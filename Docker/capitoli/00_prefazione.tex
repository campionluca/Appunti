\chapter*{Prefazione}
\addcontentsline{toc}{chapter}{Prefazione}

\section*{A chi si rivolge questo manuale}

Questi appunti sono stati pensati per studenti di istituti tecnici e professionali, sviluppatori e sistemisti che vogliono apprendere le tecnologie di containerizzazione e DevOps moderne. Il percorso è strutturato per accompagnare progressivamente dalla teoria ai container fino alla gestione di infrastrutture complesse con Docker e strumenti DevOps.

\section*{Struttura del corso}

Il corso è organizzato in 7 capitoli che coprono l'intero ecosistema Docker e le pratiche DevOps:

\textbf{Parte I - Fondamenti Container} (Capitoli 1-2)
\begin{itemize}
    \item Introduzione ai container e differenze con le macchine virtuali
    \item Architettura Docker e componenti fondamentali
    \item Comandi base per gestire container e immagini
    \item Ciclo di vita di un container
\end{itemize}

\textbf{Parte II - Creazione Immagini} (Capitolo 3)
\begin{itemize}
    \item Dockerfile: sintassi e istruzioni principali
    \item Best practices per immagini efficienti
    \item Multi-stage builds per ottimizzazione
    \item Layer caching e strategie di build
\end{itemize}

\textbf{Parte III - Orchestrazione e Networking} (Capitoli 4-5)
\begin{itemize}
    \item Docker Compose per applicazioni multi-container
    \item Networking: bridge, host, overlay
    \item Gestione volumi e persistenza dati
    \item Service discovery e load balancing
\end{itemize}

\textbf{Parte IV - Distribuzione} (Capitolo 6)
\begin{itemize}
    \item Docker Hub e registry pubblici
    \item Registry privati e sicurezza
    \item CI/CD con Docker
    \item Strategie di deployment
\end{itemize}

\section*{Prerequisiti}

Per affrontare questo corso è consigliabile avere:
\begin{itemize}
    \item Conoscenze base di Linux e comandi shell
    \item Familiarità con networking (IP, porte, protocolli)
    \item Conoscenza di almeno un linguaggio di programmazione
    \item Comprensione dei concetti di client-server
    \item (Opzionale) Esperienza con macchine virtuali
\end{itemize}

\section*{Strumenti necessari}

\textbf{Software consigliato}:
\begin{itemize}
    \item \textbf{Docker Engine}: Runtime per container Linux/Windows
    \item \textbf{Docker Desktop}: Applicazione GUI per macOS/Windows
    \item \textbf{Docker Compose}: Orchestrazione multi-container
    \item \textbf{Visual Studio Code}: Editor con estensioni Docker
    \item \textbf{Portainer}: Interfaccia web per gestione Docker
\end{itemize}

\textbf{Ambienti di sviluppo}:
\begin{itemize}
    \item \textbf{Linux}: Ubuntu 20.04+, Debian, CentOS, Fedora
    \item \textbf{Windows}: Windows 10/11 Pro con WSL2
    \item \textbf{macOS}: macOS 10.15+ con Docker Desktop
    \item \textbf{Cloud}: AWS, Azure, Google Cloud (livello free tier)
\end{itemize}

\textbf{Tool aggiuntivi}:
\begin{itemize}
    \item \textbf{Git}: Versioning del codice e Dockerfile
    \item \textbf{curl/wget}: Testing API e download
    \item \textbf{jq}: Parsing JSON per inspect e API
    \item \textbf{dive}: Analisi layer immagini Docker
\end{itemize}

\section*{Come studiare}

Per ottenere il massimo da questi appunti:

\begin{enumerate}
    \item \textbf{Installa Docker}: Configura l'ambiente sul tuo sistema
    \item \textbf{Digita i comandi}: Non copiare/incollare, scrivi manualmente
    \item \textbf{Sperimenta}: Modifica i Dockerfile e osserva i risultati
    \item \textbf{Leggi i log}: Impara a debuggare container in errore
    \item \textbf{Costruisci progetti}: Containerizza applicazioni reali
    \item \textbf{Studia i layer}: Usa \texttt{docker history} e \texttt{dive}
    \item \textbf{Pratica networking}: Testa comunicazione tra container
    \item \textbf{Ottimizza}: Riduci dimensioni immagini e tempi di build
\end{enumerate}

\begin{nota}
Questo manuale usa \textbf{Docker Engine 20.10+} e \textbf{Docker Compose V2}. La maggior parte dei comandi funziona anche su versioni precedenti, ma alcune funzionalità avanzate richiedono versioni recenti.
\end{nota}

\section*{Convenzioni tipografiche}

Nel testo vengono utilizzate le seguenti convenzioni:

\begin{itemize}
    \item \texttt{Comandi shell}: Comando da eseguire in terminale
    \item \textbf{Parole chiave}: Concetti importanti (container, image, volume)
    \item \textit{Nomi di file/path}: Riferimenti a file (Dockerfile, /var/lib/docker)
    \item Box colorati: Note, Attenzioni, Best Practices, Errori Comuni
    \item Diagrammi: Architetture e flussi con TikZ
\end{itemize}

\textbf{Formato comandi}:
\begin{lstlisting}[language=bash]
# Commento esplicativo
$ docker comando [OPZIONI] ARGOMENTO
\end{lstlisting}

\textbf{Output esempio}:
\begin{verbatim}
CONTAINER ID   IMAGE     COMMAND   STATUS
a1b2c3d4e5f6   nginx     ...       Up 2 hours
\end{verbatim}

\section*{Architettura del manuale}

\textbf{Struttura di ogni capitolo}:
\begin{enumerate}
    \item \textbf{Obiettivi}: Cosa imparerai
    \item \textbf{Teoria}: Concetti fondamentali
    \item \textbf{Pratica}: Esempi completi commentati
    \item \textbf{Diagrammi}: Visualizzazione architetture
    \item \textbf{Best Practices}: Consigli professionali
    \item \textbf{Errori Comuni}: Problemi da evitare
    \item \textbf{Debugging}: Troubleshooting e log analysis
    \item \textbf{Esercizi}: Sfide pratiche graduate
    \item \textbf{Caso di Studio}: Progetto completo
    \item \textbf{Riepilogo}: Riassunto concetti chiave
    \item \textbf{Riferimenti}: Documentazione ufficiale
\end{enumerate}

\section*{Laboratorio pratico}

Durante il corso costruirai:
\begin{itemize}
    \item \textbf{Web app multi-tier}: Frontend + Backend + Database
    \item \textbf{Microservizi}: Architettura distribuita con API
    \item \textbf{CI/CD Pipeline}: Build automatizzata e deployment
    \item \textbf{Monitoring stack}: Prometheus + Grafana
    \item \textbf{Reverse proxy}: Nginx per load balancing
\end{itemize}

\section*{Certificazioni}

Questo corso prepara per:
\begin{itemize}
    \item \textbf{Docker Certified Associate (DCA)}
    \item \textbf{Kubernetes fundamentals} (passo successivo naturale)
    \item \textbf{Linux Foundation certifications} (DevOps track)
\end{itemize}

\section*{Sito web e risorse}

Materiale aggiuntivo disponibile su:
\begin{itemize}
    \item Repository GitHub: \url{https://github.com/campionluca/Appunti}
    \item Dockerfile di esempio scaricabili
    \item Docker Compose templates per progetti comuni
    \item Script di automazione e best practices
    \item Video tutorial e screencast
    \item Community Discord per supporto
\end{itemize}

\section*{Filosofia DevOps}

Docker è uno strumento fondamentale nella cultura DevOps:

\begin{tcolorbox}[colback=blue!10, colframe=blue!60, title=Principi DevOps]
\begin{itemize}
    \item \textbf{Automation}: Automatizza build, test, deployment
    \item \textbf{CI/CD}: Integrazione e consegna continue
    \item \textbf{Infrastructure as Code}: Infrastruttura versionata
    \item \textbf{Monitoring}: Osservabilità e metriche
    \item \textbf{Collaboration}: Dev e Ops lavorano insieme
    \item \textbf{Feedback rapido}: Cicli brevi di sviluppo
\end{itemize}
\end{tcolorbox}

\section*{Container nel mondo reale}

Docker è utilizzato da:
\begin{itemize}
    \item \textbf{Startup}: Deployment rapido e scalabile
    \item \textbf{Enterprise}: Modernizzazione applicazioni legacy
    \item \textbf{Cloud providers}: AWS ECS/Fargate, Azure ACI, GCP Cloud Run
    \item \textbf{Kubernetes}: Orchestrazione container in produzione
    \item \textbf{Sviluppatori}: Ambienti consistenti dev/staging/prod
\end{itemize}

\section*{Roadmap di apprendimento}

\textbf{Percorso consigliato}:
\begin{enumerate}
    \item \textbf{Settimana 1-2}: Capitoli 1-2 (fondamenti e comandi base)
    \item \textbf{Settimana 3-4}: Capitolo 3 (Dockerfile e build)
    \item \textbf{Settimana 5-6}: Capitolo 4 (Docker Compose)
    \item \textbf{Settimana 7-8}: Capitolo 5 (networking e volumes)
    \item \textbf{Settimana 9-10}: Capitolo 6 (registry e deployment)
    \item \textbf{Settimana 11-12}: Progetto finale completo
\end{enumerate}

\section*{Progetto finale}

Al termine del corso sarai in grado di:
\begin{itemize}
    \item Containerizzare qualsiasi applicazione
    \item Creare Dockerfile ottimizzati e sicuri
    \item Orchestrare stack multi-container con Compose
    \item Configurare reti e volumi persistenti
    \item Distribuire su registry pubblici e privati
    \item Implementare CI/CD con GitHub Actions + Docker
    \item Debuggare problemi di container in produzione
\end{itemize}

\section*{Community e supporto}

\textbf{Dove trovare aiuto}:
\begin{itemize}
    \item \textbf{Docker Forums}: \url{https://forums.docker.com}
    \item \textbf{Stack Overflow}: Tag [docker] e [dockerfile]
    \item \textbf{Docker Community Slack}: Chat in tempo reale
    \item \textbf{Reddit}: r/docker per discussioni e best practices
    \item \textbf{GitHub Issues}: Report bug e feature requests
\end{itemize}

\section*{Sicurezza}

\begin{attenzione}
La sicurezza dei container è fondamentale:
\begin{itemize}
    \item Non eseguire container come root se evitabile
    \item Scansiona immagini per vulnerabilità (Trivy, Snyk)
    \item Usa immagini ufficiali da registry fidati
    \item Aggiorna regolarmente base images
    \item Limita risorse CPU/RAM per prevenire DoS
    \item Usa secrets manager per credenziali sensibili
\end{itemize}
\end{attenzione}

\section*{Ringraziamenti}

Si ringrazia:
\begin{itemize}
    \item Docker Inc. per l'eccellente documentazione ufficiale
    \item La community open source per contributi e feedback
    \item L'Istituto Tecnico Antonio Scarpa per il supporto
    \item Gli studenti che hanno testato e migliorato questi materiali
\end{itemize}

\vspace{1cm}

\begin{flushright}
\textit{Prof. Luca Campion}\\
Novembre 2025
\end{flushright}

\section*{Note sulla versione}

\textbf{Versione 1.0} - Novembre 2025
\begin{itemize}
    \item Prima release completa
    \item 7 capitoli + esempi pratici
    \item Coverage Docker: Engine, Compose, Networking, Registry
    \item Esempi testati su Docker 20.10+
    \item Diagrammi architettura con TikZ
    \item 100+ esempi di codice funzionanti
    \item Casi di studio reali da produzione
\end{itemize}

\textbf{Prossimi aggiornamenti}:
\begin{itemize}
    \item Kubernetes fundamentals (orchestrazione avanzata)
    \item Docker Swarm per clustering
    \item Security scanning e hardening
    \item Monitoring con Prometheus/Grafana
    \item Service mesh con Istio
\end{itemize}

\section*{Feedback}

Questo manuale è in continua evoluzione. Invia suggerimenti, correzioni o richieste a:
\begin{itemize}
    \item Email: \texttt{luca.campion@example.com}
    \item GitHub Issues: \url{https://github.com/campionluca/Appunti/issues}
    \item Pull Requests benvenute!
\end{itemize}

\section*{Licenza}

Questo materiale è distribuito con licenza Creative Commons BY-NC-SA 4.0:
\begin{itemize}
    \item \textbf{BY}: Attribuzione obbligatoria
    \item \textbf{NC}: Uso non commerciale
    \item \textbf{SA}: Condivisione con stessa licenza
\end{itemize}

\begin{center}
\textit{"Containers are the future of software deployment"}\\
\textbf{-- Solomon Hykes, Docker Founder}
\end{center}
