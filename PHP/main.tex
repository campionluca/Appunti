% main.tex — Documento principale LaTeX per "Appunti di Programmazione PHP"
\documentclass[a4paper,11pt]{book}

% Lingua e codifica
\usepackage[italian]{babel}
\usepackage[T1]{fontenc}
\usepackage[utf8]{inputenc}
\usepackage{fontspec}
\usepackage{csquotes}

% Layout
\usepackage{geometry}
\geometry{margin=2.5cm}

% Hyperlink e colori
\usepackage{xcolor}
\usepackage{hyperref}
\hypersetup{
    colorlinks=true,
    linkcolor=blue,
    urlcolor=blue,
    citecolor=blue
}

% Codice sorgente (PHP)
\usepackage{listings}
\usepackage{listingsutf8}
\lstdefinestyle{php}{
  language=PHP,
  basicstyle=\ttfamily\small,
  keywordstyle=\color{blue!70!black},
  commentstyle=\color{gray!70!black},
  stringstyle=\color{red!60!black},
  showstringspaces=false,
  numbers=left,
  numberstyle=\tiny,
  stepnumber=1,
  frame=single,
  breaklines=true
}
\lstset{style=php, inputencoding=utf8,
  literate={à}{{\`a}}1 {è}{{\`e}}1 {é}{{\'e}}1 {ì}{{\`i}}1 {ò}{{\`o}}1 {ù}{{\`u}}1
           {À}{{\`A}}1 {È}{{\`E}}1 {É}{{\'E}}1 {Ì}{{\`I}}1 {Ò}{{\`O}}1 {Ù}{{\`U}}1
           {–}{{-}}1 {—}{{-}}1 {‑}{{-}}1 {→}{{->}}2 {…}{{...}}3}

% Box informativi
\usepackage[skins, breakable]{tcolorbox}
\tcbset{
  colback=gray!5,
  colframe=gray!60,
  coltitle=black,
  fonttitle=\bfseries,
  boxrule=0.8pt,
  arc=2pt
}

% Tipografia
\usepackage[protrusion=true,expansion=false]{microtype}
\setlength{\emergencystretch}{3em}

% Metadati documento
\title{Appunti di Programmazione PHP}
\author{Prof. Luca Campion}
\date{\today}

\begin{document}
\maketitle
\tableofcontents

\mainmatter

% Inclusione dei capitoli
\chapter*{Prefazione}
\addcontentsline{toc}{chapter}{Prefazione}

\section*{A chi è rivolto questo libro}

Questi appunti sono stati pensati per gli studenti del quarto anno di Istituto Tecnico che stanno approfondendo la programmazione in Java. Il materiale presuppone una conoscenza di base del linguaggio (variabili, cicli, metodi, concetti fondamentali di programmazione) e si propone di consolidare e ampliare tali competenze attraverso argomenti più avanzati e pratici.

L'approccio adottato bilancia teoria ed esempi concreti, con l'obiettivo di fornire strumenti immediatamente applicabili sia nei progetti scolastici che in contesti reali.

\section*{Struttura del libro}

Il libro è organizzato in otto capitoli, ciascuno focalizzato su un argomento specifico:

\begin{enumerate}
    \item \textbf{Classi, Oggetti e Ereditarietà}: ripasso e approfondimento dei concetti fondamentali della programmazione orientata agli oggetti, con particolare attenzione agli array di oggetti e alla gerarchia tra classi.

    \item \textbf{Stream e Buffer}: gestione di flussi di dati per leggere e scrivere file, con esempi pratici di utilizzo delle classi più comuni.

    \item \textbf{Interfacce e Classi Astratte}: meccanismi per definire comportamenti comuni e creare gerarchie flessibili.

    \item \textbf{Eccezioni}: gestione degli errori a runtime attraverso il sistema delle eccezioni di Java.

    \item \textbf{ArrayList}: struttura dati dinamica per gestire collezioni di elementi in modo più flessibile rispetto agli array tradizionali.

    \item \textbf{Interfacce Grafiche}: introduzione alla creazione di applicazioni con interfaccia grafica usando Swing, inclusa la gestione degli eventi.

    \item \textbf{Model View Controller}: pattern architetturale per organizzare il codice separando logica, presentazione e controllo.

    \item \textbf{Lambda Expressions}: cenni alle espressioni lambda introdotte in Java 8, per scrivere codice più conciso ed espressivo.
\end{enumerate}

\section*{Come usare questo libro}

Ogni capitolo è strutturato per guidare l'apprendimento in modo progressivo:

\begin{itemize}
    \item Gli \textbf{obiettivi di apprendimento} all'inizio di ogni capitolo chiariscono cosa ci si aspetta di saper fare al termine dello studio.

    \item La \textbf{teoria} è presentata in modo sintetico ma completo, con definizioni chiare e schemi quando necessario.

    \item Gli \textbf{esempi di codice} sono commentati in italiano e mostrano l'applicazione pratica dei concetti. Si consiglia di digitare personalmente ogni esempio, eseguirlo e sperimentare modifiche per comprenderne il funzionamento.

    \item I \textbf{box colorati} evidenziano informazioni particolari:
    \begin{itemize}
        \item \textcolor{orange}{Arancione (Attenzione)}: punti critici da ricordare
        \item \textcolor{blue}{Blu (Nota)}: suggerimenti e best practices
        \item \textcolor{red}{Rosso (Errore Comune)}: errori frequenti da evitare
    \end{itemize}

    \item Gli \textbf{esercizi} sono suddivisi in tre livelli di difficoltà (base, intermedio, avanzato). Si consiglia di affrontarli in ordine, verificando le soluzioni commentate nell'appendice solo dopo aver tentato autonomamente.

    \item Il \textbf{riepilogo} alla fine di ogni capitolo sintetizza i concetti chiave e facilita il ripasso.
\end{itemize}

\section*{Prerequisiti}

Per affrontare efficacemente questi appunti, è necessario:

\begin{itemize}
    \item Conoscere la sintassi base di Java (tipi di dato primitivi, operatori, strutture di controllo)
    \item Saper dichiarare e utilizzare metodi
    \item Comprendere i concetti basilari di classe e oggetto
    \item Avere familiarità con array monodimensionali
    \item Disporre di un ambiente di sviluppo Java funzionante (JDK 8 o superiore, IDE come Eclipse, IntelliJ IDEA o NetBeans)
\end{itemize}

\section*{Convenzioni utilizzate}

\textbf{Codice}: tutti gli esempi di codice sono presentati con sintassi evidenziata, numerazione delle righe e commenti esplicativi.

\textbf{Nomenclatura}: si segue la convenzione Java standard (CamelCase per classi, camelCase per metodi e variabili, MAIUSCOLO per costanti).

\textbf{Terminologia}: si preferisce l'italiano quando possibile, mantenendo i termini tecnici in inglese quando consolidati nella pratica professionale (ad esempio "stream", "buffer", "exception").

\vspace{1cm}

Buono studio!

\chapter{Form HTML}

\begin{tcolorbox}[title=Mappa del capitolo]
Introduzione, Creazione di form, Gestione dati, Validazione, Sicurezza XSS, Esempi pratici, Caso di studio, Diagrammi, Esercizi, Verifica, Riepilogo, Riferimenti.
\end{tcolorbox}

\section{Obiettivi di apprendimento}
\begin{itemize}
  \item Progettare form ben strutturati per GET e POST.
  \item Gestire correttamente input lato server con normalizzazione e escaping.
  \item Applicare PRG quando si modifica stato e prevenire XSS.
\end{itemize}

\section{Teoria}
I form HTML sono la principale modalità con cui un client invia dati al server. PHP riceve i dati tramite le superglobali \verb|$_GET|, \verb|$_POST| e \verb|$_FILES|. La scelta del metodo dipende dal caso d'uso: \textbf{GET} per richieste idempotenti e query string, \textbf{POST} per invio di dati sensibili o variazioni di stato.

\section{Creazione di form (GET/POST)}
\begin{lstlisting}
<!-- Esempio GET -->
<form method="get" action="processa.php">
  <label>Query: <input type="text" name="q"></label>
  <button type="submit">Cerca</button>
</form>

<!-- Esempio POST -->
<form method="post" action="processa.php">
  <label>Email: <input type="email" name="email" required></label>
  <label>Password: <input type="password" name="password" required></label>
  <button type="submit">Invia</button>
</form>
\end{lstlisting}

\section{Gestione dei dati inviati}
\begin{lstlisting}
<?php
// processa.php
$metodo = $_SERVER['REQUEST_METHOD'];
$email  = $_POST['email'] ?? '';
$q      = $_GET['q'] ?? '';

if ($metodo === 'POST') {
    if ($email === '') {
        echo 'Email non valida';
        exit;
    }
    echo 'OK';
} else {
    echo 'Ricerca: ' . htmlspecialchars($q, ENT_QUOTES | ENT_SUBSTITUTE, 'UTF-8');
}
\end{lstlisting}

\section{Validazione degli input}
- Validazione lato client (HTML5: \texttt{required}, \texttt{type=email}) e lato server (sempre necessaria).
- Usare regex e normalizzazione dei dati.

\begin{lstlisting}
<?php
// Normalizzazione e validazione
$name = trim((string)($_POST['name'] ?? ''));
if ($name === '' || mb_strlen($name) > 100) {
    echo 'Nome obbligatorio (<=100)';
    exit;
}
\end{lstlisting}

\section{Sicurezza: prevenzione XSS}
- \textbf{XSS}: effettuare sempre escaping in output con \texttt{htmlspecialchars}.

\begin{lstlisting}
<?php
// Esempio di escaping in output
// Usare htmlspecialchars per prevenire XSS quando si rende input utente
echo 'Ricerca: ' . htmlspecialchars($q, ENT_QUOTES | ENT_SUBSTITUTE, 'UTF-8');
\end{lstlisting}

\begin{tcolorbox}[title=Best practice]
- Usare POST per dati sensibili e operazioni che modificano stato.
- Escaping sistematico dell'output (XSS).
- Validare e normalizzare sempre lato server.
- Impostare \texttt{Content-Type} e charset corretti nelle risposte.
\end{tcolorbox}

\begin{tcolorbox}[title=Errori comuni]
- Fidarsi della sola validazione lato client.
- Non effettuare escaping dell'output.
- Mescolare GET/POST senza gestire i casi distinti.
\end{tcolorbox}

\section{Esempi pratici}
\begin{lstlisting}
<?php
// Contatto semplice con normalizzazione e escape
$name  = trim((string)($_POST['name'] ?? ''));
$email = trim((string)($_POST['email'] ?? ''));
$msg   = trim((string)($_POST['message'] ?? ''));

$errors = [];
if ($name === '' || mb_strlen($name) > 100) { $errors[] = 'Nome obbligatorio (<=100)'; }
if (!preg_match('/^[^@\s]+@[^@\s]+\.[^@\s]+$/', $email)) { $errors[] = 'Email non valida'; }
if ($msg === '') { $errors[] = 'Messaggio obbligatorio'; }

if ($errors) {
    // PRG: mostra errori su pagina GET
    header('Location: /contatto_errore.php', true, 303);
    exit;
}

echo 'Grazie, ' . htmlspecialchars($name, ENT_QUOTES | ENT_SUBSTITUTE, 'UTF-8');
?>
\end{lstlisting}

\section{Caso di studio}
Progettazione di un form di login minimale con gestione errori e PRG. Si valida l'input, si normalizza e si evita qualsiasi echo diretto di dati utente senza escape.

\begin{lstlisting}
<?php
$u = trim((string)($_POST['username'] ?? ''));
$p = (string)($_POST['password'] ?? '');
if ($u === '' || $p === '') {
  header('Location: /login.php?err=1', true, 303);
  exit;
}
// Verifica credenziali (simulata)
header('Location: /dashboard.php', true, 303);
exit;
?>
\end{lstlisting}

\section{Diagrammi}
\noindent Flusso PRG: \emph{POST} (valida/aggiorna) \(\rightarrow\) \emph{303 See Other} \(\rightarrow\) \emph{GET} pagina di conferma.

\section{Esercizi}
\begin{itemize}
  \item Implementa un form di contatto con validazione lato server e PRG.
  \item Aggiungi un campo textarea e normalizza gli spazi e le nuove linee.
  \item Progetta un form di ricerca con GET e escaping dell'output.
\end{itemize}

\section{Verifica}
\begin{itemize}
  \item Vero/Falso: dopo \texttt{header('Location',...)} è obbligatorio \texttt{exit}.
  \item Quale metodo usare per mostrare risultati di ricerca? GET o POST e perché?
\end{itemize}

\section{Riepilogo}
Form ben progettati separano responsabilità tra input, elaborazione e output. La validazione lato server e l'escaping sono indispensabili; PRG evita ri-submit del POST.

\section{Riferimenti}
\begin{itemize}
  \item Manuale PHP — superglobali: \url{https://www.php.net/manual/en/reserved.variables.php}
  \item OWASP XSS: \url{https://owasp.org/www-community/attacks/xss/}
  \item HTTP Semantics (Redirect): \url{https://www.rfc-editor.org/rfc/rfc9110}
\end{itemize}

\chapter{Campi Hidden}\label{cap:hidden}

\begin{tcolorbox}[title=Mappa del capitolo]
Teoria, Utilizzo pratico, Scenari di applicazione, Sicurezza, Esempi estesi, Caso di studio, Esercizi, Verifica, Riferimenti.
\end{tcolorbox}

\section{Obiettivi di apprendimento}
\begin{itemize}
  \item Comprendere il ruolo dei campi \emph{hidden} nei form.
  \item Applicare validazioni lato server per dati non visibili.
  \item Integrare hidden con sessione e flussi multi-step mantenendo integrità.
\end{itemize}

\section{Teoria}
I campi \emph{hidden} sono input non visibili all'utente ma inviati con il form. Utili per trasmettere metadati, token di sicurezza, identificatori di stato.

\section{Utilizzo pratico}
\begin{lstlisting}
<!-- Trasmettere un ID ordine e metadati -->
<form method="post" action="checkout.php">
  <input type="hidden" name="order_id" value="12345">
  <button type="submit">Conferma</button>
</form>
\end{lstlisting}

\subsection*{Esempio esteso: token di integrità (concettuale)}
\begin{lstlisting}
<?php
// Generazione lato server (pagina precedente)
$orderId = 12345;
$secret  = 'chiave_server';
$token   = sha1($orderId . '|' . $secret); // firma semplice
?>
<form method="post" action="checkout.php">
  <input type="hidden" name="order_id" value="<?php echo (int)$orderId; ?>">
  <input type="hidden" name="token" value="<?php echo $token; ?>">
  <button type="submit">Conferma</button>
</form>

<?php
// Verifica lato server (checkout.php)
$orderId = isset($_POST['order_id']) ? (int)$_POST['order_id'] : 0;
$token   = isset($_POST['token']) ? (string)$_POST['token'] : '';
$secret  = 'chiave_server';
if ($orderId <= 0) {
    echo 'order_id non valido';
    exit;
}
$expected = sha1($orderId . '|' . $secret);
if ($token !== $expected) {
    echo 'Token non valido';
    exit;
}
// OK: procedere con checkout
?>
\end{lstlisting}

\section{Scenari di applicazione}
- Stato dell'applicazione (wizard multi-step, ID risorsa).
- Parametri non modificabili lato client.

\section{Caso di studio: wizard di checkout a più step}
\begin{itemize}
  \item Step 1: selezione prodotti; salvataggio in sessione.
  \item Step 2: indirizzo e spedizione; hidden con \verb|order_id|.
  \item Step 3: pagamento; verifica integrità di \verb|order_id| con token.
  \item Step 4: conferma; controllo finale e generazione ordine.
\end{itemize}

\section{Considerazioni sulla sicurezza}
- Non fidarsi dei valori hidden: sono modificabili lato client; validare sempre lato server.
- Usare \textbf{token} firmati o verificabili (es. HMAC) se necessario.
- Non inserire mai dati sensibili in chiaro; preferire sessione lato server.

\begin{lstlisting}
<?php
// Verifica lato server senza funzioni rimosse
$orderId = isset($_POST['order_id']) ? (int)$_POST['order_id'] : 0;
if ($orderId <= 0) {
    echo 'order_id non valido';
    exit;
}
// Continuare elaborazione
\end{lstlisting}

\begin{tcolorbox}[title=Best practice]
- Usare hidden per metadati non sensibili; mai per segreti.
- Convalidare tutti i campi hidden lato server.
 - Abbinare hidden a sessione o tracciamento lato server per integrità.
\end{tcolorbox}

\begin{tcolorbox}[title=Errori comuni]
- Presumere che hidden sia sicuro/inviolabile.
- Esporre informazioni sensibili (es. ruoli, prezzi calcolati) in hidden.
- Mancare la validazione server-side dei hidden.
\end{tcolorbox}

\section{Esercizi}
\begin{itemize}
  \item Progetta un form multi-step che usa hidden solo per ID non sensibili, mantenendo i dati in sessione.
  \item Implementa una verifica di integrità per un ID trasmesso via hidden.
\end{itemize}

\section{Verifica}
\begin{itemize}
  \item Perché i campi hidden non sono affidabili come meccanismo di sicurezza?
  \item Quali dati vanno mantenuti in sessione anziché in hidden?
\end{itemize}

\section{Riferimenti}
\begin{itemize}
  \item MDN — HTML input hidden: \url{https://developer.mozilla.org/}
  \item OWASP — Tampering dei parametri: \url{https://owasp.org/}
\end{itemize}

\chapter{Redirect mediante Header Location}

\begin{tcolorbox}[title=Mappa del capitolo]
Teoria, Implementazione corretta, Header già inviati, Best practice, Codici di stato, Esempi pratici, Caso di studio, Esercizi, Verifica, Riferimenti.
\end{tcolorbox}

\section{Teoria}
Il redirect HTTP si effettua impostando l'header \texttt{Location} e un codice di stato appropriato. Dopo aver inviato l'header, l'esecuzione deve terminare con \texttt{exit}.

\section{Implementazione corretta}
\begin{lstlisting}
<?php
// Redirect dopo una creazione (POST)
// Usare 303 See Other per evitare ri-submit del POST
header('Location: /success.php', true, 303);
exit;
\end{lstlisting}

\section{Gestione degli header già inviati}
Se è già stato generato output, l'invio degli header fallisce. Evitare output prima degli header o utilizzare buffer di output se necessario.

\section{Best practice per i redirect}
- Post/Redirect/Get (PRG) per evitare doppio submit.
- Usare codici 302/303/307 in base al contesto.
- Impostare sempre \texttt{exit} dopo l'header \texttt{Location}.
- Evitare output prima degli header: attivare output buffering o controllare flussi.

\section{Codici di stato appropriati}
- \textbf{302 Found}: redirect temporaneo generico.
- \textbf{303 See Other}: dopo POST per puntare a risorsa GET.
- \textbf{307 Temporary Redirect}: preserva il metodo (POST resta POST).
- \textbf{301 Moved Permanently}: redirect permanente (SEO, attenzione ai cache).

\begin{tcolorbox}[title=Errori comuni]
- Dimenticare \texttt{exit} dopo il redirect.
- Inviare contenuto prima degli header.
- Usare sempre 302 anche dopo POST (meglio 303 per PRG).
\end{tcolorbox}

\section{Esempi pratici}
\begin{lstlisting}
<?php
// 307: preserva il metodo
if ($_SERVER['REQUEST_METHOD'] === 'POST') {
  header('Location: /conferma.php', true, 303); // PRG
  exit;
}

// 301: migrazione URL permanente
header('Location: https://www.esempio.it/nuovo-percorso', true, 301);
exit;
?>
\end{lstlisting}

\section{Caso di studio}
Workflow di registrazione utente: POST /register \(\rightarrow\) valida e crea \(\rightarrow\) 303 See Other su /welcome.

\section{Esercizi}
\begin{itemize}
  \item Implementa PRG su un form di contatto.
  \item Esegui un redirect 307 per ripetere POST verso un endpoint differente.
\end{itemize}

\section{Verifica}
\begin{itemize}
  \item Qual è il codice preferibile dopo POST nel pattern PRG?
  \item Cosa accade se invii output prima degli header?
\end{itemize}

\section{Riferimenti}
\begin{itemize}
  \item RFC 9110: Semantics of HTTP — Redirect status codes.
  \item Manuale PHP — \texttt{header}: \url{https://www.php.net/header}
\end{itemize}

\chapter{Array}

\begin{tcolorbox}[title=Mappa del capitolo]
Teoria, Creazione e manipolazione, Funzioni principali, Iterazione, Array multidimensionali, Esempi pratici, Caso di studio, Esercizi, Verifica, Riepilogo, Riferimenti.
\end{tcolorbox}

\section{Teoria}
Gli array in PHP sono strutture flessibili che possono contenere valori di qualunque tipo. Esistono funzioni native per creare, manipolare e iterare.

\section{Creazione e manipolazione}
\begin{lstlisting}
<?php
$nums = [1, 2, 3];
$mix  = ['a', 10, true];
array_push($nums, 4); // [1,2,3,4]
$nums = array_merge([0], $nums); // [0,1,2,3,4]
unset($mix[1]); // rimuove elemento
\end{lstlisting}

\section{Funzioni principali}
- \texttt{count}, \texttt{array\_push}, \texttt{array\_pop}, \texttt{array\_shift}
- \texttt{in\_array}, \texttt{array\_search}, \texttt{array\_key\_exists}

\begin{lstlisting}
<?php
// Esempi senza funzioni di ordine superiore
// Quadrati
$squared = [];
foreach ($nums as $x) { $squared[] = $x * $x; }
// Pari
$even = [];
foreach ($nums as $x) { if ($x % 2 === 0) { $even[] = $x; } }
// Somma
$sum = 0;
foreach ($nums as $x) { $sum += $x; }
\end{lstlisting}

\section{Iterazione}
\begin{lstlisting}
<?php
for ($i=0; $i<count($nums); $i++) { echo $nums[$i]; }
foreach ($nums as $n) { echo $n; }
\end{lstlisting}

\section{Array multidimensionali}
\begin{lstlisting}
<?php
$matrix = [ [1,2], [3,4] ];
echo $matrix[1][0]; // 3
\end{lstlisting}

\begin{tcolorbox}[title=Best practice]
- Usare \texttt{foreach} per semplicità e leggibilità.
- Evitare rimuovere elementi senza reindicizzare se serve l'indice.
\end{tcolorbox}

\begin{tcolorbox}[title=Errori comuni]
- Presumere che gli indici si ricalcolino automaticamente dopo \texttt{unset}.
- Dimenticare che PHP consente tipi misti: gestire coerenza dei dati.
\end{tcolorbox}

\section{Esempi pratici}
\begin{lstlisting}
<?php
// Reindicizzazione dopo rimozione
$a = [10,20,30];
unset($a[1]);           // [0=>10, 2=>30]
$a = array_values($a);  // [0=>10, 1=>30]

// Aggregazioni
$prezzi = [10.5, 7.2, 13.0];
$totale = 0.0;
foreach ($prezzi as $p) { $totale += $p; }
?>
\end{lstlisting}

\section{Caso di studio}
Gestione di un carrello: aggiunta, rimozione, calcolo totale e sconti senza funzioni di ordine superiore.

\section{Esercizi}
\begin{itemize}
  \item Dato un array di interi, costruisci un nuovo array di quadrati usando solo cicli.
  \item Reindicizza correttamente dopo rimozioni multiple e verifica gli indici.
  \item Implementa un carrello con totale e sconto percentuale per importi \textgreater{} 100.
\end{itemize}

\section{Verifica}
\begin{itemize}
  \item Cosa restituisce \texttt{count} su un array con buchi di indici?
  \item Quando è necessario usare \texttt{array\_values}?
\end{itemize}

\section{Riepilogo}
Gli array in PHP sono dinamici e flessibili: usa cicli per trasformazioni e filtri, e reindicizza quando necessario.

\section{Riferimenti}
\begin{itemize}
  \item Manuale PHP — Arrays: \url{https://www.php.net/array}
\end{itemize}

\chapter{Array Associativi}\label{cap:array_assoc}

\begin{tcolorbox}[title=Mappa del capitolo]
Teoria, Confronto con array numerici, Funzioni utili, Esempi pratici, Conversioni, Caso di studio, Esercizi, Verifica, Riferimenti.
\end{tcolorbox}

\section{Obiettivi di apprendimento}
\begin{itemize}
  \item Modellare dati semplici con array associativi in PHP.
  \item Applicare funzioni standard per manipolare chiavi e valori.
  \item Integrare array con JSON e query string in modo sicuro.
\end{itemize}

\section{Teoria}
Gli array associativi usano chiavi stringa per mappare valori, ideali per rappresentare oggetti semplici o record.

\section{Differenze con array numerici}

Gli array associativi differiscono dagli array numerici in aspetti fondamentali. La differenza principale è che gli array associativi utilizzano chiavi esplicite (generalmente stringhe) per accedere ai valori, mentre gli array numerici usano indici numerici. Questa distinzione permette di creare strutture dati semantiche dove le chiavi descrivono il significato del valore. Per quanto riguarda l'iterazione, gli array associativi richiedono l'uso di \verb|foreach ($arr as $k=>$v)| per accedere sia alla chiave che al valore, a differenza degli array numerici dove spesso si usa semplicemente \verb|foreach ($arr as $v)|.

\begin{lstlisting}
<?php
$user = ['id'=>10, 'name'=>'Alice', 'role'=>'admin'];
echo $user['name'];
\end{lstlisting}

\section{Funzioni specifiche}

Diverse funzioni PHP sono specificamente progettate per lavorare efficacemente con array associativi. \texttt{array\_keys()} estrae tutte le chiavi di un array associativo e le restituisce come array indicizzato. \texttt{array\_values()} estrae tutti i valori e li restituisce come array indicizzato, perdendo l'informazione sulle chiavi originali. \texttt{array\_merge()} combina due o più array associativi, unendo le loro chiavi e valori. \texttt{array\_replace()} consente di sostituire i valori di un array con i valori di altri array, utile per l'override di configurazioni predefinite con valori personalizzati.

\begin{lstlisting}
<?php
$users = [
  ['id'=>1,'name'=>'A'],
  ['id'=>2,'name'=>'B'],
];
// Estrazione manuale di una colonna
$names = [];
foreach ($users as $row) { $names[] = $row['name']; } // ['A','B']
\end{lstlisting}

\subsection*{Esempi pratici aggiuntivi}
\begin{lstlisting}
<?php
// Merge con precedenza del secondo array sulle stesse chiavi
$base = ['host'=>'localhost','port'=>3306,'debug'=>false];
$override = ['debug'=>true];
$cfg = array_replace($base, $override); // ['host'=>..., 'port'=>..., 'debug'=>true]

// Ordinamento per chiave (k) e per valore (v)
$map = ['z'=>3, 'a'=>1, 'm'=>2];
ksort($map);  // ['a'=>1, 'm'=>2, 'z'=>3]
asort($map);  // ['a'=>1, 'm'=>2, 'z'=>3] (ordinato per valore crescente)

// Ridenominazione chiavi tramite costruzione manuale
$user = ['id'=>10,'name'=>'Alice'];
$renamed = [];
foreach ($user as $k=>$v) {
    $newKey = ($k==='name') ? 'nome' : $k;
    $renamed[$newKey] = $v;
}
\end{lstlisting}

\section{Conversione da/verso altri formati}

Gli array associativi PHP possono essere facilmente convertiti verso altri formati di serializzazione. Per JSON, \texttt{json\_encode} converte un array associativo in una stringa JSON, mentre \texttt{json\_decode(true)} (con il parametro \texttt{true}) deserializza una stringa JSON back in un array associativo. Per la query string (usata negli URL), la conversione richiede una costruzione manuale combinando \texttt{urlencode()} per codificare le chiavi e valori, e \texttt{implode()} per unire le coppie chiave-valore con il separatore &. Questo permette di trasformare facilmente array associativi in formato standard per la trasmissione nei parametri URL.

\begin{lstlisting}
<?php
$payload = json_encode($user, JSON_UNESCAPED_UNICODE);
$assoc   = json_decode($payload, true);
// Costruzione manuale di query string
$pairs = [];
foreach ($assoc as $k=>$v) { $pairs[] = urlencode($k).'='.urlencode((string)$v); }
$qs = implode('&', $pairs);
\end{lstlisting}

\section{Caso di studio: impostazioni applicative}
Rappresentazione di una configurazione come mappa chiave\textrightarrow{}valore con override per ambiente (sviluppo/produzione) e esportazione in query string per diagnosticare lo stato.

\section{Esercizi}
\begin{itemize}
  \item Dato un elenco di utenti, costruisci \verb|id=>nome| e ordinane le chiavi alfabeticamente.
  \item Implementa una funzione che rinomini una chiave in un array associativo senza perdere l'ordine.
\end{itemize}

\section{Verifica}
\begin{itemize}
  \item Qual è la differenza tra \texttt{array\_merge} e \texttt{array\_replace}?
  \item Come si impone l'ordinamento per chiave rispetto a per valore?
\end{itemize}

\section{Riferimenti}
\begin{itemize}
  \item Manuale PHP — Array: \url{https://www.php.net/array}
  \item JSON in PHP: \url{https://www.php.net/json}
\end{itemize}

\begin{tcolorbox}[title=Best practice]
- Validare chiavi attese e valori prima dell'uso.
- Per oggetti complessi preferire classi/DTO; usare array associativi per strutture semplici.
\end{tcolorbox}

\begin{tcolorbox}[title=Errori comuni]
- Confondere array associativi e oggetti stdClass dopo \texttt{json\_decode} (usare \texttt{true}).
- Usare \texttt{array\_merge} senza considerare override di chiavi.
\end{tcolorbox}

\chapter{Funzioni}\label{cap:funzioni}

\begin{tcolorbox}[title=Mappa del capitolo]
Obiettivi, Teoria funzioni, Definizione e sintassi, Parametri (default, type hints, variadic), Return types, Scope variabili, Arrow functions, Closures, Callback, Anonymous functions, Pure functions, Higher-order functions, Esempi pratici, Progetti, Troubleshooting, Esercizi, Riferimenti.
\end{tcolorbox}

\section{Obiettivi di apprendimento}

Al termine di questo capitolo sarai in grado di:

\begin{itemize}
    \item Definire e chiamare funzioni in PHP
    \item Utilizzare parametri con type hints e valori default
    \item Comprendere scope di variabili (locale, globale, statico)
    \item Usare arrow functions e closures
    \item Implementare callback e higher-order functions
    \item Scrivere funzioni pure e gestire side effects
    \item Applicare pattern funzionali (map, filter, reduce)
\end{itemize}

\section{Teoria: Perché le Funzioni}

Le funzioni sono blocchi di codice riutilizzabili che:

\begin{itemize}
    \item \textbf{Riducono duplicazione}: Scrivi una volta, usa ovunque
    \item \textbf{Migliorano leggibilità}: Codice auto-documentante
    \item \textbf{Facilitano testing}: Testabili indipendentemente
    \item \textbf{Incapsulano logica}: Nascondono complessità
    \item \textbf{Promuovono riuso}: Librerie e moduli
\end{itemize}

\section{Definizione e Sintassi}

\subsection{Sintassi Base}

\begin{lstlisting}[language=PHP, caption={Funzione base}]
<?php
function functionName($param1, $param2) {
    // Codice
    return $result;
}

// Chiamata
$value = functionName($arg1, $arg2);
?>
\end{lstlisting}

\subsection{Funzione con Type Hints (PHP 7+)}

\begin{lstlisting}[language=PHP, caption={Type hints consigliati}]
<?php
function greet(string $name): string {
    return "Ciao, $name!";
}

echo greet('Mario');  // Output: Ciao, Mario!

// Errore se tipo sbagliato
// greet(123);  // TypeError in strict mode
?>
\end{lstlisting}

\subsection{Strict Types}

\begin{lstlisting}[language=PHP, caption={Modalità strict types}]
<?php
declare(strict_types=1);  // PRIMA riga del file

function add(int $a, int $b): int {
    return $a + $b;
}

echo add(5, 10);     // OK: 15
// echo add('5', 10);   // TypeError: deve essere int, non string
?>
\end{lstlisting}

\begin{nota}
Usa \texttt{declare(strict\_types=1)} per forzare type checking rigoroso. È best practice per codice production.
\end{nota}

\section{Parametri delle Funzioni}

\subsection{Parametri con Valori Default}

\begin{lstlisting}[language=PHP, caption={Default parameters}]
<?php
function createUser(string $name, string $role = 'user', bool $active = true) {
    return [
        'name' => $name,
        'role' => $role,
        'active' => $active
    ];
}

// Chiama con tutti i parametri
$admin = createUser('Mario', 'admin', true);

// Usa defaults per role e active
$user = createUser('Anna');  // role='user', active=true

// Usa default solo per active
$guest = createUser('Carlo', 'guest');
?>
\end{lstlisting}

\begin{attenzione}
Parametri con default devono essere DOPO quelli obbligatori:
\begin{itemize}
    \item ✅ \texttt{function f(\$required, \$optional = 'default')}
    \item ❌ \texttt{function f(\$optional = 'default', \$required)}
\end{itemize}
\end{attenzione}

\subsection{Parametri Variadic (...)}

\begin{lstlisting}[language=PHP, caption={Variadic parameters}]
<?php
function sum(...$numbers): int {
    $total = 0;
    foreach ($numbers as $num) {
        $total += $num;
    }
    return $total;
}

echo sum(1, 2, 3);           // 6
echo sum(10, 20, 30, 40);    // 100
echo sum(5);                 // 5
echo sum();                  // 0
?>
\end{lstlisting}

\subsection{Unpacking Arguments}

\begin{lstlisting}[language=PHP, caption={Argument unpacking}]
<?php
function createPoint(int $x, int $y, int $z): array {
    return ['x' => $x, 'y' => $y, 'z' => $z];
}

$coords = [10, 20, 30];
$point = createPoint(...$coords);  // Unpacking con ...

print_r($point);
// Output: ['x' => 10, 'y' => 20, 'z' => 30]
?>
\end{lstlisting}

\subsection{Named Arguments (PHP 8+)}

\begin{lstlisting}[language=PHP, caption={Named arguments PHP 8}]
<?php
function createRect(int $width, int $height, string $color = 'black') {
    return compact('width', 'height', 'color');
}

// Posizionali (tradizionale)
$rect1 = createRect(100, 50);

// Named (PHP 8+)
$rect2 = createRect(width: 100, height: 50);

// Salta parametri intermedi
$rect3 = createRect(width: 200, color: 'red', height: 100);
?>
\end{lstlisting}

\subsection{Passaggio per Riferimento}

\begin{lstlisting}[language=PHP, caption={Pass by reference}]
<?php
// Per valore (default) - NON modifica originale
function incrementValue(int $n): void {
    $n++;  // Modifica copia locale
}

// Per riferimento - MODIFICA originale
function incrementRef(int &$n): void {
    $n++;  // Modifica variabile originale
}

$x = 10;
incrementValue($x);
echo $x;  // 10 (invariato)

incrementRef($x);
echo $x;  // 11 (modificato!)
?>
\end{lstlisting}

\begin{attenzione}
Usa passaggio per riferimento SOLO quando necessario. Preferisci return values per chiarezza.
\end{attenzione}

\section{Return Types}

\subsection{Tipi Primitivi}

\begin{lstlisting}[language=PHP, caption={Return type primitives}]
<?php
function getAge(): int {
    return 25;
}

function getPrice(): float {
    return 19.99;
}

function isActive(): bool {
    return true;
}

function getName(): string {
    return "Mario";
}
?>
\end{lstlisting}

\subsection{Tipi Complessi}

\begin{lstlisting}[language=PHP, caption={Complex return types}]
<?php
function getUser(): array {
    return ['id' => 1, 'name' => 'Mario'];
}

function getUserOrNull(): ?array {  // Nullable
    return null;  // OK
    // return ['id' => 1];  // OK anche
}

function mixed(): mixed {  // PHP 8+
    return 123;     // OK
    return "text";  // OK
    return [];      // OK
}
?>
\end{lstlisting}

\subsection{Void Return}

\begin{lstlisting}[language=PHP, caption={Void functions}]
<?php
function logMessage(string $msg): void {
    error_log($msg);
    // Non può fare return con valore
    // return $msg;  // ERRORE!

    // Solo return senza valore
    return;  // OK (opzionale)
}
?>
\end{lstlisting}

\subsection{Union Types (PHP 8+)}

\begin{lstlisting}[language=PHP, caption={Union types}]
<?php
function process(int|string $input): bool {
    if (is_int($input)) {
        return $input > 0;
    }
    return strlen($input) > 0;
}

echo process(123);      // true
echo process("hello");  // true
echo process(0);        // false
?>
\end{lstlisting}

\section{Scope delle Variabili}

\subsection{Scope Locale}

\begin{lstlisting}[language=PHP, caption={Local scope}]
<?php
function test() {
    $local = "Sono locale";
    echo $local;  // OK
}

test();
// echo $local;  // ERRORE: $local non definita fuori dalla funzione
?>
\end{lstlisting}

\subsection{Scope Globale (Evitare)}

\begin{lstlisting}[language=PHP, caption={Global scope (BAD PRACTICE)}]
<?php
$globalVar = "Globale";

function useGlobal() {
    global $globalVar;  // Dichiarazione necessaria
    echo $globalVar;
}

useGlobal();  // OK: Globale

// ALTERNATIVA MIGLIORE: passare come parametro
function useParam(string $value) {
    echo $value;
}

useParam($globalVar);  // PREFERISCI QUESTO
?>
\end{lstlisting}

\subsection{Variabili Statiche}

\begin{lstlisting}[language=PHP, caption={Static variables}]
<?php
function counter(): int {
    static $count = 0;  // Inizializzata UNA SOLA VOLTA
    $count++;
    return $count;
}

echo counter();  // 1
echo counter();  // 2
echo counter();  // 3
echo counter();  // 4
?>
\end{lstlisting}

\section{Arrow Functions (PHP 7.4+)}

\subsection{Sintassi Arrow Function}

\begin{lstlisting}[language=PHP, caption={Arrow functions}]
<?php
// Funzione tradizionale
$double1 = function($n) {
    return $n * 2;
};

// Arrow function (più concisa)
$double2 = fn($n) => $n * 2;

echo $double2(5);  // 10

// Usata con array_map
$numbers = [1, 2, 3, 4, 5];
$doubled = array_map(fn($n) => $n * 2, $numbers);
print_r($doubled);  // [2, 4, 6, 8, 10]
?>
\end{lstlisting}

\subsection{Cattura Automatica Scope}

\begin{lstlisting}[language=PHP, caption={Auto capture scope}]
<?php
$multiplier = 10;

// Closure tradizionale: serve 'use'
$func1 = function($n) use ($multiplier) {
    return $n * $multiplier;
};

// Arrow function: cattura automaticamente
$func2 = fn($n) => $n * $multiplier;

echo $func2(5);  // 50
?>
\end{lstlisting}

\section{Closures (Anonymous Functions)}

\subsection{Closure Base}

\begin{lstlisting}[language=PHP, caption={Basic closures}]
<?php
$greet = function($name) {
    return "Ciao, $name!";
};

echo $greet('Mario');  // Ciao, Mario!

// Closure come callback
$numbers = [1, 2, 3, 4, 5];
$squared = array_map(function($n) {
    return $n * $n;
}, $numbers);

print_r($squared);  // [1, 4, 9, 16, 25]
?>
\end{lstlisting}

\subsection{Closure con Use}

\begin{lstlisting}[language=PHP, caption={Closures with use}]
<?php
$prefix = "Hello";

$greet = function($name) use ($prefix) {
    return "$prefix, $name!";
};

echo $greet('World');  // Hello, World!

// Use by reference
$counter = 0;
$increment = function() use (&$counter) {
    $counter++;
};

$increment();
$increment();
echo $counter;  // 2
?>
\end{lstlisting}

\subsection{Closure Factory}

\begin{lstlisting}[language=PHP, caption={Closure factory pattern}]
<?php
function makeMultiplier(int $factor) {
    return fn($n) => $n * $factor;
}

$double = makeMultiplier(2);
$triple = makeMultiplier(3);
$quadruple = makeMultiplier(4);

echo $double(10);     // 20
echo $triple(10);     // 30
echo $quadruple(10);  // 40
?>
\end{lstlisting}

\section{Callback Functions}

\subsection{Callback con Funzioni Named}

\begin{lstlisting}[language=PHP, caption={Named function callbacks}]
<?php
function isEven(int $n): bool {
    return $n % 2 === 0;
}

$numbers = [1, 2, 3, 4, 5, 6];
$evens = array_filter($numbers, 'isEven');

print_r($evens);  // [2, 4, 6]
?>
\end{lstlisting}

\subsection{Callable Type Hint}

\begin{lstlisting}[language=PHP, caption={Callable type hint}]
<?php
function applyOperation(array $numbers, callable $operation): array {
    $result = [];
    foreach ($numbers as $num) {
        $result[] = $operation($num);
    }
    return $result;
}

$numbers = [1, 2, 3, 4, 5];

// Con closure
$squared = applyOperation($numbers, fn($n) => $n * $n);

// Con funzione named
$doubled = applyOperation($numbers, fn($n) => $n * 2);

print_r($squared);  // [1, 4, 9, 16, 25]
print_r($doubled);  // [2, 4, 6, 8, 10]
?>
\end{lstlisting}

\section{Pure Functions}

\subsection{Cos'è una Pure Function}

Una funzione è "pura" se:
\begin{enumerate}
    \item Dato lo stesso input, ritorna sempre lo stesso output (deterministica)
    \item Non ha side effects (non modifica stato esterno)
\end{enumerate}

\begin{lstlisting}[language=PHP, caption={Pure vs Impure functions}]
<?php
// PURE - deterministico, no side effects
function add(int $a, int $b): int {
    return $a + $b;
}

echo add(2, 3);  // Sempre 5

// IMPURE - usa stato esterno (time)
function getCurrentTime(): string {
    return date('H:i:s');  // Output varia
}

// IMPURE - ha side effects (scrive file)
function logMessage(string $msg): void {
    file_put_contents('log.txt', $msg, FILE_APPEND);
}

// IMPURE - modifica stato esterno
$counter = 0;
function incrementGlobal(): void {
    global $counter;
    $counter++;  // Side effect!
}
?>
\end{lstlisting}

\subsection{Vantaggi Pure Functions}

\begin{itemize}
    \item \textbf{Testabili}: Facile scrivere unit test
    \item \textbf{Prevedibili}: Nessuna sorpresa
    \item \textbf{Componibili}: Combinabili facilmente
    \item \textbf{Parallelizzabili}: Nessuna race condition
    \item \textbf{Cacheable}: Memoization possibile
\end{itemize}

\section{Higher-Order Functions}

Funzioni che:
\begin{itemize}
    \item Accettano altre funzioni come parametri
    \item Ritornano funzioni
\end{itemize}

\subsection{array\_map, array\_filter, array\_reduce}

\begin{lstlisting}[language=PHP, caption={HOF built-in}]
<?php
$numbers = [1, 2, 3, 4, 5];

// MAP: trasforma ogni elemento
$doubled = array_map(fn($n) => $n * 2, $numbers);
// [2, 4, 6, 8, 10]

// FILTER: seleziona elementi
$evens = array_filter($numbers, fn($n) => $n % 2 === 0);
// [2, 4]

// REDUCE: aggrega in un valore
$sum = array_reduce($numbers, fn($carry, $n) => $carry + $n, 0);
// 15
?>
\end{lstlisting}

\subsection{Compose e Pipe}

\begin{lstlisting}[language=PHP, caption={Function composition}]
<?php
// Compose: applica funzioni da destra a sinistra
function compose(callable ...$functions): callable {
    return fn($value) => array_reduce(
        array_reverse($functions),
        fn($carry, $fn) => $fn($carry),
        $value
    );
}

$double = fn($n) => $n * 2;
$increment = fn($n) => $n + 1;
$square = fn($n) => $n * $n;

$composed = compose($square, $increment, $double);
echo $composed(5);  // ((5 * 2) + 1)^2 = 121

// Pipe: applica funzioni da sinistra a destra
function pipe(callable ...$functions): callable {
    return fn($value) => array_reduce(
        $functions,
        fn($carry, $fn) => $fn($carry),
        $value
    );
}

$piped = pipe($double, $increment, $square);
echo $piped(5);  // ((5 * 2) + 1)^2 = 121
?>
\end{lstlisting}

\section{Esempi Pratici Completi}

\subsection{Utility Functions Library}

\begin{lstlisting}[language=PHP, caption={utils.php}]
<?php
// String utilities
function slugify(string $text): string {
    $text = strtolower($text);
    $text = preg_replace('/[^a-z0-9-]/', '-', $text);
    $text = preg_replace('/-+/', '-', $text);
    return trim($text, '-');
}

function truncate(string $text, int $length, string $suffix = '...'): string {
    if (strlen($text) <= $length) {
        return $text;
    }
    return substr($text, 0, $length - strlen($suffix)) . $suffix;
}

// Array utilities
function pluck(array $array, string $key): array {
    return array_map(fn($item) => $item[$key] ?? null, $array);
}

function groupBy(array $array, string $key): array {
    $result = [];
    foreach ($array as $item) {
        $groupKey = $item[$key] ?? 'undefined';
        $result[$groupKey][] = $item;
    }
    return $result;
}

// Validation
function validateEmail(string $email): bool {
    return filter_var($email, FILTER_VALIDATE_EMAIL) !== false;
}

function validateAge(int $age): bool {
    return $age >= 18 && $age <= 120;
}

// Usage
echo slugify("Hello World! 123");  // hello-world-123
echo truncate("Long text here", 10);  // Long te...

$users = [
    ['name' => 'Mario', 'role' => 'admin'],
    ['name' => 'Anna', 'role' => 'user'],
    ['name' => 'Carlo', 'role' => 'admin']
];

$names = pluck($users, 'name');  // ['Mario', 'Anna', 'Carlo']
$byRole = groupBy($users, 'role');
// ['admin' => [Mario, Carlo], 'user' => [Anna]]
?>
\end{lstlisting}

\section{Troubleshooting}

\subsection{Errore: "Too few arguments"}

\begin{lstlisting}[language=PHP]
function greet(string $name, string $title) {
    return "$title $name";
}

greet("Mario");  // ERROR: Too few arguments
\end{lstlisting}

Soluzione: fornisci tutti i parametri o usa valori default.

\subsection{Type Error}

\begin{lstlisting}[language=PHP]
declare(strict_types=1);

function add(int $a, int $b): int {
    return $a + $b;
}

add("5", "10");  // TypeError
\end{lstlisting}

Soluzione: passa tipi corretti o rimuovi strict_types.

\section{Esercizi}

\subsection{Esercizio 1 (Base)}
Crea funzioni utility:
\begin{itemize}
    \item \texttt{isEven(\$n)}: verifica se numero è pari
    \item \texttt{capitalize(\$str)}: prima lettera maiuscola
    \item \texttt{average(...\$numbers)}: calcola media
\end{itemize}

\subsection{Esercizio 2 (Intermedio)}
Implementa higher-order functions:
\begin{itemize}
    \item \texttt{arrayEvery(\$arr, \$predicate)}: tutti gli elementi soddisfano predicato
    \item \texttt{arraySome(\$arr, \$predicate)}: almeno un elemento soddisfa
    \item \texttt{memoize(\$fn)}: wrapper che cache i risultati
\end{itemize}

\subsection{Esercizio 3 (Avanzato)}
Sistema validazione con compose:
\begin{itemize}
    \item Singole funzioni validazione (email, lunghezza, regex)
    \item \texttt{validateAll(...\$validators)}: compone validators
    \item Applica a form registration
\end{itemize}

\section{Best Practices}

\begin{tcolorbox}[colback=green!10, colframe=green!60, title=Best Practices Funzioni]
\begin{itemize}
    \item[$\square$] Usa type hints per parametri e return types
    \item[$\square$] Preferisci funzioni pure quando possibile
    \item[$\square$] Una funzione = una responsabilità (SRP)
    \item[$\square$] Nomi descrittivi (verbi per azioni)
    \item[$\square$] Evita funzioni con troppe righe (max 20-30)
    \item[$\square$] Evita troppi parametri (max 3-4)
    \item[$\square$] Documenta con DocBlocks
    \item[$\square$] Return early per ridurre nesting
    \item[$\square$] Evita global, preferisci dependency injection
    \item[$\square$] Usa arrow functions per callback brevi
\end{itemize}
\end{tcolorbox}

\section{Riferimenti}

\begin{itemize}
    \item PHP Manual - Funzioni: \url{https://www.php.net/manual/en/language.functions.php}
    \item PSR Coding Standards: \url{https://www.php-fig.org/psr/}
    \item Clean Code PHP: \url{https://github.com/jupeter/clean-code-php}
\end{itemize}

\chapter{File di Testo}

\begin{tcolorbox}[title=Mappa del capitolo]
Teoria, Lettura e scrittura, Permessi, Bloccaggio, CSV, Esempi pratici, Caso di studio, Esercizi, Verifica, Riferimenti.
\end{tcolorbox}

\section{Teoria}
La gestione dei file include apertura, lettura/scrittura, permessi e lock per evitare corruzione in accessi concorrenti.

\section{Lettura e scrittura}
\begin{lstlisting}
<?php
$path = __DIR__ . '/dati.txt';
file_put_contents($path, "Prima riga\n", FILE_APPEND | LOCK_EX);
$contenuto = file_get_contents($path);
echo $contenuto;
\end{lstlisting}

\section{Gestione dei permessi}
- Verificare \texttt{is\_readable}, \texttt{is\_writable} e gestire errori con messaggi chiari.

\begin{lstlisting}
<?php
if (!is_writable($path)) {
    echo 'File non scrivibile';
}
\end{lstlisting}

\section{Bloccaggio dei file}
\begin{lstlisting}
<?php
$fp = fopen($path, 'c+');
if (flock($fp, LOCK_EX)) {
    fwrite($fp, "Riga sicura\n");
    fflush($fp);
    flock($fp, LOCK_UN);
}
fclose($fp);
\end{lstlisting}

\section{Operazioni su CSV}
\begin{lstlisting}
<?php
$fp = fopen(__DIR__ . '/utenti.csv', 'r');
while (($row = fgetcsv($fp, 0, ';')) !== false) {
    // $row è un array di campi
}
fclose($fp);

// Scrittura
$fp = fopen(__DIR__ . '/out.csv', 'w');
fputcsv($fp, ['id','name'], ';');
fclose($fp);
\end{lstlisting}

\begin{tcolorbox}[title=Best practice]
- Usare \texttt{LOCK\_EX} o \texttt{flock} per scritture concorrenti.
- Gestire gli errori con codici HTTP e log.
- Validare i dati prima di scrivere su disco.
\end{tcolorbox}

\begin{tcolorbox}[title=Errori comuni]
- Ignorare permessi e fallimenti di I/O.
- Non chiudere i file o non flushare i buffer.
- Usare separatori inconsistenti in CSV.
\end{tcolorbox}

\section{Esempi pratici}
\begin{lstlisting}
<?php
// Lettura riga-per-riga con controllo errori
$fp = fopen($path, 'r');
if (!$fp) { echo 'Apertura fallita'; }
while (($line = fgets($fp)) !== false) {
    echo htmlspecialchars($line, ENT_QUOTES | ENT_SUBSTITUTE, 'UTF-8');
}
fclose($fp);
?>
\end{lstlisting}

\section{Caso di studio}
Semplice logger applicativo su file con rotazione: quando la dimensione supera una soglia, rinomina il file con timestamp e riparte.

\section{Esercizi}
\begin{itemize}
  \item Implementa una funzione che scrive JSON su file con lock e controlla permessi.
  \item Leggi un CSV e costruisci un array associativo indicizzato per chiave.
\end{itemize}

\section{Verifica}
\begin{itemize}
  \item Qual è la differenza tra \texttt{LOCK\_EX} e \texttt{LOCK\_SH}?
  \item Perché è importante impostare un encoding coerente?
\end{itemize}

\section{Riferimenti}
\begin{itemize}
  \item Manuale PHP — File system: \url{https://www.php.net/manual/en/book.filesystem.php}
\end{itemize}

\chapter{Sessioni}
\label{cap:sessioni}

\begin{tcolorbox}[title=Mappa del capitolo]
Obiettivi, Teoria sessioni, Lifecycle, Configurazione sicura, Gestione dati, Security (hijacking/fixation/CSRF), Storage handlers, Esempi pratici, Progetti completi, Troubleshooting, Esercizi, Riferimenti.
\end{tcolorbox}

\section{Obiettivi di apprendimento}

Al termine di questo capitolo sarai in grado di:

\begin{itemize}
    \item Comprendere come funzionano le sessioni HTTP
    \item Avviare e configurare sessioni in modo sicuro
    \item Memorizzare e recuperare dati nelle sessioni
    \item Implementare autenticazione con sessioni
    \item Proteggere le sessioni da session hijacking e fixation
    \item Rigenerare ID sessione nei momenti critici
    \item Gestire logout e scadenza sessioni
    \item Utilizzare session handlers personalizzati
\end{itemize}

\section{Teoria: Cosa Sono le Sessioni}

\subsection{Il Problema dello Stato in HTTP}

HTTP è un protocollo \textbf{stateless} (senza stato): ogni richiesta è indipendente e il server non "ricorda" le richieste precedenti dello stesso client. Questo crea un problema:

\begin{tcolorbox}[colback=orange!10, colframe=orange!60, title=Problema]
Come fa un'applicazione web a ricordare che un utente ha fatto login? Come mantiene il carrello della spesa tra le pagine?
\end{tcolorbox}

Le \textbf{sessioni} risolvono questo problema permettendo di associare dati persistenti a un utente specifico attraverso molteplici richieste HTTP.

\subsection{Sessioni vs Cookie}

\begin{center}
\begin{tabular}{|l|p{5cm}|p{5cm}|}
\hline
\textbf{Aspetto} & \textbf{Cookie} & \textbf{Sessioni} \\
\hline
\textbf{Storage} & Client-side (browser) & Server-side (file/DB) \\
\hline
\textbf{Dimensione} & Limitata (4KB max) & Illimitata (limiti server) \\
\hline
\textbf{Sicurezza} & Visibili/modificabili dall'utente & Protetti lato server \\
\hline
\textbf{Durata} & Persistente (giorni/mesi) & Temporanea (minuti/ore) \\
\hline
\textbf{Uso tipico} & Preferenze UI, tracking & Autenticazione, carrello \\
\hline
\textbf{Traffico rete} & Inviato ad ogni richiesta & Solo ID inviato \\
\hline
\end{tabular}
\end{center}

\begin{nota}
Le sessioni \textbf{usano un cookie} (PHPSESSID) per identificare l'utente, ma i dati effettivi sono memorizzati sul server. Il cookie contiene solo l'ID univoco della sessione.
\end{nota}

\subsection{Come Funzionano le Sessioni PHP}

\begin{enumerate}
    \item Client richiede una pagina PHP
    \item Server chiama \texttt{session\_start()}
    \item PHP controlla se esiste cookie PHPSESSID:
    \begin{itemize}
        \item \textbf{NO}: Crea nuovo session ID, crea file sessione, invia cookie al client
        \item \textbf{SÌ}: Legge session ID dal cookie, carica dati da file sessione
    \end{itemize}
    \item Script accede/modifica \texttt{\$\_SESSION}
    \item Fine script: PHP salva \texttt{\$\_SESSION} nel file di sessione
    \item Richieste successive usano lo stesso session ID per accedere agli stessi dati
\end{enumerate}

\section{Session Lifecycle}

\begin{center}
\begin{tikzpicture}[
    node distance=1.2cm,
    box/.style={rectangle, draw, thick, minimum width=3.2cm, minimum height=0.7cm, align=center, font=\small},
    client/.style={box, fill=blue!20},
    server/.style={box, fill=orange!20},
    action/.style={box, fill=green!15},
    storage/.style={box, fill=yellow!20},
    arrow/.style={->, thick, >=stealth}
]
    % Prima richiesta
    \node[above, font=\small\bfseries] at (3.5,7.5) {1. Prima Richiesta (no session)};

    \node[client] (browser1) at (0,7) {Browser};
    \node[action] (req1) at (3.5,7) {GET /page.php};
    \node[server] (server1) at (7,7) {Server};

    \draw[arrow, blue] (browser1) -- (req1);
    \draw[arrow, blue] (req1) -- (server1);

    % Session start
    \node[action] (start) at (7,5.8) {session\_start()};
    \node[storage] (create) at (7,4.8) {Crea sessione\\sess\_abc123};
    \node[action] (setdata) at (7,3.8) {\$\_SESSION['user']=...};

    \draw[arrow] (server1) -- (start);
    \draw[arrow] (start) -- (create);
    \draw[arrow] (create) -- (setdata);

    % Response con Set-Cookie
    \node[action] (response1) at (3.5,3.8) {HTTP 200 OK\\Set-Cookie: PHPSESSID=abc123};
    \node[client] (browser2) at (0,3.8) {Browser\\(salva cookie)};

    \draw[arrow, red] (setdata) -- (response1);
    \draw[arrow, red] (response1) -- (browser2);

    % Seconda richiesta
    \node[above, font=\small\bfseries] at (3.5,2.5) {2. Richieste Successive};

    \node[client] (browser3) at (0,2) {Browser};
    \node[action] (req2) at (3.5,2) {GET /dashboard.php\\Cookie: PHPSESSID=abc123};
    \node[server] (server2) at (7,2) {Server};

    \draw[arrow, blue] (browser3) -- (req2);
    \draw[arrow, blue] (req2) -- (server2);

    % Load session
    \node[action] (start2) at (7,0.8) {session\_start()};
    \node[storage] (load) at (7,-0.2) {Legge sess\_abc123};
    \node[action] (use) at (7,-1.2) {Usa \$\_SESSION['user']};

    \draw[arrow] (server2) -- (start2);
    \draw[arrow] (start2) -- (load);
    \draw[arrow] (load) -- (use);

    % Response
    \node[action] (response2) at (3.5,-1.2) {HTTP 200 OK\\Contenuto personalizzato};
    \node[client] (browser4) at (0,-1.2) {Browser};

    \draw[arrow, red] (use) -- (response2);
    \draw[arrow, red] (response2) -- (browser4);

    % Logout
    \node[above, font=\small\bfseries] at (3.5,-2.5) {3. Logout};

    \node[client] (browser5) at (0,-3) {Browser};
    \node[action] (req3) at (3.5,-3) {POST /logout.php\\Cookie: PHPSESSID=abc123};
    \node[server] (server3) at (7,-3) {Server};

    \draw[arrow, blue] (browser5) -- (req3);
    \draw[arrow, blue] (req3) -- (server3);

    \node[action] (destroy) at (7,-4) {session\_destroy()};
    \node[storage] (delete) at (7,-5) {Elimina sess\_abc123};
    \node[action] (response3) at (3.5,-5) {HTTP 200 OK\\Set-Cookie: PHPSESSID=deleted};

    \draw[arrow] (server3) -- (destroy);
    \draw[arrow] (destroy) -- (delete);
    \draw[arrow] (delete) -- (response3);
    \draw[arrow, red] (response3) -- (0,-5);
\end{tikzpicture}
\end{center}

\section{Configurazione Sessioni Sicure}

\subsection{Avvio Sessione Base}

\begin{lstlisting}[language=PHP, caption={Avvio sessione semplice}]
<?php
// SEMPRE prima di qualsiasi output HTML
session_start();

// Ora puoi usare $_SESSION
$_SESSION['username'] = 'mario';
echo "Sessione avviata per: " . $_SESSION['username'];
?>
\end{lstlisting}

\begin{attenzione}
\texttt{session\_start()} DEVE essere chiamato:
\begin{itemize}
    \item Prima di qualsiasi output (HTML, echo, spazi)
    \item In ogni pagina che usa sessioni
    \item Una sola volta per richiesta
\end{itemize}
\end{attenzione}

\subsection{Configurazione Sicura Completa}

\begin{lstlisting}[language=PHP, caption={Configurazione sessioni sicure}]
<?php
// Configurazione PRIMA di session_start()

// Cookie settings sicuri
session_set_cookie_params([
    'lifetime' => 0,              // Session cookie (scade alla chiusura browser)
    'path' => '/',                // Valido per tutto il sito
    'domain' => '',               // Dominio corrente
    'secure' => true,             // Solo HTTPS
    'httponly' => true,           // Non accessibile da JavaScript
    'samesite' => 'Strict'        // Protezione CSRF
]);

// Impostazioni di sicurezza
ini_set('session.use_strict_mode', '1');        // Rifiuta session ID non generati dal server
ini_set('session.cookie_httponly', '1');        // Protezione XSS
ini_set('session.use_only_cookies', '1');       // Non accettare ID da URL
ini_set('session.cookie_samesite', 'Strict');   // Protezione CSRF

// Nome cookie personalizzato (opzionale, migliora sicurezza per oscurità)
session_name('MY_SESSION_ID');

// Avvia sessione
session_start();

// Timeout sessione (30 minuti)
if (isset($_SESSION['LAST_ACTIVITY']) &&
    (time() - $_SESSION['LAST_ACTIVITY'] > 1800)) {
    session_unset();
    session_destroy();
    header('Location: /login.php?timeout=1');
    exit;
}
$_SESSION['LAST_ACTIVITY'] = time();
?>
\end{lstlisting}

\subsection{Rigenerazione Session ID}

\begin{lstlisting}[language=PHP, caption={Rigenerazione ID nei momenti critici}]
<?php
session_start();

// Dopo login SUCCESS
function login_user($user_id) {
    // PRIMA: Rigenera ID per prevenire session fixation
    session_regenerate_id(true);  // true = elimina vecchio file sessione

    // POI: Imposta dati utente
    $_SESSION['user_id'] = $user_id;
    $_SESSION['logged_in'] = true;
    $_SESSION['login_time'] = time();
    $_SESSION['ip_address'] = $_SERVER['REMOTE_ADDR'];
    $_SESSION['user_agent'] = $_SERVER['HTTP_USER_AGENT'];
}

// Dopo cambio privilegi (es. da user a admin)
function elevate_to_admin() {
    session_regenerate_id(true);
    $_SESSION['is_admin'] = true;
}

// Periodicamente (ogni 15 minuti)
if (!isset($_SESSION['CREATED'])) {
    $_SESSION['CREATED'] = time();
} else if (time() - $_SESSION['CREATED'] > 900) {
    session_regenerate_id(true);
    $_SESSION['CREATED'] = time();
}
?>
\end{lstlisting}

\begin{tcolorbox}[colback=blue!10, colframe=blue!60, title=Quando rigenerare session ID]
\begin{itemize}
    \item \textbf{Dopo login}: Previene session fixation
    \item \textbf{Dopo cambio privilegi}: User→Admin, Guest→User
    \item \textbf{Periodicamente}: Ogni 15-30 minuti per sessioni lunghe
    \item \textbf{Prima di operazioni sensibili}: Cambio password, pagamenti
\end{itemize}
\end{tcolorbox}

\section{Gestione Dati Sessione}

\subsection{Memorizzazione Dati}

\begin{lstlisting}[language=PHP, caption={Tipi di dati in sessione}]
<?php
session_start();

// Scalari
$_SESSION['user_id'] = 123;
$_SESSION['username'] = 'mario_rossi';
$_SESSION['is_admin'] = true;

// Array
$_SESSION['permissions'] = ['read', 'write', 'delete'];
$_SESSION['cart'] = [
    'product_1' => ['qty' => 2, 'price' => 29.99],
    'product_5' => ['qty' => 1, 'price' => 149.99]
];

// Oggetti (devono essere serializzabili)
class User {
    public $id;
    public $name;
    public function __construct($id, $name) {
        $this->id = $id;
        $this->name = $name;
    }
}
$_SESSION['user_object'] = new User(123, 'Mario');

// Dati temporanei (flash messages)
$_SESSION['flash_message'] = 'Operazione completata con successo!';
?>
\end{lstlisting}

\subsection{Lettura Sicura Dati}

\begin{lstlisting}[language=PHP, caption={Accesso sicuro a dati sessione}]
<?php
session_start();

// Verifica esistenza con isset
if (isset($_SESSION['user_id'])) {
    $user_id = $_SESSION['user_id'];
} else {
    // Reindirizza al login
    header('Location: /login.php');
    exit;
}

// Operatore null coalescing (??)
$username = $_SESSION['username'] ?? 'Guest';
$is_admin = $_SESSION['is_admin'] ?? false;

// Validazione tipo
function get_user_id() {
    if (!isset($_SESSION['user_id'])) {
        return null;
    }

    $user_id = $_SESSION['user_id'];

    // Valida che sia un intero positivo
    if (!is_numeric($user_id) || $user_id <= 0) {
        // Sessione compromessa!
        session_destroy();
        return null;
    }

    return (int)$user_id;
}

// Flash messages (mostra una volta, poi rimuovi)
function get_flash_message() {
    $message = $_SESSION['flash_message'] ?? null;
    unset($_SESSION['flash_message']);
    return $message;
}
?>
\end{lstlisting}

\subsection{Rimozione Dati}

\begin{lstlisting}[language=PHP, caption={Rimuovere dati da sessione}]
<?php
session_start();

// Rimuovi singola variabile
unset($_SESSION['cart']);

// Rimuovi tutte le variabili (ma mantieni sessione attiva)
session_unset();  // equivale a $_SESSION = []

// Distruggi sessione completamente
session_destroy();

// Elimina anche il cookie
setcookie(session_name(), '', time() - 3600, '/');
?>
\end{lstlisting}

\section{Sicurezza Sessioni}

\subsection{Session Hijacking (Dirottamento)}

\textbf{Attacco}: Un attaccante ruba il session ID di una vittima e lo usa per impersonarla.

\textbf{Vettori di attacco}:
\begin{itemize}
    \item XSS: JavaScript ruba cookie (\texttt{document.cookie})
    \item Network sniffing: Intercettazione traffico HTTP
    \item Session ID in URL: \texttt{/page.php?PHPSESSID=abc123}
\end{itemize}

\textbf{Protezioni}:

\begin{lstlisting}[language=PHP, caption={Protezione da session hijacking}]
<?php
session_start();

// 1. HttpOnly cookie (previene accesso JavaScript)
ini_set('session.cookie_httponly', '1');

// 2. Secure cookie (solo HTTPS)
ini_set('session.cookie_secure', '1');

// 3. Valida IP e User-Agent (OPZIONALE, problemi con proxy)
function validate_session() {
    // Verifica IP
    if (isset($_SESSION['ip_address'])) {
        if ($_SESSION['ip_address'] !== $_SERVER['REMOTE_ADDR']) {
            session_destroy();
            die('Session hijacking detected!');
        }
    } else {
        $_SESSION['ip_address'] = $_SERVER['REMOTE_ADDR'];
    }

    // Verifica User-Agent
    if (isset($_SESSION['user_agent'])) {
        if ($_SESSION['user_agent'] !== $_SERVER['HTTP_USER_AGENT']) {
            session_destroy();
            die('Session hijacking detected!');
        }
    } else {
        $_SESSION['user_agent'] = $_SERVER['HTTP_USER_AGENT'];
    }
}

validate_session();
?>
\end{lstlisting}

\begin{attenzione}
Validare IP può causare problemi con:
\begin{itemize}
    \item Utenti mobile (IP cambia tra WiFi e 4G)
    \item Proxy/Load balancer aziendali
    \item VPN che ruotano IP
\end{itemize}
Valuta se il tuo caso d'uso lo richiede.
\end{attenzione}

\subsection{Session Fixation}

\textbf{Attacco}: L'attaccante forza la vittima a usare un session ID conosciuto dall'attaccante.

\textbf{Scenario}:
\begin{enumerate}
    \item Attaccante ottiene session ID: \texttt{PHPSESSID=malicious123}
    \item Attaccante invia link alla vittima: \texttt{http://site.com/?PHPSESSID=malicious123}
    \item Vittima fa login usando quel session ID
    \item Attaccante usa \texttt{malicious123} per accedere come la vittima
\end{enumerate}

\textbf{Protezioni}:

\begin{lstlisting}[language=PHP, caption={Protezione da session fixation}]
<?php
// 1. Use strict mode (rifiuta ID non generati dal server)
ini_set('session.use_strict_mode', '1');

// 2. Accetta ID SOLO da cookie (non da GET/POST)
ini_set('session.use_only_cookies', '1');

// 3. RIGENERA ID dopo login
function secure_login($username, $password) {
    // Verifica credenziali
    $user = authenticate($username, $password);

    if ($user) {
        // RIGENERA ID PRIMA di impostare dati sensibili
        session_regenerate_id(true);

        $_SESSION['user_id'] = $user['id'];
        $_SESSION['logged_in'] = true;

        return true;
    }

    return false;
}
?>
\end{lstlisting}

\subsection{CSRF con Sessioni}

Anche con sessioni sicure, sei vulnerabile a CSRF. Usa token CSRF:

\begin{lstlisting}[language=PHP, caption={CSRF protection con sessioni}]
<?php
session_start();

// Genera token CSRF
function generate_csrf_token() {
    if (!isset($_SESSION['csrf_token'])) {
        $_SESSION['csrf_token'] = bin2hex(random_bytes(32));
    }
    return $_SESSION['csrf_token'];
}

// Verifica token CSRF
function verify_csrf_token($token) {
    return isset($_SESSION['csrf_token']) &&
           hash_equals($_SESSION['csrf_token'], $token);
}

// Form HTML
$csrf_token = generate_csrf_token();
?>
<form method="POST" action="/transfer.php">
    <input type="hidden" name="csrf_token" value="<?= htmlspecialchars($csrf_token) ?>">
    <input type="text" name="amount">
    <button type="submit">Transfer</button>
</form>

<?php
// transfer.php
session_start();

if ($_SERVER['REQUEST_METHOD'] === 'POST') {
    $token = $_POST['csrf_token'] ?? '';

    if (!verify_csrf_token($token)) {
        die('CSRF attack detected!');
    }

    // Processa trasferimento in sicurezza
    process_transfer($_POST['amount']);
}
?>
\end{lstlisting}

\section{Logout Sicuro}

\begin{lstlisting}[language=PHP, caption={Logout completo e sicuro}]
<?php
// logout.php
session_start();

// 1. Cancella tutte le variabili di sessione
$_SESSION = [];

// 2. Elimina il cookie di sessione
if (isset($_COOKIE[session_name()])) {
    $params = session_get_cookie_params();
    setcookie(
        session_name(),
        '',
        time() - 42000,
        $params['path'],
        $params['domain'],
        $params['secure'],
        $params['httponly']
    );
}

// 3. Distruggi la sessione lato server
session_destroy();

// 4. Rigenera session ID (per sicurezza)
session_start();
session_regenerate_id(true);

// 5. Reindirizza al login
header('Location: /login.php?logout=success');
exit;
?>
\end{lstlisting}

\section{Esempio Completo: Sistema Autenticazione}

\begin{lstlisting}[language=PHP, caption={config.php - Configurazione}]
<?php
// config.php
function init_secure_session() {
    // Configurazione cookie
    session_set_cookie_params([
        'lifetime' => 0,
        'path' => '/',
        'domain' => '',
        'secure' => isset($_SERVER['HTTPS']),
        'httponly' => true,
        'samesite' => 'Strict'
    ]);

    // Impostazioni sicurezza
    ini_set('session.use_strict_mode', '1');
    ini_set('session.cookie_httponly', '1');
    ini_set('session.use_only_cookies', '1');
    ini_set('session.cookie_samesite', 'Strict');

    session_start();

    // Timeout 30 minuti
    if (isset($_SESSION['LAST_ACTIVITY']) &&
        (time() - $_SESSION['LAST_ACTIVITY'] > 1800)) {
        session_unset();
        session_destroy();
        session_start();
        $_SESSION['timeout'] = true;
    }
    $_SESSION['LAST_ACTIVITY'] = time();

    // Rigenera periodicamente
    if (!isset($_SESSION['CREATED'])) {
        $_SESSION['CREATED'] = time();
    } else if (time() - $_SESSION['CREATED'] > 900) {
        session_regenerate_id(true);
        $_SESSION['CREATED'] = time();
    }
}

function is_logged_in() {
    return isset($_SESSION['user_id']) && $_SESSION['logged_in'] === true;
}

function require_login() {
    if (!is_logged_in()) {
        header('Location: /login.php');
        exit;
    }
}

function get_user_id() {
    return $_SESSION['user_id'] ?? null;
}
?>
\end{lstlisting}

\begin{lstlisting}[language=PHP, caption={login.php - Pagina login}]
<?php
// login.php
require 'config.php';
init_secure_session();

$error = null;

if ($_SERVER['REQUEST_METHOD'] === 'POST') {
    $username = $_POST['username'] ?? '';
    $password = $_POST['password'] ?? '';

    // Connessione DB (esempio)
    $pdo = new PDO('mysql:host=localhost;dbname=app', 'user', 'pass');

    // Query preparata
    $stmt = $pdo->prepare('SELECT id, password_hash FROM users WHERE username = ?');
    $stmt->execute([$username]);
    $user = $stmt->fetch();

    if ($user && password_verify($password, $user['password_hash'])) {
        // Login SUCCESS
        session_regenerate_id(true);  // Previene session fixation

        $_SESSION['user_id'] = $user['id'];
        $_SESSION['logged_in'] = true;
        $_SESSION['login_time'] = time();

        header('Location: /dashboard.php');
        exit;
    } else {
        $error = 'Username o password errati';
    }
}
?>
<!DOCTYPE html>
<html>
<head>
    <title>Login</title>
</head>
<body>
    <h1>Login</h1>

    <?php if ($error): ?>
        <div class="error"><?= htmlspecialchars($error) ?></div>
    <?php endif; ?>

    <?php if (isset($_SESSION['timeout'])): ?>
        <div class="warning">Sessione scaduta. Effettua nuovamente il login.</div>
        <?php unset($_SESSION['timeout']); ?>
    <?php endif; ?>

    <form method="POST">
        <label>Username: <input type="text" name="username" required></label><br>
        <label>Password: <input type="password" name="password" required></label><br>
        <button type="submit">Login</button>
    </form>
</body>
</html>
\end{lstlisting}

\begin{lstlisting}[language=PHP, caption={dashboard.php - Area riservata}]
<?php
// dashboard.php
require 'config.php';
init_secure_session();
require_login();  // Reindirizza se non loggato

$user_id = get_user_id();

// Carica dati utente
$pdo = new PDO('mysql:host=localhost;dbname=app', 'user', 'pass');
$stmt = $pdo->prepare('SELECT username, email FROM users WHERE id = ?');
$stmt->execute([$user_id]);
$user = $stmt->fetch();
?>
<!DOCTYPE html>
<html>
<head>
    <title>Dashboard</title>
</head>
<body>
    <h1>Benvenuto, <?= htmlspecialchars($user['username']) ?></h1>

    <p>Email: <?= htmlspecialchars($user['email']) ?></p>

    <p>Loggato da:
        <?= htmlspecialchars(date('H:i:s', $_SESSION['login_time'])) ?>
    </p>

    <a href="/logout.php">Logout</a>
</body>
</html>
\end{lstlisting}

\section{Troubleshooting Sessioni}

\subsection{Errore: "Headers already sent"}

\begin{lstlisting}[language=PHP]
Warning: session_start(): Cannot send session cookie - headers already sent
\end{lstlisting}

\textbf{Causa}: Output HTML/spazi prima di \texttt{session\_start()}.

\textbf{Soluzione}:
\begin{itemize}
    \item Metti \texttt{session\_start()} come PRIMA istruzione
    \item Nessuno spazio/newline prima di \texttt{<?php}
    \item Nessun \texttt{echo} o HTML prima
    \item Verifica BOM in UTF-8 (usa UTF-8 without BOM)
\end{itemize}

\subsection{Sessione Non Persiste}

\textbf{Possibili cause}:
\begin{itemize}
    \item Cookie disabilitati nel browser
    \item Path cookie errato
    \item Domain cookie non corretto
    \item Permessi directory sessioni (di solito \texttt{/tmp})
\end{itemize}

\textbf{Debug}:
\begin{lstlisting}[language=PHP]
<?php
session_start();

echo "Session ID: " . session_id() . "<br>";
echo "Session Name: " . session_name() . "<br>";
echo "Session Save Path: " . session_save_path() . "<br>";
echo "Cookie Params: ";
print_r(session_get_cookie_params());

$_SESSION['test'] = 'value';
echo "<br>Session Data: ";
print_r($_SESSION);
?>
\end{lstlisting}

\subsection{Sessioni su Server Condiviso}

Su hosting condiviso, directory sessioni è condivisa. Usa directory personalizzata:

\begin{lstlisting}[language=PHP]
<?php
$custom_path = __DIR__ . '/sessions';

// Crea directory se non esiste
if (!is_dir($custom_path)) {
    mkdir($custom_path, 0700, true);
}

session_save_path($custom_path);
session_start();
?>
\end{lstlisting}

\section{Esercizi}

\subsection{Esercizio 1 (Base)}

Crea un sistema di login semplice con:
\begin{itemize}
    \item Pagina login.php con form
    \item Validazione username/password hardcoded
    \item Sessione che memorizza username
    \item Pagina dashboard.php accessibile solo se loggati
    \item Logout che invalida sessione
\end{itemize}

\subsection{Esercizio 2 (Intermedio)}

Estendi l'esercizio 1 aggiungendo:
\begin{itemize}
    \item Timeout sessione dopo 10 minuti di inattività
    \item Messaggio flash "Login effettuato con successo"
    \item Contatore numero di pagine visitate nella sessione
    \item Log timestamp ultimo accesso
\end{itemize}

\subsection{Esercizio 3 (Avanzato)}

Crea sistema autenticazione sicuro con:
\begin{itemize}
    \item Database MySQL con tabella users (username, password\_hash, email)
    \item Registrazione con password hashing (password\_hash)
    \item Login con rigenerazione session ID
    \item Protezione CSRF su tutti i form
    \item Validazione IP e User-Agent
    \item Logout completo
\end{itemize}

\section{Progetto: Carrello E-commerce}

Implementa un carrello della spesa con sessioni:

\textbf{Funzionalità richieste}:
\begin{itemize}
    \item Aggiungi prodotto al carrello (ID, quantità, prezzo)
    \item Visualizza carrello con totale
    \item Aggiorna quantità prodotto
    \item Rimuovi prodotto
    \item Svuota carrello
    \item Carrello persiste tra le pagine
    \item Timeout carrello dopo 1 ora
\end{itemize}

\textbf{Sicurezza}:
\begin{itemize}
    \item Valida ID prodotto (esiste nel DB)
    \item Valida quantità (> 0, < 100)
    \item Valida prezzo lato server (non fidarsi del client)
    \item Protezione CSRF
\end{itemize}

\section{Verifica}

\begin{enumerate}
    \item Qual è la differenza tra cookie e sessioni?
    \item Perché è importante rigenerare session ID dopo login?
    \item Cosa fa \texttt{session\_regenerate\_id(true)}? Cosa significa il parametro \texttt{true}?
    \item Quali sono i tre flag di sicurezza principali per cookie di sessione?
    \item Come si protegge da session fixation?
    \item Come si protegge da session hijacking?
    \item Quando usare \texttt{session\_unset()} vs \texttt{session\_destroy()}?
    \item Perché \texttt{session\_start()} deve essere la prima istruzione?
\end{enumerate}

\section{Best Practices Riepilogo}

\begin{tcolorbox}[colback=green!10, colframe=green!60, title=Checklist Sessioni Sicure]
\begin{itemize}
    \item[$\square$] \texttt{session\_start()} come prima istruzione
    \item[$\square$] Cookie con HttpOnly, Secure, SameSite
    \item[$\square$] \texttt{use\_strict\_mode} = 1
    \item[$\square$] \texttt{use\_only\_cookies} = 1
    \item[$\square$] Rigenerare ID dopo login
    \item[$\square$] Rigenerare ID dopo cambio privilegi
    \item[$\square$] Timeout sessione (30 minuti)
    \item[$\square$] Logout completo (unset + destroy + cookie delete)
    \item[$\square$] Protezione CSRF con token
    \item[$\square$] Solo HTTPS in produzione
    \item[$\square$] Non memorizzare password in sessione
    \item[$\square$] Validare dati sessione (tipo, range)
\end{itemize}
\end{tcolorbox}

\section{Riferimenti}

\begin{itemize}
    \item PHP Manual - Sessions: \url{https://www.php.net/manual/en/book.session.php}
    \item OWASP Session Management Cheat Sheet: \url{https://cheatsheetseries.owasp.org/cheatsheets/Session_Management_Cheat_Sheet.html}
    \item PHP: The Right Way - Security: \url{https://phptherightway.com/#security}
\end{itemize}

\chapter{Database con MySQLi}\label{cap:db_mysqli}

\begin{tcolorbox}[title=Mappa del capitolo]
Teoria, Connessione, Query preparate, Transazioni, Gestione errori, Ottimizzazione, Esempi pratici, Caso di studio, Esercizi, Verifica, Riferimenti.
\end{tcolorbox}

\section{Teoria}
L'estensione \texttt{mysqli} fornisce API procedurali e a oggetti per interagire con MySQL/MariaDB, con supporto a query preparate e transazioni.

\section{Connessione al database}
\begin{lstlisting}
<?php
$mysqli = new mysqli('localhost', 'user', 'pass', 'db');
if ($mysqli->connect_errno) {
    error_log('Connessione fallita: ' . $mysqli->connect_error);
    exit('DB non disponibile');
}
\end{lstlisting}

\section{Query preparate}
\begin{lstlisting}
<?php
$stmt = $mysqli->prepare('SELECT id, name FROM users WHERE email = ?');
$stmt->bind_param('s', $email);
$stmt->execute();
$result = $stmt->get_result();
while ($row = $result->fetch_assoc()) {
    echo $row['name'];
}
$stmt->close();
\end{lstlisting}

\section{Query non sicure (senza bind) e SQL injection}
\begin{lstlisting}
<?php
// ESEMPIO NON SICURO: concatenazione di input utente
$email = isset($_GET['email']) ? (string)$_GET['email'] : '';
$sql = "SELECT id, name FROM users WHERE email = '" . $email . "'";
$res = $mysqli->query($sql);
if ($res) {
    while ($row = $res->fetch_assoc()) { echo $row['name']; }
}
// Input malevolo: x' OR '1'='1  -> restituisce TUTTI gli utenti
// $sql diventa: SELECT id, name FROM users WHERE email = 'x' OR '1'='1'
?>
\end{lstlisting}

\subsection*{Rifacimento sicuro con parametri bind}
\begin{lstlisting}
<?php
// VERSIONE SICURA: placeholder e bind_param
$email = isset($_GET['email']) ? (string)$_GET['email'] : '';
$stmt = $mysqli->prepare('SELECT id, name FROM users WHERE email = ?');
if (!$stmt) { exit('Errore prepare'); }
$stmt->bind_param('s', $email);
$stmt->execute();
$result = $stmt->get_result();
while ($row = $result->fetch_assoc()) { echo $row['name']; }
$stmt->close();
?>
\end{lstlisting}

\begin{tcolorbox}[title=Dimostrazione]
La versione non sicura consente l'iniezione di frammenti SQL in \verb|$email|, alterando la logica della WHERE e restituendo risultati non previsti. Con i parametri bind, il valore viene inviato separatamente dall'istruzione SQL e trattato come dato, eliminando la vulnerabilità.
\end{tcolorbox}

\section{Transazioni}
\begin{lstlisting}
<?php
$mysqli->begin_transaction();
try {
    $stmt = $mysqli->prepare('UPDATE accounts SET balance = balance - ? WHERE id = ?');
    $stmt->bind_param('di', $amount, $fromId);
    $stmt->execute();

    $stmt = $mysqli->prepare('UPDATE accounts SET balance = balance + ? WHERE id = ?');
    $stmt->bind_param('di', $amount, $toId);
    $stmt->execute();

    $mysqli->commit();
} catch (Throwable $e) {
    $mysqli->rollback();
    error_log($e->getMessage());
}
\end{lstlisting}

\section{Gestione errori}
- Controllare \verb|$mysqli->errno| e \verb|$mysqli->error|; loggare errori e non esporre dettagli sensibili all'utente.
- Gestire eccezioni e timeouts; impostare charset \texttt{utf8mb4}.

\begin{lstlisting}
<?php
$mysqli->set_charset('utf8mb4');
if (!$mysqli->query('SET NAMES utf8mb4')) {
    error_log($mysqli->error);
}
\end{lstlisting}

\section{Ottimizzazione delle query}
- Indici appropriati, evitare SELECT * nelle tabelle grandi.
- Limitare i risultati, usare paginazione.
- Misurare con \texttt{EXPLAIN} e profilarne i piani.

\begin{tcolorbox}[title=Best practice]
- Usare query preparate per prevenire SQL injection.
- Gestire transazioni per operazioni atomiche.
- Impostare charset \texttt{utf8mb4} e collation coerente.
- Centralizzare la gestione della connessione e pooling se necessario.
\end{tcolorbox}

\begin{tcolorbox}[title=Errori comuni]
- Concatenare input utente nelle query.
- Dimenticare \texttt{commit}/\texttt{rollback} e lasciare transazioni aperte.
- Ignorare settaggio del charset e problemi di encoding.
\end{tcolorbox}

\section{Esempi pratici}
\begin{lstlisting}
<?php
// Inserimento utente con validazioni minime
$stmt = $mysqli->prepare('INSERT INTO users(username,email,password_hash) VALUES (?,?,?)');
$stmt->bind_param('sss', $username, $email, $hash);
$stmt->execute();
if ($stmt->errno) { error_log($stmt->error); }
$stmt->close();
?>
\end{lstlisting}

\section{Caso di studio}
Registrazione e login: creazione utente, ricerca per username/email e verifica hash. Discutere error handling e messaggi user-friendly.

\section{Esercizi}
\begin{itemize}
  \item Implementa paginazione server-side con LIMIT/OFFSET.
  \item Aggiungi indici e confronta piani di esecuzione con \texttt{EXPLAIN}.
\end{itemize}

\section{Verifica}
\begin{itemize}
  \item Quali rischi introduce la concatenazione di input utente?
  \item Perché \texttt{utf8mb4} è preferibile a \texttt{utf8}?
\end{itemize}

\section{Riferimenti}
\begin{itemize}
  \item Manuale PHP — MySQLi: \url{https://www.php.net/mysqli}
  \item MySQL Doc — Prepared Statements: \url{https://dev.mysql.com/doc/}
\end{itemize}


\backmatter
% Appendice - QR Code Risorse
\chapter*{Appendice: QR Code e Risorse}
\addcontentsline{toc}{chapter}{Appendice: QR Code e Risorse}

\begin{tcolorbox}[title=Nota]
Scansiona i QR code per aprire rapidamente le risorse online: documentazione, tutorial e strumenti.
\end{tcolorbox}

\section*{Documentazione Ufficiale}
\begin{itemize}
    \item Java SE Documentation: \href{https://docs.oracle.com/javase/}{docs.oracle.com}\newline
    \qrcode{https://docs.oracle.com/javase/}
    \item Java Tutorials (Oracle): \href{https://docs.oracle.com/javase/tutorial/}{Java Tutorials}\newline
    \qrcode{https://docs.oracle.com/javase/tutorial/}
    \item Collections Framework Guide: \href{https://docs.oracle.com/javase/8/docs/technotes/guides/collections/}{Collections Guide}\newline
    \qrcode{https://docs.oracle.com/javase/8/docs/technotes/guides/collections/}
\end{itemize}

\section*{Compilatori Online}
\begin{itemize}
    \item JDoodle Java Online: \href{https://www.jdoodle.com/online-java-compiler}{JDoodle}\newline
    \qrcode{https://www.jdoodle.com/online-java-compiler}
    \item Repl.it: \href{https://replit.com/languages/java10}{Repl.it Java}\newline
    \qrcode{https://replit.com/languages/java10}
\end{itemize}

\section*{Risorse Utili}
\begin{itemize}
    \item StackOverflow Java: \href{https://stackoverflow.com/questions/tagged/java}{SO Java}\newline
    \qrcode{https://stackoverflow.com/questions/tagged/java}
    \item GitHub Trending Java: \href{https://github.com/trending/java}{Trending Java}\newline
    \qrcode{https://github.com/trending/java}
\end{itemize}


% Bibliografia (PHP)
\chapter*{Bibliografia e Risorse}
\addcontentsline{toc}{chapter}{Bibliografia e Risorse}

\section*{Manuali Ufficiali}

\begin{enumerate}
    \item \textbf{PHP Manual} \\
    Documentazione ufficiale completa \\
    \url{https://www.php.net/manual/en/}

    \item \textbf{PHP: The Right Way} \\
    Best practices PHP moderne \\
    \url{https://phptherightway.com/}

    \item \textbf{PHP Standards Recommendations (PSR)} \\
    PHP-FIG coding standards \\
    \url{https://www.php-fig.org/psr/}
\end{enumerate}

\section*{Libri di Testo}

\begin{enumerate}
    \item \textbf{PHP \& MySQL: Novice to Ninja} \\
    Kevin Yank \\
    SitePoint, 7th Edition, 2021 \\
    ISBN: 978-1925836387

    \item \textbf{Modern PHP: New Features and Good Practices} \\
    Josh Lockhart \\
    O'Reilly Media, 2015 \\
    ISBN: 978-1491905012

    \item \textbf{PHP Objects, Patterns, and Practice} \\
    Matt Zandstra \\
    Apress, 6th Edition, 2021 \\
    ISBN: 978-1484267905

    \item \textbf{Learning PHP, MySQL \& JavaScript} \\
    Robin Nixon \\
    O'Reilly Media, 6th Edition, 2021 \\
    ISBN: 978-1492093824

    \item \textbf{PHP Cookbook} \\
    David Sklar, Adam Trachtenberg \\
    O'Reilly Media, 4th Edition, 2020 \\
    ISBN: 978-1098121327
\end{enumerate}

\section*{Sicurezza Web}

\subsection*{OWASP Resources}

\begin{itemize}
    \item \textbf{OWASP Top 10} \\
    \url{https://owasp.org/Top10/} \\
    Le 10 vulnerabilità web più critiche (aggiornato 2021)

    \item \textbf{OWASP PHP Security Cheat Sheet} \\
    \url{https://cheatsheetseries.owasp.org/cheatsheets/PHP_Configuration_Cheat_Sheet.html}

    \item \textbf{OWASP Testing Guide} \\
    \url{https://owasp.org/www-project-web-security-testing-guide/}

    \item \textbf{OWASP Secure Coding Practices} \\
    \url{https://owasp.org/www-project-secure-coding-practices-quick-reference-guide/}
\end{itemize}

\subsection*{Libri Sicurezza}

\begin{enumerate}
    \item \textbf{The Tangled Web: A Guide to Securing Modern Web Applications} \\
    Michal Zalewski \\
    No Starch Press, 2011 \\
    ISBN: 978-1593273880

    \item \textbf{Web Application Security} \\
    Andrew Hoffman \\
    O'Reilly Media, 2020 \\
    ISBN: 978-1492053118
\end{enumerate}

\section*{Framework PHP}

\subsection*{Laravel}

\begin{itemize}
    \item \textbf{Documentazione ufficiale}: \url{https://laravel.com/docs}
    \item \textbf{Laracasts}: Video tutorial premium: \url{https://laracasts.com/}
    \item \textbf{Laravel News}: \url{https://laravel-news.com/}
\end{itemize}

\subsection*{Symfony}

\begin{itemize}
    \item \textbf{Documentazione}: \url{https://symfony.com/doc/current/index.html}
    \item \textbf{Symfony Blog}: \url{https://symfony.com/blog/}
\end{itemize}

\subsection*{Altri Framework}

\begin{itemize}
    \item \textbf{CodeIgniter}: \url{https://codeigniter.com/}
    \item \textbf{Slim Framework}: \url{https://www.slimframework.com/} (microframework)
    \item \textbf{Laminas (ex Zend)}: \url{https://getlaminas.org/}
    \item \textbf{CakePHP}: \url{https://cakephp.org/}
\end{itemize}

\section*{Tutorial e Corsi Online}

\subsection*{Tutorial Gratuiti}

\begin{itemize}
    \item \textbf{W3Schools PHP Tutorial} \\
    \url{https://www.w3schools.com/php/} \\
    Tutorial interattivo per principianti

    \item \textbf{PHP Tutorial - Tutorialspoint} \\
    \url{https://www.tutorialspoint.com/php/}

    \item \textbf{Learn PHP - Codecademy} \\
    \url{https://www.codecademy.com/learn/learn-php}

    \item \textbf{PHP for Beginners - Laracasts} \\
    \url{https://laracasts.com/series/php-for-beginners} (gratuito)
\end{itemize}

\subsection*{Piattaforme Video}

\begin{itemize}
    \item \textbf{Laracasts}: PHP e Laravel (subscription) \\
    \url{https://laracasts.com/}

    \item \textbf{Udemy}: Corsi PHP vari \\
    \url{https://www.udemy.com/topic/php/}

    \item \textbf{Pluralsight}: Percorsi PHP professionali \\
    \url{https://www.pluralsight.com/paths/php}

    \item \textbf{YouTube Channels}:
    \begin{itemize}
        \item Traversy Media: \url{https://www.youtube.com/c/TraversyMedia}
        \item The Net Ninja: \url{https://www.youtube.com/c/TheNetNinja}
        \item ProgramWithGio: \url{https://www.youtube.com/c/ProgramWithGio}
    \end{itemize}
\end{itemize}

\section*{Tool e Ambiente di Sviluppo}

\subsection*{Ambienti Locali}

\begin{itemize}
    \item \textbf{XAMPP}: \url{https://www.apachefriends.org/} (Windows/Mac/Linux)
    \item \textbf{MAMP}: \url{https://www.mamp.info/} (Mac/Windows)
    \item \textbf{Laragon}: \url{https://laragon.org/} (Windows, moderno)
    \item \textbf{Docker con PHP}: \url{https://hub.docker.com/_/php}
\end{itemize}

\subsection*{IDE e Editor}

\begin{itemize}
    \item \textbf{PhpStorm}: \url{https://www.jetbrains.com/phpstorm/} (professionale, licenza studenti gratuita)
    \item \textbf{Visual Studio Code} con estensioni:
    \begin{itemize}
        \item PHP Intelephense: \url{https://marketplace.visualstudio.com/items?itemName=bmewburn.vscode-intelephense-client}
        \item PHP Debug: \url{https://marketplace.visualstudio.com/items?itemName=xdebug.php-debug}
    \end{itemize}
    \item \textbf{Sublime Text}: \url{https://www.sublimetext.com/}
    \item \textbf{NetBeans}: \url{https://netbeans.apache.org/}
\end{itemize}

\subsection*{Debugging e Profiling}

\begin{itemize}
    \item \textbf{Xdebug}: \url{https://xdebug.org/} (debugger PHP)
    \item \textbf{Blackfire.io}: \url{https://blackfire.io/} (profiling performance)
    \item \textbf{New Relic}: \url{https://newrelic.com/} (monitoring APM)
    \item \textbf{PHPStan}: \url{https://phpstan.org/} (static analysis)
    \item \textbf{Psalm}: \url{https://psalm.dev/} (static analysis)
\end{itemize}

\subsection*{Testing}

\begin{itemize}
    \item \textbf{PHPUnit}: \url{https://phpunit.de/} (unit testing)
    \item \textbf{Pest PHP}: \url{https://pestphp.com/} (testing elegante)
    \item \textbf{Codeception}: \url{https://codeception.com/} (full-stack testing)
    \item \textbf{Behat}: \url{https://docs.behat.org/} (BDD)
\end{itemize}

\subsection*{Dependency Management}

\begin{itemize}
    \item \textbf{Composer}: \url{https://getcomposer.org/} (package manager PHP)
    \item \textbf{Packagist}: \url{https://packagist.org/} (repository Composer)
\end{itemize}

\section*{Community e Forum}

\begin{itemize}
    \item \textbf{Stack Overflow} \\
    Tag [php]: \url{https://stackoverflow.com/questions/tagged/php}

    \item \textbf{Reddit r/PHP} \\
    \url{https://www.reddit.com/r/PHP/}

    \item \textbf{PHP Developers Discord} \\
    Community Discord attiva

    \item \textbf{Laravel Discord} \\
    \url{https://discord.gg/laravel}

    \item \textbf{PHP User Groups} \\
    \url{https://www.php.net/ug.php} (gruppi locali)

    \item \textbf{Conferenze PHP}:
    \begin{itemize}
        \item PHP[World]
        \item Laracon
        \item SymfonyCon
        \item PHP UK Conference
    \end{itemize}
\end{itemize}

\section*{Database e Storage}

\begin{itemize}
    \item \textbf{MySQL Documentation}: \url{https://dev.mysql.com/doc/}
    \item \textbf{MariaDB Knowledge Base}: \url{https://mariadb.com/kb/}
    \item \textbf{PostgreSQL Documentation}: \url{https://www.postgresql.org/docs/}
    \item \textbf{PDO Manual}: \url{https://www.php.net/manual/en/book.pdo.php}
    \item \textbf{MySQLi Manual}: \url{https://www.php.net/manual/en/book.mysqli.php}
    \item \textbf{Redis}: \url{https://redis.io/} (caching)
    \item \textbf{Memcached}: \url{https://www.memcached.org/}
\end{itemize}

\section*{CMS e E-commerce}

\begin{itemize}
    \item \textbf{WordPress}: \url{https://wordpress.org/}
    \begin{itemize}
        \item Codex: \url{https://codex.wordpress.org/}
        \item Developer Handbook: \url{https://developer.wordpress.org/}
    \end{itemize}

    \item \textbf{Drupal}: \url{https://www.drupal.org/}
    
    \item \textbf{Joomla}: \url{https://www.joomla.org/}

    \item \textbf{Magento}: \url{https://magento.com/} (e-commerce)

    \item \textbf{WooCommerce}: \url{https://woocommerce.com/} (WordPress e-commerce)

    \item \textbf{PrestaShop}: \url{https://www.prestashop.com/}
\end{itemize}

\section*{API e Web Services}

\begin{itemize}
    \item \textbf{REST API Tutorial}: \url{https://restfulapi.net/}
    \item \textbf{GraphQL PHP}: \url{https://webonyx.github.io/graphql-php/}
    \item \textbf{Guzzle HTTP Client}: \url{https://docs.guzzlephp.org/}
    \item \textbf{cURL Manual}: \url{https://www.php.net/manual/en/book.curl.php}
    \item \textbf{JSON Web Tokens (JWT)}: \url{https://jwt.io/}
\end{itemize}

\section*{Deployment e DevOps}

\begin{itemize}
    \item \textbf{Deployer}: \url{https://deployer.org/} (deployment tool)
    \item \textbf{Laravel Forge}: \url{https://forge.laravel.com/} (server management)
    \item \textbf{Laravel Vapor}: \url{https://vapor.laravel.com/} (serverless)
    \item \textbf{Docker Compose per PHP}: Tutorial e best practices
    \item \textbf{GitHub Actions}: \url{https://github.com/features/actions} (CI/CD)
    \item \textbf{GitLab CI/CD}: \url{https://docs.gitlab.com/ee/ci/}
\end{itemize}

\section*{Sicurezza Avanzata}

\subsection*{Penetration Testing}

\begin{itemize}
    \item \textbf{DVWA (Damn Vulnerable Web Application)} \\
    \url{https://github.com/digininja.org/dvwa} \\
    Ambiente per testare vulnerabilità

    \item \textbf{bWAPP} \\
    \url{http://www.itsecgames.com/} \\
    Buggy web application per training

    \item \textbf{WebGoat PHP} \\
    Applicazione volutamente vulnerabile per apprendimento
\end{itemize}

\subsection*{Tool Sicurezza}

\begin{itemize}
    \item \textbf{Burp Suite}: \url{https://portswigger.net/burp} (penetration testing)
    \item \textbf{OWASP ZAP}: \url{https://www.zaproxy.org/} (security scanner)
    \item \textbf{Snyk}: \url{https://snyk.io/} (vulnerability scanning)
    \item \textbf{SonarQube}: \url{https://www.sonarqube.org/} (code quality \& security)
\end{itemize}

\section*{Risorse in Italiano}

\begin{itemize}
    \item \textbf{PHP.net Italian Manual} \\
    \url{https://www.php.net/manual/it/}

    \item \textbf{HTML.it - Guida PHP} \\
    \url{https://www.html.it/guide/guida-php-di-base/}

    \item \textbf{MRW.it - Tutorial PHP} \\
    \url{https://www.mrw.it/php/}

    \item \textbf{Forum HTML.it - PHP} \\
    \url{https://forum.html.it/forum/php}

    \item \textbf{Gruppo Facebook "PHP Italia"} \\
    Community italiana attiva
\end{itemize}

\section*{Newsletter e Blog}

\begin{itemize}
    \item \textbf{PHP Weekly}: \url{http://www.phpweekly.com/}
    \item \textbf{Laravel News}: \url{https://laravel-news.com/}
    \item \textbf{Freek.dev}: \url{https://freek.dev/} (Laravel, Spatie)
    \item \textbf{PHP Annotated Monthly} (JetBrains): \url{https://blog.jetbrains.com/phpstorm/}
    \item \textbf{Scotch.io**: \url{https://scotch.io/tag/php}
\end{itemize}

\section*{Certificazioni}

\begin{itemize}
    \item \textbf{Zend Certified PHP Engineer} \\
    \url{https://www.zend.com/training/php-certification-exam} \\
    Certificazione ufficiale PHP

    \item \textbf{Laravel Certification} \\
    \url{https://exam.laravelcert.com/}

    \item \textbf{Symfony Certification} \\
    \url{https://certification.symfony.com/}

    \item \textbf{CompTIA Security+} \\
    Certificazione sicurezza generale applicabile al web
\end{itemize}

\section*{Open Source Projects (per imparare)}

\begin{itemize}
    \item \textbf{Laravel Framework}: \url{https://github.com/laravel/framework}
    \item \textbf{Symfony Components}: \url{https://github.com/symfony/symfony}
    \item \textbf{Guzzle}: \url{https://github.com/guzzle/guzzle}
    \item \textbf{Monolog}: \url{https://github.com/Seldaek/monolog}
    \item \textbf{PHPMailer}: \url{https://github.com/PHPMailer/PHPMailer}
    \item \textbf{Carbon}: \url{https://github.com/briannesbitt/Carbon} (date/time)
\end{itemize}

\section*{Standard e Specifiche}

\begin{itemize}
    \item \textbf{PSR-1}: Basic Coding Standard
    \item \textbf{PSR-2}: Coding Style Guide (deprecato, usa PSR-12)
    \item \textbf{PSR-4}: Autoloading Standard
    \item \textbf{PSR-7}: HTTP Message Interface
    \item \textbf{PSR-12}: Extended Coding Style Guide
    \item \textbf{PSR-15}: HTTP Server Request Handlers
    \item \textbf{PSR-18}: HTTP Client

    Tutti disponibili su: \url{https://www.php-fig.org/psr/}
\end{itemize}

\section*{Podcast}

\begin{itemize}
    \item \textbf{PHP Roundtable}: \url{https://www.phproundtable.com/}
    \item \textbf{Laravel Podcast}: \url{https://laravelpodcast.com/}
    \item \textbf{PHP Ugly}: \url{https://www.phpugly.com/}
    \item \textbf{No Compromises}: Focus su Laravel e best practices
\end{itemize}

\section*{Note Finali}

Le risorse elencate rappresentano una selezione curata per approfondire PHP a tutti i livelli. Si consiglia di:

\begin{itemize}
    \item Iniziare con documentazione ufficiale e PHP: The Right Way
    \item Praticare con progetti personali applicando best practices
    \item Studiare codice open-source di qualità (Laravel, Symfony)
    \item Partecipare attivamente a community online
    \item Mantenere focus su sicurezza (OWASP) in ogni fase
    \item Considerare certificazioni per validazione competenze professionali
\end{itemize}

\vspace{1cm}

\begin{center}
\textit{La sicurezza non è un prodotto, ma un processo!}
\end{center}


\end{document}

