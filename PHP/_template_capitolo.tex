% Template per Nuovo Capitolo (PHP)
\chapter{Titolo Capitolo}

\section{Introduzione}
Descrizione del problema affrontato, contesto applicativo e motivazioni pratiche.

\begin{tcolorbox}[title=Mappa del capitolo]
\textbf{Sezioni}: Introduzione, Obiettivi, Concetti, Sintassi, Esempi, Caso di studio, Diagrammi, Best practice, Errori comuni, Esercizi, Riepilogo, Riferimenti.
\end{tcolorbox}

\section{Obiettivi di Apprendimento}
\begin{itemize}
    \item Comprendere i concetti chiave e il loro uso pratico.
    \item Applicare le tecniche con esempi guidati e casi reali.
    \item Valutare le scelte progettuali con buone pratiche e anti-pattern.
\end{itemize}

\section{Concetti Fondamentali}
Definizioni, proprietà, implicazioni operative e limiti.

\section{Sintassi}
Elementi di sintassi essenziale con esempi minimali.

\section{Esempi Pratici}
Snippet completi con input, output atteso e variazioni.

\section{Caso di Studio}
Un esempio end-to-end, con requisiti, implementazione e discussione.

\section{Diagrammi e Supporti Visivi}
Descrizioni di flussi, strutture e interazioni. (Inserire figure/diagrammi ove utile.)

\section{Best Practice}
Linee guida operative, sicurezza, performance e manutenibilità.

\section{Errori Comuni}
Trappole frequenti, falsi amici, mitigazioni.

\section{Esercizi}
Attività pratiche con livello crescente e suggerimenti.

\section{Verifica}
Domande a risposta breve o vero/falso per autovalutazione.

\section{Riepilogo}
Sintesi dei concetti e collegamenti al capitolo successivo.

\section{Riferimenti}
Fonti autorevoli e link utili (manuali, standard, OWASP, PSR).

