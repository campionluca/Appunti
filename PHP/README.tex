\documentclass[11pt,a4paper]{article}
\usepackage[italian]{babel}
\usepackage[T1]{fontenc}
\usepackage[utf8]{inputenc}
\usepackage{geometry}
\geometry{margin=2.5cm}
\title{Appunti di Programmazione — Corso PHP}
\date{\small Ultimo aggiornamento: \today}
\begin{document}
\maketitle

\section*{Descrizione}
Il progetto raccoglie appunti e materiali del corso di PHP, organizzati in capitoli tecnici (LaTeX), manuale discorsivo (LaTeX) e schede di \emph{quick reference}.

\section*{Struttura}
\begin{itemize}
  \item \texttt{PHP/main.tex}: libro principale con prefazione, capitoli e appendici.
  \item \texttt{PHP/capitoli/}: capitoli tecnici in \LaTeX.
  \item \texttt{PHP/manuale/}: manuale discorsivo (\texttt{main.tex} e capitoli inclusi).
  \item \texttt{PHP/quick\_reference/}: schede sintetiche (\texttt{main.tex}).
\end{itemize}

\section*{Compilazione}
Per compilare il manuale discorsivo:
\begin{verbatim}
latexmk -pdf manuale/main.tex
\end{verbatim}
Per compilare le schede di quick reference:
\begin{verbatim}
latexmk -pdf quick_reference/main.tex
\end{verbatim}

\section*{Stato}
Migrazione da Markdown a \LaTeX completata per i capitoli introduttivi del manuale discorsivo e per la configurazione di progetto.

\end{document}

