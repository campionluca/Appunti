\chapter{Form HTML}

\begin{tcolorbox}[title=Mappa del capitolo]
Introduzione, Creazione di form, Gestione dati, Validazione, Sicurezza XSS, Esempi pratici, Caso di studio, Diagrammi, Esercizi, Verifica, Riepilogo, Riferimenti.
\end{tcolorbox}

\section{Obiettivi di apprendimento}
\begin{itemize}
  \item Progettare form ben strutturati per GET e POST.
  \item Gestire correttamente input lato server con normalizzazione e escaping.
  \item Applicare PRG quando si modifica stato e prevenire XSS.
\end{itemize}

\section{Teoria}
I form HTML sono la principale modalità con cui un client invia dati al server. PHP riceve i dati tramite le superglobali \verb|$_GET|, \verb|$_POST| e \verb|$_FILES|. La scelta del metodo dipende dal caso d'uso: \textbf{GET} per richieste idempotenti e query string, \textbf{POST} per invio di dati sensibili o variazioni di stato.

\section{Creazione di form (GET/POST)}
\begin{lstlisting}
<!-- Esempio GET -->
<form method="get" action="processa.php">
  <label>Query: <input type="text" name="q"></label>
  <button type="submit">Cerca</button>
</form>

<!-- Esempio POST -->
<form method="post" action="processa.php">
  <label>Email: <input type="email" name="email" required></label>
  <label>Password: <input type="password" name="password" required></label>
  <button type="submit">Invia</button>
</form>
\end{lstlisting}

\section{Gestione dei dati inviati}
\begin{lstlisting}
<?php
// processa.php
$metodo = $_SERVER['REQUEST_METHOD'];
$email  = $_POST['email'] ?? '';
$q      = $_GET['q'] ?? '';

if ($metodo === 'POST') {
    if ($email === '') {
        echo 'Email non valida';
        exit;
    }
    echo 'OK';
} else {
    echo 'Ricerca: ' . htmlspecialchars($q, ENT_QUOTES | ENT_SUBSTITUTE, 'UTF-8');
}
\end{lstlisting}

\section{Validazione degli input}
- Validazione lato client (HTML5: \texttt{required}, \texttt{type=email}) e lato server (sempre necessaria).
- Usare regex e normalizzazione dei dati.

\begin{lstlisting}
<?php
// Normalizzazione e validazione
$name = trim((string)($_POST['name'] ?? ''));
if ($name === '' || mb_strlen($name) > 100) {
    echo 'Nome obbligatorio (<=100)';
    exit;
}
\end{lstlisting}

\section{Sicurezza: prevenzione XSS}
- \textbf{XSS}: effettuare sempre escaping in output con \texttt{htmlspecialchars}.

\begin{lstlisting}
<?php
// Esempio di escaping in output
// Usare htmlspecialchars per prevenire XSS quando si rende input utente
echo 'Ricerca: ' . htmlspecialchars($q, ENT_QUOTES | ENT_SUBSTITUTE, 'UTF-8');
\end{lstlisting}

\begin{tcolorbox}[title=Best practice]
- Usare POST per dati sensibili e operazioni che modificano stato.
- Escaping sistematico dell'output (XSS).
- Validare e normalizzare sempre lato server.
- Impostare \texttt{Content-Type} e charset corretti nelle risposte.
\end{tcolorbox}

\begin{tcolorbox}[title=Errori comuni]
- Fidarsi della sola validazione lato client.
- Non effettuare escaping dell'output.
- Mescolare GET/POST senza gestire i casi distinti.
\end{tcolorbox}

\section{Esempi pratici}
\begin{lstlisting}
<?php
// Contatto semplice con normalizzazione e escape
$name  = trim((string)($_POST['name'] ?? ''));
$email = trim((string)($_POST['email'] ?? ''));
$msg   = trim((string)($_POST['message'] ?? ''));

$errors = [];
if ($name === '' || mb_strlen($name) > 100) { $errors[] = 'Nome obbligatorio (<=100)'; }
if (!preg_match('/^[^@\s]+@[^@\s]+\.[^@\s]+$/', $email)) { $errors[] = 'Email non valida'; }
if ($msg === '') { $errors[] = 'Messaggio obbligatorio'; }

if ($errors) {
    // PRG: mostra errori su pagina GET
    header('Location: /contatto_errore.php', true, 303);
    exit;
}

echo 'Grazie, ' . htmlspecialchars($name, ENT_QUOTES | ENT_SUBSTITUTE, 'UTF-8');
?>
\end{lstlisting}

\section{Caso di studio}
Progettazione di un form di login minimale con gestione errori e PRG. Si valida l'input, si normalizza e si evita qualsiasi echo diretto di dati utente senza escape.

\begin{lstlisting}
<?php
$u = trim((string)($_POST['username'] ?? ''));
$p = (string)($_POST['password'] ?? '');
if ($u === '' || $p === '') {
  header('Location: /login.php?err=1', true, 303);
  exit;
}
// Verifica credenziali (simulata)
header('Location: /dashboard.php', true, 303);
exit;
?>
\end{lstlisting}

\section{Diagrammi}
\noindent Flusso PRG: \emph{POST} (valida/aggiorna) \(\rightarrow\) \emph{303 See Other} \(\rightarrow\) \emph{GET} pagina di conferma.

\section{Esercizi}
\begin{itemize}
  \item Implementa un form di contatto con validazione lato server e PRG.
  \item Aggiungi un campo textarea e normalizza gli spazi e le nuove linee.
  \item Progetta un form di ricerca con GET e escaping dell'output.
\end{itemize}

\section{Verifica}
\begin{itemize}
  \item Vero/Falso: dopo \texttt{header('Location',...)} è obbligatorio \texttt{exit}.
  \item Quale metodo usare per mostrare risultati di ricerca? GET o POST e perché?
\end{itemize}

\section{Riepilogo}
Form ben progettati separano responsabilità tra input, elaborazione e output. La validazione lato server e l'escaping sono indispensabili; PRG evita ri-submit del POST.

\section{Riferimenti}
\begin{itemize}
  \item Manuale PHP — superglobali: \url{https://www.php.net/manual/en/reserved.variables.php}
  \item OWASP XSS: \url{https://owasp.org/www-community/attacks/xss/}
  \item HTTP Semantics (Redirect): \url{https://www.rfc-editor.org/rfc/rfc9110}
\end{itemize}
