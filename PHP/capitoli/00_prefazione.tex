% 00_prefazione (PHP)
\chapter*{Prefazione}
\addcontentsline{toc}{chapter}{Prefazione}

\section*{A chi è rivolto questo manuale}

Questo manuale di PHP è pensato per studenti di istituti tecnici superiori che hanno già acquisito competenze di base in HTML, CSS e programmazione (C o Java). Il corso fornisce una solida base per lo sviluppo web backend, con particolare enfasi su sicurezza, best practices e integrazione con database.

PHP (Hypertext Preprocessor) è uno dei linguaggi più utilizzati nel web development, alimentando oltre il 75\% dei siti web dinamici inclusi WordPress, Facebook, Wikipedia e milioni di applicazioni aziendali.

\section*{Prerequisiti}

Prima di affrontare questo corso è necessario avere:

\begin{itemize}
    \item Conoscenza di HTML e CSS
    \item Nozioni di base di programmazione (variabili, cicli, funzioni)
    \item Familiarità con il concetto client-server
    \item Comprensione dei database relazionali (consigliato)
\end{itemize}

\section*{Obiettivi del corso}

Al termine del corso lo studente sarà in grado di:

\begin{enumerate}
    \item \textbf{Sviluppare applicazioni web dinamiche}: Form, autenticazione, gestione sessioni
    \item \textbf{Interagire con database}: Query MySQL/MySQLi, CRUD operations, prepared statements
    \item \textbf{Implementare sicurezza web}: Protezione da OWASP Top 10, validazione input, prevenzione XSS/SQL Injection
    \item \textbf{Gestire file e upload}: Manipolazione file, validazione upload, storage sicuro
    \item \textbf{Utilizzare tecniche avanzate}: Cookie, sessioni, header HTTP, redirect
    \item \textbf{Applicare best practices}: Codice manutenibile, error handling, logging
\end{enumerate}

\section*{Perché PHP?}

\textbf{Vantaggi di PHP}:
\begin{itemize}
    \item \textbf{Facilità di apprendimento}: Sintassi intuitiva per chi conosce C/Java
    \item \textbf{Ampia diffusione}: Hosting economico, vasta community, abbondanza risorse
    \item \textbf{Integrazione database}: Supporto nativo MySQL, PostgreSQL, SQLite
    \item \textbf{Framework potenti}: Laravel, Symfony, CodeIgniter
    \item \textbf{CMS popolari}: WordPress, Drupal, Joomla basati su PHP
    \item \textbf{Performance}: PHP 7/8 ha migliorato drasticamente velocità e efficienza
\end{itemize}

\textbf{Applicazioni reali}:
\begin{itemize}
    \item Siti web dinamici e portali
    \item E-commerce (Magento, WooCommerce)
    \item Sistemi di gestione contenuti (CMS)
    \item API REST per applicazioni mobile
    \item Pannelli di amministrazione
    \item Sistemi di autenticazione e autorizzazione
\end{itemize}

\section*{Struttura del manuale}

Il libro è organizzato in moduli progressivi:

\begin{description}
    \item[Parte I --- Fondamenti] Form HTML, POST/GET, validazione input, array
    \item[Parte II --- Gestione Stato] Cookie, sessioni, autenticazione utenti
    \item[Parte III --- File e Upload] Manipolazione file, upload sicuro, validazione
    \item[Parte IV --- Database] MySQLi, prepared statements, CRUD, sicurezza
    \item[Parte V --- Sicurezza] OWASP Top 10, XSS, CSRF, SQL Injection, best practices
\end{description}

\section*{Strumenti utilizzati}

Per seguire il corso è necessario disporre di:

\subsection*{Ambiente di sviluppo locale}

\textbf{Opzione 1: XAMPP} (consigliato per Windows)
\begin{itemize}
    \item Apache web server
    \item PHP 7.4+ o 8.x
    \item MySQL/MariaDB
    \item phpMyAdmin
    \item Download: \url{https://www.apachefriends.org/}
\end{itemize}

\textbf{Opzione 2: MAMP} (macOS)
\begin{itemize}
    \item Simile a XAMPP, ottimizzato per Mac
    \item Download: \url{https://www.mamp.info/}
\end{itemize}

\textbf{Opzione 3: LAMP Stack} (Linux)
\begin{verbatim}
sudo apt update
sudo apt install apache2 php libapache2-mod-php mysql-server
sudo systemctl start apache2
sudo systemctl start mysql
\end{verbatim}

\subsection*{Editor e IDE}

\begin{itemize}
    \item \textbf{VS Code} con estensioni:
    \begin{itemize}
        \item PHP Intelephense
        \item PHP Debug
        \item MySQL (per query)
    \end{itemize}
    \item \textbf{PHPStorm}: IDE professionale (licenza studenti gratuita)
    \item \textbf{Sublime Text} con Package Control
    \item \textbf{Notepad++} (opzione base)
\end{itemize}

\subsection*{Strumenti di testing}

\begin{itemize}
    \item \textbf{Browser DevTools}: Chrome/Firefox Developer Tools
    \item \textbf{Postman}: Testing API e richieste HTTP
    \item \textbf{XAMPP Control Panel}: Gestione servizi Apache/MySQL
    \item \textbf{phpMyAdmin}: Interfaccia web per MySQL
\end{itemize}

\section*{Convenzioni tipografiche}

\begin{itemize}
    \item \texttt{\<?php echo "codice"; ?\>} --- Codice PHP inline
    \item \textbf{Variabile \$\_POST} --- Termini tecnici importanti
    \item \emph{SQL Injection} --- Concetti di sicurezza
\end{itemize}

\begin{attenzione}
Il codice presentato include commenti su vulnerabilità e best practices di sicurezza. Prestare massima attenzione ai warning su XSS, SQL Injection e CSRF.
\end{attenzione}

\section*{Focus sulla Sicurezza}

Questo corso dedica ampio spazio alla sicurezza web, coprendo:

\begin{itemize}
    \item \textbf{OWASP Top 10 2021}: Le 10 vulnerabilità più critiche
    \item \textbf{Input Validation}: Sanitizzazione e validazione input utente
    \item \textbf{Output Encoding}: Prevenzione XSS con htmlspecialchars()
    \item \textbf{Prepared Statements}: Protezione da SQL Injection
    \item \textbf{CSRF Tokens}: Prevenzione Cross-Site Request Forgery
    \item \textbf{Secure Sessions}: Gestione sicura sessioni utente
    \item \textbf{File Upload Security}: Validazione tipo, dimensione, path traversal
    \item \textbf{Password Hashing}: Uso di password\_hash() e password\_verify()
\end{itemize}

\section*{Metodologia didattica}

Ogni capitolo segue questa struttura:
\begin{enumerate}
    \item \textbf{Introduzione teorica}: Spiegazione del concetto
    \item \textbf{Esempi vulnerabili}: Codice insicuro con spiegazione dei rischi
    \item \textbf{Esempi sicuri}: Versione corretta con best practices
    \item \textbf{Esercizi guidati}: Problemi con soluzione passo-passo
    \item \textbf{Esercizi proposti}: Sfide per consolidare l'apprendimento
    \item \textbf{Checklist sicurezza}: Verifica punti critici
\end{enumerate}

\section*{Versione PHP}

Il corso copre PHP 7.4+ e PHP 8.x, evidenziando:
\begin{itemize}
    \item Nuove feature PHP 8: Named arguments, union types, match expression
    \item Deprecazioni e breaking changes
    \item Best practices per codice moderno e manutenibile
\end{itemize}

\section*{Supporto e risorse}

\begin{itemize}
    \item \textbf{Documentazione ufficiale}: \url{https://www.php.net/docs.php}
    \item \textbf{OWASP}: \url{https://owasp.org/}
    \item \textbf{PHP The Right Way}: \url{https://phptherightway.com/}
    \item \textbf{Stack Overflow}: Tag [php]
    \item \textbf{W3Schools PHP Tutorial}: \url{https://www.w3schools.com/php/}
\end{itemize}

\section*{Progetti pratici}

Il corso include progetti applicativi reali:
\begin{itemize}
    \item Sistema di login/registrazione sicuro
    \item Gestionale CRUD (Create, Read, Update, Delete)
    \item Form di contatto con validazione
    \item Upload e gestione immagini
    \item Mini-blog con commenti
    \item Carrello e-commerce base
\end{itemize}

\section*{Certificazioni e sbocchi}

Competenze PHP aprono opportunità professionali:
\begin{itemize}
    \item \textbf{Web Developer} (full-stack o backend)
    \item \textbf{WordPress Developer}
    \item \textbf{Laravel Developer}
    \item \textbf{Security Analyst} (penetration testing)
    \item \textbf{DevOps Engineer} (deployment, CI/CD)
\end{itemize}

Certificazioni rilevanti:
\begin{itemize}
    \item Zend Certified PHP Engineer
    \item Laravel Certification
    \item OWASP Web Security Certification
\end{itemize}

\section*{Note per i docenti}

Il materiale è strutturato per:
\begin{itemize}
    \item 40-60 ore di lezione (teoria + laboratorio)
    \item Approccio hands-on con esercitazioni pratiche
    \item Enfasi su sicurezza e best practices industria
    \item Preparazione a scenari reali e progetti aziendali
\end{itemize}

\section*{Ringraziamenti}

Si ringraziano:
\begin{itemize}
    \item La community OWASP per le linee guida di sicurezza
    \item Gli studenti delle classi precedenti per feedback e suggerimenti
    \item I contributori open-source di PHP e dei framework
\end{itemize}

\vspace{1cm}

\begin{center}
\textit{Buono studio e... sviluppa in sicurezza!}\\
Gli autori
\end{center}

\begin{nota}
\textbf{Disclaimer}: Gli esempi di codice vulnerabile sono presentati esclusivamente a scopo didattico per comprendere le vulnerabilità. Non utilizzare mai codice insicuro in produzione.
\end{nota}
