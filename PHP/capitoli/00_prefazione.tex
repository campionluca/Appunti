% 00_prefazione (PHP)
\chapter*{Prefazione}
\addcontentsline{toc}{chapter}{Prefazione}

\section*{A chi è rivolto questo manuale}

Questo manuale di PHP è pensato per studenti di istituti tecnici superiori che hanno già acquisito competenze di base in HTML, CSS e programmazione (C o Java). Il corso fornisce una solida base per lo sviluppo web backend, con particolare enfasi su sicurezza, best practices e integrazione con database.

PHP (Hypertext Preprocessor) è uno dei linguaggi più utilizzati nel web development, alimentando oltre il 75\% dei siti web dinamici inclusi WordPress, Facebook, Wikipedia e milioni di applicazioni aziendali.

\section*{Prerequisiti}

Prima di affrontare questo corso è necessario avere conoscenze di HTML e CSS, fondamentali per comprendere come PHP interagisce con il markup e lo stile delle pagine web. È inoltre richiesta una solida base di programmazione che includa la comprensione di concetti essenziali come variabili, cicli e funzioni. Infine, è importante avere familiarità con il concetto client-server per capire l'architettura delle applicazioni web, anche se una comprensione dei database relazionali è consigliata ma non strettamente obbligatoria all'inizio.

\section*{Obiettivi del corso}

Al termine del corso lo studente sarà in grado di:

\begin{enumerate}
    \item \textbf{Sviluppare applicazioni web dinamiche}: Form, autenticazione, gestione sessioni
    \item \textbf{Interagire con database}: Query MySQL/MySQLi, CRUD operations, prepared statements
    \item \textbf{Implementare sicurezza web}: Protezione da OWASP Top 10, validazione input, prevenzione XSS/SQL Injection
    \item \textbf{Gestire file e upload}: Manipolazione file, validazione upload, storage sicuro
    \item \textbf{Utilizzare tecniche avanzate}: Cookie, sessioni, header HTTP, redirect
    \item \textbf{Applicare best practices}: Codice manutenibile, error handling, logging
\end{enumerate}

\section*{Perché PHP?}

\textbf{Vantaggi di PHP}: PHP offre numerosi vantaggi che lo rendono il linguaggio ideale per lo sviluppo web. Innanzitutto, presenta una sintassi intuitiva particolarmente accessibile per chi ha già esperienza con C o Java. La sua ampia diffusione garantisce un hosting economico, una vasta community di sviluppatori e un'abbondanza di risorse disponibili online. PHP fornisce anche un supporto nativo eccellente per l'integrazione con database come MySQL, PostgreSQL e SQLite, permettendo di lavorare facilmente con i dati. Inoltre, è supportato da framework potenti e moderni quali Laravel, Symfony e CodeIgniter che accelerano lo sviluppo. Molti sistemi di gestione contenuti populari come WordPress, Drupal e Joomla sono costruiti su PHP. Infine, le versioni recenti PHP 7 e 8 hanno migliorato notevolmente le performance, la velocità di esecuzione e l'efficienza generale della piattaforma.

\textbf{Applicazioni reali}: Le competenze PHP apre le porte a moltissime opportunità professionali. PHP viene utilizzato per sviluppare siti web dinamici e portali di ogni tipo, piattaforme e-commerce come Magento e WooCommerce, sistemi di gestione contenuti professionali, API REST per applicazioni mobile e per l'integrazione tra servizi, pannelli di amministrazione e sistemi di gestione backend, oltre a complessi sistemi di autenticazione e autorizzazione per proteggere le applicazioni web.

\section*{Struttura del manuale}

Il libro è organizzato in moduli progressivi:

\begin{description}
    \item[Parte I --- Fondamenti] Form HTML, POST/GET, validazione input, array
    \item[Parte II --- Gestione Stato] Cookie, sessioni, autenticazione utenti
    \item[Parte III --- File e Upload] Manipolazione file, upload sicuro, validazione
    \item[Parte IV --- Database] MySQLi, prepared statements, CRUD, sicurezza
    \item[Parte V --- Sicurezza] OWASP Top 10, XSS, CSRF, SQL Injection, best practices
\end{description}

\section*{Strumenti utilizzati}

Per seguire il corso è necessario disporre di:

\subsection*{Ambiente di sviluppo locale}

\textbf{Opzione 1: XAMPP} (consigliato per Windows): XAMPP è una distribuzione completa che include il server Apache web, PHP nella versione 7.4 o superiore fino alle versioni 8.x, il sistema di gestione database MySQL o MariaDB, e phpMyAdmin per la gestione grafica dei database. Questa soluzione all-in-one è disponibile per il download presso \url{https://www.apachefriends.org/}.

\textbf{Opzione 2: MAMP} (macOS): Per gli sviluppatori che utilizzano macOS, MAMP offre una soluzione simile a XAMPP ma ottimizzata specificamente per il sistema operativo Apple, rendendo l'installazione e la configurazione più fluida. È disponibile presso \url{https://www.mamp.info/}.

\textbf{Opzione 3: LAMP Stack} (Linux):
\begin{verbatim}
sudo apt update
sudo apt install apache2 php libapache2-mod-php mysql-server
sudo systemctl start apache2
sudo systemctl start mysql
\end{verbatim}

\subsection*{Editor e IDE}

Per la stesura del codice, sono disponibili diverse opzioni a seconda delle preferenze e delle esigenze. \textbf{VS Code} è un editor leggero e moderno che con le opportune estensioni (PHP Intelephense per il supporto PHP, PHP Debug per il debugging, e MySQL per gestire query) offre un'esperienza di sviluppo completa e piacevole. \textbf{PHPStorm} è un IDE professionale completo sviluppato specificamente per PHP, che offre una licenza gratuita per studenti e fornisce funzionalità avanzate come il refactoring intelligente e il debugging integrato. \textbf{Sublime Text} è un'altra opzione popolare che, quando configurato con Package Control, permette di installare estensioni per il supporto PHP. Infine, \textbf{Notepad++} rappresenta un'opzione più basilare ma comunque funzionale per chi desidera iniziare con uno strumento minimalista.

\subsection*{Strumenti di testing}

Per il testing e il debugging, diversi strumenti facilitano il lavoro dello sviluppatore. \textbf{Browser DevTools}, disponibili in Chrome e Firefox Developer Tools, consentono di ispezionare il codice HTML, CSS e JavaScript, nonché monitorare le richieste HTTP e le risposte del server. \textbf{Postman} è uno strumento specializzato per testare API e costruire richieste HTTP complesse, permettendo di simulare client avanzati senza scrivere codice. \textbf{XAMPP Control Panel} fornisce un'interfaccia grafica per avviare, fermare e gestire i servizi Apache e MySQL, semplificando notevolmente la gestione dell'ambiente di sviluppo locale. Infine, \textbf{phpMyAdmin} è un'interfaccia web intuitiva per interagire con i database MySQL, permettendo di eseguire query, creare tabelle e gestire i dati direttamente dal browser.

\section*{Convenzioni tipografiche}

\begin{itemize}
    \item \texttt{\<?php echo "codice"; ?\>} --- Codice PHP inline
    \item \textbf{Variabile \$\_POST} --- Termini tecnici importanti
    \item \emph{SQL Injection} --- Concetti di sicurezza
\end{itemize}

\begin{attenzione}
Il codice presentato include commenti su vulnerabilità e best practices di sicurezza. Prestare massima attenzione ai warning su XSS, SQL Injection e CSRF.
\end{attenzione}

\section*{Focus sulla Sicurezza}

Questo corso dedica ampio spazio alla sicurezza web, un aspetto fondamentale nello sviluppo di applicazioni robuste e affidabili. Il curriculum affronta le vulnerabilità incluse nell'OWASP Top 10 2021, le dieci minacce di sicurezza più critiche per le applicazioni web moderne. Un focus particolare è dedicato alla validazione e sanitizzazione degli input forniti dagli utenti, componente essenziale per prevenire molti attacchi. Viene affrontato anche il concetto di output encoding, utilizzando funzioni come htmlspecialchars() per prevenire attacchi XSS (Cross-Site Scripting). Il corso insegna l'utilizzo di prepared statements per proteggere il database da SQL injection, una delle vulnerabilità più pericolose. Sono inoltre coperti i token CSRF (Cross-Site Request Forgery) per proteggere le applicazioni dalla falsificazione di richieste cross-site, la gestione sicura delle sessioni utente con appropriate impostazioni di sicurezza, le best practices per il file upload includendo validazione del tipo, della dimensione e la prevenzione di path traversal attacks, e infine le tecniche corrette di hashing delle password utilizzando funzioni come password_hash() e password_verify().

\section*{Metodologia didattica}

Ogni capitolo segue questa struttura:
\begin{enumerate}
    \item \textbf{Introduzione teorica}: Spiegazione del concetto
    \item \textbf{Esempi vulnerabili}: Codice insicuro con spiegazione dei rischi
    \item \textbf{Esempi sicuri}: Versione corretta con best practices
    \item \textbf{Esercizi guidati}: Problemi con soluzione passo-passo
    \item \textbf{Esercizi proposti}: Sfide per consolidare l'apprendimento
    \item \textbf{Checklist sicurezza}: Verifica punti critici
\end{enumerate}

\section*{Versione PHP}

Il corso copre PHP 7.4+ e PHP 8.x, evidenziando:
\begin{itemize}
    \item Nuove feature PHP 8: Named arguments, union types, match expression
    \item Deprecazioni e breaking changes
    \item Best practices per codice moderno e manutenibile
\end{itemize}

\section*{Supporto e risorse}

\begin{itemize}
    \item \textbf{Documentazione ufficiale}: \url{https://www.php.net/docs.php}
    \item \textbf{OWASP}: \url{https://owasp.org/}
    \item \textbf{PHP The Right Way}: \url{https://phptherightway.com/}
    \item \textbf{Stack Overflow}: Tag [php]
    \item \textbf{W3Schools PHP Tutorial}: \url{https://www.w3schools.com/php/}
\end{itemize}

\section*{Progetti pratici}

Il corso include progetti applicativi reali:
\begin{itemize}
    \item Sistema di login/registrazione sicuro
    \item Gestionale CRUD (Create, Read, Update, Delete)
    \item Form di contatto con validazione
    \item Upload e gestione immagini
    \item Mini-blog con commenti
    \item Carrello e-commerce base
\end{itemize}

\section*{Certificazioni e sbocchi}

Competenze PHP aprono numerose opportunità professionali nel mercato del lavoro. Gli studenti potranno intraprendere carriere come Web Developer specializzato nello sviluppo full-stack o backend, WordPress Developer per la creazione e personalizzazione di siti basati su WordPress, Laravel Developer per lo sviluppo di applicazioni enterprise-grade, Security Analyst con focus su penetration testing e sicurezza web, oppure DevOps Engineer specializzato nel deployment e nella configurazione di pipeline CI/CD.

Le certificazioni riconosciute dal settore che possono seguire il corso includono la certificazione Zend Certified PHP Engineer che attesta le competenze professionali in PHP, la Laravel Certification per chi desideri specializzarsi nel framework Laravel, e la OWASP Web Security Certification che riconosce le competenze di sicurezza web acquisite durante il corso.

\section*{Note per i docenti}

Il materiale è strutturato per un corso che comprende 40-60 ore totali di insegnamento, combinando teoria e laboratorio pratico in modo equilibrato. L'approccio è decisamente hands-on, con numerose esercitazioni pratiche che permettono agli studenti di applicare immediatamente i concetti appresi. Viene posto un'enfasi particolare sulla sicurezza web e sulle best practices riconosciute dall'industria, preparando gli studenti ad affrontare scenari reali e progetti aziendali autentici.

\section*{Ringraziamenti}

Si ringrazia la community OWASP per le linee guida di sicurezza fondamentali che hanno ispirato e informato la creazione di questo materiale didattico. Un ringraziamento va anche agli studenti delle classi precedenti i cui feedback e suggerimenti hanno contribuito al miglioramento continuo del corso. Infine, si apprezza il lavoro dei contributori open-source della community PHP e dei principali framework, le cui innovazioni e dedizione rendono possibile la creazione di materiale educativo aggiornato e rilevante.

\vspace{1cm}

\begin{center}
\textit{Buono studio e... sviluppa in sicurezza!}\\
Gli autori
\end{center}

\begin{nota}
\textbf{Disclaimer}: Gli esempi di codice vulnerabile sono presentati esclusivamente a scopo didattico per comprendere le vulnerabilità. Non utilizzare mai codice insicuro in produzione.
\end{nota}
