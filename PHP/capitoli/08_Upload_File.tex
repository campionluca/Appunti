\chapter{Upload di File}

\section{Validazione e salvataggio}
Esempio completo con controlli su MIME e dimensione.

\lstinputlisting[language=PHP, caption={Upload PHP}, label={lst:uploadphp}]{esempi/upload.php}

\subsection{Analisi tecnica dell'upload sicuro}
\begin{itemize}
\item \textbf{Funzione e scopo}: Implementazione completa di sistema di upload file con protezioni contro file injection, MIME spoofing e limitazioni di dimensione, seguendo le best practices OWASP.

\item \textbf{Componenti principali}:
\begin{itemize}
\item \texttt{declare(strict\_types=1)}: Strict typing per type safety
% Rimosso: riferimenti a CSRF e token
\item \texttt{\$_FILES} superglobal: Accesso ai metadati file upload
\item \texttt{UPLOAD\_ERR\_OK}: Controllo stato upload successivo
\item \texttt{finfo\_open(FILEINFO\_MIME\_TYPE)}: Rilevamento MIME type reale
\item \texttt{move\_uploaded\_file()}: Funzione sicura per spostamento file
\item \texttt{bin2hex(random\_bytes(8))}: Generazione nome file sicuro
\item HTTP status codes appropriati per errori
\end{itemize}

\item \textbf{Esempi pratici}: Questo pattern si applica in:
\begin{itemize}
\item Upload documenti in piattaforme enterprise
\item Sistemi di gestione contenuti (CMS)
\item Portali per submission academiche/lavorative
\item Applicazioni di file sharing sicuro
\item Backup e restore di configurazioni
\end{itemize}

\item \textbf{Varianti e alternative}:
\begin{itemize}
\item \texttt{is\_uploaded\_file()} per validazione aggiuntiva
\item Cloud storage (AWS S3, Google Cloud Storage) per scalabilità
\item Chunked upload per file di grandi dimensioni
\item Client-side validation con JavaScript prima dell'invio
\item API RESTful per integrazione con applicazioni moderne
\item Framework-specific solutions (Laravel, Symfony file upload)
\end{itemize}

\item \textbf{Best practices e insidie}:
\begin{itemize}
\item \textbf{Best practice}: Sempre validare MIME type con \verb|finfo_*|
\item \textbf{Insidia}: Trustare solo \texttt{\$_FILES['error']} per stato upload
\item \textbf{Sicurezza}: Mai trustare \texttt{\$_FILES['type']} (client-provided)
\item \textbf{Performance}: Limitare dimensione file sia client che server side
\item \textbf{Storage}: Usare directory fuori document root per file uploadati
\item \textbf{Backup}: Implementare quarantena per file sospetti
\end{itemize}

\item \textbf{Riferimenti teorici}:
\begin{itemize}
\item \textbf{OWASP File Upload}: Security guidelines per upload sicuri
% Rimosso: sezione CSRF Protection
\item \textbf{MIME Type Detection}: Magic number analysis vs extension trust
\item \textbf{File Inclusion Vulnerabilities}: Prevenzione directory traversal attacks
\item \textbf{Randomization Security}: Cryptographically secure random generation
\item \textbf{HTTP Response Codes}: Semantic status codes per error handling
\end{itemize}
\end{itemize}
