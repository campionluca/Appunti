\chapter{Array}

\begin{tcolorbox}[title=Mappa del capitolo]
Teoria, Creazione e manipolazione, Funzioni principali, Iterazione, Array multidimensionali, Esempi pratici, Caso di studio, Esercizi, Verifica, Riepilogo, Riferimenti.
\end{tcolorbox}

\section{Teoria}
Gli array in PHP sono strutture flessibili che possono contenere valori di qualunque tipo. Esistono funzioni native per creare, manipolare e iterare.

\section{Creazione e manipolazione}
\begin{lstlisting}
<?php
$nums = [1, 2, 3];
$mix  = ['a', 10, true];
array_push($nums, 4); // [1,2,3,4]
$nums = array_merge([0], $nums); // [0,1,2,3,4]
unset($mix[1]); // rimuove elemento
\end{lstlisting}

\section{Funzioni principali}
- \texttt{count}, \texttt{array\_push}, \texttt{array\_pop}, \texttt{array\_shift}
- \texttt{in\_array}, \texttt{array\_search}, \texttt{array\_key\_exists}

\begin{lstlisting}
<?php
// Esempi senza funzioni di ordine superiore
// Quadrati
$squared = [];
foreach ($nums as $x) { $squared[] = $x * $x; }
// Pari
$even = [];
foreach ($nums as $x) { if ($x % 2 === 0) { $even[] = $x; } }
// Somma
$sum = 0;
foreach ($nums as $x) { $sum += $x; }
\end{lstlisting}

\section{Iterazione}
\begin{lstlisting}
<?php
for ($i=0; $i<count($nums); $i++) { echo $nums[$i]; }
foreach ($nums as $n) { echo $n; }
\end{lstlisting}

\section{Array multidimensionali}
\begin{lstlisting}
<?php
$matrix = [ [1,2], [3,4] ];
echo $matrix[1][0]; // 3
\end{lstlisting}

\begin{tcolorbox}[title=Best practice]
- Usare \texttt{foreach} per semplicità e leggibilità.
- Evitare rimuovere elementi senza reindicizzare se serve l'indice.
\end{tcolorbox}

\begin{tcolorbox}[title=Errori comuni]
- Presumere che gli indici si ricalcolino automaticamente dopo \texttt{unset}.
- Dimenticare che PHP consente tipi misti: gestire coerenza dei dati.
\end{tcolorbox}

\section{Esempi pratici}
\begin{lstlisting}
<?php
// Reindicizzazione dopo rimozione
$a = [10,20,30];
unset($a[1]);           // [0=>10, 2=>30]
$a = array_values($a);  // [0=>10, 1=>30]

// Aggregazioni
$prezzi = [10.5, 7.2, 13.0];
$totale = 0.0;
foreach ($prezzi as $p) { $totale += $p; }
?>
\end{lstlisting}

\section{Caso di studio}
Gestione di un carrello: aggiunta, rimozione, calcolo totale e sconti senza funzioni di ordine superiore.

\section{Esercizi}
\begin{itemize}
  \item Dato un array di interi, costruisci un nuovo array di quadrati usando solo cicli.
  \item Reindicizza correttamente dopo rimozioni multiple e verifica gli indici.
  \item Implementa un carrello con totale e sconto percentuale per importi \textgreater{} 100.
\end{itemize}

\section{Verifica}
\begin{itemize}
  \item Cosa restituisce \texttt{count} su un array con buchi di indici?
  \item Quando è necessario usare \texttt{array\_values}?
\end{itemize}

\section{Riepilogo}
Gli array in PHP sono dinamici e flessibili: usa cicli per trasformazioni e filtri, e reindicizza quando necessario.

\section{Riferimenti}
\begin{itemize}
  \item Manuale PHP — Arrays: \url{https://www.php.net/array}
\end{itemize}
