% Modulo 15 — Database con SQLite e SQLAlchemy
\chapter{Database con SQLite e SQLAlchemy}

\section{Introduzione}
Persistenza dati con SQLite e ORM SQLAlchemy.

Questa descrizione spiega come modellare entità e relazioni, eseguire operazioni CRUD in modo sicuro e sfruttare transazioni per consistenza dei dati. Accenniamo a migrazioni e gestione dello schema, separando responsabilità tra layer di accesso e logica applicativa.

Integrare un database consente applicazioni stateful, archiviazione affidabile e analisi su dati strutturati, dal prototipo alla produzione.

\section{Obiettivi di Apprendimento}
\begin{itemize}
  \item Usare \texttt{sqlite3} per query semplici.
  \item Definire modelli con SQLAlchemy.
  \item Eseguire CRUD di base.
\end{itemize}

\section{Concetti Fondamentali}
\begin{tcolorbox}[title={Schema, transazioni e ORM}]
- Progetta schema con chiavi, vincoli e indici per integrità.
- Usa transazioni per coerenza (commit/rollback).
- Ciclo di vita della sessione e lazy loading (N+1 da prevenire).
- Migrazioni per evolvere lo schema in sicurezza.
\end{tcolorbox}

\section{Esempi Pratici}
\subsection{sqlite3}
\begin{lstlisting}
import sqlite3
conn = sqlite3.connect(":memory:")
c = conn.cursor()
c.execute("CREATE TABLE users(name TEXT)")
c.execute("INSERT INTO users VALUES ('Ada')")
print(list(c.execute("SELECT * FROM users")))
\end{lstlisting}

\begin{tcolorbox}[title=Spiegazione]
- \texttt{sqlite3} offre un DB leggero; usare connessione e cursore per eseguire query.
- Gestire transazioni con commit/rollback; validare input per evitare SQL injection.
- Separare accesso dati dalla logica; testare query critiche.
\end{tcolorbox}

\subsection{SQLAlchemy}
\begin{lstlisting}
from sqlalchemy import Column, Integer, String, create_engine
from sqlalchemy.orm import declarative_base, sessionmaker

Base = declarative_base()

class User(Base):
    __tablename__ = "users"
    id = Column(Integer, primary_key=True)
    name = Column(String)

engine = create_engine("sqlite:///:memory:")
Base.metadata.create_all(engine)
Session = sessionmaker(bind=engine)
sess = Session()
sess.add(User(name="Ada")); sess.commit()
print(sess.query(User).all())
\end{lstlisting}

\begin{tcolorbox}[title=Spiegazione]
- L'ORM mappa classi a tabelle; \texttt{declarative\_base} definisce il modello.
- La sessione gestisce transazioni e lifecycle; attenzione a lazy loading/N+1.
- Migrazioni (Alembic) per evolvere schema; definire vincoli e indici per integrità.
\end{tcolorbox}

\begin{tcolorbox}[title=Casi d'uso]
- CRUD per applicazioni di gestione utenti/dati.
- Report e analisi su tabelle con vincoli chiari.
- Servizi stateful con persistenza affidabile.
- Prototipi con SQLite e migrazione successiva.
\end{tcolorbox}

\section{Esercizi}
\begin{enumerate}
  \item Crea una tabella e inserisci dati da input.
  \item Implementa CRUD completo con SQLAlchemy.
  \item Aggiungi vincoli e indici.
  \item Usa transazioni e gestisci rollback.
  \item Migra schema con Alembic (concettuale).
\end{enumerate}

\section{Riepilogo}
Hai gestito dati con SQLite e modellato entità con SQLAlchemy.

\section{Contesto e Applicazioni}
\begin{tcolorbox}[title=Contesto e Applicazioni]
- Applicazioni CRUD e gestione anagrafiche.
- Reportistica e analisi dati strutturati.
- Servizi stateful con persistenza locale.
- Prototipi con SQLite e migrazione a DB esterni.
\end{tcolorbox}

\section{Approfondimenti}
\begin{tcolorbox}[title=Spiegazioni dettagliate]
\begin{itemize}
  \item \textbf{Modellazione}: definire entità, relazioni e vincoli; normalizzare dove appropriato.
  \item \textbf{Sessione e transazioni}: gestire lifecycle; commit/rollback e gestione degli errori.
  \item \textbf{Lazy vs eager}: scegliere strategie di caricamento; evitare \emph{N+1} query.
  \item \textbf{Migrazioni}: usare strumenti dedicati (Alembic) per evoluzioni dello schema.
  \item \textbf{Integrità}: vincoli, indici, e tipi corretti; prevenire inconsistenze e injection.
\end{itemize}
\end{tcolorbox}

\paragraph{Pratiche} Isolare accesso ai dati, testare query critiche, e tracciare performance; documentare decisioni di schema e mapping.
Consulta: \url{https://docs.sqlalchemy.org/}, \url{https://docs.python.org/3/library/sqlite3.html}.
