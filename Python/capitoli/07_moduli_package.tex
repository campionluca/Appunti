% Modulo 07 — Moduli e Package
\chapter{Moduli e Package}

\section{Introduzione}
Organizzare codice in moduli e package, import e percorso di ricerca.

Questa descrizione spiega come strutturare un progetto in unità coerenti, evitare conflitti di nomi e adottare convenzioni di import leggibili. Tocchiamo \texttt{\_\_init\_\_.py}, percorsi relativi/assoluti e il ruolo di \texttt{sys.path}.

Un buon design dei package rende il codice estensibile e manutenibile, facilitando test, riuso e distribuzione.

\section{Obiettivi di Apprendimento}
\begin{itemize}
  \item Creare e importare moduli.
  \item Strutturare package con \texttt{\_\_init\_\_.py}.
  \item Comprendere \texttt{sys.path} e import relativi.
\end{itemize}

\section{Concetti Fondamentali}
\begin{tcolorbox}[title=Import]
\texttt{from pkg.mod import func} e \texttt{import pkg.mod as m}
\end{tcolorbox}

\begin{tcolorbox}[title=Strutturazione e convenzioni]
- Usa import assoluti per chiarezza (evita ambiguità dei relativi profondi).
- Mantieni \texttt{\_\_init\_\_} minimale: re-export selettivo e costanti.
- Progetta API di package: cosa esporre dal livello radice e cosa no.
- Evita side-effect all'import: inizializza risorse su richiesta.
- Documenta dipendenze interne tra moduli per ridurre accoppiamento.
\end{tcolorbox}

\section{Esempi Pratici}
\subsection{Modulo semplice}
\begin{lstlisting}
# file: util.py
def say(msg):
    print(msg)

# file: app.py
import util
util.say("Ciao")
\end{lstlisting}

\begin{tcolorbox}[title=Spiegazione]
- Il modulo \texttt{util.py} definisce una funzione e viene importato in \texttt{app.py}.
- \texttt{import util} carica il modulo e ne usa il namespace: \texttt{util.say(...)}.
- Preferire import espliciti (funzioni o nomi) quando la UX lo richiede: \texttt{from util import say}.
- Evitare side-effect all'import: l'esecuzione dovrebbe avvenire in \texttt{main()} o funzioni dedicate.
\end{tcolorbox}

\subsection{Package}
\begin{lstlisting}
# pkg/__init__.py
# pkg/tools.py
def add(a, b):
    return a + b

# main.py
from pkg.tools import add
print(add(2, 3))
\end{lstlisting}

\begin{tcolorbox}[title=Spiegazione]
- Un package è una cartella (\texttt{pkg/}) con moduli; \texttt{\_\_init\_\_.py} può essere vuoto o fare re-export.
- \texttt{from pkg.tools import add} importa simboli dal modulo \texttt{tools} all'interno del package.
- Mantenere \texttt{\_\_init\_\_.py} minimale e documentare l'API pubblica con \verb|__all__| se necessario.
- Preferire import assoluti per chiarezza; usare relativi solo all'interno del package.
\end{tcolorbox}

\subsection{sys.path}
\begin{lstlisting}
import sys
print(sys.path)
\end{lstlisting}

\begin{tcolorbox}[title=Spiegazione]
- \texttt{sys.path} è la lista di percorsi in cui Python cerca moduli e package.
- Modificare \texttt{sys.path} a runtime è possibile ma va limitato a script/CLI, non a librerie.
- Per la distribuzione, usare package installabili invece di aggiungere path manualmente.
\end{tcolorbox}

\begin{tcolorbox}[title=Casi d'uso]
- Libreria interna di utilità condivisa tra progetti scollegati.
- Organizzare un'app in sottopackage (\texttt{core}, \texttt{api}, \texttt{cli}).
- Separare plugin/estensioni caricati dinamicamente.
- Re-export dell'API pubblica dal package root per UX migliore.
\end{tcolorbox}

\section{Esercizi}
\begin{enumerate}
  \item Crea un package con due moduli e import incrociati.
  \item Implementa un modulo \texttt{mathutil} con funzioni di base.
  \item Usa alias negli import e confronta leggibilità.
  \item Esplora \texttt{sys.path} e aggiungi un percorso temporaneo.
  \item Organizza un progetto in package e sottopackage.
\end{enumerate}

\section{Riepilogo}
Hai organizzato codice con moduli e package in modo pulito.

\section{Contesto e Applicazioni}
\begin{tcolorbox}[title=Contesto e Applicazioni]
- Organizzare progetti medio-grandi in package coerenti.
- Definire API pubblica del package con re-export.
- Isolare plugin/estensioni con moduli dedicati.
- Preparare distribuzione e riuso interno.
\end{tcolorbox}

\section{Approfondimenti}
\begin{tcolorbox}[title=Spiegazioni dettagliate]
\begin{itemize}
\item \textbf{Import assoluti vs relativi}: preferire import assoluti per chiarezza; relativi utili all'interno di pacchetti.
\item \verb|__init__.py|: controlla l'esposizione del pacchetto; usare \verb|__all__| per definire API pubblica.
  \item \textbf{Struttura dei moduli}: separare responsabilità e evitare cicli di dipendenze; predisporre sottopacchetti coerenti.
  \item \textbf{Ambienti e dipendenze}: congelare versioni con file di requisiti; documentare vincoli e compatibilità.
\end{itemize}
\end{tcolorbox}

\paragraph{Distribuzione} Valutare packaging con \verb|pyproject.toml| e strumenti moderni; integrare test e lint nel processo di rilascio.
Consulta: \url{https://docs.python.org/3/tutorial/modules.html}.
