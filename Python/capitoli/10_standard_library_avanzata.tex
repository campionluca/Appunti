% Modulo 10 — Standard Library Avanzata
\chapter{Standard Library Avanzata}

\section{Introduzione}
Panoramica di moduli utili: \texttt{pathlib}, \texttt{collections}, \texttt{itertools}, \texttt{functools}, \texttt{datetime}.

La descrizione evidenzia come sfruttare la libreria standard per risolvere problemi comuni senza dipendenze esterne: manipolare file e percorsi, contare e raggruppare dati, combinare sequenze e comporre funzioni. Un uso mirato accelera sviluppo e riduce bug.

Conoscere bene questi moduli evita di "reinventare la ruota" e abilita soluzioni più eleganti e concise.

\section{Obiettivi di Apprendimento}
\begin{itemize}
  \item Usare moduli della libreria standard per compiti comuni.
  \item Combinare strumenti per soluzioni eleganti.
\end{itemize}

\section{Concetti Fondamentali}
\begin{tcolorbox}[title=Selezionare gli strumenti giusti]
- \texttt{pathlib}: percorsi e file system in modo portabile.
- \texttt{collections}: contatori, code, mapping specializzati.
- \texttt{itertools}: combinazioni, accumuli, slicing di iterabili.
- \texttt{functools}: memoization, partial application, composizione.
- \texttt{datetime}: fusi orari, intervalli, formattazione.
\end{tcolorbox}

\section{Esempi Pratici}
\subsection{pathlib}
\begin{lstlisting}
from pathlib import Path
p = Path(".")
print([f.name for f in p.glob("*.tex")])
\end{lstlisting}

\begin{tcolorbox}[title=Spiegazione]
- \texttt{Path} modella percorsi in modo portabile; \texttt{glob} elenca i file per pattern.
- Preferire \texttt{Path} a \texttt{os.path}: API orientata a oggetti e più leggibile.
- Usare metodi come \texttt{read\_text()} e \texttt{write\_text()} per I/O semplice.
\end{tcolorbox}

\subsection{collections}
\begin{lstlisting}
from collections import Counter
print(Counter("banana"))
\end{lstlisting}

\begin{tcolorbox}[title=Spiegazione]
- \texttt{Counter} conta occorrenze; utile per analisi testi e frequenze.
- Supporta operazioni tra contatori (somma, sottrazione) e metodi come \texttt{most\_common()}.
- Per default dict con valori interi; considerare \texttt{defaultdict} per default custom.
\end{tcolorbox}

\subsection{itertools/functools}
\begin{lstlisting}
import itertools, functools
data = [1,2,3]
print(list(itertools.accumulate(data)))
print(functools.reduce(lambda a,b: a+b, data, 0))
\end{lstlisting}

\begin{tcolorbox}[title=Spiegazione]
- \texttt{accumulate} genera prefissi cumulativi; \texttt{reduce} combina una sequenza con una funzione.
- Usare funzioni pure per \texttt{reduce}; considerare \texttt{operator} per evitare \texttt{lambda}.
- Comporre trasformazioni con iteratori per evitare liste intermedie.
\end{tcolorbox}

\begin{tcolorbox}[title=Casi d'uso]
- Analisi testi con \texttt{Counter} e normalizzazione.
- Pipeline di trasformazione dati con \texttt{itertools}.
- Strumenti CLI e gestione file con \texttt{pathlib}.
- Cache di funzioni pure con \texttt{functools.lru\_cache}.
\end{tcolorbox}

\section{Esercizi}
\begin{enumerate}
  \item Usa \texttt{Counter} per conteggio parole in un testo.
  \item Filtra files per estensione con \texttt{pathlib}.
  \item Genera combinazioni e permutazioni con \texttt{itertools}.
  \item Applica \texttt{reduce} per calcolare un prodotto.
  \item Usa \texttt{datetime} per differenze tra date.
\end{enumerate}

\section{Riepilogo}
Hai visto moduli chiave della standard library.

\section{Contesto e Applicazioni}
\begin{tcolorbox}[title=Contesto e Applicazioni]
- Tooling per filesystem e report con \texttt{pathlib}.
- Analisi e raggruppamenti con \texttt{collections}.
- Pipeline e combinazioni con \texttt{itertools}.
- Cache e composizione funzioni con \texttt{functools}.
\end{tcolorbox}

\section{Approfondimenti}
\begin{tcolorbox}[title=Spiegazioni dettagliate]
\begin{itemize}
  \item \textbf{Collections}: \verb|deque|, \verb|Counter|, \verb|defaultdict| per pattern frequenti.
  \item \textbf{Itertools}: combinazioni, prodotti, concatenazioni; evitare ricombinazioni manuali.
  \item \textbf{Functools}: \verb|partial|, \verb|reduce|, \verb|lru_cache| per composizione e performance.
  \item \textbf{Pathlib e datetime}: gestire percorsi in modo portabile; trattare timezone e formattazioni con attenzione.
  \item \textbf{Logging e argparse}: standardizzare logging; costruire CLI coerenti e autodescrittive.
\end{itemize}
\end{tcolorbox}

\paragraph{Approccio} Preferire libreria standard quando sufficiente, riducendo dipendenze; valutare compatibilità e performance prima di introdurre terze parti. Integrare \verb|typing| e strumenti statici per contratti chiari.
Consulta: \url{https://docs.python.org/3/library/index.html}.
