% Box riutilizzabile: Guida per programmatori non-Python
\begin{tcolorbox}[title=Guida per programmatori non-Python]
Questa guida riassume come leggere e capire codice Python se provieni da altri linguaggi.

- Sintassi a indentazione: blocchi definiti da indentazione (4 spazi), niente `{}`.
- Tipi dinamici: il tipo appartiene all'oggetto; le variabili sono riferimenti.
- Truthiness: oggetti vuoti e zero sono `False`; altrimenti `True`.
- Strutture chiave: `list` (array dinamico), `dict` (mappa), `set` (insieme), `tuple` (immutabile).
- Slice e comprensioni: `seq[a:b:c]` e `[f(x) for x in xs if cond]`.
- Funzioni: prima classe, closure e default arguments; attenzione ai default mutabili.
- Errori: eccezioni con `try/except`; EAFP (fallo e gestisci) spesso preferito a LBYL.
- Moduli e package: import espliciti, nomi in \verb|snake_case|, \verb|__name__ == "__main__"| per entrypoint.
- Stile: PEP 8 per nomi, lunghezza righe, spazi; scrivere docstring.

Esempio di lettura codice:
\begin{lstlisting}
def process(items):
    """Filtra e trasforma i dati; nessun I/O qui."""
    # Comprensione: compatta, dichiarativa
    return [x * 2 for x in items if x > 0]

def main():
    data = [1, -1, 3]
    result = process(data)
    print(result)  # I/O separato dalla logica

if __name__ == "__main__":
    main()
\end{lstlisting}

Suggerimenti pratici:
- Separare logica pura (facile da testare) e I/O.
- Usare ambienti virtuali per dipendenze; evitare globali.
- Preferire nomi descrittivi e funzioni piccole; aggiungere docstring.
\end{tcolorbox}
