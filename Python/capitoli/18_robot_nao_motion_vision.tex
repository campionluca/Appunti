% Modulo 18 - Movimento e Visione con NAO (V6)
\chapter{Movimento e Visione}

\section{Contesto e Applicazioni}
\begin{tcolorbox}[title=Contesto e Applicazioni]
- Gesture e posture per interazioni HRI.
- Acquisizione immagini per analisi e riconoscimento.
- Demo di navigazione e percezione di base.
\end{tcolorbox}

\section{Approfondimenti}
Il servizio \texttt{ALMotion} controlla giunti e camminata, mentre \texttt{ALRobotPosture} gestisce posture predefinite. \texttt{ALVideoDevice} fornisce flussi dalle camere.

\begin{tcolorbox}[title=Esempi di movimento]
\begin{lstlisting}
import qi, time

session = qi.Session(); session.connect("tcp://nao.local:9559")
motion = session.service("ALMotion")
posture = session.service("ALRobotPosture")

posture.goToPosture("StandInit", 0.5)
# Muove il braccio destro (pitch/roll)
names = ["RShoulderPitch", "RShoulderRoll"]
angles = [0.5, -0.2]   # radianti
speed = 0.2
motion.setAngles(names, angles, speed)
time.sleep(1)
posture.goToPosture("Stand", 0.5)
\end{lstlisting}
\end{tcolorbox}

\begin{tcolorbox}[title=Acquisizione immagini]
\begin{lstlisting}
import qi
session = qi.Session(); session.connect("tcp://nao.local:9559")
vid = session.service("ALVideoDevice")

# Iscrizione con API esplicita per camera e spazio colore
# cameraIndex: 0 = frontale, 1 = bottom, 2 = 3D (se disponibile)
# resolution: 8=QVGA(320x240), 11=VGA(640x480), ecc.
# colorSpace: 13=BGR, 11=YUV, ecc. (vedi documentazione NAOqi)
name = vid.subscribeCamera("python_client", 0, 11, 13, 10)
frame = vid.getImageRemote(name)
vid.unsubscribe(name)

# frame è una lista: [width, height, layers, colorSpace, timestamp, data]
width, height = frame[0], frame[1]
data = frame[6]  # buffer raw (bytes)

# Scrittura grezza su file (per ispezione)
with open("frame.raw", "wb") as f:
    f.write(bytearray(data))

# Per salvare come PNG/JPEG, usare PIL (Pillow) e convertire BGR->RGB
# from PIL import Image; Image.frombytes(...)
\end{lstlisting}
\end{tcolorbox}

\paragraph{Sicurezza} Impostare posture stabili prima di muovere; evitare collisioni e mantenere velocità moderate. Analizzare eccezioni e latenza di rete.

\begin{tcolorbox}[title=Spiegazioni per non-Python]
\textbf{Liste vs tuple}: il risultato di \texttt{getImageRemote} è una lista; si accede per indice. In Python le tuple sono immutabili, le liste no.  
\textbf{Bytes}: la variabile \texttt{data} è una sequenza di byte; si può convertire con \texttt{bytearray} o \texttt{bytes} per scrittura su file.  
\textbf{Context manager}: l'istruzione \texttt{with open(...)} gestisce automaticamente apertura/chiusura file (equivalente a \texttt{try/finally}).
\end{tcolorbox}
