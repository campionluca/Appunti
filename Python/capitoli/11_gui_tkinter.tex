% Modulo 11 — GUI con Tkinter
\chapter{GUI con Tkinter}

\section{Introduzione}
Creare interfacce grafiche desktop con Tkinter: finestre, widget e eventi.

Questa descrizione spiega il ciclo di eventi, l'uso dei widget principali e come strutturare piccole applicazioni desktop rapidamente. Evidenziamo pratiche di layout e separazione tra logica e interfaccia.

Tkinter è ideale per utility locali, strumenti didattici e prototipi che richiedono input/feedback immediati.

\section{Obiettivi di Apprendimento}
\begin{itemize}
  \item Creare una finestra e aggiungere widget.
  \item Gestire eventi e callback.
  \item Organizzare layout con \texttt{pack}/\texttt{grid}.
\end{itemize}

\section{Concetti Fondamentali}
\begin{tcolorbox}[title=Event loop e layout]
- Il loop di eventi gestisce input e aggiornamenti UI.
- Scegli tra \texttt{pack}, \texttt{grid} e \texttt{place} consapevolmente.
- Evita blocchi: delega lavoro pesante a thread/processi.
- Separa logica dall'interfaccia (controller vs view).
\end{tcolorbox}

\section{Esempi Pratici}
\subsection{Finestra base}
\begin{lstlisting}
import tkinter as tk
root = tk.Tk()
root.title("Hello")
tk.Label(root, text="Ciao!").pack()
root.mainloop()
\end{lstlisting}

\begin{tcolorbox}[title=Spiegazione]
- \texttt{Tk()} crea la finestra principale; \texttt{mainloop()} avvia il ciclo eventi.
- I widget (\texttt{Label}, \texttt{Button}, ecc.) si posizionano con gestori di geometria.
- Tenere la logica separata (controller) per testabilità e manutenzione.
\end{tcolorbox}

\subsection{Button e callback}
\begin{lstlisting}
import tkinter as tk
def on_click():
    print("Clicked")
root = tk.Tk()
tk.Button(root, text="Click", command=on_click).pack()
root.mainloop()
\end{lstlisting}

\begin{tcolorbox}[title=Spiegazione]
- Le callback legano eventi UI a funzioni/metodi; evitare lavoro pesante nel thread GUI.
- Per attività lunghe, usare thread/processi e comunicare lo stato alla UI.
- Validare input e gestire errori senza bloccare l'interfaccia.
\end{tcolorbox}

\subsection{Layout grid}
\begin{lstlisting}
import tkinter as tk
root = tk.Tk()
tk.Label(root, text="Nome").grid(row=0, column=0)
tk.Entry(root).grid(row=0, column=1)
root.mainloop()
\end{lstlisting}

\begin{tcolorbox}[title=Spiegazione]
- \texttt{grid} posiziona i widget in righe/colonne; utile per form.
- Non mescolare \texttt{pack} e \texttt{grid} nello stesso container.
- Pianificare layout responsivi e gestione del focus per UX migliore.
\end{tcolorbox}

\begin{tcolorbox}[title=Casi d'uso]
- Tool desktop per gestione file o batch process.
- Applicazioni didattiche con feedback immediato.
- Pannelli di controllo per script di automazione.
- Editor semplici con campi di input e validazioni.
\end{tcolorbox}

\section{Esercizi}
\begin{enumerate}
  \item Crea una calcolatrice semplice.
  \item Implementa un form con validazione.
  \item Aggiungi un menù e scorciatoie da tastiera.
  \item Usa \texttt{grid} per un layout tabellare.
  \item Mostra un dialog di file e stampa il percorso scelto.
\end{enumerate}

\section{Riepilogo}
Hai costruito GUI di base con Tkinter.

\section{Contesto e Applicazioni}
\begin{tcolorbox}[title=Contesto e Applicazioni]
- Utility desktop per gestione batch e file.
- Prototipi didattici interattivi.
- Pannelli controllo per script e servizi locali.
- Piccole app form-based con validazioni.
\end{tcolorbox}

\section{Approfondimenti}
\begin{tcolorbox}[title=Spiegazioni dettagliate]
\begin{itemize}
  \item \textbf{Event loop}: Tk gestisce eventi e callback; evitare blocchi lunghi nel main thread.
  \item \textbf{Geometry managers}: \verb|pack|, \verb|grid| e \verb|place|; scegliere coerentemente e non mescolarli sullo stesso container.
  \item \textbf{Callback}: funzioni o metodi legati ai widget; gestire stato con oggetti per coesione.
  \item \textbf{Threading}: operazioni I/O fuori dal thread GUI; comunicare con code o eventi pianificati.
\end{itemize}
\end{tcolorbox}

\paragraph{Qualità} Separare logica della UI dalla logica applicativa; testare componenti in isolamento quando possibile.
Consulta: \url{https://docs.python.org/3/library/tk.html}.
