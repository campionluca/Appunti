% Modulo 11 — GUI con Tkinter
\chapter{GUI con Tkinter}

\begin{tcolorbox}[title=Obiettivi del capitolo]
Dopo questo capitolo saprai:
\begin{itemize}
  \item Creare finestre e dialog con Tkinter
  \item Usare widgets base (Label, Button, Entry, Text)
  \item Gestire layout con pack, grid, place
  \item Implementare eventi e callbacks
  \item Usare widgets avanzati (Combobox, Scale, Progressbar)
  \item Creare menu e dialogs
  \item Usare Canvas per grafica
  \item Threading per operazioni lunghe
  \item Applicare pattern MVC per GUI
\end{itemize}
\end{tcolorbox}

\section{Teoria}

Tkinter è la libreria GUI standard di Python (inclusa). Basata su Tk/Tcl, è:
\begin{itemize}
  \item \textbf{Cross-platform}: Windows, macOS, Linux
  \item \textbf{Lightweight}: Nessuna dipendenza esterna
  \item \textbf{Event-driven}: Main loop gestisce eventi
  \item \textbf{Semplice}: Ideale per tool, prototipi, utility
\end{itemize}

\textbf{Limitazioni}:
\begin{itemize}
  \item Look nativo solo con ttk (themed widgets)
  \item Meno widgets/features vs PyQt/wxPython
  \item Performance limitata con molti widgets
\end{itemize}

\section{Finestra Base}

\begin{lstlisting}
import tkinter as tk

# Crea finestra principale
root = tk.Tk()
root.title("My Application")
root.geometry("400x300")  # Width x Height

# Aggiungi widget
label = tk.Label(root, text="Hello, Tkinter!")
label.pack()

# Start event loop
root.mainloop()
\end{lstlisting}

\section{Widgets Base}

\subsection{Label}

\begin{lstlisting}
import tkinter as tk

root = tk.Tk()

# Label con testo
label1 = tk.Label(root, text="Simple Label")
label1.pack()

# Label con styling
label2 = tk.Label(
    root,
    text="Styled Label",
    font=("Arial", 16, "bold"),
    fg="blue",
    bg="yellow"
)
label2.pack()

# Label con variabile dinamica
text_var = tk.StringVar()
text_var.set("Dynamic Text")

label3 = tk.Label(root, textvariable=text_var)
label3.pack()

# Update text
text_var.set("Updated Text!")

root.mainloop()
\end{lstlisting}

\subsection{Button}

\begin{lstlisting}
import tkinter as tk

def on_click():
    print("Button clicked!")
    label.config(text="Button was clicked")

root = tk.Tk()

label = tk.Label(root, text="Click the button")
label.pack()

button = tk.Button(
    root,
    text="Click Me",
    command=on_click,
    bg="green",
    fg="white",
    font=("Arial", 12)
)
button.pack()

root.mainloop()
\end{lstlisting}

\subsection{Entry (Text Input)}

\begin{lstlisting}
import tkinter as tk

def submit():
    text = entry.get()
    print(f"Input: {text}")
    result_label.config(text=f"You entered: {text}")
    entry.delete(0, tk.END)  # Clear

root = tk.Tk()

entry = tk.Entry(root, width=30)
entry.pack()

button = tk.Button(root, text="Submit", command=submit)
button.pack()

result_label = tk.Label(root, text="")
result_label.pack()

root.mainloop()
\end{lstlisting}

\subsection{Text (Multi-line)}

\begin{lstlisting}
import tkinter as tk

root = tk.Tk()

text = tk.Text(root, height=10, width=40)
text.pack()

# Insert text
text.insert(tk.END, "Line 1\n")
text.insert(tk.END, "Line 2\n")

# Get text
def get_text():
    content = text.get("1.0", tk.END)  # From start to end
    print(content)

tk.Button(root, text="Get Text", command=get_text).pack()

root.mainloop()
\end{lstlisting}

\section{Layout Managers}

\subsection{Pack}

\begin{lstlisting}
import tkinter as tk

root = tk.Tk()

# Stack vertically (default)
tk.Label(root, text="Top").pack()
tk.Label(root, text="Middle").pack()
tk.Label(root, text="Bottom").pack()

# Side by side
tk.Label(root, text="Left").pack(side=tk.LEFT)
tk.Label(root, text="Right").pack(side=tk.RIGHT)

# With padding
tk.Label(root, text="Padded").pack(padx=20, pady=10)

root.mainloop()
\end{lstlisting}

\subsection{Grid}

\begin{lstlisting}
import tkinter as tk

root = tk.Tk()

# Grid layout (row, column)
tk.Label(root, text="Name:").grid(row=0, column=0, sticky=tk.W)
tk.Entry(root).grid(row=0, column=1)

tk.Label(root, text="Email:").grid(row=1, column=0, sticky=tk.W)
tk.Entry(root).grid(row=1, column=1)

tk.Label(root, text="Password:").grid(row=2, column=0, sticky=tk.W)
tk.Entry(root, show="*").grid(row=2, column=1)

tk.Button(root, text="Submit").grid(row=3, column=0, columnspan=2)

root.mainloop()
\end{lstlisting}

\section{Eventi e Bindings}

\begin{lstlisting}
import tkinter as tk

def on_key(event):
    print(f"Key pressed: {event.char}")

def on_mouse_click(event):
    print(f"Mouse clicked at ({event.x}, {event.y})")

root = tk.Tk()

# Bind keyboard
root.bind("<Key>", on_key)
root.bind("<Return>", lambda e: print("Enter pressed"))

# Bind mouse
canvas = tk.Canvas(root, width=300, height=200, bg="white")
canvas.pack()
canvas.bind("<Button-1>", on_mouse_click)  # Left click

root.mainloop()
\end{lstlisting}

\section{Widgets Avanzati (ttk)}

\begin{lstlisting}
import tkinter as tk
from tkinter import ttk

root = tk.Tk()

# Combobox
combo = ttk.Combobox(root, values=["Python", "Java", "C++"])
combo.set("Python")
combo.pack()

# Progressbar
progress = ttk.Progressbar(root, length=200, mode="determinate")
progress["maximum"] = 100
progress["value"] = 50
progress.pack()

# Spinbox
spinbox = tk.Spinbox(root, from_=0, to=100, increment=5)
spinbox.pack()

# Scale (Slider)
scale = tk.Scale(root, from_=0, to=100, orient=tk.HORIZONTAL)
scale.pack()

def show_value():
    print(f"Scale value: {scale.get()}")

tk.Button(root, text="Show Value", command=show_value).pack()

root.mainloop()
\end{lstlisting}

\section{Menu}

\begin{lstlisting}
import tkinter as tk
from tkinter import messagebox

def new_file():
    messagebox.showinfo("New", "New file created")

def open_file():
    messagebox.showinfo("Open", "Open file dialog")

def exit_app():
    root.quit()

root = tk.Tk()

# Create menubar
menubar = tk.Menu(root)
root.config(menu=menubar)

# File menu
file_menu = tk.Menu(menubar, tearoff=0)
menubar.add_cascade(label="File", menu=file_menu)
file_menu.add_command(label="New", command=new_file)
file_menu.add_command(label="Open", command=open_file)
file_menu.add_separator()
file_menu.add_command(label="Exit", command=exit_app)

# Edit menu
edit_menu = tk.Menu(menubar, tearoff=0)
menubar.add_cascade(label="Edit", menu=edit_menu)
edit_menu.add_command(label="Cut", command=lambda: print("Cut"))
edit_menu.add_command(label="Copy", command=lambda: print("Copy"))

root.mainloop()
\end{lstlisting}

\section{Dialogs}

\begin{lstlisting}
import tkinter as tk
from tkinter import messagebox, filedialog, simpledialog

root = tk.Tk()

def show_info():
    messagebox.showinfo("Info", "This is info")

def show_warning():
    messagebox.showwarning("Warning", "This is warning")

def show_error():
    messagebox.showerror("Error", "This is error")

def ask_question():
    result = messagebox.askyesno("Question", "Do you like Python?")
    print(f"Answer: {result}")

def open_file():
    filename = filedialog.askopenfilename(
        title="Select file",
        filetypes=[("Text files", "*.txt"), ("All files", "*.*")]
    )
    print(f"Selected: {filename}")

def ask_name():
    name = simpledialog.askstring("Input", "Enter your name:")
    print(f"Name: {name}")

tk.Button(root, text="Info", command=show_info).pack()
tk.Button(root, text="Warning", command=show_warning).pack()
tk.Button(root, text="Error", command=show_error).pack()
tk.Button(root, text="Question", command=ask_question).pack()
tk.Button(root, text="Open File", command=open_file).pack()
tk.Button(root, text="Ask Name", command=ask_name).pack()

root.mainloop()
\end{lstlisting}

\section{Canvas: Grafica}

\begin{lstlisting}
import tkinter as tk

root = tk.Tk()

canvas = tk.Canvas(root, width=400, height=300, bg="white")
canvas.pack()

# Draw shapes
canvas.create_line(0, 0, 400, 300, fill="red", width=2)
canvas.create_rectangle(50, 50, 150, 100, fill="blue", outline="black")
canvas.create_oval(200, 50, 300, 150, fill="green")
canvas.create_polygon(100, 200, 150, 250, 50, 250, fill="yellow")

# Draw text
canvas.create_text(200, 270, text="Hello Canvas!", font=("Arial", 16))

root.mainloop()
\end{lstlisting}

\section{Threading per Operazioni Lunghe}

\begin{lstlisting}
import tkinter as tk
import threading
import time

def long_task():
    """Simula operazione lunga"""
    button.config(state=tk.DISABLED, text="Working...")

    def worker():
        time.sleep(3)  # Simula lavoro
        # Update UI dal main thread
        root.after(0, task_completed)

    thread = threading.Thread(target=worker)
    thread.start()

def task_completed():
    button.config(state=tk.NORMAL, text="Start Task")
    label.config(text="Task completed!")

root = tk.Tk()

label = tk.Label(root, text="Ready")
label.pack()

button = tk.Button(root, text="Start Task", command=long_task)
button.pack()

root.mainloop()
\end{lstlisting}

\section{Progetto: Calculator}

\begin{lstlisting}
import tkinter as tk

class Calculator:
    def __init__(self, root):
        self.root = root
        self.root.title("Calculator")

        self.expression = ""

        # Display
        self.display = tk.Entry(root, font=("Arial", 20), justify=tk.RIGHT)
        self.display.grid(row=0, column=0, columnspan=4, sticky="ew", padx=5, pady=5)

        # Buttons
        buttons = [
            ('7', 1, 0), ('8', 1, 1), ('9', 1, 2), ('/', 1, 3),
            ('4', 2, 0), ('5', 2, 1), ('6', 2, 2), ('*', 2, 3),
            ('1', 3, 0), ('2', 3, 1), ('3', 3, 2), ('-', 3, 3),
            ('0', 4, 0), ('.', 4, 1), ('+', 4, 2), ('=', 4, 3),
        ]

        for (text, row, col) in buttons:
            button = tk.Button(
                root,
                text=text,
                font=("Arial", 16),
                command=lambda t=text: self.on_button_click(t)
            )
            button.grid(row=row, column=col, sticky="nsew", padx=2, pady=2)

        # Clear button
        tk.Button(
            root,
            text="C",
            font=("Arial", 16),
            command=self.clear
        ).grid(row=4, column=3, sticky="nsew", padx=2, pady=2)

        # Configure grid weights
        for i in range(5):
            root.grid_rowconfigure(i, weight=1)
        for i in range(4):
            root.grid_columnconfigure(i, weight=1)

    def on_button_click(self, char):
        if char == '=':
            try:
                result = eval(self.expression)
                self.display.delete(0, tk.END)
                self.display.insert(0, str(result))
                self.expression = str(result)
            except:
                self.display.delete(0, tk.END)
                self.display.insert(0, "Error")
                self.expression = ""
        else:
            self.expression += char
            self.display.delete(0, tk.END)
            self.display.insert(0, self.expression)

    def clear(self):
        self.expression = ""
        self.display.delete(0, tk.END)

root = tk.Tk()
app = Calculator(root)
root.mainloop()
\end{lstlisting}

\begin{tcolorbox}[title=Best Practices]
\begin{itemize}
  \item Separa logica business da UI (pattern MVC/MVP)
  \item Non bloccare main thread (usa threading per I/O)
  \item Usa ttk widgets per look nativo moderno
  \item Valida input prima di processare
  \item Gestisci eccezioni in callbacks
  \item Usa StringVar/IntVar per dati dinamici
  \item Organizza codice in classi per app complesse
  \item Usa grid per form, pack per semplici layout
  \item Non mescolare pack/grid nello stesso container
  \item Fai cleanup su close (close files, threads)
\end{itemize}
\end{tcolorbox}

\begin{tcolorbox}[title=Errori Comuni]
\begin{itemize}
  \item Bloccare main thread con operazioni lunghe
  \item Mescolare pack e grid nello stesso parent
  \item Non gestire eccezioni in callbacks
  \item Usare eval() senza validazione (security!)
  \item Dimenticare mainloop()
  \item Creare troppe istanze Tk() (una basta)
  \item Non configurare grid weights (resize issues)
  \item Non usare ttk (look old-school)
  \item Update UI da thread secondari (crash!)
  \item Memory leak: non cleanup widgets rimossi
\end{itemize}
\end{tcolorbox}

\section{Esercizi}

\subsection{Livello Base}
\begin{enumerate}
  \item Crea form login (username, password, button)
  \item Implementa converter temperatura (Celsius ↔ Fahrenheit)
  \item Todo list (add, display, clear)
\end{enumerate}

\subsection{Livello Intermedio}
\begin{enumerate}
  \item Text editor con menu (New, Open, Save, Exit)
  \item Calculator completa con history
  \item Image viewer con file dialog
\end{enumerate}

\subsection{Livello Avanzato}
\begin{enumerate}
  \item Drawing app con Canvas (brush, colors, clear)
  \item Database browser (SQLite) con CRUD operations
  \item File manager con tree view e operations
\end{enumerate}

\section{Riferimenti}
\begin{itemize}
  \item Tkinter Documentation: \url{https://docs.python.org/3/library/tk.html}
  \item Tkinter Tutorial: \url{https://realpython.com/python-gui-tkinter/}
  \item ttk widgets: \url{https://docs.python.org/3/library/tkinter.ttk.html}
\end{itemize}
