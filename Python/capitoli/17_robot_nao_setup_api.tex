% Modulo 17 - Setup ambiente e API NAOqi in Python
\chapter{Setup Ambiente e API NAOqi}

\section{Contesto e Applicazioni}
\begin{tcolorbox}[title=Contesto e Applicazioni]
- Preparazione ambiente di sviluppo e dipendenze.
- Connessione sicura al robot e gestione rete.
- Struttura di progetto per script e demo.
\end{tcolorbox}

\section{Approfondimenti}
Per NAOqi 2.x in Python si usa tipicamente il pacchetto \texttt{qi} per la sessione e i servizi. Organizzare un piccolo progetto con virtualenv e file di configurazione (host/IP, porte, credenziali se presenti).

\begin{tcolorbox}[title=Sessione e servizi]
\begin{lstlisting}
import qi

def connect(address="tcp://nao.local:9559"):
    session = qi.Session()
    try:
        session.connect(address)
    except RuntimeError as e:
        print("Connessione fallita:", e)
        raise
    return session

session = connect()
tts = session.service("ALTextToSpeech")
posture = session.service("ALRobotPosture")
motion = session.service("ALMotion")
\end{lstlisting}
\end{tcolorbox}

\paragraph{Networking} Preferire IP statico o hostname risolvibile; verificare porte aperte e firewall. Per contesti multi-robot, definire discovery e registri di indirizzi.

\begin{tcolorbox}[title=Spiegazione per sviluppatori non-Python]
\textbf{Virtualenv}: in Python si crea un ambiente isolato per dipendenze con \texttt{python -m venv .venv} e si attiva prima di installare pacchetti.  
\textbf{Configurazione}: usare un file \texttt{config.json} o variabili d'ambiente per host/porta, quindi leggerli con il modulo \texttt{json}/\texttt{os}.

\textbf{Pattern di connessione}:
\begin{lstlisting}
import os, qi

HOST = os.getenv("NAO_HOST", "nao.local")
PORT = int(os.getenv("NAO_PORT", "9559"))

def get_session():
    s = qi.Session()
    s.connect(f"tcp://{HOST}:{PORT}")
    return s

session = get_session()
tts = session.service("ALTextToSpeech")
tts.say("Configurazione caricata da ENV")
\end{lstlisting}

\textbf{Eccezioni e retry}: implementare un semplice \emph{retry} con backoff se la connessione fallisce; usare \texttt{time.sleep} e un numero massimo di tentativi.
\end{tcolorbox}
