% Modulo 06 — Gestione Errori
\chapter{Gestione Errori}

\section{Introduzione}
Gestire eccezioni con \texttt{try/except}, \texttt{finally} e sollevare errori custom.

La descrizione enfatizza robustezza e chiarezza: quando intercettare e quando propagare, come loggare errori in modo utile e come definire gerarchie di eccezioni significative. Mostriamo pattern come "fail fast" e cleanup con \texttt{finally}.

Una buona gestione degli errori migliora affidabilità di script, servizi web, CLI e integrazioni con file e rete.

\section{Obiettivi di Apprendimento}
\begin{itemize}
  \item Intercettare eccezioni comuni (ValueError, FileNotFoundError).
  \item Usare \texttt{try/except/else/finally}.
  \item Definire eccezioni personalizzate.
\end{itemize}

\section{Concetti Fondamentali}
\begin{tcolorbox}[title=Eccezioni]
Esempio: \texttt{raise ValueError("messaggio")}
\end{tcolorbox}
\begin{tcolorbox}[title=Best practice]
Usa messaggi informativi, intercetta eccezioni specifiche, evita \texttt{except:} generici. Logga l'errore dove ha senso e decidi se propagare o tradurre in un fallback chiaro; pulisci risorse con \texttt{finally}.
\end{tcolorbox}

\section{Esempi Pratici}
\subsection{Try/Except}
\begin{lstlisting}
try:
    x = int("12a")
except ValueError:
    print("Input non valido")
\end{lstlisting}
\begin{tcolorbox}[title=Spiegazione del codice]
- \textbf{Eccezione specifica}: `ValueError` per conversione fallita; evita `except:` generici.  
- \textbf{Messaggio}: fornisce feedback chiaro; separare validazione da logica di parsing.  
 - \textbf{Propagazione}: se l'errore non è atteso, lasciare propagare.
\end{tcolorbox}

\subsection{Finally}
\begin{lstlisting}
f = None
try:
    f = open("file.txt")
    # ... usa f
finally:
    if f:
        f.close()
\end{lstlisting}
\begin{tcolorbox}[title=Spiegazione del codice]
- \textbf{Cleanup garantito}: `finally` esegue sempre; evita leak di risorse.  
- \textbf{Preferenza}: usare `with open(...)` al posto del `try/finally` manuale.  
 - \textbf{Stato}: inizializzare `f=None` evita `UnboundLocalError` se l'apertura fallisce.
\end{tcolorbox}

\subsection{Custom Exception}
\begin{lstlisting}
class NegativeError(Exception):
    pass

def sqrt(x):
    if x < 0:
        raise NegativeError("Valore negativo")
    return x ** 0.5
\end{lstlisting}
\begin{tcolorbox}[title=Spiegazione del codice]
- \textbf{Gerarchia}: eredita da `Exception`; personalizza il significato di dominio.  
- \textbf{Contratto}: precondizione `x >= 0`; documentarla nel docstring e nelle annotazioni.  
- \textbf{Segnalazione}: usare messaggi informativi per diagnosi e UX.
\end{tcolorbox}

\begin{tcolorbox}[title=Casi d'uso]
\begin{itemize}
  \item Validazione input e regole di dominio con eccezioni chiare.
  \item Gestione errori di rete/file con retry e fallback.
  \item Traduzione di errori tecnici in messaggi per utenti/CLI.
\end{itemize}
\end{tcolorbox}

\section{Esercizi}
\begin{enumerate}
  \item Scrivi una funzione che gestisce input numerici con retry.
  \item Gestisci errori di rete simulati con \texttt{try/except}.
  \item Definisci un'eccezione custom e usala in una validazione.
  \item Usa \texttt{else} per eseguire codice solo senza eccezioni.
  \item Registra errori in un file di log.
\end{enumerate}

\section{Riepilogo}
Hai imparato a gestire e sollevare eccezioni correttamente.

\section{Contesto e Applicazioni}
\begin{tcolorbox}[title=Contesto e Applicazioni]
- Resilienza di script (retry, gestione eccezioni prevedibili).
 - Validazione input e messaggi d'errore chiari.
- Gestione risorse con context manager.
- Strategie di fallback e logging di anomalie.
\end{tcolorbox}

\section{Approfondimenti}
\begin{tcolorbox}[title=Spiegazioni dettagliate]
\begin{itemize}
  \item \textbf{try/except/else/finally}: \verb|else| esegue se non ci sono eccezioni; \verb|finally| sempre per cleanup.
  \item \textbf{Eccezioni personalizzate}: definire gerarchie chiare; documentare quando e perché vengono sollevate.
  \item \textbf{Raising e reraising}: \verb|raise| per propagare; conservare il traceback con \verb|raise| senza argomenti nel blocco \verb|except|.
\item \textbf{EAFP vs LBYL}: preferire "it's easier to ask forgiveness than permission" con gestione puntuale delle eccezioni.
  \item \textbf{Logging}: usare il modulo \verb|logging| per diagnosi e telemetria, evitare \verb|print| in codice di libreria.
\end{itemize}
\end{tcolorbox}

\begin{tcolorbox}[title={Per chi viene da C/Java/JS: modello eccezioni, retry e logging}]
\textbf{Eccezioni}: Python usa eccezioni per condizioni anomale; evita codici di ritorno speciali.  
\textbf{Specificità}: cattura eccezioni specifiche (`ValueError`, `FileNotFoundError`) e lascia propagare quelle inattese.  
\textbf{Cleanup}: preferisci context manager (`with`) rispetto a `try/finally` manuale per risorse.  
\textbf{Retry/backoff}: implementa retry con `time.sleep` o librerie; non nascondere errori irreparabili.

Esempi commentati:
\begin{lstlisting}
import logging, time
log = logging.getLogger(__name__)

def parse_int(s):
    try:
        return int(s)
    except ValueError:
        log.warning("input non numerico: %s", s)
        return None

def fetch_with_retry(fn, attempts=3, delay=0.5):
    for i in range(attempts):
        try:
            return fn()
        except TimeoutError:
            log.info("timeout, tentativo %s", i + 1)
            time.sleep(delay)
    raise RuntimeError("fallimento dopo retry")
\end{lstlisting}
\begin{tcolorbox}[title=Spiegazione del codice]
- \textbf{Logging}: usa \verb|getLogger(__name__)|; preferisci livelli (`warning`, `info`) a `print`.  
- \textbf{Retry}: intercetta eccezioni attese (`TimeoutError`) e applica backoff con `sleep`.  
- \textbf{Fallback}: dopo tentativi, solleva errore aggregato chiaro; non nascondere failure.  
- \textbf{Parse}: ritorna `None` su input non numerico; scegliere `Optional[int]` nel contratto.
\end{tcolorbox}
\end{tcolorbox}

\begin{tcolorbox}[title={Idiomi: EAFP, traduzione errori, context manager}]
- \textbf{EAFP}: prova e gestisci l'eccezione dove ha senso, invece di pre‑controllare ovunque.
- \textbf{Traduzione}: traduci eccezioni tecniche in errori di dominio (con messaggi utili).  
- \textbf{Context manager}: usa `with` per risorse e per garantire rollback/cleanup automatici.
\end{tcolorbox}

\paragraph{Consigli} Controllare granularità degli \verb|except|, evitare catture generiche; assicurare messaggi di errore chiari e azionabili.
Consulta: \url{https://docs.python.org/3/tutorial/errors.html}.
