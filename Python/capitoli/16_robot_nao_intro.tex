% Modulo 16 - Introduzione a NAO (V6) e NAOqi
\chapter{Introduzione a NAO (V6) e NAOqi}

\section{Contesto e Applicazioni}
\begin{tcolorbox}[title=Contesto e Applicazioni]
- Didattica STEM e laboratori universitari.
- Social robotics, interazione uomo-robot (HRI).
- Demo e prototipi per mostre ed eventi.
- Sperimentazione di visione, linguaggio e movimento.
\end{tcolorbox}

\section{Approfondimenti}
NAO (V6) utilizza NAOqi 2.x: un framework di servizi (\emph{AL\*}) accessibili via rete. La programmazione in Python avviene tipicamente creando una sessione (\texttt{qi}) e recuperando i servizi, come \texttt{ALTextToSpeech}, \texttt{ALMotion}, \texttt{ALRobotPosture}, \texttt{ALVideoDevice}.

\begin{tcolorbox}[title=Prime connessioni]
\begin{lstlisting}
# Connessione ai servizi NAOqi (NAO V6)
import qi

session = qi.Session()
session.connect("tcp://nao.local:9559")  # sostituisci host/IP

tts = session.service("ALTextToSpeech")
tts.say("Ciao! Sono NAO")

motion = session.service("ALMotion")
posture = session.service("ALRobotPosture")
# Passa alla postura base prima di muovere
posture.goToPosture("StandInit", 0.5)
\end{lstlisting}
\end{tcolorbox}

\paragraph{Note pratiche} Assicurare connettività (Wi-Fi/Ethernet) e versione corretta del SDK. Per stabilità, gestire eccezioni di rete e latenza; loggare operazioni critiche.

\begin{tcolorbox}[title=Guida rapida per programmatori non-Python]
Python è tipizzato dinamicamente e l'indentazione fa parte della sintassi (blocchi). Le eccezioni sono il meccanismo standard di errore; non si usano codici di ritorno come in C. Le liste (array dinamici) e i dizionari (mappe chiave->valore) sono strutture fondamentali.

- Import: \texttt{import qi} carica il modulo del SDK NAOqi.  
- Sessione: \texttt{qi.Session()} crea un oggetto che gestisce la connessione TCP verso il robot.  
- Servizi: \texttt{session.service("ALMotion")} risolve il servizio remoto e ritorna un proxy su cui invocare metodi.

\textbf{Gestione errori}: usare \texttt{try/except} per catturare \texttt{RuntimeError} quando la connessione fallisce.  
\textbf{Stringhe}: sono Unicode; passare testo direttamente a \texttt{tts.say}.  
\textbf{Convenzione}: nomi come \texttt{goToPosture} sono in stile CamelCase perché ereditati dalle API NAOqi.
\end{tcolorbox}

\begin{tcolorbox}[title=Connessione commentata passo-passo]
\begin{lstlisting}
import qi

def connect(host="nao.local", port=9559):
    """Stabilisce una sessione TCP con NAO.
    - host: hostname o IP del robot
    - port: porta NAOqi (default 9559)
    Ritorna: qi.Session connessa.
    """
    session = qi.Session()
    try:
        session.connect(f"tcp://{host}:{port}")
    except RuntimeError as e:
        # In Python le eccezioni si catturano per tipo; qui gestiamo problemi di rete
        print("Connessione fallita:", e)
        raise
    return session

session = connect()
tts = session.service("ALTextToSpeech")  # Proxy al servizio TTS
motion = session.service("ALMotion")      # Proxy al servizio di movimento
posture = session.service("ALRobotPosture")

# Prima di muovere: posizionare NAO in postura stabile
posture.goToPosture("StandInit", 0.5)  # 0.5 = fattore di velocità (0..1)
tts.say("Pronto a muovermi!")
\end{lstlisting}
\end{tcolorbox}

