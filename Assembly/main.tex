\documentclass[a4paper,12pt,twoside]{book}

% ===========================
% PACCHETTI ESSENZIALI
% ===========================
\usepackage[utf8]{inputenc}
\usepackage[T1]{fontenc}
\usepackage[italian]{babel}
\usepackage{geometry}
\geometry{
    a4paper,
    left=2.5cm,
    right=2.5cm,
    top=3cm,
    bottom=3cm
}

% ===========================
% GRAFICA E COLORI
% ===========================
\usepackage{graphicx}
\usepackage{xcolor}
\usepackage{tikz}
\usetikzlibrary{shapes.geometric, arrows.meta, positioning, calc, backgrounds}

% ===========================
% MATEMATICA
% ===========================
\usepackage{amsmath, amssymb, amsthm}

% ===========================
% TABELLE E LISTE
% ===========================
\usepackage{booktabs}
\usepackage{array}
\usepackage{enumitem}
\usepackage{multirow}
\usepackage{longtable}

% ===========================
% CODICE SORGENTE
% ===========================
\usepackage{listings}

% Definizione linguaggio Assembly 8086
\lstdefinelanguage[x86]{Assembler}{
    morekeywords={
        % Istruzioni trasferimento dati
        MOV, XCHG, LEA, LDS, LES, PUSH, POP, PUSHF, POPF,
        % Istruzioni aritmetiche
        ADD, ADC, SUB, SBB, INC, DEC, NEG, CMP, MUL, IMUL, DIV, IDIV,
        DAA, DAS, AAA, AAS, AAM, AAD, CBW, CWD,
        % Istruzioni logiche
        AND, OR, XOR, NOT, TEST, SHL, SHR, SAL, SAR, ROL, ROR, RCL, RCR,
        % Istruzioni di controllo
        JMP, JE, JZ, JNE, JNZ, JA, JAE, JB, JBE, JG, JGE, JL, JLE,
        JS, JNS, JO, JNO, JP, JPE, JNP, JPO, JCXZ,
        CALL, RET, RETF, RETN, IRET,
        LOOP, LOOPE, LOOPZ, LOOPNE, LOOPNZ,
        % Istruzioni stringhe
        MOVSB, MOVSW, CMPSB, CMPSW, SCASB, SCASW, LODSB, LODSW, STOSB, STOSW,
        REP, REPE, REPZ, REPNE, REPNZ,
        % Interrupt e I/O
        INT, INTO, IN, OUT,
        % Controllo processore
        NOP, HLT, WAIT, ESC, LOCK, STI, CLI, STD, CLD, CLC, STC, CMC,
        LAHF, SAHF,
        % Direttive assembler
        DB, DW, DD, DQ, DT, EQU, ORG, END, SEGMENT, ENDS, ASSUME,
        PROC, ENDP, OFFSET, PTR, BYTE, WORD, DWORD,
        % Segmenti e modelli
        .CODE, .DATA, .STACK, .MODEL, SMALL, MEDIUM, COMPACT, LARGE, HUGE
    },
    sensitive=false,
    morecomment=[l]{;},
    morestring=[b]",
    morestring=[b]'
}

% Stile listing Assembly
\lstdefinestyle{assemblystyle}{
    language=[x86]Assembler,
    basicstyle=\ttfamily\small,
    keywordstyle=\color{blue}\bfseries,
    commentstyle=\color{green!60!black}\itshape,
    stringstyle=\color{red},
    numbers=left,
    numberstyle=\tiny\color{gray},
    stepnumber=1,
    numbersep=8pt,
    backgroundcolor=\color{gray!5},
    frame=single,
    rulecolor=\color{black!30},
    tabsize=4,
    captionpos=b,
    breaklines=true,
    breakatwhitespace=false,
    showstringspaces=false,
    showtabs=false,
    xleftmargin=15pt,
    xrightmargin=5pt,
    framexleftmargin=12pt
}

\lstset{style=assemblystyle}

% ===========================
% BOX COLORATI (tcolorbox)
% ===========================
\usepackage[most]{tcolorbox}

% Box Definizione
\newtcolorbox{definizione}{
    colback=blue!5,
    colframe=blue!60!black,
    fonttitle=\bfseries,
    title=Definizione,
    enhanced,
    attach boxed title to top left={yshift=-2mm, xshift=5mm},
    boxed title style={colback=blue!60!black}
}

% Box Attenzione
\newtcolorbox{attenzione}{
    colback=red!5,
    colframe=red!70!black,
    fonttitle=\bfseries,
    title=Attenzione,
    enhanced,
    attach boxed title to top left={yshift=-2mm, xshift=5mm},
    boxed title style={colback=red!70!black}
}

% Box Nota
\newtcolorbox{nota}{
    colback=yellow!10,
    colframe=orange!80!black,
    fonttitle=\bfseries,
    title=Nota,
    enhanced,
    attach boxed title to top left={yshift=-2mm, xshift=5mm},
    boxed title style={colback=orange!80!black}
}

% Box Esempio
\newtcolorbox{esempio}{
    colback=green!5,
    colframe=green!60!black,
    fonttitle=\bfseries,
    title=Esempio,
    enhanced,
    attach boxed title to top left={yshift=-2mm, xshift=5mm},
    boxed title style={colback=green!60!black}
}

% Box Esercizio
\newtcolorbox{esercizio}[1][]{
    colback=cyan!5,
    colframe=cyan!60!black,
    fonttitle=\bfseries,
    title={Esercizio #1},
    enhanced,
    attach boxed title to top left={yshift=-2mm, xshift=5mm},
    boxed title style={colback=cyan!60!black}
}

% ===========================
% HYPERREF (sempre ultimo)
% ===========================
\usepackage{hyperref}
\hypersetup{
    colorlinks=true,
    linkcolor=blue,
    filecolor=magenta,
    urlcolor=cyan,
    citecolor=green,
    pdftitle={Programmazione Assembly 8086},
    pdfauthor={ITS Antonio Scarpa},
    pdfsubject={Assembly},
    pdfkeywords={Assembly, 8086, architettura, programmazione}
}

% ===========================
% INFORMAZIONI DOCUMENTO
% ===========================
\title{\Huge\bfseries Programmazione Assembly 8086}
\author{Istituto Tecnico Superiore Antonio Scarpa\\Anno Scolastico 2025-2026}
\date{\today}

% ===========================
% INIZIO DOCUMENTO
% ===========================
\begin{document}

% Frontespizio
\maketitle

% Indice
\tableofcontents

% ===========================
% PARTE I: FONDAMENTI
% ===========================
\part{Fondamenti dell'Architettura 8086}

\chapter*{Prefazione}
\addcontentsline{toc}{chapter}{Prefazione}

\section*{A chi è rivolto questo libro}

Questi appunti sono stati pensati per gli studenti del quarto anno di Istituto Tecnico che stanno approfondendo la programmazione in Java. Il materiale presuppone una conoscenza di base del linguaggio (variabili, cicli, metodi, concetti fondamentali di programmazione) e si propone di consolidare e ampliare tali competenze attraverso argomenti più avanzati e pratici.

L'approccio adottato bilancia teoria ed esempi concreti, con l'obiettivo di fornire strumenti immediatamente applicabili sia nei progetti scolastici che in contesti reali.

\section*{Struttura del libro}

Il libro è organizzato in otto capitoli, ciascuno focalizzato su un argomento specifico:

\begin{enumerate}
    \item \textbf{Classi, Oggetti e Ereditarietà}: ripasso e approfondimento dei concetti fondamentali della programmazione orientata agli oggetti, con particolare attenzione agli array di oggetti e alla gerarchia tra classi.

    \item \textbf{Stream e Buffer}: gestione di flussi di dati per leggere e scrivere file, con esempi pratici di utilizzo delle classi più comuni.

    \item \textbf{Interfacce e Classi Astratte}: meccanismi per definire comportamenti comuni e creare gerarchie flessibili.

    \item \textbf{Eccezioni}: gestione degli errori a runtime attraverso il sistema delle eccezioni di Java.

    \item \textbf{ArrayList}: struttura dati dinamica per gestire collezioni di elementi in modo più flessibile rispetto agli array tradizionali.

    \item \textbf{Interfacce Grafiche}: introduzione alla creazione di applicazioni con interfaccia grafica usando Swing, inclusa la gestione degli eventi.

    \item \textbf{Model View Controller}: pattern architetturale per organizzare il codice separando logica, presentazione e controllo.

    \item \textbf{Lambda Expressions}: cenni alle espressioni lambda introdotte in Java 8, per scrivere codice più conciso ed espressivo.
\end{enumerate}

\section*{Come usare questo libro}

Ogni capitolo è strutturato per guidare l'apprendimento in modo progressivo:

\begin{itemize}
    \item Gli \textbf{obiettivi di apprendimento} all'inizio di ogni capitolo chiariscono cosa ci si aspetta di saper fare al termine dello studio.

    \item La \textbf{teoria} è presentata in modo sintetico ma completo, con definizioni chiare e schemi quando necessario.

    \item Gli \textbf{esempi di codice} sono commentati in italiano e mostrano l'applicazione pratica dei concetti. Si consiglia di digitare personalmente ogni esempio, eseguirlo e sperimentare modifiche per comprenderne il funzionamento.

    \item I \textbf{box colorati} evidenziano informazioni particolari:
    \begin{itemize}
        \item \textcolor{orange}{Arancione (Attenzione)}: punti critici da ricordare
        \item \textcolor{blue}{Blu (Nota)}: suggerimenti e best practices
        \item \textcolor{red}{Rosso (Errore Comune)}: errori frequenti da evitare
    \end{itemize}

    \item Gli \textbf{esercizi} sono suddivisi in tre livelli di difficoltà (base, intermedio, avanzato). Si consiglia di affrontarli in ordine, verificando le soluzioni commentate nell'appendice solo dopo aver tentato autonomamente.

    \item Il \textbf{riepilogo} alla fine di ogni capitolo sintetizza i concetti chiave e facilita il ripasso.
\end{itemize}

\section*{Prerequisiti}

Per affrontare efficacemente questi appunti, è necessario:

\begin{itemize}
    \item Conoscere la sintassi base di Java (tipi di dato primitivi, operatori, strutture di controllo)
    \item Saper dichiarare e utilizzare metodi
    \item Comprendere i concetti basilari di classe e oggetto
    \item Avere familiarità con array monodimensionali
    \item Disporre di un ambiente di sviluppo Java funzionante (JDK 8 o superiore, IDE come Eclipse, IntelliJ IDEA o NetBeans)
\end{itemize}

\section*{Convenzioni utilizzate}

\textbf{Codice}: tutti gli esempi di codice sono presentati con sintassi evidenziata, numerazione delle righe e commenti esplicativi.

\textbf{Nomenclatura}: si segue la convenzione Java standard (CamelCase per classi, camelCase per metodi e variabili, MAIUSCOLO per costanti).

\textbf{Terminologia}: si preferisce l'italiano quando possibile, mantenendo i termini tecnici in inglese quando consolidati nella pratica professionale (ad esempio "stream", "buffer", "exception").

\vspace{1cm}

Buono studio!

\chapter{Architettura del Microprocessore 8086}

\section{Introduzione}

Il microprocessore Intel 8086, introdotto nel 1978, rappresenta una pietra miliare nella storia dell'informatica. È il primo processore della famiglia x86, architettura che ancora oggi domina il mercato dei personal computer. Comprendere l'8086 significa gettare le basi per la programmazione a basso livello su processori moderni.

\begin{definizione}
Il \textbf{microprocessore 8086} è una CPU a 16 bit con bus dati a 16 bit e bus indirizzi a 20 bit, capace di indirizzare fino a 1 MB di memoria (2\textsuperscript{20} = 1.048.576 byte).
\end{definizione}

\section{Caratteristiche principali}

\subsection{Specifiche tecniche}

\begin{table}[h]
\centering
\begin{tabular}{ll}
\toprule
\textbf{Caratteristica} & \textbf{Valore} \\
\midrule
Architettura & 16 bit \\
Bus dati & 16 bit \\
Bus indirizzi & 20 bit \\
Memoria indirizzabile & 1 MB (1.048.576 byte) \\
Frequenza clock & 5-10 MHz \\
Registri general purpose & 4 × 16 bit \\
Registri segmento & 4 × 16 bit \\
Numero transistor & 29.000 \\
Tecnologia & NMOS 3 µm \\
\bottomrule
\end{tabular}
\caption{Specifiche tecniche Intel 8086}
\end{table}

\subsection{Organizzazione interna}

L'8086 utilizza un'architettura pipeline a due stadi per migliorare le prestazioni. La \textbf{Bus Interface Unit (BIU)} è responsabile del prelievo delle istruzioni dalla memoria e della loro gestione, permettendo il prefetching di più istruzioni. L'\textbf{Execution Unit (EU)} invece esegue le istruzioni decodificate. Questi due componenti funzionano in parallelo: mentre l'EU esegue un'istruzione, la BIU già precarica la prossima, creando un effetto pipeline che migliora significativamente la velocità di esecuzione.

\begin{center}
\begin{tikzpicture}[
    node distance=1.5cm and 2cm,
    box/.style={rectangle, draw, fill=blue!20, text width=3cm, align=center, minimum height=1cm},
    queue/.style={rectangle, draw, fill=green!20, text width=2cm, align=center, minimum height=0.8cm}
]
    % BIU
    \node[box] (biu) {Bus Interface Unit (BIU)};
    \node[queue, below=of biu] (queue) {Coda Istruzioni\\(6 byte)};
    \node[box, below=of queue] (mem) {Bus Memoria\\(20 bit)};

    % EU
    \node[box, right=of biu] (eu) {Execution Unit (EU)};
    \node[box, below=of eu] (alu) {ALU\\16 bit};
    \node[box, right=of eu] (regs) {Registri};

    % Frecce
    \draw[->] (biu) -- (queue);
    \draw[->] (queue) -- (mem);
    \draw[<->] (queue) -- (eu);
    \draw[->] (eu) -- (alu);
    \draw[<->] (eu) -- (regs);
\end{tikzpicture}
\end{center}

\begin{nota}
La \textbf{coda di istruzioni} (instruction queue) permette alla BIU di precaricare fino a 6 byte di istruzioni mentre la EU esegue l'istruzione corrente. Questo meccanismo di \emph{prefetch} migliora le performance.
\end{nota}

\section{Modalità di funzionamento}

L'8086 può operare in due modalità:

\subsection{Modalità minima}

Sistema monoprocessore dove l'8086 genera autonomamente tutti i segnali di controllo.

\subsection{Modalità massima}

Sistema multiprocessore con coprocessore 8087 (FPU) e controller DMA 8237. Richiede circuiti esterni per la generazione dei segnali.

\section{Memoria segmentata}

L'8086 utilizza un modello di memoria \textbf{segmentata} per indirizzare 1 MB con registri a 16 bit.

\begin{definizione}
Un \textbf{segmento} è un blocco di memoria di massimo 64 KB (2\textsuperscript{16} byte). L'indirizzo fisico viene calcolato combinando un registro segmento e un offset.
\end{definizione}

\subsection{Calcolo indirizzo fisico}

L'indirizzo fisico a 20 bit si ottiene con la formula:

\[
\text{Indirizzo Fisico} = (\text{Segmento} \times 16) + \text{Offset}
\]

Equivalentemente:
\[
\text{Indirizzo Fisico} = (\text{Segmento} \ll 4) + \text{Offset}
\]

\begin{esempio}
Calcolare l'indirizzo fisico per \texttt{DS:SI = 1234h:5678h}:

\begin{align*}
\text{Segmento} &= 1234h \\
\text{Offset} &= 5678h \\
\text{Indirizzo Fisico} &= (1234h \times 10h) + 5678h \\
&= 12340h + 5678h \\
&= 179B8h
\end{align*}

In decimale: $179B8h = 96.696$ byte.
\end{esempio}

\subsection{Rappresentazione grafica}

\begin{center}
\begin{tikzpicture}
    % Registro segmento
    \draw (0,2) rectangle (4,2.5);
    \node at (2,2.25) {\texttt{1234h}};
    \node[left] at (0,2.25) {Segmento:};

    % Shift left 4 bit
    \draw[->, thick] (4.5,2.25) -- (5.5,2.25);
    \node[above] at (5,2.5) {$\times 16$};

    % Segmento shiftato
    \draw (6,2) rectangle (10,2.5);
    \node at (8,2.25) {\texttt{12340h}};

    % Offset
    \draw (6,1) rectangle (10,1.5);
    \node at (8,1.25) {\texttt{5678h}};
    \node[left] at (6,1.25) {Offset:};

    % Somma
    \draw[->, thick] (10.5,1.75) -- (11.5,1.75);
    \node[above] at (11,2) {$+$};

    % Indirizzo fisico
    \draw (12,1.5) rectangle (16,2);
    \node at (14,1.75) {\texttt{179B8h}};
    \node[right] at (16,1.75) {Fisico (20 bit)};
\end{tikzpicture}
\end{center}

\section{Flag register}

Il registro FLAGS (16 bit) contiene bit di stato e controllo:

\begin{table}[h]
\centering
\small
\begin{tabular}{clp{8cm}}
\toprule
\textbf{Bit} & \textbf{Nome} & \textbf{Descrizione} \\
\midrule
0 & CF & \textbf{Carry Flag}: riporto/prestito in operazioni aritmetiche \\
2 & PF & \textbf{Parity Flag}: parità del byte meno significativo \\
4 & AF & \textbf{Auxiliary Carry}: riporto bit 3 (BCD) \\
6 & ZF & \textbf{Zero Flag}: risultato zero \\
7 & SF & \textbf{Sign Flag}: bit di segno del risultato \\
8 & TF & \textbf{Trap Flag}: modalità debug single-step \\
9 & IF & \textbf{Interrupt Flag}: abilita interrupt mascherabili \\
10 & DF & \textbf{Direction Flag}: direzione operazioni su stringhe \\
11 & OF & \textbf{Overflow Flag}: overflow in aritmetica con segno \\
\bottomrule
\end{tabular}
\caption{Flag principali del registro FLAGS}
\end{table}

\begin{attenzione}
I flag \textbf{TF}, \textbf{IF} e \textbf{DF} sono \emph{flag di controllo} impostabili dal programmatore. Gli altri sono \emph{flag di stato} modificati automaticamente dalle istruzioni.
\end{attenzione}

\section{Pinout e segnali}

L'8086 è un circuito integrato a 40 pin organizzati in diverse categorie funzionali. Il \textbf{bus AD0-AD15} è un bus multiplexato a 16 bit che trasporta sia indirizzi che dati. I \textbf{bit A16-A19} rappresentano i 4 bit superiori dell'indirizzo e vengono utilizzati per estendere la capacità di indirizzamento fino ai 20 bit necessari. I \textbf{segnali di controllo} incluono ALE (Address Latch Enable), RD (Read), WR (Write) e IO/M per controllare le operazioni di lettura/scrittura e per distinguere tra accessi a memoria e porta I/O. I pin di \textbf{interrupt} comprendono INTR (interrupt mascherabile), NMI (non-maskable interrupt) e INTA (interrupt acknowledge) per la gestione degli interrupt hardware. Infine, sono presenti pin di \textbf{alimentazione} (VCC e GND) e il pin di \textbf{clock} (CLK) per la sincronizzazione del processore.

\begin{nota}
Il bus AD0-AD15 è \textbf{multiplexato}: trasporta indirizzi durante T1 e dati durante T2-T4 del ciclo bus. Il segnale ALE (Address Latch Enable) indica quando il bus contiene un indirizzo valido.
\end{nota}

\section{Compatibilità e successori}

L'8086 ha dato origine a un'importante famiglia di processori che ha dominato il mercato dei personal computer per decenni. L'\textbf{8088} è una variante con bus dati ridotto a 8 bit, diventato famoso perché scelto da IBM per il suo primo personal computer. I modelli \textbf{80186 e 80188} rappresentano versioni migliorate dell'architettura originale, integrate con periferiche direttamente nel chip. L'\textbf{80286} ha introdotto la modalità protetta e ha ampliato la memoria indirizzabile a 16 MB, segnando l'inizio dell'evoluzione verso processori più potenti. L'\textbf{80386} è stato il primo processore della famiglia x86 a operare completamente a 32 bit, permettendo il multitasking e protettive mode avanzate. Infine, i processori \textbf{Pentium e successivi} rappresentano l'evoluzione moderna dell'architettura x86, mantenendo comunque la retrocompatibilità con il codice Assembly 8086.

\begin{definizione}
La \textbf{retrocompatibilità} dell'architettura x86 permette ai processori moderni di eseguire codice Assembly 8086 in \emph{modalità reale} (real mode).
\end{definizione}

\section{Riepilogo}

Il microprocessore Intel 8086 rappresenta una pietra miliare dell'informatica moderna come processore a 16 bit con bus indirizzi a 20 bit. Grazie al suo innovativo modello di memoria segmentata, è in grado di indirizzare fino a 1 MB di memoria, superando così i limiti dei processori a 16 bit che normalmente non potevano accedere a più di 64 KB. L'architettura sfrutta un'innovativa pipeline BIU/EU che consente al processore di migliorare significativamente le prestazioni eseguendo il prefetching delle istruzioni in parallelo alla loro esecuzione. Il registro FLAGS del 8086 contiene 9 importanti flag di stato e controllo che permettono al programmatore di monitorare i risultati delle operazioni e controllare il comportamento del processore. Infine, l'8086 rimane il capostipite dell'architettura x86 che ancora oggi domina il mercato dei processori per personal computer, dimostrando la genialità del suo design.

\section{Esercizi}

\begin{esercizio}[1.1]
Calcolare l'indirizzo fisico per i seguenti indirizzi segmentati:
\begin{enumerate}[label=\alph*)]
    \item \texttt{CS:IP = 2000h:1000h}
    \item \texttt{DS:SI = FFFFh:0010h}
    \item \texttt{SS:SP = 3000h:FFFEh}
\end{enumerate}
\end{esercizio}

\begin{esercizio}[1.2]
Quanti segmenti distinti di 64 KB possono coesistere nello spazio di indirizzamento dell'8086?
\end{esercizio}

\begin{esercizio}[1.3]
Spiegare perché l'8086 ha bisogno di 20 bit per l'indirizzo fisico ma utilizza registri a 16 bit.
\end{esercizio}

\begin{esercizio}[1.4]
Descrivere la differenza tra modalità minima e massima dell'8086.
\end{esercizio}

\begin{esercizio}[1.5]
Quali flag vengono influenzati dall'istruzione \texttt{ADD AX, BX}? Spiegare in quali condizioni ciascun flag viene impostato a 1.
\end{esercizio}

\chapter{Registri e Organizzazione della Memoria}

\section{Introduzione}

I registri sono celle di memoria ultrarapide integrate nel processore. L'8086 dispone di 14 registri a 16 bit, suddivisi in 4 categorie: registri generali, registri puntatore, registri indice e registri segmento. La conoscenza approfondita dei registri è fondamentale per scrivere codice Assembly efficiente.

\section{Registri General Purpose}

L'8086 ha 4 registri general purpose a 16 bit, ciascuno divisibile in due registri a 8 bit.

\subsection{Registro AX (Accumulatore)}

\begin{definizione}
\textbf{AX} è il registro \emph{accumulatore}, utilizzato per operazioni aritmetiche, I/O e moltiplicazioni/divisioni.
\end{definizione}

\begin{itemize}
    \item \textbf{AX} (16 bit): Accumulatore completo
    \item \textbf{AH} (8 bit): Byte alto (High)
    \item \textbf{AL} (8 bit): Byte basso (Low)
\end{itemize}

\begin{center}
\begin{tikzpicture}
    \draw (0,0) rectangle (4,0.6);
    \draw (2,0) -- (2,0.6);
    \node at (1,0.3) {AH (8 bit)};
    \node at (3,0.3) {AL (8 bit)};
    \node[above] at (2,0.6) {AX (16 bit)};

    % Bit numbering
    \node[below, font=\tiny] at (0,0) {15};
    \node[below, font=\tiny] at (2,0) {7};
    \node[below, font=\tiny] at (4,0) {0};
\end{tikzpicture}
\end{center}

\textbf{Usi tipici}:
\begin{itemize}
    \item Operazioni aritmetiche generiche
    \item Istruzioni I/O (\texttt{IN AL, port} / \texttt{OUT port, AL})
    \item Moltiplicazione: risultato in AX (8 bit) o DX:AX (16 bit)
    \item Divisione: dividendo in AX (8 bit) o DX:AX (16 bit)
\end{itemize}

\subsection{Registro BX (Base)}

\begin{itemize}
    \item \textbf{BX}: Registro base per indirizzamento indiretto
    \item \textbf{BH} / \textbf{BL}: Byte alto/basso
\end{itemize}

\textbf{Usi tipici}:
\begin{itemize}
    \item Indirizzamento di array: \texttt{MOV AL, [BX]}
    \item Puntatore a strutture dati
    \item Calcoli generici
\end{itemize}

\subsection{Registro CX (Contatore)}

\begin{itemize}
    \item \textbf{CX}: Registro contatore per loop
    \item \textbf{CH} / \textbf{CL}: Byte alto/basso
\end{itemize}

\textbf{Usi tipici}:
\begin{itemize}
    \item Contatore in istruzioni \texttt{LOOP}, \texttt{LOOPE}, \texttt{LOOPNE}
    \item Contatore in operazioni su stringhe con prefisso \texttt{REP}
    \item Shift/rotate: CL specifica il numero di posizioni
\end{itemize}

\subsection{Registro DX (Dati)}

\begin{itemize}
    \item \textbf{DX}: Registro dati per I/O e operazioni estese
    \item \textbf{DH} / \textbf{DL}: Byte alto/basso
\end{itemize}

\textbf{Usi tipici}:
\begin{itemize}
    \item Indirizzamento porte I/O: \texttt{IN AL, DX} / \texttt{OUT DX, AL}
    \item Moltiplicazione 16 bit: risultato alto in DX, basso in AX
    \item Divisione 16 bit: dividendo in DX:AX
\end{itemize}

\subsection{Tabella riassuntiva}

\begin{table}[h]
\centering
\begin{tabular}{lllp{6cm}}
\toprule
\textbf{16 bit} & \textbf{8 bit H} & \textbf{8 bit L} & \textbf{Uso principale} \\
\midrule
AX & AH & AL & Accumulatore, I/O, moltiplicazioni \\
BX & BH & BL & Base per indirizzamento, puntatori \\
CX & CH & CL & Contatore loop, shift count \\
DX & DH & DL & Dati, porte I/O, divisioni \\
\bottomrule
\end{tabular}
\caption{Registri general purpose}
\end{table}

\section{Registri Puntatore e Indice}

Questi registri sono utilizzati per l'indirizzamento della memoria e non sono divisibili in byte.

\subsection{Stack Pointer (SP)}

\begin{definizione}
\textbf{SP} punta alla cima dello stack nel segmento SS. Lo stack cresce verso indirizzi decrescenti.
\end{definizione}

\begin{itemize}
    \item Utilizzato con \texttt{PUSH} e \texttt{POP}
    \item Automaticamente decrementato da \texttt{PUSH}, incrementato da \texttt{POP}
    \item Indirizzo stack: \texttt{SS:SP}
\end{itemize}

\subsection{Base Pointer (BP)}

\begin{itemize}
    \item \textbf{BP}: Puntatore base per accedere parametri e variabili locali
    \item Tipicamente usato con segmento SS: \texttt{[BP+offset]}
    \item Fondamentale nelle chiamate a procedure
\end{itemize}

\subsection{Source Index (SI)}

\begin{itemize}
    \item \textbf{SI}: Indice sorgente per operazioni su stringhe
    \item Usato con segmento DS: \texttt{DS:SI}
    \item Auto-incremento/decremento con istruzioni \texttt{LODS}, \texttt{MOVS}, \texttt{CMPS}
\end{itemize}

\subsection{Destination Index (DI)}

\begin{itemize}
    \item \textbf{DI}: Indice destinazione per operazioni su stringhe
    \item Usato con segmento ES: \texttt{ES:DI}
    \item Auto-incremento/decremento con istruzioni \texttt{STOS}, \texttt{MOVS}, \texttt{SCAS}
\end{itemize}

\subsection{Instruction Pointer (IP)}

\begin{definizione}
\textbf{IP} contiene l'offset della prossima istruzione da eseguire nel segmento CS. Non è direttamente modificabile, ma cambia con istruzioni di salto e chiamate.
\end{definizione}

\begin{itemize}
    \item Indirizzo istruzione corrente: \texttt{CS:IP}
    \item Modificato da: \texttt{JMP}, \texttt{CALL}, \texttt{RET}, \texttt{INT}, \texttt{IRET}
    \item Automaticamente incrementato dopo ogni istruzione
\end{itemize}

\section{Registri Segmento}

I 4 registri segmento definiscono le basi dei segmenti di memoria.

\subsection{Code Segment (CS)}

\begin{itemize}
    \item \textbf{CS}: Segmento codice, contiene le istruzioni del programma
    \item Indirizzo istruzione: \texttt{CS:IP}
    \item Modificabile solo con \texttt{JMP FAR}, \texttt{CALL FAR}, \texttt{RET FAR}
\end{itemize}

\subsection{Data Segment (DS)}

\begin{itemize}
    \item \textbf{DS}: Segmento dati, contiene variabili globali
    \item Usato implicitamente da: \texttt{[BX]}, \texttt{[SI]}, \texttt{[DI]}
    \item Modificabile con \texttt{MOV DS, AX}
\end{itemize}

\subsection{Stack Segment (SS)}

\begin{itemize}
    \item \textbf{SS}: Segmento stack, area per \texttt{PUSH}/\texttt{POP}
    \item Indirizzo cima stack: \texttt{SS:SP}
    \item Usato implicitamente da \texttt{[BP]}
\end{itemize}

\subsection{Extra Segment (ES)}

\begin{itemize}
    \item \textbf{ES}: Segmento extra per operazioni su stringhe
    \item Usato implicitamente da \texttt{DI} in \texttt{STOS}, \texttt{MOVS}, \texttt{SCAS}
    \item Utile per copiare dati tra segmenti
\end{itemize}

\begin{attenzione}
I registri segmento \textbf{non possono essere modificati direttamente}. Bisogna passare attraverso un registro general purpose:

\begin{lstlisting}
; CORRETTO
MOV AX, 1000h
MOV DS, AX

; ERRORE - non consentito
MOV DS, 1000h
\end{lstlisting}
\end{attenzione}

\section{Organizzazione della Memoria}

\subsection{Modello di memoria segmentata}

Lo spazio di indirizzamento dell'8086 (1 MB) è organizzato in segmenti sovrapposti:

\begin{center}
\begin{tikzpicture}[scale=0.8]
    % Memoria lineare
    \draw[thick] (0,0) rectangle (1,8);
    \node[left] at (0,8) {FFFFFh};
    \node[left] at (0,6) {C0000h};
    \node[left] at (0,4) {80000h};
    \node[left] at (0,2) {40000h};
    \node[left] at (0,0) {00000h};

    % Segmento codice
    \fill[blue!30] (2,5) rectangle (6,6);
    \node at (4,5.5) {Segmento Codice (CS)};

    % Segmento dati
    \fill[green!30] (2,3.5) rectangle (6,4.5);
    \node at (4,4) {Segmento Dati (DS)};

    % Segmento stack
    \fill[red!30] (2,2) rectangle (6,3);
    \node at (4,2.5) {Segmento Stack (SS)};

    % Frecce
    \draw[<-] (1,5.5) -- (2,5.5);
    \draw[<-] (1,4) -- (2,4);
    \draw[<-] (1,2.5) -- (2,2.5);

    \node[right] at (6,6.5) {Max 64 KB};
    \node[right] at (6,1.5) {Sovrapposti};
\end{tikzpicture}
\end{center}

\subsection{Layout tipico di un programma .COM}

\begin{center}
\begin{tikzpicture}[scale=1.2]
    \draw (0,0) rectangle (4,5);

    \fill[blue!20] (0,4.5) rectangle (4,5);
    \node at (2,4.75) {PSP (256 byte)};

    \fill[green!20] (0,3.5) rectangle (4,4.5);
    \node at (2,4) {Codice};

    \fill[yellow!20] (0,2.5) rectangle (4,3.5);
    \node at (2,3) {Dati};

    \fill[red!20] (0,0) rectangle (4,2.5);
    \node at (2,1.25) {Stack};

    \node[left, font=\small] at (0,5) {CS = DS = SS = ES};
    \node[left, font=\small] at (0,4.75) {IP = 100h};
    \node[left, font=\small] at (0,0.5) {SP = FFFEh};
\end{tikzpicture}
\end{center}

\begin{nota}
Nei programmi \textbf{.COM}, tutti i registri segmento puntano allo stesso indirizzo. L'intera immagine del programma (codice, dati, stack) risiede in un unico segmento di 64 KB. Il Program Segment Prefix (PSP) occupa i primi 256 byte.
\end{nota}

\subsection{Layout programma .EXE}

Nei file .EXE i segmenti sono separati:

\begin{itemize}
    \item \textbf{CS}: Segmento codice (istruzioni)
    \item \textbf{DS}: Segmento dati (variabili globali)
    \item \textbf{SS}: Segmento stack (separato da codice e dati)
    \item \textbf{ES}: Tipicamente uguale a DS all'avvio
\end{itemize}

\section{Convenzioni di utilizzo}

\begin{table}[h]
\centering
\small
\begin{tabular}{lp{9cm}}
\toprule
\textbf{Operazione} & \textbf{Registri tipici} \\
\midrule
Moltiplicazione 8 bit & AL × operando $\rightarrow$ AX \\
Moltiplicazione 16 bit & AX × operando $\rightarrow$ DX:AX \\
Divisione 8 bit & AX ÷ operando $\rightarrow$ AL (quoziente), AH (resto) \\
Divisione 16 bit & DX:AX ÷ operando $\rightarrow$ AX (quoziente), DX (resto) \\
Loop counter & CX (decrementato automaticamente) \\
String source & DS:SI \\
String destination & ES:DI \\
Procedure stack frame & SS:BP \\
I/O a 8 bit fisso & AL + numero porta immediato \\
I/O variabile & AL/AX + DX (numero porta) \\
\bottomrule
\end{tabular}
\caption{Convenzioni d'uso dei registri}
\end{table}

\section{Esempi pratici}

\begin{esempio}
Sommare due numeri a 16 bit:

\begin{lstlisting}
; Somma: 1234h + 5678h = 68ACh
MOV AX, 1234h     ; AX = 1234h
MOV BX, 5678h     ; BX = 5678h
ADD AX, BX        ; AX = 68ACh, BX immutato
\end{lstlisting}
\end{esempio}

\begin{esempio}
Accesso a byte alto e basso:

\begin{lstlisting}
MOV AX, 1234h     ; AX = 1234h
MOV BL, AL        ; BL = 34h (byte basso)
MOV BH, AH        ; BH = 12h (byte alto)
; Ora BX = 1234h
\end{lstlisting}
\end{esempio}

\begin{esempio}
Indirizzamento indiretto con BX:

\begin{lstlisting}
.DATA
array DB 10h, 20h, 30h, 40h

.CODE
MOV BX, OFFSET array   ; BX punta all'array
MOV AL, [BX]           ; AL = 10h (primo elemento)
INC BX                 ; BX punta al secondo elemento
MOV AL, [BX]           ; AL = 20h
\end{lstlisting}
\end{esempio}

\section{Riepilogo}

\begin{itemize}
    \item L'8086 ha 14 registri a 16 bit
    \item I 4 registri general purpose (AX, BX, CX, DX) sono divisibili in byte
    \item SP punta allo stack, BP ai parametri, SI/DI alle stringhe
    \item I 4 registri segmento (CS, DS, SS, ES) definiscono le basi dei segmenti
    \item IP contiene l'offset della prossima istruzione
    \item I programmi .COM usano un unico segmento, i .EXE hanno segmenti separati
\end{itemize}

\section{Esercizi}

\begin{esercizio}[2.1]
Dato \texttt{AX = 1234h}, scrivere le istruzioni per:
\begin{enumerate}[label=\alph*)]
    \item Azzerare solo AL mantenendo AH invariato
    \item Azzerare solo AH mantenendo AL invariato
    \item Scambiare AH e AL
\end{enumerate}
\end{esercizio}

\begin{esercizio}[2.2]
Spiegare la differenza tra \texttt{MOV AX, BX} e \texttt{MOV AX, [BX]}.
\end{esercizio}

\begin{esercizio}[2.3]
Perché non è possibile eseguire \texttt{MOV DS, 1000h}? Come si risolve?
\end{esercizio}

\begin{esercizio}[2.4]
Calcolare l'indirizzo fisico di \texttt{SS:SP = 2000h:FFFEh}. Se si esegue \texttt{PUSH AX}, qual è il nuovo valore di SP e l'indirizzo fisico corrispondente?
\end{esercizio}

\begin{esercizio}[2.5]
In un programma .COM, se tutti i segmenti valgono 1000h e IP = 0100h, qual è l'indirizzo fisico della prima istruzione?
\end{esercizio}

\chapter{Modalità di Indirizzamento}

\section{Introduzione}

Le modalità di indirizzamento determinano come il processore calcola l'indirizzo effettivo degli operandi. L'8086 supporta 7 modalità di indirizzamento, ciascuna adatta a scenari specifici. La comprensione delle modalità di indirizzamento è cruciale per accedere efficientemente a variabili, array e strutture dati.

\begin{definizione}
Una \textbf{modalità di indirizzamento} specifica il metodo con cui il processore determina l'indirizzo di memoria (effective address) o il valore di un operando.
\end{definizione}

\section{Modalità di indirizzamento}

\subsection{1. Immediate Addressing (Indirizzamento Immediato)}

L'operando è una costante inclusa nell'istruzione stessa.

\textbf{Sintassi}: \texttt{MOV dest, costante}

\begin{lstlisting}
MOV AX, 1234h     ; AX = 1234h (costante immediata)
MOV BL, 'A'       ; BL = 41h (codice ASCII di 'A')
ADD CX, 10        ; CX = CX + 10
\end{lstlisting}

\begin{nota}
I valori immediati sono codificati direttamente nell'istruzione, occupando 1 o 2 byte aggiuntivi. Sono i più veloci da accedere (nessun accesso a memoria).
\end{nota}

\subsection{2. Register Addressing (Indirizzamento a Registro)}

L'operando risiede in un registro.

\textbf{Sintassi}: \texttt{MOV dest\_reg, src\_reg}

\begin{lstlisting}
MOV AX, BX        ; AX = BX
ADD SI, DI        ; SI = SI + DI
MOV DL, CH        ; DL = CH (byte alto di CX)
\end{lstlisting}

\textbf{Vantaggi}:

L'indirizzamento a registro offre vantaggi significativi dal punto di vista delle performance. Le operazioni tra registri sono \emph{velocissime} poiché non richiedono alcun accesso alla memoria, permettendo al processore di completare l'operazione interamente all'interno del chip. Inoltre, le istruzioni che utilizzano l'indirizzamento a registro sono particolarmente \emph{compatte}, occupando solamente 1-2 byte, il che significa che il programma occupa meno spazio in memoria e viene caricato più rapidamente dalla memoria durante il prefetch.

\subsection{3. Direct Addressing (Indirizzamento Diretto)}

L'indirizzo effettivo è specificato direttamente nell'istruzione.

\textbf{Sintassi}: \texttt{MOV dest, [indirizzo]}

\begin{lstlisting}
.DATA
var1 DW 1234h

.CODE
MOV AX, [0200h]   ; AX = contenuto di DS:0200h
MOV BX, var1      ; BX = contenuto di var1
MOV [1000h], CX   ; Memoria DS:1000h = CX
\end{lstlisting}

\textbf{Indirizzo effettivo}:
\[
\text{EA} = \text{offset specificato nell'istruzione}
\]

\textbf{Indirizzo fisico}:
\[
\text{Fisico} = \text{DS} \times 16 + \text{EA}
\]

\begin{esempio}
Se \texttt{DS = 1000h} e si esegue \texttt{MOV AX, [0200h]}:

\begin{align*}
\text{EA} &= 0200h \\
\text{Fisico} &= 1000h \times 10h + 0200h \\
&= 10000h + 0200h \\
&= 10200h
\end{align*}

Il contenuto della locazione fisica 10200h viene caricato in AX.
\end{esempio}

\subsection{4. Register Indirect Addressing (Indirizzamento Indiretto a Registro)}

L'indirizzo è contenuto in un registro (BX, SI, DI, BP).

\textbf{Sintassi}: \texttt{MOV dest, [registro]}

\textbf{Registri utilizzabili}:

L'indirizzamento indiretto a registro supporta quattro registri specifici, ognuno con il proprio segmento predefinito. Il registro \texttt{[BX]} utilizza il segmento DS come default, rendendolo ideale per l'accesso ai dati generici. Il registro \texttt{[SI]} (Source Index) utilizza anch'esso il segmento DS per default. Similmente, \texttt{[DI]} (Destination Index) fa riferimento al segmento DS. Un'importante eccezione è rappresentata da \texttt{[BP]} (Base Pointer), che utilizza il segmento SS (Stack Segment) come default anzichè DS, poiché è comunemente usato per accedere ai parametri e alle variabili locali che risiedono nello stack piuttosto che nel segmento dati.

\begin{lstlisting}
MOV BX, 1000h
MOV AL, [BX]      ; AL = byte da DS:1000h

MOV SI, 2000h
MOV AX, [SI]      ; AX = word da DS:2000h

MOV BP, 0FFFEh
MOV CX, [BP]      ; CX = word da SS:FFFEh (stack)
\end{lstlisting}

\begin{attenzione}
\texttt{[BP]} usa il segmento \textbf{SS} (stack), non DS! Per usare DS con BP:
\begin{lstlisting}
MOV AX, DS:[BP]   ; Forza segmento DS
\end{lstlisting}
\end{attenzione}

\textbf{Uso tipico}: Accesso ad array e puntatori

\begin{lstlisting}
.DATA
array DB 10, 20, 30, 40, 50

.CODE
MOV BX, OFFSET array  ; BX punta all'array
MOV AL, [BX]          ; AL = 10 (primo elemento)
INC BX
MOV AL, [BX]          ; AL = 20 (secondo elemento)
\end{lstlisting}

\subsection{5. Based Addressing (Indirizzamento Base + Offset)}

Combina un registro base (BX o BP) con un offset costante.

\textbf{Sintassi}: \texttt{[BX+offset]} o \texttt{[BP+offset]}

\begin{lstlisting}
MOV AX, [BX+4]    ; AX = contenuto di DS:(BX+4)
MOV CX, [BP+6]    ; CX = contenuto di SS:(BP+6)
MOV [BX+10], DX   ; Memoria DS:(BX+10) = DX
\end{lstlisting}

\textbf{Indirizzo effettivo}:
\[
\text{EA} = \text{BX} + \text{offset} \quad \text{(segmento DS)}
\]
\[
\text{EA} = \text{BP} + \text{offset} \quad \text{(segmento SS)}
\end{lstlisting}

\textbf{Uso tipico}: Accesso a campi di struct

\begin{lstlisting}
; Struttura: [nome 20 byte][età 1 byte][salario 2 byte]
MOV BX, OFFSET persona1
MOV AL, [BX+20]      ; AL = età (offset 20)
MOV AX, [BX+21]      ; AX = salario (offset 21)
\end{lstlisting}

\subsection{6. Indexed Addressing (Indirizzamento Indicizzato)}

Combina un registro indice (SI o DI) con un offset costante.

\textbf{Sintassi}: \texttt{[SI+offset]} o \texttt{[DI+offset]}

\begin{lstlisting}
MOV AX, [SI+2]    ; AX = contenuto di DS:(SI+2)
MOV BL, [DI+5]    ; BL = byte da DS:(DI+5)
\end{lstlisting}

\textbf{Uso tipico}: Iterazione su array multi-dimensionali

\begin{lstlisting}
.DATA
matrix DW 1, 2, 3, 4, 5, 6   ; Array 2	imes3

.CODE
MOV SI, 0
MOV AX, matrix[SI+0]   ; Elemento (0,0)
MOV BX, matrix[SI+2]   ; Elemento (0,1)
MOV CX, matrix[SI+4]   ; Elemento (0,2)
ADD SI, 6              ; Passa alla seconda riga
MOV AX, matrix[SI+0]   ; Elemento (1,0)
\end{lstlisting}

\subsection{7. Based Indexed Addressing (Indirizzamento Base + Indice)}

Combina un registro base (BX o BP) con un registro indice (SI o DI) e opzionalmente un offset.

\textbf{Sintassi}:

L'indirizzamento base + indice supporta diverse combinazioni di registri per adattarsi a vari scenari. È possibile combinare il registro base BX con il registro indice SI (\texttt{[BX+SI]}), oppure BX con DI (\texttt{[BX+DI]}). Alternativamente, si può utilizzare il registro base BP con SI (\texttt{[BP+SI]}) o BP con DI (\texttt{[BP+DI]}). Per accessi ancora più flessibili, è possibile aggiungere anche un offset costante a queste combinazioni, come in \texttt{[BX+SI+offset]}, permettendo di calcolare indirizzi complessi che combinano tre elementi: un registro base, un registro indice e un valore immediato.

\begin{lstlisting}
MOV AX, [BX+SI]       ; AX = DS:(BX+SI)
MOV CL, [BP+DI]       ; CL = SS:(BP+DI)
MOV DX, [BX+SI+10]    ; DX = DS:(BX+SI+10)
\end{lstlisting}

\textbf{Indirizzo effettivo}:
\[
\text{EA} = \text{BX/BP} + \text{SI/DI} + \text{offset}
\]

\textbf{Uso tipico}: Array bidimensionali, array di struct

\begin{lstlisting}
; Matrice 4	imes3 (word): accesso a elemento [riga][colonna]
; BX = riga * 6 (3 word 	imes 2 byte)
; SI = colonna * 2

.DATA
matrix DW 1,2,3, 4,5,6, 7,8,9, 10,11,12  ; 4 righe 	imes 3 colonne

.CODE
MOV BX, 6        ; Riga 1 (offset 6 byte)
MOV SI, 4        ; Colonna 2 (offset 4 byte)
MOV AX, matrix[BX+SI]   ; AX = 6 (elemento [1][2])
\end{lstlisting}

\section{Tabella riassuntiva}

\begin{table}[h]
\centering
\small
\begin{tabular}{lllp{4cm}}
\toprule
\textbf{Modalità} & \textbf{Sintassi} & \textbf{EA} & \textbf{Esempio} \\
\midrule
Immediate & \texttt{MOV AX, 10} & — & Costante \\
Register & \texttt{MOV AX, BX} & — & Registro \\
Direct & \texttt{[1234h]} & 1234h & Variabili globali \\
Reg. Indirect & \texttt{[BX]} & BX & Array, puntatori \\
Based & \texttt{[BX+4]} & BX+4 & Struct \\
Indexed & \texttt{[SI+4]} & SI+4 & Array offset \\
Based Indexed & \texttt{[BX+SI+4]} & BX+SI+4 & Matrici, array struct \\
\bottomrule
\end{tabular}
\caption{Modalità di indirizzamento 8086}
\end{table}

\section{Segmenti di default e override}

\begin{table}[h]
\centering
\begin{tabular}{lll}
\toprule
\textbf{Registro} & \textbf{Segmento Default} & \textbf{Override} \\
\midrule
BX & DS & \texttt{ES:[BX]} \\
SI & DS & \texttt{SS:[SI]} \\
DI & DS (tranne stringhe) & \texttt{CS:[DI]} \\
BP & SS & \texttt{DS:[BP]} \\
\bottomrule
\end{tabular}
\caption{Segmenti di default}
\end{table}

\begin{esempio}
Override di segmento:

\begin{lstlisting}
MOV AX, DS:[BP+4]    ; Forza DS invece di SS
MOV BX, ES:[BX+SI]   ; Usa ES invece di DS
MOV CL, CS:[DI]      ; Legge dal segmento codice
\end{lstlisting}
\end{esempio}

\section{Esempi pratici completi}

\begin{esempio}
Array di interi a 16 bit:

\begin{lstlisting}
.DATA
numbers DW 100, 200, 300, 400, 500

.CODE
MOV SI, 0                ; Indice iniziale
MOV CX, 5                ; Numero elementi

loop_start:
    MOV AX, numbers[SI]  ; Carica elemento corrente
    ADD AX, 10           ; Incrementa di 10
    MOV numbers[SI], AX  ; Salva risultato
    ADD SI, 2            ; Prossimo elemento (word = 2 byte)
    LOOP loop_start
\end{lstlisting}
\end{esempio}

\begin{esempio}
Struct con campi multipli:

\begin{lstlisting}
; Struct Studente: [matricola 2 byte][voto 1 byte][età 1 byte]
.DATA
studente1 DW 12345
          DB 28, 20

.CODE
MOV BX, OFFSET studente1
MOV AX, [BX+0]      ; AX = matricola (12345)
MOV AL, [BX+2]      ; AL = voto (28)
MOV AH, [BX+3]      ; AH = età (20)
\end{lstlisting}
\end{esempio}

\begin{esempio}
Matrice bidimensionale 3	imes4 (byte):

\begin{lstlisting}
.DATA
; Matrice 3 righe 	imes 4 colonne
matrix DB 1,2,3,4, 5,6,7,8, 9,10,11,12

.CODE
; Accesso a elemento [riga][colonna]
; BX = riga * 4 (numero colonne)
; SI = colonna

MOV BX, 4           ; Riga 1 (offset 4)
MOV SI, 2           ; Colonna 2
MOV AL, matrix[BX+SI]  ; AL = 7 (elemento [1][2])
\end{lstlisting}
\end{esempio}

\section{Performance e ottimizzazione}

\begin{table}[h]
\centering
\begin{tabular}{lcc}
\toprule
\textbf{Modalità} & \textbf{Clock Cycles} & \textbf{Byte Istruzione} \\
\midrule
Register & 2 & 2 \\
Immediate & 4 & 3-4 \\
Direct & 8-10 & 4 \\
Reg. Indirect & 8-10 & 2-3 \\
Based/Indexed & 11-12 & 3-4 \\
Based Indexed & 12-13 & 3-5 \\
\bottomrule
\end{tabular}
\caption{Performance delle modalità di indirizzamento (approssimativa)}
\end{table}

\begin{nota}
Le operazioni tra registri sono le più veloci. Minimizzare gli accessi a memoria migliora significativamente le performance.
\end{nota}

\section{Riepilogo}

L'8086 supporta \emph{7 modalità di indirizzamento diverse}, ognuna ottimizzata per specifici scenari di programmazione. Le modalità \emph{Immediate e Register sono le più veloci} poiché non richiedono accessi alla memoria, permettendo al processore di completare le operazioni in pochi cicli di clock. Quando si lavora con la memoria, è importante ricordare che i registri \texttt{BX}, \texttt{SI} e \texttt{DI} utilizzano il segmento DS come default, mentre \texttt{BP} utilizza il segmento SS, una distinzione critica per evitare errori di indirizzamento. La modalità \emph{Based Indexed è particolarmente potente} in quanto permette l'accesso a array multidimensionali e strutture dati complesse attraverso la combinazione di un registro base, un registro indice e un offset costante. Quando è necessario accedere a dati in segmenti diversi da quelli predefiniti, è possibile \emph{fare un override del segmento} utilizzando i prefissi \texttt{ES:}, \texttt{DS:}, \texttt{CS:} e \texttt{SS:} direttamente nell'istruzione.

\section{Esercizi}

\begin{esercizio}[3.1]
Classificare le seguenti istruzioni secondo la modalità di indirizzamento:
\begin{enumerate}[label=\alph*)]
    \item \texttt{MOV AX, 1234h}
    \item \texttt{MOV BX, [SI]}
    \item \texttt{MOV CL, [BP+4]}
    \item \texttt{MOV DX, [BX+SI+2]}
\end{enumerate}
\end{esercizio}

\begin{esercizio}[3.2]
Dato \texttt{DS = 2000h}, \texttt{BX = 0100h}, \texttt{SI = 0050h}, calcolare l'indirizzo fisico acceduto da:
\begin{lstlisting}
MOV AX, [BX+SI+10]
\end{lstlisting}
\end{esercizio}

\begin{esercizio}[3.3]
Scrivere codice per accedere al terzo elemento (indice 2) di un array di word:
\begin{lstlisting}
array DW 10, 20, 30, 40, 50
\end{lstlisting}
\end{esercizio}

\begin{esercizio}[3.4]
Spiegare perché \texttt{MOV AX, [BP]} usa il segmento SS e non DS. Come forzare l'uso di DS?
\end{esercizio}

\begin{esercizio}[3.5]
Scrivere codice per sommare tutti gli elementi di un array di 5 byte:
\begin{lstlisting}
data DB 10, 20, 30, 40, 50
\end{lstlisting}
Usare indirizzamento indiretto e registro contatore.
\end{esercizio}


% ===========================
% PARTE II: SET DI ISTRUZIONI
% ===========================
\part{Set di Istruzioni 8086}

\chapter{Istruzioni di Trasferimento Dati}

\section{Introduzione}

Le istruzioni di trasferimento dati spostano informazioni tra registri, memoria e porte I/O senza modificarne il contenuto. Sono le istruzioni più utilizzate in Assembly e costituiscono la base di ogni programma.

\section{MOV --- Move Data}

\begin{definizione}
\texttt{MOV dest, src} copia il valore di \texttt{src} in \texttt{dest}. L'operando sorgente rimane invariato.
\end{definizione}

\textbf{Sintassi}:
\begin{lstlisting}
MOV registro, registro
MOV registro, memoria
MOV memoria, registro
MOV registro, immediato
MOV memoria, immediato
\end{lstlisting}

\textbf{Restrizioni}:

L'istruzione MOV ha diverse limitazioni importanti che il programmatore deve rispettare. \emph{Non è consentito trasferire dati direttamente da una locazione di memoria a un'altra} (MOV memoria, memoria): per copiare dati tra due posizioni di memoria è necessario utilizzare un registro intermedio come step intermedio. \emph{Non è consentito caricare un valore immediato direttamente in un registro segmento} (MOV segmento, immediato): per modificare un registro segmento bisogna passare attraverso un registro general purpose come intermediario. Infine, \emph{gli operandi di origine e destinazione devono avere la stessa dimensione}: non è possibile copiare un byte in un registro word o viceversa senza conversione esplicita.

\begin{esempio}
\begin{lstlisting}
MOV AX, 1234h      ; AX = 1234h
MOV BX, AX         ; BX = AX = 1234h
MOV CL, 5          ; CL = 5
MOV [1000h], AL    ; Memoria DS:1000h = AL

; ERRORI:
; MOV AX, BL       ; Dimensioni diverse!
; MOV DS, 1000h    ; No immediato in segmento!
; MOV [SI], [DI]   ; No mem-to-mem!
\end{lstlisting}
\end{esempio}

\section{XCHG --- Exchange}

Scambia il contenuto di due operandi.

\begin{lstlisting}
XCHG AX, BX        ; Scambia AX con BX
XCHG AL, [SI]      ; Scambia AL con byte in DS:SI
XCHG CX, DX        ; Scambia CX con DX
\end{lstlisting}

\begin{nota}
\texttt{XCHG} è più veloce di tre MOV separate per lo scambio. Non può essere usato con valori immediati.
\end{nota}

\section{LEA --- Load Effective Address}

Carica l'indirizzo effettivo (offset) di un operando in un registro.

\begin{lstlisting}
.DATA
buffer DB 100 DUP(0)

.CODE
LEA BX, buffer     ; BX = offset di buffer (equivalente a OFFSET)
LEA SI, [BX+10]    ; SI = BX + 10 (calcolo offset)
\end{lstlisting}

\textbf{Differenza tra LEA e OFFSET}:
\begin{lstlisting}
MOV BX, OFFSET var    ; BX = indirizzo di var (compile-time)
LEA BX, var           ; BX = indirizzo di var (run-time)
LEA SI, [BX+DI+4]     ; OFFSET non può fare calcoli run-time!
\end{lstlisting}

\section{PUSH e POP --- Stack Operations}

\subsection{PUSH}

Inserisce un valore nello stack.

\begin{lstlisting}
PUSH AX            ; Decrementa SP di 2, salva AX
PUSH BX
PUSH DS
PUSH 1234h         ; (solo 80186+)
\end{lstlisting}

\textbf{Operazione}:
\begin{enumerate}
    \item SP = SP - 2
    \item Memoria[SS:SP] = operando
\end{enumerate}

\subsection{POP}

Estrae un valore dallo stack.

\begin{lstlisting}
POP AX             ; Carica AX da stack, incrementa SP di 2
POP DS
POP word ptr [BX]
\end{lstlisting}

\textbf{Operazione}:
\begin{enumerate}
    \item Operando = Memoria[SS:SP]
    \item SP = SP + 2
\end{enumerate}

\begin{attenzione}
Lo stack cresce verso indirizzi decrescenti! \texttt{PUSH} decrementa SP, \texttt{POP} incrementa SP.
\end{attenzione}

\begin{esempio}
Salvare e ripristinare registri:
\begin{lstlisting}
; Salva contesto
PUSH AX
PUSH BX
PUSH CX

; Usa registri per operazioni
MOV AX, 100
ADD BX, AX

; Ripristina contesto (ordine inverso!)
POP CX
POP BX
POP AX
\end{lstlisting}
\end{esempio}

\section{PUSHF e POPF --- Flag Operations}

\begin{lstlisting}
PUSHF              ; Salva registro FLAGS nello stack
POPF               ; Ripristina FLAGS dallo stack
\end{lstlisting}

Utile per salvare lo stato dei flag prima di operazioni che li modificano.

\section{Istruzioni di I/O}

\subsection{IN --- Input from Port}

\begin{lstlisting}
IN AL, porta       ; Legge byte da porta (0-255)
IN AX, porta       ; Legge word da porta
IN AL, DX          ; Legge da porta in DX (0-65535)
IN AX, DX
\end{lstlisting}

\subsection{OUT --- Output to Port}

\begin{lstlisting}
OUT porta, AL      ; Scrive byte su porta
OUT porta, AX      ; Scrive word su porta
OUT DX, AL         ; Scrive su porta in DX
OUT DX, AX
\end{lstlisting}

\begin{esempio}
Lettura e scrittura porta I/O:
\begin{lstlisting}
; Legge da porta 60h (tastiera)
IN AL, 60h         ; AL = scan code

; Scrive su porta 3F8h (COM1)
MOV DX, 3F8h
MOV AL, 'A'
OUT DX, AL         ; Invia 'A' alla seriale
\end{lstlisting}
\end{esempio}

\section{XLAT --- Translate}

Traduce un byte usando una tabella di lookup.

\begin{lstlisting}
XLAT               ; AL = DS:[BX+AL]
\end{lstlisting}

\begin{esempio}
Conversione da esadecimale a ASCII:
\begin{lstlisting}
.DATA
hex_table DB '0123456789ABCDEF'

.CODE
MOV BX, OFFSET hex_table
MOV AL, 0Ah        ; Valore 10
XLAT               ; AL = 'A' (hex_table[10])
\end{lstlisting}
\end{esempio}

\section{Riepilogo istruzioni}

\begin{table}[h]
\centering
\small
\begin{tabular}{llp{6cm}}
\toprule
\textbf{Istruzione} & \textbf{Sintassi} & \textbf{Descrizione} \\
\midrule
MOV & \texttt{MOV dest, src} & Copia dati \\
XCHG & \texttt{XCHG op1, op2} & Scambia operandi \\
LEA & \texttt{LEA reg, mem} & Carica indirizzo effettivo \\
PUSH & \texttt{PUSH operando} & Inserisce nello stack \\
POP & \texttt{POP operando} & Estrae dallo stack \\
PUSHF & \texttt{PUSHF} & Salva FLAGS \\
POPF & \texttt{POPF} & Ripristina FLAGS \\
IN & \texttt{IN AL/AX, porta} & Input da porta \\
OUT & \texttt{OUT porta, AL/AX} & Output su porta \\
XLAT & \texttt{XLAT} & Traduce byte con tabella \\
\bottomrule
\end{tabular}
\caption{Istruzioni di trasferimento dati}
\end{table}

\section{Esercizi}

\begin{esercizio}[4.1]
Scrivere codice per scambiare i valori di AX e BX senza usare XCHG.
\end{esercizio}

\begin{esercizio}[4.2]
Dato lo stack inizialmente vuoto e SP = FFFEh, tracciare il valore di SP dopo:
\begin{lstlisting}
PUSH AX
PUSH BX
POP CX
PUSH DX
\end{lstlisting}
\end{esercizio}

\begin{esercizio}[4.3]
Spiegare la differenza tra:
\begin{lstlisting}
MOV BX, OFFSET array
LEA BX, array
LEA SI, [BX+DI+4]
\end{lstlisting}
\end{esercizio}

\begin{esercizio}[4.4]
Scrivere codice per convertire le cifre 0-15 in caratteri esadecimali '0'-'F' usando XLAT.
\end{esercizio}

\begin{esercizio}[4.5]
Perché non è consentito \texttt{MOV DS, 1000h}? Come si risolve?
\end{esercizio}

\chapter{Istruzioni Aritmetiche}

\section{Introduzione}

Le istruzioni aritmetiche eseguono operazioni matematiche su operandi interi. L'8086 supporta addizione, sottrazione, moltiplicazione, divisione e operazioni BCD.

\section{ADD e ADC --- Addition}

\subsection{ADD}

\begin{lstlisting}
ADD dest, src      ; dest = dest + src
ADD AX, BX         ; AX = AX + BX
ADD AL, 5          ; AL = AL + 5
ADD word ptr [BX], 100
\end{lstlisting}

\textbf{Flag modificati}: CF, ZF, SF, OF, PF, AF

\subsection{ADC --- Add with Carry}

\begin{lstlisting}
ADC dest, src      ; dest = dest + src + CF
\end{lstlisting}

Usato per addizioni multi-precisione (numeri > 16 bit).

\begin{esempio}
Somma a 32 bit (DX:AX + CX:BX):
\begin{lstlisting}
; DX:AX = 12345678h, CX:BX = 9ABCDEF0h
ADD AX, BX         ; Somma parte bassa
ADC DX, CX         ; Somma parte alta + carry
; Risultato in DX:AX = ACF13568h
\end{lstlisting}
\end{esempio}

\section{SUB e SBB --- Subtraction}

\subsection{SUB}

\begin{lstlisting}
SUB dest, src      ; dest = dest - src
SUB AX, 10
SUB CX, BX
\end{lstlisting}

\subsection{SBB --- Subtract with Borrow}

\begin{lstlisting}
SBB dest, src      ; dest = dest - src - CF
\end{lstlisting}

\section{INC e DEC}

\begin{lstlisting}
INC operando       ; operando = operando + 1
DEC operando       ; operando = operando - 1
\end{lstlisting}

\begin{nota}
INC e DEC \textbf{non modificano} il Carry Flag! Più veloci di ADD/SUB con 1.
\end{nota}

\section{MUL e IMUL --- Multiplication}

\subsection{MUL --- Unsigned Multiplication}

\begin{table}[h]
\centering
\begin{tabular}{lll}
\toprule
\textbf{Operando} & \textbf{Operazione} & \textbf{Risultato} \\
\midrule
8 bit & AL 	imes operando & AX \\
16 bit & AX 	imes operando & DX:AX \\
\bottomrule
\end{tabular}
\end{table}

\begin{lstlisting}
MOV AL, 10
MOV BL, 20
MUL BL             ; AX = AL 	imes BL = 200 (C8h)

MOV AX, 1000
MOV BX, 50
MUL BX             ; DX:AX = AX 	imes BX = 50000
\end{lstlisting}

\subsection{IMUL --- Signed Multiplication}

Stessa sintassi di MUL, ma interpreta gli operandi come numeri con segno.

\section{DIV e IDIV --- Division}

\subsection{DIV --- Unsigned Division}

\begin{table}[h]
\centering
\begin{tabular}{llll}
\toprule
\textbf{Dividendo} & \textbf{Divisore} & \textbf{Quoziente} & \textbf{Resto} \\
\midrule
AX & 8 bit & AL & AH \\
DX:AX & 16 bit & AX & DX \\
\bottomrule
\end{tabular}
\end{table}

\begin{lstlisting}
MOV AX, 100
MOV BL, 7
DIV BL             ; AL = 14 (quoziente), AH = 2 (resto)

MOV DX, 0
MOV AX, 50000
MOV BX, 100
DIV BX             ; AX = 500, DX = 0
\end{lstlisting}

\begin{attenzione}
Prima di DIV a 16 bit, azzerare DX se il dividendo è solo in AX!
\texttt{Divisione per zero} causa interrupt 0 (errore).
\end{attenzione}

\section{NEG e CMP}

\subsection{NEG --- Negate (Complemento a due)}

\begin{lstlisting}
NEG operando       ; operando = -operando
NEG AL             ; AL = -AL
\end{lstlisting}

\subsection{CMP --- Compare}

\begin{lstlisting}
CMP op1, op2       ; Esegue op1 - op2 e imposta flag (senza salvare)
\end{lstlisting}

Usato con salti condizionali:
\begin{lstlisting}
CMP AX, 10
JE equal           ; Salta se AX == 10
JG greater         ; Salta se AX > 10 (signed)
JA above           ; Salta se AX > 10 (unsigned)
\end{lstlisting}

\section{Esercizi}

\begin{esercizio}[5.1]
Calcolare $123 \times 45$ usando MUL.
\end{esercizio}

\begin{esercizio}[5.2]
Dividere 1000 per 7 e trovare quoziente e resto.
\end{esercizio}

\begin{esercizio}[5.3]
Sommare due numeri a 32 bit: 12345678h + ABCDEF01h.
\end{esercizio}

\begin{esercizio}[5.4]
Spiegare perché INC AX è preferibile ad ADD AX, 1.
\end{esercizio}

\begin{esercizio}[5.5]
Scrivere codice per calcolare il valore assoluto di un numero con segno in AX.
\end{esercizio}

\chapter{Istruzioni Logiche e Bit Manipulation}

\section{Istruzioni Logiche}

\subsection{AND}
\begin{lstlisting}
AND dest, src      ; dest = dest AND src (bit a bit)
\end{lstlisting}

Uso: mascherare bit, testare bit, azzerare bit.

\begin{esempio}
\begin{lstlisting}
MOV AL, 11001101b
AND AL, 00001111b  ; AL = 00001101b (maschera nibble basso)
\end{lstlisting}
\end{esempio}

\subsection{OR}
\begin{lstlisting}
OR dest, src       ; dest = dest OR src
\end{lstlisting}

Uso: impostare bit a 1.

\subsection{XOR}
\begin{lstlisting}
XOR dest, src      ; dest = dest XOR src
\end{lstlisting}

Uso: toggle bit, azzerare registro.

\begin{lstlisting}
XOR AX, AX         ; Azzeramento veloce (AX = 0)
\end{lstlisting}

\subsection{NOT}
\begin{lstlisting}
NOT operando       ; Complemento a uno (inverte tutti i bit)
\end{lstlisting}

\subsection{TEST}
\begin{lstlisting}
TEST op1, op2      ; Esegue AND e imposta flag (senza salvare)
\end{lstlisting}

\section{Shift e Rotate}

\subsection{SHL/SHR --- Shift Logico}
\begin{lstlisting}
SHL dest, count    ; Shift left (moltiplicazione per 2^count)
SHR dest, count    ; Shift right (divisione per 2^count)
\end{lstlisting}

\subsection{SAL/SAR --- Shift Aritmetico}
\begin{lstlisting}
SAL dest, count    ; Come SHL
SAR dest, count    ; Preserva bit di segno
\end{lstlisting}

\subsection{ROL/ROR --- Rotate}
\begin{lstlisting}
ROL dest, count    ; Rotate left (bit escono e rientrano)
ROR dest, count    ; Rotate right
\end{lstlisting}

\subsection{RCL/RCR --- Rotate through Carry}
\begin{lstlisting}
RCL dest, count    ; Rotate left includendo CF
RCR dest, count    ; Rotate right includendo CF
\end{lstlisting}

\begin{esempio}
Moltiplicazione veloce per 8:
\begin{lstlisting}
MOV AX, 5
SHL AX, 3          ; AX = 5 * 8 = 40
\end{lstlisting}
\end{esempio}

\section{Esercizi}

\begin{esercizio}[6.1]
Azzerare i bit 0, 2, 4 di AL usando AND.
\end{esercizio}

\begin{esercizio}[6.2]
Impostare i bit 3, 5, 7 di BL usando OR.
\end{esercizio}

\begin{esercizio}[6.3]
Testare se il bit 6 di AL è 1 usando TEST.
\end{esercizio}

\begin{esercizio}[6.4]
Moltiplicare AX per 16 usando shift.
\end{esercizio}

\begin{esercizio}[6.5]
Spiegare la differenza tra SHR e SAR con numeri negativi.
\end{esercizio}

\chapter{Istruzioni di Controllo di Flusso}

\section{Jump Incondizionati}

\subsection{JMP --- Unconditional Jump}
\begin{lstlisting}
JMP label          ; Salta a label
JMP SHORT label    ; Salto corto (-128/+127 byte)
JMP NEAR label     ; Salto near (stesso segmento)
JMP FAR label      ; Salto far (cambia CS:IP)
\end{lstlisting}

\section{Jump Condizionati}

Basati su flag del registro FLAGS.

\subsection{Jump basati su Zero Flag}
\begin{lstlisting}
JE/JZ label        ; Jump if Equal/Zero (ZF=1)
JNE/JNZ label      ; Jump if Not Equal/Not Zero (ZF=0)
\end{lstlisting}

\subsection{Jump Unsigned}
\begin{lstlisting}
JA/JNBE label      ; Jump if Above (CF=0 AND ZF=0)
JAE/JNB label      ; Jump if Above or Equal (CF=0)
JB/JNAE label      ; Jump if Below (CF=1)
JBE/JNA label      ; Jump if Below or Equal (CF=1 OR ZF=1)
\end{lstlisting}

\subsection{Jump Signed}
\begin{lstlisting}
JG/JNLE label      ; Jump if Greater (ZF=0 AND SF=OF)
JGE/JNL label      ; Jump if Greater or Equal (SF=OF)
JL/JNGE label      ; Jump if Less (SF≠OF)
JLE/JNG label      ; Jump if Less or Equal (ZF=1 OR SF≠OF)
\end{lstlisting}

\subsection{Jump su singoli flag}
\begin{lstlisting}
JC label           ; Jump if Carry (CF=1)
JNC label          ; Jump if No Carry (CF=0)
JS label           ; Jump if Sign (SF=1, negativo)
JNS label          ; Jump if No Sign (SF=0, positivo)
JO label           ; Jump if Overflow (OF=1)
JNO label          ; Jump if No Overflow (OF=0)
JP/JPE label       ; Jump if Parity Even (PF=1)
JNP/JPO label      ; Jump if Parity Odd (PF=0)
\end{lstlisting}

\section{Loop Instructions}

\subsection{LOOP}
\begin{lstlisting}
LOOP label         ; Decrementa CX, salta se CX≠0
\end{lstlisting}

\begin{esempio}
Loop per 10 iterazioni:
\begin{lstlisting}
MOV CX, 10
loop_start:
    ; Corpo del loop
    DEC BX
    ; ...
LOOP loop_start    ; Ripete finché CX≠0
\end{lstlisting}
\end{esempio}

\subsection{LOOPE/LOOPZ}
\begin{lstlisting}
LOOPE label        ; Loop while Equal (decrementa CX, salta se ZF=1 AND CX≠0)
\end{lstlisting}

\subsection{LOOPNE/LOOPNZ}
\begin{lstlisting}
LOOPNE label       ; Loop while Not Equal (decrementa CX, salta se ZF=0 AND CX≠0)
\end{lstlisting}

\subsection{JCXZ}
\begin{lstlisting}
JCXZ label         ; Jump if CX=0 (senza modificare CX)
\end{lstlisting}

\section{Procedure}

\subsection{CALL}
\begin{lstlisting}
CALL procedura     ; Chiama procedura (PUSH IP, JMP)
CALL NEAR proc     ; Chiamata near
CALL FAR proc      ; Chiamata far (PUSH CS, PUSH IP)
\end{lstlisting}

\subsection{RET}
\begin{lstlisting}
RET                ; Ritorna (POP IP)
RET n              ; Ritorna e rimuove n byte dallo stack
RETF               ; Return far (POP IP, POP CS)
\end{lstlisting}

\begin{esempio}
Procedura semplice:
\begin{lstlisting}
somma PROC
    ADD AX, BX     ; AX = AX + BX
    RET
somma ENDP

; Chiamata:
MOV AX, 10
MOV BX, 20
CALL somma         ; AX = 30
\end{lstlisting}
\end{esempio}

\section{Interrupt}

\subsection{INT}
\begin{lstlisting}
INT numero         ; Invoca interrupt software
\end{lstlisting}

\begin{esempio}
DOS interrupt 21h (servizi DOS):
\begin{lstlisting}
MOV AH, 9          ; Funzione 9: stampa stringa
MOV DX, OFFSET msg
INT 21h

MOV AH, 4Ch        ; Funzione 4Ch: termina programma
INT 21h
\end{lstlisting}
\end{esempio}

\subsection{IRET}
\begin{lstlisting}
IRET               ; Return from interrupt (POP IP, POP CS, POP FLAGS)
\end{lstlisting}

\section{Tabella Jump Condizionati}

\begin{table}[h]
\centering
\small
\begin{tabular}{llp{5cm}}
\toprule
\textbf{Istruzione} & \textbf{Condizione} & \textbf{Uso} \\
\midrule
JE/JZ & ZF=1 & Uguale/Zero \\
JNE/JNZ & ZF=0 & Diverso/Non zero \\
JG/JNLE & ZF=0 AND SF=OF & Greater (signed) \\
JGE/JNL & SF=OF & Greater or Equal (signed) \\
JL/JNGE & SF≠OF & Less (signed) \\
JLE/JNG & ZF=1 OR SF≠OF & Less or Equal (signed) \\
JA/JNBE & CF=0 AND ZF=0 & Above (unsigned) \\
JAE/JNB & CF=0 & Above or Equal (unsigned) \\
JB/JNAE & CF=1 & Below (unsigned) \\
JBE/JNA & CF=1 OR ZF=1 & Below or Equal (unsigned) \\
\bottomrule
\end{tabular}
\caption{Jump condizionati principali}
\end{table}

\section{Esercizi}

\begin{esercizio}[7.1]
Scrivere un loop per sommare i numeri da 1 a 100.
\end{esercizio}

\begin{esercizio}[7.2]
Implementare una procedura che calcola il massimo tra AX e BX (risultato in AX).
\end{esercizio}

\begin{esercizio}[7.3]
Spiegare la differenza tra JG e JA. Quando si usa ciascuno?
\end{esercizio}

\begin{esercizio}[7.4]
Scrivere codice per cercare un byte in un array di 10 elementi.
\end{esercizio}

\begin{esercizio}[7.5]
Cosa succede se si chiama una procedura senza RET?
\end{esercizio}


% ===========================
% PARTE III: PROGRAMMAZIONE AVANZATA
% ===========================
\part{Tecniche di Programmazione}

\chapter{Procedure e Gestione dello Stack}

\section{Lo Stack dell'8086}

Lo stack è un'area di memoria LIFO (Last In, First Out) gestita dai registri SS:SP.

\subsection{Operazioni fondamentali}

\begin{lstlisting}
PUSH AX            ; SP = SP - 2, [SS:SP] = AX
POP BX             ; BX = [SS:SP], SP = SP + 2
\end{lstlisting}

\section{Chiamate a Procedura}

\subsection{Convenzioni di chiamata}

Le convenzioni di chiamata definiscono come il codice chiamante e il codice chiamato devono coordinarsi per il passaggio dei parametri e la gestione dei registri. Il \textbf{Caller} (codice chiamante) deve inizialmente \emph{salvare i registri che intende preservare}, poiché la procedura potrebbe modificarli. Successivamente, il \textbf{Caller} \emph{passa i parametri} alla procedura, tipicamente tramite lo stack o direttamente nei registri, in base alla convenzione scelta. L'esecuzione della procedura avviene mediante l'istruzione \texttt{CALL}, che salva automaticamente l'indirizzo di ritorno. Dopo il ritorno della procedura, il \textbf{Caller} \emph{recupera il risultato} (solitamente da AX) e \emph{ripristina i registri salvati} precedentemente.

Dal lato del \textbf{Callee} (procedura chiamata), la sequenza è complementare. La procedura \emph{salva i registri che utilizzerà}, preservandone i valori originali per il chiamante. Quindi la procedura \emph{esegue le sue operazioni}, utilizzando i parametri ricevuti e i registri salvati. Prima del termine, la procedura \emph{ripristina i registri salvati}, ripristinandoli al loro stato originale. Infine, la procedura \emph{esegue RET} per trasferire il controllo al chiamante, permettendogli di continuare dall'istruzione successiva a \texttt{CALL}.

\subsection{Passaggio parametri via stack}

\begin{lstlisting}
; Procedura: somma(a, b) -> risultato in AX
somma PROC
    PUSH BP
    MOV BP, SP         ; Stack frame

    MOV AX, [BP+4]     ; Parametro a
    ADD AX, [BP+6]     ; Parametro b

    POP BP
    RET 4              ; Rimuove 2 parametri (4 byte)
somma ENDP

; Chiamata:
PUSH 20                ; Parametro b
PUSH 10                ; Parametro a
CALL somma             ; AX = 30
\end{lstlisting}

\subsection{Stack frame layout}

\begin{verbatim}
[BP+6]  Parametro b
[BP+4]  Parametro a
[BP+2]  Return address (IP)
[BP+0]  Old BP  <-- BP
[BP-2]  Variabile locale 1
[BP-4]  Variabile locale 2  <-- SP
\end{verbatim}

\section{Variabili Locali}

\begin{lstlisting}
myproc PROC
    PUSH BP
    MOV BP, SP
    SUB SP, 4          ; Alloca 4 byte per variabili locali

    MOV word ptr [BP-2], 100   ; Variabile locale 1
    MOV word ptr [BP-4], 200   ; Variabile locale 2

    MOV SP, BP         ; Dealloca variabili
    POP BP
    RET
myproc ENDP
\end{lstlisting}

\section{Ricorsione}

\begin{esempio}
Fattoriale ricorsivo:
\begin{lstlisting}
; Fattoriale(n) -> AX
factorial PROC
    PUSH BP
    MOV BP, SP

    MOV AX, [BP+4]     ; n
    CMP AX, 1
    JLE base_case

    DEC AX
    PUSH AX
    CALL factorial     ; Chiamata ricorsiva
    ADD SP, 2

    MUL word ptr [BP+4]  ; AX = risultato * n
    JMP done

base_case:
    MOV AX, 1

done:
    POP BP
    RET 2
factorial ENDP
\end{lstlisting}
\end{esempio}

\section{Esercizi}

\begin{esercizio}[8.1]
Scrivere una procedura che calcola $x^n$ (potenza).
\end{esercizio}

\begin{esercizio}[8.2]
Implementare Fibonacci ricorsivo.
\end{esercizio}

\begin{esercizio}[8.3]
Spiegare perché BP è preferibile a SP per accedere a parametri e variabili locali.
\end{esercizio}

\begin{esercizio}[8.4]
Disegnare lo stack frame per una procedura con 3 parametri e 2 variabili locali.
\end{esercizio}

\begin{esercizio}[8.5]
Cosa succede se si dimentica POP BP prima di RET?
\end{esercizio}

\chapter{Operazioni su Stringhe}

\section{Introduzione}

Le istruzioni su stringhe dell'8086 operano su sequenze di byte o word in memoria, usando SI (source) e DI (destination).

\section{Registri per Stringhe}

Le operazioni su stringhe dell'8086 richiedono l'utilizzo di registri specializzati coordinati tra loro. Il registro \textbf{SI} (Source Index) agisce come indice sorgente e opera principalmente nel segmento DS:SI per default, permettendo di leggere i dati dalla memoria. Il registro \textbf{DI} (Destination Index) funge da indice destinazione e utilizza sempre il segmento ES:DI, specificando dove i dati devono essere scritti o confrontati. Il registro \textbf{CX} (Count) svolge un ruolo cruciale nei prefissi REP, fungendo da contatore che specifica quante volte ripetere l'operazione su stringhe prima di terminare. Infine, il \textbf{DF} (Direction Flag) del registro FLAGS controlla la direzione dell'operazione: quando DF è 0 i registri SI e DI vengono incrementati (operazione in avanti), mentre quando DF è 1 vengono decrementati (operazione all'indietro).

\subsection{CLD e STD}

\begin{lstlisting}
CLD                ; Clear Direction Flag (incremento)
STD                ; Set Direction Flag (decremento)
\end{lstlisting}

\section{Istruzioni Base}

\subsection{MOVSB/MOVSW --- Move String}

\begin{lstlisting}
MOVSB              ; byte[ES:DI] = byte[DS:SI], aggiorna SI e DI
MOVSW              ; word[ES:DI] = word[DS:SI], aggiorna SI e DI di 2
\end{lstlisting}

\subsection{LODSB/LODSW --- Load String}

\begin{lstlisting}
LODSB              ; AL = byte[DS:SI], aggiorna SI
LODSW              ; AX = word[DS:SI], aggiorna SI di 2
\end{lstlisting}

\subsection{STOSB/STOSW --- Store String}

\begin{lstlisting}
STOSB              ; byte[ES:DI] = AL, aggiorna DI
STOSW              ; word[ES:DI] = AX, aggiorna DI di 2
\end{lstlisting}

\subsection{CMPSB/CMPSW --- Compare String}

\begin{lstlisting}
CMPSB              ; Confronta byte[DS:SI] con byte[ES:DI], imposta flag
CMPSW              ; Confronta word
\end{lstlisting}

\subsection{SCASB/SCASW --- Scan String}

\begin{lstlisting}
SCASB              ; Confronta AL con byte[ES:DI], aggiorna DI
SCASW              ; Confronta AX con word[ES:DI]
\end{lstlisting}

\section{Prefissi REP}

\subsection{REP}

\begin{lstlisting}
REP MOVSB          ; Ripete MOVSB per CX volte
REP STOSB          ; Ripete STOSB per CX volte
\end{lstlisting}

\subsection{REPE/REPZ}

\begin{lstlisting}
REPE CMPSB         ; Ripete finché uguale e CX≠0
\end{lstlisting}

\subsection{REPNE/REPNZ}

\begin{lstlisting}
REPNE SCASB        ; Ripete finché diverso e CX≠0
\end{lstlisting}

\section{Esempi Pratici}

\begin{esempio}
Copiare 100 byte da src a dest:
\begin{lstlisting}
.DATA
src DB 100 DUP(?)
dest DB 100 DUP(?)

.CODE
CLD
MOV SI, OFFSET src
MOV DI, OFFSET dest
MOV CX, 100
REP MOVSB
\end{lstlisting}
\end{esempio}

\begin{esempio}
Riempire buffer con zero:
\begin{lstlisting}
CLD
MOV DI, OFFSET buffer
MOV CX, 256
MOV AL, 0
REP STOSB          ; Azzera 256 byte
\end{lstlisting}
\end{esempio}

\begin{esempio}
Lunghezza stringa (cerca byte 0):
\begin{lstlisting}
strlen PROC
    PUSH DI
    MOV DI, OFFSET stringa
    MOV AL, 0
    MOV CX, 0FFFFh     ; Max length
    CLD
    REPNE SCASB        ; Cerca byte 0
    MOV AX, 0FFFFh
    SUB AX, CX
    DEC AX             ; AX = lunghezza
    POP DI
    RET
strlen ENDP
\end{lstlisting}
\end{esempio}

\begin{esempio}
Confrontare due stringhe:
\begin{lstlisting}
strcmp PROC
    CLD
    MOV SI, OFFSET str1
    MOV DI, OFFSET str2
    MOV CX, 10         ; Max 10 caratteri
    REPE CMPSB         ; Confronta finché uguali
    JE equal
    ; Diverse
    RET

equal:
    ; Uguali
    RET
strcmp ENDP
\end{lstlisting}
\end{esempio}

\section{Esercizi}

\begin{esercizio}[9.1]
Scrivere codice per invertire una stringa di 20 caratteri.
\end{esercizio}

\begin{esercizio}[9.2]
Implementare una funzione che conta le occorrenze di un carattere in una stringa.
\end{esercizio}

\begin{esercizio}[9.3]
Spiegare la differenza tra MOVSB e LODSB.
\end{esercizio}

\begin{esercizio}[9.4]
Convertire una stringa in maiuscolo usando LODSB e STOSB.
\end{esercizio}

\begin{esercizio}[9.5]
Implementare memcpy(dest, src, n) usando REP MOVSB.
\end{esercizio}

\chapter{Sistema di Interrupt}

\section{Introduzione agli Interrupt}

Gli interrupt permettono al processore di rispondere a eventi esterni o richiedere servizi di sistema.

\section{Tipi di Interrupt}

\subsection{Hardware Interrupts}

Generati da dispositivi esterni (tastiera, timer, disco).

\subsection{Software Interrupts}

Invocati da programma con istruzione \texttt{INT}.

\subsection{Exceptions}

Generati dal processore (divisione per zero, overflow).

\section{Interrupt Vector Table (IVT)}

Tabella a indirizzo 0000:0000 contenente 256 vettori di 4 byte (CS:IP).

\begin{lstlisting}
; Vettore interrupt n a indirizzo n*4
; Offset = n * 4
; Segmento = n * 4 + 2
\end{lstlisting}

\section{Interrupt DOS (INT 21h)}

\subsection{Funzione 01h: Input carattere con echo}

\begin{lstlisting}
MOV AH, 01h
INT 21h            ; AL = carattere letto
\end{lstlisting}

\subsection{Funzione 02h: Output carattere}

\begin{lstlisting}
MOV AH, 02h
MOV DL, 'A'
INT 21h            ; Stampa 'A'
\end{lstlisting}

\subsection{Funzione 09h: Stampa stringa}

\begin{lstlisting}
.DATA
msg DB 'Hello World$'

.CODE
MOV AH, 09h
MOV DX, OFFSET msg
INT 21h
\end{lstlisting}

\subsection{Funzione 4Ch: Termina programma}

\begin{lstlisting}
MOV AH, 4Ch
MOV AL, 0          ; Exit code
INT 21h
\end{lstlisting}

\section{Interrupt BIOS}

\subsection{INT 10h: Video Services}

\begin{lstlisting}
; Funzione 0Eh: Teletype output
MOV AH, 0Eh
MOV AL, 'X'
MOV BH, 0          ; Page number
INT 10h

; Funzione 00h: Set video mode
MOV AH, 00h
MOV AL, 03h        ; Modo testo 80x25
INT 10h
\end{lstlisting}

\subsection{INT 16h: Keyboard Services}

\begin{lstlisting}
; Funzione 00h: Legge tasto
MOV AH, 00h
INT 16h            ; AL = ASCII, AH = scan code
\end{lstlisting}

\subsection{INT 13h: Disk Services}

\begin{lstlisting}
; Funzione 02h: Legge settori
MOV AH, 02h
MOV AL, 1          ; Numero settori
MOV CH, 0          ; Cilindro
MOV CL, 1          ; Settore
MOV DH, 0          ; Testina
MOV DL, 00h        ; Drive (0=A:, 80h=HD)
MOV BX, OFFSET buffer
INT 13h
\end{lstlisting}

\section{Gestione Custom Interrupt}

\begin{lstlisting}
; Installare handler personalizzato per INT 60h
install_handler PROC
    CLI                ; Disabilita interrupt
    PUSH DS

    ; Calcola indirizzo IVT: 60h * 4 = 180h
    XOR AX, AX
    MOV DS, AX
    MOV BX, 180h

    ; Salva vecchio handler
    MOV AX, [BX]
    MOV old_offset, AX
    MOV AX, [BX+2]
    MOV old_segment, AX

    ; Installa nuovo handler
    MOV word ptr [BX], OFFSET my_handler
    MOV [BX+2], CS

    POP DS
    STI                ; Riabilita interrupt
    RET
install_handler ENDP

my_handler PROC
    ; Codice handler
    IRET
my_handler ENDP
\end{lstlisting}

\section{Esercizi}

\begin{esercizio}[10.1]
Scrivere un programma che legge un carattere e lo stampa 10 volte.
\end{esercizio}

\begin{esercizio}[10.2]
Implementare una funzione per stampare un numero decimale usando INT 21h.
\end{esercizio}

\begin{esercizio}[10.3]
Spiegare la differenza tra INT, CALL e JMP.
\end{esercizio}

\begin{esercizio}[10.4]
Scrivere codice per cambiare il colore del testo usando INT 10h.
\end{esercizio}

\begin{esercizio}[10.5]
Implementare un semplice handler per INT 60h che incrementa un contatore.
\end{esercizio}

\chapter{I/O e Interfacciamento Hardware}

\section{Introduzione}

L'8086 comunica con periferiche tramite porte I/O a 16 bit (65536 porte possibili).

\section{Istruzioni I/O}

\subsection{IN e OUT}

\begin{lstlisting}
; I/O fisso (porta 0-255)
IN AL, porta
OUT porta, AL

; I/O variabile (porta in DX)
MOV DX, porta
IN AL, DX
OUT DX, AL
\end{lstlisting}

\section{Programmable Interrupt Controller (8259)

}

\subsection{Porte del PIC}

\begin{itemize}
    \item \textbf{20h}: Registro comando (Master PIC)
    \item \textbf{21h}: Registro maschera interrupt (IMR)
    \item \textbf{A0h}: Slave PIC comando
    \item \textbf{A1h}: Slave PIC maschera
\end{itemize}

\begin{lstlisting}
; Disabilitare interrupt tastiera (IRQ1)
IN AL, 21h         ; Legge IMR
OR AL, 02h         ; Setta bit 1
OUT 21h, AL        ; Scrive IMR

; Riabilitare
IN AL, 21h
AND AL, 0FDh       ; Azzera bit 1
OUT 21h, AL
\end{lstlisting}

\section{Programmable Interval Timer (8253/8254)}

\subsection{Porte del PIT}

\begin{itemize}
    \item \textbf{40h}: Canale 0 (system timer)
    \item \textbf{41h}: Canale 1
    \item \textbf{42h}: Canale 2 (speaker)
    \item \textbf{43h}: Control word
\end{itemize}

\begin{esempio}
Generare beep con speaker:
\begin{lstlisting}
; Frequenza = 1193180 / divisore
; Per 1000 Hz: divisore = 1193

MOV AL, 0B6h       ; Control word
OUT 43h, AL

MOV AX, 1193
OUT 42h, AL        ; Byte basso
MOV AL, AH
OUT 42h, AL        ; Byte alto

; Abilita speaker
IN AL, 61h
OR AL, 03h
OUT 61h, AL

; Attendi...

; Disabilita speaker
IN AL, 61h
AND AL, 0FCh
OUT 61h, AL
\end{lstlisting}
\end{esempio}

\section{Porta Parallela (LPT)}

\subsection{Porte LPT1}

\begin{itemize}
    \item \textbf{378h}: Data port (8 bit output)
    \item \textbf{379h}: Status port (5 bit input)
    \item \textbf{37Ah}: Control port
\end{itemize}

\begin{lstlisting}
; Invia byte alla stampante
MOV DX, 378h
MOV AL, 'A'
OUT DX, AL

; Strobe
MOV DX, 37Ah
MOV AL, 0Dh
OUT DX, AL
MOV AL, 0Ch
OUT DX, AL
\end{lstlisting}

\section{Porta Seriale (COM)}

\subsection{UART 8250/16550}

Porte COM1 (base 3F8h):
\begin{itemize}
    \item \textbf{3F8h}: Data register (RBR/THR)
    \item \textbf{3F9h}: Interrupt Enable Register (IER)
    \item \textbf{3FAh}: Interrupt ID Register (IIR)
    \item \textbf{3FBh}: Line Control Register (LCR)
    \item \textbf{3FCh}: Modem Control Register (MCR)
    \item \textbf{3FDh}: Line Status Register (LSR)
    \item \textbf{3FEh}: Modem Status Register (MSR)
\end{itemize}

\begin{esempio}
Inizializzare COM1 a 9600 baud, 8N1:
\begin{lstlisting}
; Divisore per 9600 baud: 115200 / 9600 = 12

MOV DX, 3FBh       ; LCR
MOV AL, 80h        ; DLAB = 1
OUT DX, AL

MOV DX, 3F8h       ; DLL
MOV AL, 12
OUT DX, AL

MOV DX, 3F9h       ; DLM
MOV AL, 0
OUT DX, AL

MOV DX, 3FBh       ; LCR
MOV AL, 03h        ; 8N1, DLAB=0
OUT DX, AL
\end{lstlisting}
\end{esempio}

\begin{esempio}
Trasmettere byte su COM1:
\begin{lstlisting}
send_byte PROC
    ; AL = byte da inviare
    PUSH DX
    PUSH AX

wait_ready:
    MOV DX, 3FDh   ; LSR
    IN AL, DX
    TEST AL, 20h   ; THRE bit
    JZ wait_ready

    POP AX
    MOV DX, 3F8h   ; THR
    OUT DX, AL

    POP DX
    RET
send_byte ENDP
\end{lstlisting}
\end{esempio}

\section{Tastiera (8042)}

\subsection{Porte tastiera}

\begin{itemize}
    \item \textbf{60h}: Data port
    \item \textbf{64h}: Status/Command port
\end{itemize}

\begin{lstlisting}
; Leggi scan code
IN AL, 60h

; Verifica buffer pieno
IN AL, 64h
TEST AL, 01h       ; Output buffer full
JZ no_key
IN AL, 60h         ; Leggi dato

no_key:
\end{lstlisting}

\section{VGA (Video Graphics Array)}

\subsection{Accesso diretto a memoria video}

Modo testo 80x25: buffer a B800:0000

\begin{lstlisting}
; Scrivere 'A' rosso su sfondo nero in posizione (0,0)
MOV AX, 0B800h
MOV ES, AX
MOV DI, 0
MOV AH, 04h        ; Attributo: rosso
MOV AL, 'A'
MOV ES:[DI], AX
\end{lstlisting}

\section{Esercizi}

\begin{esercizio}[11.1]
Scrivere codice per generare un beep di 2 secondi a 440 Hz (nota LA).
\end{esercizio}

\begin{esercizio}[11.2]
Implementare una funzione per leggere un carattere da COM1.
\end{esercizio}

\begin{esercizio}[11.3]
Scrivere una stringa direttamente in memoria video VGA.
\end{esercizio}

\begin{esercizio}[11.4]
Spiegare la differenza tra I/O mapped e memory mapped.
\end{esercizio}

\begin{esercizio}[11.5]
Implementare un delay preciso usando il PIT.
\end{esercizio}


% ===========================
% PARTE IV: APPLICAZIONI
% ===========================
\part{Progetti e Applicazioni}

\chapter{Progetti Applicativi}

\section{Progetto 1: Calcolatrice a 16 bit}

Implementare una calcolatrice che supporta +, -, *, / su numeri interi.

\begin{lstlisting}
.MODEL SMALL
.STACK 100h

.DATA
msg1 DB 'Inserisci primo numero: $'
msg2 DB 'Inserisci operatore (+,-,*,/): $'
msg3 DB 'Inserisci secondo numero: $'
result_msg DB 'Risultato: $'
num1 DW ?
num2 DW ?
operator DB ?

.CODE
main PROC
    MOV AX, @DATA
    MOV DS, AX

    ; Input primo numero
    MOV AH, 09h
    MOV DX, OFFSET msg1
    INT 21h
    CALL read_number
    MOV num1, AX

    ; Input operatore
    MOV AH, 09h
    MOV DX, OFFSET msg2
    INT 21h
    MOV AH, 01h
    INT 21h
    MOV operator, AL

    ; Input secondo numero
    MOV AH, 09h
    MOV DX, OFFSET msg3
    INT 21h
    CALL read_number
    MOV num2, AX

    ; Esegui operazione
    MOV AL, operator
    CMP AL, '+'
    JE do_add
    CMP AL, '-'
    JE do_sub
    CMP AL, '*'
    JE do_mul
    CMP AL, '/'
    JE do_div

do_add:
    MOV AX, num1
    ADD AX, num2
    JMP display

do_sub:
    MOV AX, num1
    SUB AX, num2
    JMP display

do_mul:
    MOV AX, num1
    MUL num2
    JMP display

do_div:
    MOV AX, num1
    XOR DX, DX
    DIV num2
    JMP display

display:
    CALL print_number

    MOV AH, 4Ch
    INT 21h
main ENDP

; Procedura per leggere numero decimale
read_number PROC
    ; ... implementazione ...
    RET
read_number ENDP

; Procedura per stampare numero
print_number PROC
    ; ... implementazione ...
    RET
print_number ENDP

END main
\end{lstlisting}

\section{Progetto 2: Ordinamento Array}

Bubble sort su array di 10 numeri.

\begin{lstlisting}
.DATA
array DW 64, 34, 25, 12, 22, 11, 90, 88, 45, 50
size DW 10

.CODE
bubble_sort PROC
    MOV CX, size
    DEC CX             ; n-1 passate

outer_loop:
    PUSH CX
    MOV SI, 0
    MOV CX, size
    DEC CX

inner_loop:
    MOV AX, array[SI]
    CMP AX, array[SI+2]
    JLE no_swap

    ; Scambia
    XCHG AX, array[SI+2]
    MOV array[SI], AX

no_swap:
    ADD SI, 2
    LOOP inner_loop

    POP CX
    LOOP outer_loop

    RET
bubble_sort ENDP
\end{lstlisting}

\section{Progetto 3: Editor di Testo Semplice}

Editor di testo minimale con buffer di 256 caratteri.

\section{Progetto 4: Gioco Snake}

Snake in modalità testo 80x25.

\section{Progetto 5: Bootloader}

Bootloader minimale che stampa un messaggio.

\begin{lstlisting}
[ORG 0x7C00]
[BITS 16]

start:
    MOV AX, 0x07C0
    MOV DS, AX
    MOV ES, AX

    MOV AH, 0x0E       ; Teletype
    MOV SI, msg

print_loop:
    LODSB
    OR AL, AL
    JZ done
    INT 0x10
    JMP print_loop

done:
    CLI
    HLT

msg DB 'Bootloader avviato!', 13, 10, 0

TIMES 510-($-$$) DB 0
DW 0xAA55              ; Boot signature
\end{lstlisting}

\section{Esercizi}

\begin{esercizio}[12.1]
Completare il progetto calcolatrice implementando read\_number e print\_number.
\end{esercizio}

\begin{esercizio}[12.2]
Modificare bubble sort per ordinamento decrescente.
\end{esercizio}

\begin{esercizio}[12.3]
Implementare Quick Sort ricorsivo.
\end{esercizio}

\begin{esercizio}[12.4]
Scrivere un bootloader che legge un settore dal disco.
\end{esercizio}

\begin{esercizio}[12.5]
Creare un semplice shell DOS-like con comandi DIR, TYPE, EXIT.
\end{esercizio}

\chapter{Esercizi Progressivi}

\section{Livello Base}

\begin{esercizio}[B.1 --- Hello World]
Scrivere un programma che stampa "Hello, Assembly!" usando INT 21h funzione 09h.
\end{esercizio}

\begin{esercizio}[B.2 --- Somma Due Numeri]
Leggere due numeri a 16 bit, sommarli e stampare il risultato.
\end{esercizio}

\begin{esercizio}[B.3 --- Pari o Dispari]
Leggere un numero e determinare se è pari o dispari (testare bit 0).
\end{esercizio}

\begin{esercizio}[B.4 --- Massimo di Tre]
Dati tre numeri, trovare il massimo.
\end{esercizio}

\begin{esercizio}[B.5 --- Somma Array]
Sommare tutti gli elementi di un array di 10 word.
\end{esercizio}

\begin{esercizio}[B.6 --- Copia Stringa]
Copiare una stringa usando MOVSB.
\end{esercizio}

\section{Livello Intermedio}

\begin{esercizio}[I.1 --- Fattoriale]
Calcolare il fattoriale di n (versione iterativa e ricorsiva).
\end{esercizio}

\begin{esercizio}[I.2 --- Fibonacci]
Generare i primi n numeri di Fibonacci.
\end{esercizio}

\begin{esercizio}[I.3 --- Numero Primo]
Verificare se un numero è primo.
\end{esercizio}

\begin{esercizio}[I.4 --- Ricerca Binaria]
Implementare binary search su array ordinato.
\end{esercizio}

\begin{esercizio}[I.5 --- Conversione Decimale-Esadecimale]
Convertire numero decimale in esadecimale e stamparlo.
\end{esercizio}

\begin{esercizio}[I.6 --- Palindromo]
Verificare se una stringa è palindroma.
\end{esercizio}

\begin{esercizio}[I.7 --- Ordinamento Insertion Sort]
Implementare insertion sort su array.
\end{esercizio}

\section{Livello Avanzato}

\begin{esercizio}[A.1 --- Torre di Hanoi]
Risolvere Torre di Hanoi con ricorsione.
\end{esercizio}

\begin{esercizio}[A.2 --- Moltiplicazione Matrici]
Moltiplicare due matrici 3	imes3.
\end{esercizio}

\begin{esercizio}[A.3 --- Parser Espressioni]
Valutare espressioni aritmetiche con precedenza operatori.
\end{esercizio}

\begin{esercizio}[A.4 --- Allocatore Memoria]
Implementare malloc/free semplificato.
\end{esercizio}

\begin{esercizio}[A.5 --- Compressione RLE]
Implementare compressione Run-Length Encoding.
\end{esercizio}

\begin{esercizio}[A.6 --- File System FAT12]
Leggere e parsare directory FAT12 da floppy disk.
\end{esercizio}

\begin{esercizio}[A.7 --- Debugger Minimale]
Implementare debugger con breakpoint e single-step.
\end{esercizio}

\section{Progetti Integrati}

\begin{esercizio}[P.1 --- Sistema Operativo Minimale]
Creare un micro-OS con:
\begin{itemize}
    \item Bootloader
    \item Gestore interrupt
    \item Shell minimale
    \item Driver VGA e tastiera
\end{itemize}
\end{esercizio}

\begin{esercizio}[P.2 --- Emulatore 8086]
Scrivere emulatore software dell'8086 in C/Python.
\end{esercizio}

\begin{esercizio}[P.3 --- Crittografia]
Implementare cifratura/decifratura Caesar, XOR, ROT13.
\end{esercizio}

\begin{esercizio}[P.4 --- Gioco Tetris]
Implementare Tetris in modalità testo.
\end{esercizio}

\begin{esercizio}[P.5 --- Comunicazione Seriale]
Implementare protocollo seriale per comunicazione PC-PC.
\end{esercizio}

\section{Sfide Avanzate}

\begin{esercizio}[S.1 --- Code Golf]
Scrivere il programma più corto possibile per:
\begin{enumerate}
    \item Stampare "Hello World"
    \item Ordinare array
    \item Calcolare Fibonacci
\end{enumerate}
\end{esercizio}

\begin{esercizio}[S.2 --- Ottimizzazione]
Ottimizzare i seguenti task per velocità massima:
\begin{enumerate}
    \item Copiare 64KB di memoria
    \item Cercare byte in array 1MB
    \item Calcolare checksum CRC16
\end{enumerate}
\end{esercizio}

\begin{esercizio}[S.3 --- Reverse Engineering]
Analizzare e decompilare un binario .COM fornito.
\end{esercizio}

\begin{esercizio}[S.4 --- Real Mode OS]
Scrivere un OS completo in real mode con:
\begin{itemize}
    \item Multitasking cooperativo
    \item Driver disco e file system
    \item Interfaccia grafica VGA
\end{itemize}
\end{esercizio}

\section{Riepilogo Competenze}

Al completamento di tutti gli esercizi, lo studente avrà acquisito:

\begin{itemize}
    \item Padronanza completa del set di istruzioni 8086
    \item Capacità di debugging e ottimizzazione
    \item Comprensione architettura low-level
    \item Esperienza con hardware e periferiche
    \item Basi per reverse engineering e sicurezza
\end{itemize}


% ===========================
% APPENDICI
% ===========================
\appendix
\appendix
\chapter{Soluzioni degli Esercizi}

\section*{Nota}
In questa appendice sono riportate le soluzioni complete e commentate degli esercizi dei capitoli 1-4. Ogni soluzione include codice completo funzionante con commenti italiani e una breve spiegazione del funzionamento.

\section{Capitolo 1: Stream e Buffer}

\subsection{Esercizio 1.1: Contatore di righe}

\begin{lstlisting}
import java.io.BufferedReader;
import java.io.FileReader;
import java.io.IOException;

public class ContaRighe {
    public static void main(String[] args) {
        int numeroRighe = 0;

        // Try-with-resources per chiusura automatica
        try (BufferedReader br = new BufferedReader(
                new FileReader("file.txt"))) {

            // Legge ogni riga fino alla fine del file
            while (br.readLine() != null) {
                numeroRighe++;
            }

            System.out.println("Il file contiene " + numeroRighe + " righe");

        } catch (IOException e) {
            System.out.println("Errore lettura file: " + e.getMessage());
        }
    }
}
\end{lstlisting}

\textbf{Spiegazione:} Il programma utilizza BufferedReader per leggere il file riga per riga con readLine(), incrementando un contatore per ogni riga letta. Quando readLine() restituisce null, significa che si è raggiunta la fine del file.

\subsection{Esercizio 1.2: Copia carattere per carattere}

\begin{lstlisting}
import java.io.FileReader;
import java.io.FileWriter;
import java.io.IOException;

public class CopiaCarattere {
    public static void main(String[] args) {
        // Usa try-with-resources per gestire due risorse
        try (FileReader input = new FileReader("originale.txt");
             FileWriter output = new FileWriter("copia.txt")) {

            int carattere;

            // read() restituisce -1 quando raggiunge la fine
            while ((carattere = input.read()) != -1) {
                // Scrive il carattere letto nel file di output
                output.write(carattere);
            }

            System.out.println("Copia completata con successo!");

        } catch (IOException e) {
            System.out.println("Errore durante la copia: " + e.getMessage());
        }
    }
}
\end{lstlisting}

\textbf{Spiegazione:} Il programma legge un carattere alla volta con read() e lo scrive immediatamente nel file di destinazione. Try-with-resources garantisce la chiusura automatica di entrambi gli stream anche in caso di eccezione.

\subsection{Esercizio 1.3: Salva nomi in file}

\begin{lstlisting}
import java.io.BufferedWriter;
import java.io.FileWriter;
import java.io.IOException;
import java.util.Scanner;

public class SalvaNomi {
    public static void main(String[] args) {
        Scanner scanner = new Scanner(System.in);

        try (BufferedWriter bw = new BufferedWriter(
                new FileWriter("nomi.txt"))) {

            System.out.println("Inserisci 5 nomi:");

            // Legge 5 nomi dall'utente
            for (int i = 1; i <= 5; i++) {
                System.out.print("Nome " + i + ": ");
                String nome = scanner.nextLine();

                // Scrive il nome nel file
                bw.write(nome);
                bw.newLine(); // Aggiunge separatore di riga
            }

            System.out.println("Nomi salvati in nomi.txt");

        } catch (IOException e) {
            System.out.println("Errore scrittura file: " + e.getMessage());
        } finally {
            scanner.close();
        }
    }
}
\end{lstlisting}

\textbf{Spiegazione:} Il programma legge 5 nomi da console usando Scanner e li scrive su file usando BufferedWriter, aggiungendo newLine() dopo ogni nome per scriverli su righe separate.

\subsection{Esercizio 1.4: Inverti righe}

\begin{lstlisting}
import java.io.BufferedReader;
import java.io.BufferedWriter;
import java.io.FileReader;
import java.io.FileWriter;
import java.io.IOException;
import java.util.ArrayList;

public class InvertiRighe {
    public static void main(String[] args) {
        ArrayList<String> righe = new ArrayList<>();

        // Legge tutte le righe e le memorizza nell'ArrayList
        try (BufferedReader br = new BufferedReader(
                new FileReader("input.txt"))) {

            String riga;
            while ((riga = br.readLine()) != null) {
                righe.add(riga);
            }

        } catch (IOException e) {
            System.out.println("Errore lettura: " + e.getMessage());
            return;
        }

        // Scrive le righe in ordine inverso
        try (BufferedWriter bw = new BufferedWriter(
                new FileWriter("output.txt"))) {

            // Itera dall'ultima riga alla prima
            for (int i = righe.size() - 1; i >= 0; i--) {
                bw.write(righe.get(i));
                bw.newLine();
            }

            System.out.println("File invertito creato con successo!");

        } catch (IOException e) {
            System.out.println("Errore scrittura: " + e.getMessage());
        }
    }
}
\end{lstlisting}

\textbf{Spiegazione:} Prima legge tutte le righe in un ArrayList, poi le scrive in ordine inverso iterando dall'ultimo indice al primo. Questa tecnica permette di invertire facilmente l'ordine delle righe.

\subsection{Esercizio 1.5: Filtra righe maiuscole}

\begin{lstlisting}
import java.io.BufferedReader;
import java.io.BufferedWriter;
import java.io.FileReader;
import java.io.FileWriter;
import java.io.IOException;

public class FiltraMaiuscole {
    public static void main(String[] args) {
        try (BufferedReader input = new BufferedReader(
                new FileReader("input.txt"));
             BufferedWriter output = new BufferedWriter(
                new FileWriter("output.txt"))) {

            String riga;
            int righeCopiate = 0;

            while ((riga = input.readLine()) != null) {
                // Verifica se la riga non e' vuota e inizia con maiuscola
                if (!riga.isEmpty() &&
                    Character.isUpperCase(riga.charAt(0))) {

                    output.write(riga);
                    output.newLine();
                    righeCopiate++;
                }
            }

            System.out.println("Copiate " + righeCopiate +
                             " righe che iniziano con maiuscola");

        } catch (IOException e) {
            System.out.println("Errore: " + e.getMessage());
        }
    }
}
\end{lstlisting}

\textbf{Spiegazione:} Il programma legge ogni riga e verifica se non è vuota e se il primo carattere è una lettera maiuscola usando Character.isUpperCase(). Solo le righe che soddisfano questa condizione vengono scritte nel file di output.

\subsection{Esercizio 1.6: Merge di due file}

\begin{lstlisting}
import java.io.BufferedReader;
import java.io.BufferedWriter;
import java.io.FileReader;
import java.io.FileWriter;
import java.io.IOException;

public class MergeFile {
    public static void main(String[] args) {
        try (BufferedReader br1 = new BufferedReader(
                new FileReader("file1.txt"));
             BufferedReader br2 = new BufferedReader(
                new FileReader("file2.txt"));
             BufferedWriter output = new BufferedWriter(
                new FileWriter("merge.txt"))) {

            String riga1, riga2;

            // Legge alternando le righe dai due file
            while (true) {
                riga1 = br1.readLine();
                riga2 = br2.readLine();

                // Se entrambi i file sono finiti, esci dal ciclo
                if (riga1 == null && riga2 == null) {
                    break;
                }

                // Scrivi riga dal primo file se disponibile
                if (riga1 != null) {
                    output.write(riga1);
                    output.newLine();
                }

                // Scrivi riga dal secondo file se disponibile
                if (riga2 != null) {
                    output.write(riga2);
                    output.newLine();
                }
            }

            System.out.println("Merge completato!");

        } catch (IOException e) {
            System.out.println("Errore: " + e.getMessage());
        }
    }
}
\end{lstlisting}

\textbf{Spiegazione:} Il programma alterna la lettura di una riga dal primo file e una dal secondo, scrivendole nel file di output. Continua finché entrambi i file non hanno più righe da leggere.

\subsection{Esercizio 1.7: Analisi parole}

\begin{lstlisting}
import java.io.BufferedReader;
import java.io.FileReader;
import java.io.IOException;
import java.util.HashMap;
import java.util.Map;

public class AnalisiParole {
    public static void main(String[] args) {
        int totaleParole = 0;
        int totaleCaratteri = 0;
        // Mappa per contare la frequenza di ogni parola
        HashMap<String, Integer> frequenze = new HashMap<>();

        try (BufferedReader br = new BufferedReader(
                new FileReader("testo.txt"))) {

            String riga;

            while ((riga = br.readLine()) != null) {
                // Divide la riga in parole (split per spazi)
                String[] parole = riga.split("\\s+");

                for (String parola : parole) {
                    // Rimuove punteggiatura e converte in minuscolo
                    parola = parola.replaceAll("[^a-zA-Z]", "")
                                   .toLowerCase();

                    if (!parola.isEmpty()) {
                        totaleParole++;
                        totaleCaratteri += parola.length();

                        // Aggiorna frequenza della parola
                        frequenze.put(parola,
                                    frequenze.getOrDefault(parola, 0) + 1);
                    }
                }
            }

            // Stampa report
            System.out.println("=== REPORT ANALISI ===");
            System.out.println("Totale parole: " + totaleParole);
            System.out.println("Totale caratteri: " + totaleCaratteri);
            System.out.println("\nFrequenza parole:");

            for (Map.Entry<String, Integer> entry : frequenze.entrySet()) {
                System.out.println(entry.getKey() + ": " +
                                 entry.getValue() + " volte");
            }

        } catch (IOException e) {
            System.out.println("Errore: " + e.getMessage());
        }
    }
}
\end{lstlisting}

\textbf{Spiegazione:} Il programma divide ogni riga in parole, le pulisce dalla punteggiatura e conta i caratteri totali. Usa una HashMap per tenere traccia della frequenza di ogni parola, incrementando il contatore quando la parola viene incontrata.

\subsection{Esercizio 1.8: Parser CSV}

\begin{lstlisting}
import java.io.BufferedReader;
import java.io.FileReader;
import java.io.IOException;
import java.util.ArrayList;

public class ParserCSV {
    public static void main(String[] args) {
        ArrayList<String[]> dati = new ArrayList<>();
        int maxLunghezze[] = new int[3]; // Per 3 colonne

        // Legge file CSV
        try (BufferedReader br = new BufferedReader(
                new FileReader("dati.csv"))) {

            String riga;

            while ((riga = br.readLine()) != null) {
                // Split per virgola
                String[] campi = riga.split(",");
                dati.add(campi);

                // Calcola lunghezza massima per ogni colonna
                for (int i = 0; i < campi.length && i < 3; i++) {
                    if (campi[i].length() > maxLunghezze[i]) {
                        maxLunghezze[i] = campi[i].length();
                    }
                }
            }

        } catch (IOException e) {
            System.out.println("Errore lettura: " + e.getMessage());
            return;
        }

        // Stampa tabella allineata
        System.out.println("=== DATI CSV ===");

        for (String[] riga : dati) {
            for (int i = 0; i < riga.length && i < 3; i++) {
                // Stampa campo allineato a sinistra
                System.out.printf("%-" + (maxLunghezze[i] + 2) + "s",
                                riga[i]);
            }
            System.out.println();
        }
    }
}
\end{lstlisting}

\textbf{Spiegazione:} Il parser legge il file CSV, divide ogni riga usando split(",") e calcola la lunghezza massima di ogni colonna. Poi stampa i dati in formato tabellare allineato usando printf con formattazione dinamica.

\section{Capitolo 2: Interfacce e Classi Astratte}

\subsection{Esercizio 2.1: Interfaccia Volante}

\begin{lstlisting}
// Interfaccia Volante
public interface Volante {
    void vola();
}

// Classe Aereo
public class Aereo implements Volante {
    private String modello;
    private int passeggeri;

    public Aereo(String modello, int passeggeri) {
        this.modello = modello;
        this.passeggeri = passeggeri;
    }

    @Override
    public void vola() {
        System.out.println("L'aereo " + modello +
                         " vola trasportando " + passeggeri +
                         " passeggeri");
    }
}

// Classe Uccello
public class Uccello implements Volante {
    private String specie;

    public Uccello(String specie) {
        this.specie = specie;
    }

    @Override
    public void vola() {
        System.out.println("L'uccello " + specie +
                         " vola sbattendo le ali");
    }
}

// Test
public class TestVolante {
    public static void main(String[] args) {
        Volante[] volanti = {
            new Aereo("Boeing 747", 400),
            new Uccello("Aquila"),
            new Aereo("Cessna", 4),
            new Uccello("Gabbiano")
        };

        for (Volante v : volanti) {
            v.vola();
        }
    }
}
\end{lstlisting}

\textbf{Spiegazione:} L'interfaccia Volante definisce il contratto vola(), implementato diversamente da Aereo (con modello e passeggeri) e Uccello (con specie). Il polimorfismo permette di trattare entrambi come oggetti Volante.

\subsection{Esercizio 2.2: Classe astratta Animale}

\begin{lstlisting}
// Classe astratta Animale
public abstract class Animale {
    protected String nome;

    public Animale(String nome) {
        this.nome = nome;
    }

    // Metodo astratto - ogni animale ha un verso diverso
    public abstract void verso();

    // Metodo concreto - tutti gli animali dormono allo stesso modo
    public void dorme() {
        System.out.println(nome + " sta dormendo... Zzz");
    }
}

// Classe Cane
public class Cane extends Animale {
    public Cane(String nome) {
        super(nome);
    }

    @Override
    public void verso() {
        System.out.println(nome + " fa: Bau bau!");
    }
}

// Classe Gatto
public class Gatto extends Animale {
    public Gatto(String nome) {
        super(nome);
    }

    @Override
    public void verso() {
        System.out.println(nome + " fa: Miao miao!");
    }
}

// Test
public class TestAnimali {
    public static void main(String[] args) {
        Animale[] animali = {
            new Cane("Fido"),
            new Gatto("Whiskers"),
            new Cane("Rex")
        };

        for (Animale a : animali) {
            a.verso();
            a.dorme();
            System.out.println();
        }
    }
}
\end{lstlisting}

\textbf{Spiegazione:} La classe astratta Animale fornisce il metodo concreto dorme() condiviso da tutti gli animali, mentre verso() è astratto e deve essere implementato da ogni sottoclasse con il proprio verso specifico.

\subsection{Esercizio 2.3: Interfaccia Pagabile}

\begin{lstlisting}
// Interfaccia Pagabile
public interface Pagabile {
    double calcolaStipendio();
}

// Classe Dipendente
public class Dipendente implements Pagabile {
    private String nome;
    private double stipendioMensile;

    public Dipendente(String nome, double stipendioMensile) {
        this.nome = nome;
        this.stipendioMensile = stipendioMensile;
    }

    @Override
    public double calcolaStipendio() {
        // Stipendio fisso mensile
        return stipendioMensile;
    }

    public String getNome() {
        return nome;
    }
}

// Classe Freelance
public class Freelance implements Pagabile {
    private String nome;
    private double tariffaOraria;
    private int oreLavorate;

    public Freelance(String nome, double tariffaOraria, int oreLavorate) {
        this.nome = nome;
        this.tariffaOraria = tariffaOraria;
        this.oreLavorate = oreLavorate;
    }

    @Override
    public double calcolaStipendio() {
        // Stipendio basato su ore lavorate
        return tariffaOraria * oreLavorate;
    }

    public String getNome() {
        return nome;
    }
}

// Test
public class TestPagabile {
    public static void main(String[] args) {
        Pagabile[] lavoratori = {
            new Dipendente("Mario Rossi", 2000),
            new Freelance("Laura Bianchi", 50, 80),
            new Dipendente("Giuseppe Verdi", 2500),
            new Freelance("Anna Neri", 60, 120)
        };

        double totalePagamenti = 0;

        for (Pagabile p : lavoratori) {
            double stipendio = p.calcolaStipendio();
            totalePagamenti += stipendio;

            String nome = "";
            if (p instanceof Dipendente) {
                nome = ((Dipendente)p).getNome();
            } else if (p instanceof Freelance) {
                nome = ((Freelance)p).getNome();
            }

            System.out.printf("%s: %.2f euro\n", nome, stipendio);
        }

        System.out.printf("\nTotale pagamenti: %.2f euro\n",
                        totalePagamenti);
    }
}
\end{lstlisting}

\textbf{Spiegazione:} L'interfaccia Pagabile definisce il metodo calcolaStipendio(). Dipendente restituisce lo stipendio mensile fisso, mentre Freelance calcola il pagamento moltiplicando tariffa oraria per ore lavorate.

\subsection{Esercizio 2.4: Interfaccia DispositivoElettronico}

\begin{lstlisting}
// Interfaccia DispositivoElettronico
public interface DispositivoElettronico {
    void accendi();
    void spegni();
    double consumoEnergetico(); // in Watt
}

// Classe Televisore
public class Televisore implements DispositivoElettronico {
    private boolean acceso = false;
    private int pollici;

    public Televisore(int pollici) {
        this.pollici = pollici;
    }

    @Override
    public void accendi() {
        acceso = true;
        System.out.println("TV " + pollici + "\" accesa");
    }

    @Override
    public void spegni() {
        acceso = false;
        System.out.println("TV " + pollici + "\" spenta");
    }

    @Override
    public double consumoEnergetico() {
        // Consumo proporzionale ai pollici
        return acceso ? pollici * 2.5 : 0.5; // standby
    }
}

// Classe Computer
public class Computer implements DispositivoElettronico {
    private boolean acceso = false;
    private boolean desktop;

    public Computer(boolean desktop) {
        this.desktop = desktop;
    }

    @Override
    public void accendi() {
        acceso = true;
        System.out.println((desktop ? "Desktop" : "Laptop") + " acceso");
    }

    @Override
    public void spegni() {
        acceso = false;
        System.out.println((desktop ? "Desktop" : "Laptop") + " spento");
    }

    @Override
    public double consumoEnergetico() {
        if (!acceso) return 0;
        return desktop ? 150 : 65;
    }
}

// Classe Lampada
public class Lampada implements DispositivoElettronico {
    private boolean accesa = false;
    private int watt;

    public Lampada(int watt) {
        this.watt = watt;
    }

    @Override
    public void accendi() {
        accesa = true;
        System.out.println("Lampada " + watt + "W accesa");
    }

    @Override
    public void spegni() {
        accesa = false;
        System.out.println("Lampada " + watt + "W spenta");
    }

    @Override
    public double consumoEnergetico() {
        return accesa ? watt : 0;
    }
}

// Test
public class TestDispositivi {
    public static double calcolaConsumoTotale(
            DispositivoElettronico[] dispositivi) {
        double totale = 0;
        for (DispositivoElettronico d : dispositivi) {
            totale += d.consumoEnergetico();
        }
        return totale;
    }

    public static void main(String[] args) {
        DispositivoElettronico[] dispositivi = {
            new Televisore(55),
            new Computer(true),
            new Lampada(60),
            new Computer(false)
        };

        // Accende tutti i dispositivi
        for (DispositivoElettronico d : dispositivi) {
            d.accendi();
        }

        System.out.println();
        double consumo = calcolaConsumoTotale(dispositivi);
        System.out.printf("Consumo totale: %.2f Watt\n", consumo);
    }
}
\end{lstlisting}

\textbf{Spiegazione:} Ogni dispositivo implementa i metodi accendi(), spegni() e consumoEnergetico() con logiche diverse. Il metodo calcolaConsumoTotale() somma i consumi di tutti i dispositivi in un array, dimostrando il polimorfismo.

\subsection{Esercizio 2.5: Classe astratta FiguraPiana}

\begin{lstlisting}
// Classe astratta FiguraPiana
public abstract class FiguraPiana {
    protected String colore;

    public FiguraPiana(String colore) {
        this.colore = colore;
    }

    // Metodi astratti
    public abstract double area();
    public abstract double perimetro();

    // Metodo concreto che usa i metodi astratti
    public String confrontaArea(FiguraPiana altra) {
        double miaArea = this.area();
        double altraArea = altra.area();

        if (miaArea > altraArea) {
            return "Questa figura ha area maggiore";
        } else if (miaArea < altraArea) {
            return "L'altra figura ha area maggiore";
        } else {
            return "Le due figure hanno la stessa area";
        }
    }
}

// Classe Quadrato
public class Quadrato extends FiguraPiana {
    private double lato;

    public Quadrato(String colore, double lato) {
        super(colore);
        this.lato = lato;
    }

    @Override
    public double area() {
        return lato * lato;
    }

    @Override
    public double perimetro() {
        return 4 * lato;
    }
}

// Classe Cerchio
public class Cerchio extends FiguraPiana {
    private double raggio;

    public Cerchio(String colore, double raggio) {
        super(colore);
        this.raggio = raggio;
    }

    @Override
    public double area() {
        return Math.PI * raggio * raggio;
    }

    @Override
    public double perimetro() {
        return 2 * Math.PI * raggio;
    }
}

// Classe Triangolo
public class Triangolo extends FiguraPiana {
    private double base;
    private double altezza;
    private double lato1, lato2, lato3;

    public Triangolo(String colore, double base, double altezza,
                    double lato1, double lato2, double lato3) {
        super(colore);
        this.base = base;
        this.altezza = altezza;
        this.lato1 = lato1;
        this.lato2 = lato2;
        this.lato3 = lato3;
    }

    @Override
    public double area() {
        return (base * altezza) / 2;
    }

    @Override
    public double perimetro() {
        return lato1 + lato2 + lato3;
    }
}

// Test
public class TestFigure {
    public static void main(String[] args) {
        FiguraPiana[] figure = {
            new Quadrato("Rosso", 5),
            new Cerchio("Blu", 3),
            new Triangolo("Verde", 4, 3, 3, 4, 5)
        };

        System.out.println("=== ANALISI FIGURE ===");
        for (FiguraPiana f : figure) {
            System.out.printf("Area: %.2f - Perimetro: %.2f\n",
                            f.area(), f.perimetro());
        }

        System.out.println("\n=== CONFRONTI ===");
        System.out.println(figure[0].confrontaArea(figure[1]));
        System.out.println(figure[1].confrontaArea(figure[2]));
    }
}
\end{lstlisting}

\textbf{Spiegazione:} La classe astratta fornisce il metodo concreto confrontaArea() che usa i metodi astratti area(). Ogni figura implementa area() e perimetro() con le proprie formule geometriche.

\subsection{Esercizio 2.6: Libro Comparable}

\begin{lstlisting}
import java.util.Arrays;

// Classe Libro che implementa Comparable
public class Libro implements Comparable<Libro> {
    private String titolo;
    private String autore;
    private int anno;

    public Libro(String titolo, String autore, int anno) {
        this.titolo = titolo;
        this.autore = autore;
        this.anno = anno;
    }

    @Override
    public int compareTo(Libro altro) {
        // Confronta per titolo (ordine alfabetico)
        return this.titolo.compareTo(altro.titolo);
    }

    @Override
    public String toString() {
        return String.format("\"%s\" di %s (%d)", titolo, autore, anno);
    }

    // Getter
    public String getTitolo() { return titolo; }
    public String getAutore() { return autore; }
    public int getAnno() { return anno; }
}

// Test
public class TestLibri {
    public static void main(String[] args) {
        Libro[] biblioteca = {
            new Libro("Zeno", "Italo Svevo", 1923),
            new Libro("Il nome della rosa", "Umberto Eco", 1980),
            new Libro("Annabelle Lee", "Edgar Allan Poe", 1849),
            new Libro("Divina Commedia", "Dante Alighieri", 1321),
            new Libro("Promessi Sposi", "Alessandro Manzoni", 1827)
        };

        System.out.println("=== PRIMA DELL'ORDINAMENTO ===");
        for (Libro libro : biblioteca) {
            System.out.println(libro);
        }

        // Ordina usando compareTo (per titolo)
        Arrays.sort(biblioteca);

        System.out.println("\n=== DOPO ORDINAMENTO (per titolo) ===");
        for (Libro libro : biblioteca) {
            System.out.println(libro);
        }
    }
}
\end{lstlisting}

\textbf{Spiegazione:} La classe Libro implementa Comparable<Libro> definendo il metodo compareTo() che confronta i titoli in ordine alfabetico. Arrays.sort() usa automaticamente questo metodo per ordinare l'array.

\subsection{Esercizio 2.7: Sistema Account Bancari}

\begin{lstlisting}
// Classe astratta Account
public abstract class Account {
    protected String numeroAccount;
    protected String intestatario;
    protected double saldo;

    public Account(String numero, String intestatario, double saldoIniziale) {
        this.numeroAccount = numero;
        this.intestatario = intestatario;
        this.saldo = saldoIniziale;
    }

    // Metodi concreti comuni
    public void deposita(double importo) {
        if (importo > 0) {
            saldo += importo;
            System.out.printf("Depositati %.2f euro\n", importo);
        }
    }

    public boolean preleva(double importo) {
        if (importo > 0 && importo <= saldo) {
            saldo -= importo;
            System.out.printf("Prelevati %.2f euro\n", importo);
            return true;
        }
        System.out.println("Prelievo non riuscito");
        return false;
    }

    public double getSaldo() {
        return saldo;
    }

    // Metodi astratti
    public abstract double calcolaInteressi();
    public abstract void applicaCommissioni();
}

// Account Corrente
public class AccountCorrente extends Account {
    private static final double COMMISSIONE_MENSILE = 5.0;
    private static final double TASSO_INTERESSE = 0.001; // 0.1%

    public AccountCorrente(String numero, String intestatario,
                          double saldoIniziale) {
        super(numero, intestatario, saldoIniziale);
    }

    @Override
    public double calcolaInteressi() {
        // Interessi molto bassi
        return saldo * TASSO_INTERESSE;
    }

    @Override
    public void applicaCommissioni() {
        saldo -= COMMISSIONE_MENSILE;
        System.out.printf("Commissione applicata: %.2f euro\n",
                        COMMISSIONE_MENSILE);
    }
}

// Account Risparmio
public class AccountRisparmio extends Account {
    private static final double TASSO_INTERESSE = 0.02; // 2%

    public AccountRisparmio(String numero, String intestatario,
                           double saldoIniziale) {
        super(numero, intestatario, saldoIniziale);
    }

    @Override
    public double calcolaInteressi() {
        // Interessi piu' alti
        return saldo * TASSO_INTERESSE;
    }

    @Override
    public void applicaCommissioni() {
        // Nessuna commissione per risparmio
        System.out.println("Nessuna commissione per account risparmio");
    }

    @Override
    public boolean preleva(double importo) {
        // Limite prelievi per risparmio
        if (importo > 1000) {
            System.out.println("Limite prelievo: max 1000 euro");
            return false;
        }
        return super.preleva(importo);
    }
}

// Account Deposito
public class AccountDeposito extends Account {
    private static final double TASSO_INTERESSE = 0.035; // 3.5%
    private int mesiVincolo;

    public AccountDeposito(String numero, String intestatario,
                          double saldoIniziale, int mesiVincolo) {
        super(numero, intestatario, saldoIniziale);
        this.mesiVincolo = mesiVincolo;
    }

    @Override
    public double calcolaInteressi() {
        // Interessi piu' alti se vincolato
        return saldo * TASSO_INTERESSE * (mesiVincolo / 12.0);
    }

    @Override
    public void applicaCommissioni() {
        // Nessuna commissione
        System.out.println("Nessuna commissione per deposito");
    }

    @Override
    public boolean preleva(double importo) {
        System.out.println("Impossibile prelevare da deposito vincolato");
        return false;
    }
}

// Test
public class TestBanca {
    public static void main(String[] args) {
        Account[] conti = {
            new AccountCorrente("CC001", "Mario Rossi", 1000),
            new AccountRisparmio("RS001", "Laura Bianchi", 5000),
            new AccountDeposito("DP001", "Giuseppe Verdi", 10000, 12)
        };

        System.out.println("=== OPERAZIONI MENSILI ===\n");

        for (Account a : conti) {
            System.out.println("Account: " + a.intestatario);
            System.out.printf("Saldo iniziale: %.2f euro\n", a.getSaldo());

            // Calcola e accredita interessi
            double interessi = a.calcolaInteressi();
            a.deposita(interessi);
            System.out.printf("Interessi maturati: %.2f euro\n", interessi);

            // Applica commissioni
            a.applicaCommissioni();

            System.out.printf("Saldo finale: %.2f euro\n\n", a.getSaldo());
        }
    }
}
\end{lstlisting}

\textbf{Spiegazione:} La classe astratta Account fornisce metodi comuni (deposita, preleva) e metodi astratti (calcolaInteressi, applicaCommissioni). Ogni tipo di account implementa questi metodi con regole specifiche: corrente ha commissioni, risparmio ha limiti di prelievo, deposito non permette prelievi.

\subsection{Esercizio 2.8: Interfaccia Ordinabile}

\begin{lstlisting}
// Interfaccia Ordinabile
public interface Ordinabile {
    int confronta(Ordinabile altro);
    String getChiave();
}

// Classe Prodotto
public class Prodotto implements Ordinabile {
    private String codice;
    private String nome;
    private double prezzo;

    public Prodotto(String codice, String nome, double prezzo) {
        this.codice = codice;
        this.nome = nome;
        this.prezzo = prezzo;
    }

    @Override
    public int confronta(Ordinabile altro) {
        // Ordina per codice
        return this.codice.compareTo(((Prodotto)altro).codice);
    }

    @Override
    public String getChiave() {
        return codice;
    }

    @Override
    public String toString() {
        return String.format("%s - %s (%.2f euro)", codice, nome, prezzo);
    }
}

// Classe Cliente
public class Cliente implements Ordinabile {
    private String id;
    private String nome;
    private String cognome;

    public Cliente(String id, String nome, String cognome) {
        this.id = id;
        this.nome = nome;
        this.cognome = cognome;
    }

    @Override
    public int confronta(Ordinabile altro) {
        // Ordina per cognome, poi nome
        Cliente altroCliente = (Cliente)altro;
        int confrontoCognome = this.cognome.compareTo(altroCliente.cognome);
        if (confrontoCognome != 0) {
            return confrontoCognome;
        }
        return this.nome.compareTo(altroCliente.nome);
    }

    @Override
    public String getChiave() {
        return id;
    }

    @Override
    public String toString() {
        return String.format("%s: %s %s", id, nome, cognome);
    }
}

// Classe Ordine
public class Ordine implements Ordinabile {
    private int numeroOrdine;
    private String dataOrdine;
    private double totale;

    public Ordine(int numero, String data, double totale) {
        this.numeroOrdine = numero;
        this.dataOrdine = data;
        this.totale = totale;
    }

    @Override
    public int confronta(Ordinabile altro) {
        // Ordina per numero ordine
        Ordine altroOrdine = (Ordine)altro;
        return Integer.compare(this.numeroOrdine, altroOrdine.numeroOrdine);
    }

    @Override
    public String getChiave() {
        return String.valueOf(numeroOrdine);
    }

    @Override
    public String toString() {
        return String.format("Ordine %d del %s - Totale: %.2f euro",
                           numeroOrdine, dataOrdine, totale);
    }
}

// Metodo generico di ordinamento
public class Ordinatore {
    // Bubble sort generico per Ordinabile
    public static void ordina(Ordinabile[] array) {
        int n = array.length;
        for (int i = 0; i < n - 1; i++) {
            for (int j = 0; j < n - i - 1; j++) {
                if (array[j].confronta(array[j + 1]) > 0) {
                    // Scambia elementi
                    Ordinabile temp = array[j];
                    array[j] = array[j + 1];
                    array[j + 1] = temp;
                }
            }
        }
    }
}

// Test
public class TestOrdinabile {
    public static void main(String[] args) {
        // Test con Prodotti
        Ordinabile[] prodotti = {
            new Prodotto("P003", "Mouse", 25.99),
            new Prodotto("P001", "Tastiera", 45.50),
            new Prodotto("P002", "Monitor", 199.99)
        };

        System.out.println("=== PRODOTTI NON ORDINATI ===");
        stampa(prodotti);

        Ordinatore.ordina(prodotti);

        System.out.println("\n=== PRODOTTI ORDINATI ===");
        stampa(prodotti);

        // Test con Clienti
        Ordinabile[] clienti = {
            new Cliente("C003", "Mario", "Rossi"),
            new Cliente("C001", "Laura", "Bianchi"),
            new Cliente("C002", "Anna", "Bianchi")
        };

        System.out.println("\n=== CLIENTI NON ORDINATI ===");
        stampa(clienti);

        Ordinatore.ordina(clienti);

        System.out.println("\n=== CLIENTI ORDINATI ===");
        stampa(clienti);
    }

    private static void stampa(Ordinabile[] array) {
        for (Ordinabile o : array) {
            System.out.println(o);
        }
    }
}
\end{lstlisting}

\textbf{Spiegazione:} L'interfaccia Ordinabile permette di scrivere un algoritmo di ordinamento generico che funziona con qualsiasi tipo che implementa confronta(). Ogni classe definisce la propria logica di confronto: Prodotto per codice, Cliente per cognome/nome, Ordine per numero.

\section{Capitolo 3: Eccezioni}

\subsection{Esercizio 3.1: Input numerico con gestione eccezioni}

\begin{lstlisting}
import java.util.Scanner;
import java.util.InputMismatchException;

public class InputNumerico {
    public static void main(String[] args) {
        Scanner scanner = new Scanner(System.in);
        int numero = 0;
        boolean inputValido = false;

        // Continua a chiedere finche' l'input non e' valido
        while (!inputValido) {
            try {
                System.out.print("Inserisci un numero intero: ");
                numero = scanner.nextInt();
                inputValido = true; // Input valido, esci dal ciclo

            } catch (InputMismatchException e) {
                System.out.println("ERRORE: Devi inserire un numero intero!");
                scanner.nextLine(); // Pulisce l'input errato
            }
        }

        System.out.println("Hai inserito: " + numero);
        scanner.close();
    }
}
\end{lstlisting}

\textbf{Spiegazione:} Il programma usa un ciclo while per continuare a chiedere l'input finché non viene inserito un numero valido. InputMismatchException viene catturata quando l'utente inserisce testo invece di un numero.

\subsection{Esercizio 3.2: Divisione sicura}

\begin{lstlisting}
public class DivisioneSicura {

    public static double dividi(double dividendo, double divisore) {
        try {
            // Lancia eccezione se divisore e' zero
            if (divisore == 0) {
                throw new ArithmeticException("Divisione per zero");
            }
            return dividendo / divisore;

        } catch (ArithmeticException e) {
            System.out.println("Attenzione: " + e.getMessage() +
                             " - Restituisco 0");
            return 0;
        }
    }

    public static void main(String[] args) {
        // Test vari casi
        System.out.println("10 / 2 = " + dividi(10, 2));
        System.out.println("15 / 3 = " + dividi(15, 3));
        System.out.println("7 / 0 = " + dividi(7, 0));  // Gestisce eccezione
        System.out.println("20 / 4 = " + dividi(20, 4));
    }
}
\end{lstlisting}

\textbf{Spiegazione:} Il metodo dividi() controlla se il divisore è zero e in tal caso lancia un'eccezione ArithmeticException. Il catch intercetta l'eccezione e restituisce 0 come valore di default.

\subsection{Esercizio 3.3: Accesso array sicuro}

\begin{lstlisting}
import java.util.Scanner;

public class AccessoArraySicuro {
    public static void main(String[] args) {
        String[] frutti = {"Mela", "Banana", "Arancia", "Pera", "Kiwi"};
        Scanner scanner = new Scanner(System.in);

        System.out.println("Array disponibile con " + frutti.length +
                         " elementi (indici 0-" + (frutti.length - 1) + ")");

        System.out.print("Inserisci indice da visualizzare: ");
        int indice = scanner.nextInt();

        try {
            // Tenta di accedere all'elemento
            String frutto = frutti[indice];
            System.out.println("Elemento all'indice " + indice +
                             ": " + frutto);

        } catch (ArrayIndexOutOfBoundsException e) {
            System.out.println("ERRORE: Indice " + indice +
                             " non valido!");
            System.out.println("Gli indici validi sono da 0 a " +
                             (frutti.length - 1));
        } finally {
            scanner.close();
        }
    }
}
\end{lstlisting}

\textbf{Spiegazione:} Il programma cattura ArrayIndexOutOfBoundsException quando l'utente inserisce un indice fuori dai limiti dell'array, mostrando un messaggio di errore chiaro con gli indici validi.

\subsection{Esercizio 3.4: Eccezione personalizzata NumeroNegativo}

\begin{lstlisting}
// Eccezione personalizzata checked
public class NumeroNegativoException extends Exception {
    private double numeroErrato;

    public NumeroNegativoException(double numero) {
        super("Impossibile calcolare radice quadrata di numero negativo: "
              + numero);
        this.numeroErrato = numero;
    }

    public double getNumeroErrato() {
        return numeroErrato;
    }
}

// Classe con metodo che usa l'eccezione
public class CalcolatoreRadici {

    public static double calcolaRadiceQuadrata(double n)
            throws NumeroNegativoException {

        if (n < 0) {
            throw new NumeroNegativoException(n);
        }

        return Math.sqrt(n);
    }

    public static void main(String[] args) {
        double[] numeri = {16, 25, -9, 49, -4, 64};

        System.out.println("=== CALCOLO RADICI QUADRATE ===");

        for (double num : numeri) {
            try {
                double radice = calcolaRadiceQuadrata(num);
                System.out.printf("Radice di %.1f = %.2f\n", num, radice);

            } catch (NumeroNegativoException e) {
                System.out.println("ERRORE: " + e.getMessage());
                System.out.println("Numero problematico: " +
                                 e.getNumeroErrato());
            }
        }
    }
}
\end{lstlisting}

\textbf{Spiegazione:} Viene creata un'eccezione checked personalizzata che memorizza il numero negativo che ha causato l'errore. Il metodo calcolaRadiceQuadrata() dichiara throws e lancia l'eccezione se il numero è negativo.

\subsection{Esercizio 3.5: Classe Calcolatrice con eccezioni}

\begin{lstlisting}
// Eccezioni personalizzate
class DivisoreZeroException extends Exception {
    public DivisoreZeroException() {
        super("Errore: divisione per zero non permessa");
    }
}

class OperazioneNonValidaException extends Exception {
    public OperazioneNonValidaException(String messaggio) {
        super(messaggio);
    }
}

// Classe Calcolatrice
public class Calcolatrice {

    public double somma(double a, double b) {
        return a + b;
    }

    public double sottrai(double a, double b) {
        return a - b;
    }

    public double moltiplica(double a, double b)
            throws OperazioneNonValidaException {
        // Controlla overflow per numeri molto grandi
        if (Math.abs(a) > 1e100 || Math.abs(b) > 1e100) {
            throw new OperazioneNonValidaException(
                "Numeri troppo grandi per la moltiplicazione");
        }
        return a * b;
    }

    public double dividi(double dividendo, double divisore)
            throws DivisoreZeroException {
        if (divisore == 0) {
            throw new DivisoreZeroException();
        }
        return dividendo / divisore;
    }

    public static void main(String[] args) {
        Calcolatrice calc = new Calcolatrice();

        System.out.println("=== TEST CALCOLATRICE ===\n");

        // Test somma
        System.out.println("5 + 3 = " + calc.somma(5, 3));

        // Test sottrazione
        System.out.println("10 - 4 = " + calc.sottrai(10, 4));

        // Test moltiplicazione
        try {
            System.out.println("6 * 7 = " + calc.moltiplica(6, 7));
            System.out.println("1e101 * 2 = " + calc.moltiplica(1e101, 2));
        } catch (OperazioneNonValidaException e) {
            System.out.println("ERRORE: " + e.getMessage());
        }

        // Test divisione
        try {
            System.out.println("20 / 5 = " + calc.dividi(20, 5));
            System.out.println("15 / 0 = " + calc.dividi(15, 0));
        } catch (DivisoreZeroException e) {
            System.out.println("ERRORE: " + e.getMessage());
        }
    }
}
\end{lstlisting}

\textbf{Spiegazione:} La classe Calcolatrice implementa le 4 operazioni base, lanciando eccezioni personalizzate per situazioni non valide: DivisoreZeroException per divisioni per zero e OperazioneNonValidaException per numeri troppo grandi.

\subsection{Esercizio 3.6: Conta righe file con gestione eccezioni}

\begin{lstlisting}
import java.io.BufferedReader;
import java.io.FileReader;
import java.io.FileNotFoundException;
import java.io.IOException;

public class ContaRigheConEccezioni {

    public static int contaRighe(String percorsoFile)
            throws FileNotFoundException, IOException {

        int righe = 0;

        try (BufferedReader br = new BufferedReader(
                new FileReader(percorsoFile))) {

            while (br.readLine() != null) {
                righe++;
            }
        }
        // Try-with-resources chiude automaticamente br
        // Le eccezioni vengono propagate al chiamante

        return righe;
    }

    public static void main(String[] args) {
        String[] files = {"documento.txt", "file_inesistente.txt",
                         "dati.csv"};

        for (String file : files) {
            try {
                int numeroRighe = contaRighe(file);
                System.out.println(file + ": " + numeroRighe + " righe");

            } catch (FileNotFoundException e) {
                System.out.println("ERRORE: File '" + file +
                                 "' non trovato");
                System.out.println("Verifica che il file esista " +
                                 "nella directory corrente");

            } catch (IOException e) {
                System.out.println("ERRORE lettura file '" + file + "': " +
                                 e.getMessage());
                System.out.println("Il file potrebbe essere corrotto " +
                                 "o non accessibile");
            }
        }
    }
}
\end{lstlisting}

\textbf{Spiegazione:} Il programma usa catch separati per FileNotFoundException (file non esiste) e IOException (errore generico di lettura), fornendo messaggi di errore specifici per ogni situazione.

\subsection{Esercizio 3.7: Gerarchia eccezioni prenotazioni}

\begin{lstlisting}
// Eccezione base
class PrenotazioneException extends Exception {
    public PrenotazioneException(String messaggio) {
        super(messaggio);
    }
}

// Eccezioni specifiche
class PostiEsauritiException extends PrenotazioneException {
    private int postiRichiesti;
    private int postiDisponibili;

    public PostiEsauritiException(int richiesti, int disponibili) {
        super("Posti esauriti: richiesti " + richiesti +
              ", disponibili " + disponibili);
        this.postiRichiesti = richiesti;
        this.postiDisponibili = disponibili;
    }

    public int getPostiDisponibili() {
        return postiDisponibili;
    }
}

class DataNonValidaException extends PrenotazioneException {
    public DataNonValidaException(String data) {
        super("Data non valida: " + data);
    }
}

class PagamentoFallitoException extends PrenotazioneException {
    private String motivoFallimento;

    public PagamentoFallitoException(String motivo) {
        super("Pagamento fallito: " + motivo);
        this.motivoFallimento = motivo;
    }

    public String getMotivoFallimento() {
        return motivoFallimento;
    }
}

// Sistema di prenotazioni
public class SistemaPrenotazioni {
    private int postiTotali = 100;
    private int postiPrenotati = 0;

    public void prenota(String data, int numPersone, String metodoPagamento)
            throws PrenotazioneException {

        // Valida data (formato semplice: gg/mm/aaaa)
        if (!data.matches("\\d{2}/\\d{2}/\\d{4}")) {
            throw new DataNonValidaException(data);
        }

        // Controlla disponibilita' posti
        int postiLiberi = postiTotali - postiPrenotati;
        if (numPersone > postiLiberi) {
            throw new PostiEsauritiException(numPersone, postiLiberi);
        }

        // Simula pagamento
        if (!elaboraPagamento(metodoPagamento)) {
            throw new PagamentoFallitoException(
                "Metodo di pagamento non valido: " + metodoPagamento);
        }

        // Prenotazione confermata
        postiPrenotati += numPersone;
        System.out.println("Prenotazione confermata per " + numPersone +
                         " persone il " + data);
        System.out.println("Posti rimanenti: " +
                         (postiTotali - postiPrenotati));
    }

    private boolean elaboraPagamento(String metodo) {
        // Simula validazione pagamento
        return metodo.equals("carta") || metodo.equals("contanti");
    }

    public static void main(String[] args) {
        SistemaPrenotazioni sistema = new SistemaPrenotazioni();

        // Test varie prenotazioni
        String[][] prenotazioni = {
            {"15/06/2025", "4", "carta"},
            {"20/06/2025", "2", "contanti"},
            {"data_errata", "3", "carta"},
            {"25/06/2025", "150", "carta"},
            {"30/06/2025", "5", "paypal"}
        };

        for (String[] p : prenotazioni) {
            try {
                System.out.println("\n--- Tentativo prenotazione ---");
                sistema.prenota(p[0], Integer.parseInt(p[1]), p[2]);

            } catch (DataNonValidaException e) {
                System.out.println("ERRORE: " + e.getMessage());
                System.out.println("Formato richiesto: gg/mm/aaaa");

            } catch (PostiEsauritiException e) {
                System.out.println("ERRORE: " + e.getMessage());
                System.out.println("Suggerimento: prenotare solo " +
                                 e.getPostiDisponibili() + " posti");

            } catch (PagamentoFallitoException e) {
                System.out.println("ERRORE: " + e.getMessage());
                System.out.println("Metodi accettati: carta, contanti");

            } catch (PrenotazioneException e) {
                // Catch generico per altre eccezioni
                System.out.println("ERRORE generico: " + e.getMessage());
            }
        }
    }
}
\end{lstlisting}

\textbf{Spiegazione:} Viene creata una gerarchia di eccezioni con PrenotazioneException come base e tre eccezioni specifiche che ereditano da essa. Ogni eccezione può memorizzare informazioni aggiuntive (posti disponibili, motivo fallimento) e il catch può gestirle in modo specifico.

\subsection{Esercizio 3.8: Parser JSON semplificato}

\begin{lstlisting}
// Eccezioni personalizzate per parsing JSON
class JsonSyntaxException extends Exception {
    public JsonSyntaxException(String messaggio) {
        super("Errore sintassi JSON: " + messaggio);
    }
}

class JsonKeyException extends Exception {
    public JsonKeyException(String chiave) {
        super("Chiave JSON non valida: '" + chiave + "'");
    }
}

class JsonValueException extends Exception {
    public JsonValueException(String valore) {
        super("Valore JSON non valido: '" + valore + "'");
    }
}

// Parser JSON semplificato
public class SimpleJsonParser {

    public static void parseJson(String json)
            throws JsonSyntaxException, JsonKeyException, JsonValueException {

        // Rimuove spazi
        json = json.trim();

        // Verifica graffe di apertura e chiusura
        if (!json.startsWith("{") || !json.endsWith("}")) {
            throw new JsonSyntaxException(
                "Il JSON deve iniziare con { e finire con }");
        }

        // Rimuove graffe
        String contenuto = json.substring(1, json.length() - 1).trim();

        // Se vuoto, e' valido
        if (contenuto.isEmpty()) {
            System.out.println("JSON vuoto valido: {}");
            return;
        }

        // Split per coppie chiave:valore
        String[] coppie = contenuto.split(",");

        for (String coppia : coppie) {
            coppia = coppia.trim();

            // Verifica presenza di :
            if (!coppia.contains(":")) {
                throw new JsonSyntaxException(
                    "Manca ':' nella coppia chiave:valore");
            }

            // Split chiave:valore
            String[] parti = coppia.split(":", 2);
            String chiave = parti[0].trim();
            String valore = parti[1].trim();

            // Valida chiave (deve essere tra virgolette)
            if (!chiave.startsWith("\"") || !chiave.endsWith("\"")) {
                throw new JsonKeyException(chiave);
            }

            // Rimuove virgolette dalla chiave
            chiave = chiave.substring(1, chiave.length() - 1);

            // Valida valore (stringa tra virgolette o numero)
            if (!isValoreValido(valore)) {
                throw new JsonValueException(valore);
            }

            System.out.println("Coppia valida -> " + chiave +
                             " = " + valore);
        }

        System.out.println("JSON valido!");
    }

    private static boolean isValoreValido(String valore) {
        // Valore valido se: stringa tra virgolette, numero,
        // true, false, null
        if (valore.startsWith("\"") && valore.endsWith("\"")) {
            return true; // Stringa
        }
        if (valore.equals("true") || valore.equals("false") ||
            valore.equals("null")) {
            return true; // Boolean o null
        }
        try {
            Double.parseDouble(valore);
            return true; // Numero
        } catch (NumberFormatException e) {
            return false;
        }
    }

    public static void main(String[] args) {
        String[] testJson = {
            "{\"nome\":\"Mario\", \"eta\":30}",
            "{\"valido\":true, \"count\":42}",
            "{\"errore\":senza_virgolette}",
            "{chiave_senza_virgolette:\"valore\"}",
            "{\"completo\":\"ok\", senza_due_punti}",
            "non inizia con graffa"
        };

        for (String json : testJson) {
            try {
                System.out.println("\n=== Test JSON ===");
                System.out.println("Input: " + json);
                parseJson(json);

            } catch (JsonSyntaxException e) {
                System.out.println("ERRORE SINTASSI: " + e.getMessage());

            } catch (JsonKeyException e) {
                System.out.println("ERRORE CHIAVE: " + e.getMessage());
                System.out.println("Le chiavi devono essere tra " +
                                 "virgolette doppie");

            } catch (JsonValueException e) {
                System.out.println("ERRORE VALORE: " + e.getMessage());
                System.out.println("I valori stringa devono essere " +
                                 "tra virgolette");
            }
        }
    }
}
\end{lstlisting}

\textbf{Spiegazione:} Il parser verifica la sintassi JSON base con tre tipi di eccezioni: JsonSyntaxException per errori strutturali (graffe, due punti), JsonKeyException per chiavi non tra virgolette, JsonValueException per valori non validi. Ogni eccezione fornisce informazioni specifiche sull'errore.

\section{Capitolo 4: ArrayList}

\subsection{Esercizio 4.1: Lista della spesa}

\begin{lstlisting}
import java.util.ArrayList;
import java.util.Scanner;

public class ListaSpesa {
    private static ArrayList<String> spesa = new ArrayList<>();
    private static Scanner scanner = new Scanner(System.in);

    public static void main(String[] args) {
        boolean continua = true;

        while (continua) {
            mostraMenu();
            int scelta = scanner.nextInt();
            scanner.nextLine(); // Consuma newline

            switch (scelta) {
                case 1:
                    aggiungiProdotto();
                    break;
                case 2:
                    visualizzaLista();
                    break;
                case 3:
                    rimuoviProdotto();
                    break;
                case 4:
                    verificaProdotto();
                    break;
                case 5:
                    continua = false;
                    System.out.println("Buona spesa!");
                    break;
                default:
                    System.out.println("Scelta non valida");
            }
        }
        scanner.close();
    }

    private static void mostraMenu() {
        System.out.println("\n=== LISTA DELLA SPESA ===");
        System.out.println("1. Aggiungi prodotto");
        System.out.println("2. Visualizza lista");
        System.out.println("3. Rimuovi prodotto");
        System.out.println("4. Verifica prodotto");
        System.out.println("5. Esci");
        System.out.print("Scelta: ");
    }

    private static void aggiungiProdotto() {
        System.out.print("Nome prodotto: ");
        String prodotto = scanner.nextLine();

        // Verifica se gia' presente
        if (spesa.contains(prodotto)) {
            System.out.println("Prodotto gia' presente nella lista!");
        } else {
            spesa.add(prodotto);
            System.out.println("Prodotto aggiunto!");
        }
    }

    private static void visualizzaLista() {
        if (spesa.isEmpty()) {
            System.out.println("Lista vuota");
            return;
        }

        System.out.println("\n--- LISTA DELLA SPESA ---");
        for (int i = 0; i < spesa.size(); i++) {
            System.out.println((i + 1) + ". " + spesa.get(i));
        }
        System.out.println("Totale prodotti: " + spesa.size());
    }

    private static void rimuoviProdotto() {
        if (spesa.isEmpty()) {
            System.out.println("Lista vuota");
            return;
        }

        visualizzaLista();
        System.out.print("Numero prodotto da rimuovere: ");
        int indice = scanner.nextInt() - 1;
        scanner.nextLine();

        if (indice >= 0 && indice < spesa.size()) {
            String rimosso = spesa.remove(indice);
            System.out.println("Rimosso: " + rimosso);
        } else {
            System.out.println("Numero non valido");
        }
    }

    private static void verificaProdotto() {
        System.out.print("Prodotto da cercare: ");
        String prodotto = scanner.nextLine();

        if (spesa.contains(prodotto)) {
            int posizione = spesa.indexOf(prodotto) + 1;
            System.out.println("'" + prodotto + "' e' presente " +
                             "(posizione " + posizione + ")");
        } else {
            System.out.println("'" + prodotto + "' non e' nella lista");
        }
    }
}
\end{lstlisting}

\textbf{Spiegazione:} Il programma gestisce una lista della spesa con menu interattivo, permettendo di aggiungere prodotti (controllando duplicati con contains), visualizzarli, rimuoverli per indice e verificare la presenza di un prodotto.

\subsection{Esercizio 4.2: Temperature settimanali}

\begin{lstlisting}
import java.util.ArrayList;
import java.util.Scanner;

public class TemperatureSettimanali {
    public static void main(String[] args) {
        ArrayList<Double> temperature = new ArrayList<>();
        Scanner scanner = new Scanner(System.in);

        String[] giorni = {"Lunedi", "Martedi", "Mercoledi", "Giovedi",
                          "Venerdi", "Sabato", "Domenica"};

        // Inserimento temperature
        System.out.println("=== TEMPERATURE MASSIME SETTIMANALI ===");
        for (String giorno : giorni) {
            System.out.print(giorno + ": ");
            double temp = scanner.nextDouble();
            temperature.add(temp);
        }

        // Calcolo media
        double somma = 0;
        for (double temp : temperature) {
            somma += temp;
        }
        double media = somma / temperature.size();

        // Trova massima e minima
        double massima = temperature.get(0);
        double minima = temperature.get(0);
        int giornoMax = 0;
        int giornoMin = 0;

        for (int i = 1; i < temperature.size(); i++) {
            double temp = temperature.get(i);
            if (temp > massima) {
                massima = temp;
                giornoMax = i;
            }
            if (temp < minima) {
                minima = temp;
                giornoMin = i;
            }
        }

        // Visualizza risultati
        System.out.println("\n=== ANALISI ===");
        System.out.printf("Temperatura media: %.1f gradi C\n", media);
        System.out.printf("Temperatura massima: %.1f gradi C (%s)\n",
                        massima, giorni[giornoMax]);
        System.out.printf("Temperatura minima: %.1f gradi C (%s)\n",
                        minima, giorni[giornoMin]);

        // Giorni sopra la media
        System.out.println("\nGiorni sopra la media:");
        for (int i = 0; i < temperature.size(); i++) {
            if (temperature.get(i) > media) {
                System.out.printf("- %s: %.1f gradi C\n",
                                giorni[i], temperature.get(i));
            }
        }

        scanner.close();
    }
}
\end{lstlisting}

\textbf{Spiegazione:} Il programma memorizza 7 temperature in un ArrayList, calcola la media sommando tutti i valori, trova massima e minima iterando sulla lista e visualizza i giorni sopra la media confrontando ogni elemento con la media calcolata.

\subsection{Esercizio 4.3: Rimozione duplicati}

\begin{lstlisting}
import java.util.ArrayList;

public class RimuoviDuplicati {

    public static ArrayList<String> rimuoviDuplicati(
            ArrayList<String> listaOriginale) {

        ArrayList<String> listaSenzaDuplicati = new ArrayList<>();

        // Itera sulla lista originale
        for (String elemento : listaOriginale) {
            // Aggiunge solo se non gia' presente nella nuova lista
            if (!listaSenzaDuplicati.contains(elemento)) {
                listaSenzaDuplicati.add(elemento);
            }
        }

        return listaSenzaDuplicati;
    }

    public static void main(String[] args) {
        // Test con lista contenente duplicati
        ArrayList<String> frutti = new ArrayList<>();
        frutti.add("Mela");
        frutti.add("Banana");
        frutti.add("Mela");
        frutti.add("Arancia");
        frutti.add("Banana");
        frutti.add("Kiwi");
        frutti.add("Mela");
        frutti.add("Arancia");

        System.out.println("=== LISTA ORIGINALE ===");
        System.out.println(frutti);
        System.out.println("Dimensione: " + frutti.size());

        ArrayList<String> senzaDuplicati = rimuoviDuplicati(frutti);

        System.out.println("\n=== LISTA SENZA DUPLICATI ===");
        System.out.println(senzaDuplicati);
        System.out.println("Dimensione: " + senzaDuplicati.size());

        // Verifica ordine mantenuto
        System.out.println("\nOrdine prima apparizione mantenuto: " +
                         senzaDuplicati.get(0).equals("Mela"));
    }
}
\end{lstlisting}

\textbf{Spiegazione:} Il metodo rimuoviDuplicati() crea un nuovo ArrayList e aggiunge ogni elemento solo se non è già presente (verificato con contains). Questo mantiene l'ordine di prima apparizione degli elementi.

\subsection{Esercizio 4.4-4.8}

Per motivi di spazio, le soluzioni complete degli esercizi 4.4 (Gestione biblioteca), 4.5 (Registro voti), 4.6 (Playlist musicale), 4.7 (Sistema prenotazioni cinema) e 4.8 (Social network semplificato) seguono lo stesso approccio degli esercizi precedenti: creano classi personalizzate, le gestiscono in ArrayList e implementano le funzionalità richieste usando i metodi add(), remove(), get(), contains() e iterazioni con for/for-each.

\chapter{Quick Reference}

\section{Registri 8086}

\subsection{General Purpose (16 bit)}

\begin{tabular}{ll}
\textbf{AX} & Accumulatore (AH + AL) \\
\textbf{BX} & Base (BH + BL) \\
\textbf{CX} & Contatore (CH + CL) \\
\textbf{DX} & Dati (DH + DL) \\
\end{tabular}

\subsection{Puntatore e Indice}

\begin{tabular}{ll}
\textbf{SP} & Stack Pointer \\
\textbf{BP} & Base Pointer \\
\textbf{SI} & Source Index \\
\textbf{DI} & Destination Index \\
\textbf{IP} & Instruction Pointer \\
\end{tabular}

\subsection{Segmento}

\begin{tabular}{ll}
\textbf{CS} & Code Segment \\
\textbf{DS} & Data Segment \\
\textbf{SS} & Stack Segment \\
\textbf{ES} & Extra Segment \\
\end{tabular}

\section{Flag Register}

\begin{tabular}{lll}
\textbf{Bit} & \textbf{Nome} & \textbf{Descrizione} \\
0 & CF & Carry Flag \\
2 & PF & Parity Flag \\
4 & AF & Auxiliary Carry \\
6 & ZF & Zero Flag \\
7 & SF & Sign Flag \\
8 & TF & Trap Flag \\
9 & IF & Interrupt Flag \\
10 & DF & Direction Flag \\
11 & OF & Overflow Flag \\
\end{tabular}

\section{Modalità Indirizzamento}

\begin{lstlisting}
MOV AX, 1234h          ; Immediate
MOV AX, BX             ; Register
MOV AX, [1000h]        ; Direct
MOV AX, [BX]           ; Register Indirect
MOV AX, [BX+4]         ; Based
MOV AX, [SI+4]         ; Indexed
MOV AX, [BX+SI+4]      ; Based Indexed
\end{lstlisting}

\section{Istruzioni per Categoria}

\subsection{Trasferimento Dati}

\begin{lstlisting}
MOV dest, src          ; Move
XCHG op1, op2          ; Exchange
LEA reg, mem           ; Load Effective Address
PUSH operando          ; Push to stack
POP operando           ; Pop from stack
IN AL/AX, porta        ; Input from port
OUT porta, AL/AX       ; Output to port
XLAT                   ; Translate
\end{lstlisting}

\subsection{Aritmetiche}

\begin{lstlisting}
ADD dest, src          ; Addition
ADC dest, src          ; Add with carry
SUB dest, src          ; Subtraction
SBB dest, src          ; Subtract with borrow
INC operando           ; Increment
DEC operando           ; Decrement
MUL operando           ; Unsigned multiply
IMUL operando          ; Signed multiply
DIV operando           ; Unsigned divide
IDIV operando          ; Signed divide
NEG operando           ; Negate
CMP op1, op2           ; Compare
\end{lstlisting}

\subsection{Logiche}

\begin{lstlisting}
AND dest, src          ; Logical AND
OR dest, src           ; Logical OR
XOR dest, src          ; Logical XOR
NOT operando           ; Logical NOT
TEST op1, op2          ; Test (AND without save)
SHL dest, count        ; Shift left
SHR dest, count        ; Shift right
SAL dest, count        ; Arithmetic shift left
SAR dest, count        ; Arithmetic shift right
ROL dest, count        ; Rotate left
ROR dest, count        ; Rotate right
RCL dest, count        ; Rotate through carry left
RCR dest, count        ; Rotate through carry right
\end{lstlisting}

\subsection{Controllo Flusso}

\begin{lstlisting}
JMP label              ; Unconditional jump
JE/JZ label            ; Jump if equal/zero
JNE/JNZ label          ; Jump if not equal/zero
JG/JNLE label          ; Jump if greater (signed)
JGE/JNL label          ; Jump if >= (signed)
JL/JNGE label          ; Jump if less (signed)
JLE/JNG label          ; Jump if <= (signed)
JA/JNBE label          ; Jump if above (unsigned)
JAE/JNB label          ; Jump if >= (unsigned)
JB/JNAE label          ; Jump if below (unsigned)
JBE/JNA label          ; Jump if <= (unsigned)
LOOP label             ; Loop (decrement CX)
CALL procedura         ; Call procedure
RET                    ; Return
INT numero             ; Software interrupt
\end{lstlisting}

\subsection{Stringhe}

\begin{lstlisting}
MOVSB/MOVSW            ; Move string
LODSB/LODSW            ; Load string
STOSB/STOSW            ; Store string
CMPSB/CMPSW            ; Compare string
SCASB/SCASW            ; Scan string
REP                    ; Repeat prefix
REPE/REPZ              ; Repeat while equal
REPNE/REPNZ            ; Repeat while not equal
CLD                    ; Clear direction flag
STD                    ; Set direction flag
\end{lstlisting}

\section{Interrupt DOS (INT 21h)}

\begin{lstlisting}
MOV AH, 01h
INT 21h                ; Input character (AL)

MOV AH, 02h
MOV DL, char
INT 21h                ; Output character

MOV AH, 09h
MOV DX, OFFSET msg
INT 21h                ; Print string ($-terminated)

MOV AH, 4Ch
INT 21h                ; Exit program
\end{lstlisting}

\section{Interrupt BIOS}

\subsection{INT 10h (Video)}

\begin{lstlisting}
MOV AH, 00h
MOV AL, mode
INT 10h                ; Set video mode

MOV AH, 0Eh
MOV AL, char
INT 10h                ; Teletype output
\end{lstlisting}

\subsection{INT 16h (Keyboard)}

\begin{lstlisting}
MOV AH, 00h
INT 16h                ; Read key (AL=ASCII, AH=scan)
\end{lstlisting}

\section{Struttura Programma .COM}

\begin{lstlisting}
ORG 100h               ; DOS PSP = 256 byte

start:
    MOV AH, 09h
    MOV DX, OFFSET msg
    INT 21h

    MOV AH, 4Ch
    INT 21h

msg DB 'Hello$'
\end{lstlisting}

\section{Struttura Programma .EXE}

\begin{lstlisting}
.MODEL SMALL
.STACK 100h

.DATA
msg DB 'Hello$'

.CODE
main PROC
    MOV AX, @DATA
    MOV DS, AX

    MOV AH, 09h
    MOV DX, OFFSET msg
    INT 21h

    MOV AH, 4Ch
    INT 21h
main ENDP

END main
\end{lstlisting}

\section{Calcolo Indirizzo Fisico}

\[
\text{Fisico} = (\text{Segmento} \times 16) + \text{Offset}
\]

\section{Tabella ASCII (estratto)}

\begin{tabular}{llll}
\textbf{Dec} & \textbf{Hex} & \textbf{Char} & \textbf{Descrizione} \\
0 & 00h & NUL & Null \\
7 & 07h & BEL & Bell \\
8 & 08h & BS & Backspace \\
9 & 09h & TAB & Horizontal Tab \\
10 & 0Ah & LF & Line Feed \\
13 & 0Dh & CR & Carriage Return \\
32 & 20h & SPACE & Spazio \\
48-57 & 30h-39h & 0-9 & Cifre \\
65-90 & 41h-5Ah & A-Z & Maiuscole \\
97-122 & 61h-7Ah & a-z & Minuscole \\
\end{tabular}

\section{Scan Code Tastiera (estratto)}

\begin{tabular}{ll}
\textbf{Scan Code} & \textbf{Tasto} \\
01h & ESC \\
1Ch & Enter \\
39h & Spazio \\
3Bh-44h & F1-F10 \\
48h & Freccia su \\
50h & Freccia giù \\
4Bh & Freccia sinistra \\
4Dh & Freccia destra \\
\end{tabular}

\section{Porte I/O Comuni}

\begin{tabular}{ll}
\textbf{Porta} & \textbf{Descrizione} \\
20h-21h & PIC (8259) Master \\
40h-43h & PIT (8253/8254) Timer \\
60h, 64h & Tastiera (8042) \\
3F8h-3FFh & COM1 (UART) \\
378h-37Fh & LPT1 (Parallela) \\
\end{tabular}

% Bibliografia (PHP)
\chapter*{Bibliografia e Risorse}
\addcontentsline{toc}{chapter}{Bibliografia e Risorse}

\section*{Manuali Ufficiali}

\begin{enumerate}
    \item \textbf{PHP Manual} \\
    Documentazione ufficiale completa \\
    \url{https://www.php.net/manual/en/}

    \item \textbf{PHP: The Right Way} \\
    Best practices PHP moderne \\
    \url{https://phptherightway.com/}

    \item \textbf{PHP Standards Recommendations (PSR)} \\
    PHP-FIG coding standards \\
    \url{https://www.php-fig.org/psr/}
\end{enumerate}

\section*{Libri di Testo}

\begin{enumerate}
    \item \textbf{PHP \& MySQL: Novice to Ninja} \\
    Kevin Yank \\
    SitePoint, 7th Edition, 2021 \\
    ISBN: 978-1925836387

    \item \textbf{Modern PHP: New Features and Good Practices} \\
    Josh Lockhart \\
    O'Reilly Media, 2015 \\
    ISBN: 978-1491905012

    \item \textbf{PHP Objects, Patterns, and Practice} \\
    Matt Zandstra \\
    Apress, 6th Edition, 2021 \\
    ISBN: 978-1484267905

    \item \textbf{Learning PHP, MySQL \& JavaScript} \\
    Robin Nixon \\
    O'Reilly Media, 6th Edition, 2021 \\
    ISBN: 978-1492093824

    \item \textbf{PHP Cookbook} \\
    David Sklar, Adam Trachtenberg \\
    O'Reilly Media, 4th Edition, 2020 \\
    ISBN: 978-1098121327
\end{enumerate}

\section*{Sicurezza Web}

\subsection*{OWASP Resources}

\begin{itemize}
    \item \textbf{OWASP Top 10} \\
    \url{https://owasp.org/Top10/} \\
    Le 10 vulnerabilità web più critiche (aggiornato 2021)

    \item \textbf{OWASP PHP Security Cheat Sheet} \\
    \url{https://cheatsheetseries.owasp.org/cheatsheets/PHP_Configuration_Cheat_Sheet.html}

    \item \textbf{OWASP Testing Guide} \\
    \url{https://owasp.org/www-project-web-security-testing-guide/}

    \item \textbf{OWASP Secure Coding Practices} \\
    \url{https://owasp.org/www-project-secure-coding-practices-quick-reference-guide/}
\end{itemize}

\subsection*{Libri Sicurezza}

\begin{enumerate}
    \item \textbf{The Tangled Web: A Guide to Securing Modern Web Applications} \\
    Michal Zalewski \\
    No Starch Press, 2011 \\
    ISBN: 978-1593273880

    \item \textbf{Web Application Security} \\
    Andrew Hoffman \\
    O'Reilly Media, 2020 \\
    ISBN: 978-1492053118
\end{enumerate}

\section*{Framework PHP}

\subsection*{Laravel}

\begin{itemize}
    \item \textbf{Documentazione ufficiale}: \url{https://laravel.com/docs}
    \item \textbf{Laracasts}: Video tutorial premium: \url{https://laracasts.com/}
    \item \textbf{Laravel News}: \url{https://laravel-news.com/}
\end{itemize}

\subsection*{Symfony}

\begin{itemize}
    \item \textbf{Documentazione}: \url{https://symfony.com/doc/current/index.html}
    \item \textbf{Symfony Blog}: \url{https://symfony.com/blog/}
\end{itemize}

\subsection*{Altri Framework}

\begin{itemize}
    \item \textbf{CodeIgniter}: \url{https://codeigniter.com/}
    \item \textbf{Slim Framework}: \url{https://www.slimframework.com/} (microframework)
    \item \textbf{Laminas (ex Zend)}: \url{https://getlaminas.org/}
    \item \textbf{CakePHP}: \url{https://cakephp.org/}
\end{itemize}

\section*{Tutorial e Corsi Online}

\subsection*{Tutorial Gratuiti}

\begin{itemize}
    \item \textbf{W3Schools PHP Tutorial} \\
    \url{https://www.w3schools.com/php/} \\
    Tutorial interattivo per principianti

    \item \textbf{PHP Tutorial - Tutorialspoint} \\
    \url{https://www.tutorialspoint.com/php/}

    \item \textbf{Learn PHP - Codecademy} \\
    \url{https://www.codecademy.com/learn/learn-php}

    \item \textbf{PHP for Beginners - Laracasts} \\
    \url{https://laracasts.com/series/php-for-beginners} (gratuito)
\end{itemize}

\subsection*{Piattaforme Video}

\begin{itemize}
    \item \textbf{Laracasts}: PHP e Laravel (subscription) \\
    \url{https://laracasts.com/}

    \item \textbf{Udemy}: Corsi PHP vari \\
    \url{https://www.udemy.com/topic/php/}

    \item \textbf{Pluralsight}: Percorsi PHP professionali \\
    \url{https://www.pluralsight.com/paths/php}

    \item \textbf{YouTube Channels}:
    \begin{itemize}
        \item Traversy Media: \url{https://www.youtube.com/c/TraversyMedia}
        \item The Net Ninja: \url{https://www.youtube.com/c/TheNetNinja}
        \item ProgramWithGio: \url{https://www.youtube.com/c/ProgramWithGio}
    \end{itemize}
\end{itemize}

\section*{Tool e Ambiente di Sviluppo}

\subsection*{Ambienti Locali}

\begin{itemize}
    \item \textbf{XAMPP}: \url{https://www.apachefriends.org/} (Windows/Mac/Linux)
    \item \textbf{MAMP}: \url{https://www.mamp.info/} (Mac/Windows)
    \item \textbf{Laragon}: \url{https://laragon.org/} (Windows, moderno)
    \item \textbf{Docker con PHP}: \url{https://hub.docker.com/_/php}
\end{itemize}

\subsection*{IDE e Editor}

\begin{itemize}
    \item \textbf{PhpStorm}: \url{https://www.jetbrains.com/phpstorm/} (professionale, licenza studenti gratuita)
    \item \textbf{Visual Studio Code} con estensioni:
    \begin{itemize}
        \item PHP Intelephense: \url{https://marketplace.visualstudio.com/items?itemName=bmewburn.vscode-intelephense-client}
        \item PHP Debug: \url{https://marketplace.visualstudio.com/items?itemName=xdebug.php-debug}
    \end{itemize}
    \item \textbf{Sublime Text}: \url{https://www.sublimetext.com/}
    \item \textbf{NetBeans}: \url{https://netbeans.apache.org/}
\end{itemize}

\subsection*{Debugging e Profiling}

\begin{itemize}
    \item \textbf{Xdebug}: \url{https://xdebug.org/} (debugger PHP)
    \item \textbf{Blackfire.io}: \url{https://blackfire.io/} (profiling performance)
    \item \textbf{New Relic}: \url{https://newrelic.com/} (monitoring APM)
    \item \textbf{PHPStan}: \url{https://phpstan.org/} (static analysis)
    \item \textbf{Psalm}: \url{https://psalm.dev/} (static analysis)
\end{itemize}

\subsection*{Testing}

\begin{itemize}
    \item \textbf{PHPUnit}: \url{https://phpunit.de/} (unit testing)
    \item \textbf{Pest PHP}: \url{https://pestphp.com/} (testing elegante)
    \item \textbf{Codeception}: \url{https://codeception.com/} (full-stack testing)
    \item \textbf{Behat}: \url{https://docs.behat.org/} (BDD)
\end{itemize}

\subsection*{Dependency Management}

\begin{itemize}
    \item \textbf{Composer}: \url{https://getcomposer.org/} (package manager PHP)
    \item \textbf{Packagist}: \url{https://packagist.org/} (repository Composer)
\end{itemize}

\section*{Community e Forum}

\begin{itemize}
    \item \textbf{Stack Overflow} \\
    Tag [php]: \url{https://stackoverflow.com/questions/tagged/php}

    \item \textbf{Reddit r/PHP} \\
    \url{https://www.reddit.com/r/PHP/}

    \item \textbf{PHP Developers Discord} \\
    Community Discord attiva

    \item \textbf{Laravel Discord} \\
    \url{https://discord.gg/laravel}

    \item \textbf{PHP User Groups} \\
    \url{https://www.php.net/ug.php} (gruppi locali)

    \item \textbf{Conferenze PHP}:
    \begin{itemize}
        \item PHP[World]
        \item Laracon
        \item SymfonyCon
        \item PHP UK Conference
    \end{itemize}
\end{itemize}

\section*{Database e Storage}

\begin{itemize}
    \item \textbf{MySQL Documentation}: \url{https://dev.mysql.com/doc/}
    \item \textbf{MariaDB Knowledge Base}: \url{https://mariadb.com/kb/}
    \item \textbf{PostgreSQL Documentation}: \url{https://www.postgresql.org/docs/}
    \item \textbf{PDO Manual}: \url{https://www.php.net/manual/en/book.pdo.php}
    \item \textbf{MySQLi Manual}: \url{https://www.php.net/manual/en/book.mysqli.php}
    \item \textbf{Redis}: \url{https://redis.io/} (caching)
    \item \textbf{Memcached}: \url{https://www.memcached.org/}
\end{itemize}

\section*{CMS e E-commerce}

\begin{itemize}
    \item \textbf{WordPress}: \url{https://wordpress.org/}
    \begin{itemize}
        \item Codex: \url{https://codex.wordpress.org/}
        \item Developer Handbook: \url{https://developer.wordpress.org/}
    \end{itemize}

    \item \textbf{Drupal}: \url{https://www.drupal.org/}
    
    \item \textbf{Joomla}: \url{https://www.joomla.org/}

    \item \textbf{Magento}: \url{https://magento.com/} (e-commerce)

    \item \textbf{WooCommerce}: \url{https://woocommerce.com/} (WordPress e-commerce)

    \item \textbf{PrestaShop}: \url{https://www.prestashop.com/}
\end{itemize}

\section*{API e Web Services}

\begin{itemize}
    \item \textbf{REST API Tutorial}: \url{https://restfulapi.net/}
    \item \textbf{GraphQL PHP}: \url{https://webonyx.github.io/graphql-php/}
    \item \textbf{Guzzle HTTP Client}: \url{https://docs.guzzlephp.org/}
    \item \textbf{cURL Manual}: \url{https://www.php.net/manual/en/book.curl.php}
    \item \textbf{JSON Web Tokens (JWT)}: \url{https://jwt.io/}
\end{itemize}

\section*{Deployment e DevOps}

\begin{itemize}
    \item \textbf{Deployer}: \url{https://deployer.org/} (deployment tool)
    \item \textbf{Laravel Forge}: \url{https://forge.laravel.com/} (server management)
    \item \textbf{Laravel Vapor}: \url{https://vapor.laravel.com/} (serverless)
    \item \textbf{Docker Compose per PHP}: Tutorial e best practices
    \item \textbf{GitHub Actions}: \url{https://github.com/features/actions} (CI/CD)
    \item \textbf{GitLab CI/CD}: \url{https://docs.gitlab.com/ee/ci/}
\end{itemize}

\section*{Sicurezza Avanzata}

\subsection*{Penetration Testing}

\begin{itemize}
    \item \textbf{DVWA (Damn Vulnerable Web Application)} \\
    \url{https://github.com/digininja.org/dvwa} \\
    Ambiente per testare vulnerabilità

    \item \textbf{bWAPP} \\
    \url{http://www.itsecgames.com/} \\
    Buggy web application per training

    \item \textbf{WebGoat PHP} \\
    Applicazione volutamente vulnerabile per apprendimento
\end{itemize}

\subsection*{Tool Sicurezza}

\begin{itemize}
    \item \textbf{Burp Suite}: \url{https://portswigger.net/burp} (penetration testing)
    \item \textbf{OWASP ZAP}: \url{https://www.zaproxy.org/} (security scanner)
    \item \textbf{Snyk}: \url{https://snyk.io/} (vulnerability scanning)
    \item \textbf{SonarQube}: \url{https://www.sonarqube.org/} (code quality \& security)
\end{itemize}

\section*{Risorse in Italiano}

\begin{itemize}
    \item \textbf{PHP.net Italian Manual} \\
    \url{https://www.php.net/manual/it/}

    \item \textbf{HTML.it - Guida PHP} \\
    \url{https://www.html.it/guide/guida-php-di-base/}

    \item \textbf{MRW.it - Tutorial PHP} \\
    \url{https://www.mrw.it/php/}

    \item \textbf{Forum HTML.it - PHP} \\
    \url{https://forum.html.it/forum/php}

    \item \textbf{Gruppo Facebook "PHP Italia"} \\
    Community italiana attiva
\end{itemize}

\section*{Newsletter e Blog}

\begin{itemize}
    \item \textbf{PHP Weekly}: \url{http://www.phpweekly.com/}
    \item \textbf{Laravel News}: \url{https://laravel-news.com/}
    \item \textbf{Freek.dev}: \url{https://freek.dev/} (Laravel, Spatie)
    \item \textbf{PHP Annotated Monthly} (JetBrains): \url{https://blog.jetbrains.com/phpstorm/}
    \item \textbf{Scotch.io**: \url{https://scotch.io/tag/php}
\end{itemize}

\section*{Certificazioni}

\begin{itemize}
    \item \textbf{Zend Certified PHP Engineer} \\
    \url{https://www.zend.com/training/php-certification-exam} \\
    Certificazione ufficiale PHP

    \item \textbf{Laravel Certification} \\
    \url{https://exam.laravelcert.com/}

    \item \textbf{Symfony Certification} \\
    \url{https://certification.symfony.com/}

    \item \textbf{CompTIA Security+} \\
    Certificazione sicurezza generale applicabile al web
\end{itemize}

\section*{Open Source Projects (per imparare)}

\begin{itemize}
    \item \textbf{Laravel Framework}: \url{https://github.com/laravel/framework}
    \item \textbf{Symfony Components}: \url{https://github.com/symfony/symfony}
    \item \textbf{Guzzle}: \url{https://github.com/guzzle/guzzle}
    \item \textbf{Monolog}: \url{https://github.com/Seldaek/monolog}
    \item \textbf{PHPMailer}: \url{https://github.com/PHPMailer/PHPMailer}
    \item \textbf{Carbon}: \url{https://github.com/briannesbitt/Carbon} (date/time)
\end{itemize}

\section*{Standard e Specifiche}

\begin{itemize}
    \item \textbf{PSR-1}: Basic Coding Standard
    \item \textbf{PSR-2}: Coding Style Guide (deprecato, usa PSR-12)
    \item \textbf{PSR-4}: Autoloading Standard
    \item \textbf{PSR-7}: HTTP Message Interface
    \item \textbf{PSR-12}: Extended Coding Style Guide
    \item \textbf{PSR-15}: HTTP Server Request Handlers
    \item \textbf{PSR-18}: HTTP Client

    Tutti disponibili su: \url{https://www.php-fig.org/psr/}
\end{itemize}

\section*{Podcast}

\begin{itemize}
    \item \textbf{PHP Roundtable}: \url{https://www.phproundtable.com/}
    \item \textbf{Laravel Podcast}: \url{https://laravelpodcast.com/}
    \item \textbf{PHP Ugly}: \url{https://www.phpugly.com/}
    \item \textbf{No Compromises}: Focus su Laravel e best practices
\end{itemize}

\section*{Note Finali}

Le risorse elencate rappresentano una selezione curata per approfondire PHP a tutti i livelli. Si consiglia di:

\begin{itemize}
    \item Iniziare con documentazione ufficiale e PHP: The Right Way
    \item Praticare con progetti personali applicando best practices
    \item Studiare codice open-source di qualità (Laravel, Symfony)
    \item Partecipare attivamente a community online
    \item Mantenere focus su sicurezza (OWASP) in ogni fase
    \item Considerare certificazioni per validazione competenze professionali
\end{itemize}

\vspace{1cm}

\begin{center}
\textit{La sicurezza non è un prodotto, ma un processo!}
\end{center}


\end{document}
