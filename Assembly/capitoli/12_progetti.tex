\chapter{Progetti Applicativi}

\section{Progetto 1: Calcolatrice a 16 bit}

Implementare una calcolatrice che supporta +, -, *, / su numeri interi.

\begin{lstlisting}
.MODEL SMALL
.STACK 100h

.DATA
msg1 DB 'Inserisci primo numero: $'
msg2 DB 'Inserisci operatore (+,-,*,/): $'
msg3 DB 'Inserisci secondo numero: $'
result_msg DB 'Risultato: $'
num1 DW ?
num2 DW ?
operator DB ?

.CODE
main PROC
    MOV AX, @DATA
    MOV DS, AX

    ; Input primo numero
    MOV AH, 09h
    MOV DX, OFFSET msg1
    INT 21h
    CALL read_number
    MOV num1, AX

    ; Input operatore
    MOV AH, 09h
    MOV DX, OFFSET msg2
    INT 21h
    MOV AH, 01h
    INT 21h
    MOV operator, AL

    ; Input secondo numero
    MOV AH, 09h
    MOV DX, OFFSET msg3
    INT 21h
    CALL read_number
    MOV num2, AX

    ; Esegui operazione
    MOV AL, operator
    CMP AL, '+'
    JE do_add
    CMP AL, '-'
    JE do_sub
    CMP AL, '*'
    JE do_mul
    CMP AL, '/'
    JE do_div

do_add:
    MOV AX, num1
    ADD AX, num2
    JMP display

do_sub:
    MOV AX, num1
    SUB AX, num2
    JMP display

do_mul:
    MOV AX, num1
    MUL num2
    JMP display

do_div:
    MOV AX, num1
    XOR DX, DX
    DIV num2
    JMP display

display:
    CALL print_number

    MOV AH, 4Ch
    INT 21h
main ENDP

; Procedura per leggere numero decimale
read_number PROC
    ; ... implementazione ...
    RET
read_number ENDP

; Procedura per stampare numero
print_number PROC
    ; ... implementazione ...
    RET
print_number ENDP

END main
\end{lstlisting}

\section{Progetto 2: Ordinamento Array}

Bubble sort su array di 10 numeri.

\begin{lstlisting}
.DATA
array DW 64, 34, 25, 12, 22, 11, 90, 88, 45, 50
size DW 10

.CODE
bubble_sort PROC
    MOV CX, size
    DEC CX             ; n-1 passate

outer_loop:
    PUSH CX
    MOV SI, 0
    MOV CX, size
    DEC CX

inner_loop:
    MOV AX, array[SI]
    CMP AX, array[SI+2]
    JLE no_swap

    ; Scambia
    XCHG AX, array[SI+2]
    MOV array[SI], AX

no_swap:
    ADD SI, 2
    LOOP inner_loop

    POP CX
    LOOP outer_loop

    RET
bubble_sort ENDP
\end{lstlisting}

\section{Progetto 3: Editor di Testo Semplice}

Editor di testo minimale con buffer di 256 caratteri.

\section{Progetto 4: Gioco Snake}

Snake in modalità testo 80x25.

\section{Progetto 5: Bootloader}

Bootloader minimale che stampa un messaggio.

\begin{lstlisting}
[ORG 0x7C00]
[BITS 16]

start:
    MOV AX, 0x07C0
    MOV DS, AX
    MOV ES, AX

    MOV AH, 0x0E       ; Teletype
    MOV SI, msg

print_loop:
    LODSB
    OR AL, AL
    JZ done
    INT 0x10
    JMP print_loop

done:
    CLI
    HLT

msg DB 'Bootloader avviato!', 13, 10, 0

TIMES 510-($-$$) DB 0
DW 0xAA55              ; Boot signature
\end{lstlisting}

\section{Esercizi}

\begin{esercizio}[12.1]
Completare il progetto calcolatrice implementando read\_number e print\_number.
\end{esercizio}

\begin{esercizio}[12.2]
Modificare bubble sort per ordinamento decrescente.
\end{esercizio}

\begin{esercizio}[12.3]
Implementare Quick Sort ricorsivo.
\end{esercizio}

\begin{esercizio}[12.4]
Scrivere un bootloader che legge un settore dal disco.
\end{esercizio}

\begin{esercizio}[12.5]
Creare un semplice shell DOS-like con comandi DIR, TYPE, EXIT.
\end{esercizio}
