\chapter{Istruzioni Aritmetiche}

\section{Introduzione}

Le istruzioni aritmetiche eseguono operazioni matematiche su operandi interi. L'8086 supporta addizione, sottrazione, moltiplicazione, divisione e operazioni BCD.

\section{ADD e ADC --- Addition}

\subsection{ADD}

\begin{lstlisting}
ADD dest, src      ; dest = dest + src
ADD AX, BX         ; AX = AX + BX
ADD AL, 5          ; AL = AL + 5
ADD word ptr [BX], 100
\end{lstlisting}

\textbf{Flag modificati}: CF, ZF, SF, OF, PF, AF

\subsection{ADC --- Add with Carry}

\begin{lstlisting}
ADC dest, src      ; dest = dest + src + CF
\end{lstlisting}

Usato per addizioni multi-precisione (numeri > 16 bit).

\begin{esempio}
Somma a 32 bit (DX:AX + CX:BX):
\begin{lstlisting}
; DX:AX = 12345678h, CX:BX = 9ABCDEF0h
ADD AX, BX         ; Somma parte bassa
ADC DX, CX         ; Somma parte alta + carry
; Risultato in DX:AX = ACF13568h
\end{lstlisting}
\end{esempio}

\section{SUB e SBB --- Subtraction}

\subsection{SUB}

\begin{lstlisting}
SUB dest, src      ; dest = dest - src
SUB AX, 10
SUB CX, BX
\end{lstlisting}

\subsection{SBB --- Subtract with Borrow}

\begin{lstlisting}
SBB dest, src      ; dest = dest - src - CF
\end{lstlisting}

\section{INC e DEC}

\begin{lstlisting}
INC operando       ; operando = operando + 1
DEC operando       ; operando = operando - 1
\end{lstlisting}

\begin{nota}
INC e DEC \textbf{non modificano} il Carry Flag! Più veloci di ADD/SUB con 1.
\end{nota}

\section{MUL e IMUL --- Multiplication}

\subsection{MUL --- Unsigned Multiplication}

\begin{table}[h]
\centering
\begin{tabular}{lll}
\toprule
\textbf{Operando} & \textbf{Operazione} & \textbf{Risultato} \\
\midrule
8 bit & AL 	imes operando & AX \\
16 bit & AX 	imes operando & DX:AX \\
\bottomrule
\end{tabular}
\end{table}

\begin{lstlisting}
MOV AL, 10
MOV BL, 20
MUL BL             ; AX = AL 	imes BL = 200 (C8h)

MOV AX, 1000
MOV BX, 50
MUL BX             ; DX:AX = AX 	imes BX = 50000
\end{lstlisting}

\subsection{IMUL --- Signed Multiplication}

Stessa sintassi di MUL, ma interpreta gli operandi come numeri con segno.

\section{DIV e IDIV --- Division}

\subsection{DIV --- Unsigned Division}

\begin{table}[h]
\centering
\begin{tabular}{llll}
\toprule
\textbf{Dividendo} & \textbf{Divisore} & \textbf{Quoziente} & \textbf{Resto} \\
\midrule
AX & 8 bit & AL & AH \\
DX:AX & 16 bit & AX & DX \\
\bottomrule
\end{tabular}
\end{table}

\begin{lstlisting}
MOV AX, 100
MOV BL, 7
DIV BL             ; AL = 14 (quoziente), AH = 2 (resto)

MOV DX, 0
MOV AX, 50000
MOV BX, 100
DIV BX             ; AX = 500, DX = 0
\end{lstlisting}

\begin{attenzione}
Prima di DIV a 16 bit, azzerare DX se il dividendo è solo in AX!
\texttt{Divisione per zero} causa interrupt 0 (errore).
\end{attenzione}

\section{NEG e CMP}

\subsection{NEG --- Negate (Complemento a due)}

\begin{lstlisting}
NEG operando       ; operando = -operando
NEG AL             ; AL = -AL
\end{lstlisting}

\subsection{CMP --- Compare}

\begin{lstlisting}
CMP op1, op2       ; Esegue op1 - op2 e imposta flag (senza salvare)
\end{lstlisting}

Usato con salti condizionali:
\begin{lstlisting}
CMP AX, 10
JE equal           ; Salta se AX == 10
JG greater         ; Salta se AX > 10 (signed)
JA above           ; Salta se AX > 10 (unsigned)
\end{lstlisting}

\section{Esercizi}

\begin{esercizio}[5.1]
Calcolare $123 \times 45$ usando MUL.
\end{esercizio}

\begin{esercizio}[5.2]
Dividere 1000 per 7 e trovare quoziente e resto.
\end{esercizio}

\begin{esercizio}[5.3]
Sommare due numeri a 32 bit: 12345678h + ABCDEF01h.
\end{esercizio}

\begin{esercizio}[5.4]
Spiegare perché INC AX è preferibile ad ADD AX, 1.
\end{esercizio}

\begin{esercizio}[5.5]
Scrivere codice per calcolare il valore assoluto di un numero con segno in AX.
\end{esercizio}
