\chapter{Sistema di Interrupt}

\section{Introduzione agli Interrupt}

Gli interrupt permettono al processore di rispondere a eventi esterni o richiedere servizi di sistema.

\section{Tipi di Interrupt}

\subsection{Hardware Interrupts}

Generati da dispositivi esterni (tastiera, timer, disco).

\subsection{Software Interrupts}

Invocati da programma con istruzione \texttt{INT}.

\subsection{Exceptions}

Generati dal processore (divisione per zero, overflow).

\section{Interrupt Vector Table (IVT)}

Tabella a indirizzo 0000:0000 contenente 256 vettori di 4 byte (CS:IP).

\begin{lstlisting}
; Vettore interrupt n a indirizzo n*4
; Offset = n * 4
; Segmento = n * 4 + 2
\end{lstlisting}

\section{Interrupt DOS (INT 21h)}

\subsection{Funzione 01h: Input carattere con echo}

\begin{lstlisting}
MOV AH, 01h
INT 21h            ; AL = carattere letto
\end{lstlisting}

\subsection{Funzione 02h: Output carattere}

\begin{lstlisting}
MOV AH, 02h
MOV DL, 'A'
INT 21h            ; Stampa 'A'
\end{lstlisting}

\subsection{Funzione 09h: Stampa stringa}

\begin{lstlisting}
.DATA
msg DB 'Hello World$'

.CODE
MOV AH, 09h
MOV DX, OFFSET msg
INT 21h
\end{lstlisting}

\subsection{Funzione 4Ch: Termina programma}

\begin{lstlisting}
MOV AH, 4Ch
MOV AL, 0          ; Exit code
INT 21h
\end{lstlisting}

\section{Interrupt BIOS}

\subsection{INT 10h: Video Services}

\begin{lstlisting}
; Funzione 0Eh: Teletype output
MOV AH, 0Eh
MOV AL, 'X'
MOV BH, 0          ; Page number
INT 10h

; Funzione 00h: Set video mode
MOV AH, 00h
MOV AL, 03h        ; Modo testo 80x25
INT 10h
\end{lstlisting}

\subsection{INT 16h: Keyboard Services}

\begin{lstlisting}
; Funzione 00h: Legge tasto
MOV AH, 00h
INT 16h            ; AL = ASCII, AH = scan code
\end{lstlisting}

\subsection{INT 13h: Disk Services}

\begin{lstlisting}
; Funzione 02h: Legge settori
MOV AH, 02h
MOV AL, 1          ; Numero settori
MOV CH, 0          ; Cilindro
MOV CL, 1          ; Settore
MOV DH, 0          ; Testina
MOV DL, 00h        ; Drive (0=A:, 80h=HD)
MOV BX, OFFSET buffer
INT 13h
\end{lstlisting}

\section{Gestione Custom Interrupt}

\begin{lstlisting}
; Installare handler personalizzato per INT 60h
install_handler PROC
    CLI                ; Disabilita interrupt
    PUSH DS

    ; Calcola indirizzo IVT: 60h * 4 = 180h
    XOR AX, AX
    MOV DS, AX
    MOV BX, 180h

    ; Salva vecchio handler
    MOV AX, [BX]
    MOV old_offset, AX
    MOV AX, [BX+2]
    MOV old_segment, AX

    ; Installa nuovo handler
    MOV word ptr [BX], OFFSET my_handler
    MOV [BX+2], CS

    POP DS
    STI                ; Riabilita interrupt
    RET
install_handler ENDP

my_handler PROC
    ; Codice handler
    IRET
my_handler ENDP
\end{lstlisting}

\section{Esercizi}

\begin{esercizio}[10.1]
Scrivere un programma che legge un carattere e lo stampa 10 volte.
\end{esercizio}

\begin{esercizio}[10.2]
Implementare una funzione per stampare un numero decimale usando INT 21h.
\end{esercizio}

\begin{esercizio}[10.3]
Spiegare la differenza tra INT, CALL e JMP.
\end{esercizio}

\begin{esercizio}[10.4]
Scrivere codice per cambiare il colore del testo usando INT 10h.
\end{esercizio}

\begin{esercizio}[10.5]
Implementare un semplice handler per INT 60h che incrementa un contatore.
\end{esercizio}
