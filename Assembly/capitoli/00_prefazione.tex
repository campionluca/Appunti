\chapter*{Prefazione}
\addcontentsline{toc}{chapter}{Prefazione}

\section*{A chi è rivolto questo manuale}

Questo manuale di programmazione Assembly 8086 è pensato per studenti di istituti tecnici superiori che desiderano comprendere a fondo il funzionamento del microprocessore e acquisire competenze di programmazione a basso livello. Il corso fornisce una solida base teorica e pratica sulla programmazione Assembly, fondamentale per chi intende specializzarsi in sistemi embedded, reverse engineering, ottimizzazione del codice o sicurezza informatica.

\section*{Prerequisiti}

Prima di affrontare questo corso è necessario avere:

\begin{itemize}
    \item Conoscenza di base della programmazione (C o linguaggi simili)
    \item Comprensione dei sistemi di numerazione (binario, esadecimale)
    \item Nozioni di base di architettura dei computer
    \item Familiarità con i concetti di variabili, cicli e condizioni
\end{itemize}

\section*{Obiettivi del corso}

Al termine del corso lo studente sarà in grado di:

\begin{enumerate}
    \item \textbf{Comprendere l'architettura 8086}: registri, segmenti, modalità di indirizzamento
    \item \textbf{Scrivere programmi Assembly}: utilizzando il set completo di istruzioni
    \item \textbf{Gestire procedure e stack}: chiamate a funzione, passaggio parametri
    \item \textbf{Programmare interrupt}: gestione I/O e servizi DOS
    \item \textbf{Ottimizzare il codice}: tecniche per efficienza e performance
    \item \textbf{Debugging}: identificare e correggere errori in Assembly
\end{enumerate}

\section*{Strumenti utilizzati}

Per seguire il corso è necessario disporre di:

\begin{itemize}
    \item \textbf{Emulatore 8086}: EMU8086, DOSBox, NASM, MASM
    \item \textbf{Debugger}: Debug.exe, TD (Turbo Debugger), GDB
    \item \textbf{Editor di testo}: Notepad++, VS Code, TASM IDE
    \item \textbf{Sistema operativo}: Windows/Linux/macOS con emulatore DOS
\end{itemize}

\subsection*{EMU8086 (Consigliato)}

EMU8086 è un ambiente integrato che combina assembler, disassembler ed emulatore. Permette di:
\begin{itemize}
    \item Scrivere e assemblare codice Assembly 8086
    \item Eseguire programmi step-by-step
    \item Visualizzare registri e memoria in tempo reale
    \item Debuggare con breakpoint
\end{itemize}

\textbf{Download}: \url{https://emu8086-microprocessor-emulator.en.softonic.com/}

\subsection*{NASM (Netwide Assembler)}

NASM è un assembler open-source multi-piattaforma:

\begin{lstlisting}[language=bash, style=assemblystyle]
# Installazione su Linux
sudo apt-get install nasm

# Assemblaggio e linking
nasm -f elf program.asm -o program.o
ld -m elf_i386 program.o -o program

# Esecuzione
./program
\end{lstlisting}

\subsection*{DOSBox}

Per eseguire programmi .COM e .EXE su sistemi moderni:

\begin{lstlisting}[language=bash, style=assemblystyle]
# Installazione
sudo apt-get install dosbox  # Linux
brew install dosbox           # macOS

# Esecuzione
dosbox
mount c ~/assembly
c:
program.exe
\end{lstlisting}

\section*{Struttura del manuale}

Il libro è organizzato in 4 parti:

\begin{description}
    \item[Parte I --- Fondamenti (Cap. 1-3)] Architettura 8086, registri, memoria, modalità di indirizzamento
    \item[Parte II --- Set di Istruzioni (Cap. 4-7)] Istruzioni di trasferimento dati, aritmetiche, logiche, controllo di flusso
    \item[Parte III --- Tecniche Avanzate (Cap. 8-11)] Procedure, stack, stringhe, interrupt, I/O hardware
    \item[Parte IV --- Applicazioni (Cap. 12-13)] Progetti completi ed esercizi pratici
\end{description}

\section*{Convenzioni tipografiche}

\begin{itemize}
    \item \texttt{MOV AX, BX} --- Codice Assembly inline
    \item \textbf{Registro AX} --- Termini tecnici importanti
    \item \emph{little-endian} --- Concetti da memorizzare
\end{itemize}

\begin{nota}
I codici di esempio sono testati con EMU8086 e NASM. Alcune istruzioni potrebbero richiedere adattamenti per altri assembler.
\end{nota}

\section*{Metodologia didattica}

Ogni capitolo segue questa struttura:
\begin{enumerate}
    \item \textbf{Introduzione teorica}: Spiegazione del concetto
    \item \textbf{Esempi commentati}: Codice Assembly con spiegazioni dettagliate
    \item \textbf{Esercizi guidati}: Problemi con soluzione passo-passo
    \item \textbf{Esercizi proposti}: Sfide per consolidare l'apprendimento
    \item \textbf{Riepilogo}: Punti chiave del capitolo
\end{enumerate}

\section*{Supporto e risorse}

\begin{itemize}
    \item \textbf{Manuale Intel 8086}: \url{https://www.intel.com/}
    \item \textbf{OSDev Wiki}: \url{https://wiki.osdev.org/}
    \item \textbf{Assembly Language Tutorial}: \url{https://www.tutorialspoint.com/assembly_programming/}
    \item \textbf{Stack Overflow}: Tag [assembly] e [x86]
\end{itemize}

\section*{Ringraziamenti}

Si ringraziano tutti gli studenti che hanno contribuito con feedback e suggerimenti al miglioramento di questo materiale didattico.

\vspace{1cm}

\begin{flushright}
\textit{Buono studio!}\\
Gli autori
\end{flushright}
