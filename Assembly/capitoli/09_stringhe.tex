\chapter{Operazioni su Stringhe}

\section{Introduzione}

Le istruzioni su stringhe dell'8086 operano su sequenze di byte o word in memoria, usando SI (source) e DI (destination).

\section{Registri per Stringhe}

Le operazioni su stringhe dell'8086 richiedono l'utilizzo di registri specializzati coordinati tra loro. Il registro \textbf{SI} (Source Index) agisce come indice sorgente e opera principalmente nel segmento DS:SI per default, permettendo di leggere i dati dalla memoria. Il registro \textbf{DI} (Destination Index) funge da indice destinazione e utilizza sempre il segmento ES:DI, specificando dove i dati devono essere scritti o confrontati. Il registro \textbf{CX} (Count) svolge un ruolo cruciale nei prefissi REP, fungendo da contatore che specifica quante volte ripetere l'operazione su stringhe prima di terminare. Infine, il \textbf{DF} (Direction Flag) del registro FLAGS controlla la direzione dell'operazione: quando DF è 0 i registri SI e DI vengono incrementati (operazione in avanti), mentre quando DF è 1 vengono decrementati (operazione all'indietro).

\subsection{CLD e STD}

\begin{lstlisting}
CLD                ; Clear Direction Flag (incremento)
STD                ; Set Direction Flag (decremento)
\end{lstlisting}

\section{Istruzioni Base}

\subsection{MOVSB/MOVSW --- Move String}

\begin{lstlisting}
MOVSB              ; byte[ES:DI] = byte[DS:SI], aggiorna SI e DI
MOVSW              ; word[ES:DI] = word[DS:SI], aggiorna SI e DI di 2
\end{lstlisting}

\subsection{LODSB/LODSW --- Load String}

\begin{lstlisting}
LODSB              ; AL = byte[DS:SI], aggiorna SI
LODSW              ; AX = word[DS:SI], aggiorna SI di 2
\end{lstlisting}

\subsection{STOSB/STOSW --- Store String}

\begin{lstlisting}
STOSB              ; byte[ES:DI] = AL, aggiorna DI
STOSW              ; word[ES:DI] = AX, aggiorna DI di 2
\end{lstlisting}

\subsection{CMPSB/CMPSW --- Compare String}

\begin{lstlisting}
CMPSB              ; Confronta byte[DS:SI] con byte[ES:DI], imposta flag
CMPSW              ; Confronta word
\end{lstlisting}

\subsection{SCASB/SCASW --- Scan String}

\begin{lstlisting}
SCASB              ; Confronta AL con byte[ES:DI], aggiorna DI
SCASW              ; Confronta AX con word[ES:DI]
\end{lstlisting}

\section{Prefissi REP}

\subsection{REP}

\begin{lstlisting}
REP MOVSB          ; Ripete MOVSB per CX volte
REP STOSB          ; Ripete STOSB per CX volte
\end{lstlisting}

\subsection{REPE/REPZ}

\begin{lstlisting}
REPE CMPSB         ; Ripete finché uguale e CX≠0
\end{lstlisting}

\subsection{REPNE/REPNZ}

\begin{lstlisting}
REPNE SCASB        ; Ripete finché diverso e CX≠0
\end{lstlisting}

\section{Esempi Pratici}

\begin{esempio}
Copiare 100 byte da src a dest:
\begin{lstlisting}
.DATA
src DB 100 DUP(?)
dest DB 100 DUP(?)

.CODE
CLD
MOV SI, OFFSET src
MOV DI, OFFSET dest
MOV CX, 100
REP MOVSB
\end{lstlisting}
\end{esempio}

\begin{esempio}
Riempire buffer con zero:
\begin{lstlisting}
CLD
MOV DI, OFFSET buffer
MOV CX, 256
MOV AL, 0
REP STOSB          ; Azzera 256 byte
\end{lstlisting}
\end{esempio}

\begin{esempio}
Lunghezza stringa (cerca byte 0):
\begin{lstlisting}
strlen PROC
    PUSH DI
    MOV DI, OFFSET stringa
    MOV AL, 0
    MOV CX, 0FFFFh     ; Max length
    CLD
    REPNE SCASB        ; Cerca byte 0
    MOV AX, 0FFFFh
    SUB AX, CX
    DEC AX             ; AX = lunghezza
    POP DI
    RET
strlen ENDP
\end{lstlisting}
\end{esempio}

\begin{esempio}
Confrontare due stringhe:
\begin{lstlisting}
strcmp PROC
    CLD
    MOV SI, OFFSET str1
    MOV DI, OFFSET str2
    MOV CX, 10         ; Max 10 caratteri
    REPE CMPSB         ; Confronta finché uguali
    JE equal
    ; Diverse
    RET

equal:
    ; Uguali
    RET
strcmp ENDP
\end{lstlisting}
\end{esempio}

\section{Esercizi}

\begin{esercizio}[9.1]
Scrivere codice per invertire una stringa di 20 caratteri.
\end{esercizio}

\begin{esercizio}[9.2]
Implementare una funzione che conta le occorrenze di un carattere in una stringa.
\end{esercizio}

\begin{esercizio}[9.3]
Spiegare la differenza tra MOVSB e LODSB.
\end{esercizio}

\begin{esercizio}[9.4]
Convertire una stringa in maiuscolo usando LODSB e STOSB.
\end{esercizio}

\begin{esercizio}[9.5]
Implementare memcpy(dest, src, n) usando REP MOVSB.
\end{esercizio}
