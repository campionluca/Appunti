\chapter{Bibliografia e Risorse}

\section{Manuali Ufficiali}

\begin{enumerate}
    \item \textbf{Intel 8086/8088 User's Manual} \\
    Intel Corporation, 1981 \\
    Manuale ufficiale completo dell'architettura 8086. \\
    \url{https://www.intel.com/}

    \item \textbf{The Art of Assembly Language Programming} \\
    Randall Hyde \\
    Guida completa alla programmazione Assembly. \\
    \url{https://www.plantation-productions.com/Webster/}

    \item \textbf{PC Assembly Language} \\
    Paul A. Carter \\
    Introduzione moderna all'Assembly x86. \\
    \url{https://pacman128.github.io/pcasm/}
\end{enumerate}

\section{Libri di Testo}

\begin{enumerate}
    \item \textbf{Assembly Language for x86 Processors} \\
    Kip R. Irvine \\
    Pearson, 7th Edition, 2014 \\
    ISBN: 978-0133769401

    \item \textbf{The 8086/8088 Family: Design, Programming, and Interfacing} \\
    John E. Uffenbeck \\
    Prentice Hall, 2002 \\
    ISBN: 978-0130406859

    \item \textbf{Microprocessor 8086: Architecture, Programming and Interfacing} \\
    Mathur Sunil \\
    PHI Learning, 2010 \\
    ISBN: 978-8120340688
\end{enumerate}

\section{Risorse Online}

\subsection{Tutorial e Guide}

\begin{itemize}
    \item \textbf{Tutorialspoint Assembly Programming} \\
    \url{https://www.tutorialspoint.com/assembly_programming/}

    \item \textbf{OSDev Wiki} \\
    \url{https://wiki.osdev.org/} \\
    Risorsa completa per sviluppo sistemi operativi

    \item \textbf{x86 Assembly Guide (Yale)} \\
    \url{https://flint.cs.yale.edu/cs421/papers/x86-asm/asm.html}

    \item \textbf{8086 Instruction Reference} \\
    \url{http://www.gabrielececchetti.it/Teaching/CalcolatoriElettronici/Docs/i8086_instruction_set.pdf}
\end{itemize}

\subsection{Emulatori e Tools}

\begin{itemize}
    \item \textbf{EMU8086} \\
    \url{https://emu8086-microprocessor-emulator.en.softonic.com/} \\
    Emulatore completo con IDE integrato

    \item \textbf{NASM (Netwide Assembler)} \\
    \url{https://www.nasm.us/} \\
    Assembler open-source multi-piattaforma

    \item \textbf{MASM (Microsoft Macro Assembler)} \\
    \url{https://docs.microsoft.com/en-us/cpp/assembler/masm/} \\
    Assembler ufficiale Microsoft

    \item \textbf{DOSBox} \\
    \url{https://www.dosbox.com/} \\
    Emulatore DOS per eseguire programmi 16-bit

    \item \textbf{TASM (Turbo Assembler)} \\
    Borland Turbo Assembler, compatibile MASM

    \item \textbf{Debug.exe} \\
    Debugger DOS integrato, utile per disassemblaggio

    \item \textbf{TD (Turbo Debugger)} \\
    Debugger avanzato con interfaccia grafica
\end{itemize}

\subsection{Simulatori Online}

\begin{itemize}
    \item \textbf{Compiler Explorer (Godbolt)} \\
    \url{https://godbolt.org/} \\
    Compila C/C++ e mostra Assembly risultante

    \item \textbf{Online x86 Assembler} \\
    \url{https://defuse.ca/online-x86-assembler.htm}

    \item \textbf{8086 Emulator Online} \\
    \url{https://www.tutorialspoint.com/compile_assembly_online.php}
\end{itemize}

\section{Comunità e Forum}

\begin{itemize}
    \item \textbf{Stack Overflow} \\
    Tag: [assembly], [x86], [8086] \\
    \url{https://stackoverflow.com/questions/tagged/assembly}

    \item \textbf{Reddit r/asm} \\
    \url{https://www.reddit.com/r/asm/}

    \item \textbf{OSDev Forums} \\
    \url{https://forum.osdev.org/}

    \item \textbf{Flat Assembler Board} \\
    \url{https://board.flatassembler.net/}
\end{itemize}

\section{Video Corsi}

\begin{itemize}
    \item \textbf{Ben Eater --- Building an 8-bit Computer} \\
    \url{https://www.youtube.com/c/BenEater} \\
    Serie fantastica su architettura computer

    \item \textbf{Davy Wybiral --- Assembly Language Programming} \\
    \url{https://www.youtube.com/playlist?list=PLmxT2pVYo5LB5EzTPZGfFN0c2GDiSXgQe}

    \item \textbf{Code Vault --- x86 Assembly} \\
    \url{https://www.youtube.com/c/CodeVault}
\end{itemize}

\section{Progetti e Codice Esempio}

\begin{itemize}
    \item \textbf{GitHub --- Assembly Examples} \\
    \url{https://github.com/topics/assembly-8086}

    \item \textbf{OSDev Bare Bones} \\
    \url{https://wiki.osdev.org/Bare_Bones} \\
    Kernel minimale in Assembly

    \item \textbf{Bootlin (ex Compiler Explorer)} \\
    Confronto codice C vs Assembly
\end{itemize}

\section{Specifiche Hardware}

\begin{itemize}
    \item \textbf{Intel Datasheets} \\
    \url{https://www.intel.com/content/www/us/en/design/resource-design-center.html}

    \item \textbf{8259 PIC Datasheet} \\
    Programmable Interrupt Controller

    \item \textbf{8253/8254 PIT Datasheet} \\
    Programmable Interval Timer

    \item \textbf{8042 Keyboard Controller}

    \item \textbf{16550 UART Datasheet} \\
    Universal Asynchronous Receiver/Transmitter
\end{itemize}

\section{Reverse Engineering}

\begin{itemize}
    \item \textbf{IDA Pro} \\
    \url{https://hex-rays.com/ida-pro/} \\
    Disassembler e debugger professionale

    \item \textbf{Ghidra} \\
    \url{https://ghidra-sre.org/} \\
    Tool NSA open-source per reverse engineering

    \item \textbf{Radare2} \\
    \url{https://rada.re/} \\
    Framework open-source per RE

    \item \textbf{x64dbg} \\
    \url{https://x64dbg.com/} \\
    Debugger Windows open-source
\end{itemize}

\section{Sicurezza e Exploit Development}

\begin{itemize}
    \item \textbf{Exploit Education} \\
    \url{https://exploit.education/} \\
    Wargames per imparare exploit

    \item \textbf{Smashing The Stack For Fun And Profit} \\
    Aleph One, Phrack Magazine \\
    Articolo classico su buffer overflow

    \item \textbf{Shellcode Database} \\
    \url{http://shell-storm.org/shellcode/}
\end{itemize}

\section{Standard e Riferimenti}

\begin{itemize}
    \item \textbf{System V ABI} \\
    Application Binary Interface standard

    \item \textbf{Intel® 64 and IA-32 Architectures Software Developer Manuals} \\
    \url{https://software.intel.com/content/www/us/en/develop/articles/intel-sdm.html}

    \item \textbf{AMD64 Architecture Programmer's Manual}

    \item \textbf{NASM Documentation} \\
    \url{https://www.nasm.us/docs.php}
\end{itemize}

\section{Risorse in Italiano}

\begin{itemize}
    \item \textbf{Appunti di Calcolatori Elettronici} \\
    Università varie, reperibili online

    \item \textbf{Forum HTML.it --- Assembly} \\
    \url{https://forum.html.it/forum/assembly}

    \item \textbf{Wikibooks --- Programmazione x86 Assembly} \\
    \url{https://it.wikibooks.org/wiki/Programmazione_x86_Assembly}
\end{itemize}

\section{Riviste e Pubblicazioni}

\begin{itemize}
    \item \textbf{Dr. Dobb's Journal} \\
    Articoli storici su Assembly e ottimizzazione

    \item \textbf{Phrack Magazine} \\
    \url{http://phrack.org/} \\
    E-zine su hacking e programmazione low-level

    \item \textbf{MSDN Magazine} \\
    Articoli tecnici Microsoft
\end{itemize}

\section{Note Finali}

Questo elenco non è esaustivo ma rappresenta un punto di partenza solido per approfondire la programmazione Assembly 8086. Si consiglia di:

\begin{itemize}
    \item Praticare regolarmente scrivendo codice
    \item Partecipare a comunità online per confronto
    \item Studiare codice esistente (open source)
    \item Sperimentare con progetti personali
    \item Usare debugger per comprendere il funzionamento interno
\end{itemize}

\vspace{1cm}

\begin{center}
\textit{La programmazione Assembly è un'arte che richiede pratica costante.\\
Non scoraggiatevi di fronte alle difficoltà iniziali!}
\end{center}
