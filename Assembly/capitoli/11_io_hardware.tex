\chapter{I/O e Interfacciamento Hardware}

\section{Introduzione}

L'8086 comunica con periferiche tramite porte I/O a 16 bit (65536 porte possibili).

\section{Istruzioni I/O}

\subsection{IN e OUT}

\begin{lstlisting}
; I/O fisso (porta 0-255)
IN AL, porta
OUT porta, AL

; I/O variabile (porta in DX)
MOV DX, porta
IN AL, DX
OUT DX, AL
\end{lstlisting}

\section{Programmable Interrupt Controller (8259)

}

\subsection{Porte del PIC}

\begin{itemize}
    \item \textbf{20h}: Registro comando (Master PIC)
    \item \textbf{21h}: Registro maschera interrupt (IMR)
    \item \textbf{A0h}: Slave PIC comando
    \item \textbf{A1h}: Slave PIC maschera
\end{itemize}

\begin{lstlisting}
; Disabilitare interrupt tastiera (IRQ1)
IN AL, 21h         ; Legge IMR
OR AL, 02h         ; Setta bit 1
OUT 21h, AL        ; Scrive IMR

; Riabilitare
IN AL, 21h
AND AL, 0FDh       ; Azzera bit 1
OUT 21h, AL
\end{lstlisting}

\section{Programmable Interval Timer (8253/8254)}

\subsection{Porte del PIT}

\begin{itemize}
    \item \textbf{40h}: Canale 0 (system timer)
    \item \textbf{41h}: Canale 1
    \item \textbf{42h}: Canale 2 (speaker)
    \item \textbf{43h}: Control word
\end{itemize}

\begin{esempio}
Generare beep con speaker:
\begin{lstlisting}
; Frequenza = 1193180 / divisore
; Per 1000 Hz: divisore = 1193

MOV AL, 0B6h       ; Control word
OUT 43h, AL

MOV AX, 1193
OUT 42h, AL        ; Byte basso
MOV AL, AH
OUT 42h, AL        ; Byte alto

; Abilita speaker
IN AL, 61h
OR AL, 03h
OUT 61h, AL

; Attendi...

; Disabilita speaker
IN AL, 61h
AND AL, 0FCh
OUT 61h, AL
\end{lstlisting}
\end{esempio}

\section{Porta Parallela (LPT)}

\subsection{Porte LPT1}

\begin{itemize}
    \item \textbf{378h}: Data port (8 bit output)
    \item \textbf{379h}: Status port (5 bit input)
    \item \textbf{37Ah}: Control port
\end{itemize}

\begin{lstlisting}
; Invia byte alla stampante
MOV DX, 378h
MOV AL, 'A'
OUT DX, AL

; Strobe
MOV DX, 37Ah
MOV AL, 0Dh
OUT DX, AL
MOV AL, 0Ch
OUT DX, AL
\end{lstlisting}

\section{Porta Seriale (COM)}

\subsection{UART 8250/16550}

Porte COM1 (base 3F8h):
\begin{itemize}
    \item \textbf{3F8h}: Data register (RBR/THR)
    \item \textbf{3F9h}: Interrupt Enable Register (IER)
    \item \textbf{3FAh}: Interrupt ID Register (IIR)
    \item \textbf{3FBh}: Line Control Register (LCR)
    \item \textbf{3FCh}: Modem Control Register (MCR)
    \item \textbf{3FDh}: Line Status Register (LSR)
    \item \textbf{3FEh}: Modem Status Register (MSR)
\end{itemize}

\begin{esempio}
Inizializzare COM1 a 9600 baud, 8N1:
\begin{lstlisting}
; Divisore per 9600 baud: 115200 / 9600 = 12

MOV DX, 3FBh       ; LCR
MOV AL, 80h        ; DLAB = 1
OUT DX, AL

MOV DX, 3F8h       ; DLL
MOV AL, 12
OUT DX, AL

MOV DX, 3F9h       ; DLM
MOV AL, 0
OUT DX, AL

MOV DX, 3FBh       ; LCR
MOV AL, 03h        ; 8N1, DLAB=0
OUT DX, AL
\end{lstlisting}
\end{esempio}

\begin{esempio}
Trasmettere byte su COM1:
\begin{lstlisting}
send_byte PROC
    ; AL = byte da inviare
    PUSH DX
    PUSH AX

wait_ready:
    MOV DX, 3FDh   ; LSR
    IN AL, DX
    TEST AL, 20h   ; THRE bit
    JZ wait_ready

    POP AX
    MOV DX, 3F8h   ; THR
    OUT DX, AL

    POP DX
    RET
send_byte ENDP
\end{lstlisting}
\end{esempio}

\section{Tastiera (8042)}

\subsection{Porte tastiera}

\begin{itemize}
    \item \textbf{60h}: Data port
    \item \textbf{64h}: Status/Command port
\end{itemize}

\begin{lstlisting}
; Leggi scan code
IN AL, 60h

; Verifica buffer pieno
IN AL, 64h
TEST AL, 01h       ; Output buffer full
JZ no_key
IN AL, 60h         ; Leggi dato

no_key:
\end{lstlisting}

\section{VGA (Video Graphics Array)}

\subsection{Accesso diretto a memoria video}

Modo testo 80x25: buffer a B800:0000

\begin{lstlisting}
; Scrivere 'A' rosso su sfondo nero in posizione (0,0)
MOV AX, 0B800h
MOV ES, AX
MOV DI, 0
MOV AH, 04h        ; Attributo: rosso
MOV AL, 'A'
MOV ES:[DI], AX
\end{lstlisting}

\section{Esercizi}

\begin{esercizio}[11.1]
Scrivere codice per generare un beep di 2 secondi a 440 Hz (nota LA).
\end{esercizio}

\begin{esercizio}[11.2]
Implementare una funzione per leggere un carattere da COM1.
\end{esercizio}

\begin{esercizio}[11.3]
Scrivere una stringa direttamente in memoria video VGA.
\end{esercizio}

\begin{esercizio}[11.4]
Spiegare la differenza tra I/O mapped e memory mapped.
\end{esercizio}

\begin{esercizio}[11.5]
Implementare un delay preciso usando il PIT.
\end{esercizio}
