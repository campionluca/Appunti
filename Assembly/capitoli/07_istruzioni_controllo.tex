\chapter{Istruzioni di Controllo di Flusso}

\section{Jump Incondizionati}

\subsection{JMP --- Unconditional Jump}
\begin{lstlisting}
JMP label          ; Salta a label
JMP SHORT label    ; Salto corto (-128/+127 byte)
JMP NEAR label     ; Salto near (stesso segmento)
JMP FAR label      ; Salto far (cambia CS:IP)
\end{lstlisting}

\section{Jump Condizionati}

Basati su flag del registro FLAGS.

\subsection{Jump basati su Zero Flag}
\begin{lstlisting}
JE/JZ label        ; Jump if Equal/Zero (ZF=1)
JNE/JNZ label      ; Jump if Not Equal/Not Zero (ZF=0)
\end{lstlisting}

\subsection{Jump Unsigned}
\begin{lstlisting}
JA/JNBE label      ; Jump if Above (CF=0 AND ZF=0)
JAE/JNB label      ; Jump if Above or Equal (CF=0)
JB/JNAE label      ; Jump if Below (CF=1)
JBE/JNA label      ; Jump if Below or Equal (CF=1 OR ZF=1)
\end{lstlisting}

\subsection{Jump Signed}
\begin{lstlisting}
JG/JNLE label      ; Jump if Greater (ZF=0 AND SF=OF)
JGE/JNL label      ; Jump if Greater or Equal (SF=OF)
JL/JNGE label      ; Jump if Less (SF≠OF)
JLE/JNG label      ; Jump if Less or Equal (ZF=1 OR SF≠OF)
\end{lstlisting}

\subsection{Jump su singoli flag}
\begin{lstlisting}
JC label           ; Jump if Carry (CF=1)
JNC label          ; Jump if No Carry (CF=0)
JS label           ; Jump if Sign (SF=1, negativo)
JNS label          ; Jump if No Sign (SF=0, positivo)
JO label           ; Jump if Overflow (OF=1)
JNO label          ; Jump if No Overflow (OF=0)
JP/JPE label       ; Jump if Parity Even (PF=1)
JNP/JPO label      ; Jump if Parity Odd (PF=0)
\end{lstlisting}

\section{Loop Instructions}

\subsection{LOOP}
\begin{lstlisting}
LOOP label         ; Decrementa CX, salta se CX≠0
\end{lstlisting}

\begin{esempio}
Loop per 10 iterazioni:
\begin{lstlisting}
MOV CX, 10
loop_start:
    ; Corpo del loop
    DEC BX
    ; ...
LOOP loop_start    ; Ripete finché CX≠0
\end{lstlisting}
\end{esempio}

\subsection{LOOPE/LOOPZ}
\begin{lstlisting}
LOOPE label        ; Loop while Equal (decrementa CX, salta se ZF=1 AND CX≠0)
\end{lstlisting}

\subsection{LOOPNE/LOOPNZ}
\begin{lstlisting}
LOOPNE label       ; Loop while Not Equal (decrementa CX, salta se ZF=0 AND CX≠0)
\end{lstlisting}

\subsection{JCXZ}
\begin{lstlisting}
JCXZ label         ; Jump if CX=0 (senza modificare CX)
\end{lstlisting}

\section{Procedure}

\subsection{CALL}
\begin{lstlisting}
CALL procedura     ; Chiama procedura (PUSH IP, JMP)
CALL NEAR proc     ; Chiamata near
CALL FAR proc      ; Chiamata far (PUSH CS, PUSH IP)
\end{lstlisting}

\subsection{RET}
\begin{lstlisting}
RET                ; Ritorna (POP IP)
RET n              ; Ritorna e rimuove n byte dallo stack
RETF               ; Return far (POP IP, POP CS)
\end{lstlisting}

\begin{esempio}
Procedura semplice:
\begin{lstlisting}
somma PROC
    ADD AX, BX     ; AX = AX + BX
    RET
somma ENDP

; Chiamata:
MOV AX, 10
MOV BX, 20
CALL somma         ; AX = 30
\end{lstlisting}
\end{esempio}

\section{Interrupt}

\subsection{INT}
\begin{lstlisting}
INT numero         ; Invoca interrupt software
\end{lstlisting}

\begin{esempio}
DOS interrupt 21h (servizi DOS):
\begin{lstlisting}
MOV AH, 9          ; Funzione 9: stampa stringa
MOV DX, OFFSET msg
INT 21h

MOV AH, 4Ch        ; Funzione 4Ch: termina programma
INT 21h
\end{lstlisting}
\end{esempio}

\subsection{IRET}
\begin{lstlisting}
IRET               ; Return from interrupt (POP IP, POP CS, POP FLAGS)
\end{lstlisting}

\section{Tabella Jump Condizionati}

\begin{table}[h]
\centering
\small
\begin{tabular}{llp{5cm}}
\toprule
\textbf{Istruzione} & \textbf{Condizione} & \textbf{Uso} \\
\midrule
JE/JZ & ZF=1 & Uguale/Zero \\
JNE/JNZ & ZF=0 & Diverso/Non zero \\
JG/JNLE & ZF=0 AND SF=OF & Greater (signed) \\
JGE/JNL & SF=OF & Greater or Equal (signed) \\
JL/JNGE & SF≠OF & Less (signed) \\
JLE/JNG & ZF=1 OR SF≠OF & Less or Equal (signed) \\
JA/JNBE & CF=0 AND ZF=0 & Above (unsigned) \\
JAE/JNB & CF=0 & Above or Equal (unsigned) \\
JB/JNAE & CF=1 & Below (unsigned) \\
JBE/JNA & CF=1 OR ZF=1 & Below or Equal (unsigned) \\
\bottomrule
\end{tabular}
\caption{Jump condizionati principali}
\end{table}

\section{Esercizi}

\begin{esercizio}[7.1]
Scrivere un loop per sommare i numeri da 1 a 100.
\end{esercizio}

\begin{esercizio}[7.2]
Implementare una procedura che calcola il massimo tra AX e BX (risultato in AX).
\end{esercizio}

\begin{esercizio}[7.3]
Spiegare la differenza tra JG e JA. Quando si usa ciascuno?
\end{esercizio}

\begin{esercizio}[7.4]
Scrivere codice per cercare un byte in un array di 10 elementi.
\end{esercizio}

\begin{esercizio}[7.5]
Cosa succede se si chiama una procedura senza RET?
\end{esercizio}
