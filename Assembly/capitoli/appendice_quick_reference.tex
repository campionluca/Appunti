\chapter{Quick Reference}

\section{Registri 8086}

\subsection{General Purpose (16 bit)}

\begin{tabular}{ll}
\textbf{AX} & Accumulatore (AH + AL) \\
\textbf{BX} & Base (BH + BL) \\
\textbf{CX} & Contatore (CH + CL) \\
\textbf{DX} & Dati (DH + DL) \\
\end{tabular}

\subsection{Puntatore e Indice}

\begin{tabular}{ll}
\textbf{SP} & Stack Pointer \\
\textbf{BP} & Base Pointer \\
\textbf{SI} & Source Index \\
\textbf{DI} & Destination Index \\
\textbf{IP} & Instruction Pointer \\
\end{tabular}

\subsection{Segmento}

\begin{tabular}{ll}
\textbf{CS} & Code Segment \\
\textbf{DS} & Data Segment \\
\textbf{SS} & Stack Segment \\
\textbf{ES} & Extra Segment \\
\end{tabular}

\section{Flag Register}

\begin{tabular}{lll}
\textbf{Bit} & \textbf{Nome} & \textbf{Descrizione} \\
0 & CF & Carry Flag \\
2 & PF & Parity Flag \\
4 & AF & Auxiliary Carry \\
6 & ZF & Zero Flag \\
7 & SF & Sign Flag \\
8 & TF & Trap Flag \\
9 & IF & Interrupt Flag \\
10 & DF & Direction Flag \\
11 & OF & Overflow Flag \\
\end{tabular}

\section{Modalità Indirizzamento}

\begin{lstlisting}
MOV AX, 1234h          ; Immediate
MOV AX, BX             ; Register
MOV AX, [1000h]        ; Direct
MOV AX, [BX]           ; Register Indirect
MOV AX, [BX+4]         ; Based
MOV AX, [SI+4]         ; Indexed
MOV AX, [BX+SI+4]      ; Based Indexed
\end{lstlisting}

\section{Istruzioni per Categoria}

\subsection{Trasferimento Dati}

\begin{lstlisting}
MOV dest, src          ; Move
XCHG op1, op2          ; Exchange
LEA reg, mem           ; Load Effective Address
PUSH operando          ; Push to stack
POP operando           ; Pop from stack
IN AL/AX, porta        ; Input from port
OUT porta, AL/AX       ; Output to port
XLAT                   ; Translate
\end{lstlisting}

\subsection{Aritmetiche}

\begin{lstlisting}
ADD dest, src          ; Addition
ADC dest, src          ; Add with carry
SUB dest, src          ; Subtraction
SBB dest, src          ; Subtract with borrow
INC operando           ; Increment
DEC operando           ; Decrement
MUL operando           ; Unsigned multiply
IMUL operando          ; Signed multiply
DIV operando           ; Unsigned divide
IDIV operando          ; Signed divide
NEG operando           ; Negate
CMP op1, op2           ; Compare
\end{lstlisting}

\subsection{Logiche}

\begin{lstlisting}
AND dest, src          ; Logical AND
OR dest, src           ; Logical OR
XOR dest, src          ; Logical XOR
NOT operando           ; Logical NOT
TEST op1, op2          ; Test (AND without save)
SHL dest, count        ; Shift left
SHR dest, count        ; Shift right
SAL dest, count        ; Arithmetic shift left
SAR dest, count        ; Arithmetic shift right
ROL dest, count        ; Rotate left
ROR dest, count        ; Rotate right
RCL dest, count        ; Rotate through carry left
RCR dest, count        ; Rotate through carry right
\end{lstlisting}

\subsection{Controllo Flusso}

\begin{lstlisting}
JMP label              ; Unconditional jump
JE/JZ label            ; Jump if equal/zero
JNE/JNZ label          ; Jump if not equal/zero
JG/JNLE label          ; Jump if greater (signed)
JGE/JNL label          ; Jump if >= (signed)
JL/JNGE label          ; Jump if less (signed)
JLE/JNG label          ; Jump if <= (signed)
JA/JNBE label          ; Jump if above (unsigned)
JAE/JNB label          ; Jump if >= (unsigned)
JB/JNAE label          ; Jump if below (unsigned)
JBE/JNA label          ; Jump if <= (unsigned)
LOOP label             ; Loop (decrement CX)
CALL procedura         ; Call procedure
RET                    ; Return
INT numero             ; Software interrupt
\end{lstlisting}

\subsection{Stringhe}

\begin{lstlisting}
MOVSB/MOVSW            ; Move string
LODSB/LODSW            ; Load string
STOSB/STOSW            ; Store string
CMPSB/CMPSW            ; Compare string
SCASB/SCASW            ; Scan string
REP                    ; Repeat prefix
REPE/REPZ              ; Repeat while equal
REPNE/REPNZ            ; Repeat while not equal
CLD                    ; Clear direction flag
STD                    ; Set direction flag
\end{lstlisting}

\section{Interrupt DOS (INT 21h)}

\begin{lstlisting}
MOV AH, 01h
INT 21h                ; Input character (AL)

MOV AH, 02h
MOV DL, char
INT 21h                ; Output character

MOV AH, 09h
MOV DX, OFFSET msg
INT 21h                ; Print string ($-terminated)

MOV AH, 4Ch
INT 21h                ; Exit program
\end{lstlisting}

\section{Interrupt BIOS}

\subsection{INT 10h (Video)}

\begin{lstlisting}
MOV AH, 00h
MOV AL, mode
INT 10h                ; Set video mode

MOV AH, 0Eh
MOV AL, char
INT 10h                ; Teletype output
\end{lstlisting}

\subsection{INT 16h (Keyboard)}

\begin{lstlisting}
MOV AH, 00h
INT 16h                ; Read key (AL=ASCII, AH=scan)
\end{lstlisting}

\section{Struttura Programma .COM}

\begin{lstlisting}
ORG 100h               ; DOS PSP = 256 byte

start:
    MOV AH, 09h
    MOV DX, OFFSET msg
    INT 21h

    MOV AH, 4Ch
    INT 21h

msg DB 'Hello$'
\end{lstlisting}

\section{Struttura Programma .EXE}

\begin{lstlisting}
.MODEL SMALL
.STACK 100h

.DATA
msg DB 'Hello$'

.CODE
main PROC
    MOV AX, @DATA
    MOV DS, AX

    MOV AH, 09h
    MOV DX, OFFSET msg
    INT 21h

    MOV AH, 4Ch
    INT 21h
main ENDP

END main
\end{lstlisting}

\section{Calcolo Indirizzo Fisico}

\[
\text{Fisico} = (\text{Segmento} \times 16) + \text{Offset}
\]

\section{Tabella ASCII (estratto)}

\begin{tabular}{llll}
\textbf{Dec} & \textbf{Hex} & \textbf{Char} & \textbf{Descrizione} \\
0 & 00h & NUL & Null \\
7 & 07h & BEL & Bell \\
8 & 08h & BS & Backspace \\
9 & 09h & TAB & Horizontal Tab \\
10 & 0Ah & LF & Line Feed \\
13 & 0Dh & CR & Carriage Return \\
32 & 20h & SPACE & Spazio \\
48-57 & 30h-39h & 0-9 & Cifre \\
65-90 & 41h-5Ah & A-Z & Maiuscole \\
97-122 & 61h-7Ah & a-z & Minuscole \\
\end{tabular}

\section{Scan Code Tastiera (estratto)}

\begin{tabular}{ll}
\textbf{Scan Code} & \textbf{Tasto} \\
01h & ESC \\
1Ch & Enter \\
39h & Spazio \\
3Bh-44h & F1-F10 \\
48h & Freccia su \\
50h & Freccia giù \\
4Bh & Freccia sinistra \\
4Dh & Freccia destra \\
\end{tabular}

\section{Porte I/O Comuni}

\begin{tabular}{ll}
\textbf{Porta} & \textbf{Descrizione} \\
20h-21h & PIC (8259) Master \\
40h-43h & PIT (8253/8254) Timer \\
60h, 64h & Tastiera (8042) \\
3F8h-3FFh & COM1 (UART) \\
378h-37Fh & LPT1 (Parallela) \\
\end{tabular}
