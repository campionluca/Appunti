\chapter{Design Responsivo}
\label{cap:responsive}

\section{Mobile-first approach}

Il design responsivo inizia dai dispositivi mobili (320px) e scala verso schermi più grandi:

\begin{lstlisting}[language=CSS]
/* Mobile first: base per tutti */
body { font-size: 14px; }
.container { width: 100%; }

/* Tablet: 768px e oltre */
@media (min-width: 768px) {
  body { font-size: 16px; }
  .container { width: 90%; }
}

/* Desktop: 1024px e oltre */
@media (min-width: 1024px) {
  body { font-size: 18px; }
  .container { width: 80%; }
  max-width: 1200px;
}
\end{lstlisting}

\section{Meta tag viewport}

Essenziale per responsività:

\begin{lstlisting}[language=HTML]
<meta name="viewport" content="width=device-width, initial-scale=1.0" />
\end{lstlisting}

\begin{attenzione}
Senza il meta tag viewport, i dispositivi mobili mostreranno la pagina desktop rimpicciolita, non responsive.
\end{attenzione}

\section{Breakpoints standard}

\begin{itemize}
  \item \textbf{Mobile}: 320px - 480px
  \item \textbf{Tablet}: 768px - 1024px
  \item \textbf{Desktop}: 1024px+
  \item \textbf{Large}: 1920px+
\end{itemize}

\section{Immagini responsive}

\begin{lstlisting}[language=CSS]
img {
  max-width: 100%;
  height: auto;
}
\end{lstlisting}

\begin{lstlisting}[language=HTML]
<picture>
  <source media="(min-width: 1024px)" srcset="large.jpg" />
  <source media="(min-width: 768px)" srcset="medium.jpg" />
  <img src="small.jpg" alt="Descrizione" />
</picture>
\end{lstlisting}

\section{Esercizi}

\subsection{Esercizio 1 (Base)}
Crea una pagina con 2 colonne che diventa 1 colonna su mobile (max 768px).

\subsection{Esercizio 2 (Intermedio)}
Implementa 3 breakpoint diversi (320, 768, 1024) con layout e font-size che cambiano.

\subsection{Esercizio 3 (Avanzato)}
Crea un sito completo responsivo con navbar mobile-first, menu hamburger (CSS), e layout che adatta a tutti i dispositivi.

\section{Riepilogo}

\begin{itemize}
  \item Mobile-first: inizia da mobile e scala
  \item Meta viewport è obbligatorio
  \item Media queries: @media (min-width: ...)
  \item Breakpoint standard: 320, 768, 1024, 1920
  \item Immagini: max-width 100%, height auto
\end{itemize}
