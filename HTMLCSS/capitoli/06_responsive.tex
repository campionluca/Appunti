\chapter{Design Responsivo}
\label{cap:responsive}

\section{Introduzione al Responsive Web Design}

Il \textbf{Responsive Web Design} (RWD) è un approccio al web design che garantisce un'esperienza ottimale su tutti i dispositivi: desktop, tablet, smartphone. Coniato da Ethan Marcotte nel 2010, è diventato lo standard per lo sviluppo web moderno.

\subsection{Principi Fondamentali}

Il responsive design si basa su tre pilastri:

\begin{enumerate}
  \item \textbf{Griglia fluida}: Layout basato su percentuali invece di pixel fissi
  \item \textbf{Immagini flessibili}: Immagini che si adattano al contenitore (\texttt{max-width: 100\%})
  \item \textbf{Media queries}: Regole CSS che si applicano solo a determinate dimensioni schermo
\end{enumerate}

\subsection{Perché è Importante}

Il responsive design non è più una semplice opzione, ma una necessità imprescindibile nel web moderno. Dal punto di vista del traffico, oltre il 60\% del traffico web proviene da dispositivi mobili, il che significa che ignorare i mobile user significa perdere la maggior parte dei tuoi visitatori. Per quanto riguarda la SEO, è importante sapere che Google penalizza attivamente i siti che non sono mobile-friendly, influenzando negativamente il tuo ranking nei risultati di ricerca. L'esperienza utente è fondamentale: molti utenti abbandonano immediatamente siti non ottimizzati per mobile, causando perdite significative di conversioni. Dal punto di vista economico, il responsive design offre un grande vantaggio: puoi mantenere una sola soluzione web che funziona su tutti i dispositivi, invece di sviluppare e gestire due siti separati (uno per desktop e uno per mobile). Infine, la manutenzione è semplificata poiché hai un solo codebase da aggiornare e mantenere, riducendo i costi e i tempi di sviluppo.

\section{Meta Tag Viewport}

Il meta tag viewport è \textbf{obbligatorio} per il responsive design. Controlla come la pagina viene visualizzata su dispositivi mobili.

\subsection{Sintassi Base}

\begin{lstlisting}[language=HTML]
<!DOCTYPE html>
<html lang="it">
<head>
  <meta charset="UTF-8">
  <meta name="viewport" content="width=device-width, initial-scale=1.0">
  <title>Pagina Responsive</title>
</head>
<body>
  <!-- Contenuto -->
</body>
</html>
\end{lstlisting}

\subsection{Parametri Viewport}

\begin{description}
  \item[width=device-width] Larghezza = larghezza dispositivo (invece di 980px default)
  \item[initial-scale=1.0] Zoom iniziale 100\% (1:1)
  \item[maximum-scale=5.0] Zoom massimo permesso (default 5.0)
  \item[minimum-scale=0.25] Zoom minimo permesso (default 0.25)
  \item[user-scalable=yes] Permetti zoom (default yes, NON disabilitare!)
\end{description}

\begin{attenzione}
\textbf{NON usare mai \texttt{user-scalable=no} o \texttt{maximum-scale=1.0}}. Impedire lo zoom viola le linee guida di accessibilità WCAG e peggiora drasticamente l'esperienza utente per persone ipovedenti.
\end{attenzione}

\begin{lstlisting}[language=HTML]
<!-- ❌ MAL: Disabilita zoom -->
<meta name="viewport" content="width=device-width, initial-scale=1.0, user-scalable=no">

<!-- ✅ BENE: Permetti zoom -->
<meta name="viewport" content="width=device-width, initial-scale=1.0">
\end{lstlisting}

\section{Approcci al Responsive Design}

\subsection{Mobile-First Approach}

Design che parte da mobile (320px) e scala verso desktop. \textbf{Questo è l'approccio consigliato}. L'approccio mobile-first offre numerosi vantaggi rispetto all'alternativa tradizionale. Innanzitutto, ti costringe a dare priorità ai contenuti essenziali, riducendo il clutter e migliorando la chiarezza complessiva del sito. Dal punto di vista delle performance, i siti mobile-first hanno performance migliori su dispositivi mobili perché c'è meno CSS da sovrascrivere con media queries successive. L'approccio si basa sul concetto di progressive enhancement: parti da un'esperienza minimalista su mobile e aggiungi gradualmente funzionalità e complessità via via che gli schermi diventano più grandi. Infine, questo approccio è SEO-friendly poiché Google utilizza mobile-first indexing, il che significa che il motore di ricerca indicizza principalmente la versione mobile del tuo sito, premiando i designer che seguono questo approccio.

\begin{lstlisting}[language=CSS]
/* Mobile first: base per tutti i dispositivi */
body {
  font-size: 14px;
  padding: 1rem;
}

.container {
  width: 100%;
  margin: 0 auto;
}

.grid {
  display: grid;
  grid-template-columns: 1fr; /* 1 colonna su mobile */
  gap: 1rem;
}

/* Tablet: 768px e oltre (aggiungi complessità) */
@media (min-width: 768px) {
  body {
    font-size: 16px;
    padding: 2rem;
  }

  .container {
    width: 90%;
    max-width: 1200px;
  }

  .grid {
    grid-template-columns: repeat(2, 1fr); /* 2 colonne */
  }
}

/* Desktop: 1024px e oltre */
@media (min-width: 1024px) {
  body {
    font-size: 18px;
  }

  .grid {
    grid-template-columns: repeat(3, 1fr); /* 3 colonne */
  }
}

/* Large desktop: 1440px e oltre */
@media (min-width: 1440px) {
  .container {
    max-width: 1400px;
  }

  .grid {
    grid-template-columns: repeat(4, 1fr); /* 4 colonne */
  }
}
\end{lstlisting}

\subsection{Desktop-First Approach}

Design che parte da desktop e si riduce verso mobile (deprecato ma ancora in uso legacy).

\begin{lstlisting}[language=CSS]
/* Desktop first: base per desktop */
body {
  font-size: 18px;
}

.grid {
  display: grid;
  grid-template-columns: repeat(4, 1fr);
}

/* Tablet: 1024px e sotto (rimuovi complessità) */
@media (max-width: 1024px) {
  .grid {
    grid-template-columns: repeat(2, 1fr);
  }
}

/* Mobile: 768px e sotto */
@media (max-width: 768px) {
  body {
    font-size: 14px;
  }

  .grid {
    grid-template-columns: 1fr;
  }
}
\end{lstlisting}

\begin{nota}
Mobile-first usa \texttt{min-width} (scala verso l'alto), desktop-first usa \texttt{max-width} (scala verso il basso).
\end{nota}

\section{Media Queries}

Le media queries permettono di applicare CSS condizionalmente basandosi su caratteristiche dispositivo.

\subsection{Sintassi Base}

\begin{lstlisting}[language=CSS]
@media media-type and (media-feature) {
  /* Regole CSS */
}
\end{lstlisting}

\subsection{Media Types}

\begin{itemize}
  \item \texttt{all}: Tutti i dispositivi (default)
  \item \texttt{screen}: Schermi (desktop, tablet, mobile)
  \item \texttt{print}: Stampa
  \item \texttt{speech}: Screen reader (accessibilità)
\end{itemize}

\begin{lstlisting}[language=CSS]
/* CSS per stampa */
@media print {
  body {
    font-size: 12pt;
    color: black;
    background: white;
  }

  .no-print {
    display: none; /* Nascondi navbar, footer in stampa */
  }
}
\end{lstlisting}

\subsection{Media Features}

\subsubsection{Width-based}

\begin{lstlisting}[language=CSS]
/* Larghezza esatta */
@media (width: 768px) { }

/* Larghezza minima (mobile-first) */
@media (min-width: 768px) { }

/* Larghezza massima (desktop-first) */
@media (max-width: 768px) { }

/* Range */
@media (min-width: 768px) and (max-width: 1024px) { }
\end{lstlisting}

\subsubsection{Height-based}

\begin{lstlisting}[language=CSS]
/* Altezza minima (utile per header fissi) */
@media (min-height: 600px) {
  .sticky-header {
    position: fixed;
    top: 0;
  }
}
\end{lstlisting}

\subsubsection{Orientation}

\begin{lstlisting}[language=CSS]
/* Landscape (orizzontale): width > height */
@media (orientation: landscape) {
  .image {
    max-width: 50%;
  }
}

/* Portrait (verticale): height > width */
@media (orientation: portrait) {
  .image {
    max-width: 100%;
  }
}
\end{lstlisting}

\subsubsection{Aspect Ratio}

\begin{lstlisting}[language=CSS]
/* Ratio 16:9 */
@media (aspect-ratio: 16/9) {
  .video-container {
    /* Ottimizzato per 16:9 */
  }
}

/* Minimo aspect ratio */
@media (min-aspect-ratio: 16/9) { }
\end{lstlisting}

\subsubsection{Resolution (Retina Display)}

\begin{lstlisting}[language=CSS]
/* Schermi Retina (2x pixel density) */
@media (min-resolution: 2dppx) {
  .logo {
    background-image: url('logo@2x.png');
  }
}

/* Oppure con -webkit- prefix */
@media (-webkit-min-device-pixel-ratio: 2) {
  .logo {
    background-image: url('logo@2x.png');
  }
}
\end{lstlisting}

\subsubsection{Hover Capability}

\begin{lstlisting}[language=CSS]
/* Dispositivi con hover (mouse) */
@media (hover: hover) {
  .button:hover {
    background-color: #333;
  }
}

/* Dispositivi senza hover (touch) */
@media (hover: none) {
  .button:active {
    background-color: #333;
  }
}
\end{lstlisting}

\subsubsection{Pointer Precision}

\begin{lstlisting}[language=CSS]
/* Pointer fine (mouse) */
@media (pointer: fine) {
  .button {
    padding: 0.5rem 1rem;
    font-size: 14px;
  }
}

/* Pointer coarse (touch) */
@media (pointer: coarse) {
  .button {
    padding: 1rem 2rem; /* Bottoni più grandi per touch */
    font-size: 16px;
  }
}
\end{lstlisting}

\subsection{Combinare Media Queries}

\begin{lstlisting}[language=CSS]
/* AND: Entrambe le condizioni devono essere vere */
@media (min-width: 768px) and (max-width: 1024px) {
  /* Solo tablet */
}

/* OR: Almeno una condizione deve essere vera (usa virgola) */
@media (max-width: 768px), (orientation: portrait) {
  /* Mobile O qualsiasi dispositivo verticale */
}

/* NOT: Nega la condizione */
@media not print {
  /* Tutti tranne stampa */
}
\end{lstlisting}

\section{Breakpoints Standard}

Non esistono breakpoints "universali", ma queste sono dimensioni comuni:

\begin{table}[h]
\centering
\begin{tabular}{|l|l|l|}
\hline
\textbf{Dispositivo} & \textbf{Range} & \textbf{Media Query} \\
\hline
Mobile Small & 320px - 480px & Base (no media query) \\
\hline
Mobile Large & 481px - 767px & \texttt{@media (min-width: 481px)} \\
\hline
Tablet Portrait & 768px - 1023px & \texttt{@media (min-width: 768px)} \\
\hline
Tablet Landscape / Desktop Small & 1024px - 1279px & \texttt{@media (min-width: 1024px)} \\
\hline
Desktop & 1280px - 1439px & \texttt{@media (min-width: 1280px)} \\
\hline
Large Desktop & 1440px+ & \texttt{@media (min-width: 1440px)} \\
\hline
4K / UHD & 1920px+ & \texttt{@media (min-width: 1920px)} \\
\hline
\end{tabular}
\caption{Breakpoints Standard}
\end{table}

\subsection{Framework Breakpoints}

\textbf{Bootstrap 5}:
\begin{itemize}
  \item xs: < 576px (extra small)
  \item sm: ≥ 576px (small)
  \item md: ≥ 768px (medium)
  \item lg: ≥ 992px (large)
  \item xl: ≥ 1200px (extra large)
  \item xxl: ≥ 1400px (extra extra large)
\end{itemize}

\textbf{Tailwind CSS}:
\begin{itemize}
  \item sm: 640px
  \item md: 768px
  \item lg: 1024px
  \item xl: 1280px
  \item 2xl: 1536px
\end{itemize}

\section{Immagini Responsive}

\subsection{CSS max-width}

Tecnica base per immagini responsive:

\begin{lstlisting}[language=CSS]
img {
  max-width: 100%;  /* Mai più grande del container */
  height: auto;     /* Mantieni aspect ratio */
  display: block;   /* Rimuovi spazio sotto immagine */
}
\end{lstlisting}

\subsection{srcset e sizes}

HTML5 permette di specificare immagini diverse per risoluzioni diverse:

\begin{lstlisting}[language=HTML]
<!-- srcset: Immagini diverse per densità pixel -->
<img
  src="image-800.jpg"
  srcset="
    image-400.jpg 400w,
    image-800.jpg 800w,
    image-1200.jpg 1200w
  "
  sizes="(max-width: 768px) 100vw, 50vw"
  alt="Descrizione immagine"
/>
\end{lstlisting}

\textbf{Spiegazione}:
\begin{itemize}
  \item \texttt{srcset}: Lista di immagini con larghezze (400w, 800w, 1200w)
  \item \texttt{sizes}: Istruzioni su quando usare quale immagine
  \item \texttt{(max-width: 768px) 100vw}: Su mobile, immagine = 100\% viewport width
  \item \texttt{50vw}: Su desktop, immagine = 50\% viewport width
  \item Browser sceglie automaticamente l'immagine ottimale
\end{itemize}

\subsection{Picture Element}

Per controllo completo su art direction:

\begin{lstlisting}[language=HTML]
<picture>
  <!-- Desktop: immagine orizzontale -->
  <source
    media="(min-width: 1024px)"
    srcset="desktop-landscape.jpg"
  />

  <!-- Tablet: immagine quadrata -->
  <source
    media="(min-width: 768px)"
    srcset="tablet-square.jpg"
  />

  <!-- Mobile: immagine verticale -->
  <img
    src="mobile-portrait.jpg"
    alt="Descrizione"
  />
</picture>
\end{lstlisting}

\subsection{Formati Moderni (WebP, AVIF)}

\begin{lstlisting}[language=HTML]
<picture>
  <!-- AVIF: formato più efficiente (2020+) -->
  <source
    type="image/avif"
    srcset="image.avif"
  />

  <!-- WebP: fallback (supporto ampio) -->
  <source
    type="image/webp"
    srcset="image.webp"
  />

  <!-- JPEG: fallback finale -->
  <img src="image.jpg" alt="Descrizione" />
</picture>
\end{lstlisting}

\section{Tipografia Responsive}

\subsection{Fluid Typography con clamp()}

La funzione \texttt{clamp()} permette dimensioni fluide tra min e max:

\begin{lstlisting}[language=CSS]
/* clamp(min, preferred, max) */
h1 {
  font-size: clamp(2rem, 5vw, 4rem);
  /* Min 2rem, preferito 5% viewport, max 4rem */
}

body {
  font-size: clamp(1rem, 2.5vw, 1.25rem);
}

.container {
  padding: clamp(1rem, 5vw, 3rem);
}
\end{lstlisting}

\subsection{Viewport Units}

\begin{itemize}
  \item \texttt{vw}: 1\% della larghezza viewport
  \item \texttt{vh}: 1\% dell'altezza viewport
  \item \texttt{vmin}: 1\% della dimensione minore (min(vw, vh))
  \item \texttt{vmax}: 1\% della dimensione maggiore (max(vw, vh))
\end{itemize}

\begin{lstlisting}[language=CSS]
h1 {
  font-size: 5vw; /* 5% della larghezza schermo */
}

/* Hero full-screen */
.hero {
  height: 100vh; /* 100% altezza viewport */
}
\end{lstlisting}

\begin{attenzione}
Usa \texttt{clamp()} invece di solo \texttt{vw} per evitare testo troppo piccolo su mobile o troppo grande su desktop.
\end{attenzione}

\subsection{Media Queries per Font Size}

\begin{lstlisting}[language=CSS]
/* Mobile first */
body {
  font-size: 16px;
  line-height: 1.5;
}

h1 { font-size: 2rem; }   /* 32px */
h2 { font-size: 1.5rem; } /* 24px */

@media (min-width: 768px) {
  body {
    font-size: 18px;
  }

  h1 { font-size: 2.5rem; } /* 45px */
  h2 { font-size: 2rem; }   /* 36px */
}

@media (min-width: 1024px) {
  body {
    font-size: 20px;
  }

  h1 { font-size: 3rem; }   /* 60px */
  h2 { font-size: 2.25rem; } /* 45px */
}
\end{lstlisting}

\section{Touch-Friendly Design}

\subsection{Dimensioni Minime Touch Target}

Le linee guida WCAG raccomandano:
\begin{itemize}
  \item \textbf{Minimo}: 44×44 pixel (Apple), 48×48 pixel (Google)
  \item \textbf{Consigliato}: 48×48 pixel con 8px di spazio tra elementi
\end{itemize}

\begin{lstlisting}[language=CSS]
/* ❌ Troppo piccolo per touch */
.button {
  padding: 5px 10px;
  font-size: 12px;
}

/* ✅ Touch-friendly */
.button {
  min-height: 48px;
  min-width: 48px;
  padding: 12px 24px;
  font-size: 16px;
  margin: 8px; /* Spazio tra bottoni */
}
\end{lstlisting}

\subsection{Hover vs Touch}

\begin{lstlisting}[language=CSS]
/* Desktop: hover */
@media (hover: hover) {
  .card:hover {
    transform: translateY(-5px);
    box-shadow: 0 10px 20px rgba(0,0,0,0.2);
  }
}

/* Mobile: touch (usa :active invece di :hover) */
@media (hover: none) {
  .card:active {
    transform: scale(0.98);
  }
}
\end{lstlisting}

\subsection{Evitare Hover-only UI}

\begin{lstlisting}[language=CSS]
/* ❌ MAL: Menu visibile solo su hover (inaccessibile su touch) */
.menu {
  display: none;
}

.dropdown:hover .menu {
  display: block;
}

/* ✅ BENE: Menu toggle con JavaScript */
.menu.active {
  display: block;
}
\end{lstlisting}

\section{Layout Responsive Patterns}

\subsection{Column Drop}

Colonne che "cadono" una sotto l'altra su mobile:

\begin{lstlisting}[language=CSS]
.container {
  display: grid;
  gap: 20px;
}

/* Mobile: 1 colonna */
@media (min-width: 480px) {
  .container {
    grid-template-columns: repeat(2, 1fr);
  }
}

/* Tablet: 3 colonne */
@media (min-width: 768px) {
  .container {
    grid-template-columns: repeat(3, 1fr);
  }
}

/* Desktop: 4 colonne */
@media (min-width: 1024px) {
  .container {
    grid-template-columns: repeat(4, 1fr);
  }
}
\end{lstlisting}

\subsection{Layout Shifter}

Layout completamente diversi per dispositivi diversi:

\begin{lstlisting}[language=CSS]
/* Mobile: verticale */
.layout {
  display: flex;
  flex-direction: column;
}

/* Desktop: header full-width, 2 colonne sotto */
@media (min-width: 1024px) {
  .layout {
    display: grid;
    grid-template-columns: 300px 1fr;
    grid-template-areas:
      "header header"
      "sidebar main";
  }

  .header { grid-area: header; }
  .sidebar { grid-area: sidebar; }
  .main { grid-area: main; }
}
\end{lstlisting}

\subsection{Off Canvas}

Menu nascosto fuori schermo su mobile:

\begin{lstlisting}[language=CSS]
.sidebar {
  position: fixed;
  left: -300px; /* Nascosto */
  width: 300px;
  height: 100vh;
  transition: left 0.3s;
}

.sidebar.open {
  left: 0; /* Visibile quando attivato */
}

/* Desktop: sempre visibile */
@media (min-width: 1024px) {
  .sidebar {
    position: static;
    left: 0;
  }
}
\end{lstlisting}

\section{Testing Responsive}

\subsection{Browser DevTools}

Tutti i browser moderni hanno strumenti per testare responsive:

\textbf{Chrome DevTools}:
\begin{enumerate}
  \item F12 o Ctrl+Shift+I
  \item Toggle Device Toolbar (Ctrl+Shift+M)
  \item Seleziona dispositivo: iPhone, iPad, Pixel, Galaxy, ecc.
  \item Testa orientamento portrait/landscape
  \item Testa touch events
\end{enumerate}

\textbf{Firefox Responsive Design Mode}:
\begin{enumerate}
  \item Ctrl+Shift+M
  \item Seleziona preset o custom size
  \item Testa DPR (Device Pixel Ratio) per Retina
\end{enumerate}

\subsection{Device Testing}

\textbf{Test su dispositivi reali}:
\begin{itemize}
  \item Emulatori non sempre riproducono perfettamente
  \item Testa su almeno: 1 iPhone, 1 Android, 1 tablet
  \item Usa servizi come BrowserStack, LambdaTest (cloud testing)
\end{itemize}

\textbf{Online Tools}:
\begin{itemize}
  \item \textbf{Responsinator}: \url{http://www.responsinator.com/}
  \item \textbf{Am I Responsive}: \url{http://ami.responsivedesign.is/}
  \item \textbf{BrowserStack}: \url{https://www.browserstack.com/}
  \item \textbf{LambdaTest}: \url{https://www.lambdatest.com/}
\end{itemize}

\section{Performance Responsive}

\subsection{Lazy Loading Immagini}

\begin{lstlisting}[language=HTML]
<!-- Native lazy loading (Chrome 76+) -->
<img src="image.jpg" alt="Descrizione" loading="lazy" />
\end{lstlisting}

\subsection{Preload Critical Resources}

\begin{lstlisting}[language=HTML]
<head>
  <!-- Preload font critico -->
  <link rel="preload" href="font.woff2" as="font" type="font/woff2" crossorigin />

  <!-- Preload immagine hero -->
  <link rel="preload" href="hero.jpg" as="image" />
</head>
\end{lstlisting}

\subsection{Evitare Layout Shift (CLS)}

\begin{lstlisting}[language=HTML]
<!-- ❌ MAL: Immagine senza dimensioni causa layout shift -->
<img src="image.jpg" alt="Descrizione" />

<!-- ✅ BENE: Specifica aspect ratio -->
<img
  src="image.jpg"
  alt="Descrizione"
  width="800"
  height="600"
  style="aspect-ratio: 4/3;"
/>
\end{lstlisting}

\begin{lstlisting}[language=CSS]
/* CSS aspect ratio */
.image-container {
  aspect-ratio: 16 / 9;
  overflow: hidden;
}

.image-container img {
  width: 100%;
  height: 100%;
  object-fit: cover;
}
\end{lstlisting}

\section{Accessibility nel Responsive}

\subsection{Zoom e Scaling}

\begin{lstlisting}[language=HTML]
<!-- ✅ BENE: Permetti zoom -->
<meta name="viewport" content="width=device-width, initial-scale=1.0">

<!-- ❌ MAL: Disabilita zoom (viola WCAG) -->
<meta name="viewport" content="width=device-width, initial-scale=1.0, user-scalable=no">
\end{lstlisting}

\subsection{Focus Visibile}

\begin{lstlisting}[language=CSS]
/* Assicurati che focus sia visibile su tutti i dispositivi */
a:focus,
button:focus {
  outline: 2px solid #007bff;
  outline-offset: 2px;
}

/* Non rimuovere MAI outline senza alternativa */
/* ❌ button:focus { outline: none; } */
\end{lstlisting}

\subsection{Skip Links}

\begin{lstlisting}[language=HTML]
<body>
  <a href="#main-content" class="skip-link">Salta al contenuto</a>
  <header><!-- Navbar --></header>
  <main id="main-content"><!-- Contenuto --></main>
</body>
\end{lstlisting}

\begin{lstlisting}[language=CSS]
.skip-link {
  position: absolute;
  top: -40px;
  left: 0;
  background: #000;
  color: white;
  padding: 8px;
  text-decoration: none;
  z-index: 100;
}

.skip-link:focus {
  top: 0; /* Visibile quando riceve focus */
}
\end{lstlisting}

\section{Esercizi}

\subsection{Esercizio 1 (Base): Layout 2 Colonne Responsive}

Crea una pagina con 2 colonne che diventa 1 colonna su mobile:
\begin{itemize}
  \item Desktop (≥768px): 2 colonne (30\% sidebar, 70\% main)
  \item Mobile (<768px): 1 colonna verticale (sidebar sopra, main sotto)
  \item Aggiungi meta viewport
  \item Font size: 16px mobile, 18px desktop
\end{itemize}

\subsection{Esercizio 2 (Base): Immagini Responsive}

Crea una gallery di 6 immagini responsive:
\begin{itemize}
  \item Immagini con \texttt{max-width: 100\%}
  \item 3 colonne desktop, 2 tablet, 1 mobile
  \item Usa \texttt{srcset} con 3 dimensioni diverse
  \item Aspect ratio 16:9
\end{itemize}

\subsection{Esercizio 3 (Intermedio): Navbar Responsive}

Crea una navbar con menu hamburger:
\begin{itemize}
  \item Desktop: menu orizzontale sempre visibile
  \item Mobile: hamburger icon, menu verticale nascosto
  \item Transizione smooth per apertura menu
  \item Touch-friendly: bottoni min 48×48px
  \item Usa solo CSS (no JavaScript)
\end{itemize}

\subsection{Esercizio 4 (Intermedio): Tipografia Fluida}

Implementa tipografia fluida con \texttt{clamp()}:
\begin{itemize}
  \item H1: min 2rem, max 4rem
  \item H2: min 1.5rem, max 3rem
  \item Body: min 1rem, max 1.25rem
  \item Padding container: min 1rem, max 3rem
  \item Testa su 320px e 1920px
\end{itemize}

\subsection{Esercizio 5 (Avanzato): Dashboard Responsive}

Crea un dashboard responsive completo:
\begin{itemize}
  \item Desktop: header + sidebar (250px) + main + widget area
  \item Tablet: header + main + sidebar sotto
  \item Mobile: tutto verticale con sidebar off-canvas
  \item 4 breakpoint: 320px, 768px, 1024px, 1440px
  \item Grid CSS per layout
\end{itemize}

\subsection{Esercizio 6 (Avanzato): Art Direction con Picture}

Crea un hero section con art direction:
\begin{itemize}
  \item Desktop: immagine orizzontale 16:9
  \item Tablet: immagine quadrata 1:1
  \item Mobile: immagine verticale 9:16
  \item Usa \texttt{<picture>} con 3 immagini diverse
  \item Testo overlay responsive
  \item Formati: AVIF, WebP, JPEG (fallback)
\end{itemize}

\subsection{Esercizio 7 (Avanzato): Performance Optimization}

Ottimizza una pagina per performance:
\begin{itemize}
  \item Lazy loading per tutte le immagini non above-the-fold
  \item Preload font e immagine hero
  \item Aspect ratio per evitare layout shift
  \item Responsive images con srcset
  \item Media query per stampa
  \item Test Lighthouse: punteggio Performance >90
\end{itemize}

\section{Best Practices}

\subsection{Design}

\begin{itemize}
  \item \textbf{Mobile-first}: Parti da mobile, scala verso desktop
  \item \textbf{Content-first}: Priorità ai contenuti essenziali
  \item \textbf{Touch-friendly}: Bottoni min 48×48px, spazio 8px
  \item \textbf{Leggibilità}: Font size min 16px, line-height 1.5+
  \item \textbf{Contrast}: Ratio min 4.5:1 per testo normale
\end{itemize}

\subsection{Codice}

\begin{itemize}
  \item Usa \texttt{clamp()} per dimensioni fluide
  \item Preferisci \texttt{min-width} (mobile-first) a \texttt{max-width}
  \item Usa \texttt{em/rem} invece di \texttt{px} per scalabilità
  \item Raggruppa media queries per breakpoint
  \item Testa su dispositivi reali
\end{itemize}

\subsection{Performance}

\begin{itemize}
  \item Lazy loading immagini
  \item \texttt{srcset} e \texttt{<picture>} per immagini responsive
  \item Formati moderni: WebP, AVIF
  \item Aspect ratio per evitare layout shift
  \item Riduci dimensioni immagini (max 200KB per immagine)
\end{itemize}

\subsection{Accessibilità}

\begin{itemize}
  \item NON disabilitare zoom
  \item Focus visibile su tutti gli elementi interattivi
  \item Skip links per navigazione da tastiera
  \item ARIA labels per elementi interattivi
  \item Test con screen reader (NVDA, JAWS, VoiceOver)
\end{itemize}

\section{Checklist Responsive}

\subsection{HTML}

\begin{itemize}
  \item[$\square$] Meta viewport presente e corretto
  \item[$\square$] Immagini con \texttt{alt} descrittivo
  \item[$\square$] \texttt{srcset} per immagini critiche
  \item[$\square$] \texttt{loading="lazy"} per immagini sotto fold
  \item[$\square$] Semantic HTML (header, nav, main, article, footer)
\end{itemize}

\subsection{CSS}

\begin{itemize}
  \item[$\square$] Mobile-first approach
  \item[$\square$] Almeno 3 breakpoint (mobile, tablet, desktop)
  \item[$\square$] Tipografia fluida o responsive
  \item[$\square$] Immagini max-width 100\%
  \item[$\square$] Touch target min 48×48px
  \item[$\square$] Focus visibile
  \item[$\square$] Media query per stampa
\end{itemize}

\subsection{Testing}

\begin{itemize}
  \item[$\square$] Testato su Chrome DevTools (3+ dispositivi)
  \item[$\square$] Testato su almeno 1 dispositivo reale
  \item[$\square$] Testato landscape e portrait
  \item[$\square$] Testato zoom 200\%
  \item[$\square$] Lighthouse Performance >90
  \item[$\square$] Lighthouse Accessibility >90
\end{itemize}

\section{Risorse Utili}

\begin{itemize}
  \item \textbf{Responsive Design Checker}: \url{https://responsivedesignchecker.com/}
  \item \textbf{Google Mobile-Friendly Test}: \url{https://search.google.com/test/mobile-friendly}
  \item \textbf{Can I Use}: \url{https://caniuse.com/} (supporto feature browser)
  \item \textbf{MDN Media Queries}: \url{https://developer.mozilla.org/en-US/docs/Web/CSS/Media_Queries}
  \item \textbf{A Book Apart - Responsive Web Design}: Ethan Marcotte
  \item \textbf{This Is Responsive}: \url{https://bradfrost.github.io/this-is-responsive/}
\end{itemize}

\section{Riepilogo}

\subsection{Concetti Chiave}

\begin{itemize}
  \item \textbf{Responsive Web Design}: Un sito che si adatta a tutti i dispositivi
  \item \textbf{Mobile-first}: Approccio consigliato (parti da mobile, scala verso desktop)
  \item \textbf{Meta viewport}: Obbligatorio per responsive (\texttt{width=device-width, initial-scale=1.0})
  \item \textbf{Media queries}: CSS condizionale per breakpoint
  \item \textbf{Breakpoints}: 320px, 768px, 1024px, 1440px (standard)
\end{itemize}

\subsection{Tecniche Essenziali}

\begin{itemize}
  \item Immagini: \texttt{max-width: 100\%}, \texttt{srcset}, \texttt{<picture>}
  \item Tipografia: \texttt{clamp()}, viewport units, media queries
  \item Layout: Flexbox, Grid, mobile-first patterns
  \item Touch: Min 48×48px, \texttt{@media (hover: none)}
  \item Performance: Lazy loading, preload, aspect ratio
\end{itemize}

\subsection{Best Practices}

\begin{itemize}
  \item Mobile-first approach
  \item Content-first (priorità contenuti essenziali)
  \item Touch-friendly (48×48px min)
  \item Accessibilità (zoom, focus, skip links)
  \item Performance (lazy loading, srcset, WebP)
  \item Test su dispositivi reali
  \item NON disabilitare zoom (WCAG)
\end{itemize}
