\chapter{DOM Manipulation Avanzato}
\label{cap:javascript_dom}

\section{Document Object Model (DOM)}

Il DOM (Document Object Model) è una rappresentazione ad albero del documento HTML. Ogni elemento, attributo e testo è un nodo che può essere accesso e modificato con JavaScript.

\begin{nota}
Il DOM è l'interfaccia tra JavaScript e HTML. Quando modifichi il DOM con JavaScript, il browser aggiorna automaticamente la pagina visualizzata.
\end{nota}

\section{Traversing il DOM}

\subsection{Relazioni tra nodi}

\begin{lstlisting}[language=JavaScript]
// Accedere ai nodi parent, child e sibling
const element = document.getElementById("main");

element.parentElement;           // Elemento genitore
element.parentNode;              // Nodo genitore (può essere text node)
element.children;                // Array di elementi figli
element.childNodes;              // Array di tutti i nodi figli (inclusi text)
element.firstChild;              // Primo nodo figlio
element.lastChild;               // Ultimo nodo figlio
element.firstElementChild;       // Primo elemento figlio
element.lastElementChild;        // Ultimo elemento figlio
element.nextSibling;             // Prossimo nodo fratello
element.nextElementSibling;      // Prossimo elemento fratello
element.previousSibling;         // Nodo fratello precedente
element.previousElementSibling;  // Elemento fratello precedente
\end{lstlisting}

\subsection{Attraversare gerarchie complesse}

\begin{lstlisting}[language=HTML]
<div id="root">
  <header>
    <nav>
      <ul id="menu">
        <li><a href="#home">Home</a></li>
        <li><a href="#about">About</a></li>
      </ul>
    </nav>
  </header>
  <main id="content">
    <section class="card">...</section>
  </main>
</div>
\end{lstlisting}

\begin{lstlisting}[language=JavaScript]
const link = document.querySelector("a[href='#home']");

// Salire verso l'alto
const li = link.parentElement;           // <li>
const ul = li.parentElement;             // <ul id="menu">
const nav = ul.parentElement;            // <nav>
const header = nav.parentElement;        // <header>

// Discendere verso il basso
const root = document.getElementById("root");
const main = root.querySelector("main");
const firstSection = main.firstElementChild; // <section class="card">

// Metodo più diretto: closest()
const listItem = link.closest("li");  // Trova primo <li> antenato
const container = link.closest("[id]"); // Trova primo antenato con id
\end{lstlisting}

\section{Selettori CSS Avanzati}

\subsection{querySelector e querySelectorAll}

\begin{lstlisting}[language=JavaScript]
// querySelector restituisce il primo match
const button = document.querySelector("button.primary");
const item = document.querySelector("#main .item");
const input = document.querySelector("input[name='email']");

// querySelectorAll restituisce NodeList
const buttons = document.querySelectorAll("button");
const items = document.querySelectorAll(".card");
const inputs = document.querySelectorAll("input[type='text']");

// Convertire NodeList a Array
const itemsArray = Array.from(items);
const itemsArray2 = [...items]; // Usando spread operator
\end{lstlisting}

\subsection{Selettori CSS3}

\begin{lstlisting}[language=CSS]
/* Pseudo-class selectors */
li:first-child                   /* Primo figlio */
li:last-child                    /* Ultimo figlio */
li:nth-child(2)                  /* Secondo figlio */
li:nth-child(odd)                /* Figli dispari */
li:nth-of-type(2)                /* Secondo del suo tipo */

/* Attribute selectors */
input[type="email"]              /* Input email */
input[name^="user"]              /* Name inizia con "user" */
input[name$="code"]              /* Name finisce con "code" */
input[name*="temp"]              /* Name contiene "temp" */

/* Combinatori */
div.container > p                /* Figli diretti */
.parent p                        /* Tutti i discendenti */
h1 + p                           /* Elemento adiacente */
h1 ~ p                           /* Elementi fratelli seguenti */
\end{lstlisting}

\begin{lstlisting}[language=JavaScript]
// Usare selettori avanzati con querySelector
const firstItem = document.querySelector("li:first-child");
const emailInput = document.querySelector("input[type='email']");
const oddListItems = document.querySelectorAll("li:nth-child(odd)");

// Filtrare con selezione dinamica
const activeButtons = document.querySelectorAll("button:not(.disabled)");
const visibleCards = document.querySelectorAll(".card:not(.hidden)");
\end{lstlisting}

\section{Event Delegation}

\subsection{Problema: Troppi Event Listener}

\begin{lstlisting}[language=HTML]
<ul id="tasks">
  <li data-id="1">Compra latte <button class="delete">X</button></li>
  <li data-id="2">Studia JS <button class="delete">X</button></li>
  <li data-id="3">Esercizio <button class="delete">X</button></li>
  <!-- Migliaia di item... -->
</ul>
\end{lstlisting}

\begin{lstlisting}[language=JavaScript]
// SBAGLIATO: Aggiungere listener a ogni bottone
document.querySelectorAll(".delete").forEach(btn => {
  btn.addEventListener("click", handleDelete);
});
// Questo è inefficiente se aggiungi/rimuovi molti item!

// CORRETTO: Delegare al genitore
const taskList = document.getElementById("tasks");
taskList.addEventListener("click", (event) => {
  // Controllare se il click è su un bottone delete
  if (event.target.classList.contains("delete")) {
    const taskItem = event.target.parentElement;
    const taskId = taskItem.dataset.id;
    handleDelete(taskId);
    taskItem.remove();
  }
});
\end{lstlisting}

\begin{attenzione}
Event delegation funziona con tutti gli eventi che bubblano (click, change, input, etc.). Per eventi che non bubblano (focus, load), usa il terzo parametro \texttt{capture: true}.
\end{attenzione}

\subsection{Proprietà di event}

\begin{lstlisting}[language=JavaScript]
document.addEventListener("click", (event) => {
  console.log(event.target);         // Elemento cliccato
  console.log(event.currentTarget);  // Elemento con listener
  console.log(event.bubbles);        // Booleano: bubblare?
  console.log(event.preventDefault()); // Annullare azione
  console.log(event.stopPropagation()); // Bloccare bubbling
  console.log(event.type);           // "click"
});
\end{lstlisting}

\section{Element Creation e Manipulation}

\subsection{Creare e aggiungere elementi}

\begin{lstlisting}[language=JavaScript]
// Creare elementi
const div = document.createElement("div");
const p = document.createElement("p");
const button = document.createElement("button");

// Aggiungere classi e attributi
div.className = "card";
div.classList.add("featured");
div.setAttribute("data-type", "product");
button.textContent = "Click me";

// Aggiungere al DOM
document.body.appendChild(div);
div.appendChild(p);
div.appendChild(button);

// Altre operazioni di aggiunta
div.insertBefore(p, button);        // Inserire prima di elemento
div.insertAdjacentHTML("beforeend", "<span>Testo</span>");
\end{lstlisting}

\subsection{innerHTML vs textContent vs innerText}

\begin{lstlisting}[language=JavaScript]
const div = document.getElementById("content");

// innerHTML: Parsed come HTML
div.innerHTML = "<strong>Testo</strong>"; // Crea <strong>

// textContent: Puro testo
div.textContent = "<strong>Testo</strong>"; // Mostra stringa

// innerText: Testo visibile (considera CSS)
div.innerText = "Solo testo"; // Ignora elementi nascosti

// Preferire textContent per sicurezza
const userInput = getUserInput();
div.textContent = userInput; // Sicuro da XSS injection
\end{lstlisting}

\begin{nota}
Usa \texttt{textContent} per inserire testo utente. Usa \texttt{innerHTML} solo se hai totale controllo del contenuto. Mai usare \texttt{innerHTML} con dati da utenti!
\end{nota}

\subsection{Clonare elementi}

\begin{lstlisting}[language=HTML]
<template id="card-template">
  <div class="card">
    <h3 class="title"></h3>
    <p class="description"></p>
    <button class="delete">Elimina</button>
  </div>
</template>
\end{lstlisting}

\begin{lstlisting}[language=JavaScript]
// Clonare da template
const template = document.getElementById("card-template");
const clone = template.content.cloneNode(true);

// Modificare il clone
clone.querySelector(".title").textContent = "My Card";
clone.querySelector(".description").textContent = "Description here";

// Aggiungere al DOM
document.getElementById("container").appendChild(clone);
\end{lstlisting}

\subsection{Rimuovere elementi}

\begin{lstlisting}[language=JavaScript]
// Rimuovere elemento specifico
const element = document.getElementById("to-delete");
element.remove();
element.parentElement.removeChild(element); // Alternativa

// Svuotare un contenitore
const container = document.getElementById("container");
container.innerHTML = ""; // Opzione 1
while (container.firstChild) {
  container.removeChild(container.firstChild); // Opzione 2
}
\end{lstlisting}

\section{Manipolazione di Classi e Attributi}

\subsection{ClassList API}

\begin{lstlisting}[language=JavaScript]
const button = document.getElementById("save-btn");

// Aggiungere classe
button.classList.add("active");
button.classList.add("primary", "large"); // Più classi

// Rimuovere classe
button.classList.remove("disabled");

// Toggle: aggiungere se assente, rimuovere se presente
button.classList.toggle("selected");
button.classList.toggle("highlight", true); // Forza add
button.classList.toggle("highlight", false); // Forza remove

// Controllare classe
if (button.classList.contains("active")) {
  console.log("Button è attivo");
}

// Iterare le classi
button.classList.forEach(className => {
  console.log(className);
});
\end{lstlisting}

\subsection{Attributi personalizzati (data-*)}

\begin{lstlisting}[language=HTML]
<div class="product" data-id="123" data-price="29.99"
     data-in-stock="true">
  Nike Shoes
</div>
\end{lstlisting}

\begin{lstlisting}[language=JavaScript]
const product = document.querySelector(".product");

// Accedere ai data attributes
console.log(product.dataset.id);      // "123"
console.log(product.dataset.price);   // "29.99"
console.log(product.dataset.inStock); // "true"

// Modificare data attributes
product.dataset.id = "456";
product.dataset.available = "false";

// Accesso diretto agli attributi
product.setAttribute("data-discount", "10%");
console.log(product.getAttribute("data-discount")); // "10%"
\end{lstlisting}

\section{Esercizio Pratico: Dynamic Form Builder}

\subsection{Specifica}

Crea una app che permette all'utente di costruire un form dinamicamente:
- Aggiungere campi (text, email, select, checkbox)
- Personalizzare label e placeholder
- Rimuovere campi
- Generare il codice HTML del form
- Salvare il form in localStorage
- Resettare il form

\subsection{HTML/CSS/JavaScript}

\begin{lstlisting}[language=HTML]
<!DOCTYPE html>
<html lang="it">
<head>
  <meta charset="UTF-8">
  <meta name="viewport" content="width=device-width, initial-scale=1.0">
  <title>Dynamic Form Builder</title>
  <style>
    body {
      font-family: Arial, sans-serif;
      max-width: 1200px;
      margin: 0 auto;
      padding: 20px;
      background: #f5f5f5;
    }

    .container {
      display: grid;
      grid-template-columns: 1fr 1fr;
      gap: 30px;
    }

    .builder {
      background: white;
      padding: 20px;
      border-radius: 8px;
      box-shadow: 0 2px 8px rgba(0,0,0,0.1);
    }

    .preview {
      background: white;
      padding: 20px;
      border-radius: 8px;
      box-shadow: 0 2px 8px rgba(0,0,0,0.1);
    }

    h1 { margin-top: 0; }

    .controls {
      margin-bottom: 20px;
      display: flex;
      gap: 10px;
      flex-wrap: wrap;
    }

    button {
      padding: 8px 12px;
      background: #007bff;
      color: white;
      border: none;
      border-radius: 4px;
      cursor: pointer;
      font-size: 14px;
    }

    button:hover { background: #0056b3; }
    button.danger { background: #dc3545; }
    button.danger:hover { background: #c82333; }
    button.success { background: #28a745; }
    button.success:hover { background: #218838; }

    .field {
      margin-bottom: 15px;
      padding: 15px;
      background: #f9f9f9;
      border: 1px solid #ddd;
      border-radius: 4px;
      position: relative;
    }

    .field input, .field select, .field textarea {
      width: 100%;
      padding: 8px;
      margin: 5px 0;
      border: 1px solid #ccc;
      border-radius: 4px;
      font-size: 14px;
      box-sizing: border-box;
    }

    .field label {
      display: block;
      font-weight: bold;
      margin: 10px 0 5px 0;
    }

    .field button.remove {
      position: absolute;
      top: 10px;
      right: 10px;
      padding: 5px 10px;
      font-size: 12px;
    }

    #code-output {
      background: #f0f0f0;
      padding: 15px;
      border-radius: 4px;
      font-family: monospace;
      font-size: 12px;
      overflow-x: auto;
      white-space: pre-wrap;
      word-wrap: break-word;
      max-height: 300px;
      overflow-y: auto;
    }
  </style>
</head>
<body>
  <h1>Dynamic Form Builder</h1>

  <div class="container">
    <div class="builder">
      <h2>Builder</h2>
      <div class="controls">
        <button onclick="addTextField()">+ Text</button>
        <button onclick="addEmailField()">+ Email</button>
        <button onclick="addSelectField()">+ Select</button>
        <button onclick="addCheckboxField()">+ Checkbox</button>
        <button onclick="addTextareaField()">+ Textarea</button>
      </div>
      <div id="fields-container"></div>
      <div class="controls" style="margin-top: 20px;">
        <button class="success" onclick="generateCode()">Genera Codice</button>
        <button class="success" onclick="saveForm()">Salva Form</button>
        <button class="danger" onclick="resetForm()">Resetta</button>
      </div>
    </div>

    <div class="preview">
      <h2>Anteprima Form</h2>
      <form id="preview-form"></form>
      <h3>Codice HTML</h3>
      <div id="code-output"></div>
    </div>
  </div>

  <script>
    class FormBuilder {
      constructor() {
        this.fields = [];
        this.fieldId = 0;
        this.loadForm();
      }

      addField(type) {
        const id = this.fieldId++;
        const field = {
          id, type,
          label: `Campo ${id + 1}`,
          placeholder: "",
          options: type === "select" ? ["Opzione 1"] : [],
          required: false
        };
        this.fields.push(field);
        this.render();
      }

      removeField(id) {
        this.fields = this.fields.filter(f => f.id !== id);
        this.render();
      }

      updateField(id, updates) {
        const field = this.fields.find(f => f.id === id);
        if (field) {
          Object.assign(field, updates);
          this.render();
        }
      }

      render() {
        const container = document.getElementById("fields-container");
        container.innerHTML = "";

        this.fields.forEach(field => {
          const fieldDiv = document.createElement("div");
          fieldDiv.className = "field";
          fieldDiv.innerHTML = `
            <button class="remove danger" onclick="formBuilder.removeField(${field.id})">X</button>
            <label>Label:</label>
            <input type="text" value="${field.label}"
              onchange="formBuilder.updateField(${field.id}, {label: this.value})">

            <label>Placeholder:</label>
            <input type="text" value="${field.placeholder}"
              onchange="formBuilder.updateField(${field.id}, {placeholder: this.value})"
              ${field.type === "select" || field.type === "checkbox" ? "disabled" : ""}>

            ${field.type === "select" ? `
              <label>Opzioni (una per riga):</label>
              <textarea style="width: 100%; height: 80px;"
                onchange="formBuilder.updateField(${field.id}, {options: this.value.split('\\n').filter(o => o)})">
                ${field.options.join("\n")}
              </textarea>
            ` : ""}

            <label>
              <input type="checkbox" ${field.required ? "checked" : ""}
                onchange="formBuilder.updateField(${field.id}, {required: this.checked})">
              Campo obbligatorio
            </label>

            <small style="color: #666;">Tipo: ${field.type}</small>
          `;
          container.appendChild(fieldDiv);
        });

        this.updatePreview();
      }

      updatePreview() {
        const preview = document.getElementById("preview-form");
        preview.innerHTML = "";

        this.fields.forEach(field => {
          const label = document.createElement("label");
          label.textContent = field.label;
          label.style.display = "block";
          label.style.marginTop = "10px";
          label.style.fontWeight = "bold";
          preview.appendChild(label);

          if (field.type === "text" || field.type === "email") {
            const input = document.createElement("input");
            input.type = field.type;
            input.placeholder = field.placeholder;
            input.required = field.required;
            input.style.width = "100%";
            input.style.padding = "8px";
            input.style.margin = "5px 0 10px 0";
            input.style.border = "1px solid #ccc";
            input.style.borderRadius = "4px";
            input.style.boxSizing = "border-box";
            preview.appendChild(input);
          }
        });

        const submitBtn = document.createElement("button");
        submitBtn.type = "submit";
        submitBtn.textContent = "Invia";
        submitBtn.style.marginTop = "10px";
        submitBtn.style.padding = "10px 20px";
        preview.appendChild(submitBtn);
      }

      generateCode() {
        let html = '<form>\n';
        this.fields.forEach(field => {
          html += `  <label>${field.label}</label>\n`;
          html += `  <input type="${field.type}" `;
          html += `placeholder="${field.placeholder}" `;
          html += `${field.required ? 'required' : ''}>\n\n`;
        });
        html += '  <button type="submit">Invia</button>\n</form>';
        document.getElementById("code-output").textContent = html;
      }

      saveForm() {
        localStorage.setItem("form-builder", JSON.stringify(this.fields));
        alert("Form salvato!");
      }

      loadForm() {
        const saved = localStorage.getItem("form-builder");
        if (saved) {
          this.fields = JSON.parse(saved);
          this.fieldId = Math.max(...this.fields.map(f => f.id)) + 1;
          this.render();
        }
      }

      reset() {
        this.fields = [];
        this.fieldId = 0;
        this.render();
        document.getElementById("code-output").textContent = "";
      }
    }

    const formBuilder = new FormBuilder();

    function addTextField() { formBuilder.addField("text"); }
    function addEmailField() { formBuilder.addField("email"); }
    function addSelectField() { formBuilder.addField("select"); }
    function addCheckboxField() { formBuilder.addField("checkbox"); }
    function addTextareaField() { formBuilder.addField("textarea"); }
    function generateCode() { formBuilder.generateCode(); }
    function saveForm() { formBuilder.saveForm(); }
    function resetForm() { formBuilder.reset(); }
  </script>
</body>
</html>
\end{lstlisting}

\section{Esercizi}

\subsection{Esercizio 1 (Base)}
Scrivi una funzione che seleziona tutti i bottoni della pagina e aggiunge la classe "highlighted" solo a quelli che contengono la parola "submit".

\subsection{Esercizio 2 (Intermedio)}
Crea una funzione che con event delegation ascolta click su una lista e, quando clicchi su un item, colora il background di giallo. Quando clicchi di nuovo, lo rimuove.

\subsection{Esercizio 3 (Avanzato)}
Crea una tabella HTML dove ogni riga ha un bottone "Modifica" e "Elimina". Usa event delegation per gestire tutti i click e mostra in console quale riga è stata cliccata.

\section{Riepilogo}

\begin{itemize}
  \item Il DOM è una rappresentazione ad albero del documento HTML
  \item \texttt{parentElement}, \texttt{children}, \texttt{nextSibling} per traversing
  \item \texttt{querySelector} e \texttt{querySelectorAll} con selettori CSS avanzati
  \item Event delegation riduce il numero di listener quando hai molti elementi
  \item \texttt{createElement} per creare elementi dinamicamente
  \item \texttt{classList} per manipolare classi in modo pulito
  \item \texttt{dataset} per accedere a attributi data-*
  \item Template tag \texttt{<template>} perfetto per clonare elementi
  \item Preferisci \texttt{textContent} per sicurezza rispetto a \texttt{innerHTML}
\end{itemize}
