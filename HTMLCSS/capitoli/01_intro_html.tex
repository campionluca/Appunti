\chapter{Introduzione a HTML5}
\label{cap:intro_html}

\section{Cos'è HTML?}

HTML (HyperText Markup Language) è il linguaggio di marcatura per creare pagine web. Non è un linguaggio di programmazione come Java (vedi il capitolo sulla programmazione orientata agli oggetti nel corso di Quarta), ma un modo per strutturare e descrivere il contenuto usando tag e attributi.

\section{Struttura base di un documento HTML5}

\subsection{Anatomia dei tag HTML}
Un tag HTML è composto da tre parti fondamentali. La prima parte è il tag di apertura, che ha la forma \texttt{<nome>} e indica l'inizio dell'elemento. La seconda parte è il contenuto, che può essere semplice testo o altri tag annidati che formano una struttura gerarchica. Infine, la terza parte è il tag di chiusura, scritto come \texttt{</nome>}, che marca la fine dell'elemento e completa la struttura del tag.

Esempio di tag con contenuto:
\begin{lstlisting}[language=HTML]
<p>Questo è un paragrafo</p>
<div>
    <p>Questo paragrafo è annidato nel div</p>
</div>
\end{lstlisting}

\subsection{Tag auto-chiudenti (self-closing tags)}
Alcuni elementi HTML non hanno contenuto. Questi sono chiamati \textbf{tag auto-chiudenti} o \textbf{self-closing tags}. Per garantire la massima compatibilità con XHTML e XML, è importante chiudere esplicitamente questi tag:

\begin{lstlisting}[language=HTML]
<img src="immagine.jpg" alt="Una descrizione" />
<input type="text" placeholder="Inserisci testo" />
<meta charset="UTF-8" />
<link rel="stylesheet" href="stile.css" />
<area shape="rect" coords="0,0,100,100" href="link.html" />
\end{lstlisting}

\begin{nota}
Anche se HTML5 permette di omettere la chiusura (\texttt{/>}), è consigliabile usare sempre la sintassi XML/XHTML con chiusura esplicita. Questo approccio garantisce la compatibilità con XHTML e XML, rendendo il codice più consistente e leggibile. Inoltre, aiuta a evitare problemi con parser XML e facilita la migrazione tra diversi formati, assicurando che il tuo codice possa essere facilmente convertito o integrato con altri sistemi.
\end{nota}

\begin{nota}
Nota: In passato si utilizzavano i tag \texttt{<br>} e \texttt{<hr>} per creare interruzioni di riga e linee orizzontali. Queste pratiche sono ora considerate obsolete poiché mescolano presentazione e contenuto. È preferibile utilizzare CSS per gestire lo spazio e le separazioni visive.

Invece di \texttt{<br>}, puoi utilizzare approcci più semantici come elementi blocco appropriati (\texttt{<p>}, \texttt{<div>}) che creano naturalmente interruzioni di linea, oppure puoi controllare lo spazio verticale attraverso le proprietà CSS \texttt{margin} o \texttt{padding}, o ancora gestire la formattazione del testo con la proprietà CSS \texttt{white-space}.

Invece di \texttt{<hr>}, dovresti preferire soluzioni CSS come la proprietà \texttt{border} per creare linee divisorie, oppure utilizzare elementi semantici con stili appropriati, o ancora sfruttare elementi strutturali come \texttt{<section>} che forniscono una separazione logica e visiva del contenuto.
\end{nota}

\subsection{Commenti HTML}
I commenti in HTML sono utili per documentare il codice e vengono ignorati dal browser. La sintassi è:
\begin{lstlisting}[language=HTML]
<!-- Questo è un commento -->
<!-- 
    I commenti possono
    essere su più righe
-->
\end{lstlisting}

I commenti sono strumenti molto utili durante lo sviluppo. Puoi utilizzarli per documentare sezioni complesse del codice, spiegando il ragionamento dietro decisioni di design o struttura. Sono particolarmente utili quando hai bisogno di disabilitare temporaneamente parti di HTML durante il debugging o il refactoring. Infine, i commenti permettono di aggiungere note per altri sviluppatori che lavoreranno sul tuo codice, facilitando la comprensione e la collaborazione nel team.

\subsection{Attributi HTML}
Gli attributi forniscono informazioni aggiuntive ai tag e hanno questa sintassi:
\begin{lstlisting}[language=HTML]
<tag attributo="valore">contenuto</tag>
\end{lstlisting}

\subsection{Gli attributi id e class}
Gli attributi \texttt{id} e \texttt{class} sono fondamentali per identificare e raggruppare elementi HTML:

\begin{description}
  \item[\texttt{id}] È un identificatore \textbf{univoco} per un elemento nella pagina. Deve essere unico nel documento, il che significa che non possono coesistere due elementi con lo stesso \texttt{id}. Viene utilizzato per identificare un elemento specifico e risulta particolarmente utile per JavaScript quando hai bisogno di accedere a un elemento preciso. L'attributo \texttt{id} è anche fondamentale per creare collegamenti interni alla pagina (bookmark). Nel CSS, ti riferisci a un elemento con id usando il selettore \texttt{\#} (es: \texttt{\#header}).
  Esempio: \texttt{<div id="header">}

  \item[\texttt{class}] Definisce una o più categorie a cui l'elemento appartiene. Un elemento può avere multiple classi, separate da spazi, permettendoti di combinare diversi stili e comportamenti. Diversamente da \texttt{id}, la stessa classe può essere usata su molteplici elementi della pagina, rendendola ideale per applicare stili comuni a gruppi di elementi. Nel CSS, ti riferisci a una classe usando il selettore \texttt{.} (es: \texttt{.container}).
  Esempio: \texttt{<div class="container blue">}
\end{description}

\begin{nota}
Le differenze principali tra questi due attributi sono concettuali e pratiche. L'attributo \texttt{id} funziona come una carta d'identità: è unico e identifica un singolo elemento specifico nella pagina. Al contrario, l'attributo \texttt{class} funziona come un gruppo o categoria, e può essere condiviso tra più elementi, permettendo di applicare gli stessi stili o comportamenti a diversi componenti della pagina.
\end{nota}

\subsection{Attributi per collegamenti e risorse}

\subsubsection{L'attributo href}
L'attributo \texttt{href} (Hypertext REFerence) definisce il collegamento a una risorsa ed è utilizzato principalmente con il tag \texttt{<a>} per creare link. L'attributo \texttt{href} può contenere diversi tipi di valori, a seconda della destinazione desiderata. Puoi specificare URL assoluti (come \texttt{https://www.esempio.com}) per collegamenti a siti web esterni, oppure URL relativi (come \texttt{./immagini/foto.jpg}) per risorse all'interno del tuo progetto. È possibile anche creare collegamenti interni alla pagina stessa usando anchor link (come \texttt{\#sezione}), stabilire collegamenti email (come \texttt{mailto:esempio@email.com}) che permettono agli utenti di inviare messaggi, oppure creare link telefonici (come \texttt{tel:+390123456789}) che attivano applicazioni di composizione su dispositivi mobili.

Esempi di utilizzo:
\begin{lstlisting}[language=HTML]
<!-- Link a sito web -->
<a href="https://www.esempio.com/">Visita il sito</a>

<!-- Link a sezione della pagina -->
<a href="#introduzione">Vai all'introduzione</a>

<!-- Link a email -->
<a href="mailto:info@esempio.com">Contattaci</a>
\end{lstlisting}

\subsubsection{L'attributo src}
L'attributo \texttt{src} (source) specifica la fonte di una risorsa multimediale ed è utilizzato con diversi elementi HTML che caricano contenuti esterni. Troverai \texttt{src} utilizzato con il tag \texttt{<img>} per le immagini, con \texttt{<video>} per i filmati, con \texttt{<audio>} per i suoni, e con \texttt{<script>} per file JavaScript. Proprio come \texttt{href}, \texttt{src} può contenere diversi tipi di percorsi. Puoi specificare percorsi assoluti (come \texttt{https://cdn.esempio.com/immagine.jpg}) che puntano a risorse su server remoti, oppure percorsi relativi (come \texttt{./assets/logo.png}) che riferiscono risorse locali nel tuo progetto. In alcuni casi avanzati, puoi anche utilizzare Data URI (come \texttt{data:image/png;base64,...}) per incorporare direttamente il contenuto nel markup HTML.

Esempi di utilizzo:
\begin{lstlisting}[language=HTML]
<!-- Immagine da URL assoluto -->
<img src="https://esempio.com/foto.jpg" alt="Descrizione" />

<!-- Immagine da percorso relativo -->
<img src="./immagini/logo.png" alt="Logo" />

<!-- Video con sorgenti multiple -->
<video controls="controls">
    <source src="video.mp4" type="video/mp4" />
    <source src="video.webm" type="video/webm" />
</video>
\end{lstlisting}

\begin{nota}
Differenze principali:
\begin{itemize}
  \item \texttt{href}: per collegamenti e riferimenti (dove vuoi andare)
  \item \texttt{src}: per contenuti da incorporare (cosa vuoi mostrare)
\end{itemize}
\end{nota}

\subsection{Annidamento dei tag}
I tag possono essere annidati, cioè contenuti dentro altri tag, ma devono seguire regole precise per assicurare un markup valido. I tag devono essere chiusi nell'ordine inverso rispetto a quello in cui sono stati aperti, garantendo coerenza strutturale. Inoltre, ogni tag figlio deve essere completamente contenuto dentro il suo tag genitore, mantenendo una corretta gerarchia. Un accorgimento pratico e molto utile è utilizzare l'indentazione appropriata nel codice, che aiuta visivamente a comprendere la struttura e la profondità di annidamento degli elementi.

Esempio corretto:
\begin{lstlisting}[language=HTML]
<div class="container">
    <header>
        <h1>Titolo</h1>
        <nav>
            <a href="pagina1.html">Link 1</a>
            <a href="/about/index.html">Chi siamo</a>
            <a href="../contatti.html">Contatti</a>
        </nav>
    </header>
</div>
\end{lstlisting}

Esempio errato:
\begin{lstlisting}[language=HTML]
<div><p>Questo è <!-- NON FARE COSÌ -->
    <span>sbagliato</div></p></span>
\end{lstlisting}

\begin{attenzione}
L'annidamento errato può causare comportamenti imprevedibili nel rendering della pagina e non passa la validazione HTML.
\end{attenzione}

\subsection{Commenti HTML}
I commenti in HTML servono per documentare il codice e non vengono visualizzati nel browser, rendendoli ideali per note interne. I commenti sono particolarmente utili per spiegare sezioni complesse di codice, aiutando te e gli altri sviluppatori a comprendere il ragionamento dietro determinate scelte. Durante lo sviluppo e il debugging, i commenti permettono di disabilitare temporaneamente parti di codice senza eliminarle completamente. Puoi anche utilizzarli per organizzare il codice in sezioni logiche ben definite, facilitando la navigazione in file di grandi dimensioni. Infine, i commenti sono uno strumento essenziale di comunicazione con altri sviluppatori, permettendoti di condividere informazioni importanti e best practices all'interno del team.

Sintassi dei commenti:
\begin{lstlisting}[language=HTML]
<!-- Questo è un commento su una riga -->

<!-- 
  Questo è un commento
  su più righe
-->

<!-- Non usare -- dentro i commenti -->

<div class="header">
  <!-- TODO: aggiungere logo -->
  <h1>Titolo</h1>
</div>

<!--[if IE 8]>
  <link href="ie8only.css" rel="stylesheet">
<![endif]-->
\end{lstlisting}

\begin{nota}
I commenti HTML possono contenere qualsiasi testo tranne \texttt{--} doppio trattino, che può causare problemi di parsing.
\end{nota}

\begin{attenzione}
I commenti sono visibili nel codice sorgente della pagina. Non inserire informazioni sensibili (password, chiavi API, ecc.) nei commenti.
\end{attenzione}

\subsection{Struttura completa documento}
Ecco la struttura completa di un documento HTML5:

\begin{lstlisting}[language=HTML]
<!DOCTYPE html>
<html lang="it">
<head>
  <meta charset="UTF-8" />
  <meta name="viewport" content="width=device-width, initial-scale=1.0" />
  <title>Titolo della pagina</title>
  <link rel="stylesheet" href="style.css" />
</head>
<body>
  <header>
    <h1>Intestazione principale</h1>
  </header>
  <main>
    <p>Contenuto principale della pagina</p>
  </main>
  <footer>
    <p>&copy; 2025 - Tutti i diritti riservati</p>
  </footer>
</body>
</html>
\end{lstlisting}

\subsection{Elementi principali}

\begin{description}
  \item[\texttt{<!DOCTYPE html>}] Dichiara che il documento è HTML5, quindi deve rispettare le regole di questa versione del linguaggio.
  \item[\texttt{<html>}] Elemento radice del documento
  \item[\texttt{<head>}] Contiene metadati e link a risorse
  \item[\texttt{<body>}] Contiene il contenuto visibile
\end{description}

\section{Tag semantici}

HTML5 introduce tag semantici che descrivono il significato del contenuto:

\begin{lstlisting}[language=HTML]
<header>Intestazione del sito</header>
<nav>Barra di navigazione</nav>
<main>Contenuto principale</main>
<section>Una sezione tematica</section>
<article>Un articolo indipendente</article>
<aside>Barra laterale</aside>
<footer>Piè di pagina</footer>
\end{lstlisting}

\begin{attenzione}
La semantica HTML è importante per l'accessibilità: i lettori di schermo leggono il significato dei tag, non solo il testo.
\end{attenzione}

\section{Meta tag essenziali}

I meta tag sono elementi HTML che forniscono metadati sul documento HTML. Questi tag non sono visualizzati nella pagina ma contengono informazioni importanti per browser, motori di ricerca e altri servizi web.

\subsection{Sintassi generale}
Un meta tag ha questa struttura:
\begin{lstlisting}[language=HTML]
<meta name="nome" content="valore">
<!-- oppure -->
<meta http-equiv="nome" content="valore">
<!-- oppure -->
<meta charset="codifica">
\end{lstlisting}

\subsection{Meta tag fondamentali}
Ecco i meta tag più importanti e il loro utilizzo:

\begin{lstlisting}[language=HTML]
<meta charset="UTF-8" />
<meta name="viewport" content="width=device-width, initial-scale=1.0" />
<meta name="description" content="Breve descrizione della pagina" />
<meta name="keywords" content="html, css, web, sviluppo" />
\end{lstlisting}

\begin{description}
  \item[\texttt{charset}] Specifica la codifica dei caratteri del documento. UTF-8 è lo standard moderno che supporta caratteri internazionali e emoji.
  
  \item[\texttt{viewport}] Controlla come la pagina si adatta ai dispositivi mobili. Contiene parametri come \texttt{width=device-width}, che imposta la larghezza della pagina alla larghezza effettiva del dispositivo, garantendo che il layout si adatti correttamente su schermi di diverse dimensioni. Inoltre include \texttt{initial-scale=1.0}, che imposta il livello di zoom iniziale al 100%, assicurando che la pagina non sia ingrandita o rimpicciolita quando viene caricata per la prima volta.
  
  \item[\texttt{description}] Fornisce una breve descrizione della pagina (massimo 155 caratteri). Viene mostrata nei risultati di ricerca di Google.
  
  \item[\texttt{keywords}] Elenca parole chiave rilevanti per la pagina. Oggi ha meno importanza per il SEO ma è ancora utilizzato da alcuni motori di ricerca.
\end{description}

\begin{attenzione}
Il meta tag viewport è cruciale per il responsive design. Senza di esso, i siti mobile-friendly non funzioneranno correttamente sui dispositivi mobili.
\end{attenzione}

\begin{nota}
Il meta tag \texttt{viewport} è fondamentale per il responsive design. Senza di esso, i dispositivi mobili non visualizzeranno la pagina correttamente.
\end{nota}


\section{Riepilogo}

\begin{itemize}
  \item HTML è il linguaggio di marcatura del web
  \item \texttt{<!DOCTYPE html>} identifica la versione HTML5
  \item Tag semantici descrivono il significato del contenuto
  \item Meta tag sono fondamentali per accessibilità e responsive
  \item Sempre validare il codice HTML
\end{itemize}
