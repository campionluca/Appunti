\chapter{Introduzione a HTML5}
\label{cap:intro_html}

\section{Cos'è HTML?}

HTML (HyperText Markup Language) è il linguaggio di marcatura per creare pagine web. Non è un linguaggio di programmazione come Java (vedi il capitolo sulla programmazione orientata agli oggetti nel corso di Quarta), ma un modo per strutturare e descrivere il contenuto usando tag e attributi.

\section{Struttura base di un documento HTML5}

\subsection{Anatomia dei tag HTML}
Un tag HTML è composto da:
\begin{itemize}
  \item \textbf{Tag di apertura}: \texttt{<nome>}
  \item \textbf{Contenuto}: testo o altri tag annidati
  \item \textbf{Tag di chiusura}: \texttt{</nome>}
\end{itemize}

Esempio di tag con contenuto:
\begin{lstlisting}[language=HTML]
<p>Questo è un paragrafo</p>
<div>
    <p>Questo paragrafo è annidato nel div</p>
</div>
\end{lstlisting}

\subsection{Tag auto-chiudenti (self-closing tags)}
Alcuni elementi HTML non hanno contenuto. Questi sono chiamati \textbf{tag auto-chiudenti} o \textbf{self-closing tags}. Per garantire la massima compatibilità con XHTML e XML, è importante chiudere esplicitamente questi tag:

\begin{lstlisting}[language=HTML]
<img src="immagine.jpg" alt="Una descrizione" />
<input type="text" placeholder="Inserisci testo" />
<meta charset="UTF-8" />
<link rel="stylesheet" href="stile.css" />
<area shape="rect" coords="0,0,100,100" href="link.html" />
\end{lstlisting}

\begin{nota}
Anche se HTML5 permette di omettere la chiusura (\texttt{/>}), è consigliabile usare sempre la sintassi XML/XHTML con chiusura esplicita per:
\begin{itemize}
  \item Garantire la compatibilità con XHTML e XML
  \item Rendere il codice più consistente e leggibile
  \item Evitare problemi con parser XML
  \item Facilitare la migrazione tra diversi formati
\end{itemize}
\end{nota}

\begin{nota}
Nota: In passato si utilizzavano i tag \texttt{<br>} e \texttt{<hr>} per creare interruzioni di riga e linee orizzontali. Queste pratiche sono ora considerate obsolete poiché mescolano presentazione e contenuto. È preferibile utilizzare CSS per gestire lo spazio e le separazioni visive.

Invece di \texttt{<br>}, usa:
\begin{itemize}
  \item Elementi blocco appropriati (\texttt{<p>}, \texttt{<div>})
  \item CSS \texttt{margin} o \texttt{padding}
  \item Proprietà CSS \texttt{white-space}
\end{itemize}

Invece di \texttt{<hr>}, usa:
\begin{itemize}
  \item CSS \texttt{border}
  \item Elementi semantici con stili appropriati
  \item \texttt{<section>} o altri elementi strutturali
\end{itemize}
\end{nota}

\subsection{Commenti HTML}
I commenti in HTML sono utili per documentare il codice e vengono ignorati dal browser. La sintassi è:
\begin{lstlisting}[language=HTML]
<!-- Questo è un commento -->
<!-- 
    I commenti possono
    essere su più righe
-->
\end{lstlisting}

I commenti sono utili per:
\begin{itemize}
  \item Documentare sezioni complesse del codice
  \item Disabilitare temporaneamente parti di HTML
  \item Aggiungere note per altri sviluppatori
\end{itemize}

\subsection{Attributi HTML}
Gli attributi forniscono informazioni aggiuntive ai tag e hanno questa sintassi:
\begin{lstlisting}[language=HTML]
<tag attributo="valore">contenuto</tag>
\end{lstlisting}

\subsection{Gli attributi id e class}
Gli attributi \texttt{id} e \texttt{class} sono fondamentali per identificare e raggruppare elementi HTML:

\begin{description}
  \item[\texttt{id}] È un identificatore \textbf{univoco} per un elemento nella pagina:
  \begin{itemize}
    \item Deve essere unico nel documento (non ci possono essere due elementi con lo stesso \texttt{id})
    \item Viene usato per identificare un elemento specifico
    \item Utile per JavaScript e per collegamenti interni alla pagina
    \item Si riferisce con \texttt{\#} nel CSS (es: \texttt{\#header})
  \end{itemize}
  Esempio: \texttt{<div id="header">}

  \item[\texttt{class}] Definisce una o più categorie a cui l'elemento appartiene:
  \begin{itemize}
    \item Un elemento può avere multiple classi (separate da spazi)
    \item La stessa classe può essere usata su più elementi
    \item Utile per applicare stili comuni a gruppi di elementi
    \item Si riferisce con \texttt{.} nel CSS (es: \texttt{.container})
  \end{itemize}
  Esempio: \texttt{<div class="container blue">}
\end{description}

\begin{nota}
Differenze principali:
\begin{itemize}
  \item \texttt{id}: come la carta d'identità, è unico e identifica un singolo elemento
  \item \texttt{class}: come un gruppo o categoria, può essere condiviso tra più elementi
\end{itemize}
\end{nota}

\subsection{Attributi per collegamenti e risorse}

\subsubsection{L'attributo href}
L'attributo \texttt{href} (Hypertext REFerence) definisce il collegamento a una risorsa:

\begin{itemize}
  \item Usato principalmente con il tag \texttt{<a>} per creare link
  \item Può contenere:
    \begin{itemize}
      \item URL assoluti: \texttt{https://www.esempio.com}
      \item URL relativi: \texttt{./immagini/foto.jpg}
      \item Collegamenti interni: \texttt{\#sezione}
      \item Collegamenti email: \texttt{mailto:esempio@email.com}
      \item Collegamenti telefonici: \texttt{tel:+390123456789}
    \end{itemize}
\end{itemize}

Esempi di utilizzo:
\begin{lstlisting}[language=HTML]
<!-- Link a sito web -->
<a href="https://www.esempio.com/">Visita il sito</a>

<!-- Link a sezione della pagina -->
<a href="#introduzione">Vai all'introduzione</a>

<!-- Link a email -->
<a href="mailto:info@esempio.com">Contattaci</a>
\end{lstlisting}

\subsubsection{L'attributo src}
L'attributo \texttt{src} (source) specifica la fonte di una risorsa multimediale:

\begin{itemize}
  \item Utilizzato con elementi che caricano contenuti esterni:
    \begin{itemize}
      \item \texttt{<img>} per immagini
      \item \texttt{<video>} per filmati
      \item \texttt{<audio>} per suoni
      \item \texttt{<script>} per JavaScript
    \end{itemize}
  \item Può contenere:
    \begin{itemize}
      \item Percorsi assoluti: \texttt{https://cdn.esempio.com/immagine.jpg}
      \item Percorsi relativi: \texttt{./assets/logo.png}
      \item Data URI: \texttt{data:image/png;base64,...}
    \end{itemize}
\end{itemize}

Esempi di utilizzo:
\begin{lstlisting}[language=HTML]
<!-- Immagine da URL assoluto -->
<img src="https://esempio.com/foto.jpg" alt="Descrizione" />

<!-- Immagine da percorso relativo -->
<img src="./immagini/logo.png" alt="Logo" />

<!-- Video con sorgenti multiple -->
<video controls="controls">
    <source src="video.mp4" type="video/mp4" />
    <source src="video.webm" type="video/webm" />
</video>
\end{lstlisting}

\begin{nota}
Differenze principali:
\begin{itemize}
  \item \texttt{href}: per collegamenti e riferimenti (dove vuoi andare)
  \item \texttt{src}: per contenuti da incorporare (cosa vuoi mostrare)
\end{itemize}
\end{nota}

\subsection{Annidamento dei tag}
I tag possono essere annidati (contenuti dentro altri tag) ma devono seguire regole precise:
\begin{itemize}
  \item I tag devono essere chiusi nell'ordine inverso di apertura
  \item Ogni tag figlio deve essere completamente contenuto nel genitore
  \item L'indentazione aiuta a visualizzare la struttura
\end{itemize}

Esempio corretto:
\begin{lstlisting}[language=HTML]
<div class="container">
    <header>
        <h1>Titolo</h1>
        <nav>
            <a href="pagina1.html">Link 1</a>
            <a href="/about/index.html">Chi siamo</a>
            <a href="../contatti.html">Contatti</a>
        </nav>
    </header>
</div>
\end{lstlisting}

Esempio errato:
\begin{lstlisting}[language=HTML]
<div><p>Questo è <!-- NON FARE COSÌ -->
    <span>sbagliato</div></p></span>
\end{lstlisting}

\begin{attenzione}
L'annidamento errato può causare comportamenti imprevedibili nel rendering della pagina e non passa la validazione HTML.
\end{attenzione}

\subsection{Commenti HTML}
I commenti in HTML servono per documentare il codice e non vengono visualizzati nel browser. Sono utili per:
\begin{itemize}
  \item Spiegare sezioni complesse di codice
  \item Disabilitare temporaneamente parti di codice
  \item Organizzare il codice in sezioni logiche
  \item Comunicare con altri sviluppatori
\end{itemize}

Sintassi dei commenti:
\begin{lstlisting}[language=HTML]
<!-- Questo è un commento su una riga -->

<!-- 
  Questo è un commento
  su più righe
-->

<!-- Non usare -- dentro i commenti -->

<div class="header">
  <!-- TODO: aggiungere logo -->
  <h1>Titolo</h1>
</div>

<!--[if IE 8]>
  <link href="ie8only.css" rel="stylesheet">
<![endif]-->
\end{lstlisting}

\begin{nota}
I commenti HTML possono contenere qualsiasi testo tranne \texttt{--} doppio trattino, che può causare problemi di parsing.
\end{nota}

\begin{attenzione}
I commenti sono visibili nel codice sorgente della pagina. Non inserire informazioni sensibili (password, chiavi API, ecc.) nei commenti.
\end{attenzione}

\subsection{Struttura completa documento}
Ecco la struttura completa di un documento HTML5:

\begin{lstlisting}[language=HTML]
<!DOCTYPE html>
<html lang="it">
<head>
  <meta charset="UTF-8" />
  <meta name="viewport" content="width=device-width, initial-scale=1.0" />
  <title>Titolo della pagina</title>
  <link rel="stylesheet" href="style.css" />
</head>
<body>
  <header>
    <h1>Intestazione principale</h1>
  </header>
  <main>
    <p>Contenuto principale della pagina</p>
  </main>
  <footer>
    <p>&copy; 2025 - Tutti i diritti riservati</p>
  </footer>
</body>
</html>
\end{lstlisting}

\subsection{Elementi principali}

\begin{description}
  \item[\texttt{<!DOCTYPE html>}] Dichiara che il documento è HTML5, quindi deve rispettare le regole di questa versione del linguaggio.
  \item[\texttt{<html>}] Elemento radice del documento
  \item[\texttt{<head>}] Contiene metadati e link a risorse
  \item[\texttt{<body>}] Contiene il contenuto visibile
\end{description}

\section{Tag semantici}

HTML5 introduce tag semantici che descrivono il significato del contenuto:

\begin{lstlisting}[language=HTML]
<header>Intestazione del sito</header>
<nav>Barra di navigazione</nav>
<main>Contenuto principale</main>
<section>Una sezione tematica</section>
<article>Un articolo indipendente</article>
<aside>Barra laterale</aside>
<footer>Piè di pagina</footer>
\end{lstlisting}

\begin{attenzione}
La semantica HTML è importante per l'accessibilità: i lettori di schermo leggono il significato dei tag, non solo il testo.
\end{attenzione}

\section{Meta tag essenziali}

I meta tag sono elementi HTML che forniscono metadati sul documento HTML. Questi tag non sono visualizzati nella pagina ma contengono informazioni importanti per browser, motori di ricerca e altri servizi web.

\subsection{Sintassi generale}
Un meta tag ha questa struttura:
\begin{lstlisting}[language=HTML]
<meta name="nome" content="valore">
<!-- oppure -->
<meta http-equiv="nome" content="valore">
<!-- oppure -->
<meta charset="codifica">
\end{lstlisting}

\subsection{Meta tag fondamentali}
Ecco i meta tag più importanti e il loro utilizzo:

\begin{lstlisting}[language=HTML]
<meta charset="UTF-8" />
<meta name="viewport" content="width=device-width, initial-scale=1.0" />
<meta name="description" content="Breve descrizione della pagina" />
<meta name="keywords" content="html, css, web, sviluppo" />
\end{lstlisting}

\begin{description}
  \item[\texttt{charset}] Specifica la codifica dei caratteri del documento. UTF-8 è lo standard moderno che supporta caratteri internazionali e emoji.
  
  \item[\texttt{viewport}] Controlla come la pagina si adatta ai dispositivi mobili:
  \begin{itemize}
    \item \texttt{width=device-width}: imposta la larghezza della pagina alla larghezza del dispositivo
    \item \texttt{initial-scale=1.0}: imposta il livello di zoom iniziale
  \end{itemize}
  
  \item[\texttt{description}] Fornisce una breve descrizione della pagina (massimo 155 caratteri). Viene mostrata nei risultati di ricerca di Google.
  
  \item[\texttt{keywords}] Elenca parole chiave rilevanti per la pagina. Oggi ha meno importanza per il SEO ma è ancora utilizzato da alcuni motori di ricerca.
\end{description}

\begin{attenzione}
Il meta tag viewport è cruciale per il responsive design. Senza di esso, i siti mobile-friendly non funzioneranno correttamente sui dispositivi mobili.
\end{attenzione}

\begin{nota}
Il meta tag \texttt{viewport} è fondamentale per il responsive design. Senza di esso, i dispositivi mobili non visualizzeranno la pagina correttamente.
\end{nota}


\section{Riepilogo}

\begin{itemize}
  \item HTML è il linguaggio di marcatura del web
  \item \texttt{<!DOCTYPE html>} identifica la versione HTML5
  \item Tag semantici descrivono il significato del contenuto
  \item Meta tag sono fondamentali per accessibilità e responsive
  \item Sempre validare il codice HTML
\end{itemize}
