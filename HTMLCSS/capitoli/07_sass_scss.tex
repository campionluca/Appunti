\chapter{SASS e SCSS}
\label{cap:sass_scss}

\section{Cos'è SASS/SCSS?}

SASS (Syntactically Awesome StyleSheets) è un preprocessore CSS. SCSS è la sintassi moderna di SASS che è un superset di CSS (tutto il CSS valido è valido SCSS).

\section{Variabili}

Le variabili SCSS permettono di riutilizzare valori:

\begin{lstlisting}[language=CSS]
$primary-color: #3498db;
$secondary-color: #2ecc71;
$base-font-size: 16px;
$spacing: 20px;

body {
  font-size: $base-font-size;
  color: $primary-color;
}

.button {
  background-color: $primary-color;
  padding: $spacing;
  margin: $spacing / 2;
}
\end{lstlisting}

\section{Nesting}

Nesting riduce ripetizione di selettori:

\begin{lstlisting}[language=CSS]
.navbar {
  background-color: $primary-color;
  padding: $spacing;

  .logo {
    font-size: 24px;
    color: white;
  }

  .menu {
    display: flex;
    gap: $spacing;

    a {
      color: white;
      text-decoration: none;

      &:hover {
        color: $secondary-color;
      }
    }
  }
}
\end{lstlisting}

\begin{nota}
L'ampersand (\&) rappresenta il selettore genitore. \texttt{\&:hover} diventa \texttt{.navbar .menu a:hover}.
\end{nota}

\section{Mixin}

Mixin sono funzioni riutilizzabili:

\begin{lstlisting}[language=CSS]
@mixin flex-center {
  display: flex;
  justify-content: center;
  align-items: center;
}

@mixin responsive($breakpoint) {
  @if $breakpoint == "tablet" {
    @media (min-width: 768px) { @content; }
  }
  @if $breakpoint == "desktop" {
    @media (min-width: 1024px) { @content; }
  }
}

.button {
  @include flex-center;
  padding: $spacing;
}

@include responsive("tablet") {
  .container { width: 90%; }
}
\end{lstlisting}

\section{Import e modularità}

Organizza SCSS in file modulari:

\begin{lstlisting}[language=CSS]
// main.scss
@import "variables";    // _variables.scss
@import "mixins";       // _mixins.scss
@import "navbar";       // _navbar.scss
@import "buttons";      // _buttons.scss
@import "responsive";   // _responsive.scss
\end{lstlisting}

\section{Compilazione SCSS}

Compilare SCSS a CSS con Node.js:

\begin{lstlisting}[language=bash]
# Installare sass
npm install -g sass

# Compilare una volta
sass scss/main.scss css/style.css

# Watch mode (recompila su cambiamento)
sass --watch scss:css

# Minificato
sass --style=compressed scss/main.scss css/style.min.css
\end{lstlisting}

\section{Esercizi}

\subsection{Esercizio 1 (Base)}
Crea file \texttt{\_variables.scss} con colori, font-size, spacing. Importa in \texttt{main.scss} e usali in almeno 5 selettori.

\subsection{Esercizio 2 (Intermedio)}
Crea un mixin \texttt{@mixin card} che stilizza le card con padding, border, shadow, e font. Usalo per multiple card diverse.

\section{Riepilogo}

\begin{itemize}
  \item SCSS è superset di CSS
  \item Variabili: \$variableName
  \item Nesting: selettori annidati, ampersand per genitore
  \item Mixin: funzioni riutilizzabili con @mixin e @include
  \item Import: modularità con @import
  \item Compilazione: sass scss:css
\end{itemize}
