\chapter{Prefazione}
\label{cap:prefazione}

\section*{Benvenuti nel corso HTML e CSS}

Questo corso offre una guida completa e pratica per imparare HTML5, CSS3 e SASS/SCSS, i linguaggi fondamentali dello sviluppo web moderno.

\section*{Obiettivi del corso}

Al termine di questo corso sarai in grado di strutturare correttamente documenti HTML5 semantici, sfruttando i tag semantici per migliorare accessibilità e SEO. Apprenderai a applicare CSS per styling e layout avanzati, padroneggiando selettori, specificity e il box model. Svilupperai competenze nella creazione di form funzionali e accessibili, comprendendo come implementare validazione e garantire un'esperienza inclusiva per tutti gli utenti.

Acquisirai la capacità di implementare design responsivo che funziona perfettamente su dispositivi mobile, tablet e desktop, utilizzando media queries e approcci mobile-first. Dominerai Flexbox e CSS Grid, due strumenti fondamentali per creare layout moderni e flessibili. Imparerai a lavorare con SASS/SCSS per gestire fogli di stile complessi in modo modulare e mantenibile.

Le tue competenze si estenderanno a JavaScript, permettendoti di scrivere codice per aggiungere interattività alle tue pagine web. Sarai capace di manipolare il DOM e gestire eventi, interagendo efficacemente con gli elementi della pagina. Comprenderai come lavorare con API esterne, utilizzare fetch per richiedere dati e gestire operazioni asincrone con async/await.

Infine, acquisirai conoscenze su come ottimizzare pagine web per migliori performance e accessibilità, garantendo un'esperienza eccellente per tutti gli utenti. Apprenderai anche come distribuire siti web su piattaforme cloud, rendendo i tuoi progetti disponibili online.

\section*{Struttura del corso}

Questo corso è organizzato in 12 capitoli + esercizi:

\begin{enumerate}
  \item \textbf{Capitolo 1}: Introduzione HTML5
  \item \textbf{Capitolo 2}: Tag di blocco e riga
  \item \textbf{Capitolo 3}: Form e input
  \item \textbf{Capitolo 4}: CSS fondamentale
  \item \textbf{Capitolo 5}: Flexbox e Grid
  \item \textbf{Capitolo 6}: Design responsivo
  \item \textbf{Capitolo 7}: SASS/SCSS
  \item \textbf{Capitolo 8}: Esercizi e progetti
  \item \textbf{Capitolo 9}: JavaScript Basics
  \item \textbf{Capitolo 10}: JavaScript Avanzato
  \item \textbf{Capitolo 11}: Framework e Deployment
  \item \textbf{Capitolo 12}: Bibliografia
\end{enumerate}

\section*{Come usare questo materiale}

\begin{description}
  \item[Per lo studente] Leggi la teoria, scrivi il codice, pratica con gli esercizi
  \item[Per l'insegnante] Adatta il contenuto al ritmo della tua classe
\end{description}

\section*{Requisiti}

\begin{itemize}
  \item Editor di testo (VS Code, Sublime Text, Atom)
  \item Browser moderno (Chrome, Firefox, Safari)
  \item Node.js (facoltativo, per SCSS)
\end{itemize}

\section*{Collegamento con altri corsi}

Questo corso web completa la formazione tecnica fornendo competenze complementari ai corsi precedenti. Nel corso di Terza hai appreso il linguaggio C, sviluppando fondamenta solide di logica procedurale e programmazione a basso livello. In Quarta-Part1, hai affrontato Java Avanzato, acquisendo competenze in Object-Oriented Programming, design patterns e sviluppo backend. Ora, in Quarta-Part2, approfondisci HTML5, CSS3 e JavaScript, gli strumenti essenziali per il frontend web moderno.

Insieme, questi corsi formano un percorso didattico coerente e progressivo che ti porta dalla programmazione strutturata di base fino alla creazione di applicazioni web complete, spostandosi dalla logica procedurale alla programmazione orientata agli oggetti fino alla realizzazione di interfacce web interattive e responsivi.

\section*{Note importanti}

Tutti gli esempi sono testati e funzionanti. Non esitare a sperimentare modificando il codice e osservando i risultati nei browser. La pratica è fondamentale per imparare il web development.

---

\textit{Versione 1.0 -- Novembre 2025}
