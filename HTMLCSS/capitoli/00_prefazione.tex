\chapter{Prefazione}
\label{cap:prefazione}

\section*{Benvenuti nel corso HTML e CSS}

Questo corso offre una guida completa e pratica per imparare HTML5, CSS3 e SASS/SCSS, i linguaggi fondamentali dello sviluppo web moderno.

\section*{Obiettivi del corso}

Al termine di questo corso sarai in grado di:

\begin{itemize}
  \item Strutturare correttamente documenti HTML5 semantici
  \item Applicare CSS per styling e layout avanzati
  \item Creare form funzionali e accessibili
  \item Implementare design responsivo per mobile/tablet/desktop
  \item Usare Flexbox e CSS Grid per layout moderni
  \item Lavorare con SASS/SCSS per CSS modulare
  \item Scrivere JavaScript per interattività web
  \item Manipolare il DOM e gestire eventi
  \item Lavorare con API, fetch, async/await
  \item Ottimizzare pagine web per performance e accessibilità
  \item Distribuire siti web su piattaforme cloud
\end{itemize}

\section*{Struttura del corso}

Questo corso è organizzato in 12 capitoli + esercizi:

\begin{enumerate}
  \item \textbf{Capitolo 1}: Introduzione HTML5
  \item \textbf{Capitolo 2}: Tag di blocco e riga
  \item \textbf{Capitolo 3}: Form e input
  \item \textbf{Capitolo 4}: CSS fondamentale
  \item \textbf{Capitolo 5}: Flexbox e Grid
  \item \textbf{Capitolo 6}: Design responsivo
  \item \textbf{Capitolo 7}: SASS/SCSS
  \item \textbf{Capitolo 8}: Esercizi e progetti
  \item \textbf{Capitolo 9}: JavaScript Basics
  \item \textbf{Capitolo 10}: JavaScript Avanzato
  \item \textbf{Capitolo 11}: Framework e Deployment
  \item \textbf{Capitolo 12}: Bibliografia
\end{enumerate}

\section*{Come usare questo materiale}

\begin{description}
  \item[Per lo studente] Leggi la teoria, scrivi il codice, pratica con gli esercizi
  \item[Per l'insegnante] Adatta il contenuto al ritmo della tua classe
\end{description}

\section*{Requisiti}

\begin{itemize}
  \item Editor di testo (VS Code, Sublime Text, Atom)
  \item Browser moderno (Chrome, Firefox, Safari)
  \item Node.js (facoltativo, per SCSS)
\end{itemize}

\section*{Collegamento con altri corsi}

Questo corso web completa la formazione tecnica:

\begin{itemize}
  \item \textbf{Terza}: Linguaggio C per logica procedurale e programmazione a basso livello
  \item \textbf{Quarta-Part1}: Java Avanzato per OOP, design patterns e backend
  \item \textbf{Quarta-Part2}: HTML5, CSS3, JavaScript per frontend web moderno
\end{itemize}

Insieme formano un percorso didattico coerente che spazia dalla programmazione strutturata al web development completo.

\section*{Note importanti}

Tutti gli esempi sono testati e funzionanti. Non esitare a sperimentare modificando il codice e osservando i risultati nei browser. La pratica è fondamentale per imparare il web development.

---

\textit{Versione 1.0 -- Novembre 2025}
