\chapter{Bibliografia e Risorse}
\label{cap:bibliografia}

\section{Documentazione ufficiale}

Consulta MDN Web Docs per HTML e CSS, W3C per specifiche tecniche, Can I Use per compatibilità browser.

\section{Tutorial e Learning}

CodePen, JsFiddle, CSS-Tricks per articoli e tutorial. Flexbox Froggy e Grid Garden per giochi interattivi.

\section{Tool online}

W3C HTML Validator, W3C CSS Validator, WebAIM WCAG Checker per validazione. Sass Playground per compilazione SCSS online.

\section{Editor e IDE}

VS Code (consigliato con Live Server), Sublime Text, WebStorm, Atom per sviluppo HTML/CSS.

\section{Editor e IDE}

\begin{itemize}
  \item \textbf{VS Code} (Gratuito) - Raccomandato
    \begin{itemize}
      \item Estensioni: Live Server, CSS Intellisense, Prettier
    \end{itemize}
  \item \textbf{Sublime Text} (Pagato, prova gratuita)
  \item \textbf{WebStorm} (Pagato, ma completo)
  \item \textbf{Atom} (Gratuito, da GitHub)
\end{itemize}

\section{Browser Developer Tools}

\begin{itemize}
  \item \textbf{Chrome DevTools}: F12 o Ctrl+Shift+I
  \item \textbf{Firefox Developer Edition}: F12
  \item \textbf{Safari Web Inspector}: Cmd+Option+I
  \item Strumenti essenziali: Inspector, Console, Network, Performance
\end{itemize}

\section{Libri consigliati}

\begin{itemize}
  \item ``HTML \& CSS: Design and Build Websites'' - Jon Duckett
  \item ``Responsive Web Design'' - Ethan Marcotte
  \item ``CSS Secrets'' - Lea Verou
  \item ``The Pragmatic Programmer'' - (contiene best practices web)
\end{itemize}

\section{Community e Forum}

\begin{itemize}
  \item \textbf{Stack Overflow}: \url{https://stackoverflow.com/} - Q\&A
  \item \textbf{CSS-Tricks Community}: Forum e discussioni
  \item \textbf{Reddit r/webdev}: Comunità web developers
  \item \textbf{GitHub}: Cerca progetti open source per imparare
\end{itemize}

\section{Collegamento con altri corsi}

Come visto in questo corso:

\begin{itemize}
  \item \textbf{Terza (Linguaggio C)}: Introduzione a C - Logica e algoritmi
  \item \textbf{Quarta (Java)}: Classi, Oggetti, Ereditarietà e Package - OOP e backend
  \item \textbf{Quarta (HTML/CSS)}: Questo corso - Frontend web
\end{itemize}

---

\textit{Ultimo aggiornamento: Novembre 2025}
