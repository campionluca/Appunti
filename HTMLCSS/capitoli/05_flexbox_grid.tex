\chapter{Flexbox e Grid}
\label{cap:flexbox_grid}

\section{Flexbox}

Flexbox è un modo moderno per creare layout flessibili:

\begin{lstlisting}[language=CSS]
.container {
  display: flex;
  flex-direction: row;      /* row, column */
  justify-content: center;  /* allineamento asse principale */
  align-items: center;      /* allineamento asse trasversale */
  gap: 20px;               /* spazio tra elementi */
}

.item {
  flex: 1;                 /* crescita uguale */
  flex-basis: 200px;       /* larghezza base */
}
\end{lstlisting}

\subsection{Proprietà container}

\begin{itemize}
  \item \texttt{flex-direction}: row, column, row-reverse, column-reverse
  \item \texttt{justify-content}: flex-start, center, flex-end, space-between, space-around
  \item \texttt{align-items}: flex-start, center, flex-end, stretch, baseline
  \item \texttt{flex-wrap}: nowrap, wrap, wrap-reverse
\end{itemize}

\section{CSS Grid}

Grid è un sistema di layout bidimensionale:

\begin{lstlisting}[language=CSS]
.grid-container {
  display: grid;
  grid-template-columns: 1fr 2fr 1fr;
  grid-template-rows: auto 1fr auto;
  gap: 20px;
}

.grid-item {
  grid-column: 1 / 3;      /* colonne 1-2 */
  grid-row: 1 / 2;        /* riga 1 */
}
\end{lstlisting}

\begin{nota}
\texttt{1fr} significa "1 frazione" dello spazio disponibile. Se hai 3 colonne con \texttt{1fr 2fr 1fr}, la seconda colonna sarà il doppio delle altre.
\end{nota}

\section{Flexbox vs Grid}

\begin{itemize}
  \item \textbf{Flexbox}: Layout 1D (riga o colonna)
  \item \textbf{Grid}: Layout 2D (righe e colonne simultaneamente)
  \item Usa flexbox per navbar, card layout
  \item Usa grid per layout pagina completa
\end{itemize}

\section{Esercizi}

\subsection{Esercizio 1 (Base)}
Crea un navbar usando flexbox con logo a sinistra e menu a destra.

\subsection{Esercizio 2 (Intermedio)}
Crea una griglia 3x3 di card usando CSS Grid. Fai in modo che una card occupi 2 colonne.

\subsection{Esercizio 3 (Avanzato)}
Crea un layout di pagina completo con header (grid), sidebar e main content (flex), footer usando entrambi.

\section{Riepilogo}

\begin{itemize}
  \item Flexbox per layout monodimensionali
  \item Grid per layout bidimensionali
  \item \texttt{justify-content} e \texttt{align-items} per allineamento
  \item \texttt{gap} per spazi uniformi
  \item \texttt{1fr} per unità di frazione
\end{itemize}
