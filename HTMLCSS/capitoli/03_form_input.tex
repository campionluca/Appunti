\chapter{Form e Input}
\label{cap:form_input}

\section{Introduzione ai Form}

I form (moduli) sono elementi fondamentali del web che permettono l'interazione tra utente e sito web. I form hanno moltissime applicazioni pratiche: permettono di raccogliere dati dagli utenti in contesti come registrazione, login e contatti. Sono utilizzati per implementare funzionalità di ricerca, consentendo agli utenti di trovare contenuti specifici. I form facilitano il caricamento di file, come quando vuoi permettere agli utenti di caricare foto o documenti. Rappresentano il canale principale per raccogliere feedback e commenti dagli utenti, migliorando così il tuo servizio. Nei contesti di e-commerce, i form sono essenziali per completare transazioni come acquisti e prenotazioni. Infine, i form permettono agli utenti di configurare le loro preferenze personali, migliorando l'esperienza d'uso.

Il processo di gestione di un form prevede tre fasi:
\begin{enumerate}
  \item \textbf{Raccolta dati}: L'utente compila i campi del form
  \item \textbf{Invio}: I dati vengono inviati al server tramite una richiesta HTTP
  \item \textbf{Elaborazione}: Il server processa i dati e restituisce una risposta
\end{enumerate}

\begin{nota}
I form sono il ponte tra il frontend (interfaccia utente) e il backend (logica del server). La sicurezza e la validazione dei dati sono cruciali in questo processo.
\end{nota}

\section{Elemento form}

Il tag \texttt{<form>} crea un modulo per raccogliere dati dall'utente. È simile ai metodi di una classe Java (vedi il corso di Quarta sulla programmazione orientata agli oggetti) che accettano parametri:

\begin{lstlisting}[language=HTML]
<form action="paginaDiArrivo.html" method="POST" name="contactForm">
  <label for="email">Email:</label>
  <input type="email" id="email" name="email" required>

  <button type="submit">Invia</button>
  <button type="reset">Cancella</button>
</form>
\end{lstlisting}

\subsection{Attributi form}

\begin{description}
  \item[\texttt{action}] URL dove inviare i dati del form
  \item[\texttt{method}] GET o POST (modalità invio dati)
  \item[\texttt{name}] Nome identificativo del form
  \item[\texttt{enctype}] Tipo di codifica (application/x-www-form-urlencoded, multipart/form-data)
\end{description}

\section{Input types}

HTML5 offre molti tipi di input per diversi dati:
\begin{lstlisting}[language=HTML]
<input type="text" placeholder="Testo generico" />
<input type="password" placeholder="Password" />
<input type="email" placeholder="Email" />
<input type="number" min="0" max="100" step="5" />
<input type="date" />
<input type="checkbox" id="terms" /> <label for="terms">Accetto i termini</label>
<input type="radio" name="option" value="1" id="opt1" /> <label for="opt1">Opzione 1</label>
<select>
  <option>Scegli un'opzione</option>
</select>
<textarea rows="5" cols="40"></textarea>
\end{lstlisting}

%TODO: Inserire immagine del codice sopra


\section{Label e accessibilità}

Il tag \texttt{<label>} associa il testo all'input:

\begin{lstlisting}[language=HTML]
<label for="username">Username:</label>
<input type="text" id="username" name="username">

<label for="subscribe">Iscrivimi alla newsletter</label>
  <input type="checkbox" id="subscribe" name="subscribe">
\end{lstlisting}

\begin{nota}
L'attributo \texttt{for} del label deve corrispondere all'attributo \texttt{id} dell'input per accessibilità screen reader.
\end{nota}

\section{Validazione HTML5}

HTML5 supporta validazione lato client:

\begin{lstlisting}[language=HTML]
<input type="email" required />
<input type="number" min="0" max="10" />
<input type="text" pattern="[A-Za-z]+" title="Solo lettere" />
<input type="password" minlength="8" />
\end{lstlisting}

\begin{attenzione}
La validazione HTML5 non sostituisce la validazione server-side. Sempre convalidare i dati sul server poiché l'utente potrebbe modificare il codice client.
\end{attenzione}

\section{Riepilogo}

\begin{itemize}
  \item I form sono lo strumento principale per l'interazione utente-server
  \item L'elemento \texttt{<form>} ha attributi essenziali:
    \begin{itemize}
      \item \texttt{action}: dove inviare i dati
      \item \texttt{method}: come inviarli (GET/POST)
      \item \texttt{enctype}: tipo di codifica
    \end{itemize}
  \item HTML5 offre diversi tipi di input specializzati:
    \begin{itemize}
      \item \texttt{text}, \texttt{email}, \texttt{password}, \texttt{number}
      \item \texttt{date}, \texttt{checkbox}, \texttt{radio}
      \item \texttt{file}, \texttt{textarea}, \texttt{select}
    \end{itemize}
  \item Ogni \texttt{input} dovrebbe avere un \texttt{label} associato
  \item La validazione client-side (HTML5) va sempre accompagnata da validazione server-side
  \item L'accessibilità è fondamentale (label, attributi \texttt{for} e \texttt{id})
\end{itemize}
