\chapter{Tag di Blocco e Riga}
\label{cap:tag_blocco_riga}

\section{Differenza tra blocco e riga}

In HTML, i tag sono classificati in due categorie:

\begin{description}
  \item[Block-level] L'elemento occupa tutta la larghezza disponibile e inizia su una nuova riga
  \item[Inline] L'elemento occupa solo lo spazio necessario e rimane sulla stessa riga
\end{description}

\section{Tag di blocco}

I tag di blocco iniziano su una nuova riga:

\begin{lstlisting}[language=HTML]
<!-- Esempio di elementi di blocco HTML -->
<div class="container">Contenitore generico</div>
<p>Questo è un paragrafo di testo</p>
<!-- I titoli hanno diversi livelli di importanza -->
<h1>Titolo principale</h1>
<h2>Titolo secondario</h2>
<section>Una sezione tematica</section>
<article>Un articolo completo</article>
<!-- Le liste possono essere ordinate o non ordinate -->
<ul>
  <li>Lista elemento non ordinato</li>
  <li>Lista elemento non ordinato</li>
</ul>
<ol>
  <li>Lista elemento ordinato</li>
  <li>Lista elemento ordinato</li>
</ol>
\end{lstlisting}

%TODO: Inserire l'immagine del codice sopra

\subsection{Tag blocco comuni}

\begin{itemize}
  \item \texttt{<div>}: Contenitore generico
  \item \texttt{<p>}: Paragrafo
  \item \texttt{<h1>} a \texttt{<h6>}: Heading (titoli)
  \item \texttt{<section>}, \texttt{<article>}, \texttt{<header>}, \texttt{<footer>}: Semantici
  \item \texttt{<ul>}, \texttt{<ol>}, \texttt{<li>}: Liste
  \item \texttt{<blockquote>}: Citazione in blocco
\end{itemize}

\section{Tag inline}

I tag inline non creano nuove righe e occupano solo lo spazio del contenuto:

\begin{lstlisting}[language=HTML]
<span class="highlight">testo importante</span>
<a href="https://example.com">collegamento</a>
<strong>testo forte (bold)</strong>
<em>testo enfatizzato (italic)</em>
<code>codice inline</code>
<img src="immagine.jpg" alt="Descrizione" />
\end{lstlisting}

%TODO: Questa sezione va spostata nei css

\section{Proprietà display CSS}

La proprietà CSS \texttt{display} controlla come un elemento viene visualizzato:

\begin{lstlisting}[language=CSS]
/* Block-level */
display: block;
display: flex;
display: grid;

/* Inline */
display: inline;
display: inline-block;
display: none;
\end{lstlisting}

\begin{nota}
Con \texttt{display: inline-block}, un elemento inline può avere larghezza e altezza come un blocco, ma rimane sulla stessa riga degli elementi vicini.
\end{nota}

\section{Riepilogo}

\begin{itemize}
  \item Tag blocco occupano tutta la larghezza disponibile
  \item Tag inline occupano solo lo spazio del contenuto
  \item \texttt{display: block/inline/inline-block} controlla il comportamento
  \item Proprietà CSS può modificare il comportamento di default
  \item Semantica HTML rimane importante indipendentemente da display
\end{itemize}
