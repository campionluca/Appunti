% main.tex - bozza minimale per compilazione
\documentclass[a4paper,11pt,oneside]{book}

% Pacchetti essenziali
\usepackage[T1]{fontenc}
\usepackage[utf8]{inputenc}
\usepackage{lmodern}    
\usepackage{lmodern}         
\usepackage{geometry}
\usepackage{graphicx}
\usepackage{float}
\usepackage{listings}
\usepackage{xcolor}
\usepackage{hyperref}
\usepackage{url}
\usepackage{fancyhdr}
\usepackage{amsmath}
\usepackage{tcolorbox}
\usepackage{tikz}
\usetikzlibrary{shapes,arrows,positioning,trees,shadows,fit}
\usepackage{titling}            % ← Per il \subtitle

% ... resto del preambolo identico ...
\usetikzlibrary{shapes,arrows,positioning,trees,shadows,fit}

% Configurazione
\geometry{margin=2.5cm}
\setlength{\parindent}{0pt}
\setlength{\parskip}{6pt}

% Definizione colori per HTML
\definecolor{htmltagcolor}{RGB}{0,0,255}      % Blu brillante per i tag
\definecolor{htmlattrcolor}{RGB}{128,0,255}   % Viola brillante per gli attributi
\definecolor{htmlvalcolor}{RGB}{255,0,0}      % Rosso brillante per i valori
\definecolor{htmlcommentcolor}{RGB}{0,128,0}  % Verde per i commenti

% Definizione base del linguaggio HTML
\lstdefinelanguage{HTML}{
  sensitive=true,
  keywords={DOCTYPE,html,head,meta,tag,link,title,video,source,body,div,p,span,a,img,ul,ol,li,br,h1,h2,h3,h4,h5,h6,form,input,label,script,style,button,section,article,header,footer,nav,main,aside,blockquote,strong,em,code,pre,select,textarea,option,area,hr},
  ndkeywords={class,id,href,src,rel,type,attributo,charset,lang,control,media,alt,name,content,action,method,enctype,placeholder,required,pattern,min,max,step,minlength,maxlength,rows,cols,value,for,title,shape,coords},
  comment=[s][\color{htmlcommentcolor}]{<!--}{-->},
  morestring=[s]{"}{"}
}

% Configurazione per HTML
\lstdefinestyle{HTML}{
  language=HTML,
  basicstyle=\ttfamily\small,
  keywordstyle=\color{htmltagcolor},
  ndkeywordstyle=\color{htmlattrcolor},
  commentstyle=\color{htmlcommentcolor},
  stringstyle=\color{htmlvalcolor},
  morecomment=[s][\color{htmlcommentcolor}]{<!--}{-->},
  morestring=[b]",
  numbers=left,
  numberstyle=\tiny,
  showspaces=false,
  showstringspaces=false,
  breaklines=true,
  frame=single,
  rulecolor=\color{black!30},
  backgroundcolor=\color{gray!5}
}
\lstset{style=HTML,
    escapeinside={(*@}{@*)},
    xleftmargin=15pt,
    xrightmargin=15pt,
    float=htbp,
    aboveskip=15pt,
    belowskip=15pt,literate=
        {à}{{\`a}}1 {è}{{\`e}}1 {ì}{{\`i}}1 {ò}{{\`o}}1 {ù}{{\`u}}1
        {À}{{\`A}}1 {È}{{\`E}}1 {Ì}{{\`I}}1 {Ò}{{\`O}}1 {Ù}{{\`U}}1
        {á}{{\'a}}1 {é}{{\'e}}1 {í}{{\'i}}1 {ó}{{\'o}}1 {ú}{{\'u}}1
        {Á}{{\'A}}1 {É}{{\'E}}1 {Í}{{\'I}}1 {Ó}{{\'O}}1 {Ú}{{\'U}}1
        {più}{{pi\`u}}1 {può}{{pu\`o}}1
}

% Supporto linguaggi (mapping to supported languages)
\lstalias{bash}{sh}
\lstalias{html}{html}
\lstalias{css}{sh}
\lstalias{javascript}{java}
\lstalias{text}{sh}

% Box Colorati per Note Speciali
\newtcolorbox{attenzione}{
    colback=yellow!10,
    colframe=orange!80,
    title=Attenzione,
    fonttitle=\bfseries
}

\newtcolorbox{nota}{
    colback=blue!5,
    colframe=blue!60,
    title=Nota,
    fonttitle=\bfseries
}

\newtcolorbox{errore}{
    colback=red!5,
    colframe=red!60,
    title=Errore Comune,
    fonttitle=\bfseries
}

\begin{document}
\frontmatter
\title{Programmazione lato client}
%\subtitle{HTML5, CSS3, SCSS e JavaScript}
\author{Prof. Luca Campion}
\date{\today}
\maketitle
\tableofcontents
%\listoffigures
%\lstlistoflistings
\mainmatter

% Prefazione
%\chapter*{Prefazione}
\addcontentsline{toc}{chapter}{Prefazione}

\section*{A chi è rivolto questo libro}

Questi appunti sono stati pensati per gli studenti del quarto anno di Istituto Tecnico che stanno approfondendo la programmazione in Java. Il materiale presuppone una conoscenza di base del linguaggio (variabili, cicli, metodi, concetti fondamentali di programmazione) e si propone di consolidare e ampliare tali competenze attraverso argomenti più avanzati e pratici.

L'approccio adottato bilancia teoria ed esempi concreti, con l'obiettivo di fornire strumenti immediatamente applicabili sia nei progetti scolastici che in contesti reali.

\section*{Struttura del libro}

Il libro è organizzato in otto capitoli, ciascuno focalizzato su un argomento specifico:

\begin{enumerate}
    \item \textbf{Classi, Oggetti e Ereditarietà}: ripasso e approfondimento dei concetti fondamentali della programmazione orientata agli oggetti, con particolare attenzione agli array di oggetti e alla gerarchia tra classi.

    \item \textbf{Stream e Buffer}: gestione di flussi di dati per leggere e scrivere file, con esempi pratici di utilizzo delle classi più comuni.

    \item \textbf{Interfacce e Classi Astratte}: meccanismi per definire comportamenti comuni e creare gerarchie flessibili.

    \item \textbf{Eccezioni}: gestione degli errori a runtime attraverso il sistema delle eccezioni di Java.

    \item \textbf{ArrayList}: struttura dati dinamica per gestire collezioni di elementi in modo più flessibile rispetto agli array tradizionali.

    \item \textbf{Interfacce Grafiche}: introduzione alla creazione di applicazioni con interfaccia grafica usando Swing, inclusa la gestione degli eventi.

    \item \textbf{Model View Controller}: pattern architetturale per organizzare il codice separando logica, presentazione e controllo.

    \item \textbf{Lambda Expressions}: cenni alle espressioni lambda introdotte in Java 8, per scrivere codice più conciso ed espressivo.
\end{enumerate}

\section*{Come usare questo libro}

Ogni capitolo è strutturato per guidare l'apprendimento in modo progressivo:

\begin{itemize}
    \item Gli \textbf{obiettivi di apprendimento} all'inizio di ogni capitolo chiariscono cosa ci si aspetta di saper fare al termine dello studio.

    \item La \textbf{teoria} è presentata in modo sintetico ma completo, con definizioni chiare e schemi quando necessario.

    \item Gli \textbf{esempi di codice} sono commentati in italiano e mostrano l'applicazione pratica dei concetti. Si consiglia di digitare personalmente ogni esempio, eseguirlo e sperimentare modifiche per comprenderne il funzionamento.

    \item I \textbf{box colorati} evidenziano informazioni particolari:
    \begin{itemize}
        \item \textcolor{orange}{Arancione (Attenzione)}: punti critici da ricordare
        \item \textcolor{blue}{Blu (Nota)}: suggerimenti e best practices
        \item \textcolor{red}{Rosso (Errore Comune)}: errori frequenti da evitare
    \end{itemize}

    \item Gli \textbf{esercizi} sono suddivisi in tre livelli di difficoltà (base, intermedio, avanzato). Si consiglia di affrontarli in ordine, verificando le soluzioni commentate nell'appendice solo dopo aver tentato autonomamente.

    \item Il \textbf{riepilogo} alla fine di ogni capitolo sintetizza i concetti chiave e facilita il ripasso.
\end{itemize}

\section*{Prerequisiti}

Per affrontare efficacemente questi appunti, è necessario:

\begin{itemize}
    \item Conoscere la sintassi base di Java (tipi di dato primitivi, operatori, strutture di controllo)
    \item Saper dichiarare e utilizzare metodi
    \item Comprendere i concetti basilari di classe e oggetto
    \item Avere familiarità con array monodimensionali
    \item Disporre di un ambiente di sviluppo Java funzionante (JDK 8 o superiore, IDE come Eclipse, IntelliJ IDEA o NetBeans)
\end{itemize}

\section*{Convenzioni utilizzate}

\textbf{Codice}: tutti gli esempi di codice sono presentati con sintassi evidenziata, numerazione delle righe e commenti esplicativi.

\textbf{Nomenclatura}: si segue la convenzione Java standard (CamelCase per classi, camelCase per metodi e variabili, MAIUSCOLO per costanti).

\textbf{Terminologia}: si preferisce l'italiano quando possibile, mantenendo i termini tecnici in inglese quando consolidati nella pratica professionale (ad esempio "stream", "buffer", "exception").

\vspace{1cm}

Buono studio!


% Capitoli HTML/CSS
\chapter{Introduzione a HTML5}
\label{cap:intro_html}

\section{Cos'è HTML?}

HTML (HyperText Markup Language) è il linguaggio di marcatura per creare pagine web. Non è un linguaggio di programmazione come Java (vedi il capitolo sulla programmazione orientata agli oggetti nel corso di Quarta), ma un modo per strutturare e descrivere il contenuto usando tag e attributi.

\section{Struttura base di un documento HTML5}

\subsection{Anatomia dei tag HTML}
Un tag HTML è composto da tre parti fondamentali. La prima parte è il tag di apertura, che ha la forma \texttt{<nome>} e indica l'inizio dell'elemento. La seconda parte è il contenuto, che può essere semplice testo o altri tag annidati che formano una struttura gerarchica. Infine, la terza parte è il tag di chiusura, scritto come \texttt{</nome>}, che marca la fine dell'elemento e completa la struttura del tag.

Esempio di tag con contenuto:
\begin{lstlisting}[language=HTML]
<p>Questo è un paragrafo</p>
<div>
    <p>Questo paragrafo è annidato nel div</p>
</div>
\end{lstlisting}

\subsection{Tag auto-chiudenti (self-closing tags)}
Alcuni elementi HTML non hanno contenuto. Questi sono chiamati \textbf{tag auto-chiudenti} o \textbf{self-closing tags}. Per garantire la massima compatibilità con XHTML e XML, è importante chiudere esplicitamente questi tag:

\begin{lstlisting}[language=HTML]
<img src="immagine.jpg" alt="Una descrizione" />
<input type="text" placeholder="Inserisci testo" />
<meta charset="UTF-8" />
<link rel="stylesheet" href="stile.css" />
<area shape="rect" coords="0,0,100,100" href="link.html" />
\end{lstlisting}

\begin{nota}
Anche se HTML5 permette di omettere la chiusura (\texttt{/>}), è consigliabile usare sempre la sintassi XML/XHTML con chiusura esplicita. Questo approccio garantisce la compatibilità con XHTML e XML, rendendo il codice più consistente e leggibile. Inoltre, aiuta a evitare problemi con parser XML e facilita la migrazione tra diversi formati, assicurando che il tuo codice possa essere facilmente convertito o integrato con altri sistemi.
\end{nota}

\begin{nota}
Nota: In passato si utilizzavano i tag \texttt{<br>} e \texttt{<hr>} per creare interruzioni di riga e linee orizzontali. Queste pratiche sono ora considerate obsolete poiché mescolano presentazione e contenuto. È preferibile utilizzare CSS per gestire lo spazio e le separazioni visive.

Invece di \texttt{<br>}, puoi utilizzare approcci più semantici come elementi blocco appropriati (\texttt{<p>}, \texttt{<div>}) che creano naturalmente interruzioni di linea, oppure puoi controllare lo spazio verticale attraverso le proprietà CSS \texttt{margin} o \texttt{padding}, o ancora gestire la formattazione del testo con la proprietà CSS \texttt{white-space}.

Invece di \texttt{<hr>}, dovresti preferire soluzioni CSS come la proprietà \texttt{border} per creare linee divisorie, oppure utilizzare elementi semantici con stili appropriati, o ancora sfruttare elementi strutturali come \texttt{<section>} che forniscono una separazione logica e visiva del contenuto.
\end{nota}

\subsection{Commenti HTML}
I commenti in HTML sono utili per documentare il codice e vengono ignorati dal browser. La sintassi è:
\begin{lstlisting}[language=HTML]
<!-- Questo è un commento -->
<!-- 
    I commenti possono
    essere su più righe
-->
\end{lstlisting}

I commenti sono strumenti molto utili durante lo sviluppo. Puoi utilizzarli per documentare sezioni complesse del codice, spiegando il ragionamento dietro decisioni di design o struttura. Sono particolarmente utili quando hai bisogno di disabilitare temporaneamente parti di HTML durante il debugging o il refactoring. Infine, i commenti permettono di aggiungere note per altri sviluppatori che lavoreranno sul tuo codice, facilitando la comprensione e la collaborazione nel team.

\subsection{Attributi HTML}
Gli attributi forniscono informazioni aggiuntive ai tag e hanno questa sintassi:
\begin{lstlisting}[language=HTML]
<tag attributo="valore">contenuto</tag>
\end{lstlisting}

\subsection{Gli attributi id e class}
Gli attributi \texttt{id} e \texttt{class} sono fondamentali per identificare e raggruppare elementi HTML:

\begin{description}
  \item[\texttt{id}] È un identificatore \textbf{univoco} per un elemento nella pagina. Deve essere unico nel documento, il che significa che non possono coesistere due elementi con lo stesso \texttt{id}. Viene utilizzato per identificare un elemento specifico e risulta particolarmente utile per JavaScript quando hai bisogno di accedere a un elemento preciso. L'attributo \texttt{id} è anche fondamentale per creare collegamenti interni alla pagina (bookmark). Nel CSS, ti riferisci a un elemento con id usando il selettore \texttt{\#} (es: \texttt{\#header}).
  Esempio: \texttt{<div id="header">}

  \item[\texttt{class}] Definisce una o più categorie a cui l'elemento appartiene. Un elemento può avere multiple classi, separate da spazi, permettendoti di combinare diversi stili e comportamenti. Diversamente da \texttt{id}, la stessa classe può essere usata su molteplici elementi della pagina, rendendola ideale per applicare stili comuni a gruppi di elementi. Nel CSS, ti riferisci a una classe usando il selettore \texttt{.} (es: \texttt{.container}).
  Esempio: \texttt{<div class="container blue">}
\end{description}

\begin{nota}
Le differenze principali tra questi due attributi sono concettuali e pratiche. L'attributo \texttt{id} funziona come una carta d'identità: è unico e identifica un singolo elemento specifico nella pagina. Al contrario, l'attributo \texttt{class} funziona come un gruppo o categoria, e può essere condiviso tra più elementi, permettendo di applicare gli stessi stili o comportamenti a diversi componenti della pagina.
\end{nota}

\subsection{Attributi per collegamenti e risorse}

\subsubsection{L'attributo href}
L'attributo \texttt{href} (Hypertext REFerence) definisce il collegamento a una risorsa ed è utilizzato principalmente con il tag \texttt{<a>} per creare link. L'attributo \texttt{href} può contenere diversi tipi di valori, a seconda della destinazione desiderata. Puoi specificare URL assoluti (come \texttt{https://www.esempio.com}) per collegamenti a siti web esterni, oppure URL relativi (come \texttt{./immagini/foto.jpg}) per risorse all'interno del tuo progetto. È possibile anche creare collegamenti interni alla pagina stessa usando anchor link (come \texttt{\#sezione}), stabilire collegamenti email (come \texttt{mailto:esempio@email.com}) che permettono agli utenti di inviare messaggi, oppure creare link telefonici (come \texttt{tel:+390123456789}) che attivano applicazioni di composizione su dispositivi mobili.

Esempi di utilizzo:
\begin{lstlisting}[language=HTML]
<!-- Link a sito web -->
<a href="https://www.esempio.com/">Visita il sito</a>

<!-- Link a sezione della pagina -->
<a href="#introduzione">Vai all'introduzione</a>

<!-- Link a email -->
<a href="mailto:info@esempio.com">Contattaci</a>
\end{lstlisting}

\subsubsection{L'attributo src}
L'attributo \texttt{src} (source) specifica la fonte di una risorsa multimediale ed è utilizzato con diversi elementi HTML che caricano contenuti esterni. Troverai \texttt{src} utilizzato con il tag \texttt{<img>} per le immagini, con \texttt{<video>} per i filmati, con \texttt{<audio>} per i suoni, e con \texttt{<script>} per file JavaScript. Proprio come \texttt{href}, \texttt{src} può contenere diversi tipi di percorsi. Puoi specificare percorsi assoluti (come \texttt{https://cdn.esempio.com/immagine.jpg}) che puntano a risorse su server remoti, oppure percorsi relativi (come \texttt{./assets/logo.png}) che riferiscono risorse locali nel tuo progetto. In alcuni casi avanzati, puoi anche utilizzare Data URI (come \texttt{data:image/png;base64,...}) per incorporare direttamente il contenuto nel markup HTML.

Esempi di utilizzo:
\begin{lstlisting}[language=HTML]
<!-- Immagine da URL assoluto -->
<img src="https://esempio.com/foto.jpg" alt="Descrizione" />

<!-- Immagine da percorso relativo -->
<img src="./immagini/logo.png" alt="Logo" />

<!-- Video con sorgenti multiple -->
<video controls="controls">
    <source src="video.mp4" type="video/mp4" />
    <source src="video.webm" type="video/webm" />
</video>
\end{lstlisting}

\begin{nota}
Differenze principali:
\begin{itemize}
  \item \texttt{href}: per collegamenti e riferimenti (dove vuoi andare)
  \item \texttt{src}: per contenuti da incorporare (cosa vuoi mostrare)
\end{itemize}
\end{nota}

\subsection{Annidamento dei tag}
I tag possono essere annidati, cioè contenuti dentro altri tag, ma devono seguire regole precise per assicurare un markup valido. I tag devono essere chiusi nell'ordine inverso rispetto a quello in cui sono stati aperti, garantendo coerenza strutturale. Inoltre, ogni tag figlio deve essere completamente contenuto dentro il suo tag genitore, mantenendo una corretta gerarchia. Un accorgimento pratico e molto utile è utilizzare l'indentazione appropriata nel codice, che aiuta visivamente a comprendere la struttura e la profondità di annidamento degli elementi.

Esempio corretto:
\begin{lstlisting}[language=HTML]
<div class="container">
    <header>
        <h1>Titolo</h1>
        <nav>
            <a href="pagina1.html">Link 1</a>
            <a href="/about/index.html">Chi siamo</a>
            <a href="../contatti.html">Contatti</a>
        </nav>
    </header>
</div>
\end{lstlisting}

Esempio errato:
\begin{lstlisting}[language=HTML]
<div><p>Questo è <!-- NON FARE COSÌ -->
    <span>sbagliato</div></p></span>
\end{lstlisting}

\begin{attenzione}
L'annidamento errato può causare comportamenti imprevedibili nel rendering della pagina e non passa la validazione HTML.
\end{attenzione}

\subsection{Commenti HTML}
I commenti in HTML servono per documentare il codice e non vengono visualizzati nel browser, rendendoli ideali per note interne. I commenti sono particolarmente utili per spiegare sezioni complesse di codice, aiutando te e gli altri sviluppatori a comprendere il ragionamento dietro determinate scelte. Durante lo sviluppo e il debugging, i commenti permettono di disabilitare temporaneamente parti di codice senza eliminarle completamente. Puoi anche utilizzarli per organizzare il codice in sezioni logiche ben definite, facilitando la navigazione in file di grandi dimensioni. Infine, i commenti sono uno strumento essenziale di comunicazione con altri sviluppatori, permettendoti di condividere informazioni importanti e best practices all'interno del team.

Sintassi dei commenti:
\begin{lstlisting}[language=HTML]
<!-- Questo è un commento su una riga -->

<!-- 
  Questo è un commento
  su più righe
-->

<!-- Non usare -- dentro i commenti -->

<div class="header">
  <!-- TODO: aggiungere logo -->
  <h1>Titolo</h1>
</div>

<!--[if IE 8]>
  <link href="ie8only.css" rel="stylesheet">
<![endif]-->
\end{lstlisting}

\begin{nota}
I commenti HTML possono contenere qualsiasi testo tranne \texttt{--} doppio trattino, che può causare problemi di parsing.
\end{nota}

\begin{attenzione}
I commenti sono visibili nel codice sorgente della pagina. Non inserire informazioni sensibili (password, chiavi API, ecc.) nei commenti.
\end{attenzione}

\subsection{Struttura completa documento}
Ecco la struttura completa di un documento HTML5:

\begin{lstlisting}[language=HTML]
<!DOCTYPE html>
<html lang="it">
<head>
  <meta charset="UTF-8" />
  <meta name="viewport" content="width=device-width, initial-scale=1.0" />
  <title>Titolo della pagina</title>
  <link rel="stylesheet" href="style.css" />
</head>
<body>
  <header>
    <h1>Intestazione principale</h1>
  </header>
  <main>
    <p>Contenuto principale della pagina</p>
  </main>
  <footer>
    <p>&copy; 2025 - Tutti i diritti riservati</p>
  </footer>
</body>
</html>
\end{lstlisting}

\subsection{Elementi principali}

\begin{description}
  \item[\texttt{<!DOCTYPE html>}] Dichiara che il documento è HTML5, quindi deve rispettare le regole di questa versione del linguaggio.
  \item[\texttt{<html>}] Elemento radice del documento
  \item[\texttt{<head>}] Contiene metadati e link a risorse
  \item[\texttt{<body>}] Contiene il contenuto visibile
\end{description}

\section{Tag semantici}

HTML5 introduce tag semantici che descrivono il significato del contenuto:

\begin{lstlisting}[language=HTML]
<header>Intestazione del sito</header>
<nav>Barra di navigazione</nav>
<main>Contenuto principale</main>
<section>Una sezione tematica</section>
<article>Un articolo indipendente</article>
<aside>Barra laterale</aside>
<footer>Piè di pagina</footer>
\end{lstlisting}

\begin{attenzione}
La semantica HTML è importante per l'accessibilità: i lettori di schermo leggono il significato dei tag, non solo il testo.
\end{attenzione}

\section{Meta tag essenziali}

I meta tag sono elementi HTML che forniscono metadati sul documento HTML. Questi tag non sono visualizzati nella pagina ma contengono informazioni importanti per browser, motori di ricerca e altri servizi web.

\subsection{Sintassi generale}
Un meta tag ha questa struttura:
\begin{lstlisting}[language=HTML]
<meta name="nome" content="valore">
<!-- oppure -->
<meta http-equiv="nome" content="valore">
<!-- oppure -->
<meta charset="codifica">
\end{lstlisting}

\subsection{Meta tag fondamentali}
Ecco i meta tag più importanti e il loro utilizzo:

\begin{lstlisting}[language=HTML]
<meta charset="UTF-8" />
<meta name="viewport" content="width=device-width, initial-scale=1.0" />
<meta name="description" content="Breve descrizione della pagina" />
<meta name="keywords" content="html, css, web, sviluppo" />
\end{lstlisting}

\begin{description}
  \item[\texttt{charset}] Specifica la codifica dei caratteri del documento. UTF-8 è lo standard moderno che supporta caratteri internazionali e emoji.
  
  \item[\texttt{viewport}] Controlla come la pagina si adatta ai dispositivi mobili. Contiene parametri come \texttt{width=device-width}, che imposta la larghezza della pagina alla larghezza effettiva del dispositivo, garantendo che il layout si adatti correttamente su schermi di diverse dimensioni. Inoltre include \texttt{initial-scale=1.0}, che imposta il livello di zoom iniziale al 100%, assicurando che la pagina non sia ingrandita o rimpicciolita quando viene caricata per la prima volta.
  
  \item[\texttt{description}] Fornisce una breve descrizione della pagina (massimo 155 caratteri). Viene mostrata nei risultati di ricerca di Google.
  
  \item[\texttt{keywords}] Elenca parole chiave rilevanti per la pagina. Oggi ha meno importanza per il SEO ma è ancora utilizzato da alcuni motori di ricerca.
\end{description}

\begin{attenzione}
Il meta tag viewport è cruciale per il responsive design. Senza di esso, i siti mobile-friendly non funzioneranno correttamente sui dispositivi mobili.
\end{attenzione}

\begin{nota}
Il meta tag \texttt{viewport} è fondamentale per il responsive design. Senza di esso, i dispositivi mobili non visualizzeranno la pagina correttamente.
\end{nota}


\section{Riepilogo}

\begin{itemize}
  \item HTML è il linguaggio di marcatura del web
  \item \texttt{<!DOCTYPE html>} identifica la versione HTML5
  \item Tag semantici descrivono il significato del contenuto
  \item Meta tag sono fondamentali per accessibilità e responsive
  \item Sempre validare il codice HTML
\end{itemize}

\chapter{Tag di Blocco e Riga}
\label{cap:tag_blocco_riga}

\section{Differenza tra blocco e riga}

In HTML, i tag sono classificati in due categorie:

\begin{description}
  \item[Block-level] L'elemento occupa tutta la larghezza disponibile e inizia su una nuova riga
  \item[Inline] L'elemento occupa solo lo spazio necessario e rimane sulla stessa riga
\end{description}

\section{Tag di blocco}

I tag di blocco iniziano su una nuova riga:

\begin{lstlisting}[language=HTML]
<!-- Esempio di elementi di blocco HTML -->
<div class="container">Contenitore generico</div>
<p>Questo è un paragrafo di testo</p>
<!-- I titoli hanno diversi livelli di importanza -->
<h1>Titolo principale</h1>
<h2>Titolo secondario</h2>
<section>Una sezione tematica</section>
<article>Un articolo completo</article>
<!-- Le liste possono essere ordinate o non ordinate -->
<ul>
  <li>Lista elemento non ordinato</li>
  <li>Lista elemento non ordinato</li>
</ul>
<ol>
  <li>Lista elemento ordinato</li>
  <li>Lista elemento ordinato</li>
</ol>
\end{lstlisting}

%TODO: Inserire l'immagine del codice sopra

\subsection{Tag blocco comuni}

\begin{itemize}
  \item \texttt{<div>}: Contenitore generico
  \item \texttt{<p>}: Paragrafo
  \item \texttt{<h1>} a \texttt{<h6>}: Heading (titoli)
  \item \texttt{<section>}, \texttt{<article>}, \texttt{<header>}, \texttt{<footer>}: Semantici
  \item \texttt{<ul>}, \texttt{<ol>}, \texttt{<li>}: Liste
  \item \texttt{<blockquote>}: Citazione in blocco
\end{itemize}

\section{Tag inline}

I tag inline non creano nuove righe e occupano solo lo spazio del contenuto:

\begin{lstlisting}[language=HTML]
<span class="highlight">testo importante</span>
<a href="https://example.com">collegamento</a>
<strong>testo forte (bold)</strong>
<em>testo enfatizzato (italic)</em>
<code>codice inline</code>
<img src="immagine.jpg" alt="Descrizione" />
\end{lstlisting}

%TODO: Questa sezione va spostata nei css

\section{Proprietà display CSS}

La proprietà CSS \texttt{display} controlla come un elemento viene visualizzato:

\begin{lstlisting}[language=CSS]
/* Block-level */
display: block;
display: flex;
display: grid;

/* Inline */
display: inline;
display: inline-block;
display: none;
\end{lstlisting}

\begin{nota}
Con \texttt{display: inline-block}, un elemento inline può avere larghezza e altezza come un blocco, ma rimane sulla stessa riga degli elementi vicini.
\end{nota}

\section{Riepilogo}

\begin{itemize}
  \item Tag blocco occupano tutta la larghezza disponibile
  \item Tag inline occupano solo lo spazio del contenuto
  \item \texttt{display: block/inline/inline-block} controlla il comportamento
  \item Proprietà CSS può modificare il comportamento di default
  \item Semantica HTML rimane importante indipendentemente da display
\end{itemize}

\chapter{Form e Input}
\label{cap:form_input}

\section{Introduzione ai Form}

I form (moduli) sono elementi fondamentali del web che permettono l'interazione tra utente e sito web. I form hanno moltissime applicazioni pratiche: permettono di raccogliere dati dagli utenti in contesti come registrazione, login e contatti. Sono utilizzati per implementare funzionalità di ricerca, consentendo agli utenti di trovare contenuti specifici. I form facilitano il caricamento di file, come quando vuoi permettere agli utenti di caricare foto o documenti. Rappresentano il canale principale per raccogliere feedback e commenti dagli utenti, migliorando così il tuo servizio. Nei contesti di e-commerce, i form sono essenziali per completare transazioni come acquisti e prenotazioni. Infine, i form permettono agli utenti di configurare le loro preferenze personali, migliorando l'esperienza d'uso.

Il processo di gestione di un form prevede tre fasi:
\begin{enumerate}
  \item \textbf{Raccolta dati}: L'utente compila i campi del form
  \item \textbf{Invio}: I dati vengono inviati al server tramite una richiesta HTTP
  \item \textbf{Elaborazione}: Il server processa i dati e restituisce una risposta
\end{enumerate}

\begin{nota}
I form sono il ponte tra il frontend (interfaccia utente) e il backend (logica del server). La sicurezza e la validazione dei dati sono cruciali in questo processo.
\end{nota}

\section{Elemento form}

Il tag \texttt{<form>} crea un modulo per raccogliere dati dall'utente. È simile ai metodi di una classe Java (vedi il corso di Quarta sulla programmazione orientata agli oggetti) che accettano parametri:

\begin{lstlisting}[language=HTML]
<form action="paginaDiArrivo.html" method="POST" name="contactForm">
  <label for="email">Email:</label>
  <input type="email" id="email" name="email" required>

  <button type="submit">Invia</button>
  <button type="reset">Cancella</button>
</form>
\end{lstlisting}

\subsection{Attributi form}

\begin{description}
  \item[\texttt{action}] URL dove inviare i dati del form
  \item[\texttt{method}] GET o POST (modalità invio dati)
  \item[\texttt{name}] Nome identificativo del form
  \item[\texttt{enctype}] Tipo di codifica (application/x-www-form-urlencoded, multipart/form-data)
\end{description}

\section{Input types}

HTML5 offre molti tipi di input per diversi dati:
\begin{lstlisting}[language=HTML]
<input type="text" placeholder="Testo generico" />
<input type="password" placeholder="Password" />
<input type="email" placeholder="Email" />
<input type="number" min="0" max="100" step="5" />
<input type="date" />
<input type="checkbox" id="terms" /> <label for="terms">Accetto i termini</label>
<input type="radio" name="option" value="1" id="opt1" /> <label for="opt1">Opzione 1</label>
<select>
  <option>Scegli un'opzione</option>
</select>
<textarea rows="5" cols="40"></textarea>
\end{lstlisting}

%TODO: Inserire immagine del codice sopra


\section{Label e accessibilità}

Il tag \texttt{<label>} associa il testo all'input:

\begin{lstlisting}[language=HTML]
<label for="username">Username:</label>
<input type="text" id="username" name="username">

<label for="subscribe">Iscrivimi alla newsletter</label>
  <input type="checkbox" id="subscribe" name="subscribe">
\end{lstlisting}

\begin{nota}
L'attributo \texttt{for} del label deve corrispondere all'attributo \texttt{id} dell'input per accessibilità screen reader.
\end{nota}

\section{Validazione HTML5}

HTML5 supporta validazione lato client:

\begin{lstlisting}[language=HTML]
<input type="email" required />
<input type="number" min="0" max="10" />
<input type="text" pattern="[A-Za-z]+" title="Solo lettere" />
<input type="password" minlength="8" />
\end{lstlisting}

\begin{attenzione}
La validazione HTML5 non sostituisce la validazione server-side. Sempre convalidare i dati sul server poiché l'utente potrebbe modificare il codice client.
\end{attenzione}

\section{Riepilogo}

\begin{itemize}
  \item I form sono lo strumento principale per l'interazione utente-server
  \item L'elemento \texttt{<form>} ha attributi essenziali:
    \begin{itemize}
      \item \texttt{action}: dove inviare i dati
      \item \texttt{method}: come inviarli (GET/POST)
      \item \texttt{enctype}: tipo di codifica
    \end{itemize}
  \item HTML5 offre diversi tipi di input specializzati:
    \begin{itemize}
      \item \texttt{text}, \texttt{email}, \texttt{password}, \texttt{number}
      \item \texttt{date}, \texttt{checkbox}, \texttt{radio}
      \item \texttt{file}, \texttt{textarea}, \texttt{select}
    \end{itemize}
  \item Ogni \texttt{input} dovrebbe avere un \texttt{label} associato
  \item La validazione client-side (HTML5) va sempre accompagnata da validazione server-side
  \item L'accessibilità è fondamentale (label, attributi \texttt{for} e \texttt{id})
\end{itemize}

%\chapter{CSS Fondamentale}
\label{cap:css_base}

\section{Cos'è CSS?}

CSS (Cascading Style Sheets) è usato per stilizzare gli elementi HTML. Mentre HTML (\ref{cap:intro_html}) descrive la struttura, CSS controlla l'aspetto visivo.

\section{Selettori CSS}

I selettori CSS identificano quali elementi stilizzare:

\begin{lstlisting}[language=CSS]
/* Selettore elemento */
p { color: blue; }

/* Selettore classe */
.highlight { background: yellow; }

/* Selettore ID */
#header { margin: 0; }

/* Selettore attributo */
input[type="email"] { border: 1px solid blue; }

/* Combinatori */
div p { color: red; }        /* Discendente */
div > p { color: green; }    /* Figlio diretto */
h1 + p { color: gray; }      /* Elemento adiacente */
\end{lstlisting}

\section{Specificity}

La specificity determina quale regola CSS ha priorità quando più regole si applicano allo stesso elemento:

\begin{lstlisting}[language=CSS]
/* Specificity: 0,0,1 */
p { color: blue; }

/* Specificity: 0,1,0 */
.class { color: green; }

/* Specificity: 0,1,1 */
p.class { color: red; }

/* Specificity: 1,0,0 */
#id { color: yellow; }
\end{lstlisting}

\begin{nota}
Regola generale della specificity: ID (1,0,0) > Classe (0,1,0) > Elemento (0,0,1). Evita \texttt{!important} quando possibile.
\end{nota}

\section{Box model}

Ogni elemento HTML è un rettangolo composto da:

\begin{lstlisting}[language=CSS]
div {
  width: 300px;
  height: 200px;
  padding: 20px;        /* Spazio interno */
  border: 2px solid;    /* Bordo */
  margin: 10px;         /* Spazio esterno */
  box-sizing: border-box;
}
\end{lstlisting}

\subsection{Componenti del box model}

\begin{description}
  \item[Content] Area con il contenuto
  \item[Padding] Spazio interno tra content e border
  \item[Border] Linea intorno al padding
  \item[Margin] Spazio esterno tra border e elementi vicini
\end{description}

\section{Proprietà CSS comuni}

\begin{itemize}
  \item \texttt{color}: Colore del testo
  \item \texttt{background-color}: Colore sfondo
  \item \texttt{font-size}: Dimensione font
  \item \texttt{text-align}: Allineamento testo (left, center, right, justify)
  \item \texttt{display}: Tipo di visualizzazione (block, inline, flex, grid)
  \item \texttt{width}, \texttt{height}: Dimensioni elemento
  \item \texttt{border}: Bordo elemento
  \item \texttt{box-shadow}: Ombra elemento
\end{itemize}

\section{Unità di misura}

\begin{description}
  \item[px] Pixel (unità assoluta, non scalare)
  \item[em] Relativo alla font-size del genitore
  \item[rem] Relativo alla font-size dell'elemento radice (html)
  \item[\%] Percentuale del genitore
  \item[vw/vh] Viewport width/height (schermo utente)
\end{description}

\begin{attenzione}
Preferisci \texttt{rem} e \texttt{\%} a \texttt{px} per layout più responsive e accessibili.
\end{attenzione}

\section{Esercizi}

\subsection{Esercizio 1 (Base)}
Calcola la specificity di vari selettori CSS. Dato: \texttt{div.card p.text}, \texttt{\#header .nav}, \texttt{p}, \texttt{.class} - classifica per specificity.

\subsection{Esercizio 2 (Intermedio)}
Crea una pagina stilizzata usando selettori diversi e proprietà box model. Includi: colori, padding, margin, border, width.

\subsection{Esercizio 3 (Avanzato)}
Crea un layout complesso usando CSS senza framework. Includi: card, header, navigation bar, footer con stili responsive.

\section{Esercizio Pratico Completo: Card Portfolio}

Crea una pagina con card stilizzate per mostrare progetti personali.

\subsection{Requisiti}

\begin{itemize}
  \item 3 card con: immagine, titolo, descrizione, link
  \item Card hanno padding, border, box-shadow
  \item Responsive: ridimensionati per mobile
  \item Colori coerenti
  \item Hover effects (bonus)
\end{itemize}

\subsection{Soluzione HTML}

\begin{lstlisting}[language=HTML]
<!DOCTYPE html>
<html lang="it">
<head>
  <meta charset="UTF-8" />
  <meta name="viewport" content="width=device-width, initial-scale=1.0" />
  <title>Portfolio</title>
  <link rel="stylesheet" href="style.css" />
</head>
<body>
  <div class="container">
    <h1>Miei Progetti</h1>

    <div class="card">
      <img src="project1.jpg" alt="Progetto 1">
      <div class="card-content">
        <h2>Sito di Vendita</h2>
        <p>Un e-commerce moderno con Flexbox e Grid</p>
        <a href="/progetti/ecommerce.html" class="btn">Visualizza</a>
      </div>
    </div>

    <div class="card">
      <img src="project2.jpg" alt="Progetto 2">
      <div class="card-content">
        <h2>Blog Responsivo</h2>
        <p>Blog con SCSS e media queries</p>
        <a href="/progetti/blog/index.html" class="btn">Visualizza</a>
      </div>
    </div>

    <div class="card">
      <img src="project3.jpg" alt="Progetto 3">
      <div class="card-content">
        <h2>App Todo</h2>
        <p>Todo list con JavaScript puro</p>
        <a href="/progetti/todo-app/demo.html" class="btn">Visualizza</a>
      </div>
    </div>
  </div>
</body>
</html>
\end{lstlisting}

\subsection{Soluzione CSS (style.css)}

\begin{lstlisting}[language=CSS]
/* Reset */
* {
  margin: 0;
  padding: 0;
  box-sizing: border-box;
}

body {
  font-family: Arial, sans-serif;
  background-color: #f5f5f5;
  padding: 20px;
}

.container {
  max-width: 1200px;
  margin: 0 auto;
}

h1 {
  text-align: center;
  margin-bottom: 40px;
  color: #333;
}

/* Card styling */
.card {
  background-color: white;
  border-radius: 8px;
  overflow: hidden;
  box-shadow: 0 2px 10px rgba(0, 0, 0, 0.1);
  margin-bottom: 20px;
  transition: transform 0.3s ease;
}

.card:hover {
  transform: translateY(-5px);
  box-shadow: 0 5px 20px rgba(0, 0, 0, 0.2);
}

.card img {
  width: 100%;
  height: 250px;
  object-fit: cover;
}

.card-content {
  padding: 20px;
}

.card h2 {
  color: #333;
  margin-bottom: 10px;
}

.card p {
  color: #666;
  margin-bottom: 15px;
  line-height: 1.5;
}

.btn {
  display: inline-block;
  padding: 10px 20px;
  background-color: #007bff;
  color: white;
  text-decoration: none;
  border-radius: 4px;
  transition: background-color 0.3s;
}

.btn:hover {
  background-color: #0056b3;
}

/* Responsive per mobile */
@media (max-width: 768px) {
  .container {
    padding: 0;
  }

  h1 {
    font-size: 1.5rem;
  }

  .card {
    margin-bottom: 15px;
  }
}
\end{lstlisting}

\subsection{Spiegazione Dettagliata}

\subsubsection{Reset CSS}
\texttt{* { margin: 0; padding: 0; }} rimuove margini/padding di default del browser. \texttt{box-sizing: border-box} fa sì che padding/border siano inclusi nella larghezza.

\subsubsection{Card Styling}
\begin{itemize}
  \item \texttt{border-radius: 8px} arrotonda gli angoli
  \item \texttt{box-shadow} crea ombra sottile (4 valori: offset-x, offset-y, blur, color)
  \item \texttt{overflow: hidden} taglia l'immagine agli angoli arrotondati
  \item \texttt{transition} crea effetti smooth
\end{itemize}

\subsubsection{Hover Effects}
\texttt{:hover} applica stili quando il mouse passa sulla card. \texttt{transform: translateY(-5px)} sposta la card di 5px verso l'alto.

\subsubsection{Responsive}
\texttt{@media (max-width: 768px)} applica stili per schermi piccoli. Card diventano a schermo intero su mobile.

\subsection{Estensioni}

\begin{enumerate}
  \item Usa Flexbox per disporre card in riga (vedi Cap. 5)
  \item Aggiungi più colori diversi per ogni card
  \item Crea varianti (primaria, secondaria, success) usando classi
  \item Implementa dark mode con media query \texttt{prefers-color-scheme}
\end{enumerate}

\section{Riepilogo}

\begin{itemize}
  \item CSS stila gli elementi HTML selezionandoli con selettori
  \item Specificity: ID > Classe > Elemento
  \item Box model: content, padding, border, margin
  \item Proprietà CSS controllano aspetto visivo
  \item Unità di misura: px, em, rem, \%, vw/vh
\end{itemize}

%\chapter{Flexbox e Grid}
\label{cap:flexbox_grid}

\section{Introduzione ai Layout Moderni}

Prima di Flexbox e Grid, i layout CSS erano implementati con tecniche complesse e poco intuitive: float, position, table-cell. I layout moderni CSS3 risolvono questi problemi offrendo:

\begin{itemize}
  \item \textbf{Flexbox}: Layout monodimensionale (riga o colonna)
  \item \textbf{CSS Grid}: Layout bidimensionale (righe e colonne simultaneamente)
  \item \textbf{Allineamento intuitivo}: Centramento verticale/orizzontale semplice
  \item \textbf{Responsività}: Adattamento automatico a dimensioni schermo
  \item \textbf{Ordinamento visuale}: Cambio ordine elementi senza modificare HTML
\end{itemize}

\begin{nota}
Flexbox e Grid non sono in competizione: sono complementari. Usa Flexbox per componenti (navbar, card) e Grid per layout di pagina completi.
\end{nota}

\section{Flexbox}

\subsection{Concetti Fondamentali}

Flexbox (Flexible Box Layout) è un modello di layout monodimensionale che permette di distribuire spazio e allineare elementi lungo un \textbf{asse principale} (main axis) e un \textbf{asse trasversale} (cross axis).

\subsubsection{Terminologia Flexbox}

\begin{description}
  \item[Flex container] Elemento genitore con \texttt{display: flex} o \texttt{display: inline-flex}
  \item[Flex items] Elementi figli diretti del container
  \item[Main axis] Asse principale (orizzontale con row, verticale con column)
  \item[Cross axis] Asse perpendicolare al main axis
  \item[Main start/end] Inizio e fine dell'asse principale
  \item[Cross start/end] Inizio e fine dell'asse trasversale
\end{description}

\subsection{Proprietà del Flex Container}

\subsubsection{1. display}

Attiva il contesto Flexbox:

\begin{lstlisting}[language=CSS]
.container {
  display: flex;        /* Block-level flex container */
  /* oppure */
  display: inline-flex; /* Inline-level flex container */
}
\end{lstlisting}

\subsubsection{2. flex-direction}

Definisce la direzione dell'asse principale:

\begin{lstlisting}[language=CSS]
.container {
  flex-direction: row;            /* Default: sinistra → destra */
  flex-direction: row-reverse;    /* destra → sinistra */
  flex-direction: column;         /* alto → basso */
  flex-direction: column-reverse; /* basso → alto */
}
\end{lstlisting}

\subsubsection{3. flex-wrap}

Controlla se gli elementi vanno a capo:

\begin{lstlisting}[language=CSS]
.container {
  flex-wrap: nowrap;       /* Default: tutti su una riga */
  flex-wrap: wrap;         /* Wrapping su più righe */
  flex-wrap: wrap-reverse; /* Wrapping invertito */
}
\end{lstlisting}

\subsubsection{4. justify-content}

Allineamento lungo l'asse principale:

\begin{lstlisting}[language=CSS]
.container {
  justify-content: flex-start;    /* Inizio (default) */
  justify-content: flex-end;      /* Fine */
  justify-content: center;        /* Centro */
  justify-content: space-between; /* Spazio tra elementi */
  justify-content: space-around;  /* Spazio intorno elementi */
  justify-content: space-evenly;  /* Spazio uniforme */
}
\end{lstlisting}

\textbf{Differenze}:
\begin{itemize}
  \item \texttt{space-between}: Primo e ultimo elemento attaccati ai bordi
  \item \texttt{space-around}: Spazio uguale intorno a ogni elemento (bordi = metà spazio)
  \item \texttt{space-evenly}: Spazio identico ovunque (inclusi bordi)
\end{itemize}

\subsubsection{5. align-items}

Allineamento lungo l'asse trasversale:

\begin{lstlisting}[language=CSS]
.container {
  align-items: stretch;    /* Default: riempie altezza */
  align-items: flex-start; /* Inizio cross axis */
  align-items: flex-end;   /* Fine cross axis */
  align-items: center;     /* Centro cross axis */
  align-items: baseline;   /* Allinea baseline testo */
}
\end{lstlisting}

\subsubsection{6. align-content}

Allinea righe multiple quando c'è wrapping:

\begin{lstlisting}[language=CSS]
.container {
  flex-wrap: wrap;
  align-content: flex-start;
  align-content: flex-end;
  align-content: center;
  align-content: space-between;
  align-content: space-around;
  align-content: stretch; /* Default */
}
\end{lstlisting}

\begin{attenzione}
\texttt{align-content} funziona solo con \texttt{flex-wrap: wrap} e quando ci sono più righe.
\end{attenzione}

\subsubsection{7. gap}

Spazio tra elementi (introdotto in CSS3):

\begin{lstlisting}[language=CSS]
.container {
  display: flex;
  gap: 20px;           /* Spazio uniforme */
  gap: 20px 10px;      /* row-gap column-gap */
  row-gap: 20px;
  column-gap: 10px;
}
\end{lstlisting}

\subsection{Proprietà dei Flex Items}

\subsubsection{1. flex-grow}

Fattore di crescita (default 0):

\begin{lstlisting}[language=CSS]
.item1 { flex-grow: 0; } /* Non cresce */
.item2 { flex-grow: 1; } /* Cresce proporzionalmente */
.item3 { flex-grow: 2; } /* Cresce doppio rispetto a item2 */
\end{lstlisting}

Se spazio disponibile = 300px e item2 ha \texttt{flex-grow: 1} e item3 ha \texttt{flex-grow: 2}:
\begin{itemize}
  \item item2 riceve: 300px × (1/3) = 100px
  \item item3 riceve: 300px × (2/3) = 200px
\end{itemize}

\subsubsection{2. flex-shrink}

Fattore di riduzione quando non c'è spazio (default 1):

\begin{lstlisting}[language=CSS]
.item {
  flex-shrink: 1; /* Default: si riduce */
  flex-shrink: 0; /* Non si riduce mai */
}
\end{lstlisting}

\subsubsection{3. flex-basis}

Dimensione base dell'elemento prima della distribuzione spazio:

\begin{lstlisting}[language=CSS]
.item {
  flex-basis: auto;  /* Default: dimensione contenuto */
  flex-basis: 200px; /* Larghezza base fissa */
  flex-basis: 50%;   /* Percentuale del container */
}
\end{lstlisting}

\subsubsection{4. flex (shorthand)}

Combina grow, shrink, basis:

\begin{lstlisting}[language=CSS]
.item {
  flex: 1;              /* grow=1, shrink=1, basis=0% */
  flex: 0 1 auto;       /* Default: grow=0, shrink=1, basis=auto */
  flex: 2 1 300px;      /* grow=2, shrink=1, basis=300px */
  flex: none;           /* 0 0 auto (rigido) */
}
\end{lstlisting}

\textbf{Valori comuni}:
\begin{itemize}
  \item \texttt{flex: 1} → Elementi con larghezza uguale e flessibili
  \item \texttt{flex: none} → Elementi rigidi (dimensione contenuto)
  \item \texttt{flex: 0 1 auto} → Elementi che si riducono ma non crescono
\end{itemize}

\subsubsection{5. align-self}

Override di \texttt{align-items} per singolo elemento:

\begin{lstlisting}[language=CSS]
.container {
  align-items: flex-start; /* Tutti allineati all'inizio */
}

.special-item {
  align-self: center; /* Solo questo centrato */
}
\end{lstlisting}

\subsubsection{6. order}

Cambia ordine visuale (default 0):

\begin{lstlisting}[language=CSS]
.item1 { order: 3; }
.item2 { order: 1; }
.item3 { order: 2; }
/* Ordine visuale: item2 → item3 → item1 */
\end{lstlisting}

\subsection{Esempi Pratici Flexbox}

\subsubsection{Esempio 1: Navbar Responsive}

\begin{lstlisting}[language=HTML]
<nav class="navbar">
  <div class="logo">MySite</div>
  <ul class="menu">
    <li><a href="#">Home</a></li>
    <li><a href="#">About</a></li>
    <li><a href="#">Contact</a></li>
  </ul>
</nav>
\end{lstlisting}

\begin{lstlisting}[language=CSS]
.navbar {
  display: flex;
  justify-content: space-between;
  align-items: center;
  padding: 1rem 2rem;
  background-color: #333;
}

.logo {
  font-size: 1.5rem;
  color: white;
}

.menu {
  display: flex;
  gap: 2rem;
  list-style: none;
}

.menu a {
  color: white;
  text-decoration: none;
}
\end{lstlisting}

\subsubsection{Esempio 2: Card Layout Responsive}

\begin{lstlisting}[language=HTML]
<div class="card-container">
  <div class="card">Card 1</div>
  <div class="card">Card 2</div>
  <div class="card">Card 3</div>
  <div class="card">Card 4</div>
</div>
\end{lstlisting}

\begin{lstlisting}[language=CSS]
.card-container {
  display: flex;
  flex-wrap: wrap;
  gap: 20px;
  padding: 20px;
}

.card {
  flex: 1 1 300px;  /* Cresce, si riduce, base 300px */
  background: #f0f0f0;
  padding: 2rem;
  border-radius: 8px;
}

/* Su mobile: 1 colonna */
@media (max-width: 768px) {
  .card {
    flex: 1 1 100%;
  }
}
\end{lstlisting}

\subsubsection{Esempio 3: Centramento Perfetto}

\begin{lstlisting}[language=CSS]
.container {
  display: flex;
  justify-content: center; /* Centro orizzontale */
  align-items: center;     /* Centro verticale */
  height: 100vh;           /* Altezza viewport */
}

.box {
  width: 200px;
  height: 200px;
  background: #3498db;
  color: white;
}
\end{lstlisting}

\section{CSS Grid}

\subsection{Concetti Fondamentali}

CSS Grid è un sistema di layout bidimensionale che permette di creare layout complessi con righe e colonne simultaneamente.

\subsubsection{Terminologia Grid}

\begin{description}
  \item[Grid container] Elemento con \texttt{display: grid}
  \item[Grid items] Elementi figli diretti
  \item[Grid line] Linee di divisione orizzontali/verticali (numerate da 1)
  \item[Grid track] Spazio tra due linee (riga o colonna)
  \item[Grid cell] Singola cella (intersezione riga-colonna)
  \item[Grid area] Area rettangolare composta da celle
\end{description}

\subsection{Proprietà del Grid Container}

\subsubsection{1. display}

\begin{lstlisting}[language=CSS]
.container {
  display: grid;        /* Block-level grid */
  display: inline-grid; /* Inline-level grid */
}
\end{lstlisting}

\subsubsection{2. grid-template-columns/rows}

Definisce dimensioni colonne e righe:

\begin{lstlisting}[language=CSS]
.container {
  /* 3 colonne: 200px, 1fr, 2fr */
  grid-template-columns: 200px 1fr 2fr;

  /* 3 righe: auto, 300px, auto */
  grid-template-rows: auto 300px auto;

  /* Ripeti pattern: 4 colonne uguali */
  grid-template-columns: repeat(4, 1fr);

  /* Auto-fill responsive */
  grid-template-columns: repeat(auto-fill, minmax(200px, 1fr));
}
\end{lstlisting}

\textbf{Unità di misura}:
\begin{itemize}
  \item \texttt{px, \%, em, rem}: Unità fisse o relative
  \item \texttt{fr}: Frazione dello spazio disponibile (flessibile)
  \item \texttt{auto}: Dimensione basata su contenuto
  \item \texttt{minmax(min, max)}: Range di dimensioni
  \item \texttt{min-content, max-content}: Basato su contenuto
\end{itemize}

\subsubsection{3. grid-template-areas}

Definisce aree nominate:

\begin{lstlisting}[language=CSS]
.container {
  display: grid;
  grid-template-columns: 200px 1fr 200px;
  grid-template-rows: auto 1fr auto;
  grid-template-areas:
    "header header  header"
    "sidebar main   ads"
    "footer footer  footer";
}

.header  { grid-area: header; }
.sidebar { grid-area: sidebar; }
.main    { grid-area: main; }
.ads     { grid-area: ads; }
.footer  { grid-area: footer; }
\end{lstlisting}

\begin{nota}
\texttt{grid-template-areas} rende il codice molto leggibile e mantenibile. Puoi "vedere" il layout nel CSS.
\end{nota}

\subsubsection{4. gap}

Spazio tra righe e colonne:

\begin{lstlisting}[language=CSS]
.container {
  gap: 20px;           /* row-gap e column-gap uguali */
  gap: 20px 10px;      /* row-gap column-gap */
  row-gap: 20px;
  column-gap: 10px;
}
\end{lstlisting}

\subsubsection{5. justify-items / align-items}

Allineamento elementi dentro celle:

\begin{lstlisting}[language=CSS]
.container {
  /* Allineamento orizzontale */
  justify-items: start | end | center | stretch;

  /* Allineamento verticale */
  align-items: start | end | center | stretch;
}
\end{lstlisting}

\subsubsection{6. justify-content / align-content}

Allineamento griglia dentro container:

\begin{lstlisting}[language=CSS]
.container {
  width: 100%;
  height: 100vh;

  /* Allinea griglia orizzontalmente */
  justify-content: start | end | center | space-between | space-around;

  /* Allinea griglia verticalmente */
  align-content: start | end | center | space-between | space-around;
}
\end{lstlisting}

\subsection{Proprietà dei Grid Items}

\subsubsection{1. grid-column / grid-row}

Posizionamento esplicito:

\begin{lstlisting}[language=CSS]
.item {
  /* Colonne da linea 1 a 3 (occupa 2 colonne) */
  grid-column: 1 / 3;
  grid-column-start: 1;
  grid-column-end: 3;

  /* Righe da linea 2 a 4 */
  grid-row: 2 / 4;

  /* Span syntax: occupa 2 colonne */
  grid-column: span 2;

  /* Combinato */
  grid-column: 1 / span 2; /* Inizia a 1, occupa 2 */
}
\end{lstlisting}

\subsubsection{2. grid-area}

Shorthand per posizionamento:

\begin{lstlisting}[language=CSS]
.item {
  /* grid-row-start / grid-column-start / grid-row-end / grid-column-end */
  grid-area: 1 / 1 / 3 / 4;

  /* Oppure usa nome area */
  grid-area: header;
}
\end{lstlisting}

\subsubsection{3. justify-self / align-self}

Override allineamento per singolo elemento:

\begin{lstlisting}[language=CSS]
.special-item {
  justify-self: center; /* Orizzontale */
  align-self: end;      /* Verticale */
}
\end{lstlisting}

\subsection{Esempi Pratici Grid}

\subsubsection{Esempio 1: Layout Pagina Completo}

\begin{lstlisting}[language=HTML]
<div class="page-layout">
  <header>Header</header>
  <nav>Navigation</nav>
  <main>Main Content</main>
  <aside>Sidebar</aside>
  <footer>Footer</footer>
</div>
\end{lstlisting}

\begin{lstlisting}[language=CSS]
.page-layout {
  display: grid;
  grid-template-columns: 200px 1fr 250px;
  grid-template-rows: auto 1fr auto;
  grid-template-areas:
    "header  header  header"
    "nav     main    aside"
    "footer  footer  footer";
  gap: 20px;
  height: 100vh;
}

header { grid-area: header; background: #333; color: white; padding: 1rem; }
nav    { grid-area: nav; background: #f0f0f0; padding: 1rem; }
main   { grid-area: main; background: white; padding: 2rem; }
aside  { grid-area: aside; background: #f9f9f9; padding: 1rem; }
footer { grid-area: footer; background: #333; color: white; padding: 1rem; }

@media (max-width: 768px) {
  .page-layout {
    grid-template-columns: 1fr;
    grid-template-areas:
      "header"
      "nav"
      "main"
      "aside"
      "footer";
  }
}
\end{lstlisting}

\subsubsection{Esempio 2: Gallery Responsive}

\begin{lstlisting}[language=CSS]
.gallery {
  display: grid;
  grid-template-columns: repeat(auto-fill, minmax(250px, 1fr));
  gap: 20px;
  padding: 20px;
}

.gallery img {
  width: 100%;
  height: 250px;
  object-fit: cover;
  border-radius: 8px;
}
\end{lstlisting}

\textbf{Spiegazione \texttt{auto-fill}}:
\begin{itemize}
  \item \texttt{auto-fill}: Crea tante colonne quante ci stanno (min 250px)
  \item \texttt{minmax(250px, 1fr)}: Colonne tra 250px e 1fr
  \item Risultato: Gallery completamente responsive senza media queries
\end{itemize}

\subsubsection{Esempio 3: Dashboard Card Layout}

\begin{lstlisting}[language=CSS]
.dashboard {
  display: grid;
  grid-template-columns: repeat(4, 1fr);
  grid-template-rows: repeat(3, 200px);
  gap: 20px;
}

.card-large {
  grid-column: span 2;
  grid-row: span 2;
}

.card-wide {
  grid-column: span 2;
}

.card-tall {
  grid-row: span 2;
}
\end{lstlisting}

\section{Flexbox vs Grid: Quando Usarli}

\begin{table}[h]
\centering
\begin{tabular}{|p{5cm}|p{5cm}|}
\hline
\textbf{Flexbox} & \textbf{Grid} \\
\hline
Layout monodimensionale (riga o colonna) & Layout bidimensionale (righe e colonne) \\
\hline
Componenti UI (navbar, menu, card) & Layout pagina completa \\
\hline
Distribuzione spazio dinamica & Posizionamento esplicito elementi \\
\hline
Allineamento semplice & Sovrapposizione elementi possibile \\
\hline
Crescita/riduzione flessibile & Griglia fissa o responsive \\
\hline
Content-first (dimensione da contenuto) & Layout-first (struttura predefinita) \\
\hline
\end{tabular}
\caption{Confronto Flexbox vs CSS Grid}
\end{table}

\subsection{Quando Usare Flexbox}

\begin{itemize}
  \item Navbar orizzontale o verticale
  \item Allineamento elementi in fila
  \item Centrare elementi (orizzontale/verticale)
  \item Card con altezza uguale
  \item Form con label e input allineati
  \item Footer con elementi distribuiti
\end{itemize}

\subsection{Quando Usare Grid}

\begin{itemize}
  \item Layout pagina completo (header, sidebar, main, footer)
  \item Gallery di immagini
  \item Dashboard con card di dimensioni diverse
  \item Griglia prodotti e-commerce
  \item Calendario o tabella
  \item Magazine-style layout
\end{itemize}

\subsection{Combinazione Flexbox + Grid}

Spesso si usano insieme:

\begin{lstlisting}[language=CSS]
/* Grid per layout pagina */
.page {
  display: grid;
  grid-template-areas: "header" "main" "footer";
}

/* Flexbox per navbar dentro header */
.header {
  display: flex;
  justify-content: space-between;
  align-items: center;
}

/* Grid per gallery dentro main */
.gallery {
  display: grid;
  grid-template-columns: repeat(auto-fill, minmax(200px, 1fr));
}
\end{lstlisting}

\section{Funzioni CSS Avanzate per Grid}

\subsection{repeat()}

\begin{lstlisting}[language=CSS]
/* 4 colonne uguali */
grid-template-columns: repeat(4, 1fr);

/* Equivalente a: */
grid-template-columns: 1fr 1fr 1fr 1fr;

/* Ripeti pattern complesso */
grid-template-columns: repeat(2, 100px 1fr);
/* Risultato: 100px 1fr 100px 1fr */
\end{lstlisting}

\subsection{minmax()}

\begin{lstlisting}[language=CSS]
/* Colonne tra 200px e 400px */
grid-template-columns: repeat(3, minmax(200px, 400px));

/* Responsive: minimo 250px, massimo 1fr (flessibile) */
grid-template-columns: repeat(auto-fill, minmax(250px, 1fr));
\end{lstlisting}

\subsection{auto-fill vs auto-fit}

\begin{lstlisting}[language=CSS]
/* auto-fill: Crea celle extra vuote se c'è spazio */
grid-template-columns: repeat(auto-fill, minmax(200px, 1fr));

/* auto-fit: Espande celle esistenti per riempire spazio */
grid-template-columns: repeat(auto-fit, minmax(200px, 1fr));
\end{lstlisting}

\textbf{Differenza}:
\begin{itemize}
  \item \texttt{auto-fill}: 3 elementi su schermo largo → crea 3 colonne + celle vuote
  \item \texttt{auto-fit}: 3 elementi su schermo largo → 3 colonne espanse al 100\%
\end{itemize}

\section{Esercizi}

\subsection{Esercizio 1 (Base): Navbar Flexbox}

Crea una navbar usando Flexbox con:
\begin{itemize}
  \item Logo a sinistra
  \item Menu con link centrati
  \item Pulsante "Login" a destra
  \item Allineamento verticale centrato
  \item Gap 2rem tra elementi menu
\end{itemize}

\subsection{Esercizio 2 (Base): Card Layout Flexbox}

Crea un container con 4 card usando Flexbox:
\begin{itemize}
  \item Card con altezza uguale
  \item Wrapping su mobile (max-width 768px)
  \item Gap 20px tra card
  \item Ogni card con flex-basis 300px
\end{itemize}

\subsection{Esercizio 3 (Intermedio): Griglia Prodotti Grid}

Crea una griglia 3x3 di prodotti usando CSS Grid:
\begin{itemize}
  \item 3 colonne uguali su desktop
  \item 2 colonne su tablet (max 1024px)
  \item 1 colonna su mobile (max 768px)
  \item Gap 20px
  \item Un prodotto "featured" che occupa 2 colonne
\end{itemize}

\subsection{Esercizio 4 (Intermedio): Dashboard Layout}

Crea un dashboard usando Grid con:
\begin{itemize}
  \item Header full-width (grid-column: 1 / -1)
  \item Sidebar sinistra 250px
  \item Main content area (1fr)
  \item Widget area destra 300px
  \item Footer full-width
  \item Altezza totale 100vh
\end{itemize}

\subsection{Esercizio 5 (Avanzato): Layout Magazine}

Crea un layout magazine-style con Grid:
\begin{itemize}
  \item 1 articolo grande (2 colonne × 2 righe)
  \item 4 articoli medi (1 colonna × 1 riga)
  \item Responsive: mobile 1 colonna, tablet 2 colonne, desktop 4 colonne
  \item Usa \texttt{grid-template-areas}
\end{itemize}

\subsection{Esercizio 6 (Avanzato): Gallery Responsive Auto}

Crea una gallery completamente responsive usando:
\begin{itemize}
  \item \texttt{grid-template-columns: repeat(auto-fit, minmax(250px, 1fr))}
  \item Aspect ratio 1:1 per immagini
  \item Hover effect: scale(1.05)
  \item Gap 15px
  \item Nessuna media query richiesta
\end{itemize}

\subsection{Esercizio 7 (Avanzato): Combinazione Flexbox + Grid}

Crea una pagina completa combinando Flexbox e Grid:
\begin{itemize}
  \item Grid per layout pagina (header, main, footer)
  \item Flexbox per navbar dentro header
  \item Grid per gallery prodotti dentro main
  \item Flexbox per footer con 3 sezioni distribuite
  \item Responsive su mobile: tutto verticale
\end{itemize}

\section{Best Practices}

\subsection{Flexbox}

\begin{itemize}
  \item Usa \texttt{flex} shorthand invece di grow/shrink/basis separati
  \item Usa \texttt{gap} invece di margin per spaziatura
  \item Mobile-first: \texttt{flex-direction: column} su mobile, \texttt{row} su desktop
  \item Evita \texttt{justify-content: space-between} con numero variabile elementi
  \item Per centramento: \texttt{justify-content: center; align-items: center;}
\end{itemize}

\subsection{Grid}

\begin{itemize}
  \item Usa \texttt{fr} per layout flessibili, \texttt{px} per dimensioni fisse
  \item Preferisci \texttt{grid-template-areas} per layout complessi (leggibilità)
  \item Usa \texttt{auto-fit/auto-fill} per gallery responsive senza media queries
  \item Definisci grid gap a livello container, non margin su item
  \item Nomina le grid lines per posizionamento più chiaro
\end{itemize}

\subsection{Generale}

\begin{itemize}
  \item Flexbox per componenti, Grid per layout
  \item Combina i due sistemi quando necessario
  \item Testa su browser diversi (supporto IE11 limitato)
  \item Usa CSS Grid per overlay/sovrapposizione elementi
  \item Mobile-first: layout semplice su mobile, complesso su desktop
\end{itemize}

\section{Supporto Browser}

\begin{itemize}
  \item \textbf{Flexbox}: Supportato da IE11+, tutti i browser moderni
  \item \textbf{Grid}: Supportato da IE11 (con prefissi -ms-), tutti i browser moderni
  \item \textbf{gap in Flexbox}: Chrome 84+, Firefox 63+, Safari 14.1+ (2021)
  \item \textbf{Fallback}: Usa feature queries \texttt{@supports}
\end{itemize}

\begin{lstlisting}[language=CSS]
/* Fallback per browser senza grid */
.container {
  display: flex; /* Fallback */
}

@supports (display: grid) {
  .container {
    display: grid;
    grid-template-columns: repeat(3, 1fr);
  }
}
\end{lstlisting}

\section{Risorse Utili}

\begin{itemize}
  \item \textbf{Flexbox Froggy}: \url{https://flexboxfroggy.com/} (gioco interattivo)
  \item \textbf{Grid Garden}: \url{https://cssgridgarden.com/}
  \item \textbf{CSS-Tricks Flexbox Guide}: \url{https://css-tricks.com/snippets/css/a-guide-to-flexbox/}
  \item \textbf{CSS-Tricks Grid Guide}: \url{https://css-tricks.com/snippets/css/complete-guide-grid/}
  \item \textbf{MDN Flexbox}: \url{https://developer.mozilla.org/en-US/docs/Web/CSS/CSS_Flexible_Box_Layout}
  \item \textbf{MDN Grid}: \url{https://developer.mozilla.org/en-US/docs/Web/CSS/CSS_Grid_Layout}
\end{itemize}

\section{Riepilogo}

\subsection{Flexbox}

\begin{itemize}
  \item Layout monodimensionale (riga o colonna)
  \item Proprietà container: display, flex-direction, flex-wrap, justify-content, align-items, align-content, gap
  \item Proprietà item: flex-grow, flex-shrink, flex-basis, flex (shorthand), order, align-self
  \item Perfetto per navbar, card, allineamenti semplici
  \item \texttt{justify-content}: asse principale, \texttt{align-items}: asse trasversale
\end{itemize}

\subsection{Grid}

\begin{itemize}
  \item Layout bidimensionale (righe e colonne)
  \item Proprietà container: display, grid-template-columns/rows, grid-template-areas, gap, justify/align-items, justify/align-content
  \item Proprietà item: grid-column/row, grid-area, justify/align-self
  \item Funzioni: repeat(), minmax(), auto-fill, auto-fit
  \item Perfetto per layout pagina, gallery, dashboard
  \item \texttt{fr}: frazione spazio disponibile
\end{itemize}

\subsection{Quando Usare}

\begin{itemize}
  \item \textbf{Flexbox}: Componenti UI, allineamento monodimensionale, distribuzione spazio
  \item \textbf{Grid}: Layout completi, posizionamento bidimensionale, griglie
  \item \textbf{Insieme}: Grid per struttura pagina + Flexbox per componenti interni
\end{itemize}

%\chapter{Design Responsivo}
\label{cap:responsive}

\section{Mobile-first approach}

Il design responsivo inizia dai dispositivi mobili (320px) e scala verso schermi più grandi:

\begin{lstlisting}[language=CSS]
/* Mobile first: base per tutti */
body { font-size: 14px; }
.container { width: 100%; }

/* Tablet: 768px e oltre */
@media (min-width: 768px) {
  body { font-size: 16px; }
  .container { width: 90%; }
}

/* Desktop: 1024px e oltre */
@media (min-width: 1024px) {
  body { font-size: 18px; }
  .container { width: 80%; }
  max-width: 1200px;
}
\end{lstlisting}

\section{Meta tag viewport}

Essenziale per responsività:

\begin{lstlisting}[language=HTML]
<meta name="viewport" content="width=device-width, initial-scale=1.0" />
\end{lstlisting}

\begin{attenzione}
Senza il meta tag viewport, i dispositivi mobili mostreranno la pagina desktop rimpicciolita, non responsive.
\end{attenzione}

\section{Breakpoints standard}

\begin{itemize}
  \item \textbf{Mobile}: 320px - 480px
  \item \textbf{Tablet}: 768px - 1024px
  \item \textbf{Desktop}: 1024px+
  \item \textbf{Large}: 1920px+
\end{itemize}

\section{Immagini responsive}

\begin{lstlisting}[language=CSS]
img {
  max-width: 100%;
  height: auto;
}
\end{lstlisting}

\begin{lstlisting}[language=HTML]
<picture>
  <source media="(min-width: 1024px)" srcset="large.jpg" />
  <source media="(min-width: 768px)" srcset="medium.jpg" />
  <img src="small.jpg" alt="Descrizione" />
</picture>
\end{lstlisting}

\section{Esercizi}

\subsection{Esercizio 1 (Base)}
Crea una pagina con 2 colonne che diventa 1 colonna su mobile (max 768px).

\subsection{Esercizio 2 (Intermedio)}
Implementa 3 breakpoint diversi (320, 768, 1024) con layout e font-size che cambiano.

\subsection{Esercizio 3 (Avanzato)}
Crea un sito completo responsivo con navbar mobile-first, menu hamburger (CSS), e layout che adatta a tutti i dispositivi.

\section{Riepilogo}

\begin{itemize}
  \item Mobile-first: inizia da mobile e scala
  \item Meta viewport è obbligatorio
  \item Media queries: @media (min-width: ...)
  \item Breakpoint standard: 320, 768, 1024, 1920
  \item Immagini: max-width 100%, height auto
\end{itemize}

%\chapter{SASS e SCSS}
\label{cap:sass_scss}

\section{Cos'è SASS/SCSS?}

SASS (Syntactically Awesome StyleSheets) è un preprocessore CSS. SCSS è la sintassi moderna di SASS che è un superset di CSS (tutto il CSS valido è valido SCSS).

\section{Variabili}

Le variabili SCSS permettono di riutilizzare valori:

\begin{lstlisting}[language=CSS]
$primary-color: #3498db;
$secondary-color: #2ecc71;
$base-font-size: 16px;
$spacing: 20px;

body {
  font-size: $base-font-size;
  color: $primary-color;
}

.button {
  background-color: $primary-color;
  padding: $spacing;
  margin: $spacing / 2;
}
\end{lstlisting}

\section{Nesting}

Nesting riduce ripetizione di selettori:

\begin{lstlisting}[language=CSS]
.navbar {
  background-color: $primary-color;
  padding: $spacing;

  .logo {
    font-size: 24px;
    color: white;
  }

  .menu {
    display: flex;
    gap: $spacing;

    a {
      color: white;
      text-decoration: none;

      &:hover {
        color: $secondary-color;
      }
    }
  }
}
\end{lstlisting}

\begin{nota}
L'ampersand (\&) rappresenta il selettore genitore. \texttt{\&:hover} diventa \texttt{.navbar .menu a:hover}.
\end{nota}

\section{Mixin}

Mixin sono funzioni riutilizzabili:

\begin{lstlisting}[language=CSS]
@mixin flex-center {
  display: flex;
  justify-content: center;
  align-items: center;
}

@mixin responsive($breakpoint) {
  @if $breakpoint == "tablet" {
    @media (min-width: 768px) { @content; }
  }
  @if $breakpoint == "desktop" {
    @media (min-width: 1024px) { @content; }
  }
}

.button {
  @include flex-center;
  padding: $spacing;
}

@include responsive("tablet") {
  .container { width: 90%; }
}
\end{lstlisting}

\section{Import e modularità}

Organizza SCSS in file modulari:

\begin{lstlisting}[language=CSS]
// main.scss
@import "variables";    // _variables.scss
@import "mixins";       // _mixins.scss
@import "navbar";       // _navbar.scss
@import "buttons";      // _buttons.scss
@import "responsive";   // _responsive.scss
\end{lstlisting}

\section{Compilazione SCSS}

Compilare SCSS a CSS con Node.js:

\begin{lstlisting}[language=bash]
# Installare sass
npm install -g sass

# Compilare una volta
sass scss/main.scss css/style.css

# Watch mode (recompila su cambiamento)
sass --watch scss:css

# Minificato
sass --style=compressed scss/main.scss css/style.min.css
\end{lstlisting}

\section{Esercizi}

\subsection{Esercizio 1 (Base)}
Crea file \texttt{\_variables.scss} con colori, font-size, spacing. Importa in \texttt{main.scss} e usali in almeno 5 selettori.

\subsection{Esercizio 2 (Intermedio)}
Crea un mixin \texttt{@mixin card} che stilizza le card con padding, border, shadow, e font. Usalo per multiple card diverse.

\section{Riepilogo}

\begin{itemize}
  \item SCSS è superset di CSS
  \item Variabili: \$variableName
  \item Nesting: selettori annidati, ampersand per genitore
  \item Mixin: funzioni riutilizzabili con @mixin e @include
  \item Import: modularità con @import
  \item Compilazione: sass scss:css
\end{itemize}

%\chapter{Esercizi e Progetti}
\label{cap:esercizi}

\section{Esercizi base}

\subsection{Esercizio 1.1}
Crea un documento HTML5 valido con struttura semantica e meta tag viewport.

\subsection{Esercizio 1.2}
Crea una pagina con almeno 5 tag semantici diversi e stilizzali con CSS base.

\subsection{Esercizio 2.1}
Crea un form di login con email, password, e checkbox "ricordami".

\subsection{Esercizio 2.2}
Crea un form di registrazione con validazione HTML5 per tutti i campi.

\subsection{Esercizio 3.1}
Calcola specificity per 10 selettori CSS diversi e ordinali per priorità.

\subsection{Esercizio 3.2}
Crea un layout con header, sidebar, main, footer usando CSS box model.

\section{Esercizi intermedi}

\subsection{Esercizio 4.1}
Crea un navbar responsivo usando flexbox che diventa menu verticale su mobile.

\subsection{Esercizio 4.2}
Crea una griglia 3x3 usando CSS Grid con una card che occupa 2 colonne.

\subsection{Esercizio 5.1}
Implementa 3 breakpoint (320, 768, 1024) con layout che cambia completamente.

\subsection{Esercizio 5.2}
Crea pagina con immagini responsive usando tag \texttt{<picture>}.

\section{Progetti guidati}

\subsection{Progetto Uno: Portfolio Personale}
\textbf{Capitoli}: 01-04 (HTML5 + CSS base)

Crea un portfolio personale con:
\begin{itemize}
  \item Header con nome e tagline
  \item Sezione "Chi sono"
  \item Sezione "Progetti" (gallery di 3-4 progetti)
  \item Form contatti
  \item Footer con link social
\end{itemize}

\textbf{Requisiti}:
\begin{itemize}
  \item HTML5 semantico
  \item CSS styling completo
  \item Responsive (mobile + desktop)
  \item Form validato
\end{itemize}

\subsubsection{Soluzione HTML}

\begin{lstlisting}[language=HTML]
<!DOCTYPE html>
<html lang="it">
<head>
  <meta charset="UTF-8">
  <meta name="viewport" content="width=device-width, initial-scale=1.0">
  <title>Portfolio - Marco Rossi</title>
  <link rel="stylesheet" href="style.css">
</head>
<body>
  <header>
    <h1>Marco Rossi</h1>
    <p class="tagline">Full-Stack Developer & Designer</p>
  </header>

  <nav>
    <ul>
      <li><a href="#about">Chi sono</a></li>
      <li><a href="#projects">Progetti</a></li>
      <li><a href="#contact">Contatti</a></li>
    </ul>
  </nav>

  <main>
    <section id="about" class="section-about">
      <h2>Chi sono</h2>
      <p>Sviluppatore web con 3 anni di esperienza. Specializzato in front-end e back-end con focus su user experience.</p>
      <div class="skills">
        <span class="skill-badge">HTML5</span>
        <span class="skill-badge">CSS3</span>
        <span class="skill-badge">JavaScript</span>
        <span class="skill-badge">React</span>
        <span class="skill-badge">Node.js</span>
      </div>
    </section>

    <section id="projects" class="section-projects">
      <h2>I miei Progetti</h2>
      <div class="projects-grid">
        <article class="project-card">
          <h3>E-commerce App</h3>
          <p>Piattaforma di vendita online con React e Node.js</p>
          <a href="#" class="btn-project">Vedi Progetto</a>
        </article>
        <article class="project-card">
          <h3>Social Network</h3>
          <p>Rete sociale con autenticazione e messaggi real-time</p>
          <a href="#" class="btn-project">Vedi Progetto</a>
        </article>
        <article class="project-card">
          <h3>Blog CMS</h3>
          <p>Sistema di gestione contenuti con dashboard admin</p>
          <a href="#" class="btn-project">Vedi Progetto</a>
        </article>
      </div>
    </section>

    <section id="contact" class="section-contact">
      <h2>Contattami</h2>
      <form class="contact-form">
        <input type="text" placeholder="Il tuo nome" required>
        <input type="email" placeholder="La tua email" required>
        <textarea placeholder="Messaggio" rows="5" required></textarea>
        <button type="submit">Invia</button>
      </form>
    </section>
  </main>

  <footer>
    <p>&copy; 2025 Marco Rossi. Tutti i diritti riservati.</p>
    <div class="social-links">
      <a href="#">GitHub</a>
      <a href="#">LinkedIn</a>
      <a href="#">Twitter</a>
    </div>
  </footer>
</body>
</html>
\end{lstlisting}

\subsubsection{Soluzione CSS}

\begin{lstlisting}[language=CSS]
* {
  margin: 0;
  padding: 0;
  box-sizing: border-box;
}

body {
  font-family: 'Segoe UI', Tahoma, Geneva, Verdana, sans-serif;
  line-height: 1.6;
  color: #333;
}

header {
  background: linear-gradient(135deg, #667eea 0%, #764ba2 100%);
  color: white;
  padding: 60px 20px;
  text-align: center;
}

header h1 {
  font-size: 3rem;
  margin-bottom: 10px;
}

header .tagline {
  font-size: 1.2rem;
  font-weight: 300;
}

nav {
  background: #333;
  padding: 0;
  position: sticky;
  top: 0;
  z-index: 100;
}

nav ul {
  list-style: none;
  display: flex;
  justify-content: center;
}

nav a {
  display: block;
  color: white;
  padding: 15px 20px;
  text-decoration: none;
  transition: background 0.3s;
}

nav a:hover {
  background: #667eea;
}

main {
  max-width: 1200px;
  margin: 0 auto;
  padding: 40px 20px;
}

.section-about {
  text-align: center;
  margin-bottom: 60px;
}

.section-about h2 {
  font-size: 2rem;
  margin-bottom: 20px;
}

.section-about p {
  font-size: 1.1rem;
  margin-bottom: 30px;
  color: #666;
}

.skills {
  display: flex;
  flex-wrap: wrap;
  justify-content: center;
  gap: 10px;
}

.skill-badge {
  background: #667eea;
  color: white;
  padding: 8px 15px;
  border-radius: 20px;
  font-size: 0.9rem;
}

.section-projects h2 {
  font-size: 2rem;
  margin-bottom: 40px;
  text-align: center;
}

.projects-grid {
  display: grid;
  grid-template-columns: repeat(auto-fit, minmax(300px, 1fr));
  gap: 30px;
  margin-bottom: 60px;
}

.project-card {
  background: white;
  border: 1px solid #eee;
  border-radius: 8px;
  padding: 30px;
  text-align: center;
  transition: transform 0.3s, box-shadow 0.3s;
}

.project-card:hover {
  transform: translateY(-10px);
  box-shadow: 0 10px 30px rgba(0, 0, 0, 0.2);
}

.project-card h3 {
  font-size: 1.3rem;
  margin-bottom: 15px;
}

.project-card p {
  color: #666;
  margin-bottom: 20px;
}

.btn-project {
  display: inline-block;
  background: #667eea;
  color: white;
  padding: 10px 20px;
  border-radius: 5px;
  text-decoration: none;
  transition: background 0.3s;
}

.btn-project:hover {
  background: #764ba2;
}

.section-contact h2 {
  font-size: 2rem;
  margin-bottom: 30px;
  text-align: center;
}

.contact-form {
  max-width: 600px;
  margin: 0 auto 60px;
  display: flex;
  flex-direction: column;
  gap: 15px;
}

.contact-form input,
.contact-form textarea {
  padding: 12px;
  border: 1px solid #ddd;
  border-radius: 5px;
  font-family: inherit;
}

.contact-form button {
  background: #667eea;
  color: white;
  padding: 12px;
  border: none;
  border-radius: 5px;
  cursor: pointer;
  font-size: 1rem;
  transition: background 0.3s;
}

.contact-form button:hover {
  background: #764ba2;
}

footer {
  background: #333;
  color: white;
  padding: 40px 20px;
  text-align: center;
}

footer p {
  margin-bottom: 15px;
}

.social-links {
  display: flex;
  justify-content: center;
  gap: 15px;
}

.social-links a {
  color: #667eea;
  text-decoration: none;
  transition: color 0.3s;
}

.social-links a:hover {
  color: #764ba2;
}

/* Responsive */
@media (max-width: 768px) {
  header h1 {
    font-size: 2rem;
  }

  nav ul {
    flex-direction: column;
  }

  nav a {
    padding: 12px 20px;
  }

  .projects-grid {
    grid-template-columns: 1fr;
  }

  main {
    padding: 20px 10px;
  }
}
\end{lstlisting}

\subsection{Progetto Due: E-commerce Landing}
\textbf{Capitoli}: 05-06 (Flexbox/Grid + Responsive)

Crea landing page e-commerce con:
\begin{itemize}
  \item Hero section
  \item Griglia prodotti (3 colonne su desktop, 1 su mobile)
  \item Card prodotto con immagine, titolo, prezzo
  \item Newsletter signup form
  \item Footer
\end{itemize}

\textbf{Requisiti}:
\begin{itemize}
  \item CSS Grid per layout
  \item Responsive breakpoint 768px
  \item Hover effects su card
  \item Form validazione
\end{itemize}

\subsubsection{Soluzione HTML}

\begin{lstlisting}[language=HTML]
<!DOCTYPE html>
<html lang="it">
<head>
  <meta charset="UTF-8">
  <meta name="viewport" content="width=device-width, initial-scale=1.0">
  <title>TechShop - E-commerce</title>
  <link rel="stylesheet" href="style.css">
</head>
<body>
  <header>
    <nav class="navbar">
      <h1 class="logo">TechShop</h1>
      <ul class="nav-menu">
        <li><a href="#home">Home</a></li>
        <li><a href="#products">Prodotti</a></li>
        <li><a href="#newsletter">Newsletter</a></li>
      </ul>
    </nav>
  </header>

  <section id="home" class="hero">
    <div class="hero-content">
      <h1>Benvenuto su TechShop</h1>
      <p>I migliori gadget tecnologici a prezzi competitivi</p>
      <a href="#products" class="cta-btn">Scopri i Prodotti</a>
    </div>
  </section>

  <section id="products" class="products-section">
    <h2>I Nostri Prodotti</h2>
    <div class="products-grid">
      <div class="product-card">
        <div class="product-image" style="background: #f0f0f0;">
          <img alt="Laptop" style="width: 100%; height: 200px;">
        </div>
        <h3>Laptop Pro</h3>
        <p class="description">Processore Intel i7, 16GB RAM, SSD 512GB</p>
        <p class="price">€999.99</p>
        <button class="btn-add">Aggiungi al Carrello</button>
      </div>

      <div class="product-card">
        <div class="product-image" style="background: #f0f0f0;">
          <img alt="Smartphone" style="width: 100%; height: 200px;">
        </div>
        <h3>Smartphone X</h3>
        <p class="description">Display OLED, Fotocamera 108MP, 5G</p>
        <p class="price">€699.99</p>
        <button class="btn-add">Aggiungi al Carrello</button>
      </div>

      <div class="product-card">
        <div class="product-image" style="background: #f0f0f0;">
          <img alt="Smartwatch" style="width: 100%; height: 200px;">
        </div>
        <h3>Smartwatch Elite</h3>
        <p class="description">Batteria 7 giorni, Monitoraggio Salute, GPS</p>
        <p class="price">€299.99</p>
        <button class="btn-add">Aggiungi al Carrello</button>
      </div>

      <div class="product-card">
        <div class="product-image" style="background: #f0f0f0;">
          <img alt="Cuffie" style="width: 100%; height: 200px;">
        </div>
        <h3>Cuffie Wireless</h3>
        <p class="description">ANC, Bluetooth 5.0, Qualità Audio Hi-Fi</p>
        <p class="price">€199.99</p>
        <button class="btn-add">Aggiungi al Carrello</button>
      </div>

      <div class="product-card">
        <div class="product-image" style="background: #f0f0f0;">
          <img alt="Tablet" style="width: 100%; height: 200px;">
        </div>
        <h3>Tablet Ultra</h3>
        <p class="description">11 pollici, 128GB Storage, Stylus inclusa</p>
        <p class="price">€449.99</p>
        <button class="btn-add">Aggiungi al Carrello</button>
      </div>

      <div class="product-card">
        <div class="product-image" style="background: #f0f0f0;">
          <img alt="Powerbank" style="width: 100%; height: 200px;">
        </div>
        <h3>PowerBank 30000mAh</h3>
        <p class="description">Carica veloce, Porte Multiple USB-C</p>
        <p class="price">€79.99</p>
        <button class="btn-add">Aggiungi al Carrello</button>
      </div>
    </div>
  </section>

  <section id="newsletter" class="newsletter">
    <div class="newsletter-content">
      <h2>Iscriviti alla Newsletter</h2>
      <p>Ricevi offerte esclusive e sconti speciali</p>
      <form class="newsletter-form">
        <input type="email" placeholder="La tua email" required>
        <button type="submit">Iscriviti</button>
      </form>
    </div>
  </section>

  <footer>
    <div class="footer-content">
      <div class="footer-section">
        <h4>About</h4>
        <p>TechShop - Il tuo negozio di gadget online</p>
      </div>
      <div class="footer-section">
        <h4>Links</h4>
        <ul>
          <li><a href="#">Chi Siamo</a></li>
          <li><a href="#">Contatti</a></li>
          <li><a href="#">Privacy</a></li>
        </ul>
      </div>
      <div class="footer-section">
        <h4>Follow Us</h4>
        <ul>
          <li><a href="#">Facebook</a></li>
          <li><a href="#">Instagram</a></li>
          <li><a href="#">Twitter</a></li>
        </ul>
      </div>
    </div>
    <p class="footer-bottom">&copy; 2025 TechShop. Tutti i diritti riservati.</p>
  </footer>
</body>
</html>
\end{lstlisting}

\subsubsection{Soluzione CSS}

\begin{lstlisting}[language=CSS]
* {
  margin: 0;
  padding: 0;
  box-sizing: border-box;
}

body {
  font-family: 'Segoe UI', Tahoma, Geneva, Verdana, sans-serif;
  color: #333;
  line-height: 1.6;
}

header {
  background: white;
  box-shadow: 0 2px 10px rgba(0, 0, 0, 0.1);
  position: sticky;
  top: 0;
  z-index: 100;
}

.navbar {
  display: flex;
  justify-content: space-between;
  align-items: center;
  padding: 1rem 2rem;
  max-width: 1200px;
  margin: 0 auto;
}

.logo {
  font-size: 1.8rem;
  font-weight: bold;
  color: #ff6b6b;
}

.nav-menu {
  list-style: none;
  display: flex;
  gap: 2rem;
}

.nav-menu a {
  text-decoration: none;
  color: #333;
  transition: color 0.3s;
}

.nav-menu a:hover {
  color: #ff6b6b;
}

.hero {
  background: linear-gradient(135deg, #667eea 0%, #764ba2 100%);
  color: white;
  padding: 100px 20px;
  text-align: center;
  min-height: 400px;
  display: flex;
  align-items: center;
  justify-content: center;
}

.hero-content h1 {
  font-size: 3rem;
  margin-bottom: 1rem;
}

.hero-content p {
  font-size: 1.3rem;
  margin-bottom: 2rem;
}

.cta-btn {
  display: inline-block;
  background: #ff6b6b;
  color: white;
  padding: 12px 30px;
  border-radius: 5px;
  text-decoration: none;
  transition: background 0.3s;
  font-weight: bold;
}

.cta-btn:hover {
  background: #ff5252;
}

.products-section {
  max-width: 1200px;
  margin: 60px auto;
  padding: 0 20px;
}

.products-section h2 {
  text-align: center;
  font-size: 2.5rem;
  margin-bottom: 50px;
  color: #333;
}

.products-grid {
  display: grid;
  grid-template-columns: repeat(auto-fit, minmax(250px, 1fr));
  gap: 30px;
}

.product-card {
  background: white;
  border: 1px solid #eee;
  border-radius: 8px;
  overflow: hidden;
  transition: transform 0.3s, box-shadow 0.3s;
}

.product-card:hover {
  transform: translateY(-10px);
  box-shadow: 0 10px 30px rgba(0, 0, 0, 0.15);
}

.product-image {
  width: 100%;
  height: 200px;
  background: #f5f5f5;
  overflow: hidden;
}

.product-card h3 {
  padding: 15px 15px 5px;
  font-size: 1.2rem;
}

.product-card .description {
  padding: 0 15px;
  color: #666;
  font-size: 0.9rem;
}

.product-card .price {
  padding: 15px 15px 10px;
  font-size: 1.5rem;
  font-weight: bold;
  color: #ff6b6b;
}

.btn-add {
  width: calc(100% - 30px);
  margin: 0 15px 15px;
  padding: 10px;
  background: #667eea;
  color: white;
  border: none;
  border-radius: 5px;
  cursor: pointer;
  transition: background 0.3s;
}

.btn-add:hover {
  background: #764ba2;
}

.newsletter {
  background: #f8f9fa;
  padding: 60px 20px;
  text-align: center;
}

.newsletter-content {
  max-width: 600px;
  margin: 0 auto;
}

.newsletter-content h2 {
  font-size: 2rem;
  margin-bottom: 10px;
}

.newsletter-content p {
  margin-bottom: 30px;
  color: #666;
}

.newsletter-form {
  display: flex;
  gap: 10px;
}

.newsletter-form input {
  flex: 1;
  padding: 12px;
  border: 1px solid #ddd;
  border-radius: 5px;
}

.newsletter-form button {
  padding: 12px 30px;
  background: #ff6b6b;
  color: white;
  border: none;
  border-radius: 5px;
  cursor: pointer;
  transition: background 0.3s;
}

.newsletter-form button:hover {
  background: #ff5252;
}

footer {
  background: #333;
  color: white;
  padding: 40px 20px;
}

.footer-content {
  max-width: 1200px;
  margin: 0 auto 30px;
  display: grid;
  grid-template-columns: repeat(auto-fit, minmax(200px, 1fr));
  gap: 30px;
}

.footer-section h4 {
  margin-bottom: 15px;
}

.footer-section ul {
  list-style: none;
}

.footer-section ul li {
  margin-bottom: 10px;
}

.footer-section a {
  color: #ccc;
  text-decoration: none;
  transition: color 0.3s;
}

.footer-section a:hover {
  color: #ff6b6b;
}

.footer-bottom {
  text-align: center;
  border-top: 1px solid #555;
  padding-top: 20px;
}

/* Responsive */
@media (max-width: 768px) {
  .navbar {
    flex-direction: column;
    gap: 1rem;
  }

  .nav-menu {
    flex-direction: column;
    gap: 1rem;
  }

  .hero-content h1 {
    font-size: 2rem;
  }

  .products-grid {
    grid-template-columns: 1fr;
  }

  .newsletter-form {
    flex-direction: column;
  }
}
\end{lstlisting}

\subsection{Progetto Tre: Blog Responsivo}
\textbf{Capitoli}: 07 (SASS/SCSS)

Crea blog con:
\begin{itemize}
  \item Homepage con lista articoli
  \item Pagina singolo articolo
  \item Sidebar con categorie
  \item Search form
  \item Sistema design con SCSS
\end{itemize}

\textbf{Requisiti}:
\begin{itemize}
  \item Crea file variabili.scss con colori e font
  \item Crea file mixins.scss con flex-center e media-queries
  \item Modularità: navbar.scss, cards.scss, responsive.scss
  \item Compila SCSS a CSS
\end{itemize}

\subsubsection{Soluzione HTML}

\begin{lstlisting}[language=HTML]
<!DOCTYPE html>
<html lang="it">
<head>
  <meta charset="UTF-8">
  <meta name="viewport" content="width=device-width, initial-scale=1.0">
  <title>TechBlog - Tutorial e Articoli</title>
  <link rel="stylesheet" href="style.css">
</head>
<body>
  <header>
    <nav class="navbar">
      <h1 class="logo">TechBlog</h1>
      <ul class="nav-menu">
        <li><a href="#home">Home</a></li>
        <li><a href="#articles">Articoli</a></li>
        <li><a href="#about">Chi Siamo</a></li>
      </ul>
    </nav>
  </header>

  <main class="container">
    <section class="content">
      <div class="search-box">
        <input type="search" placeholder="Cerca articoli...">
        <button>Cerca</button>
      </div>

      <article class="article-card">
        <div class="article-header">
          <h2>Guida Completa a Flexbox</h2>
          <span class="category">CSS</span>
        </div>
        <p class="date">8 Novembre 2025</p>
        <p class="excerpt">Scopri tutto quello che devi sapere su Flexbox, uno dei layout moderni più potenti del CSS contemporaneo...</p>
        <a href="#" class="read-more">Leggi Articolo</a>
      </article>

      <article class="article-card">
        <div class="article-header">
          <h2>JavaScript Async/Await Spiegato</h2>
          <span class="category">JavaScript</span>
        </div>
        <p class="date">5 Novembre 2025</p>
        <p class="excerpt">Impara a gestire operazioni asincrone in JavaScript con async/await, la sintassi moderna più leggibile...</p>
        <a href="#" class="read-more">Leggi Articolo</a>
      </article>

      <article class="article-card">
        <div class="article-header">
          <h2>SCSS: Variabili e Mixin</h2>
          <span class="category">SCSS</span>
        </div>
        <p class="date">2 Novembre 2025</p>
        <p class="excerpt">Trasforma il tuo CSS con SCSS: scopri variabili, nesting e mixin per codice più manutenibile e DRY...</p>
        <a href="#" class="read-more">Leggi Articolo</a>
      </article>

      <article class="article-card">
        <div class="article-header">
          <h2>Responsive Design: Mobile First</h2>
          <span class="category">Web Design</span>
        </div>
        <p class="date">30 Ottobre 2025</p>
        <p class="excerpt">Impara l'approccio mobile-first per creare siti responsive che funzionano su tutti i dispositivi...</p>
        <a href="#" class="read-more">Leggi Articolo</a>
      </article>
    </section>

    <aside class="sidebar">
      <div class="widget">
        <h3>Categorie</h3>
        <ul>
          <li><a href="#">HTML</a> <span class="count">(5)</span></li>
          <li><a href="#">CSS</a> <span class="count">(8)</span></li>
          <li><a href="#">JavaScript</a> <span class="count">(12)</span></li>
          <li><a href="#">SCSS</a> <span class="count">(4)</span></li>
          <li><a href="#">Web Design</a> <span class="count">(6)</span></li>
        </ul>
      </div>

      <div class="widget">
        <h3>Articoli Recenti</h3>
        <ul class="recent-posts">
          <li><a href="#">Guida Completa a Flexbox</a></li>
          <li><a href="#">JavaScript Async/Await</a></li>
          <li><a href="#">SCSS: Variabili e Mixin</a></li>
        </ul>
      </div>

      <div class="widget">
        <h3>Newsletter</h3>
        <form class="newsletter-mini">
          <input type="email" placeholder="La tua email" required>
          <button type="submit">Iscriviti</button>
        </form>
      </div>
    </aside>
  </main>

  <footer>
    <p>&copy; 2025 TechBlog. Tutti i diritti riservati.</p>
  </footer>
</body>
</html>
\end{lstlisting}

\subsubsection{Soluzione SCSS}

\begin{lstlisting}[language=CSS]
// variabili.scss
$primary-color: #667eea;
$secondary-color: #764ba2;
$text-color: #333;
$light-gray: #f5f5f5;
$border-color: #ddd;
$success-color: #4caf50;

$font-size-base: 1rem;
$font-size-lg: 1.5rem;
$font-size-xl: 2rem;
$line-height-base: 1.6;

// mixins.scss
@mixin flex-center {
  display: flex;
  justify-content: center;
  align-items: center;
}

@mixin flex-between {
  display: flex;
  justify-content: space-between;
  align-items: center;
}

@mixin box-shadow {
  box-shadow: 0 2px 10px rgba(0, 0, 0, 0.1);
}

@mixin transition($property: all) {
  transition: $property 0.3s ease;
}

@mixin media-tablet {
  @media (max-width: 768px) {
    @content;
  }
}

@mixin media-mobile {
  @media (max-width: 480px) {
    @content;
  }
}

// main.scss
* {
  margin: 0;
  padding: 0;
  box-sizing: border-box;
}

body {
  font-family: 'Segoe UI', Tahoma, Geneva, Verdana, sans-serif;
  color: $text-color;
  background: white;
  line-height: $line-height-base;
}

header {
  background: white;
  @include box-shadow;
  position: sticky;
  top: 0;
  z-index: 100;
}

.navbar {
  @include flex-between;
  padding: 1rem 2rem;
  max-width: 1200px;
  margin: 0 auto;

  .logo {
    font-size: $font-size-lg;
    font-weight: bold;
    color: $primary-color;
  }

  .nav-menu {
    list-style: none;
    display: flex;
    gap: 2rem;

    a {
      text-decoration: none;
      color: $text-color;
      @include transition;

      &:hover {
        color: $primary-color;
      }
    }
  }
}

.container {
  display: grid;
  grid-template-columns: 1fr 300px;
  gap: 30px;
  max-width: 1200px;
  margin: 40px auto;
  padding: 0 20px;

  @include media-tablet {
    grid-template-columns: 1fr;
  }
}

.content {
  .search-box {
    @include flex-between;
    gap: 10px;
    margin-bottom: 30px;

    input {
      flex: 1;
      padding: 12px;
      border: 1px solid $border-color;
      border-radius: 5px;
    }

    button {
      padding: 12px 20px;
      background: $primary-color;
      color: white;
      border: none;
      border-radius: 5px;
      cursor: pointer;
      @include transition;

      &:hover {
        background: $secondary-color;
      }
    }
  }
}

.article-card {
  background: white;
  border: 1px solid $border-color;
  border-radius: 8px;
  padding: 25px;
  margin-bottom: 25px;
  @include box-shadow;
  @include transition;

  &:hover {
    @include box-shadow;
    transform: translateY(-5px);
  }

  .article-header {
    @include flex-between;
    gap: 15px;
    margin-bottom: 10px;

    h2 {
      font-size: 1.4rem;
      color: $text-color;
    }

    .category {
      background: $primary-color;
      color: white;
      padding: 5px 12px;
      border-radius: 20px;
      font-size: 0.85rem;
      white-space: nowrap;
    }
  }

  .date {
    color: #999;
    font-size: 0.9rem;
    margin-bottom: 10px;
  }

  .excerpt {
    color: #666;
    margin-bottom: 15px;
    line-height: 1.6;
  }

  .read-more {
    display: inline-block;
    color: $primary-color;
    text-decoration: none;
    font-weight: bold;
    @include transition;

    &:hover {
      color: $secondary-color;
    }
  }
}

.sidebar {
  .widget {
    background: $light-gray;
    padding: 20px;
    border-radius: 8px;
    margin-bottom: 25px;

    h3 {
      font-size: $font-size-lg;
      margin-bottom: 15px;
      color: $primary-color;
    }

    ul {
      list-style: none;

      li {
        padding: 8px 0;
        border-bottom: 1px solid $border-color;

        a {
          text-decoration: none;
          color: $text-color;
          @include transition;

          &:hover {
            color: $primary-color;
          }
        }

        .count {
          color: #999;
          font-size: 0.9rem;
        }
      }

      li:last-child {
        border: none;
      }
    }

    .recent-posts li {
      border: none;
      padding: 10px 0;
    }
  }

  .newsletter-mini {
    display: flex;
    flex-direction: column;
    gap: 10px;

    input {
      padding: 10px;
      border: 1px solid $border-color;
      border-radius: 5px;
    }

    button {
      padding: 10px;
      background: $primary-color;
      color: white;
      border: none;
      border-radius: 5px;
      cursor: pointer;
      @include transition;

      &:hover {
        background: $secondary-color;
      }
    }
  }
}

footer {
  background: $text-color;
  color: white;
  text-align: center;
  padding: 30px 20px;
  margin-top: 40px;
}

// Responsive
@include media-tablet {
  .navbar {
    flex-direction: column;
    gap: 1rem;
    text-align: center;
  }

  .nav-menu {
    gap: 1rem;
  }
}

@include media-mobile {
  .container {
    padding: 0 10px;
  }

  .article-card {
    padding: 15px;

    .article-header {
      flex-direction: column;
      align-items: flex-start;

      h2 {
        font-size: 1.2rem;
      }
    }
  }

  .sidebar {
    display: none;
  }
}
\end{lstlisting}

\subsection{Progetto Quattro: Dashboard Admin}
\textbf{Capitoli}: 01-07 (Integrazione completa)

Crea dashboard admin con:
\begin{itemize}
  \item Sidebar navigation (collapsible su mobile)
  \item Header con user profile
  \item Griglia statistiche
  \item Tabella dati
  \item Form modifica profilo
  \item Responsive su tutti dispositivi
\end{itemize}

\textbf{Requisiti}:
\begin{itemize}
  \item HTML5 semantico
  \item CSS Grid per layout principale
  \item Flexbox per componenti
  \item SCSS modularizzato
  \item Responsive (320, 768, 1024)
  \item Form validazione
\end{itemize}

\subsubsection{Soluzione HTML}

\begin{lstlisting}[language=HTML]
<!DOCTYPE html>
<html lang="it">
<head>
  <meta charset="UTF-8">
  <meta name="viewport" content="width=device-width, initial-scale=1.0">
  <title>Admin Dashboard</title>
  <link rel="stylesheet" href="style.css">
</head>
<body>
  <div class="dashboard">
    <aside class="sidebar">
      <div class="sidebar-header">
        <h2>Dashboard</h2>
        <button class="menu-toggle" id="menu-toggle">☰</button>
      </div>
      <nav class="sidebar-nav">
        <a href="#" class="nav-item active">
          <span class="icon">📊</span> Dashboard
        </a>
        <a href="#" class="nav-item">
          <span class="icon">👥</span> Utenti
        </a>
        <a href="#" class="nav-item">
          <span class="icon">📝</span> Articoli
        </a>
        <a href="#" class="nav-item">
          <span class="icon">⚙️</span> Impostazioni
        </a>
        <a href="#" class="nav-item">
          <span class="icon">📄</span> Rapporti
        </a>
      </nav>
    </aside>

    <div class="main-content">
      <header class="top-header">
        <div class="header-left">
          <h1>Benvenuto, Admin!</h1>
        </div>
        <div class="header-right">
          <div class="user-profile">
            <img src="data:image/svg+xml,%3Csvg xmlns='http://www.w3.org/2000/svg' width='40' height='40'%3E%3Ccircle cx='20' cy='20' r='20' fill='%23ddd'/%3E%3C/svg%3E" alt="User">
            <div class="user-info">
              <p class="user-name">Marco Rossi</p>
              <p class="user-role">Administrator</p>
            </div>
          </div>
          <button class="logout-btn">Logout</button>
        </div>
      </header>

      <main class="content">
        <section class="stats-grid">
          <div class="stat-card">
            <div class="stat-icon">📈</div>
            <h3>Utenti Totali</h3>
            <p class="stat-value">1,234</p>
            <p class="stat-change">+12% questo mese</p>
          </div>
          <div class="stat-card">
            <div class="stat-icon">💰</div>
            <h3>Ricavi</h3>
            <p class="stat-value">€45,231</p>
            <p class="stat-change">+8% questo mese</p>
          </div>
          <div class="stat-card">
            <div class="stat-icon">📊</div>
            <h3>Ordini</h3>
            <p class="stat-value">856</p>
            <p class="stat-change">-2% questo mese</p>
          </div>
          <div class="stat-card">
            <div class="stat-icon">⭐</div>
            <h3>Rating</h3>
            <p class="stat-value">4.8/5</p>
            <p class="stat-change">+0.2 questo mese</p>
          </div>
        </section>

        <section class="data-section">
          <div class="section-header">
            <h2>Ultimi Utenti</h2>
            <button class="btn-add">+ Nuovo Utente</button>
          </div>
          <div class="table-responsive">
            <table class="data-table">
              <thead>
                <tr>
                  <th>Nome</th>
                  <th>Email</th>
                  <th>Data Iscrizione</th>
                  <th>Status</th>
                  <th>Azioni</th>
                </tr>
              </thead>
              <tbody>
                <tr>
                  <td>Marco Rossi</td>
                  <td>marco@example.com</td>
                  <td>01/01/2025</td>
                  <td><span class="badge active">Attivo</span></td>
                  <td>
                    <button class="btn-small">Modifica</button>
                    <button class="btn-small btn-danger">Elimina</button>
                  </td>
                </tr>
                <tr>
                  <td>Anna Bianchi</td>
                  <td>anna@example.com</td>
                  <td>05/02/2025</td>
                  <td><span class="badge active">Attivo</span></td>
                  <td>
                    <button class="btn-small">Modifica</button>
                    <button class="btn-small btn-danger">Elimina</button>
                  </td>
                </tr>
                <tr>
                  <td>Luigi Verdi</td>
                  <td>luigi@example.com</td>
                  <td>10/02/2025</td>
                  <td><span class="badge inactive">Inattivo</span></td>
                  <td>
                    <button class="btn-small">Modifica</button>
                    <button class="btn-small btn-danger">Elimina</button>
                  </td>
                </tr>
              </tbody>
            </table>
          </div>
        </section>

        <section class="form-section">
          <h2>Modifica Profilo</h2>
          <form class="profile-form">
            <div class="form-group">
              <label>Nome Completo</label>
              <input type="text" placeholder="Marco Rossi" required>
            </div>
            <div class="form-group">
              <label>Email</label>
              <input type="email" placeholder="marco@example.com" required>
            </div>
            <div class="form-row">
              <div class="form-group">
                <label>Telefono</label>
                <input type="tel" placeholder="+39 3xx xxxx xxx">
              </div>
              <div class="form-group">
                <label>Città</label>
                <input type="text" placeholder="Roma">
              </div>
            </div>
            <div class="form-group">
              <label>Biografia</label>
              <textarea rows="4" placeholder="Descrivi te stesso..."></textarea>
            </div>
            <button type="submit" class="btn-submit">Salva Modifiche</button>
          </form>
        </section>
      </main>
    </div>
  </div>

  <script>
    document.getElementById('menu-toggle').addEventListener('click', function() {
      document.querySelector('.sidebar').classList.toggle('collapsed');
    });
  </script>
</body>
</html>
\end{lstlisting}

\subsubsection{Soluzione CSS (Main)}

\begin{lstlisting}[language=CSS]
* {
  margin: 0;
  padding: 0;
  box-sizing: border-box;
}

body {
  font-family: 'Segoe UI', Tahoma, Geneva, Verdana, sans-serif;
  background: #f5f7fa;
  color: #333;
}

.dashboard {
  display: grid;
  grid-template-columns: 250px 1fr;
  min-height: 100vh;
}

/* Sidebar */
.sidebar {
  background: linear-gradient(135deg, #667eea 0%, #764ba2 100%);
  color: white;
  padding: 20px;
  position: fixed;
  left: 0;
  top: 0;
  width: 250px;
  height: 100vh;
  overflow-y: auto;
  z-index: 100;
}

.sidebar-header {
  display: flex;
  justify-content: space-between;
  align-items: center;
  margin-bottom: 30px;
}

.sidebar-header h2 {
  font-size: 1.5rem;
}

.menu-toggle {
  display: none;
  background: none;
  border: none;
  color: white;
  font-size: 1.5rem;
  cursor: pointer;
}

.sidebar-nav {
  display: flex;
  flex-direction: column;
  gap: 10px;
}

.nav-item {
  display: flex;
  align-items: center;
  gap: 12px;
  padding: 12px 15px;
  color: white;
  text-decoration: none;
  border-radius: 8px;
  transition: background 0.3s;
}

.nav-item:hover,
.nav-item.active {
  background: rgba(255, 255, 255, 0.2);
}

.nav-item .icon {
  font-size: 1.3rem;
}

/* Main Content */
.main-content {
  margin-left: 250px;
  background: #f5f7fa;
  min-height: 100vh;
}

.top-header {
  background: white;
  padding: 20px 40px;
  display: flex;
  justify-content: space-between;
  align-items: center;
  box-shadow: 0 2px 10px rgba(0, 0, 0, 0.1);
}

.header-left h1 {
  font-size: 1.8rem;
  color: #333;
}

.header-right {
  display: flex;
  align-items: center;
  gap: 20px;
}

.user-profile {
  display: flex;
  align-items: center;
  gap: 12px;
}

.user-profile img {
  width: 40px;
  height: 40px;
  border-radius: 50%;
}

.user-name {
  font-weight: bold;
  margin: 0;
}

.user-role {
  font-size: 0.85rem;
  color: #666;
  margin: 0;
}

.logout-btn {
  padding: 8px 15px;
  background: #ff6b6b;
  color: white;
  border: none;
  border-radius: 5px;
  cursor: pointer;
  transition: background 0.3s;
}

.logout-btn:hover {
  background: #ff5252;
}

/* Content */
.content {
  padding: 40px;
}

.stats-grid {
  display: grid;
  grid-template-columns: repeat(auto-fit, minmax(250px, 1fr));
  gap: 25px;
  margin-bottom: 40px;
}

.stat-card {
  background: white;
  padding: 25px;
  border-radius: 10px;
  box-shadow: 0 2px 10px rgba(0, 0, 0, 0.1);
  text-align: center;
}

.stat-icon {
  font-size: 2.5rem;
  margin-bottom: 15px;
}

.stat-card h3 {
  color: #666;
  font-size: 0.95rem;
  margin-bottom: 10px;
  text-transform: uppercase;
}

.stat-value {
  font-size: 2rem;
  font-weight: bold;
  color: #667eea;
  margin-bottom: 5px;
}

.stat-change {
  color: #4caf50;
  font-size: 0.9rem;
}

.data-section {
  background: white;
  padding: 25px;
  border-radius: 10px;
  margin-bottom: 40px;
  box-shadow: 0 2px 10px rgba(0, 0, 0, 0.1);
}

.section-header {
  display: flex;
  justify-content: space-between;
  align-items: center;
  margin-bottom: 25px;
}

.section-header h2 {
  font-size: 1.5rem;
}

.btn-add {
  padding: 10px 20px;
  background: #667eea;
  color: white;
  border: none;
  border-radius: 5px;
  cursor: pointer;
  transition: background 0.3s;
}

.btn-add:hover {
  background: #764ba2;
}

.table-responsive {
  overflow-x: auto;
}

.data-table {
  width: 100%;
  border-collapse: collapse;
}

.data-table th {
  background: #f5f7fa;
  padding: 15px;
  text-align: left;
  font-weight: bold;
  border-bottom: 2px solid #ddd;
}

.data-table td {
  padding: 15px;
  border-bottom: 1px solid #eee;
}

.data-table tbody tr:hover {
  background: #f9f9f9;
}

.badge {
  padding: 5px 12px;
  border-radius: 20px;
  font-size: 0.85rem;
  font-weight: bold;
}

.badge.active {
  background: #c8e6c9;
  color: #2e7d32;
}

.badge.inactive {
  background: #ffcccc;
  color: #c62828;
}

.btn-small {
  padding: 6px 12px;
  background: #667eea;
  color: white;
  border: none;
  border-radius: 4px;
  cursor: pointer;
  margin-right: 5px;
  transition: background 0.3s;
}

.btn-small:hover {
  background: #764ba2;
}

.btn-small.btn-danger {
  background: #ff6b6b;
}

.btn-small.btn-danger:hover {
  background: #ff5252;
}

.form-section {
  background: white;
  padding: 25px;
  border-radius: 10px;
  box-shadow: 0 2px 10px rgba(0, 0, 0, 0.1);
}

.form-section h2 {
  margin-bottom: 25px;
  font-size: 1.5rem;
}

.profile-form {
  display: flex;
  flex-direction: column;
  gap: 20px;
}

.form-group {
  display: flex;
  flex-direction: column;
}

.form-group label {
  font-weight: bold;
  margin-bottom: 8px;
  color: #333;
}

.form-group input,
.form-group textarea {
  padding: 12px;
  border: 1px solid #ddd;
  border-radius: 5px;
  font-family: inherit;
  font-size: 1rem;
}

.form-group input:focus,
.form-group textarea:focus {
  outline: none;
  border-color: #667eea;
  box-shadow: 0 0 0 3px rgba(102, 126, 234, 0.1);
}

.form-row {
  display: grid;
  grid-template-columns: 1fr 1fr;
  gap: 20px;
}

.btn-submit {
  align-self: flex-start;
  padding: 12px 30px;
  background: #667eea;
  color: white;
  border: none;
  border-radius: 5px;
  cursor: pointer;
  font-size: 1rem;
  transition: background 0.3s;
}

.btn-submit:hover {
  background: #764ba2;
}

/* Responsive */
@media (max-width: 1024px) {
  .dashboard {
    grid-template-columns: 1fr;
  }

  .sidebar {
    width: 200px;
  }

  .main-content {
    margin-left: 200px;
  }

  .content {
    padding: 20px;
  }

  .form-row {
    grid-template-columns: 1fr;
  }
}

@media (max-width: 768px) {
  .dashboard {
    grid-template-columns: 1fr;
  }

  .sidebar {
    position: fixed;
    left: -250px;
    transition: left 0.3s;
    width: 250px;
  }

  .sidebar.collapsed {
    left: 0;
  }

  .main-content {
    margin-left: 0;
  }

  .menu-toggle {
    display: block;
  }

  .top-header {
    padding: 15px 20px;
    flex-direction: column;
    gap: 15px;
    align-items: flex-start;
  }

  .header-right {
    width: 100%;
    justify-content: space-between;
  }

  .stats-grid {
    grid-template-columns: 1fr;
  }

  .content {
    padding: 15px;
  }

  .data-table {
    font-size: 0.9rem;
  }

  .data-table th,
  .data-table td {
    padding: 10px;
  }

  .section-header {
    flex-direction: column;
    gap: 10px;
    align-items: flex-start;
  }

  .btn-add {
    width: 100%;
  }

  .table-responsive {
    overflow-x: auto;
  }

  .form-row {
    grid-template-columns: 1fr;
  }
}

@media (max-width: 480px) {
  .top-header {
    flex-direction: column;
    align-items: flex-start;
  }

  .stats-grid {
    grid-template-columns: 1fr;
  }

  .stat-card {
    padding: 15px;
  }

  .stat-value {
    font-size: 1.5rem;
  }

  .section-header h2 {
    font-size: 1.2rem;
  }

  .profile-form {
    gap: 15px;
  }

  .btn-small {
    display: block;
    width: 100%;
    margin-bottom: 5px;
  }
}
\end{lstlisting}

\section{Riepilogo}

Completa questi esercizi e progetti in ordine. Ogni progetto integra i concetti dei capitoli precedenti e prepara al prossimo.


% Capitoli JavaScript
%\chapter{DOM Manipulation Avanzato}
\label{cap:javascript_dom}

\section{Document Object Model (DOM)}

Il DOM (Document Object Model) è una rappresentazione ad albero del documento HTML. Ogni elemento, attributo e testo è un nodo che può essere accesso e modificato con JavaScript.

\begin{nota}
Il DOM è l'interfaccia tra JavaScript e HTML. Quando modifichi il DOM con JavaScript, il browser aggiorna automaticamente la pagina visualizzata.
\end{nota}

\section{Traversing il DOM}

\subsection{Relazioni tra nodi}

\begin{lstlisting}[language=JavaScript]
// Accedere ai nodi parent, child e sibling
const element = document.getElementById("main");

element.parentElement;           // Elemento genitore
element.parentNode;              // Nodo genitore (può essere text node)
element.children;                // Array di elementi figli
element.childNodes;              // Array di tutti i nodi figli (inclusi text)
element.firstChild;              // Primo nodo figlio
element.lastChild;               // Ultimo nodo figlio
element.firstElementChild;       // Primo elemento figlio
element.lastElementChild;        // Ultimo elemento figlio
element.nextSibling;             // Prossimo nodo fratello
element.nextElementSibling;      // Prossimo elemento fratello
element.previousSibling;         // Nodo fratello precedente
element.previousElementSibling;  // Elemento fratello precedente
\end{lstlisting}

\subsection{Attraversare gerarchie complesse}

\begin{lstlisting}[language=HTML]
<div id="root">
  <header>
    <nav>
      <ul id="menu">
        <li><a href="#home">Home</a></li>
        <li><a href="#about">About</a></li>
      </ul>
    </nav>
  </header>
  <main id="content">
    <section class="card">...</section>
  </main>
</div>
\end{lstlisting}

\begin{lstlisting}[language=JavaScript]
const link = document.querySelector("a[href='#home']");

// Salire verso l'alto
const li = link.parentElement;           // <li>
const ul = li.parentElement;             // <ul id="menu">
const nav = ul.parentElement;            // <nav>
const header = nav.parentElement;        // <header>

// Discendere verso il basso
const root = document.getElementById("root");
const main = root.querySelector("main");
const firstSection = main.firstElementChild; // <section class="card">

// Metodo più diretto: closest()
const listItem = link.closest("li");  // Trova primo <li> antenato
const container = link.closest("[id]"); // Trova primo antenato con id
\end{lstlisting}

\section{Selettori CSS Avanzati}

\subsection{querySelector e querySelectorAll}

\begin{lstlisting}[language=JavaScript]
// querySelector restituisce il primo match
const button = document.querySelector("button.primary");
const item = document.querySelector("#main .item");
const input = document.querySelector("input[name='email']");

// querySelectorAll restituisce NodeList
const buttons = document.querySelectorAll("button");
const items = document.querySelectorAll(".card");
const inputs = document.querySelectorAll("input[type='text']");

// Convertire NodeList a Array
const itemsArray = Array.from(items);
const itemsArray2 = [...items]; // Usando spread operator
\end{lstlisting}

\subsection{Selettori CSS3}

\begin{lstlisting}[language=CSS]
/* Pseudo-class selectors */
li:first-child                   /* Primo figlio */
li:last-child                    /* Ultimo figlio */
li:nth-child(2)                  /* Secondo figlio */
li:nth-child(odd)                /* Figli dispari */
li:nth-of-type(2)                /* Secondo del suo tipo */

/* Attribute selectors */
input[type="email"]              /* Input email */
input[name^="user"]              /* Name inizia con "user" */
input[name$="code"]              /* Name finisce con "code" */
input[name*="temp"]              /* Name contiene "temp" */

/* Combinatori */
div.container > p                /* Figli diretti */
.parent p                        /* Tutti i discendenti */
h1 + p                           /* Elemento adiacente */
h1 ~ p                           /* Elementi fratelli seguenti */
\end{lstlisting}

\begin{lstlisting}[language=JavaScript]
// Usare selettori avanzati con querySelector
const firstItem = document.querySelector("li:first-child");
const emailInput = document.querySelector("input[type='email']");
const oddListItems = document.querySelectorAll("li:nth-child(odd)");

// Filtrare con selezione dinamica
const activeButtons = document.querySelectorAll("button:not(.disabled)");
const visibleCards = document.querySelectorAll(".card:not(.hidden)");
\end{lstlisting}

\section{Event Delegation}

\subsection{Problema: Troppi Event Listener}

\begin{lstlisting}[language=HTML]
<ul id="tasks">
  <li data-id="1">Compra latte <button class="delete">X</button></li>
  <li data-id="2">Studia JS <button class="delete">X</button></li>
  <li data-id="3">Esercizio <button class="delete">X</button></li>
  <!-- Migliaia di item... -->
</ul>
\end{lstlisting}

\begin{lstlisting}[language=JavaScript]
// SBAGLIATO: Aggiungere listener a ogni bottone
document.querySelectorAll(".delete").forEach(btn => {
  btn.addEventListener("click", handleDelete);
});
// Questo è inefficiente se aggiungi/rimuovi molti item!

// CORRETTO: Delegare al genitore
const taskList = document.getElementById("tasks");
taskList.addEventListener("click", (event) => {
  // Controllare se il click è su un bottone delete
  if (event.target.classList.contains("delete")) {
    const taskItem = event.target.parentElement;
    const taskId = taskItem.dataset.id;
    handleDelete(taskId);
    taskItem.remove();
  }
});
\end{lstlisting}

\begin{attenzione}
Event delegation funziona con tutti gli eventi che bubblano (click, change, input, etc.). Per eventi che non bubblano (focus, load), usa il terzo parametro \texttt{capture: true}.
\end{attenzione}

\subsection{Proprietà di event}

\begin{lstlisting}[language=JavaScript]
document.addEventListener("click", (event) => {
  console.log(event.target);         // Elemento cliccato
  console.log(event.currentTarget);  // Elemento con listener
  console.log(event.bubbles);        // Booleano: bubblare?
  console.log(event.preventDefault()); // Annullare azione
  console.log(event.stopPropagation()); // Bloccare bubbling
  console.log(event.type);           // "click"
});
\end{lstlisting}

\section{Element Creation e Manipulation}

\subsection{Creare e aggiungere elementi}

\begin{lstlisting}[language=JavaScript]
// Creare elementi
const div = document.createElement("div");
const p = document.createElement("p");
const button = document.createElement("button");

// Aggiungere classi e attributi
div.className = "card";
div.classList.add("featured");
div.setAttribute("data-type", "product");
button.textContent = "Click me";

// Aggiungere al DOM
document.body.appendChild(div);
div.appendChild(p);
div.appendChild(button);

// Altre operazioni di aggiunta
div.insertBefore(p, button);        // Inserire prima di elemento
div.insertAdjacentHTML("beforeend", "<span>Testo</span>");
\end{lstlisting}

\subsection{innerHTML vs textContent vs innerText}

\begin{lstlisting}[language=JavaScript]
const div = document.getElementById("content");

// innerHTML: Parsed come HTML
div.innerHTML = "<strong>Testo</strong>"; // Crea <strong>

// textContent: Puro testo
div.textContent = "<strong>Testo</strong>"; // Mostra stringa

// innerText: Testo visibile (considera CSS)
div.innerText = "Solo testo"; // Ignora elementi nascosti

// Preferire textContent per sicurezza
const userInput = getUserInput();
div.textContent = userInput; // Sicuro da XSS injection
\end{lstlisting}

\begin{nota}
Usa \texttt{textContent} per inserire testo utente. Usa \texttt{innerHTML} solo se hai totale controllo del contenuto. Mai usare \texttt{innerHTML} con dati da utenti!
\end{nota}

\subsection{Clonare elementi}

\begin{lstlisting}[language=HTML]
<template id="card-template">
  <div class="card">
    <h3 class="title"></h3>
    <p class="description"></p>
    <button class="delete">Elimina</button>
  </div>
</template>
\end{lstlisting}

\begin{lstlisting}[language=JavaScript]
// Clonare da template
const template = document.getElementById("card-template");
const clone = template.content.cloneNode(true);

// Modificare il clone
clone.querySelector(".title").textContent = "My Card";
clone.querySelector(".description").textContent = "Description here";

// Aggiungere al DOM
document.getElementById("container").appendChild(clone);
\end{lstlisting}

\subsection{Rimuovere elementi}

\begin{lstlisting}[language=JavaScript]
// Rimuovere elemento specifico
const element = document.getElementById("to-delete");
element.remove();
element.parentElement.removeChild(element); // Alternativa

// Svuotare un contenitore
const container = document.getElementById("container");
container.innerHTML = ""; // Opzione 1
while (container.firstChild) {
  container.removeChild(container.firstChild); // Opzione 2
}
\end{lstlisting}

\section{Manipolazione di Classi e Attributi}

\subsection{ClassList API}

\begin{lstlisting}[language=JavaScript]
const button = document.getElementById("save-btn");

// Aggiungere classe
button.classList.add("active");
button.classList.add("primary", "large"); // Più classi

// Rimuovere classe
button.classList.remove("disabled");

// Toggle: aggiungere se assente, rimuovere se presente
button.classList.toggle("selected");
button.classList.toggle("highlight", true); // Forza add
button.classList.toggle("highlight", false); // Forza remove

// Controllare classe
if (button.classList.contains("active")) {
  console.log("Button è attivo");
}

// Iterare le classi
button.classList.forEach(className => {
  console.log(className);
});
\end{lstlisting}

\subsection{Attributi personalizzati (data-*)}

\begin{lstlisting}[language=HTML]
<div class="product" data-id="123" data-price="29.99"
     data-in-stock="true">
  Nike Shoes
</div>
\end{lstlisting}

\begin{lstlisting}[language=JavaScript]
const product = document.querySelector(".product");

// Accedere ai data attributes
console.log(product.dataset.id);      // "123"
console.log(product.dataset.price);   // "29.99"
console.log(product.dataset.inStock); // "true"

// Modificare data attributes
product.dataset.id = "456";
product.dataset.available = "false";

// Accesso diretto agli attributi
product.setAttribute("data-discount", "10%");
console.log(product.getAttribute("data-discount")); // "10%"
\end{lstlisting}

\section{Esercizio Pratico: Dynamic Form Builder}

\subsection{Specifica}

Crea una app che permette all'utente di costruire un form dinamicamente:
- Aggiungere campi (text, email, select, checkbox)
- Personalizzare label e placeholder
- Rimuovere campi
- Generare il codice HTML del form
- Salvare il form in localStorage
- Resettare il form

\subsection{HTML/CSS/JavaScript}

\begin{lstlisting}[language=HTML]
<!DOCTYPE html>
<html lang="it">
<head>
  <meta charset="UTF-8">
  <meta name="viewport" content="width=device-width, initial-scale=1.0">
  <title>Dynamic Form Builder</title>
  <style>
    body {
      font-family: Arial, sans-serif;
      max-width: 1200px;
      margin: 0 auto;
      padding: 20px;
      background: #f5f5f5;
    }

    .container {
      display: grid;
      grid-template-columns: 1fr 1fr;
      gap: 30px;
    }

    .builder {
      background: white;
      padding: 20px;
      border-radius: 8px;
      box-shadow: 0 2px 8px rgba(0,0,0,0.1);
    }

    .preview {
      background: white;
      padding: 20px;
      border-radius: 8px;
      box-shadow: 0 2px 8px rgba(0,0,0,0.1);
    }

    h1 { margin-top: 0; }

    .controls {
      margin-bottom: 20px;
      display: flex;
      gap: 10px;
      flex-wrap: wrap;
    }

    button {
      padding: 8px 12px;
      background: #007bff;
      color: white;
      border: none;
      border-radius: 4px;
      cursor: pointer;
      font-size: 14px;
    }

    button:hover { background: #0056b3; }
    button.danger { background: #dc3545; }
    button.danger:hover { background: #c82333; }
    button.success { background: #28a745; }
    button.success:hover { background: #218838; }

    .field {
      margin-bottom: 15px;
      padding: 15px;
      background: #f9f9f9;
      border: 1px solid #ddd;
      border-radius: 4px;
      position: relative;
    }

    .field input, .field select, .field textarea {
      width: 100%;
      padding: 8px;
      margin: 5px 0;
      border: 1px solid #ccc;
      border-radius: 4px;
      font-size: 14px;
      box-sizing: border-box;
    }

    .field label {
      display: block;
      font-weight: bold;
      margin: 10px 0 5px 0;
    }

    .field button.remove {
      position: absolute;
      top: 10px;
      right: 10px;
      padding: 5px 10px;
      font-size: 12px;
    }

    #code-output {
      background: #f0f0f0;
      padding: 15px;
      border-radius: 4px;
      font-family: monospace;
      font-size: 12px;
      overflow-x: auto;
      white-space: pre-wrap;
      word-wrap: break-word;
      max-height: 300px;
      overflow-y: auto;
    }
  </style>
</head>
<body>
  <h1>Dynamic Form Builder</h1>

  <div class="container">
    <div class="builder">
      <h2>Builder</h2>
      <div class="controls">
        <button onclick="addTextField()">+ Text</button>
        <button onclick="addEmailField()">+ Email</button>
        <button onclick="addSelectField()">+ Select</button>
        <button onclick="addCheckboxField()">+ Checkbox</button>
        <button onclick="addTextareaField()">+ Textarea</button>
      </div>
      <div id="fields-container"></div>
      <div class="controls" style="margin-top: 20px;">
        <button class="success" onclick="generateCode()">Genera Codice</button>
        <button class="success" onclick="saveForm()">Salva Form</button>
        <button class="danger" onclick="resetForm()">Resetta</button>
      </div>
    </div>

    <div class="preview">
      <h2>Anteprima Form</h2>
      <form id="preview-form"></form>
      <h3>Codice HTML</h3>
      <div id="code-output"></div>
    </div>
  </div>

  <script>
    class FormBuilder {
      constructor() {
        this.fields = [];
        this.fieldId = 0;
        this.loadForm();
      }

      addField(type) {
        const id = this.fieldId++;
        const field = {
          id, type,
          label: `Campo ${id + 1}`,
          placeholder: "",
          options: type === "select" ? ["Opzione 1"] : [],
          required: false
        };
        this.fields.push(field);
        this.render();
      }

      removeField(id) {
        this.fields = this.fields.filter(f => f.id !== id);
        this.render();
      }

      updateField(id, updates) {
        const field = this.fields.find(f => f.id === id);
        if (field) {
          Object.assign(field, updates);
          this.render();
        }
      }

      render() {
        const container = document.getElementById("fields-container");
        container.innerHTML = "";

        this.fields.forEach(field => {
          const fieldDiv = document.createElement("div");
          fieldDiv.className = "field";
          fieldDiv.innerHTML = `
            <button class="remove danger" onclick="formBuilder.removeField(${field.id})">X</button>
            <label>Label:</label>
            <input type="text" value="${field.label}"
              onchange="formBuilder.updateField(${field.id}, {label: this.value})">

            <label>Placeholder:</label>
            <input type="text" value="${field.placeholder}"
              onchange="formBuilder.updateField(${field.id}, {placeholder: this.value})"
              ${field.type === "select" || field.type === "checkbox" ? "disabled" : ""}>

            ${field.type === "select" ? `
              <label>Opzioni (una per riga):</label>
              <textarea style="width: 100%; height: 80px;"
                onchange="formBuilder.updateField(${field.id}, {options: this.value.split('\\n').filter(o => o)})">
                ${field.options.join("\n")}
              </textarea>
            ` : ""}

            <label>
              <input type="checkbox" ${field.required ? "checked" : ""}
                onchange="formBuilder.updateField(${field.id}, {required: this.checked})">
              Campo obbligatorio
            </label>

            <small style="color: #666;">Tipo: ${field.type}</small>
          `;
          container.appendChild(fieldDiv);
        });

        this.updatePreview();
      }

      updatePreview() {
        const preview = document.getElementById("preview-form");
        preview.innerHTML = "";

        this.fields.forEach(field => {
          const label = document.createElement("label");
          label.textContent = field.label;
          label.style.display = "block";
          label.style.marginTop = "10px";
          label.style.fontWeight = "bold";
          preview.appendChild(label);

          if (field.type === "text" || field.type === "email") {
            const input = document.createElement("input");
            input.type = field.type;
            input.placeholder = field.placeholder;
            input.required = field.required;
            input.style.width = "100%";
            input.style.padding = "8px";
            input.style.margin = "5px 0 10px 0";
            input.style.border = "1px solid #ccc";
            input.style.borderRadius = "4px";
            input.style.boxSizing = "border-box";
            preview.appendChild(input);
          }
        });

        const submitBtn = document.createElement("button");
        submitBtn.type = "submit";
        submitBtn.textContent = "Invia";
        submitBtn.style.marginTop = "10px";
        submitBtn.style.padding = "10px 20px";
        preview.appendChild(submitBtn);
      }

      generateCode() {
        let html = '<form>\n';
        this.fields.forEach(field => {
          html += `  <label>${field.label}</label>\n`;
          html += `  <input type="${field.type}" `;
          html += `placeholder="${field.placeholder}" `;
          html += `${field.required ? 'required' : ''}>\n\n`;
        });
        html += '  <button type="submit">Invia</button>\n</form>';
        document.getElementById("code-output").textContent = html;
      }

      saveForm() {
        localStorage.setItem("form-builder", JSON.stringify(this.fields));
        alert("Form salvato!");
      }

      loadForm() {
        const saved = localStorage.getItem("form-builder");
        if (saved) {
          this.fields = JSON.parse(saved);
          this.fieldId = Math.max(...this.fields.map(f => f.id)) + 1;
          this.render();
        }
      }

      reset() {
        this.fields = [];
        this.fieldId = 0;
        this.render();
        document.getElementById("code-output").textContent = "";
      }
    }

    const formBuilder = new FormBuilder();

    function addTextField() { formBuilder.addField("text"); }
    function addEmailField() { formBuilder.addField("email"); }
    function addSelectField() { formBuilder.addField("select"); }
    function addCheckboxField() { formBuilder.addField("checkbox"); }
    function addTextareaField() { formBuilder.addField("textarea"); }
    function generateCode() { formBuilder.generateCode(); }
    function saveForm() { formBuilder.saveForm(); }
    function resetForm() { formBuilder.reset(); }
  </script>
</body>
</html>
\end{lstlisting}

\section{Esercizi}

\subsection{Esercizio 1 (Base)}
Scrivi una funzione che seleziona tutti i bottoni della pagina e aggiunge la classe "highlighted" solo a quelli che contengono la parola "submit".

\subsection{Esercizio 2 (Intermedio)}
Crea una funzione che con event delegation ascolta click su una lista e, quando clicchi su un item, colora il background di giallo. Quando clicchi di nuovo, lo rimuove.

\subsection{Esercizio 3 (Avanzato)}
Crea una tabella HTML dove ogni riga ha un bottone "Modifica" e "Elimina". Usa event delegation per gestire tutti i click e mostra in console quale riga è stata cliccata.

\section{Riepilogo}

\begin{itemize}
  \item Il DOM è una rappresentazione ad albero del documento HTML
  \item \texttt{parentElement}, \texttt{children}, \texttt{nextSibling} per traversing
  \item \texttt{querySelector} e \texttt{querySelectorAll} con selettori CSS avanzati
  \item Event delegation riduce il numero di listener quando hai molti elementi
  \item \texttt{createElement} per creare elementi dinamicamente
  \item \texttt{classList} per manipolare classi in modo pulito
  \item \texttt{dataset} per accedere a attributi data-*
  \item Template tag \texttt{<template>} perfetto per clonare elementi
  \item Preferisci \texttt{textContent} per sicurezza rispetto a \texttt{innerHTML}
\end{itemize}

%\chapter{Async/Await e Fetch API}
\label{cap:javascript_async}

\section{Cos'è l'Asincronia}

Il codice asincrono permette al browser di continuare a eseguire altre operazioni mentre attende il completamento di operazioni lunghe (richieste di rete, lettura file, timer).

\begin{nota}
JavaScript è single-threaded. L'asincronia non crea nuovi thread, ma usa event loop e callback per eseguire operazioni senza bloccare il thread principale.
\end{nota}

\section{Promises}

Una Promise rappresenta il risultato futuro di un'operazione asincrona: riuscita (resolved) o fallita (rejected).

\subsection{Creare Promise}

\begin{lstlisting}[language=JavaScript]
// Creare una Promise
const myPromise = new Promise((resolve, reject) => {
  setTimeout(() => {
    const success = true;
    if (success) {
      resolve("Operazione completata!");
    } else {
      reject("Operazione fallita!");
    }
  }, 2000);
});

// Usare la Promise con .then() e .catch()
myPromise
  .then(result => console.log(result))
  .catch(error => console.error(error));
\end{lstlisting}

\subsection{Concatenare Promise (Promise Chains)}

\begin{lstlisting}[language=JavaScript]
// Promise chain: ogni .then() ritorna una nuova Promise
fetch("/api/users/1")
  .then(response => response.json())
  .then(user => {
    console.log("Utente:", user.name);
    return fetch(`/api/posts/${user.id}`);
  })
  .then(response => response.json())
  .then(posts => console.log("Post:", posts))
  .catch(error => console.error("Errore:", error));
\end{lstlisting}

\begin{attenzione}
Se dimentichi \texttt{return} in un .then(), la Promise successiva riceverà \texttt{undefined}. Sempre ritornare il valore/Promise.
\end{attenzione}

\subsection{Promise.all() e Promise.race()}

\begin{lstlisting}[language=JavaScript]
// Promise.all: attendi che TUTTE le Promise si risolvano
Promise.all([
  fetch("/api/users").then(r => r.json()),
  fetch("/api/posts").then(r => r.json()),
  fetch("/api/comments").then(r => r.json())
])
  .then(([users, posts, comments]) => {
    console.log("Tutti i dati caricati:", users, posts, comments);
  })
  .catch(error => console.error("Errore:", error));

// Promise.race: ritorna il risultato della prima Promise risolta
Promise.race([
  fetch("/api/slow").then(r => r.json()),
  new Promise((_, reject) => setTimeout(() => reject("Timeout"), 5000))
])
  .then(result => console.log("Primo risultato:", result))
  .catch(error => console.error("Errore:", error));
\end{lstlisting}

\section{Async e Await}

\subsection{Sintassi async/await}

Async/await rende il codice asincrono più leggibile, come se fosse sincrono.

\begin{lstlisting}[language=JavaScript]
// Funzione asincrona
async function fetchUser(userId) {
  try {
    const response = await fetch(`/api/users/${userId}`);
    const user = await response.json();
    console.log("Utente:", user);
    return user;
  } catch (error) {
    console.error("Errore:", error);
  }
}

// Chiamare la funzione asincrona
fetchUser(1);
\end{lstlisting}

\begin{nota}
\texttt{await} pausa l'esecuzione della funzione finché la Promise non si risolve. \texttt{await} funziona solo dentro una funzione \texttt{async}.
\end{nota}

\subsection{Error Handling con try/catch}

\begin{lstlisting}[language=JavaScript]
async function loadData() {
  try {
    // Codice che potrebbe fallire
    const response = await fetch("/api/data");
    if (!response.ok) {
      throw new Error(`HTTP ${response.status}`);
    }
    const data = await response.json();
    console.log("Dati:", data);
    return data;
  } catch (error) {
    console.error("Errore durante il caricamento:", error.message);
    // Gestire l'errore (mostrare messaggio, riprovare, etc.)
  } finally {
    console.log("Operazione completata (successo o fallimento)");
  }
}
\end{lstlisting}

\subsection{Async con funzioni freccia}

\begin{lstlisting}[language=JavaScript]
// Funzione freccia asincrona
const fetchData = async () => {
  const response = await fetch("/api/data");
  return await response.json();
};

// Usare il risultato
fetchData().then(data => console.log(data));

// O usare await nella funzione padre
async function main() {
  const data = await fetchData();
  console.log(data);
}
\end{lstlisting}

\subsection{Eseguire Promise in parallelo}

\begin{lstlisting}[language=JavaScript]
// SBAGLIATO: sequenziale (troppo lento)
async function loadDataSequential() {
  const user = await fetch("/api/user").then(r => r.json());
  const posts = await fetch("/api/posts").then(r => r.json());
  const comments = await fetch("/api/comments").then(r => r.json());
  return { user, posts, comments };
}

// CORRETTO: parallelo (più veloce)
async function loadDataParallel() {
  const [user, posts, comments] = await Promise.all([
    fetch("/api/user").then(r => r.json()),
    fetch("/api/posts").then(r => r.json()),
    fetch("/api/comments").then(r => r.json())
  ]);
  return { user, posts, comments };
}
\end{lstlisting}

\section{Fetch API}

\subsection{GET Request}

\begin{lstlisting}[language=JavaScript]
// Forma semplice
const data = await fetch("/api/users/1")
  .then(r => r.json());

// Forma completa con headers
const response = await fetch("https://api.example.com/users", {
  method: "GET",
  headers: {
    "Content-Type": "application/json",
    "Authorization": "Bearer token123"
  }
});

if (!response.ok) {
  throw new Error(`HTTP ${response.status}`);
}

const data = await response.json();
console.log(data);
\end{lstlisting}

\subsection{POST Request}

\begin{lstlisting}[language=JavaScript]
async function createUser(user) {
  const response = await fetch("/api/users", {
    method: "POST",
    headers: {
      "Content-Type": "application/json"
    },
    body: JSON.stringify({
      name: user.name,
      email: user.email,
      age: user.age
    })
  });

  if (!response.ok) {
    throw new Error("Errore nella creazione dell'utente");
  }

  const newUser = await response.json();
  return newUser;
}

// Usare la funzione
createUser({ name: "Marco", email: "marco@example.com", age: 25 })
  .then(user => console.log("Utente creato:", user))
  .catch(error => console.error(error));
\end{lstlisting}

\subsection{PUT e DELETE Request}

\begin{lstlisting}[language=JavaScript]
// PUT: Aggiornare risorsa
async function updateUser(userId, updates) {
  const response = await fetch(`/api/users/${userId}`, {
    method: "PUT",
    headers: { "Content-Type": "application/json" },
    body: JSON.stringify(updates)
  });
  return await response.json();
}

// DELETE: Eliminare risorsa
async function deleteUser(userId) {
  const response = await fetch(`/api/users/${userId}`, {
    method: "DELETE"
  });
  if (!response.ok) {
    throw new Error("Errore nell'eliminazione");
  }
  return response.status === 204 ? null : response.json();
}

// Usare
updateUser(1, { name: "Marco Rossi", age: 26 });
deleteUser(1);
\end{lstlisting}

\begin{attenzione}
Ricorda di controllare \texttt{response.ok} dopo fetch. Una risposta con status 404 non genera automaticamente un'eccezione; devi controllare manualmente!
\end{attenzione}

\subsection{Gestire JSON e FormData}

\begin{lstlisting}[language=JavaScript]
// Inviare JSON
async function sendJSON() {
  const response = await fetch("/api/data", {
    method: "POST",
    headers: { "Content-Type": "application/json" },
    body: JSON.stringify({ message: "Ciao" })
  });
  return response.json();
}

// Inviare FormData (file upload)
async function uploadFile(file) {
  const formData = new FormData();
  formData.append("file", file);
  formData.append("title", "Mio file");

  const response = await fetch("/api/upload", {
    method: "POST",
    body: formData
    // NON settare Content-Type, il browser lo farà
  });
  return response.json();
}

// Usare
const fileInput = document.querySelector("input[type='file']");
fileInput.addEventListener("change", (e) => {
  uploadFile(e.target.files[0]);
});
\end{lstlisting}

\section{Esercizio Pratico: Weather App}

\subsection{Specifica}

Crea una weather app che:
- Accetta una città in input
- Fetcha dati meteo da API pubblica (OpenWeatherMap o JSONPlaceholder)
- Mostra temperatura, descrizione, umidità
- Salva le ultime 5 città cercate
- Permette di ricercare città con debounce

\subsection{HTML/CSS/JavaScript}

\begin{lstlisting}[language=HTML]
<!DOCTYPE html>
<html lang="it">
<head>
  <meta charset="UTF-8">
  <meta name="viewport" content="width=device-width, initial-scale=1.0">
  <title>Weather App</title>
  <style>
    body {
      font-family: Arial, sans-serif;
      max-width: 600px;
      margin: 0 auto;
      padding: 20px;
      background: linear-gradient(135deg, #667eea 0%, #764ba2 100%);
      min-height: 100vh;
    }

    .container {
      background: white;
      padding: 30px;
      border-radius: 12px;
      box-shadow: 0 8px 32px rgba(0,0,0,0.2);
    }

    h1 {
      color: #333;
      text-align: center;
      margin-top: 0;
    }

    .search-box {
      margin-bottom: 30px;
    }

    .search-box input {
      width: 100%;
      padding: 12px;
      font-size: 16px;
      border: 2px solid #ddd;
      border-radius: 8px;
      box-sizing: border-box;
      transition: border-color 0.3s;
    }

    .search-box input:focus {
      outline: none;
      border-color: #667eea;
    }

    .weather-info {
      background: linear-gradient(135deg, #667eea 0%, #764ba2 100%);
      color: white;
      padding: 20px;
      border-radius: 8px;
      margin-bottom: 20px;
      display: none;
    }

    .weather-info.show {
      display: block;
    }

    .temp {
      font-size: 48px;
      font-weight: bold;
      margin-bottom: 10px;
    }

    .description {
      font-size: 18px;
      margin-bottom: 15px;
      text-transform: capitalize;
    }

    .details {
      display: grid;
      grid-template-columns: 1fr 1fr;
      gap: 15px;
      font-size: 14px;
    }

    .detail-item {
      background: rgba(255,255,255,0.2);
      padding: 10px;
      border-radius: 4px;
    }

    .error {
      background: #fee;
      color: #c00;
      padding: 15px;
      border-radius: 8px;
      display: none;
      margin-bottom: 20px;
    }

    .error.show {
      display: block;
    }

    .history {
      margin-top: 30px;
      padding-top: 20px;
      border-top: 1px solid #ddd;
    }

    .history h3 {
      margin-top: 0;
      color: #333;
    }

    .history-list {
      display: flex;
      flex-wrap: wrap;
      gap: 10px;
    }

    .history-item {
      background: #f0f0f0;
      padding: 8px 12px;
      border-radius: 20px;
      cursor: pointer;
      transition: all 0.3s;
      font-size: 14px;
    }

    .history-item:hover {
      background: #667eea;
      color: white;
    }

    .loading {
      text-align: center;
      color: #667eea;
      display: none;
      margin: 20px 0;
    }

    .loading.show {
      display: block;
    }
  </style>
</head>
<body>
  <div class="container">
    <h1>Meteo App</h1>

    <div class="search-box">
      <input
        type="text"
        id="city-input"
        placeholder="Scrivi il nome di una città..."
        autocomplete="off"
      >
    </div>

    <div class="loading" id="loading">Caricamento...</div>

    <div class="error" id="error-message"></div>

    <div class="weather-info" id="weather-info">
      <div class="temp" id="temp"></div>
      <div class="description" id="description"></div>
      <div class="details">
        <div class="detail-item">
          <strong>Umidità:</strong> <span id="humidity"></span>%
        </div>
        <div class="detail-item">
          <strong>Vento:</strong> <span id="wind"></span> km/h
        </div>
        <div class="detail-item">
          <strong>Sensazione:</strong> <span id="feels"></span>$^\circ$C
        </div>
        <div class="detail-item">
          <strong>Città:</strong> <span id="city-name"></span>
        </div>
      </div>
    </div>

    <div class="history" id="history-section" style="display: none;">
      <h3>Ultime ricerche</h3>
      <div class="history-list" id="history-list"></div>
    </div>
  </div>

  <script>
    class WeatherApp {
      constructor() {
        this.apiKey = "8b3e2f8f5c0c4b1a8d5e6f7g8h9i0j";
        this.history = JSON.parse(localStorage.getItem("weather-history")) || [];
        this.debounceTimer = null;
        this.init();
      }

      init() {
        const input = document.getElementById("city-input");
        input.addEventListener("input", (e) => this.handleSearch(e.target.value));
        this.renderHistory();
      }

      handleSearch(query) {
        clearTimeout(this.debounceTimer);

        if (!query) return;

        this.debounceTimer = setTimeout(() => {
          this.searchCity(query);
        }, 500);
      }

      async searchCity(city) {
        try {
          this.showLoading(true);
          this.hideError();

          // Usare API pubblica (esempio con JSONPlaceholder mock)
          const response = await fetch(
            `https://api.openweathermap.org/data/2.5/weather?q=${city}&appid=${this.apiKey}&units=metric&lang=it`
          );

          if (!response.ok) {
            if (response.status === 404) {
              throw new Error("Città non trovata");
            }
            throw new Error("Errore nel caricamento dei dati meteo");
          }

          const data = await response.json();
          this.displayWeather(data);
          this.addToHistory(city);
        } catch (error) {
          this.showError(error.message);
        } finally {
          this.showLoading(false);
        }
      }

      displayWeather(data) {
        const info = document.getElementById("weather-info");
        document.getElementById("temp").textContent = Math.round(data.main.temp) + "$^\circ$C";
        document.getElementById("description").textContent = data.weather[0].description;
        document.getElementById("humidity").textContent = data.main.humidity;
        document.getElementById("wind").textContent = Math.round(data.wind.speed * 3.6);
        document.getElementById("feels").textContent = Math.round(data.main.feels_like);
        document.getElementById("city-name").textContent = data.name + ", " + data.sys.country;

        info.classList.add("show");
      }

      addToHistory(city) {
        this.history = this.history.filter(c => c.toLowerCase() !== city.toLowerCase());
        this.history.unshift(city);
        this.history = this.history.slice(0, 5);
        localStorage.setItem("weather-history", JSON.stringify(this.history));
        this.renderHistory();
      }

      renderHistory() {
        if (this.history.length === 0) {
          document.getElementById("history-section").style.display = "none";
          return;
        }

        document.getElementById("history-section").style.display = "block";
        const list = document.getElementById("history-list");
        list.innerHTML = "";

        this.history.forEach(city => {
          const item = document.createElement("div");
          item.className = "history-item";
          item.textContent = city;
          item.onclick = () => this.searchCity(city);
          list.appendChild(item);
        });
      }

      showLoading(show) {
        document.getElementById("loading").classList.toggle("show", show);
      }

      showError(message) {
        const error = document.getElementById("error-message");
        error.textContent = message;
        error.classList.add("show");
      }

      hideError() {
        document.getElementById("error-message").classList.remove("show");
      }
    }

    new WeatherApp();
  </script>
</body>
</html>
\end{lstlisting}

\begin{nota}
Nota: l'API key è fittizia. Registrati su openweathermap.org per una vera API key.
\end{nota}

\section{Esercizi}

\subsection{Esercizio 1 (Intermedio)}
Fetch una lista di post da JSONPlaceholder (/posts) e mostra titolo e corpo in una lista.

\subsection{Esercizio 2 (Intermedio)}
Crea una funzione che fetcha 3 risorse diverse in parallelo e le mostra tutte.

\subsection{Esercizio 3 (Avanzato)}
Crea un form che invia dati POST a una API. Gestisci success e errori con try/catch.

\subsection{Esercizio 4 (Avanzato)}
Implementa un retry mechanism: se una fetch fallisce, ritenta fino a 3 volte prima di rinunciare.

\section{Riepilogo}

\begin{itemize}
  \item Promise rappresentano risultati futuri di operazioni asincrone
  \item .then() e .catch() per gestire Promise
  \item \texttt{async/await} è zucchero sintattico per Promise chains
  \item \texttt{await} pausa l'esecuzione finché Promise non si risolve
  \item \texttt{try/catch/finally} per error handling con async/await
  \item Fetch API con \texttt{fetch(url, options)} per richieste HTTP
  \item Controllare sempre \texttt{response.ok} dopo fetch
  \item \texttt{response.json()} per parsare JSON
  \item Combinare fetch con async/await per codice leggibile
  \item Usare \texttt{Promise.all()} per parallelizzare richieste
\end{itemize}

%\chapter{React/Vue Intro e Deployment}
\label{cap:javascript_framework}

\section{Cos'è un Framework JavaScript}

Un framework JavaScript (come React e Vue) permette di costruire interfacce utente complesse in modo organizzato, con component riutilizzabili, gestione dello stato e rendering reattivo.

\begin{nota}
React è una libreria per costruire UI con un approccio basato su componenti. Vue è un framework più leggero con sintassi template. Entrambi rendono la gestione di pagine dinamiche più semplice.
\end{nota}

\section{React Basics}

\subsection{JSX Syntax}

JSX permette di scrivere HTML-like syntax dentro JavaScript.

\begin{lstlisting}[language=JavaScript]
// JSX viene trasformato in React.createElement()
// Questo JSX:
const element = <h1 className="greeting">Hello World</h1>;

// Diventa:
const element = React.createElement("h1", { className: "greeting" }, "Hello World");

// Componenti JSX
function Greeting() {
  return <h1>Ciao, sono un componente!</h1>;
}

// Usare il componente
const app = <Greeting />;
\end{lstlisting}

\begin{attenzione}
Nota: \texttt{class} in HTML diventa \texttt{className} in JSX. Usa \texttt{htmlFor} invece di \texttt{for}.
\end{attenzione}

\subsection{Componenti Funzionali}

\begin{lstlisting}[language=JavaScript]
// Componente funzionale semplice
function Welcome(props) {
  return <h1>Benvenuto, {props.name}!</h1>;
}

// Usare il componente
<Welcome name="Marco" />  // Output: <h1>Benvenuto, Marco!</h1>

// Componente con più contenuto
function Card({ title, description, imageUrl }) {
  return (
    <div className="card">
      <img src={imageUrl} alt={title} />
      <h2>{title}</h2>
      <p>{description}</p>
    </div>
  );
}

// Destructuring dei props
function Button({ label, onClick, disabled }) {
  return (
    <button onClick={onClick} disabled={disabled}>
      {label}
    </button>
  );
}
\end{lstlisting}

\subsection{Hook: useState}

\begin{lstlisting}[language=JavaScript]
import { useState } from 'react';

// Componente con stato
function Counter() {
  const [count, setCount] = useState(0);

  return (
    <div>
      <p>Contatore: {count}</p>
      <button onClick={() => setCount(count + 1)}>
        Incrementa
      </button>
      <button onClick={() => setCount(0)}>
        Reset
      </button>
    </div>
  );
}

// Stato complesso
function Form() {
  const [formData, setFormData] = useState({
    name: "",
    email: "",
    message: ""
  });

  const handleChange = (e) => {
    const { name, value } = e.target;
    setFormData({
      ...formData,
      [name]: value
    });
  };

  return (
    <form onSubmit={(e) => { e.preventDefault(); console.log(formData); }}>
      <input name="name" value={formData.name} onChange={handleChange} />
      <input name="email" value={formData.email} onChange={handleChange} />
      <textarea name="message" value={formData.message} onChange={handleChange} />
      <button type="submit">Invia</button>
    </form>
  );
}
\end{lstlisting}

\subsection{Hook: useEffect}

\begin{lstlisting}[language=JavaScript]
import { useState, useEffect } from 'react';

// Componente con side effect
function UserProfile({ userId }) {
  const [user, setUser] = useState(null);
  const [loading, setLoading] = useState(true);

  useEffect(() => {
    // Questo codice runs quando componente monta o userId cambia
    const fetchUser = async () => {
      try {
        const response = await fetch(`/api/users/${userId}`);
        const data = await response.json();
        setUser(data);
      } catch (error) {
        console.error("Errore:", error);
      } finally {
        setLoading(false);
      }
    };

    fetchUser();
  }, [userId]); // Dependency array

  if (loading) return <p>Caricamento...</p>;
  if (!user) return <p>Utente non trovato</p>;

  return (
    <div>
      <h1>{user.name}</h1>
      <p>Email: {user.email}</p>
    </div>
  );
}

// useEffect senza dipendenze: runs solo al mount
useEffect(() => {
  console.log("Componente montato!");
  return () => console.log("Componente smontato!");
}, []);
\end{lstlisting}

\begin{nota}
Dependency array: [] = runs solo al mount; [dep1, dep2] = runs quando dipendenze cambiano; niente = runs dopo ogni render (pericoloso!).
\end{nota}

\subsection{Lista di Elementi}

\begin{lstlisting}[language=JavaScript]
function TodoList() {
  const [todos, setTodos] = useState([
    { id: 1, text: "Compra latte" },
    { id: 2, text: "Studia React" }
  ]);

  const deleteTodo = (id) => {
    setTodos(todos.filter(todo => todo.id !== id));
  };

  return (
    <ul>
      {todos.map(todo => (
        <li key={todo.id}>
          {todo.text}
          <button onClick={() => deleteTodo(todo.id)}>Elimina</button>
        </li>
      ))}
    </ul>
  );
}
\end{lstlisting}

\section{Vue Basics}

\subsection{Template Syntax}

\begin{lstlisting}[language=HTML]
<!-- Interpolazione -->
<div>{{ message }}</div>
<div>{{ firstName + " " + lastName }}</div>

<!-- Attributi -->
<img :src="imageUrl" :alt="imageAlt" />
<div :class="{ active: isActive }"></div>
<div :style="{ color: dynamicColor }"></div>

<!-- Attributi booleani -->
<button :disabled="isLoading">Submit</button>

<!-- Espressioni brevi -->
<div v-bind:id="dynamicId"></div> <!-- Lungo -->
<div :id="dynamicId"></div>       <!-- Corto -->
\end{lstlisting}

\subsection{Direttive Condizionali}

\begin{lstlisting}[language=HTML]
<!-- v-if, v-else-if, v-else -->
<div v-if="status === 'loading'">Caricamento...</div>
<div v-else-if="status === 'error'">Errore!</div>
<div v-else>Completato!</div>

<!-- v-show: usa display:none invece di rimuovere dal DOM -->
<div v-show="isVisible">Mostra/nascondi con CSS</div>

<!-- v-if vs v-show: v-if è meglio per toggle rari, v-show per frequenti -->
\end{lstlisting}

\subsection{Direttive di Iterazione}

\begin{lstlisting}[language=HTML]
<!-- v-for con array -->
<ul>
  <li v-for="item in items" :key="item.id">
    {{ item.name }}
  </li>
</ul>

<!-- v-for con indice -->
<div v-for="(item, index) in items" :key="index">
  {{ index }}: {{ item }}
</div>

<!-- v-for con oggetto -->
<div v-for="(value, key) in user" :key="key">
  {{ key }}: {{ value }}
</div>

<!-- v-for annidati -->
<ul>
  <li v-for="category in categories" :key="category.id">
    {{ category.name }}
    <ul>
      <li v-for="item in category.items" :key="item.id">
        {{ item }}
      </li>
    </ul>
  </li>
</ul>
\end{lstlisting}

\begin{attenzione}
Sempre usa :key nei v-for. Non usare indice come key se la lista può cambiare ordine!
\end{attenzione}

\subsection{Event Handling}

\begin{lstlisting}[language=HTML]
<!-- v-on o @ shorthand -->
<button v-on:click="handleClick">Click me</button>
<button @click="handleClick">Click me (shorthand)</button>

<!-- Passare argomenti -->
<button @click="handleClick('param1', 42)">Click</button>

<!-- Accedere all'evento -->
<input @keydown.enter="handleEnter" />
<button @click.right="handleRightClick">Right click</button>
<button @click.prevent="handleClick">Previeni default</button>
\end{lstlisting}

\begin{lstlisting}[language=JavaScript]
export default {
  data() {
    return {
      message: "Ciao Vue!",
      count: 0
    };
  },
  methods: {
    handleClick() {
      this.count++;
      console.log("Cliccato!");
    },
    handleEnter() {
      this.message = "Enter premuto!";
    }
  }
};
\end{lstlisting}

\subsection{Two-Way Binding}

\begin{lstlisting}[language=HTML]
<!-- v-model per input bidirrezionali -->
<input v-model="name" placeholder="Nome" />
<p>{{ name }}</p>

<!-- Con checkbox -->
<input type="checkbox" v-model="agreed" />
<p v-if="agreed">Accettato!</p>

<!-- Con select -->
<select v-model="selectedCity">
  <option value="">Seleziona città</option>
  <option value="roma">Roma</option>
  <option value="milano">Milano</option>
</select>

<!-- Con radio -->
<input type="radio" value="yes" v-model="answer" /> Sì
<input type="radio" value="no" v-model="answer" /> No
\end{lstlisting}

\begin{lstlisting}[language=JavaScript]
export default {
  data() {
    return {
      name: "",
      agreed: false,
      selectedCity: "",
      answer: ""
    };
  }
};
\end{lstlisting}

\section{Component Lifecycle}

\subsection{React Lifecycle (hooks)}

\begin{lstlisting}[language=JavaScript]
function MyComponent() {
  // Mount
  useEffect(() => {
    console.log("Componente montato");
    return () => console.log("Componente smontato");
  }, []);

  // Update (quando dipendenza cambia)
  useEffect(() => {
    console.log("Componente aggiornato");
  }, [someDependency]);

  // Update generale
  useEffect(() => {
    console.log("Componente montato o aggiornato");
  });

  return <div>Contenuto</div>;
}
\end{lstlisting}

\subsection{Vue Lifecycle}

\begin{lstlisting}[language=JavaScript]
export default {
  created() {
    // Componente creato, dati disponibili
    console.log("created");
  },
  mounted() {
    // Componente inserito nel DOM
    console.log("mounted - accedi a template");
  },
  updated() {
    // Dati sono cambiati, DOM aggiornato
    console.log("updated");
  },
  unmounted() {
    // Componente rimosso dal DOM
    console.log("unmounted - cleanup");
  }
};
\end{lstlisting}

\section{State Management}

\subsection{Props per passare dati}

\begin{lstlisting}[language=JavaScript]
// React
function Parent() {
  const [count, setCount] = useState(0);
  return <Child count={count} onIncrement={() => setCount(count + 1)} />;
}

function Child({ count, onIncrement }) {
  return (
    <div>
      <p>{count}</p>
      <button onClick={onIncrement}>Incrementa</button>
    </div>
  );
}

// Vue
<!-- Parent.vue -->
<template>
  <Child :count="count" @increment="count++" />
</template>

<script>
import Child from './Child.vue';
export default {
  components: { Child },
  data() { return { count: 0 }; }
};
</script>

<!-- Child.vue -->
<template>
  <div>
    <p>{{ count }}</p>
    <button @click="$emit('increment')">Incrementa</button>
  </div>
</template>

<script>
export default {
  props: ['count']
};
</script>
\end{lstlisting}

\section{Deployment}

\subsection{Netlify}

\begin{lstlisting}[language=bash]
# 1. Installare Netlify CLI
npm install -g netlify-cli

# 2. Buildare il progetto
npm run build

# 3. Deployare
netlify deploy --prod --dir=dist

# O configurare deploys automatici
# Collegare repo GitHub a Netlify, settare:
# - Build command: npm run build
# - Publish directory: dist
\end{lstlisting}

\subsection{Vercel (per Next.js)}

\begin{lstlisting}[language=bash]
# 1. Installare Vercel CLI
npm install -g vercel

# 2. Deployare
vercel --prod

# O collegare GitHub repo a vercel.com
# Vercel auto-detecta Next.js e configure tutto
\end{lstlisting}

\subsection{GitHub Pages}

\begin{lstlisting}[language=bash]
# Per progetti React/Vue statici
# 1. Aggiungere a package.json:
"homepage": "https://username.github.io/repo-name"

# 2. Installare gh-pages
npm install --save-dev gh-pages

# 3. Aggiungere scripts:
"scripts": {
  "deploy": "npm run build && gh-pages -d build",
  "build": "react-scripts build"
}

# 4. Deployare
npm run deploy

# Oppure manualmente in GitHub:
# - Creare branch gh-pages
# - Pushare build files
# - Settings > Pages > Source: gh-pages
\end{lstlisting}

\section{Esercizio Pratico: React Todo App}

\subsection{Componenti}

\begin{lstlisting}[language=JavaScript]
// App.jsx
import { useState, useEffect } from 'react';
import './App.css';

function App() {
  const [todos, setTodos] = useState([]);
  const [input, setInput] = useState('');

  // Caricare todos da localStorage
  useEffect(() => {
    const saved = localStorage.getItem('todos');
    if (saved) setTodos(JSON.parse(saved));
  }, []);

  // Salvare todos quando cambiano
  useEffect(() => {
    localStorage.setItem('todos', JSON.stringify(todos));
  }, [todos]);

  const addTodo = () => {
    if (!input.trim()) return;
    setTodos([...todos, { id: Date.now(), text: input, done: false }]);
    setInput('');
  };

  const toggleTodo = (id) => {
    setTodos(todos.map(todo =>
      todo.id === id ? { ...todo, done: !todo.done } : todo
    ));
  };

  const deleteTodo = (id) => {
    setTodos(todos.filter(todo => todo.id !== id));
  };

  return (
    <div className="app">
      <h1>My Todos</h1>
      <div className="input-group">
        <input
          value={input}
          onChange={(e) => setInput(e.target.value)}
          onKeyPress={(e) => e.key === 'Enter' && addTodo()}
          placeholder="Aggiungi task..."
        />
        <button onClick={addTodo}>Aggiungi</button>
      </div>
      <ul className="todo-list">
        {todos.map(todo => (
          <li key={todo.id} className={todo.done ? 'done' : ''}>
            <input
              type="checkbox"
              checked={todo.done}
              onChange={() => toggleTodo(todo.id)}
            />
            <span>{todo.text}</span>
            <button onClick={() => deleteTodo(todo.id)}>X</button>
          </li>
        ))}
      </ul>
      <p>{todos.filter(t => !t.done).length} task rimanenti</p>
    </div>
  );
}

export default App;
\end{lstlisting}

\subsection{Styling}

\begin{lstlisting}[language=CSS]
/* App.css */
.app {
  max-width: 500px;
  margin: 50px auto;
  padding: 20px;
  font-family: Arial, sans-serif;
}

h1 {
  color: #333;
  text-align: center;
}

.input-group {
  display: flex;
  gap: 10px;
  margin-bottom: 20px;
}

.input-group input {
  flex: 1;
  padding: 10px;
  border: 1px solid #ddd;
  border-radius: 4px;
  font-size: 14px;
}

.input-group button {
  padding: 10px 20px;
  background: #007bff;
  color: white;
  border: none;
  border-radius: 4px;
  cursor: pointer;
  font-size: 14px;
}

.input-group button:hover {
  background: #0056b3;
}

.todo-list {
  list-style: none;
  padding: 0;
}

.todo-list li {
  display: flex;
  align-items: center;
  padding: 10px;
  border-bottom: 1px solid #eee;
  gap: 10px;
}

.todo-list li.done {
  opacity: 0.6;
}

.todo-list li.done span {
  text-decoration: line-through;
}

.todo-list button {
  margin-left: auto;
  padding: 5px 10px;
  background: #dc3545;
  color: white;
  border: none;
  border-radius: 4px;
  cursor: pointer;
  font-size: 12px;
}
\end{lstlisting}

\section{Esercizi}

\subsection{Esercizio 1 (Base)}
Crea un componente React che mostra un contatore con bottoni +/- e Reset.

\subsection{Esercizio 2 (Intermedio)}
Crea un form con input text, email, select. Raccogli i dati e mostrali in una preview sotto il form.

\subsection{Esercizio 3 (Avanzato)}
Crea un componente React che fetcha una lista di utenti da JSONPlaceholder e li mostra in una tabella. Aggiungi possibilità di filtrare per nome.

\subsection{Esercizio 4 (Avanzato)}
Converti il React Todo App in Vue usando script setup (composition API).

\section{Riepilogo}

\begin{itemize}
  \item React usa JSX per scrivere componenti in modo dichiarativo
  \item \texttt{useState} per gestire stato locale nel componente
  \item \texttt{useEffect} per side effects come fetch dati
  \item Props per passare dati da parent a child
  \item Vue usa template syntax per separare HTML da JS
  \item \texttt{v-if}, \texttt{v-for}, \texttt{v-model} direttive comuni di Vue
  \item Lifecycle hooks per mount/update/unmount
  \item Tutti i framework moderni rendono le applicazioni più mantenibili
  \item Deployment su Netlify, Vercel, GitHub Pages molto semplice
  \item Build step necessario prima del deploy (npm run build)
\end{itemize}


%\backmatter
\chapter{Bibliografia e Risorse}
\label{cap:bibliografia}

\section{Documentazione ufficiale}

Consulta MDN Web Docs per HTML e CSS, W3C per specifiche tecniche, Can I Use per compatibilità browser.

\section{Tutorial e Learning}

CodePen, JsFiddle, CSS-Tricks per articoli e tutorial. Flexbox Froggy e Grid Garden per giochi interattivi.

\section{Tool online}

W3C HTML Validator, W3C CSS Validator, WebAIM WCAG Checker per validazione. Sass Playground per compilazione SCSS online.

\section{Editor e IDE}

VS Code (consigliato con Live Server), Sublime Text, WebStorm, Atom per sviluppo HTML/CSS.

\section{Editor e IDE}

\begin{itemize}
  \item \textbf{VS Code} (Gratuito) - Raccomandato
    \begin{itemize}
      \item Estensioni: Live Server, CSS Intellisense, Prettier
    \end{itemize}
  \item \textbf{Sublime Text} (Pagato, prova gratuita)
  \item \textbf{WebStorm} (Pagato, ma completo)
  \item \textbf{Atom} (Gratuito, da GitHub)
\end{itemize}

\section{Browser Developer Tools}

\begin{itemize}
  \item \textbf{Chrome DevTools}: F12 o Ctrl+Shift+I
  \item \textbf{Firefox Developer Edition}: F12
  \item \textbf{Safari Web Inspector}: Cmd+Option+I
  \item Strumenti essenziali: Inspector, Console, Network, Performance
\end{itemize}

\section{Libri consigliati}

\begin{itemize}
  \item ``HTML \& CSS: Design and Build Websites'' - Jon Duckett
  \item ``Responsive Web Design'' - Ethan Marcotte
  \item ``CSS Secrets'' - Lea Verou
  \item ``The Pragmatic Programmer'' - (contiene best practices web)
\end{itemize}

\section{Community e Forum}

\begin{itemize}
  \item \textbf{Stack Overflow}: \url{https://stackoverflow.com/} - Q\&A
  \item \textbf{CSS-Tricks Community}: Forum e discussioni
  \item \textbf{Reddit r/webdev}: Comunità web developers
  \item \textbf{GitHub}: Cerca progetti open source per imparare
\end{itemize}

\section{Collegamento con altri corsi}

Come visto in questo corso:

\begin{itemize}
  \item \textbf{Terza (Linguaggio C)}: Introduzione a C - Logica e algoritmi
  \item \textbf{Quarta (Java)}: Classi, Oggetti, Ereditarietà e Package - OOP e backend
  \item \textbf{Quarta (HTML/CSS)}: Questo corso - Frontend web
\end{itemize}

---

\textit{Ultimo aggiornamento: Novembre 2025}


\end{document}
