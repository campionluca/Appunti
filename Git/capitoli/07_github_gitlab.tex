\chapter{GitHub e GitLab}

\section*{Introduzione}
GitHub e GitLab sono piattaforme di hosting per repository Git che offrono strumenti di collaborazione, CI/CD, issue tracking e molto altro. Questo capitolo esplora le funzionalità principali di entrambe le piattaforme, con focus su GitHub che è la più diffusa.

\section*{Obiettivi di apprendimento}
\begin{itemize}
    \item Comprendere differenze tra GitHub e GitLab
    \item Gestire repository remoti su piattaforme cloud
    \item Usare issues per tracciare bug e feature
    \item Organizzare progetti con GitHub Projects
    \item Creare e gestire Pull Request
    \item Automatizzare workflow con GitHub Actions
    \item Configurare webhooks per integrazioni
    \item Pubblicare siti statici con GitHub Pages
\end{itemize}

\section{GitHub vs GitLab: Confronto}

\subsection{Caratteristiche Principali}

\begin{center}
\begin{tabular}{|l|l|l|}
\hline
\textbf{Caratteristica} & \textbf{GitHub} & \textbf{GitLab} \\
\hline
Hosting & Cloud / Enterprise & Cloud / Self-hosted \\
Repository privati & Illimitati (gratis) & Illimitati (gratis) \\
CI/CD integrato & GitHub Actions & GitLab CI (più maturo) \\
Issue tracking & Sì & Sì (più avanzato) \\
Wiki & Sì & Sì \\
Container registry & Sì & Sì (integrato) \\
Code review & Pull Request & Merge Request \\
Community & Più grande & Crescente \\
\hline
\end{tabular}
\end{center}

\begin{tcolorbox}[colback=blue!10, colframe=blue!60, title=Quale Scegliere?]
\begin{itemize}
    \item \textbf{GitHub}: Preferito per open source, community enorme, integrazioni abbondanti
    \item \textbf{GitLab}: Migliore per CI/CD complesso, self-hosting, DevOps completo
    \item Entrambi offrono funzionalità simili per la maggior parte dei casi d'uso
\end{itemize}
\end{tcolorbox}

\section{Repository su GitHub}

\subsection{Creare un Repository}

\begin{lstlisting}
# Via web: github.com -> New repository
# Nome: my-project
# Visibilità: Public / Private
# Initialize: README, .gitignore, License

# Collegare repository locale a GitHub
git remote add origin https://github.com/username/my-project.git

# Push iniziale
git branch -M main
git push -u origin main
\end{lstlisting}

\subsection{Clonare Repository}

\begin{lstlisting}
# Clone via HTTPS
git clone https://github.com/username/repo.git

# Clone via SSH (raccomandato)
git clone git@github.com:username/repo.git

# Clone di branch specifico
git clone -b develop https://github.com/username/repo.git

# Clone shallow (solo ultimo commit, più veloce)
git clone --depth 1 https://github.com/username/repo.git
\end{lstlisting}

\subsection{SSH Keys per GitHub}

\begin{lstlisting}
# Generare chiave SSH
ssh-keygen -t ed25519 -C "your_email@example.com"

# Aggiungere a ssh-agent
eval "$(ssh-agent -s)"
ssh-add ~/.ssh/id_ed25519

# Copiare chiave pubblica
cat ~/.ssh/id_ed25519.pub

# Incollare in GitHub:
# Settings -> SSH and GPG keys -> New SSH key

# Testare connessione
ssh -T git@github.com
# Hi username! You've successfully authenticated...
\end{lstlisting}

\section{Issues: Tracciamento Attività}

\subsection{Cosa Sono le Issues}

Le issues sono il sistema di ticket per tracciare:
\begin{itemize}
    \item Bug reports
    \item Feature requests
    \item Domande e discussioni
    \item To-do list per il progetto
\end{itemize}

\subsection{Anatomia di una Issue}

\begin{tcolorbox}[colback=gray!10, colframe=gray!60, title=Esempio Issue]
\textbf{Titolo}: Login button not responsive on mobile

\textbf{Descrizione}:
\begin{lstlisting}
## Bug Description
The login button on mobile devices doesn't respond to touch events.

## Steps to Reproduce
1. Open app on mobile (iOS/Android)
2. Navigate to login page
3. Tap login button
4. Nothing happens

## Expected Behavior
Should submit login form

## Environment
- Device: iPhone 12
- OS: iOS 15.2
- Browser: Safari

## Screenshots
[attach screenshot]
\end{lstlisting}

\textbf{Labels}: bug, mobile, high-priority

\textbf{Assignee}: @developer-name

\textbf{Milestone}: v1.2.0
\end{tcolorbox}

\subsection{Comandi Issue da Git}

\begin{lstlisting}
# Referenziare issue in commit
git commit -m "Fix login button (#42)"

# Chiudere issue automaticamente
git commit -m "Fix mobile login

Fixes #42
Closes #43"

# Quando fai push, issues #42 e #43 vengono chiuse
\end{lstlisting}

\subsection{Labels e Milestones}

\begin{lstlisting}
# Labels comuni:
bug              # Errori da fixare
enhancement      # Nuove feature
documentation    # Miglioramenti doc
good first issue # Per nuovi contributor
help wanted      # Richiesta aiuto
wontfix          # Non verrà implementato
duplicate        # Duplicato di altra issue

# Milestones:
v1.0.0 - Initial Release (15 issues)
v1.1.0 - Mobile Support (8 issues)
v2.0.0 - Major Refactor (23 issues)
\end{lstlisting}

\subsection{Issue Templates}

File \texttt{.github/ISSUE\_TEMPLATE/bug\_report.md}:
\begin{lstlisting}
---
name: Bug Report
about: Report a bug
title: '[BUG] '
labels: bug
assignees: ''
---

## Describe the bug
A clear description of what the bug is.

## To Reproduce
Steps to reproduce:
1. Go to '...'
2. Click on '...'
3. See error

## Expected behavior
What you expected to happen.

## Screenshots
If applicable, add screenshots.

## Environment
- OS: [e.g. Windows 10]
- Browser: [e.g. Chrome 95]
- Version: [e.g. 1.2.0]
\end{lstlisting}

\section{GitHub Projects: Project Management}

\subsection{Creazione Project Board}

\begin{enumerate}
    \item Repository → Projects → New Project
    \item Scegli template: Kanban, Table, Roadmap
    \item Configura colonne: To Do, In Progress, Done
\end{enumerate}

\subsection{Automazione Project Board}

\begin{lstlisting}
# Workflow automatico:
- Issue aperta → colonna "To Do"
- PR aperta → colonna "In Progress"
- PR merged → colonna "Done"

# Configurazione automazione:
Project Settings → Workflows → Enable automation
\end{lstlisting}

\subsection{Esempio Kanban Board}

\begin{tcolorbox}[colback=blue!10, colframe=blue!60, title=Sprint Board Example]
\textbf{Backlog}: 12 issues

\textbf{To Do} (Sprint corrente):
\begin{itemize}
    \item \#45 Implement user authentication
    \item \#46 Add password reset
    \item \#47 Create dashboard
\end{itemize}

\textbf{In Progress}:
\begin{itemize}
    \item \#42 Fix mobile login (assigned to @dev1)
    \item \#43 Optimize database queries (assigned to @dev2)
\end{itemize}

\textbf{Code Review}:
\begin{itemize}
    \item \#40 Add email notifications (PR \#105)
\end{itemize}

\textbf{Done} (questo sprint):
\begin{itemize}
    \item \#38 Setup CI/CD pipeline
    \item \#39 Configure production server
\end{itemize}
\end{tcolorbox}

\section{Pull Requests}

\subsection{Workflow Pull Request}

\begin{lstlisting}
# 1. Crea branch per feature
git checkout -b feature/user-auth

# 2. Implementa e committa
git add .
git commit -m "Add user authentication system"

# 3. Push branch
git push -u origin feature/user-auth

# 4. Apri PR su GitHub
# Repository -> Pull Requests -> New Pull Request
# Base: main <- Compare: feature/user-auth

# 5. Code review e discussione

# 6. Approva e merge
# Merge pull request (create merge commit)
# Squash and merge (combina commit)
# Rebase and merge (linear history)
\end{lstlisting}

\subsection{Template Pull Request}

File \texttt{.github/PULL\_REQUEST\_TEMPLATE.md}:
\begin{lstlisting}
## Description
Brief description of changes.

## Type of Change
- [ ] Bug fix
- [ ] New feature
- [ ] Breaking change
- [ ] Documentation update

## Related Issues
Fixes #(issue)

## Testing
Describe tests performed:
- [ ] Unit tests pass
- [ ] Integration tests pass
- [ ] Manual testing completed

## Checklist
- [ ] Code follows style guidelines
- [ ] Self-review completed
- [ ] Comments added for complex code
- [ ] Documentation updated
- [ ] No new warnings generated

## Screenshots
If applicable.
\end{lstlisting}

\subsection{Code Review Best Practices}

\begin{tcolorbox}[colback=green!10, colframe=green!60, title=Come Fare Code Review]
\textbf{Reviewer}:
\begin{itemize}
    \item Sii costruttivo e specifico nei commenti
    \item Suggerisci soluzioni, non solo problemi
    \item Approva quando il codice è sufficientemente buono
    \item Usa "Request changes" solo per problemi bloccanti
\end{itemize}

\textbf{Author}:
\begin{itemize}
    \item Mantieni PR piccole (< 400 righe)
    \item Descrivi chiaramente le modifiche
    \item Rispondi a tutti i commenti
    \item Non prendere critiche personalmente
\end{itemize}
\end{tcolorbox}

\subsection{Protected Branches}

\begin{lstlisting}
# Settings -> Branches -> Add rule

Branch name pattern: main

Protezioni:
[X] Require pull request before merging
    [X] Require approvals (minimo 1-2)
    [X] Dismiss stale reviews
[X] Require status checks to pass
    [X] Require branches to be up to date
    - CI/CD tests
    - Code quality checks
[X] Require linear history
[X] Include administrators
\end{lstlisting}

\section{GitHub Actions: CI/CD}

\subsection{Cos'è GitHub Actions}

GitHub Actions automatizza workflow di sviluppo: testing, building, deployment, e altro. I workflow sono definiti in file YAML in \texttt{.github/workflows/}.

\subsection{Workflow Base: Test su Push}

File \texttt{.github/workflows/test.yml}:
\begin{lstlisting}[language=bash]
name: Run Tests

on:
  push:
    branches: [ main, develop ]
  pull_request:
    branches: [ main ]

jobs:
  test:
    runs-on: ubuntu-latest

    steps:
    - name: Checkout code
      uses: actions/checkout@v3

    - name: Setup Node.js
      uses: actions/setup-node@v3
      with:
        node-version: '18'

    - name: Install dependencies
      run: npm ci

    - name: Run tests
      run: npm test

    - name: Upload coverage
      uses: codecov/codecov-action@v3
      with:
        files: ./coverage/coverage.xml
\end{lstlisting}

\subsection{Workflow Avanzato: Matrix Testing}

\begin{lstlisting}[language=bash]
name: Matrix Test

on: [push, pull_request]

jobs:
  test:
    runs-on: ${{ matrix.os }}
    strategy:
      matrix:
        os: [ubuntu-latest, windows-latest, macos-latest]
        node: [14, 16, 18]

    steps:
    - uses: actions/checkout@v3
    - name: Setup Node ${{ matrix.node }}
      uses: actions/setup-node@v3
      with:
        node-version: ${{ matrix.node }}
    - run: npm ci
    - run: npm test
\end{lstlisting}

\subsection{Deployment Automatico}

\begin{lstlisting}[language=bash]
name: Deploy to Production

on:
  push:
    branches: [ main ]

jobs:
  deploy:
    runs-on: ubuntu-latest

    steps:
    - uses: actions/checkout@v3

    - name: Build application
      run: |
        npm ci
        npm run build

    - name: Deploy to server
      uses: appleboy/ssh-action@master
      with:
        host: ${{ secrets.SERVER_HOST }}
        username: ${{ secrets.SERVER_USER }}
        key: ${{ secrets.SSH_PRIVATE_KEY }}
        script: |
          cd /var/www/app
          git pull origin main
          npm install
          pm2 restart app
\end{lstlisting}

\subsection{Secrets e Environment Variables}

\begin{lstlisting}
# Settings -> Secrets and variables -> Actions

Secrets (encrypted):
- DATABASE_URL
- API_KEY
- SSH_PRIVATE_KEY

# Uso nel workflow:
env:
  DATABASE_URL: ${{ secrets.DATABASE_URL }}

# O direttamente:
run: echo "${{ secrets.API_KEY }}" > .env
\end{lstlisting}

\section{Webhooks}

\subsection{Cos'è un Webhook}

Un webhook è un HTTP callback che GitHub invia al tuo server quando avvengono eventi specifici nel repository.

\subsection{Configurazione Webhook}

\begin{lstlisting}
# Settings -> Webhooks -> Add webhook

Payload URL: https://your-server.com/webhook
Content type: application/json
Secret: your-secret-token

Events:
[X] Push
[X] Pull request
[X] Issues
[ ] Just the push event
[ ] Send me everything
\end{lstlisting}

\subsection{Esempio Server Webhook (Node.js)}

\begin{lstlisting}[language=bash]
const express = require('express');
const crypto = require('crypto');
const app = express();

app.use(express.json());

// Verifica firma GitHub
function verifySignature(req) {
    const signature = req.headers['x-hub-signature-256'];
    const hmac = crypto.createHmac('sha256', process.env.WEBHOOK_SECRET);
    const digest = 'sha256=' + hmac.update(
        JSON.stringify(req.body)
    ).digest('hex');
    return signature === digest;
}

app.post('/webhook', (req, res) => {
    if (!verifySignature(req)) {
        return res.status(403).send('Invalid signature');
    }

    const event = req.headers['x-github-event'];
    const payload = req.body;

    switch(event) {
        case 'push':
            console.log(`Push to ${payload.ref}`);
            console.log(`Commits: ${payload.commits.length}`);
            // Trigger deployment, tests, etc.
            break;

        case 'pull_request':
            console.log(`PR #${payload.number}: ${payload.action}`);
            // Notify team, run checks, etc.
            break;

        case 'issues':
            console.log(`Issue #${payload.issue.number}: ${payload.action}`);
            // Add to project, notify, etc.
            break;
    }

    res.status(200).send('Webhook received');
});

app.listen(3000, () => console.log('Webhook server running'));
\end{lstlisting}

\section{GitHub Pages}

\subsection{Pubblicare Sito Statico}

\begin{lstlisting}
# Opzione 1: Branch gh-pages
git checkout -b gh-pages
# Aggiungi file HTML/CSS/JS
git add .
git commit -m "Initial site"
git push origin gh-pages

# Settings -> Pages -> Source: gh-pages branch

# Sito disponibile a:
# https://username.github.io/repository-name/
\end{lstlisting}

\subsection{Opzione 2: GitHub Actions per Deploy}

File \texttt{.github/workflows/deploy-pages.yml}:
\begin{lstlisting}[language=bash]
name: Deploy to GitHub Pages

on:
  push:
    branches: [ main ]

permissions:
  contents: read
  pages: write
  id-token: write

jobs:
  build:
    runs-on: ubuntu-latest
    steps:
      - uses: actions/checkout@v3

      - name: Setup Node
        uses: actions/setup-node@v3
        with:
          node-version: '18'

      - name: Install and build
        run: |
          npm ci
          npm run build

      - name: Upload artifact
        uses: actions/upload-pages-artifact@v1
        with:
          path: ./dist

  deploy:
    environment:
      name: github-pages
      url: ${{ steps.deployment.outputs.page_url }}
    runs-on: ubuntu-latest
    needs: build
    steps:
      - name: Deploy to GitHub Pages
        id: deployment
        uses: actions/deploy-pages@v1
\end{lstlisting}

\subsection{Custom Domain}

\begin{lstlisting}
# Aggiungi file CNAME nel repository
echo "www.yoursite.com" > CNAME
git add CNAME
git commit -m "Add custom domain"
git push

# Configura DNS:
# CNAME record: www -> username.github.io

# Settings -> Pages -> Custom domain
# Abilita HTTPS
\end{lstlisting}

\section{GitLab: Differenze Chiave}

\subsection{GitLab CI/CD}

File \texttt{.gitlab-ci.yml}:
\begin{lstlisting}[language=bash]
stages:
  - test
  - build
  - deploy

test:
  stage: test
  image: node:18
  script:
    - npm ci
    - npm test
  coverage: '/Coverage: \d+\.\d+%/'

build:
  stage: build
  script:
    - npm run build
  artifacts:
    paths:
      - dist/

deploy_production:
  stage: deploy
  script:
    - ./deploy.sh
  only:
    - main
  when: manual
\end{lstlisting}

\subsection{Merge Requests vs Pull Requests}

\begin{itemize}
    \item \textbf{Nome diverso}: GitLab usa "Merge Request", GitHub usa "Pull Request"
    \item \textbf{Funzionalità simili}: Code review, discussione, approval
    \item \textbf{GitLab extra}: Merge trains, merge when pipeline succeeds
\end{itemize}

\section{Esercizi}

\subsection{Esercizio 1: Setup Repository}

\begin{enumerate}
    \item Crea account GitHub (se non hai già)
    \item Crea nuovo repository pubblico "git-practice"
    \item Aggiungi README, .gitignore (Node), License (MIT)
    \item Clona localmente e aggiungi file
    \item Push modifiche
\end{enumerate}

\subsection{Esercizio 2: Issue Tracking}

\begin{enumerate}
    \item Crea 3 issues: 1 bug, 1 feature, 1 documentation
    \item Aggiungi labels appropriate
    \item Crea milestone "v1.0"
    \item Assegna issues al milestone
    \item Chiudi una issue tramite commit message
\end{enumerate}

\subsection{Esercizio 3: Pull Request Workflow}

\begin{enumerate}
    \item Crea branch \texttt{feature/add-calculator}
    \item Implementa semplice calcolatrice
    \item Push branch e apri PR
    \item Aggiungi descrizione dettagliata
    \item Simula review (commenta il tuo codice)
    \item Merge PR
\end{enumerate}

\subsection{Esercizio 4: GitHub Actions}

\begin{enumerate}
    \item Crea workflow che esegue test su push
    \item Aggiungi badge status al README
    \item Configura matrix testing per Node 16 e 18
    \item Aggiungi step per code coverage
    \item Verifica che workflow funzioni
\end{enumerate}

\subsection{Esercizio 5: GitHub Pages}

\begin{enumerate}
    \item Crea semplice sito HTML nel repository
    \item Configura GitHub Pages da branch main
    \item Verifica che sito sia online
    \item Modifica sito e verifica deploy automatico
    \item Bonus: Usa Jekyll o framework static site
\end{enumerate}

\begin{tcolorbox}[colback=green!10, colframe=green!60, title=Risorse]
\begin{itemize}
    \item GitHub Docs: \url{https://docs.github.com}
    \item GitHub Actions Marketplace: \url{https://github.com/marketplace?type=actions}
    \item GitLab Docs: \url{https://docs.gitlab.com}
    \item GitHub Learning Lab: \url{https://lab.github.com}
\end{itemize}
\end{tcolorbox}
