\chapter*{Prefazione}
\addcontentsline{toc}{chapter}{Prefazione}

\section*{A chi si rivolge questo manuale}

Questi appunti sono dedicati a studenti, sviluppatori e chiunque voglia comprendere a fondo Git e il controllo di versione. Il percorso è strutturato per accompagnare progressivamente dalle basi teoriche del version control alla gestione avanzata di repository collaborativi, workflow moderni e best practices professionali.

\section*{Perché Git è fondamentale}

Git è lo standard de facto per il controllo di versione nel mondo dello sviluppo software. Non si tratta solo di un tool tecnico, ma di una competenza professionale essenziale che caratterizza lo sviluppo moderno.

Innanzitutto, Git permette di tracciare ogni modifica al codice nel tempo, creando una cronologia completa e immutabile di tutti i cambiamenti apportati al progetto. Questa capacità di registrazione consente di comprendere come il codice si è evoluto e facilita enormemente la manutenzione.

Inoltre, Git facilita la collaborazione tra team distribuiti geograficamente, consentendo a decine o anche centinaia di sviluppatori di lavorare simultaneamente sullo stesso progetto senza interferi fra loro. Ogni sviluppatore lavora nel proprio ambiente locale e sincronizza il lavoro quando è pronto, garantendo un coordinamento efficiente.

Sul piano strategico, Git abilita workflow complessi di sviluppo come feature branches isolati, hotfix urgenti per problemi in produzione e gestione di release organizzate. Questi workflow strutturati permettono ai team di lavorare con disciplina e precisione, separando nettamente il codice stabile da quello in sviluppo.

Dal punto di vista della sicurezza e dell'affidabilità, Git fornisce protezione contro perdite di dati e regressioni indesiderate. La struttura distribuita garantisce che ogni clone del repository sia un backup completo, e il sistema di checksum SHA-1 assicura l'integrità dei dati.

Infine, Git è diventato uno standard richiesto praticamente da ogni azienda tech del mondo. La competenza nell'uso di Git è ormai tanto essenziale quanto sapere programmare, rendendolo un requisito quasi universale per qualunque posizione di sviluppo.

\section*{Struttura del corso}

Il corso è organizzato in 10 capitoli che coprono l'intero ecosistema Git:

\textbf{Parte I - Fondamenti} (Capitoli 1-2)
\begin{itemize}
    \item Storia e filosofia del version control
    \item Differenze tra VCS centralizzati e distribuiti
    \item Concetti fondamentali: repository, commit, staging area
    \item Primi comandi Git: init, add, commit, status, log
    \item File .gitignore e gestione file da escludere
\end{itemize}

\textbf{Parte II - Branching e Merging} (Capitolo 3)
\begin{itemize}
    \item Creazione e gestione branch
    \item Switching tra branch con checkout/switch
    \item Merge: fast-forward e three-way merge
    \item Rebase per linearizzare la storia
    \item Risoluzione conflitti
\end{itemize}

\textbf{Parte III - Repository Remoti} (Capitolo 4)
\begin{itemize}
    \item Configurazione remote
    \item Clone di repository esistenti
    \item Fetch, pull e push
    \item Tracking branches
    \item Best practices per sincronizzazione
\end{itemize}

\textbf{Parte IV - Workflow Collaborativi} (Capitolo 5)
\begin{itemize}
    \item Git Flow: workflow strutturato per release
    \item GitHub Flow: workflow semplificato per CI/CD
    \item Trunk-Based Development
    \item Fork e Pull Request
    \item Code review e collaborazione
\end{itemize}

\textbf{Parte V - Argomenti Avanzati} (Capitoli 6-10)
\begin{itemize}
    \item Git avanzato: stash, cherry-pick, bisect, reflog
    \item GitHub e GitLab: piattaforme collaborative
    \item Best practices professionali
    \item Troubleshooting e recovery
    \item Introduzione a CI/CD con Git
\end{itemize}

\section*{Prerequisiti}

Per affrontare questo corso con profitto è consigliabile avere già acquisito una certa dimestichezza con la riga di comando (terminal/bash/cmd), poiché tutti gli esempi sono basati su interfaccia testuale. Una conoscenza base di un linguaggio di programmazione è utile per comprendere gli esempi, anche se non è strettamente necessaria per imparare i concetti fondamentali di Git. È importante anche possedere una comprensione dei concetti di base del filesystem come file e directory. Infine, anche se opzionale, un'esperienza precedente con sviluppo software collaborativo aiuta a contextualizzare meglio il valore e l'importanza dei concetti presentati.

\section*{Strumenti necessari}

\textbf{Software indispensabile}:
\begin{itemize}
    \item \textbf{Git}: Sistema di controllo versione (2.30+)
    \item \textbf{Terminal/Bash}: Shell per eseguire comandi Git
    \item \textbf{Editor di testo}: VS Code, Vim, Nano, Sublime Text
\end{itemize}

\textbf{Piattaforme Git (consigliato)}:
\begin{itemize}
    \item \textbf{GitHub}: Piattaforma leader per hosting repository
    \item \textbf{GitLab}: Alternativa completa con CI/CD integrato
    \item \textbf{Bitbucket}: Soluzione Atlassian per team enterprise
\end{itemize}

\textbf{GUI Git (opzionale)}:
\begin{itemize}
    \item \textbf{GitKraken}: Client grafico potente e intuitivo
    \item \textbf{SourceTree}: GUI gratuita di Atlassian
    \item \textbf{GitHub Desktop}: Integrazione nativa GitHub
    \item \textbf{VS Code Git}: Integrazione Git in VS Code
\end{itemize}

\section*{Come installare Git}

\textbf{Linux (Ubuntu/Debian)}:
\begin{verbatim}
sudo apt update
sudo apt install git
\end{verbatim}

\textbf{macOS}:
\begin{verbatim}
brew install git
\end{verbatim}

\textbf{Windows}:
Scarica Git for Windows da \url{https://git-scm.com/download/win}

\textbf{Verifica installazione}:
\begin{verbatim}
git --version
# Output: git version 2.39.0 (o superiore)
\end{verbatim}

\section*{Come studiare}

Per ottenere il massimo da questi appunti:

\begin{enumerate}
    \item \textbf{Leggi attentamente la teoria}: Ogni concetto è spiegato con diagrammi e esempi
    \item \textbf{Pratica ogni comando}: Crea repository di prova e sperimenta
    \item \textbf{Disegna i diagrammi}: Visualizza branch, merge e workflow su carta
    \item \textbf{Commetti errori}: Sbagliare è parte dell'apprendimento (usa repository di test!)
    \item \textbf{Risolvi gli esercizi}: Prova autonomamente prima di vedere le soluzioni
    \item \textbf{Collabora}: Trova un partner e praticate workflow collaborativi
    \item \textbf{Leggi i log}: Analizza la storia dei commit per capire il flusso
\end{enumerate}

\begin{tcolorbox}[colback=orange!10, colframe=orange!60, title=Nota Importante]
Git ha una curva di apprendimento ripida inizialmente. \textbf{Non scoraggiarti!} La maggior parte degli sviluppatori usa quotidianamente solo 10-15 comandi. Questo manuale copre tutto lo spettro, ma inizierai con i comandi base e acquisirai padronanza gradualmente.
\end{tcolorbox}

\section*{Convenzioni tipografiche}

Nel testo vengono utilizzate le seguenti convenzioni:

\begin{itemize}
    \item \texttt{Comandi Git}: In carattere monospace (\texttt{git commit})
    \item \textbf{Concetti chiave}: In grassetto (repository, branch, commit)
    \item \textit{Parametri/opzioni}: In corsivo (-m, --amend)
    \item Box colorati: Best Practices, Attenzioni, Errori Comuni, Esempi
\end{itemize}

\textbf{Struttura comandi}:
\begin{verbatim}
$ git comando [opzioni] <argomenti>
# Commento esplicativo
\end{verbatim}

Il simbolo \texttt{\$} indica il prompt della shell (non va digitato).

\section*{Filosofia di Git}

Git è stato creato da Linus Torvalds nel 2005 per gestire lo sviluppo del kernel Linux. I principi fondamentali sono:

\begin{description}
    \item[\textbf{Distribuito}] Ogni sviluppatore ha una copia completa della storia
    \item[\textbf{Veloce}] Operazioni locali senza latenza di rete
    \item[\textbf{Integrità}] Checksum SHA-1 garantisce integrità dei dati
    \item[\textbf{Immutabilità}] La storia non viene riscritta (salvo operazioni esplicite)
    \item[\textbf{Branching leggero}] Creare branch è veloce e incoraggiato
\end{description}

\section*{Sito web e risorse}

Materiale aggiuntivo disponibile su:
\begin{itemize}
    \item Repository GitHub: \url{https://github.com/campionluca/Appunti}
    \item Documentazione ufficiale Git: \url{https://git-scm.com/doc}
    \item Pro Git Book (gratuito): \url{https://git-scm.com/book/it/v2}
    \item GitHub Learning Lab: \url{https://lab.github.com}
    \item Visualizing Git: \url{https://git-school.github.io/visualizing-git/}
\end{itemize}

\section*{Ringraziamenti}

Si ringrazia:
\begin{itemize}
    \item Linus Torvalds e la community Git per aver creato questo straordinario tool
    \item La community open source per innumerevoli tutorial e risorse
    \item L'Istituto Tecnico Antonio Scarpa per il supporto nella realizzazione
    \item Tutti gli studenti che con domande e feedback hanno migliorato questo materiale
\end{itemize}

\vspace{1cm}

\begin{flushright}
\textit{Prof. Luca Campion}\\
Novembre 2025
\end{flushright}

\section*{Note sulla versione}

\textbf{Versione 1.0} - Novembre 2025
\begin{itemize}
    \item Prima release completa
    \item 10 capitoli + 2 appendici
    \item Coverage: Git base, branching, remote, workflow, advanced
    \item Esempi pratici con Git 2.30+
    \item Diagrammi TikZ per visualizzare branch e merge
    \item Esercizi progressivi per ogni capitolo
    \item Best practices professionali aggiornate
\end{itemize}

\section*{Convenzioni usate negli esempi}

\textbf{Repository di esempio}: Useremo repository di prova chiamati:
\begin{itemize}
    \item \texttt{mio-progetto}: Repository locale base
    \item \texttt{website}: Progetto web di esempio
    \item \texttt{app}: Applicazione esempio per branching
\end{itemize}

\textbf{Utenti di esempio}:
\begin{itemize}
    \item Alice: Sviluppatrice senior
    \item Bob: Sviluppatore junior
    \item Charlie: DevOps engineer
\end{itemize}

\textbf{Branch comuni}:
\begin{itemize}
    \item \texttt{main/master}: Branch principale di produzione
    \item \texttt{develop}: Branch di sviluppo
    \item \texttt{feature/nome}: Branch per nuove funzionalità
    \item \texttt{hotfix/nome}: Branch per fix urgenti
\end{itemize}

\begin{tcolorbox}[colback=blue!10, colframe=blue!60, title=Prima di iniziare]
Configura Git con il tuo nome e email (richiesto per commit):

\begin{verbatim}
$ git config --global user.name "Tuo Nome"
$ git config --global user.email "tua.email@example.com"
\end{verbatim}

Verifica la configurazione:
\begin{verbatim}
$ git config --list
\end{verbatim}
\end{tcolorbox}

\section*{Licenza e utilizzo}

Questo materiale è rilasciato con licenza Creative Commons BY-SA 4.0. Sei libero di:
\begin{itemize}
    \item Condividere e distribuire il materiale
    \item Adattare e modificare per scopi didattici
    \item Utilizzare anche per scopi commerciali
\end{itemize}

A patto di:
\begin{itemize}
    \item Attribuire la paternità originale
    \item Condividere derivati con la stessa licenza
    \item Indicare eventuali modifiche apportate
\end{itemize}

Buono studio e buon Git!
