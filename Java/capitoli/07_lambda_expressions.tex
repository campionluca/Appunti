% Capitolo 7 - Lambda Expressions (cenni)
\chapter{Lambda Expressions}
\label{cap:lambda_expressions}
\label{cap:lambda}

\section{Obiettivi di apprendimento}
Al termine di questo capitolo sarai in grado di:
\begin{itemize}
    \item Comprendere il concetto di lambda expression
    \item Riconoscere la sintassi delle lambda in Java
    \item Utilizzare lambda per semplificare il codice
    \item Applicare lambda con Comparator, ActionListener e Predicate
    \item Confrontare lambda con classi anonime
\end{itemize}

\section{Cos'è una lambda expression}

Una \textbf{lambda expression} (o semplicemente lambda) è una funzione anonima che può essere passata come argomento o assegnata a una variabile. Le lambda sono state introdotte in Java 8 per supportare elementi di \textbf{programmazione funzionale}.

\subsection{Cenni sulla programmazione funzionale}
La programmazione funzionale tratta le funzioni come cittadini di prima classe (first-class citizens), attribuendo loro lo stesso status dei dati. Questo significa che è possibile passare funzioni come parametri ad altre funzioni, permettendo di comporre comportamenti complessi da blocchi semplici. Inoltre, è possibile restituire funzioni come valori di ritorno da altre funzioni, creando funzioni che generano e personalizzano altri comportamenti. Infine, la programmazione funzionale consente di assegnare funzioni a variabili e strutture dati, trattandole esattamente come si farebbe con i dati ordinari, aprendo possibilità di composizione e trasformazione molto potenti.

Le lambda in Java permettono di scrivere codice più conciso ed espressivo, riducendo il boilerplate necessario per le classi anonime.

\subsection{Interfacce funzionali}
Le lambda in Java funzionano con le \textbf{interfacce funzionali} (per un approfondimento sul concetto di interfaccia, vedi \autoref{cap:interfacce_classi_astratte}), ovvero interfacce con un solo metodo astratto. Esempi comuni:
\begin{itemize}
    \item \texttt{Runnable} - con il metodo \texttt{oggetto.run()}
    \item \texttt{Comparator<T>} - con il metodo \texttt{comparator.compare(T o1, T o2)}
    \item \texttt{ActionListener} - con il metodo \texttt{listener.actionPerformed(ActionEvent e)}
    \item \texttt{Predicate<T>} - con il metodo \texttt{predicate.test(T elemento)}
\end{itemize}

\section{Sintassi base}

La sintassi generale di una lambda è:
\begin{verbatim}
(parametri) -> espressione
\end{verbatim}

oppure, per blocchi di codice:
\begin{verbatim}
(parametri) -> { istruzioni; }
\end{verbatim}

\subsection{Forma 1: Sintassi completa}

\begin{lstlisting}
// Lambda con tipo esplicito dei parametri
Comparator<Integer> comp1 = (Integer a, Integer b) -> {
    return a.compareTo(b);
};
\end{lstlisting}

Questa forma specifica esplicitamente:
\begin{itemize}
    \item I tipi dei parametri (\texttt{Integer a} e \texttt{Integer b})
    \item Le parentesi graffe per il blocco di codice
    \item L'istruzione \texttt{return} per restituire il valore dell'espressione \texttt{a.compareTo(b)}
\end{itemize}

\subsection{Forma 2: Sintassi abbreviata}

\begin{lstlisting}
// Lambda senza tipo esplicito (inferito dal compilatore)
Comparator<Integer> comp2 = (a, b) -> {
    return a.compareTo(b);
};
\end{lstlisting}

Il compilatore inferisce automaticamente i tipi dei parametri dal contesto.

\subsection{Forma 3: Sintassi single-statement}

\begin{lstlisting}
// Lambda con espressione singola (senza return esplicito)
Comparator<Integer> comp3 = (a, b) -> a.compareTo(b);
\end{lstlisting}

Per espressioni singole:
\begin{itemize}
    \item Si omettono le parentesi graffe
    \item Si omette la parola chiave \texttt{return}
    \item Il valore dell'espressione è restituito implicitamente
\end{itemize}

\begin{nota}
Quando la lambda ha un solo parametro, si possono omettere anche le parentesi:
\begin{lstlisting}
// Un parametro: parentesi opzionali
Predicate<String> isEmpty = s -> s.isEmpty();

// Zero o piu parametri: parentesi obbligatorie
Runnable r = () -> System.out.println("Hello");
Comparator<Integer> c = (a, b) -> a - b;
\end{lstlisting}

Nel primo esempio, \texttt{s.isEmpty()} invoca il metodo dell'oggetto \texttt{String} rappresentato dal parametro \texttt{s}. Nel secondo esempio, il metodo \texttt{System.out.println("Hello")} viene richiamato sull'oggetto \texttt{PrintStream} di sistema. Nel terzo esempio, l'espressione \texttt{a - b} restituisce la differenza tra i due parametri interi.
\end{nota}

\section{Quando usare le lambda}

Le lambda sono particolarmente utili in diversi contesti comuni in Java.

\subsection{Comparator per ordinamento}

Le lambda semplificano notevolmente la creazione di comparatori personalizzati (per informazioni sulle interfacce, vedi \autoref{cap:interfacce_classi_astratte}).

\begin{lstlisting}
import java.util.ArrayList;
import java.util.Collections;
import java.util.Comparator;
import java.util.List;

public class EsempioComparator {
    public static void main(String[] args) {
        List<String> nomi = new ArrayList<>();
        nomi.add("Mario");
        nomi.add("Anna");
        nomi.add("Luca");
        nomi.add("Giulia");

        // Ordinamento per lunghezza con lambda
        // Il metodo Collections.sort() accetta la lista nomi e un Comparator
        // La lambda invoca s1.length() e s2.length() per confrontare le stringhe
        Collections.sort(nomi, (s1, s2) -> s1.length() - s2.length());
        System.out.println("Per lunghezza: " + nomi);

        // Ordinamento alfabetico inverso con lambda
        // Il metodo s2.compareTo(s1) confronta le stringhe in ordine inverso
        Collections.sort(nomi, (s1, s2) -> s2.compareTo(s1));
        System.out.println("Alfabetico inverso: " + nomi);

        // Method reference (forma ancora piu compatta)
        // String::compareTo e' un riferimento al metodo compareTo della classe String
        Collections.sort(nomi, String::compareTo);
        System.out.println("Alfabetico: " + nomi);
    }
}
\end{lstlisting}

\subsection{ActionListener per GUI}

Le lambda rendono più leggibile la gestione degli eventi nelle interfacce grafiche (per un approfondimento sulle GUI e gli eventi, vedi \autoref{cap:interfacce_grafiche}).

\begin{lstlisting}
import javax.swing.*;
import java.awt.*;

public class EsempioActionListener extends JFrame {
    private JLabel etichetta;
    private int contatore = 0;

    public EsempioActionListener() {
        // Configura la finestra usando metodi dell'oggetto JFrame
        setTitle("Lambda con ActionListener");
        setSize(300, 150);
        setDefaultCloseOperation(EXIT_ON_CLOSE);
        setLayout(new BorderLayout());

        etichetta = new JLabel("Contatore: 0", SwingConstants.CENTER);
        JButton pulsante = new JButton("Clicca");

        // ActionListener con lambda (forma concisa)
        // Il metodo pulsante.addActionListener() accetta una lambda
        // che implementa il metodo actionPerformed(ActionEvent e)
        pulsante.addActionListener(e -> {
            contatore++;
            // Il metodo etichetta.setText() aggiorna il testo della label
            etichetta.setText("Contatore: " + contatore);
        });

        // Il metodo this.add() (ereditato da JFrame) aggiunge i componenti
        add(etichetta, BorderLayout.CENTER);
        add(pulsante, BorderLayout.SOUTH);

        setVisible(true);
    }

    public static void main(String[] args) {
        // SwingUtilities.invokeLater() esegue la lambda nel thread della GUI
        SwingUtilities.invokeLater(() -> new EsempioActionListener());
    }
}
\end{lstlisting}

\subsection{Predicate per filtri}

\texttt{Predicate<T>} è un'interfaccia funzionale che rappresenta un predicato (funzione booleana) su un oggetto.

\begin{lstlisting}
import java.util.ArrayList;
import java.util.List;
import java.util.function.Predicate;

public class EsempioPredicate {
    public static void main(String[] args) {
        List<Integer> numeri = new ArrayList<>();
        for (int i = 1; i <= 10; i++) {
            // Il metodo numeri.add() aggiunge elementi alla lista
            numeri.add(i);
        }

        // Filtra numeri pari usando una lambda
        System.out.println("Numeri pari:");
        filtra(numeri, n -> n % 2 == 0);

        // Filtra numeri maggiori di 5
        System.out.println("\nNumeri > 5:");
        filtra(numeri, n -> n > 5);

        // Filtra numeri dispari e minori di 7
        System.out.println("\nNumeri dispari < 7:");
        filtra(numeri, n -> n % 2 != 0 && n < 7);
    }

    // Metodo generico che accetta un Predicate come parametro
    public static <T> void filtra(List<T> lista, Predicate<T> condizione) {
        for (T elemento : lista) {
            // Il metodo condizione.test() verifica se l'elemento soddisfa il predicato
            if (condizione.test(elemento)) {
                System.out.println(elemento);
            }
        }
    }
}
\end{lstlisting}

\section{Confronto con classi anonime}

Prima dell'introduzione delle lambda, si utilizzavano le classi anonime per implementare interfacce funzionali. Le lambda offrono una sintassi molto più concisa.

\subsection{Confronto side-by-side: Comparator}

\begin{lstlisting}
import java.util.ArrayList;
import java.util.Collections;
import java.util.Comparator;
import java.util.List;

public class ConfrontoSintassi {
    public static void main(String[] args) {
        List<String> parole = new ArrayList<>();
        parole.add("Java");
        parole.add("Python");
        parole.add("C");
        parole.add("JavaScript");

        // === CLASSE ANONIMA (sintassi verbosa) ===
        // Il metodo Collections.sort() accetta un oggetto Comparator
        // Qui viene creata una classe anonima che implementa Comparator<String>
        Collections.sort(parole, new Comparator<String>() {
            @Override
            public int compare(String s1, String s2) {
                // I metodi s1.length() e s2.length() restituiscono le lunghezze
                return s1.length() - s2.length();
            }
        });
        System.out.println("Con classe anonima: " + parole);

        // === LAMBDA (sintassi concisa) ===
        // Stessa funzionalita' ma con lambda expression
        Collections.sort(parole, (s1, s2) -> s1.length() - s2.length());
        System.out.println("Con lambda: " + parole);

        // === METHOD REFERENCE (forma piu compatta) ===
        // Il metodo statico Comparator.comparingInt() crea un comparatore
        // String::length e' un method reference al metodo length() di String
        Collections.sort(parole, Comparator.comparingInt(String::length));
        System.out.println("Con method reference: " + parole);
    }
}
\end{lstlisting}

\subsection{Confronto side-by-side: ActionListener}

\begin{lstlisting}
import javax.swing.*;
import java.awt.event.ActionEvent;
import java.awt.event.ActionListener;

public class ConfrontoActionListener {
    public static void main(String[] args) {
        JButton btn1 = new JButton("Classe Anonima");
        JButton btn2 = new JButton("Lambda");

        // === CLASSE ANONIMA ===
        // Il metodo btn1.addActionListener() accetta un oggetto ActionListener
        // Qui viene creata una classe anonima che implementa ActionListener
        btn1.addActionListener(new ActionListener() {
            @Override
            public void actionPerformed(ActionEvent e) {
                System.out.println("Click con classe anonima");
            }
        });

        // === LAMBDA ===
        // Il metodo btn2.addActionListener() accetta una lambda che implementa
        // il metodo actionPerformed(ActionEvent e) dell'interfaccia ActionListener
        btn2.addActionListener(e ->
            System.out.println("Click con lambda")
        );
    }
}
\end{lstlisting}

\subsection{Vantaggi delle lambda}
\begin{itemize}
    \item \textbf{Concisione}: meno codice boilerplate
    \item \textbf{Leggibilità}: focus sull'azione, non sulla struttura
    \item \textbf{Manutenibilità}: codice più chiaro e diretto
\end{itemize}

\subsection{Quando usare classi anonime}
Le classi anonime sono ancora utili quando:
\begin{itemize}
    \item L'interfaccia ha più di un metodo astratto
    \item Serve mantenere stato interno complesso
    \item Si vogliono override multipli o aggiungere campi
\end{itemize}

\section{Esempi avanzati}

Questa sezione presenta esempi più complessi che mostrano il potere delle lambda in situazioni realistiche.

\subsection{Esempio 1: Comparator con oggetti personalizzati}

Quando si lavora con oggetti personalizzati, le lambda semplificano enormemente il codice di ordinamento.

\subsubsection{Versione con classe anonima (prima di Java 8)}

\begin{lstlisting}
import java.util.ArrayList;
import java.util.Collections;
import java.util.Comparator;
import java.util.List;

class Studente {
    private String nome;
    private String cognome;
    private double media;

    public Studente(String nome, String cognome, double media) {
        this.nome = nome;
        this.cognome = cognome;
        this.media = media;
    }

    public String getNome() { return nome; }
    public String getCognome() { return cognome; }
    public double getMedia() { return media; }

    @Override
    public String toString() {
        return cognome + " " + nome + " (media: " + media + ")";
    }
}

public class EsempioComparatorClasseAnonima {
    public static void main(String[] args) {
        List<Studente> classe = new ArrayList<>();
        classe.add(new Studente("Mario", "Rossi", 7.5));
        classe.add(new Studente("Anna", "Bianchi", 8.2));
        classe.add(new Studente("Luca", "Verdi", 6.8));
        classe.add(new Studente("Giulia", "Neri", 9.1));

        // === CLASSE ANONIMA: ordinamento per media (decrescente) ===
        // Il metodo Collections.sort() accetta un oggetto Comparator
        // creato con una classe anonima verbosa
        Collections.sort(classe, new Comparator<Studente>() {
            @Override
            public int compare(Studente s1, Studente s2) {
                // I metodi s1.getMedia() e s2.getMedia() restituiscono double
                // Double.compare() confronta i valori in modo sicuro
                // Ordine invertito (s2, s1) per avere decrescente
                return Double.compare(s2.getMedia(), s1.getMedia());
            }
        });

        System.out.println("Studenti ordinati per media (decrescente):");
        for (Studente s : classe) {
            System.out.println(s);
        }
    }
}
\end{lstlisting}

\subsubsection{Versione con lambda (da Java 8 in poi)}

\begin{lstlisting}
import java.util.ArrayList;
import java.util.Comparator;
import java.util.List;

// La classe Studente rimane identica

public class EsempioComparatorLambda {
    public static void main(String[] args) {
        List<Studente> classe = new ArrayList<>();
        classe.add(new Studente("Mario", "Rossi", 7.5));
        classe.add(new Studente("Anna", "Bianchi", 8.2));
        classe.add(new Studente("Luca", "Verdi", 6.8));
        classe.add(new Studente("Giulia", "Neri", 9.1));

        // === LAMBDA: stessa funzionalita', codice molto piu conciso ===
        // Il metodo List.sort() (da Java 8) accetta un Comparator
        // La lambda (s1, s2) -> ... implementa il metodo compare()
        classe.sort((s1, s2) -> Double.compare(s2.getMedia(), s1.getMedia()));

        System.out.println("Studenti ordinati per media (decrescente):");
        classe.forEach(s -> System.out.println(s));

        // === METHOD REFERENCE: forma ancora piu elegante ===
        // Comparator.comparingDouble() crea un comparatore
        // Studente::getMedia e' un method reference al getter
        // reversed() inverte l'ordine (da crescente a decrescente)
        classe.sort(Comparator.comparingDouble(Studente::getMedia).reversed());

        System.out.println("\nOrdinamento multiplo (cognome, poi nome):");
        // Comparator.comparing() permette ordinamenti concatenati
        classe.sort(Comparator.comparing(Studente::getCognome)
                              .thenComparing(Studente::getNome));

        classe.forEach(System.out::println);
    }
}
\end{lstlisting}

\subsubsection{Analisi delle differenze}

\begin{itemize}
    \item \textbf{Righe di codice}: La classe anonima richiede 7 righe solo per definire il comparatore, la lambda ne richiede solo 1
    \item \textbf{Leggibilità}: La lambda permette di vedere immediatamente la logica di confronto senza boilerplate
    \item \textbf{Flessibilità}: I method reference e i metodi di Comparator (come \texttt{reversed()}, \texttt{thenComparing()}) offrono potenti combinazioni
    \item \textbf{Manutenibilità}: Codice più compatto = meno errori e modifiche più rapide
\end{itemize}

\subsection{Esempio 2: Lambda con Stream API}

Le Stream API (introdotte in Java 8) sfruttano pesantemente le lambda per elaborare collezioni in modo dichiarativo e funzionale.

\begin{lstlisting}
import java.util.ArrayList;
import java.util.List;
import java.util.stream.Collectors;

class Prodotto {
    private String nome;
    private String categoria;
    private double prezzo;
    private int giacenza;

    public Prodotto(String nome, String categoria, double prezzo, int giacenza) {
        this.nome = nome;
        this.categoria = categoria;
        this.prezzo = prezzo;
        this.giacenza = giacenza;
    }

    public String getNome() { return nome; }
    public String getCategoria() { return categoria; }
    public double getPrezzo() { return prezzo; }
    public int getGiacenza() { return giacenza; }

    @Override
    public String toString() {
        return nome + " (" + categoria + ") - EUR " + prezzo + " [" + giacenza + " pz]";
    }
}

public class EsempioStreamAPI {
    public static void main(String[] args) {
        List<Prodotto> catalogo = new ArrayList<>();
        catalogo.add(new Prodotto("Laptop", "Elettronica", 899.99, 15));
        catalogo.add(new Prodotto("Mouse", "Elettronica", 25.50, 120));
        catalogo.add(new Prodotto("Tastiera", "Elettronica", 65.00, 80));
        catalogo.add(new Prodotto("Sedia", "Arredamento", 199.99, 30));
        catalogo.add(new Prodotto("Scrivania", "Arredamento", 349.00, 12));
        catalogo.add(new Prodotto("Lampada", "Arredamento", 45.90, 5));
        catalogo.add(new Prodotto("Monitor", "Elettronica", 279.00, 25));

        // === ESEMPIO 1: Filtrare e raccogliere ===
        // Trova tutti i prodotti di elettronica con prezzo > 50 EUR
        System.out.println("=== Elettronica oltre 50 EUR ===");
        List<Prodotto> elettronicaCostosa = catalogo.stream()
            // filter() accetta un Predicate (lambda che restituisce boolean)
            .filter(p -> p.getCategoria().equals("Elettronica"))
            .filter(p -> p.getPrezzo() > 50)
            // collect() raccoglie i risultati in una nuova lista
            .collect(Collectors.toList());

        elettronicaCostosa.forEach(p -> System.out.println(p));

        // === ESEMPIO 2: Trasformare con map() ===
        // Estrai solo i nomi dei prodotti in giacenza scarsa (< 20 pz)
        System.out.println("\n=== Prodotti con giacenza scarsa ===");
        List<String> prodottiScarsi = catalogo.stream()
            .filter(p -> p.getGiacenza() < 20)
            // map() trasforma ogni Prodotto in una String (il nome)
            .map(p -> p.getNome())
            .collect(Collectors.toList());

        System.out.println(prodottiScarsi);

        // === ESEMPIO 3: Pipeline complessa ===
        // Calcola il valore totale del magazzino per categoria "Arredamento"
        System.out.println("\n=== Valore magazzino Arredamento ===");
        double valoreArredamento = catalogo.stream()
            .filter(p -> p.getCategoria().equals("Arredamento"))
            // map() calcola il valore (prezzo * giacenza) per ogni prodotto
            .map(p -> p.getPrezzo() * p.getGiacenza())
            // mapToDouble() converte Stream<Double> in DoubleStream
            .mapToDouble(Double::doubleValue)
            // sum() somma tutti i valori
            .sum();

        System.out.println("Valore totale: EUR " + valoreArredamento);

        // === ESEMPIO 4: Ordinamento e limite ===
        // Top 3 prodotti piu costosi
        System.out.println("\n=== Top 3 prodotti piu costosi ===");
        catalogo.stream()
            // sorted() accetta un Comparator (lambda)
            .sorted((p1, p2) -> Double.compare(p2.getPrezzo(), p1.getPrezzo()))
            // limit() prende solo i primi 3 elementi
            .limit(3)
            .forEach(p -> System.out.println(p.getNome() + " - EUR " + p.getPrezzo()));

        // === ESEMPIO 5: Conteggio e statistiche ===
        System.out.println("\n=== Statistiche Elettronica ===");
        long numProdottiElettronica = catalogo.stream()
            .filter(p -> p.getCategoria().equals("Elettronica"))
            // count() restituisce il numero di elementi
            .count();

        double prezzoMedio = catalogo.stream()
            .filter(p -> p.getCategoria().equals("Elettronica"))
            .mapToDouble(Prodotto::getPrezzo)
            // average() restituisce un OptionalDouble
            .average()
            // orElse() fornisce un valore di default se lo stream e' vuoto
            .orElse(0.0);

        System.out.println("Numero prodotti: " + numProdottiElettronica);
        System.out.println("Prezzo medio: EUR " + String.format("%.2f", prezzoMedio));
    }
}
\end{lstlisting}

\subsubsection{Operazioni Stream principali con lambda}

\begin{itemize}
    \item \textbf{filter(Predicate)}: Filtra elementi che soddisfano una condizione
    \begin{lstlisting}
.filter(p -> p.getPrezzo() > 100)
    \end{lstlisting}

    \item \textbf{map(Function)}: Trasforma ogni elemento in un altro tipo
    \begin{lstlisting}
.map(p -> p.getNome())  // Prodotto -> String
    \end{lstlisting}

    \item \textbf{sorted(Comparator)}: Ordina gli elementi
    \begin{lstlisting}
.sorted((a, b) -> Double.compare(b.getPrezzo(), a.getPrezzo()))
    \end{lstlisting}

    \item \textbf{forEach(Consumer)}: Esegue un'azione su ogni elemento
    \begin{lstlisting}
.forEach(p -> System.out.println(p))
    \end{lstlisting}

    \item \textbf{collect(Collector)}: Raccoglie i risultati in una collezione
    \begin{lstlisting}
.collect(Collectors.toList())
    \end{lstlisting}
\end{itemize}

\subsubsection{Vantaggi degli Stream con lambda}

\begin{itemize}
    \item \textbf{Codice dichiarativo}: Si specifica \emph{cosa} fare, non \emph{come} farlo
    \item \textbf{Pipeline leggibili}: Operazioni concatenate che si leggono come frasi
    \item \textbf{Separazione delle responsabilità}: Ogni operazione ha un compito preciso
    \item \textbf{Potenziale parallelizzazione}: Usando \texttt{.parallelStream()}, il codice può eseguire in parallelo senza modifiche
\end{itemize}

\begin{nota}
Gli Stream sono una tecnica avanzata. Quando elabori collezioni:
\begin{itemize}
    \item Per operazioni semplici, i cicli tradizionali possono essere più chiari
    \item Per trasformazioni complesse o multiple, gli Stream migliorano la leggibilità
    \item Gli Stream non modificano la collezione originale (sono immutabili)
    \item Ogni Stream può essere usato una sola volta
\end{itemize}
\end{nota}

\section{Riepilogo}

In questo capitolo abbiamo introdotto le lambda expressions in Java:

\begin{itemize}
    \item \textbf{Lambda}: funzioni anonime per programmazione funzionale
    \item \textbf{Sintassi}: tre forme (completa, abbreviata, single-statement)
    \item \textbf{Interfacce funzionali}: interfacce con un solo metodo astratto (approfondite in \autoref{cap:interfacce_classi_astratte})
    \item \textbf{Casi d'uso comuni}: utilizzo con \texttt{Comparator} per ordinamento di collezioni (vedi \autoref{cap:arraylist}), \texttt{ActionListener} per gestione eventi GUI (vedi \autoref{cap:interfacce_grafiche}), e \texttt{Predicate} per filtri
    \item \textbf{Vantaggi}: codice più conciso e leggibile rispetto a classi anonime
\end{itemize}

Le lambda sono uno strumento potente per scrivere codice più espressivo e manutenibile, e sono ampiamente utilizzate nelle API moderne di Java.

\begin{nota}
Concetti avanzati correlati alle lambda (non trattati in questo capitolo):
\begin{itemize}
    \item \textbf{Stream API}: elaborazione dichiarativa di collezioni con lambda
    \item \textbf{Method reference}: forma ancora più compatta (\texttt{String::length})
    \item \textbf{Closure}: cattura di variabili dal contesto esterno
    \item \textbf{Interfacce funzionali comuni}: Function, Consumer, Supplier, BiFunction
\end{itemize}

Questi argomenti possono essere approfonditi in corsi avanzati di Java.
\end{nota}
