\chapter*{Prefazione}
\addcontentsline{toc}{chapter}{Prefazione}

\section*{A chi è rivolto questo libro}

Questi appunti sono stati pensati per gli studenti del quarto anno di Istituto Tecnico che stanno approfondendo la programmazione in Java. Il materiale presuppone una conoscenza di base del linguaggio (variabili, cicli, metodi, concetti fondamentali di programmazione) e si propone di consolidare e ampliare tali competenze attraverso argomenti più avanzati e pratici.

L'approccio adottato bilancia teoria ed esempi concreti, con l'obiettivo di fornire strumenti immediatamente applicabili sia nei progetti scolastici che in contesti reali.

\section*{Struttura del libro}

Il libro è organizzato in otto capitoli, ciascuno focalizzato su un argomento specifico:

\begin{enumerate}
    \item \textbf{Classi, Oggetti e Ereditarietà}: ripasso e approfondimento dei concetti fondamentali della programmazione orientata agli oggetti, con particolare attenzione agli array di oggetti e alla gerarchia tra classi.

    \item \textbf{Stream e Buffer}: gestione di flussi di dati per leggere e scrivere file, con esempi pratici di utilizzo delle classi più comuni.

    \item \textbf{Interfacce e Classi Astratte}: meccanismi per definire comportamenti comuni e creare gerarchie flessibili.

    \item \textbf{Eccezioni}: gestione degli errori a runtime attraverso il sistema delle eccezioni di Java.

    \item \textbf{ArrayList}: struttura dati dinamica per gestire collezioni di elementi in modo più flessibile rispetto agli array tradizionali.

    \item \textbf{Interfacce Grafiche}: introduzione alla creazione di applicazioni con interfaccia grafica usando Swing, inclusa la gestione degli eventi.

    \item \textbf{Model View Controller}: pattern architetturale per organizzare il codice separando logica, presentazione e controllo.

    \item \textbf{Lambda Expressions}: cenni alle espressioni lambda introdotte in Java 8, per scrivere codice più conciso ed espressivo.
\end{enumerate}

\section*{Come usare questo libro}

Ogni capitolo è strutturato per guidare l'apprendimento in modo progressivo:

\begin{itemize}
    \item Gli \textbf{obiettivi di apprendimento} all'inizio di ogni capitolo chiariscono cosa ci si aspetta di saper fare al termine dello studio.

    \item La \textbf{teoria} è presentata in modo sintetico ma completo, con definizioni chiare e schemi quando necessario.

    \item Gli \textbf{esempi di codice} sono commentati in italiano e mostrano l'applicazione pratica dei concetti. Si consiglia di digitare personalmente ogni esempio, eseguirlo e sperimentare modifiche per comprenderne il funzionamento.

    \item I \textbf{box colorati} evidenziano informazioni particolari:
    \begin{itemize}
        \item \textcolor{orange}{Arancione (Attenzione)}: punti critici da ricordare
        \item \textcolor{blue}{Blu (Nota)}: suggerimenti e best practices
        \item \textcolor{red}{Rosso (Errore Comune)}: errori frequenti da evitare
    \end{itemize}

    \item Gli \textbf{esercizi} sono suddivisi in tre livelli di difficoltà (base, intermedio, avanzato). Si consiglia di affrontarli in ordine, verificando le soluzioni commentate nell'appendice solo dopo aver tentato autonomamente.

    \item Il \textbf{riepilogo} alla fine di ogni capitolo sintetizza i concetti chiave e facilita il ripasso.
\end{itemize}

\section*{Prerequisiti}

Per affrontare efficacemente questi appunti, è necessario:

\begin{itemize}
    \item Conoscere la sintassi base di Java (tipi di dato primitivi, operatori, strutture di controllo)
    \item Saper dichiarare e utilizzare metodi
    \item Comprendere i concetti basilari di classe e oggetto
    \item Avere familiarità con array monodimensionali
    \item Disporre di un ambiente di sviluppo Java funzionante (JDK 8 o superiore, IDE come Eclipse, IntelliJ IDEA o NetBeans)
\end{itemize}

\section*{Convenzioni utilizzate}

\textbf{Codice}: tutti gli esempi di codice sono presentati con sintassi evidenziata, numerazione delle righe e commenti esplicativi.

\textbf{Nomenclatura}: si segue la convenzione Java standard (CamelCase per classi, camelCase per metodi e variabili, MAIUSCOLO per costanti).

\textbf{Terminologia}: si preferisce l'italiano quando possibile, mantenendo i termini tecnici in inglese quando consolidati nella pratica professionale (ad esempio "stream", "buffer", "exception").

\vspace{1cm}

Buono studio!
