\chapter*{Bibliografia e Sitografia}
\addcontentsline{toc}{chapter}{Bibliografia}

\section*{Libri Consigliati}

\begin{itemize}
    \item \textbf{Oracle Java Documentation} - La documentazione ufficiale di Java, disponibile online, è la risorsa più completa e aggiornata per approfondire ogni aspetto del linguaggio.

    \item \textbf{Effective Java} di Joshua Bloch - Un classico per chi vuole scrivere codice Java di qualità, con best practices e pattern consolidati.

    \item \textbf{Head First Java} di Kathy Sierra e Bert Bates - Un'introduzione visuale e coinvolgente ai concetti di Java, particolarmente adatta per chi inizia.

    \item \textbf{Java: The Complete Reference} di Herbert Schildt - Riferimento enciclopedico per tutti gli argomenti Java, dalla sintassi di base alle API avanzate.
\end{itemize}

\section*{Risorse Online}

\begin{itemize}
    \item \textbf{Oracle Java Tutorials} \\
    \url{https://docs.oracle.com/javase/tutorial/} \\
    Tutorial ufficiali organizzati per argomento, con esempi scaricabili.

    \item \textbf{Java API Documentation} \\
    \url{https://docs.oracle.com/en/java/javase/} \\
    Documentazione completa di tutte le classi e metodi della libreria standard Java.

    \item \textbf{Stack Overflow} \\
    \url{https://stackoverflow.com/questions/tagged/java} \\
    Community di sviluppatori dove trovare risposte a problemi specifici.

    \item \textbf{Baeldung} \\
    \url{https://www.baeldung.com/} \\
    Tutorial pratici e aggiornati su Java e tecnologie correlate.

    \item \textbf{Java Code Geeks} \\
    \url{https://www.javacodegeeks.com/} \\
    Articoli, tutorial e best practices per sviluppatori Java.

    \item \textbf{GitHub} \\
    \url{https://github.com/} \\
    Piattaforma per esplorare progetti open source Java e imparare da codice reale.
\end{itemize}

\section*{Strumenti di Sviluppo}

\begin{itemize}
    \item \textbf{IntelliJ IDEA} - IDE potente con versione gratuita (Community Edition)
    \item \textbf{Eclipse} - IDE open source molto diffuso nell'ambiente educativo
    \item \textbf{Visual Studio Code} - Editor leggero con estensioni per Java
    \item \textbf{NetBeans} - IDE ufficiale Oracle, particolarmente adatto per principianti
\end{itemize}
