% Capitolo 8 - Esercizi
\chapter{Esercizi}
\label{cap:esercizi}

Questo capitolo raccoglie tutti gli esercizi organizzati per argomento e livello di difficoltà. Ogni sezione corrisponde a un capitolo del libro e permette di applicare praticamente i concetti teorici esposti nei capitoli precedenti. Gli esercizi sono suddivisi in:

\begin{itemize}
    \item \textbf{Livello Base}: Applicazione diretta dei concetti fondamentali
    \item \textbf{Livello Intermedio}: Combinazione di più concetti e ragionamento
    \item \textbf{Livello Avanzato}: Problemi complessi che richiedono progettazione
\end{itemize}

Le soluzioni complete e commentate di tutti gli esercizi sono disponibili nell'Appendice.

\section{Esercizi su Classi, Oggetti ed Ereditarietà}

Gli esercizi di questa sezione consolidano i concetti presentati in \autoref{cap:classi_oggetti_ereditarieta}.

\subsection{Esercizio 1.1 - Livello Base}
\textbf{Classe Libro}

Crea una classe \texttt{Libro} con i seguenti attributi privati:
\begin{itemize}
    \item \texttt{titolo} (String)
    \item \texttt{autore} (String)
    \item \texttt{numeroPagine} (int)
    \item \texttt{prezzoCopertina} (double)
\end{itemize}

Implementa:
\begin{itemize}
    \item Costruttore con tutti i parametri
    \item Metodi getter e setter per tutti gli attributi
    \item Metodo \texttt{libro.applicaSconto(double percentuale)} che riduce il prezzo del libro in base alla percentuale fornita
    \item Metodo \texttt{libro.toString()} che restituisce una descrizione completa del libro
\end{itemize}

Crea una classe di test che crea 3 libri, applica uno sconto del 15\% e stampa le informazioni.

\subsection{Esercizio 1.2 - Livello Base}
\textbf{Array di Prodotti}

Crea una classe \texttt{Prodotto} con attributi privati: codice, nome, quantita, prezzo e relativi metodi getter/setter. Scrivi un programma che:
\begin{itemize}
    \item Crea un array di 5 oggetti \texttt{Prodotto}
    \item Calcola il valore totale del magazzino usando \texttt{prodotto.getQuantita() * prodotto.getPrezzo()} per ogni prodotto
    \item Trova e stampa il prodotto con la quantità minima confrontando \texttt{prodotto.getQuantita()} di ciascun prodotto
    \item Stampa tutti i prodotti con quantità inferiore a 10 (soglia di riordino) verificando \texttt{prodotto.getQuantita() < 10}
\end{itemize}

\subsection{Esercizio 1.3 - Livello Intermedio}
\textbf{Gerarchia Dipendenti}

Crea una classe base \texttt{Dipendente} con attributi: nome, cognome, stipendioBase. Crea due sottoclassi che applicano il concetto di ereditarietà (vedi \autoref{cap:classi_oggetti_ereditarieta}):
\begin{itemize}
    \item \texttt{DipendenteFullTime}: aggiunge l'attributo bonusAnnuale e il metodo \texttt{dipendenteFullTime.calcolaStipendioAnnuale()} che calcola stipendioBase * 12 + bonusAnnuale
    \item \texttt{DipendentePartTime}: aggiunge l'attributo oreLavorate e il metodo \texttt{dipendentePartTime.calcolaStipendioMensile()} che calcola stipendioBase * oreLavorate
\end{itemize}

Implementa l'override del metodo \texttt{dipendente.mostraDettagli()} in entrambe le sottoclassi per visualizzare le informazioni specifiche. Crea un array polimorfo di dipendenti e calcola il costo totale del personale.

\subsection{Esercizio 1.4 - Livello Intermedio}
\textbf{Sistema Biblioteca}

Crea una classe \texttt{ElementoBiblioteca} con attributi comuni (titolo, anno, codice). Crea sottoclassi:
\begin{itemize}
    \item \texttt{Libro}: aggiunge attributi autore e numeroPagine
    \item \texttt{Rivista}: aggiunge attributi numeroEdizione e mese
    \item \texttt{DVD}: aggiunge attributi durata e regista
\end{itemize}

Implementa un metodo \texttt{elemento.calcolaPenaleRitardo(int giorniRitardo)} con logiche diverse per ogni tipo (ad esempio: i libri hanno una penale base, le riviste hanno una penale ridotta, i DVD hanno una penale maggiorata). Crea un array polimorfo e calcola le penali totali per diversi scenari di ritardo.

\subsection{Esercizio 1.5 - Livello Avanzato}
\textbf{Sistema Gestione Forme Geometriche}

Crea una gerarchia di classi per forme geometriche (approfondisci il concetto di classi astratte in \autoref{cap:interfacce_classi_astratte}):
\begin{itemize}
    \item Classe base astratta \texttt{Forma} con metodi astratti \texttt{forma.calcolaArea()} e \texttt{forma.calcolaPerimetro()}
    \item Sottoclassi concrete: \texttt{Cerchio}, \texttt{Rettangolo}, \texttt{Triangolo}, \texttt{Quadrato} (che estende Rettangolo)
\end{itemize}

Implementa:
\begin{itemize}
    \item Un array polimorfo di forme
    \item Metodo statico \texttt{Forma.trovaFormaAreaMassima(Forma[] forme)} per trovare la forma con area massima
    \item Metodo statico \texttt{Forma.ordinaPerPerimetro(Forma[] forme)} per ordinare le forme per perimetro crescente
    \item Metodo statico \texttt{Forma.calcolaAreaTotale(Forma[] forme)} per calcolare l'area totale di tutte le forme
    \item Metodo statico \texttt{Forma.contaFormeSopraSoglia(Forma[] forme, double soglia)} per contare quante forme hanno area superiore a una soglia
\end{itemize}

\begin{errore}
\textbf{Errore comune}: Dimenticare di inizializzare gli oggetti in un array.

\begin{lstlisting}
Studente[] classe = new Studente[3];
String nome = classe[0].getNome(); // NullPointerException!
// Errore: classe[0] e' null, non possiamo chiamare getNome()
\end{lstlisting}

Dopo aver creato l'array, ogni elemento deve essere istanziato prima di poter invocare metodi su di esso:
\begin{lstlisting}
classe[0] = new Studente("001", "Mario", 7.5);
String nome = classe[0].getNome(); // Ora funziona!
\end{lstlisting}
\end{errore}

\section{Esercizi su Stream e Buffer}

Gli esercizi di questa sezione approfondiscono i concetti di gestione file e I/O presentati in \autoref{cap:stream_buffer}.

\subsection{Esercizi Base}

\textbf{Esercizio 2.1:} Scrivi un programma che legge un file di testo e conta il numero di righe, stampando il risultato.

\textbf{Esercizio 2.2:} Crea un programma che copia il contenuto di un file carattere per carattere in un altro file, utilizzando FileReader e FileWriter.

\textbf{Esercizio 2.3:} Scrivi un programma che chiede all'utente di inserire 5 nomi (da console) e li salva in un file di testo, uno per riga.

\subsection{Esercizi Intermedi}

\textbf{Esercizio 2.4:} Scrivi un programma che legge un file di testo e crea un nuovo file con le righe in ordine inverso (l'ultima riga diventa la prima).

\textbf{Esercizio 2.5:} Crea un programma che legge un file di testo e copia in un nuovo file solo le righe che iniziano con una lettera maiuscola.

\textbf{Esercizio 2.6:} Implementa un programma che unisce (merge) due file di testo in un terzo file, alternando le righe dei due file sorgente.

\subsection{Esercizi Avanzati}

\textbf{Esercizio 2.7:} Scrivi un programma che analizza un file di testo e produce un report con:
\begin{itemize}
    \item Numero totale di parole
    \item Numero totale di caratteri (esclusi spazi)
    \item Frequenza di ogni parola (quante volte appare)
\end{itemize}

\textbf{Esercizio 2.8:} Crea un semplice parser CSV che legge un file con dati separati da virgola (es: nome,cognome,età) e li stampa in formato tabellare allineato.

\section{Esercizi su Interfacce e Classi Astratte}

Gli esercizi di questa sezione permettono di praticare i concetti di astrazione, interfacce e polimorfismo presentati in \autoref{cap:interfacce_classi_astratte}.

\subsection{Esercizi Base}

\textbf{Esercizio 3.1:} Crea un'interfaccia \texttt{Volante} con metodo astratto \texttt{volante.vola()} che deve essere implementato dalle classi. Implementa l'interfaccia in due classi: \texttt{Aereo} e \texttt{Uccello}, dove ciascuna classe fornisce la propria implementazione del metodo \texttt{oggetto.vola()}.

\textbf{Esercizio 3.2:} Crea una classe astratta \texttt{Animale} con metodo astratto \texttt{animale.verso()} e metodo concreto \texttt{animale.dorme()}. Implementa due classi concrete: \texttt{Cane} (che implementa \texttt{cane.verso()} restituendo "Bau!") e \texttt{Gatto} (che implementa \texttt{gatto.verso()} restituendo "Miao!").

\textbf{Esercizio 3.3:} Crea un'interfaccia \texttt{Pagabile} con metodo \texttt{lavoratore.calcolaStipendio()}. Implementala in due classi \texttt{Dipendente} (con \texttt{dipendente.calcolaStipendio()} basato su stipendio fisso mensile) e \texttt{Freelance} (con \texttt{freelance.calcolaStipendio()} basato su ore lavorate e tariffa oraria) con logiche di calcolo diverse.

\subsection{Esercizi Intermedi}

\textbf{Esercizio 3.4:} Implementa un'interfaccia \texttt{DispositivoElettronico} con metodi \texttt{dispositivo.accendi()}, \texttt{dispositivo.spegni()} e \texttt{dispositivo.consumoEnergetico()}. Crea 3 classi che la implementano (ad esempio: \texttt{Televisore}, \texttt{Frigorifero}, \texttt{Lavatrice}) e un metodo statico \texttt{DispositivoElettronico.calcolaConsumoTotale(DispositivoElettronico[] dispositivi)} che calcola il consumo totale di un array di dispositivi.

\textbf{Esercizio 3.5:} Crea una classe astratta \texttt{FiguraPiana} con metodi astratti \texttt{figura.calcolaArea()} e \texttt{figura.calcolaPerimetro()}, e un metodo concreto \texttt{figura.confrontaArea(FiguraPiana altra)} che confronta l'area della figura corrente con quella di un'altra figura. Implementa almeno 3 figure concrete (ad esempio: \texttt{Cerchio}, \texttt{Rettangolo}, \texttt{Triangolo}).

\textbf{Esercizio 3.6:} Implementa l'interfaccia \texttt{Comparable<Libro>} per una classe \texttt{Libro}, specificando nel metodo \texttt{libro.compareTo(Libro altro)} il confronto alfabetico per titolo. Crea un array di libri e ordinalo usando \texttt{Arrays.sort(libri)}.

\subsection{Esercizi Avanzati}

\textbf{Esercizio 3.7:} Progetta un sistema per gestire diversi tipi di account bancari (Corrente, Risparmio, Deposito). Usa una classe astratta \texttt{Account} con metodi astratti \texttt{account.calcolaInteressi()} e \texttt{account.applicaCommissioni()}, e metodi concreti \texttt{account.deposita(double importo)} e \texttt{account.preleva(double importo)} per deposito/prelievo. Ogni tipo di account ha regole diverse per calcolo interessi e applicazione commissioni.

\textbf{Esercizio 3.8:} Crea un'interfaccia \texttt{Ordinabile} con metodi \texttt{ordinabile.confronta(Ordinabile altro)} (restituisce int negativo, zero o positivo) e \texttt{ordinabile.getChiave()} (restituisce la chiave di ordinamento). Implementala in classi \texttt{Prodotto}, \texttt{Cliente}, \texttt{Ordine}. Scrivi un metodo statico generico \texttt{Ordinabile.ordina(Ordinabile[] elementi)} che ordina un array di qualsiasi tipo che implementa \texttt{Ordinabile}.

\section{Esercizi su Eccezioni}

Gli esercizi di questa sezione consolidano la gestione degli errori e delle eccezioni in Java, argomento trattato in \autoref{cap:eccezioni}.

\subsection{Esercizi Base}

\textbf{Esercizio 4.1:} Scrivi un programma che chiede all'utente di inserire un numero da console e gestisce l'eccezione se l'input non è un numero valido.

\textbf{Esercizio 4.2:} Crea un metodo che calcola la divisione tra due numeri e gestisce l'eccezione di divisione per zero, restituendo 0 in quel caso.

\textbf{Esercizio 4.3:} Scrivi un programma che legge un array e stampa l'elemento all'indice richiesto dall'utente, gestendo l'eccezione di indice fuori range.

\subsection{Esercizi Intermedi}

\textbf{Esercizio 4.4:} Implementa un metodo \texttt{Matematica.calcolaRadiceQuadrata(double n)} che lancia un'eccezione checked personalizzata \texttt{NumeroNegativoException} se il parametro n è negativo. Il metodo deve restituire la radice quadrata se n è non negativo.

\textbf{Esercizio 4.5:} Crea una classe \texttt{Calcolatrice} con metodi \texttt{calcolatrice.somma(double a, double b)}, \texttt{calcolatrice.sottrazione(double a, double b)}, \texttt{calcolatrice.moltiplicazione(double a, double b)} e \texttt{calcolatrice.divisione(double a, double b)}. Il metodo di divisione deve lanciare un'eccezione \texttt{ArithmeticException} quando il divisore è zero.

\textbf{Esercizio 4.6:} Scrivi un programma che legge un file riga per riga e conta le righe. Gestisci sia l'eccezione di file non trovato che eventuali errori di lettura, usando blocchi catch separati.

\subsection{Esercizi Avanzati}

\textbf{Esercizio 4.7:} Implementa una gerarchia di eccezioni personalizzate per un sistema di prenotazioni:
\begin{itemize}
    \item \texttt{PrenotazioneException} (base)
    \item \texttt{PostiEsauritiException}
    \item \texttt{DataNonValidaException}
    \item \texttt{PagamentoFallitoException}
\end{itemize}
Crea una classe \texttt{SistemaPrenotazioni} che usa queste eccezioni.

\textbf{Esercizio 4.8:} Scrivi un parser JSON semplificato che legge una stringa e verifica se contiene una struttura valida (coppie chiave:valore tra graffe). Crea eccezioni personalizzate per diversi tipi di errori: \texttt{JsonSyntaxException}, \texttt{JsonKeyException}, \texttt{JsonValueException}.

\section{Esercizi su ArrayList}

Gli esercizi di questa sezione permettono di praticare l'uso delle collezioni dinamiche in Java, argomento presentato in \autoref{cap:arraylist}.

\subsection{Esercizi base}

\begin{enumerate}
    \item \textbf{Lista della spesa}: Crea un programma che gestisce una lista della spesa usando un \texttt{ArrayList<String>}. L'utente deve poter invocare metodi come \texttt{listaSpesa.add(String prodotto)} per aggiungere prodotti, visualizzare l'intera lista con un ciclo, rimuovere prodotti con \texttt{listaSpesa.remove(String prodotto)} e verificare se un prodotto è già presente con \texttt{listaSpesa.contains(String prodotto)}.

    \item \textbf{Temperature settimanali}: Scrivi un programma che memorizza le temperature massime di una settimana in un \texttt{ArrayList<Double>}. Implementa metodi per calcolare e visualizzare la temperatura media iterando sulla lista, trovare la temperatura massima usando \texttt{Collections.max(temperature)} e quella minima usando \texttt{Collections.min(temperature)}.

    \item \textbf{Rimozione duplicati}: Dato un \texttt{ArrayList<String>} contenente elementi duplicati, scrivi un metodo \texttt{rimuoviDuplicati(ArrayList<String> lista)} che restituisce un nuovo \texttt{ArrayList<String>} contenente solo gli elementi unici, mantenendo l'ordine di prima apparizione usando \texttt{listaUnica.contains(elemento)} per verificare duplicati.
\end{enumerate}

\subsection{Esercizi intermedi}

\begin{enumerate}
    \item \textbf{Gestione biblioteca}: Crea una classe \texttt{Libro} con attributi titolo, autore, anno e ISBN. Implementa una classe \texttt{Biblioteca} con un \texttt{ArrayList<Libro>} e metodi come \texttt{biblioteca.aggiungiLibro(Libro libro)}, \texttt{biblioteca.cercaPerAutore(String autore)} che restituisce un \texttt{ArrayList<Libro>}, \texttt{biblioteca.ordinaPerAnno()} che usa \texttt{libri.sort()} con un comparatore, e \texttt{biblioteca.rimuoviPerISBN(String isbn)} che itera sulla lista per trovare e rimuovere il libro.

    \item \textbf{Registro voti}: Realizza un sistema di gestione voti per una classe. Crea una classe \texttt{Studente} con attributi nome, cognome e un \texttt{ArrayList<Double> voti}. Implementa metodi come \texttt{studente.aggiungiVoto(double voto)} usando \texttt{voti.add()}, \texttt{studente.calcolaMedia()} iterando sui voti, un metodo statico \texttt{Studente.trovaMediaPiuAlta(ArrayList<Studente> studenti)} e un metodo \texttt{Studente.filtraPerMedia(ArrayList<Studente> studenti, double soglia)} che restituisce studenti con media superiore alla soglia.

    \item \textbf{Gestione playlist musicale}: Crea una classe \texttt{Canzone} con attributi titolo, artista e durata in secondi. Implementa una classe \texttt{Playlist} con un \texttt{ArrayList<Canzone>} e metodi come \texttt{playlist.aggiungiCanzone(Canzone canzone)}, \texttt{playlist.rimuoviCanzone(Canzone canzone)}, \texttt{playlist.calcolaDurataTotale()} sommando le durate, \texttt{playlist.cercaPerArtista(String artista)} che filtra le canzoni e \texttt{playlist.riproduciCasuale()} che usa \texttt{Collections.shuffle(canzoni)}.
\end{enumerate}

\subsection{Esercizi avanzati}

\begin{enumerate}
    \item \textbf{Sistema prenotazioni}: Sviluppa un sistema di prenotazioni per un cinema. Crea classi \texttt{Film} (con attributi titolo, durata, genere), \texttt{Posto} (con attributi fila, numero, occupato), \texttt{Sala} (con attributi numero e \texttt{ArrayList<Posto> posti}) e \texttt{Prenotazione} (con attributi film, sala, \texttt{ArrayList<Posto> postiPrenotati}, nomeCliente). Implementa metodi come \texttt{sala.visualizzaPostiLiberi()}, \texttt{posto.prenota()}, \texttt{prenotazione.conferma()} che aggiorna i posti usando \texttt{posto.setOccupato(true)} e \texttt{prenotazione.annulla()} che libera i posti.

    \item \textbf{Social network semplificato}: Progetta un sistema base di social network dove ogni \texttt{Utente} ha attributi nome, \texttt{ArrayList<Utente> amici} e \texttt{ArrayList<String> post}. Implementa metodi come \texttt{utente.aggiungiAmico(Utente amico)} che usa sia \texttt{this.amici.add(amico)} che \texttt{amico.amici.add(this)} per gestire la bidirezionalità, \texttt{utente.rimuoviAmico(Utente amico)}, \texttt{utente.pubblicaPost(String post)} usando \texttt{post.add()}, \texttt{utente.visualizzaFeed()} che itera sui post propri e degli amici, e un metodo statico \texttt{Utente.cercaUtente(ArrayList<Utente> utenti, String nome)}.
\end{enumerate}

\section{Esercizi su Interfacce Grafiche}

Gli esercizi di questa sezione permettono di sviluppare competenze nella creazione di interfacce grafiche con Swing, come spiegato in \autoref{cap:interfacce_grafiche}.

\subsection{Esercizio 6.1 - Livello Base}
\textbf{Calcolatrice Semplice}

Crea una GUI con:
\begin{itemize}
    \item Due oggetti \texttt{JTextField} per inserire due numeri (usa \texttt{campo1.getText()} per ottenere il valore)
    \item Quattro pulsanti \texttt{JButton} per le operazioni: +, -, *, /
    \item Una \texttt{JLabel} per visualizzare il risultato (usa \texttt{etichetta.setText(String testo)} per aggiornarlo)
\end{itemize}

Quando l'utente clicca su un pulsante (gestito con un \texttt{ActionListener}), esegui l'operazione corrispondente e mostra il risultato usando \texttt{risultato.setText()}. Gestisci il caso di divisione per zero mostrando un messaggio di errore con \texttt{JOptionPane.showMessageDialog()}.

\subsection{Esercizio 6.2 - Livello Base}
\textbf{Cambio Colore}

Crea una finestra con:
\begin{itemize}
    \item Un oggetto \texttt{JPanel} centrale colorato
    \item Tre pulsanti \texttt{JButton}: "Rosso", "Verde", "Blu"
\end{itemize}

Quando l'utente clicca su un pulsante, cambia il colore di sfondo del pannello centrale usando \texttt{pannello.setBackground(Color colore)} e chiama \texttt{pannello.repaint()} per aggiornare la visualizzazione.

\subsection{Esercizio 6.3 - Livello Intermedio}
\textbf{Lista di Task}

Crea un'applicazione per gestire una lista di task con:
\begin{itemize}
    \item Un oggetto \texttt{JTextField} per inserire un nuovo task (usa \texttt{campoTask.getText()} per ottenere il testo)
    \item Un pulsante \texttt{JButton} "Aggiungi" per aggiungere il task
    \item Una \texttt{JTextArea} per visualizzare tutti i task (usa \texttt{areaTask.append(String testo)} per aggiungere righe)
    \item Un pulsante \texttt{JButton} "Cancella Tutto" per svuotare la lista (usa \texttt{areaTask.setText("")})
\end{itemize}

Ogni task aggiunto deve essere numerato progressivamente. Dopo aver aggiunto un task, svuota il campo di input con \texttt{campoTask.setText("")}.

\subsection{Esercizio 6.4 - Livello Intermedio}
\textbf{Convertitore di Temperature}

Crea una GUI che converte temperature tra Celsius e Fahrenheit:
\begin{itemize}
    \item Un oggetto \texttt{JTextField} per inserire la temperatura (usa \texttt{campoTemp.getText()} per ottenere il valore)
    \item Due oggetti \texttt{JRadioButton} in un \texttt{ButtonGroup} per selezionare la direzione di conversione (C to F oppure F to C) - usa \texttt{radioButton.isSelected()} per verificare quale è selezionato
    \item Un pulsante \texttt{JButton} "Converti"
    \item Una \texttt{JLabel} per visualizzare il risultato (usa \texttt{etichettaRisultato.setText(String testo)})
\end{itemize}

Formule: F = C * 9/5 + 32, C = (F - 32) * 5/9

\subsection{Esercizio 6.5 - Livello Avanzato}
\textbf{Disegno con il Mouse}

Crea un'applicazione di disegno semplice che:
\begin{itemize}
    \item Usa un oggetto \texttt{JPanel} personalizzato per disegnare
    \item Permette di disegnare linee trascinando il mouse (implementa \texttt{MouseMotionListener} e usa i metodi \texttt{pannello.mouseDragged(MouseEvent e)} per catturare i movimenti)
    \item Ha pulsanti \texttt{JButton} per scegliere il colore del disegno (rosso, verde, blu, nero) che modificano una variabile di stato \texttt{coloreCorrente}
    \item Ha un pulsante \texttt{JButton} "Cancella" per pulire il disegno (svuota la lista dei segmenti e chiama \texttt{pannello.repaint()})
    \item Tiene traccia di tutti i segmenti disegnati in un \texttt{ArrayList} di oggetti \texttt{Segmento}
\end{itemize}

Suggerimento: fai l'override del metodo \texttt{pannello.paintComponent(Graphics g)} del JPanel personalizzato per disegnare tutti i segmenti usando \texttt{g.drawLine()}.

\begin{errore}
\textbf{Errore comune}: Dimenticare di chiamare \texttt{finestra.setVisible(true)}.

Se la finestra non appare, assicurati di aver chiamato il metodo \texttt{finestra.setVisible(true)} sull'oggetto \texttt{JFrame} dopo aver configurato tutti i componenti. Inoltre, verifica di aver impostato una dimensione adeguata con \texttt{finestra.setSize(int larghezza, int altezza)} o \texttt{finestra.pack()}.
\end{errore}

\section{Esercizi su Model-View-Controller}

Gli esercizi di questa sezione permettono di praticare il pattern architetturale MVC, spiegato in dettaglio in \autoref{cap:model_view_controller}.

\subsection{Esercizio 7.1 - Livello Base}
\textbf{Contatore MVC}

Implementa un'applicazione contatore seguendo il pattern MVC:
\begin{itemize}
    \item \textbf{Model}: classe \texttt{ContatoreModel} che contiene l'attributo \texttt{valore} del contatore e metodi \texttt{model.increment()}, \texttt{model.decrement()}, \texttt{model.reset()} e \texttt{model.getValore()}
    \item \textbf{View}: classe \texttt{ContatoreView} che estende \texttt{JFrame}, mostra il valore corrente in una \texttt{JLabel} (accessibile con \texttt{view.getLabelValore()}) e ha tre pulsanti \texttt{JButton} (+, -, Reset) accessibili con getter
    \item \textbf{Controller}: classe \texttt{ContatoreController} che riceve riferimenti a Model e View, gestisce gli eventi dei pulsanti e chiama i metodi del Model (es: \texttt{model.increment()}) e aggiorna la View (es: \texttt{view.aggiorna(model.getValore())})
\end{itemize}

\subsection{Esercizio 7.2 - Livello Base}
\textbf{Calcolatrice MVC}

Crea una calcolatrice semplice con il pattern MVC:
\begin{itemize}
    \item \textbf{Model}: classe \texttt{CalcolatriceModel} con metodi \texttt{model.somma(double a, double b)}, \texttt{model.sottrazione(double a, double b)}, \texttt{model.moltiplicazione(double a, double b)}, \texttt{model.divisione(double a, double b)}
    \item \textbf{View}: classe \texttt{CalcolatriceView} con due \texttt{JTextField} per input (accessibili con \texttt{view.getCampo1()} e \texttt{view.getCampo2()}), pulsanti \texttt{JButton} per operazioni, e una \texttt{JLabel} per il risultato (aggiornabile con \texttt{view.setRisultato(String testo)})
    \item \textbf{Controller}: classe \texttt{CalcolatriceController} che gestisce i click sui pulsanti, ottiene i valori con \texttt{campo.getText()}, chiama i metodi appropriati del Model e aggiorna la View con il risultato
\end{itemize}

\subsection{Esercizio 7.3 - Livello Intermedio}
\textbf{Rubrica Telefonica MVC}

Implementa una rubrica telefonica:
\begin{itemize}
    \item \textbf{Model}: classe \texttt{RubricaModel} che gestisce un \texttt{ArrayList<Contatto>} con metodi \texttt{model.aggiungiContatto(Contatto c)}, \texttt{model.rimuoviContatto(String nome)}, \texttt{model.cercaContatto(String nome)} che restituisce un \texttt{Contatto}, e \texttt{model.getTuttiContatti()} che restituisce l'intera lista
    \item \textbf{View}: classe \texttt{RubricaView} con form di inserimento (\texttt{JTextField} per nome e telefono), una \texttt{JList} o \texttt{JTextArea} per visualizzare i contatti, un campo \texttt{JTextField} per la ricerca
    \item \textbf{Controller}: classe \texttt{RubricaController} che coordina le operazioni CRUD chiamando i metodi del Model (es: \texttt{model.aggiungiContatto()}) e aggiorna la View con \texttt{view.aggiornaLista(model.getTuttiContatti())}
\end{itemize}

\subsection{Esercizio 7.4 - Livello Intermedio}
\textbf{Gestione Studenti MVC}

Crea un sistema di gestione studenti:
\begin{itemize}
    \item \textbf{Model}: classe \texttt{Studente} (con attributi matricola, nome, \texttt{ArrayList<Double> voti} e metodo \texttt{studente.calcolaMedia()}) e classe \texttt{RegistroModel} con \texttt{ArrayList<Studente>} e metodi \texttt{model.aggiungiStudente(Studente s)}, \texttt{model.rimuoviStudente(String matricola)}, \texttt{model.getStudenti()}
    \item \textbf{View}: classe \texttt{RegistroView} con form inserimento studente (\texttt{JTextField} per matricola e nome), una \texttt{JTable} per visualizzare gli studenti, pannello per statistiche con \texttt{JLabel}
    \item \textbf{Controller}: classe \texttt{RegistroController} che gestisce le operazioni CRUD chiamando metodi del Model e aggiorna la View, calcola statistiche iterando su \texttt{model.getStudenti()}
\end{itemize}

Aggiungi funzionalità per ordinare studenti per nome usando \texttt{Collections.sort(studenti, comparatore)} o per media voti usando un comparatore personalizzato.

\begin{errore}
\textbf{Errore comune}: Mettere logica di business nel Controller.

\begin{lstlisting}
// SBAGLIATO: logica nel Controller
class AddTaskListener implements ActionListener {
    public void actionPerformed(ActionEvent e) {
        String task = view.getInputTask();
        if (task.trim().isEmpty()) { // Validazione nel controller!
            return;
        }
        model.addTask(task); // Il controller delega al model
    }
}
\end{lstlisting}

La validazione deve stare nel Model. Il Controller deve solo delegare:
\begin{lstlisting}
// CORRETTO: logica nel Model, Controller che delega
// Nel Model:
public boolean addTask(String task) {
    if (task == null || task.trim().isEmpty()) {
        return false; // Validazione nel model
    }
    todoList.add(task.trim());
    return true;
}

// Nel Controller:
class AddTaskListener implements ActionListener {
    public void actionPerformed(ActionEvent e) {
        String task = view.getInputTask();
        boolean successo = model.addTask(task);
        if (successo) {
            view.aggiornaLista(model.getTasks());
        }
    }
}
\end{lstlisting}
\end{errore}

\section{Esercizi su Lambda Expressions}

Gli esercizi di questa sezione permettono di sperimentare la programmazione funzionale in Java con le lambda expressions, argomento trattato in \autoref{cap:lambda_expressions}.

\subsection{Esercizio 8.1 - Livello Base}
\textbf{Ordinamento Studenti}

Data un \texttt{ArrayList<Studente>} (con attributi nome e voto), usa espressioni lambda per:
\begin{itemize}
    \item Ordinarli per voto crescente usando \texttt{studenti.sort((s1, s2) -> Double.compare(s1.getVoto(), s2.getVoto()))}
    \item Ordinarli per voto decrescente usando \texttt{studenti.sort((s1, s2) -> Double.compare(s2.getVoto(), s1.getVoto()))}
    \item Ordinarli alfabeticamente per nome usando \texttt{studenti.sort((s1, s2) -> s1.getNome().compareTo(s2.getNome()))}
\end{itemize}

Confronta la sintassi con lambda e quella equivalente con classe anonima \texttt{Comparator}.

\subsection{Esercizio 8.2 - Livello Intermedio}
\textbf{Filtro Prodotti}

Crea una classe \texttt{Prodotto} con attributi: nome, prezzo, categoria.

Implementa un metodo statico generico \texttt{Prodotto.filtra(ArrayList<Prodotto> prodotti, Predicate<Prodotto> condizione)} che accetta un \texttt{Predicate<Prodotto>} e filtra la lista usando \texttt{condizione.test(prodotto)}. Usa espressioni lambda per creare diversi predicati:
\begin{itemize}
    \item Trovare prodotti con prezzo maggiore di 50 euro: \texttt{p -> p.getPrezzo() > 50}
    \item Trovare prodotti della categoria "Elettronica": \texttt{p -> p.getCategoria().equals("Elettronica")}
    \item Trovare prodotti con nome che inizia per "A": \texttt{p -> p.getNome().startsWith("A")}
    \item Combinare più condizioni usando \texttt{Predicate.and()}: \texttt{p -> p.getPrezzo() < 100 && p.getCategoria().equals("Libri")}
\end{itemize}

\subsection{Esercizio 8.3 - Livello Avanzato}
\textbf{GUI con Lambda}

Crea una calcolatrice grafica che:
\begin{itemize}
    \item Ha due oggetti \texttt{JTextField} per inserire i numeri
    \item Ha quattro pulsanti \texttt{JButton} per operazioni (+, -, *, /)
    \item Mostra il risultato in una \texttt{JLabel}
\end{itemize}

\textbf{Requisiti}:
\begin{itemize}
    \item Usa espressioni lambda per tutti gli \texttt{ActionListener}, ad esempio: \texttt{pulsanteSomma.addActionListener(e -> \{...\})}
    \item Crea un'interfaccia funzionale \texttt{Operazione} con metodo astratto \texttt{operazione.calcola(double a, double b)} che restituisce double
    \item Assegna a ciascun pulsante una lambda che implementa l'operazione, ad esempio: \texttt{Operazione somma = (a, b) -> a + b}
    \item Gestisci errori (divisione per zero con \texttt{b == 0}, input non valido con \texttt{try-catch} su \texttt{Double.parseDouble()}) usando blocchi di codice nelle lambda
\end{itemize}

\begin{nota}
Le soluzioni complete e commentate di tutti questi esercizi sono disponibili nell'Appendice. Si consiglia di provare a risolvere gli esercizi autonomamente prima di consultare le soluzioni.
\end{nota}
