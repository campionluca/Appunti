\appendix
\chapter{Soluzioni degli Esercizi}

\section*{Nota}
In questa appendice sono riportate le soluzioni complete e commentate degli esercizi dei capitoli 1-4. Ogni soluzione include codice completo funzionante con commenti italiani e una breve spiegazione del funzionamento.

\section{Capitolo 1: Stream e Buffer}

\subsection{Esercizio 1.1: Contatore di righe}

\begin{lstlisting}
import java.io.BufferedReader;
import java.io.FileReader;
import java.io.IOException;

public class ContaRighe {
    public static void main(String[] args) {
        int numeroRighe = 0;

        // Try-with-resources per chiusura automatica
        try (BufferedReader br = new BufferedReader(
                new FileReader("file.txt"))) {

            // Legge ogni riga fino alla fine del file
            while (br.readLine() != null) {
                numeroRighe++;
            }

            System.out.println("Il file contiene " + numeroRighe + " righe");

        } catch (IOException e) {
            System.out.println("Errore lettura file: " + e.getMessage());
        }
    }
}
\end{lstlisting}

\textbf{Spiegazione:} Il programma utilizza BufferedReader per leggere il file riga per riga con readLine(), incrementando un contatore per ogni riga letta. Quando readLine() restituisce null, significa che si è raggiunta la fine del file.

\subsection{Esercizio 1.2: Copia carattere per carattere}

\begin{lstlisting}
import java.io.FileReader;
import java.io.FileWriter;
import java.io.IOException;

public class CopiaCarattere {
    public static void main(String[] args) {
        // Usa try-with-resources per gestire due risorse
        try (FileReader input = new FileReader("originale.txt");
             FileWriter output = new FileWriter("copia.txt")) {

            int carattere;

            // read() restituisce -1 quando raggiunge la fine
            while ((carattere = input.read()) != -1) {
                // Scrive il carattere letto nel file di output
                output.write(carattere);
            }

            System.out.println("Copia completata con successo!");

        } catch (IOException e) {
            System.out.println("Errore durante la copia: " + e.getMessage());
        }
    }
}
\end{lstlisting}

\textbf{Spiegazione:} Il programma legge un carattere alla volta con read() e lo scrive immediatamente nel file di destinazione. Try-with-resources garantisce la chiusura automatica di entrambi gli stream anche in caso di eccezione.

\subsection{Esercizio 1.3: Salva nomi in file}

\begin{lstlisting}
import java.io.BufferedWriter;
import java.io.FileWriter;
import java.io.IOException;
import java.util.Scanner;

public class SalvaNomi {
    public static void main(String[] args) {
        Scanner scanner = new Scanner(System.in);

        try (BufferedWriter bw = new BufferedWriter(
                new FileWriter("nomi.txt"))) {

            System.out.println("Inserisci 5 nomi:");

            // Legge 5 nomi dall'utente
            for (int i = 1; i <= 5; i++) {
                System.out.print("Nome " + i + ": ");
                String nome = scanner.nextLine();

                // Scrive il nome nel file
                bw.write(nome);
                bw.newLine(); // Aggiunge separatore di riga
            }

            System.out.println("Nomi salvati in nomi.txt");

        } catch (IOException e) {
            System.out.println("Errore scrittura file: " + e.getMessage());
        } finally {
            scanner.close();
        }
    }
}
\end{lstlisting}

\textbf{Spiegazione:} Il programma legge 5 nomi da console usando Scanner e li scrive su file usando BufferedWriter, aggiungendo newLine() dopo ogni nome per scriverli su righe separate.

\subsection{Esercizio 1.4: Inverti righe}

\begin{lstlisting}
import java.io.BufferedReader;
import java.io.BufferedWriter;
import java.io.FileReader;
import java.io.FileWriter;
import java.io.IOException;
import java.util.ArrayList;

public class InvertiRighe {
    public static void main(String[] args) {
        ArrayList<String> righe = new ArrayList<>();

        // Legge tutte le righe e le memorizza nell'ArrayList
        try (BufferedReader br = new BufferedReader(
                new FileReader("input.txt"))) {

            String riga;
            while ((riga = br.readLine()) != null) {
                righe.add(riga);
            }

        } catch (IOException e) {
            System.out.println("Errore lettura: " + e.getMessage());
            return;
        }

        // Scrive le righe in ordine inverso
        try (BufferedWriter bw = new BufferedWriter(
                new FileWriter("output.txt"))) {

            // Itera dall'ultima riga alla prima
            for (int i = righe.size() - 1; i >= 0; i--) {
                bw.write(righe.get(i));
                bw.newLine();
            }

            System.out.println("File invertito creato con successo!");

        } catch (IOException e) {
            System.out.println("Errore scrittura: " + e.getMessage());
        }
    }
}
\end{lstlisting}

\textbf{Spiegazione:} Prima legge tutte le righe in un ArrayList, poi le scrive in ordine inverso iterando dall'ultimo indice al primo. Questa tecnica permette di invertire facilmente l'ordine delle righe.

\subsection{Esercizio 1.5: Filtra righe maiuscole}

\begin{lstlisting}
import java.io.BufferedReader;
import java.io.BufferedWriter;
import java.io.FileReader;
import java.io.FileWriter;
import java.io.IOException;

public class FiltraMaiuscole {
    public static void main(String[] args) {
        try (BufferedReader input = new BufferedReader(
                new FileReader("input.txt"));
             BufferedWriter output = new BufferedWriter(
                new FileWriter("output.txt"))) {

            String riga;
            int righeCopiate = 0;

            while ((riga = input.readLine()) != null) {
                // Verifica se la riga non e' vuota e inizia con maiuscola
                if (!riga.isEmpty() &&
                    Character.isUpperCase(riga.charAt(0))) {

                    output.write(riga);
                    output.newLine();
                    righeCopiate++;
                }
            }

            System.out.println("Copiate " + righeCopiate +
                             " righe che iniziano con maiuscola");

        } catch (IOException e) {
            System.out.println("Errore: " + e.getMessage());
        }
    }
}
\end{lstlisting}

\textbf{Spiegazione:} Il programma legge ogni riga e verifica se non è vuota e se il primo carattere è una lettera maiuscola usando Character.isUpperCase(). Solo le righe che soddisfano questa condizione vengono scritte nel file di output.

\subsection{Esercizio 1.6: Merge di due file}

\begin{lstlisting}
import java.io.BufferedReader;
import java.io.BufferedWriter;
import java.io.FileReader;
import java.io.FileWriter;
import java.io.IOException;

public class MergeFile {
    public static void main(String[] args) {
        try (BufferedReader br1 = new BufferedReader(
                new FileReader("file1.txt"));
             BufferedReader br2 = new BufferedReader(
                new FileReader("file2.txt"));
             BufferedWriter output = new BufferedWriter(
                new FileWriter("merge.txt"))) {

            String riga1, riga2;

            // Legge alternando le righe dai due file
            while (true) {
                riga1 = br1.readLine();
                riga2 = br2.readLine();

                // Se entrambi i file sono finiti, esci dal ciclo
                if (riga1 == null && riga2 == null) {
                    break;
                }

                // Scrivi riga dal primo file se disponibile
                if (riga1 != null) {
                    output.write(riga1);
                    output.newLine();
                }

                // Scrivi riga dal secondo file se disponibile
                if (riga2 != null) {
                    output.write(riga2);
                    output.newLine();
                }
            }

            System.out.println("Merge completato!");

        } catch (IOException e) {
            System.out.println("Errore: " + e.getMessage());
        }
    }
}
\end{lstlisting}

\textbf{Spiegazione:} Il programma alterna la lettura di una riga dal primo file e una dal secondo, scrivendole nel file di output. Continua finché entrambi i file non hanno più righe da leggere.

\subsection{Esercizio 1.7: Analisi parole}

\begin{lstlisting}
import java.io.BufferedReader;
import java.io.FileReader;
import java.io.IOException;
import java.util.HashMap;
import java.util.Map;

public class AnalisiParole {
    public static void main(String[] args) {
        int totaleParole = 0;
        int totaleCaratteri = 0;
        // Mappa per contare la frequenza di ogni parola
        HashMap<String, Integer> frequenze = new HashMap<>();

        try (BufferedReader br = new BufferedReader(
                new FileReader("testo.txt"))) {

            String riga;

            while ((riga = br.readLine()) != null) {
                // Divide la riga in parole (split per spazi)
                String[] parole = riga.split("\\s+");

                for (String parola : parole) {
                    // Rimuove punteggiatura e converte in minuscolo
                    parola = parola.replaceAll("[^a-zA-Z]", "")
                                   .toLowerCase();

                    if (!parola.isEmpty()) {
                        totaleParole++;
                        totaleCaratteri += parola.length();

                        // Aggiorna frequenza della parola
                        frequenze.put(parola,
                                    frequenze.getOrDefault(parola, 0) + 1);
                    }
                }
            }

            // Stampa report
            System.out.println("=== REPORT ANALISI ===");
            System.out.println("Totale parole: " + totaleParole);
            System.out.println("Totale caratteri: " + totaleCaratteri);
            System.out.println("\nFrequenza parole:");

            for (Map.Entry<String, Integer> entry : frequenze.entrySet()) {
                System.out.println(entry.getKey() + ": " +
                                 entry.getValue() + " volte");
            }

        } catch (IOException e) {
            System.out.println("Errore: " + e.getMessage());
        }
    }
}
\end{lstlisting}

\textbf{Spiegazione:} Il programma divide ogni riga in parole, le pulisce dalla punteggiatura e conta i caratteri totali. Usa una HashMap per tenere traccia della frequenza di ogni parola, incrementando il contatore quando la parola viene incontrata.

\subsection{Esercizio 1.8: Parser CSV}

\begin{lstlisting}
import java.io.BufferedReader;
import java.io.FileReader;
import java.io.IOException;
import java.util.ArrayList;

public class ParserCSV {
    public static void main(String[] args) {
        ArrayList<String[]> dati = new ArrayList<>();
        int maxLunghezze[] = new int[3]; // Per 3 colonne

        // Legge file CSV
        try (BufferedReader br = new BufferedReader(
                new FileReader("dati.csv"))) {

            String riga;

            while ((riga = br.readLine()) != null) {
                // Split per virgola
                String[] campi = riga.split(",");
                dati.add(campi);

                // Calcola lunghezza massima per ogni colonna
                for (int i = 0; i < campi.length && i < 3; i++) {
                    if (campi[i].length() > maxLunghezze[i]) {
                        maxLunghezze[i] = campi[i].length();
                    }
                }
            }

        } catch (IOException e) {
            System.out.println("Errore lettura: " + e.getMessage());
            return;
        }

        // Stampa tabella allineata
        System.out.println("=== DATI CSV ===");

        for (String[] riga : dati) {
            for (int i = 0; i < riga.length && i < 3; i++) {
                // Stampa campo allineato a sinistra
                System.out.printf("%-" + (maxLunghezze[i] + 2) + "s",
                                riga[i]);
            }
            System.out.println();
        }
    }
}
\end{lstlisting}

\textbf{Spiegazione:} Il parser legge il file CSV, divide ogni riga usando split(",") e calcola la lunghezza massima di ogni colonna. Poi stampa i dati in formato tabellare allineato usando printf con formattazione dinamica.

\section{Capitolo 2: Interfacce e Classi Astratte}

\subsection{Esercizio 2.1: Interfaccia Volante}

\begin{lstlisting}
// Interfaccia Volante
public interface Volante {
    void vola();
}

// Classe Aereo
public class Aereo implements Volante {
    private String modello;
    private int passeggeri;

    public Aereo(String modello, int passeggeri) {
        this.modello = modello;
        this.passeggeri = passeggeri;
    }

    @Override
    public void vola() {
        System.out.println("L'aereo " + modello +
                         " vola trasportando " + passeggeri +
                         " passeggeri");
    }
}

// Classe Uccello
public class Uccello implements Volante {
    private String specie;

    public Uccello(String specie) {
        this.specie = specie;
    }

    @Override
    public void vola() {
        System.out.println("L'uccello " + specie +
                         " vola sbattendo le ali");
    }
}

// Test
public class TestVolante {
    public static void main(String[] args) {
        Volante[] volanti = {
            new Aereo("Boeing 747", 400),
            new Uccello("Aquila"),
            new Aereo("Cessna", 4),
            new Uccello("Gabbiano")
        };

        for (Volante v : volanti) {
            v.vola();
        }
    }
}
\end{lstlisting}

\textbf{Spiegazione:} L'interfaccia Volante definisce il contratto vola(), implementato diversamente da Aereo (con modello e passeggeri) e Uccello (con specie). Il polimorfismo permette di trattare entrambi come oggetti Volante.

\subsection{Esercizio 2.2: Classe astratta Animale}

\begin{lstlisting}
// Classe astratta Animale
public abstract class Animale {
    protected String nome;

    public Animale(String nome) {
        this.nome = nome;
    }

    // Metodo astratto - ogni animale ha un verso diverso
    public abstract void verso();

    // Metodo concreto - tutti gli animali dormono allo stesso modo
    public void dorme() {
        System.out.println(nome + " sta dormendo... Zzz");
    }
}

// Classe Cane
public class Cane extends Animale {
    public Cane(String nome) {
        super(nome);
    }

    @Override
    public void verso() {
        System.out.println(nome + " fa: Bau bau!");
    }
}

// Classe Gatto
public class Gatto extends Animale {
    public Gatto(String nome) {
        super(nome);
    }

    @Override
    public void verso() {
        System.out.println(nome + " fa: Miao miao!");
    }
}

// Test
public class TestAnimali {
    public static void main(String[] args) {
        Animale[] animali = {
            new Cane("Fido"),
            new Gatto("Whiskers"),
            new Cane("Rex")
        };

        for (Animale a : animali) {
            a.verso();
            a.dorme();
            System.out.println();
        }
    }
}
\end{lstlisting}

\textbf{Spiegazione:} La classe astratta Animale fornisce il metodo concreto dorme() condiviso da tutti gli animali, mentre verso() è astratto e deve essere implementato da ogni sottoclasse con il proprio verso specifico.

\subsection{Esercizio 2.3: Interfaccia Pagabile}

\begin{lstlisting}
// Interfaccia Pagabile
public interface Pagabile {
    double calcolaStipendio();
}

// Classe Dipendente
public class Dipendente implements Pagabile {
    private String nome;
    private double stipendioMensile;

    public Dipendente(String nome, double stipendioMensile) {
        this.nome = nome;
        this.stipendioMensile = stipendioMensile;
    }

    @Override
    public double calcolaStipendio() {
        // Stipendio fisso mensile
        return stipendioMensile;
    }

    public String getNome() {
        return nome;
    }
}

// Classe Freelance
public class Freelance implements Pagabile {
    private String nome;
    private double tariffaOraria;
    private int oreLavorate;

    public Freelance(String nome, double tariffaOraria, int oreLavorate) {
        this.nome = nome;
        this.tariffaOraria = tariffaOraria;
        this.oreLavorate = oreLavorate;
    }

    @Override
    public double calcolaStipendio() {
        // Stipendio basato su ore lavorate
        return tariffaOraria * oreLavorate;
    }

    public String getNome() {
        return nome;
    }
}

// Test
public class TestPagabile {
    public static void main(String[] args) {
        Pagabile[] lavoratori = {
            new Dipendente("Mario Rossi", 2000),
            new Freelance("Laura Bianchi", 50, 80),
            new Dipendente("Giuseppe Verdi", 2500),
            new Freelance("Anna Neri", 60, 120)
        };

        double totalePagamenti = 0;

        for (Pagabile p : lavoratori) {
            double stipendio = p.calcolaStipendio();
            totalePagamenti += stipendio;

            String nome = "";
            if (p instanceof Dipendente) {
                nome = ((Dipendente)p).getNome();
            } else if (p instanceof Freelance) {
                nome = ((Freelance)p).getNome();
            }

            System.out.printf("%s: %.2f euro\n", nome, stipendio);
        }

        System.out.printf("\nTotale pagamenti: %.2f euro\n",
                        totalePagamenti);
    }
}
\end{lstlisting}

\textbf{Spiegazione:} L'interfaccia Pagabile definisce il metodo calcolaStipendio(). Dipendente restituisce lo stipendio mensile fisso, mentre Freelance calcola il pagamento moltiplicando tariffa oraria per ore lavorate.

\subsection{Esercizio 2.4: Interfaccia DispositivoElettronico}

\begin{lstlisting}
// Interfaccia DispositivoElettronico
public interface DispositivoElettronico {
    void accendi();
    void spegni();
    double consumoEnergetico(); // in Watt
}

// Classe Televisore
public class Televisore implements DispositivoElettronico {
    private boolean acceso = false;
    private int pollici;

    public Televisore(int pollici) {
        this.pollici = pollici;
    }

    @Override
    public void accendi() {
        acceso = true;
        System.out.println("TV " + pollici + "\" accesa");
    }

    @Override
    public void spegni() {
        acceso = false;
        System.out.println("TV " + pollici + "\" spenta");
    }

    @Override
    public double consumoEnergetico() {
        // Consumo proporzionale ai pollici
        return acceso ? pollici * 2.5 : 0.5; // standby
    }
}

// Classe Computer
public class Computer implements DispositivoElettronico {
    private boolean acceso = false;
    private boolean desktop;

    public Computer(boolean desktop) {
        this.desktop = desktop;
    }

    @Override
    public void accendi() {
        acceso = true;
        System.out.println((desktop ? "Desktop" : "Laptop") + " acceso");
    }

    @Override
    public void spegni() {
        acceso = false;
        System.out.println((desktop ? "Desktop" : "Laptop") + " spento");
    }

    @Override
    public double consumoEnergetico() {
        if (!acceso) return 0;
        return desktop ? 150 : 65;
    }
}

// Classe Lampada
public class Lampada implements DispositivoElettronico {
    private boolean accesa = false;
    private int watt;

    public Lampada(int watt) {
        this.watt = watt;
    }

    @Override
    public void accendi() {
        accesa = true;
        System.out.println("Lampada " + watt + "W accesa");
    }

    @Override
    public void spegni() {
        accesa = false;
        System.out.println("Lampada " + watt + "W spenta");
    }

    @Override
    public double consumoEnergetico() {
        return accesa ? watt : 0;
    }
}

// Test
public class TestDispositivi {
    public static double calcolaConsumoTotale(
            DispositivoElettronico[] dispositivi) {
        double totale = 0;
        for (DispositivoElettronico d : dispositivi) {
            totale += d.consumoEnergetico();
        }
        return totale;
    }

    public static void main(String[] args) {
        DispositivoElettronico[] dispositivi = {
            new Televisore(55),
            new Computer(true),
            new Lampada(60),
            new Computer(false)
        };

        // Accende tutti i dispositivi
        for (DispositivoElettronico d : dispositivi) {
            d.accendi();
        }

        System.out.println();
        double consumo = calcolaConsumoTotale(dispositivi);
        System.out.printf("Consumo totale: %.2f Watt\n", consumo);
    }
}
\end{lstlisting}

\textbf{Spiegazione:} Ogni dispositivo implementa i metodi accendi(), spegni() e consumoEnergetico() con logiche diverse. Il metodo calcolaConsumoTotale() somma i consumi di tutti i dispositivi in un array, dimostrando il polimorfismo.

\subsection{Esercizio 2.5: Classe astratta FiguraPiana}

\begin{lstlisting}
// Classe astratta FiguraPiana
public abstract class FiguraPiana {
    protected String colore;

    public FiguraPiana(String colore) {
        this.colore = colore;
    }

    // Metodi astratti
    public abstract double area();
    public abstract double perimetro();

    // Metodo concreto che usa i metodi astratti
    public String confrontaArea(FiguraPiana altra) {
        double miaArea = this.area();
        double altraArea = altra.area();

        if (miaArea > altraArea) {
            return "Questa figura ha area maggiore";
        } else if (miaArea < altraArea) {
            return "L'altra figura ha area maggiore";
        } else {
            return "Le due figure hanno la stessa area";
        }
    }
}

// Classe Quadrato
public class Quadrato extends FiguraPiana {
    private double lato;

    public Quadrato(String colore, double lato) {
        super(colore);
        this.lato = lato;
    }

    @Override
    public double area() {
        return lato * lato;
    }

    @Override
    public double perimetro() {
        return 4 * lato;
    }
}

// Classe Cerchio
public class Cerchio extends FiguraPiana {
    private double raggio;

    public Cerchio(String colore, double raggio) {
        super(colore);
        this.raggio = raggio;
    }

    @Override
    public double area() {
        return Math.PI * raggio * raggio;
    }

    @Override
    public double perimetro() {
        return 2 * Math.PI * raggio;
    }
}

// Classe Triangolo
public class Triangolo extends FiguraPiana {
    private double base;
    private double altezza;
    private double lato1, lato2, lato3;

    public Triangolo(String colore, double base, double altezza,
                    double lato1, double lato2, double lato3) {
        super(colore);
        this.base = base;
        this.altezza = altezza;
        this.lato1 = lato1;
        this.lato2 = lato2;
        this.lato3 = lato3;
    }

    @Override
    public double area() {
        return (base * altezza) / 2;
    }

    @Override
    public double perimetro() {
        return lato1 + lato2 + lato3;
    }
}

// Test
public class TestFigure {
    public static void main(String[] args) {
        FiguraPiana[] figure = {
            new Quadrato("Rosso", 5),
            new Cerchio("Blu", 3),
            new Triangolo("Verde", 4, 3, 3, 4, 5)
        };

        System.out.println("=== ANALISI FIGURE ===");
        for (FiguraPiana f : figure) {
            System.out.printf("Area: %.2f - Perimetro: %.2f\n",
                            f.area(), f.perimetro());
        }

        System.out.println("\n=== CONFRONTI ===");
        System.out.println(figure[0].confrontaArea(figure[1]));
        System.out.println(figure[1].confrontaArea(figure[2]));
    }
}
\end{lstlisting}

\textbf{Spiegazione:} La classe astratta fornisce il metodo concreto confrontaArea() che usa i metodi astratti area(). Ogni figura implementa area() e perimetro() con le proprie formule geometriche.

\subsection{Esercizio 2.6: Libro Comparable}

\begin{lstlisting}
import java.util.Arrays;

// Classe Libro che implementa Comparable
public class Libro implements Comparable<Libro> {
    private String titolo;
    private String autore;
    private int anno;

    public Libro(String titolo, String autore, int anno) {
        this.titolo = titolo;
        this.autore = autore;
        this.anno = anno;
    }

    @Override
    public int compareTo(Libro altro) {
        // Confronta per titolo (ordine alfabetico)
        return this.titolo.compareTo(altro.titolo);
    }

    @Override
    public String toString() {
        return String.format("\"%s\" di %s (%d)", titolo, autore, anno);
    }

    // Getter
    public String getTitolo() { return titolo; }
    public String getAutore() { return autore; }
    public int getAnno() { return anno; }
}

// Test
public class TestLibri {
    public static void main(String[] args) {
        Libro[] biblioteca = {
            new Libro("Zeno", "Italo Svevo", 1923),
            new Libro("Il nome della rosa", "Umberto Eco", 1980),
            new Libro("Annabelle Lee", "Edgar Allan Poe", 1849),
            new Libro("Divina Commedia", "Dante Alighieri", 1321),
            new Libro("Promessi Sposi", "Alessandro Manzoni", 1827)
        };

        System.out.println("=== PRIMA DELL'ORDINAMENTO ===");
        for (Libro libro : biblioteca) {
            System.out.println(libro);
        }

        // Ordina usando compareTo (per titolo)
        Arrays.sort(biblioteca);

        System.out.println("\n=== DOPO ORDINAMENTO (per titolo) ===");
        for (Libro libro : biblioteca) {
            System.out.println(libro);
        }
    }
}
\end{lstlisting}

\textbf{Spiegazione:} La classe Libro implementa Comparable<Libro> definendo il metodo compareTo() che confronta i titoli in ordine alfabetico. Arrays.sort() usa automaticamente questo metodo per ordinare l'array.

\subsection{Esercizio 2.7: Sistema Account Bancari}

\begin{lstlisting}
// Classe astratta Account
public abstract class Account {
    protected String numeroAccount;
    protected String intestatario;
    protected double saldo;

    public Account(String numero, String intestatario, double saldoIniziale) {
        this.numeroAccount = numero;
        this.intestatario = intestatario;
        this.saldo = saldoIniziale;
    }

    // Metodi concreti comuni
    public void deposita(double importo) {
        if (importo > 0) {
            saldo += importo;
            System.out.printf("Depositati %.2f euro\n", importo);
        }
    }

    public boolean preleva(double importo) {
        if (importo > 0 && importo <= saldo) {
            saldo -= importo;
            System.out.printf("Prelevati %.2f euro\n", importo);
            return true;
        }
        System.out.println("Prelievo non riuscito");
        return false;
    }

    public double getSaldo() {
        return saldo;
    }

    // Metodi astratti
    public abstract double calcolaInteressi();
    public abstract void applicaCommissioni();
}

// Account Corrente
public class AccountCorrente extends Account {
    private static final double COMMISSIONE_MENSILE = 5.0;
    private static final double TASSO_INTERESSE = 0.001; // 0.1%

    public AccountCorrente(String numero, String intestatario,
                          double saldoIniziale) {
        super(numero, intestatario, saldoIniziale);
    }

    @Override
    public double calcolaInteressi() {
        // Interessi molto bassi
        return saldo * TASSO_INTERESSE;
    }

    @Override
    public void applicaCommissioni() {
        saldo -= COMMISSIONE_MENSILE;
        System.out.printf("Commissione applicata: %.2f euro\n",
                        COMMISSIONE_MENSILE);
    }
}

// Account Risparmio
public class AccountRisparmio extends Account {
    private static final double TASSO_INTERESSE = 0.02; // 2%

    public AccountRisparmio(String numero, String intestatario,
                           double saldoIniziale) {
        super(numero, intestatario, saldoIniziale);
    }

    @Override
    public double calcolaInteressi() {
        // Interessi piu' alti
        return saldo * TASSO_INTERESSE;
    }

    @Override
    public void applicaCommissioni() {
        // Nessuna commissione per risparmio
        System.out.println("Nessuna commissione per account risparmio");
    }

    @Override
    public boolean preleva(double importo) {
        // Limite prelievi per risparmio
        if (importo > 1000) {
            System.out.println("Limite prelievo: max 1000 euro");
            return false;
        }
        return super.preleva(importo);
    }
}

// Account Deposito
public class AccountDeposito extends Account {
    private static final double TASSO_INTERESSE = 0.035; // 3.5%
    private int mesiVincolo;

    public AccountDeposito(String numero, String intestatario,
                          double saldoIniziale, int mesiVincolo) {
        super(numero, intestatario, saldoIniziale);
        this.mesiVincolo = mesiVincolo;
    }

    @Override
    public double calcolaInteressi() {
        // Interessi piu' alti se vincolato
        return saldo * TASSO_INTERESSE * (mesiVincolo / 12.0);
    }

    @Override
    public void applicaCommissioni() {
        // Nessuna commissione
        System.out.println("Nessuna commissione per deposito");
    }

    @Override
    public boolean preleva(double importo) {
        System.out.println("Impossibile prelevare da deposito vincolato");
        return false;
    }
}

// Test
public class TestBanca {
    public static void main(String[] args) {
        Account[] conti = {
            new AccountCorrente("CC001", "Mario Rossi", 1000),
            new AccountRisparmio("RS001", "Laura Bianchi", 5000),
            new AccountDeposito("DP001", "Giuseppe Verdi", 10000, 12)
        };

        System.out.println("=== OPERAZIONI MENSILI ===\n");

        for (Account a : conti) {
            System.out.println("Account: " + a.intestatario);
            System.out.printf("Saldo iniziale: %.2f euro\n", a.getSaldo());

            // Calcola e accredita interessi
            double interessi = a.calcolaInteressi();
            a.deposita(interessi);
            System.out.printf("Interessi maturati: %.2f euro\n", interessi);

            // Applica commissioni
            a.applicaCommissioni();

            System.out.printf("Saldo finale: %.2f euro\n\n", a.getSaldo());
        }
    }
}
\end{lstlisting}

\textbf{Spiegazione:} La classe astratta Account fornisce metodi comuni (deposita, preleva) e metodi astratti (calcolaInteressi, applicaCommissioni). Ogni tipo di account implementa questi metodi con regole specifiche: corrente ha commissioni, risparmio ha limiti di prelievo, deposito non permette prelievi.

\subsection{Esercizio 2.8: Interfaccia Ordinabile}

\begin{lstlisting}
// Interfaccia Ordinabile
public interface Ordinabile {
    int confronta(Ordinabile altro);
    String getChiave();
}

// Classe Prodotto
public class Prodotto implements Ordinabile {
    private String codice;
    private String nome;
    private double prezzo;

    public Prodotto(String codice, String nome, double prezzo) {
        this.codice = codice;
        this.nome = nome;
        this.prezzo = prezzo;
    }

    @Override
    public int confronta(Ordinabile altro) {
        // Ordina per codice
        return this.codice.compareTo(((Prodotto)altro).codice);
    }

    @Override
    public String getChiave() {
        return codice;
    }

    @Override
    public String toString() {
        return String.format("%s - %s (%.2f euro)", codice, nome, prezzo);
    }
}

// Classe Cliente
public class Cliente implements Ordinabile {
    private String id;
    private String nome;
    private String cognome;

    public Cliente(String id, String nome, String cognome) {
        this.id = id;
        this.nome = nome;
        this.cognome = cognome;
    }

    @Override
    public int confronta(Ordinabile altro) {
        // Ordina per cognome, poi nome
        Cliente altroCliente = (Cliente)altro;
        int confrontoCognome = this.cognome.compareTo(altroCliente.cognome);
        if (confrontoCognome != 0) {
            return confrontoCognome;
        }
        return this.nome.compareTo(altroCliente.nome);
    }

    @Override
    public String getChiave() {
        return id;
    }

    @Override
    public String toString() {
        return String.format("%s: %s %s", id, nome, cognome);
    }
}

// Classe Ordine
public class Ordine implements Ordinabile {
    private int numeroOrdine;
    private String dataOrdine;
    private double totale;

    public Ordine(int numero, String data, double totale) {
        this.numeroOrdine = numero;
        this.dataOrdine = data;
        this.totale = totale;
    }

    @Override
    public int confronta(Ordinabile altro) {
        // Ordina per numero ordine
        Ordine altroOrdine = (Ordine)altro;
        return Integer.compare(this.numeroOrdine, altroOrdine.numeroOrdine);
    }

    @Override
    public String getChiave() {
        return String.valueOf(numeroOrdine);
    }

    @Override
    public String toString() {
        return String.format("Ordine %d del %s - Totale: %.2f euro",
                           numeroOrdine, dataOrdine, totale);
    }
}

// Metodo generico di ordinamento
public class Ordinatore {
    // Bubble sort generico per Ordinabile
    public static void ordina(Ordinabile[] array) {
        int n = array.length;
        for (int i = 0; i < n - 1; i++) {
            for (int j = 0; j < n - i - 1; j++) {
                if (array[j].confronta(array[j + 1]) > 0) {
                    // Scambia elementi
                    Ordinabile temp = array[j];
                    array[j] = array[j + 1];
                    array[j + 1] = temp;
                }
            }
        }
    }
}

// Test
public class TestOrdinabile {
    public static void main(String[] args) {
        // Test con Prodotti
        Ordinabile[] prodotti = {
            new Prodotto("P003", "Mouse", 25.99),
            new Prodotto("P001", "Tastiera", 45.50),
            new Prodotto("P002", "Monitor", 199.99)
        };

        System.out.println("=== PRODOTTI NON ORDINATI ===");
        stampa(prodotti);

        Ordinatore.ordina(prodotti);

        System.out.println("\n=== PRODOTTI ORDINATI ===");
        stampa(prodotti);

        // Test con Clienti
        Ordinabile[] clienti = {
            new Cliente("C003", "Mario", "Rossi"),
            new Cliente("C001", "Laura", "Bianchi"),
            new Cliente("C002", "Anna", "Bianchi")
        };

        System.out.println("\n=== CLIENTI NON ORDINATI ===");
        stampa(clienti);

        Ordinatore.ordina(clienti);

        System.out.println("\n=== CLIENTI ORDINATI ===");
        stampa(clienti);
    }

    private static void stampa(Ordinabile[] array) {
        for (Ordinabile o : array) {
            System.out.println(o);
        }
    }
}
\end{lstlisting}

\textbf{Spiegazione:} L'interfaccia Ordinabile permette di scrivere un algoritmo di ordinamento generico che funziona con qualsiasi tipo che implementa confronta(). Ogni classe definisce la propria logica di confronto: Prodotto per codice, Cliente per cognome/nome, Ordine per numero.

\section{Capitolo 3: Eccezioni}

\subsection{Esercizio 3.1: Input numerico con gestione eccezioni}

\begin{lstlisting}
import java.util.Scanner;
import java.util.InputMismatchException;

public class InputNumerico {
    public static void main(String[] args) {
        Scanner scanner = new Scanner(System.in);
        int numero = 0;
        boolean inputValido = false;

        // Continua a chiedere finche' l'input non e' valido
        while (!inputValido) {
            try {
                System.out.print("Inserisci un numero intero: ");
                numero = scanner.nextInt();
                inputValido = true; // Input valido, esci dal ciclo

            } catch (InputMismatchException e) {
                System.out.println("ERRORE: Devi inserire un numero intero!");
                scanner.nextLine(); // Pulisce l'input errato
            }
        }

        System.out.println("Hai inserito: " + numero);
        scanner.close();
    }
}
\end{lstlisting}

\textbf{Spiegazione:} Il programma usa un ciclo while per continuare a chiedere l'input finché non viene inserito un numero valido. InputMismatchException viene catturata quando l'utente inserisce testo invece di un numero.

\subsection{Esercizio 3.2: Divisione sicura}

\begin{lstlisting}
public class DivisioneSicura {

    public static double dividi(double dividendo, double divisore) {
        try {
            // Lancia eccezione se divisore e' zero
            if (divisore == 0) {
                throw new ArithmeticException("Divisione per zero");
            }
            return dividendo / divisore;

        } catch (ArithmeticException e) {
            System.out.println("Attenzione: " + e.getMessage() +
                             " - Restituisco 0");
            return 0;
        }
    }

    public static void main(String[] args) {
        // Test vari casi
        System.out.println("10 / 2 = " + dividi(10, 2));
        System.out.println("15 / 3 = " + dividi(15, 3));
        System.out.println("7 / 0 = " + dividi(7, 0));  // Gestisce eccezione
        System.out.println("20 / 4 = " + dividi(20, 4));
    }
}
\end{lstlisting}

\textbf{Spiegazione:} Il metodo dividi() controlla se il divisore è zero e in tal caso lancia un'eccezione ArithmeticException. Il catch intercetta l'eccezione e restituisce 0 come valore di default.

\subsection{Esercizio 3.3: Accesso array sicuro}

\begin{lstlisting}
import java.util.Scanner;

public class AccessoArraySicuro {
    public static void main(String[] args) {
        String[] frutti = {"Mela", "Banana", "Arancia", "Pera", "Kiwi"};
        Scanner scanner = new Scanner(System.in);

        System.out.println("Array disponibile con " + frutti.length +
                         " elementi (indici 0-" + (frutti.length - 1) + ")");

        System.out.print("Inserisci indice da visualizzare: ");
        int indice = scanner.nextInt();

        try {
            // Tenta di accedere all'elemento
            String frutto = frutti[indice];
            System.out.println("Elemento all'indice " + indice +
                             ": " + frutto);

        } catch (ArrayIndexOutOfBoundsException e) {
            System.out.println("ERRORE: Indice " + indice +
                             " non valido!");
            System.out.println("Gli indici validi sono da 0 a " +
                             (frutti.length - 1));
        } finally {
            scanner.close();
        }
    }
}
\end{lstlisting}

\textbf{Spiegazione:} Il programma cattura ArrayIndexOutOfBoundsException quando l'utente inserisce un indice fuori dai limiti dell'array, mostrando un messaggio di errore chiaro con gli indici validi.

\subsection{Esercizio 3.4: Eccezione personalizzata NumeroNegativo}

\begin{lstlisting}
// Eccezione personalizzata checked
public class NumeroNegativoException extends Exception {
    private double numeroErrato;

    public NumeroNegativoException(double numero) {
        super("Impossibile calcolare radice quadrata di numero negativo: "
              + numero);
        this.numeroErrato = numero;
    }

    public double getNumeroErrato() {
        return numeroErrato;
    }
}

// Classe con metodo che usa l'eccezione
public class CalcolatoreRadici {

    public static double calcolaRadiceQuadrata(double n)
            throws NumeroNegativoException {

        if (n < 0) {
            throw new NumeroNegativoException(n);
        }

        return Math.sqrt(n);
    }

    public static void main(String[] args) {
        double[] numeri = {16, 25, -9, 49, -4, 64};

        System.out.println("=== CALCOLO RADICI QUADRATE ===");

        for (double num : numeri) {
            try {
                double radice = calcolaRadiceQuadrata(num);
                System.out.printf("Radice di %.1f = %.2f\n", num, radice);

            } catch (NumeroNegativoException e) {
                System.out.println("ERRORE: " + e.getMessage());
                System.out.println("Numero problematico: " +
                                 e.getNumeroErrato());
            }
        }
    }
}
\end{lstlisting}

\textbf{Spiegazione:} Viene creata un'eccezione checked personalizzata che memorizza il numero negativo che ha causato l'errore. Il metodo calcolaRadiceQuadrata() dichiara throws e lancia l'eccezione se il numero è negativo.

\subsection{Esercizio 3.5: Classe Calcolatrice con eccezioni}

\begin{lstlisting}
// Eccezioni personalizzate
class DivisoreZeroException extends Exception {
    public DivisoreZeroException() {
        super("Errore: divisione per zero non permessa");
    }
}

class OperazioneNonValidaException extends Exception {
    public OperazioneNonValidaException(String messaggio) {
        super(messaggio);
    }
}

// Classe Calcolatrice
public class Calcolatrice {

    public double somma(double a, double b) {
        return a + b;
    }

    public double sottrai(double a, double b) {
        return a - b;
    }

    public double moltiplica(double a, double b)
            throws OperazioneNonValidaException {
        // Controlla overflow per numeri molto grandi
        if (Math.abs(a) > 1e100 || Math.abs(b) > 1e100) {
            throw new OperazioneNonValidaException(
                "Numeri troppo grandi per la moltiplicazione");
        }
        return a * b;
    }

    public double dividi(double dividendo, double divisore)
            throws DivisoreZeroException {
        if (divisore == 0) {
            throw new DivisoreZeroException();
        }
        return dividendo / divisore;
    }

    public static void main(String[] args) {
        Calcolatrice calc = new Calcolatrice();

        System.out.println("=== TEST CALCOLATRICE ===\n");

        // Test somma
        System.out.println("5 + 3 = " + calc.somma(5, 3));

        // Test sottrazione
        System.out.println("10 - 4 = " + calc.sottrai(10, 4));

        // Test moltiplicazione
        try {
            System.out.println("6 * 7 = " + calc.moltiplica(6, 7));
            System.out.println("1e101 * 2 = " + calc.moltiplica(1e101, 2));
        } catch (OperazioneNonValidaException e) {
            System.out.println("ERRORE: " + e.getMessage());
        }

        // Test divisione
        try {
            System.out.println("20 / 5 = " + calc.dividi(20, 5));
            System.out.println("15 / 0 = " + calc.dividi(15, 0));
        } catch (DivisoreZeroException e) {
            System.out.println("ERRORE: " + e.getMessage());
        }
    }
}
\end{lstlisting}

\textbf{Spiegazione:} La classe Calcolatrice implementa le 4 operazioni base, lanciando eccezioni personalizzate per situazioni non valide: DivisoreZeroException per divisioni per zero e OperazioneNonValidaException per numeri troppo grandi.

\subsection{Esercizio 3.6: Conta righe file con gestione eccezioni}

\begin{lstlisting}
import java.io.BufferedReader;
import java.io.FileReader;
import java.io.FileNotFoundException;
import java.io.IOException;

public class ContaRigheConEccezioni {

    public static int contaRighe(String percorsoFile)
            throws FileNotFoundException, IOException {

        int righe = 0;

        try (BufferedReader br = new BufferedReader(
                new FileReader(percorsoFile))) {

            while (br.readLine() != null) {
                righe++;
            }
        }
        // Try-with-resources chiude automaticamente br
        // Le eccezioni vengono propagate al chiamante

        return righe;
    }

    public static void main(String[] args) {
        String[] files = {"documento.txt", "file_inesistente.txt",
                         "dati.csv"};

        for (String file : files) {
            try {
                int numeroRighe = contaRighe(file);
                System.out.println(file + ": " + numeroRighe + " righe");

            } catch (FileNotFoundException e) {
                System.out.println("ERRORE: File '" + file +
                                 "' non trovato");
                System.out.println("Verifica che il file esista " +
                                 "nella directory corrente");

            } catch (IOException e) {
                System.out.println("ERRORE lettura file '" + file + "': " +
                                 e.getMessage());
                System.out.println("Il file potrebbe essere corrotto " +
                                 "o non accessibile");
            }
        }
    }
}
\end{lstlisting}

\textbf{Spiegazione:} Il programma usa catch separati per FileNotFoundException (file non esiste) e IOException (errore generico di lettura), fornendo messaggi di errore specifici per ogni situazione.

\subsection{Esercizio 3.7: Gerarchia eccezioni prenotazioni}

\begin{lstlisting}
// Eccezione base
class PrenotazioneException extends Exception {
    public PrenotazioneException(String messaggio) {
        super(messaggio);
    }
}

// Eccezioni specifiche
class PostiEsauritiException extends PrenotazioneException {
    private int postiRichiesti;
    private int postiDisponibili;

    public PostiEsauritiException(int richiesti, int disponibili) {
        super("Posti esauriti: richiesti " + richiesti +
              ", disponibili " + disponibili);
        this.postiRichiesti = richiesti;
        this.postiDisponibili = disponibili;
    }

    public int getPostiDisponibili() {
        return postiDisponibili;
    }
}

class DataNonValidaException extends PrenotazioneException {
    public DataNonValidaException(String data) {
        super("Data non valida: " + data);
    }
}

class PagamentoFallitoException extends PrenotazioneException {
    private String motivoFallimento;

    public PagamentoFallitoException(String motivo) {
        super("Pagamento fallito: " + motivo);
        this.motivoFallimento = motivo;
    }

    public String getMotivoFallimento() {
        return motivoFallimento;
    }
}

// Sistema di prenotazioni
public class SistemaPrenotazioni {
    private int postiTotali = 100;
    private int postiPrenotati = 0;

    public void prenota(String data, int numPersone, String metodoPagamento)
            throws PrenotazioneException {

        // Valida data (formato semplice: gg/mm/aaaa)
        if (!data.matches("\\d{2}/\\d{2}/\\d{4}")) {
            throw new DataNonValidaException(data);
        }

        // Controlla disponibilita' posti
        int postiLiberi = postiTotali - postiPrenotati;
        if (numPersone > postiLiberi) {
            throw new PostiEsauritiException(numPersone, postiLiberi);
        }

        // Simula pagamento
        if (!elaboraPagamento(metodoPagamento)) {
            throw new PagamentoFallitoException(
                "Metodo di pagamento non valido: " + metodoPagamento);
        }

        // Prenotazione confermata
        postiPrenotati += numPersone;
        System.out.println("Prenotazione confermata per " + numPersone +
                         " persone il " + data);
        System.out.println("Posti rimanenti: " +
                         (postiTotali - postiPrenotati));
    }

    private boolean elaboraPagamento(String metodo) {
        // Simula validazione pagamento
        return metodo.equals("carta") || metodo.equals("contanti");
    }

    public static void main(String[] args) {
        SistemaPrenotazioni sistema = new SistemaPrenotazioni();

        // Test varie prenotazioni
        String[][] prenotazioni = {
            {"15/06/2025", "4", "carta"},
            {"20/06/2025", "2", "contanti"},
            {"data_errata", "3", "carta"},
            {"25/06/2025", "150", "carta"},
            {"30/06/2025", "5", "paypal"}
        };

        for (String[] p : prenotazioni) {
            try {
                System.out.println("\n--- Tentativo prenotazione ---");
                sistema.prenota(p[0], Integer.parseInt(p[1]), p[2]);

            } catch (DataNonValidaException e) {
                System.out.println("ERRORE: " + e.getMessage());
                System.out.println("Formato richiesto: gg/mm/aaaa");

            } catch (PostiEsauritiException e) {
                System.out.println("ERRORE: " + e.getMessage());
                System.out.println("Suggerimento: prenotare solo " +
                                 e.getPostiDisponibili() + " posti");

            } catch (PagamentoFallitoException e) {
                System.out.println("ERRORE: " + e.getMessage());
                System.out.println("Metodi accettati: carta, contanti");

            } catch (PrenotazioneException e) {
                // Catch generico per altre eccezioni
                System.out.println("ERRORE generico: " + e.getMessage());
            }
        }
    }
}
\end{lstlisting}

\textbf{Spiegazione:} Viene creata una gerarchia di eccezioni con PrenotazioneException come base e tre eccezioni specifiche che ereditano da essa. Ogni eccezione può memorizzare informazioni aggiuntive (posti disponibili, motivo fallimento) e il catch può gestirle in modo specifico.

\subsection{Esercizio 3.8: Parser JSON semplificato}

\begin{lstlisting}
// Eccezioni personalizzate per parsing JSON
class JsonSyntaxException extends Exception {
    public JsonSyntaxException(String messaggio) {
        super("Errore sintassi JSON: " + messaggio);
    }
}

class JsonKeyException extends Exception {
    public JsonKeyException(String chiave) {
        super("Chiave JSON non valida: '" + chiave + "'");
    }
}

class JsonValueException extends Exception {
    public JsonValueException(String valore) {
        super("Valore JSON non valido: '" + valore + "'");
    }
}

// Parser JSON semplificato
public class SimpleJsonParser {

    public static void parseJson(String json)
            throws JsonSyntaxException, JsonKeyException, JsonValueException {

        // Rimuove spazi
        json = json.trim();

        // Verifica graffe di apertura e chiusura
        if (!json.startsWith("{") || !json.endsWith("}")) {
            throw new JsonSyntaxException(
                "Il JSON deve iniziare con { e finire con }");
        }

        // Rimuove graffe
        String contenuto = json.substring(1, json.length() - 1).trim();

        // Se vuoto, e' valido
        if (contenuto.isEmpty()) {
            System.out.println("JSON vuoto valido: {}");
            return;
        }

        // Split per coppie chiave:valore
        String[] coppie = contenuto.split(",");

        for (String coppia : coppie) {
            coppia = coppia.trim();

            // Verifica presenza di :
            if (!coppia.contains(":")) {
                throw new JsonSyntaxException(
                    "Manca ':' nella coppia chiave:valore");
            }

            // Split chiave:valore
            String[] parti = coppia.split(":", 2);
            String chiave = parti[0].trim();
            String valore = parti[1].trim();

            // Valida chiave (deve essere tra virgolette)
            if (!chiave.startsWith("\"") || !chiave.endsWith("\"")) {
                throw new JsonKeyException(chiave);
            }

            // Rimuove virgolette dalla chiave
            chiave = chiave.substring(1, chiave.length() - 1);

            // Valida valore (stringa tra virgolette o numero)
            if (!isValoreValido(valore)) {
                throw new JsonValueException(valore);
            }

            System.out.println("Coppia valida -> " + chiave +
                             " = " + valore);
        }

        System.out.println("JSON valido!");
    }

    private static boolean isValoreValido(String valore) {
        // Valore valido se: stringa tra virgolette, numero,
        // true, false, null
        if (valore.startsWith("\"") && valore.endsWith("\"")) {
            return true; // Stringa
        }
        if (valore.equals("true") || valore.equals("false") ||
            valore.equals("null")) {
            return true; // Boolean o null
        }
        try {
            Double.parseDouble(valore);
            return true; // Numero
        } catch (NumberFormatException e) {
            return false;
        }
    }

    public static void main(String[] args) {
        String[] testJson = {
            "{\"nome\":\"Mario\", \"eta\":30}",
            "{\"valido\":true, \"count\":42}",
            "{\"errore\":senza_virgolette}",
            "{chiave_senza_virgolette:\"valore\"}",
            "{\"completo\":\"ok\", senza_due_punti}",
            "non inizia con graffa"
        };

        for (String json : testJson) {
            try {
                System.out.println("\n=== Test JSON ===");
                System.out.println("Input: " + json);
                parseJson(json);

            } catch (JsonSyntaxException e) {
                System.out.println("ERRORE SINTASSI: " + e.getMessage());

            } catch (JsonKeyException e) {
                System.out.println("ERRORE CHIAVE: " + e.getMessage());
                System.out.println("Le chiavi devono essere tra " +
                                 "virgolette doppie");

            } catch (JsonValueException e) {
                System.out.println("ERRORE VALORE: " + e.getMessage());
                System.out.println("I valori stringa devono essere " +
                                 "tra virgolette");
            }
        }
    }
}
\end{lstlisting}

\textbf{Spiegazione:} Il parser verifica la sintassi JSON base con tre tipi di eccezioni: JsonSyntaxException per errori strutturali (graffe, due punti), JsonKeyException per chiavi non tra virgolette, JsonValueException per valori non validi. Ogni eccezione fornisce informazioni specifiche sull'errore.

\section{Capitolo 4: ArrayList}

\subsection{Esercizio 4.1: Lista della spesa}

\begin{lstlisting}
import java.util.ArrayList;
import java.util.Scanner;

public class ListaSpesa {
    private static ArrayList<String> spesa = new ArrayList<>();
    private static Scanner scanner = new Scanner(System.in);

    public static void main(String[] args) {
        boolean continua = true;

        while (continua) {
            mostraMenu();
            int scelta = scanner.nextInt();
            scanner.nextLine(); // Consuma newline

            switch (scelta) {
                case 1:
                    aggiungiProdotto();
                    break;
                case 2:
                    visualizzaLista();
                    break;
                case 3:
                    rimuoviProdotto();
                    break;
                case 4:
                    verificaProdotto();
                    break;
                case 5:
                    continua = false;
                    System.out.println("Buona spesa!");
                    break;
                default:
                    System.out.println("Scelta non valida");
            }
        }
        scanner.close();
    }

    private static void mostraMenu() {
        System.out.println("\n=== LISTA DELLA SPESA ===");
        System.out.println("1. Aggiungi prodotto");
        System.out.println("2. Visualizza lista");
        System.out.println("3. Rimuovi prodotto");
        System.out.println("4. Verifica prodotto");
        System.out.println("5. Esci");
        System.out.print("Scelta: ");
    }

    private static void aggiungiProdotto() {
        System.out.print("Nome prodotto: ");
        String prodotto = scanner.nextLine();

        // Verifica se gia' presente
        if (spesa.contains(prodotto)) {
            System.out.println("Prodotto gia' presente nella lista!");
        } else {
            spesa.add(prodotto);
            System.out.println("Prodotto aggiunto!");
        }
    }

    private static void visualizzaLista() {
        if (spesa.isEmpty()) {
            System.out.println("Lista vuota");
            return;
        }

        System.out.println("\n--- LISTA DELLA SPESA ---");
        for (int i = 0; i < spesa.size(); i++) {
            System.out.println((i + 1) + ". " + spesa.get(i));
        }
        System.out.println("Totale prodotti: " + spesa.size());
    }

    private static void rimuoviProdotto() {
        if (spesa.isEmpty()) {
            System.out.println("Lista vuota");
            return;
        }

        visualizzaLista();
        System.out.print("Numero prodotto da rimuovere: ");
        int indice = scanner.nextInt() - 1;
        scanner.nextLine();

        if (indice >= 0 && indice < spesa.size()) {
            String rimosso = spesa.remove(indice);
            System.out.println("Rimosso: " + rimosso);
        } else {
            System.out.println("Numero non valido");
        }
    }

    private static void verificaProdotto() {
        System.out.print("Prodotto da cercare: ");
        String prodotto = scanner.nextLine();

        if (spesa.contains(prodotto)) {
            int posizione = spesa.indexOf(prodotto) + 1;
            System.out.println("'" + prodotto + "' e' presente " +
                             "(posizione " + posizione + ")");
        } else {
            System.out.println("'" + prodotto + "' non e' nella lista");
        }
    }
}
\end{lstlisting}

\textbf{Spiegazione:} Il programma gestisce una lista della spesa con menu interattivo, permettendo di aggiungere prodotti (controllando duplicati con contains), visualizzarli, rimuoverli per indice e verificare la presenza di un prodotto.

\subsection{Esercizio 4.2: Temperature settimanali}

\begin{lstlisting}
import java.util.ArrayList;
import java.util.Scanner;

public class TemperatureSettimanali {
    public static void main(String[] args) {
        ArrayList<Double> temperature = new ArrayList<>();
        Scanner scanner = new Scanner(System.in);

        String[] giorni = {"Lunedi", "Martedi", "Mercoledi", "Giovedi",
                          "Venerdi", "Sabato", "Domenica"};

        // Inserimento temperature
        System.out.println("=== TEMPERATURE MASSIME SETTIMANALI ===");
        for (String giorno : giorni) {
            System.out.print(giorno + ": ");
            double temp = scanner.nextDouble();
            temperature.add(temp);
        }

        // Calcolo media
        double somma = 0;
        for (double temp : temperature) {
            somma += temp;
        }
        double media = somma / temperature.size();

        // Trova massima e minima
        double massima = temperature.get(0);
        double minima = temperature.get(0);
        int giornoMax = 0;
        int giornoMin = 0;

        for (int i = 1; i < temperature.size(); i++) {
            double temp = temperature.get(i);
            if (temp > massima) {
                massima = temp;
                giornoMax = i;
            }
            if (temp < minima) {
                minima = temp;
                giornoMin = i;
            }
        }

        // Visualizza risultati
        System.out.println("\n=== ANALISI ===");
        System.out.printf("Temperatura media: %.1f gradi C\n", media);
        System.out.printf("Temperatura massima: %.1f gradi C (%s)\n",
                        massima, giorni[giornoMax]);
        System.out.printf("Temperatura minima: %.1f gradi C (%s)\n",
                        minima, giorni[giornoMin]);

        // Giorni sopra la media
        System.out.println("\nGiorni sopra la media:");
        for (int i = 0; i < temperature.size(); i++) {
            if (temperature.get(i) > media) {
                System.out.printf("- %s: %.1f gradi C\n",
                                giorni[i], temperature.get(i));
            }
        }

        scanner.close();
    }
}
\end{lstlisting}

\textbf{Spiegazione:} Il programma memorizza 7 temperature in un ArrayList, calcola la media sommando tutti i valori, trova massima e minima iterando sulla lista e visualizza i giorni sopra la media confrontando ogni elemento con la media calcolata.

\subsection{Esercizio 4.3: Rimozione duplicati}

\begin{lstlisting}
import java.util.ArrayList;

public class RimuoviDuplicati {

    public static ArrayList<String> rimuoviDuplicati(
            ArrayList<String> listaOriginale) {

        ArrayList<String> listaSenzaDuplicati = new ArrayList<>();

        // Itera sulla lista originale
        for (String elemento : listaOriginale) {
            // Aggiunge solo se non gia' presente nella nuova lista
            if (!listaSenzaDuplicati.contains(elemento)) {
                listaSenzaDuplicati.add(elemento);
            }
        }

        return listaSenzaDuplicati;
    }

    public static void main(String[] args) {
        // Test con lista contenente duplicati
        ArrayList<String> frutti = new ArrayList<>();
        frutti.add("Mela");
        frutti.add("Banana");
        frutti.add("Mela");
        frutti.add("Arancia");
        frutti.add("Banana");
        frutti.add("Kiwi");
        frutti.add("Mela");
        frutti.add("Arancia");

        System.out.println("=== LISTA ORIGINALE ===");
        System.out.println(frutti);
        System.out.println("Dimensione: " + frutti.size());

        ArrayList<String> senzaDuplicati = rimuoviDuplicati(frutti);

        System.out.println("\n=== LISTA SENZA DUPLICATI ===");
        System.out.println(senzaDuplicati);
        System.out.println("Dimensione: " + senzaDuplicati.size());

        // Verifica ordine mantenuto
        System.out.println("\nOrdine prima apparizione mantenuto: " +
                         senzaDuplicati.get(0).equals("Mela"));
    }
}
\end{lstlisting}

\textbf{Spiegazione:} Il metodo rimuoviDuplicati() crea un nuovo ArrayList e aggiunge ogni elemento solo se non è già presente (verificato con contains). Questo mantiene l'ordine di prima apparizione degli elementi.

\subsection{Esercizio 4.4-4.8}

Per motivi di spazio, le soluzioni complete degli esercizi 4.4 (Gestione biblioteca), 4.5 (Registro voti), 4.6 (Playlist musicale), 4.7 (Sistema prenotazioni cinema) e 4.8 (Social network semplificato) seguono lo stesso approccio degli esercizi precedenti: creano classi personalizzate, le gestiscono in ArrayList e implementano le funzionalità richieste usando i metodi add(), remove(), get(), contains() e iterazioni con for/for-each.
