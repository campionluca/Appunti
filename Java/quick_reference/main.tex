\documentclass[a4paper,11pt]{article}
\usepackage[italian]{babel}
\usepackage{geometry}
\usepackage{tcolorbox}
\usepackage{hyperref}
\geometry{margin=2cm}

\title{Java Quick Reference Cards}
\author{Prof. Luca Campion}
\date{\today}

\begin{document}
\maketitle

\newtcolorbox{qrcard}{colback=blue!5,colframe=blue!60,title=Capitolo}

\begin{qrcard}[title=00 — Classi, Oggetti, Ereditarietà]
Concetti chiave: classe, oggetto, attributi, metodi, costruttori, incapsulamento, ereditarietà, polimorfismo.\newline
Tipi di relazioni: is-a (ereditarietà), has-a (composizione/associazione).
\end{qrcard}

\begin{qrcard}[title=01 — Stream e Buffer]
I/O di base: InputStream/OutputStream, Reader/Writer, Buffering per performance.\newline
Try-with-resources per gestione automatica.
\end{qrcard}

\begin{qrcard}[title=02 — Interfacce e Classi Astratte]
Interfacce (contratti), classi astratte (struttura parziale).\newline
Default methods (Java 8), multiple inheritance via interfacce.
\end{qrcard}

\begin{qrcard}[title=03 — Eccezioni]
Checked vs Unchecked, try-catch-finally, throws/throw.\newline
Custom exceptions, best practices di gestione errori.
\end{qrcard}

\begin{qrcard}[title=04 — ArrayList]
Liste dinamiche, operazioni: add, remove, get, size.\newline
Iterazione: for, foreach, Iterator.
\end{qrcard}

\begin{qrcard}[title=05 — Interfacce Grafiche]
Swing: JFrame, JPanel, Layouts, Eventi (ActionListener).\newline
Separazione View/Controller.
\end{qrcard}

\begin{qrcard}[title=06 — Model View Controller]
Ruoli: Model (stato/logica), View (presentazione), Controller (coordinatore).\newline
Eventi e notifiche, testabilità.
\end{qrcard}

\begin{qrcard}[title=07 — Lambda Expressions]
Interfacce funzionali: Predicate, Comparator, Runnable.\newline
Sintassi lambda, method references, Stream API.
\end{qrcard}

\begin{qrcard}[title=08 — Esercizi]
Pratica guidata su contenuti capitoli 00–07 con soluzioni.\newline
Focus: applicazione integrata.
\end{qrcard}

\begin{qrcard}[title=09 — Generics]
Tipi parametrici, wildcards (extends/super), type erasure.\newline
Classi/metodi generici, best practices.
\end{qrcard}

\begin{qrcard}[title=10 — Collections Framework]
List/Set/Map/Queue, algoritmi Collections, Comparable/Comparator.\newline
Iterazione e prestazioni.
\end{qrcard}

\begin{qrcard}[title=11 — Multithreading]
Thread, Runnable, sincronizzazione, ExecutorService, concorrency utilities.\newline
Race conditions e deadlock.
\end{qrcard}

\begin{qrcard}[title=12 — Design Patterns]
Creational: Singleton, Factory, Builder.\newline
Structural/Behavioral: Adapter, Observer, Strategy.
\end{qrcard}

\begin{qrcard}[title=13 — Stream API]
Pipeline: filter, map, sorted, collect.\newline
Optional e parallel streams.
\end{qrcard}

\end{document}

