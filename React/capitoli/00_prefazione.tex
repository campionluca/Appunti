\chapter{Prefazione}

\section{Benvenuti nel Mondo di React}

React è una delle librerie JavaScript più popolari e influenti per la costruzione di interfacce utente moderne. Nata nei laboratori di Facebook (ora Meta) nel 2011 e rilasciata come progetto open source nel 2013, React ha rivoluzionato il modo in cui sviluppiamo applicazioni web, introducendo concetti innovativi che hanno cambiato il panorama dello sviluppo frontend.

\subsection{La Storia di React}

React è stato creato da Jordan Walke, un ingegnere software di Facebook, per risolvere problemi specifici che l'azienda stava affrontando con le sue applicazioni web sempre più complesse. Il News Feed di Facebook richiedeva aggiornamenti dinamici continui, e i pattern tradizionali di manipolazione del DOM stavano diventando difficili da gestire e mantenere.

\textbf{Timeline Storica:}

\begin{itemize}
    \item \textbf{2011}: Jordan Walke crea il primo prototipo di React chiamato "FaxJS"
    \item \textbf{2012}: React viene utilizzato internamente per Instagram
    \item \textbf{Maggio 2013}: React viene rilasciato come open source alla JSConf US
    \item \textbf{2014}: Introduzione del Flux pattern per la gestione dello stato
    \item \textbf{2015}: React Native viene rilasciato, portando React allo sviluppo mobile
    \item \textbf{2016}: React Fiber viene annunciato (nuova architettura di rendering)
    \item \textbf{2017}: React 16 viene rilasciato con supporto completo per Fiber
    \item \textbf{2019}: React Hooks vengono introdotti in React 16.8
    \item \textbf{2020}: Concurrent Mode (sperimentale) viene presentato
    \item \textbf{2022}: React 18 introduce Concurrent Rendering, Suspense migliorato
    \item \textbf{2024}: React 19 introduce Server Components e Actions
\end{itemize}

\subsection{La Rivoluzione React}

Prima di React, lo sviluppo frontend seguiva principalmente due approcci:

\textbf{1. Manipolazione Diretta del DOM:}

\begin{lstlisting}[language=JavaScript, caption=Approccio tradizionale con jQuery]
// Codice jQuery per aggiornare l'interfaccia
$('#counter').text('0');

$('#increment').click(function() {
    var currentValue = parseInt($('#counter').text());
    $('#counter').text(currentValue + 1);

    // Gestione manuale della logica di visualizzazione
    if (currentValue + 1 > 10) {
        $('#counter').addClass('high-value');
    }

    // Aggiornamento di altri elementi correlati
    $('#status').text('Counter incrementato');
});
\end{lstlisting}

\textbf{2. Template Engine con Two-Way Binding:}

\begin{lstlisting}[language=JavaScript, caption=Approccio AngularJS]
// AngularJS controller
app.controller('CounterController', function($scope) {
    $scope.counter = 0;

    $scope.increment = function() {
        $scope.counter++;
    };
});
\end{lstlisting}

\textbf{La Soluzione React:}

React ha introdotto un approccio completamente nuovo basato su componenti, dichiaratività e unidirezionalità dei dati:

\begin{lstlisting}[language=JavaScript, caption=Approccio React moderno]
import { useState } from 'react';

function Counter() {
    const [count, setCount] = useState(0);

    return (
        <div>
            <p className={count > 10 ? 'high-value' : ''}>
                Contatore: {count}
            </p>
            <button onClick={() => setCount(count + 1)}>
                Incrementa
            </button>
            <p>Stato: Counter {count > 0 ? 'incrementato' : 'iniziale'}</p>
        </div>
    );
}
\end{lstlisting}

\section{Perché Scegliere React?}

\subsection{1. Component-Based Architecture}

React promuove la costruzione di UI come composizione di componenti riutilizzabili e indipendenti. Ogni componente gestisce il proprio stato e può essere combinato per creare interfacce complesse.

\textbf{Vantaggi:}

\begin{itemize}
    \item \textbf{Riutilizzabilità}: Scrivi una volta, usa ovunque
    \item \textbf{Manutenibilità}: Ogni componente è isolato e testabile
    \item \textbf{Scalabilità}: Facile aggiungere nuove funzionalità
    \item \textbf{Collaborazione}: Team diversi possono lavorare su componenti diversi
\end{itemize}

\begin{lstlisting}[language=JavaScript, caption=Esempio di composizione di componenti]
// Componente Button riutilizzabile
function Button({ onClick, children, variant = 'primary' }) {
    return (
        <button
            className={`btn btn-${variant}`}
            onClick={onClick}
        >
            {children}
        </button>
    );
}

// Componente Card riutilizzabile
function Card({ title, description, onAction }) {
    return (
        <div className="card">
            <h3>{title}</h3>
            <p>{description}</p>
            <Button onClick={onAction} variant="secondary">
                Leggi di più
            </Button>
        </div>
    );
}

// Composizione per creare una dashboard
function Dashboard() {
    const handleCardAction = (id) => {
        console.log(`Azione su card ${id}`);
    };

    return (
        <div className="dashboard">
            <h1>Dashboard</h1>
            <div className="cards-grid">
                <Card
                    title="Vendite"
                    description="Visualizza le vendite mensili"
                    onAction={() => handleCardAction('sales')}
                />
                <Card
                    title="Utenti"
                    description="Gestisci gli utenti registrati"
                    onAction={() => handleCardAction('users')}
                />
                <Card
                    title="Analytics"
                    description="Analizza le metriche"
                    onAction={() => handleCardAction('analytics')}
                />
            </div>
        </div>
    );
}
\end{lstlisting}

\subsection{2. Virtual DOM e Performance}

Il Virtual DOM è una delle innovazioni chiave di React. Invece di manipolare direttamente il DOM (operazione costosa), React mantiene una rappresentazione virtuale leggera dell'UI in memoria.

\textbf{Come Funziona:}

\begin{enumerate}
    \item Quando lo stato cambia, React crea un nuovo Virtual DOM tree
    \item React confronta il nuovo tree con quello precedente (diffing)
    \item React calcola il set minimo di cambiamenti necessari (reconciliation)
    \item React applica solo questi cambiamenti al DOM reale (batch update)
\end{enumerate}

\textbf{Vantaggi in Termini di Performance:}

\begin{itemize}
    \item \textbf{Aggiornamenti Efficienti}: Solo le parti modificate vengono aggiornate
    \item \textbf{Batch Updates}: Più aggiornamenti vengono raggruppati
    \item \textbf{Predictable}: Il rendering è prevedibile e deterministico
\end{itemize}

\begin{lstlisting}[language=JavaScript, caption=Virtual DOM in azione]
function TodoList() {
    const [todos, setTodos] = useState([
        { id: 1, text: 'Imparare React', completed: false },
        { id: 2, text: 'Costruire un progetto', completed: false },
        { id: 3, text: 'Deploy in produzione', completed: false }
    ]);

    // Quando toggli un todo, React:
    // 1. Crea un nuovo Virtual DOM
    // 2. Confronta con il precedente
    // 3. Aggiorna solo il singolo elemento modificato nel DOM reale
    const toggleTodo = (id) => {
        setTodos(todos.map(todo =>
            todo.id === id
                ? { ...todo, completed: !todo.completed }
                : todo
        ));
    };

    return (
        <ul>
            {todos.map(todo => (
                <li
                    key={todo.id}
                    onClick={() => toggleTodo(todo.id)}
                    style={{
                        textDecoration: todo.completed ? 'line-through' : 'none'
                    }}
                >
                    {todo.text}
                </li>
            ))}
        </ul>
    );
}
\end{lstlisting}

\subsection{3. Approccio Dichiarativo}

React permette di descrivere \textit{cosa} vuoi che l'interfaccia mostri, non \textit{come} manipolare il DOM per ottenerlo. Questo rende il codice più leggibile, prevedibile e facile da debuggare.

\textbf{Imperativo vs Dichiarativo:}

\begin{lstlisting}[language=JavaScript, caption=Approccio Imperativo (Vanilla JS)]
// Descrive COME fare qualcosa
const form = document.getElementById('login-form');
const errorDiv = document.getElementById('error');
const submitBtn = document.getElementById('submit');

form.addEventListener('submit', function(e) {
    e.preventDefault();

    // Rimuovi errori precedenti
    errorDiv.innerHTML = '';
    errorDiv.classList.remove('visible');

    // Disabilita il bottone
    submitBtn.disabled = true;
    submitBtn.textContent = 'Caricamento...';

    // Fai la richiesta
    fetch('/api/login', { method: 'POST', body: new FormData(form) })
        .then(response => {
            if (!response.ok) {
                // Mostra errore
                errorDiv.textContent = 'Login fallito';
                errorDiv.classList.add('visible');
                // Riabilita bottone
                submitBtn.disabled = false;
                submitBtn.textContent = 'Login';
            } else {
                // Redirect
                window.location.href = '/dashboard';
            }
        });
});
\end{lstlisting}

\begin{lstlisting}[language=JavaScript, caption=Approccio Dichiarativo (React)]
// Descrive COSA mostrare
function LoginForm() {
    const [isLoading, setIsLoading] = useState(false);
    const [error, setError] = useState(null);

    const handleSubmit = async (e) => {
        e.preventDefault();
        setIsLoading(true);
        setError(null);

        try {
            const response = await fetch('/api/login', {
                method: 'POST',
                body: new FormData(e.target)
            });

            if (!response.ok) {
                setError('Login fallito');
            } else {
                window.location.href = '/dashboard';
            }
        } finally {
            setIsLoading(false);
        }
    };

    // React si occupa di aggiornare il DOM basandosi sullo stato
    return (
        <form onSubmit={handleSubmit}>
            {error && <div className="error visible">{error}</div>}

            <input type="email" name="email" required />
            <input type="password" name="password" required />

            <button type="submit" disabled={isLoading}>
                {isLoading ? 'Caricamento...' : 'Login'}
            </button>
        </form>
    );
}
\end{lstlisting}

\subsection{4. Ecosistema Ricco e Community Attiva}

React beneficia di un ecosistema vasto e maturo:

\textbf{Librerie e Tool Popolari:}

\begin{itemize}
    \item \textbf{State Management}: Redux, MobX, Zustand, Jotai, Recoil
    \item \textbf{Routing}: React Router, TanStack Router, Wouter
    \item \textbf{Data Fetching}: TanStack Query, SWR, Apollo Client
    \item \textbf{Forms}: React Hook Form, Formik, Final Form
    \item \textbf{UI Components}: Material-UI, Ant Design, Chakra UI, Shadcn/ui
    \item \textbf{Animation}: Framer Motion, React Spring, GSAP
    \item \textbf{Testing}: React Testing Library, Jest, Vitest
    \item \textbf{Dev Tools}: React DevTools, Storybook, ESLint, Prettier
    \item \textbf{Frameworks}: Next.js, Remix, Gatsby
    \item \textbf{Mobile}: React Native, Expo
\end{itemize}

\subsection{5. Unidirezionalità dei Dati}

React implementa un flusso di dati unidirezionale (one-way data flow), che rende il comportamento dell'applicazione più prevedibile e facile da debuggare.

\begin{lstlisting}[language=JavaScript, caption=Flusso dati unidirezionale]
// I dati fluiscono sempre dall'alto verso il basso
function App() {
    const [user, setUser] = useState({
        name: 'Mario Rossi',
        role: 'Admin'
    });

    // App passa dati ai componenti figli
    return (
        <div>
            <Header user={user} />
            <Dashboard user={user} onUpdateUser={setUser} />
        </div>
    );
}

function Header({ user }) {
    // Header riceve dati e li visualizza
    // Non può modificarli direttamente
    return (
        <header>
            <h1>Benvenuto, {user.name}</h1>
            <span>Ruolo: {user.role}</span>
        </header>
    );
}

function Dashboard({ user, onUpdateUser }) {
    // Dashboard può richiedere modifiche tramite callback
    const handleRoleChange = () => {
        onUpdateUser({
            ...user,
            role: user.role === 'Admin' ? 'User' : 'Admin'
        });
    };

    return (
        <div>
            <p>Dashboard per {user.name}</p>
            <button onClick={handleRoleChange}>
                Cambia Ruolo
            </button>
        </div>
    );
}
\end{lstlisting}

\subsection{6. Learn Once, Write Anywhere}

Con React puoi costruire:

\begin{itemize}
    \item \textbf{Web Apps}: Single Page Applications con React
    \item \textbf{Mobile Apps}: iOS e Android con React Native
    \item \textbf{Desktop Apps}: Con Electron + React
    \item \textbf{Static Sites}: Con Gatsby o Next.js
    \item \textbf{Server-Side Rendering}: Con Next.js o Remix
    \item \textbf{VR/AR}: Con React 360 o React Three Fiber
\end{itemize}

\begin{lstlisting}[language=JavaScript, caption=Stesso componente per Web e Mobile]
// components/Button.js - Funziona sia su web che mobile
import { TouchableOpacity, Text } from 'react-native'; // Mobile
// import './Button.css'; // Web

function Button({ onPress, title, style }) {
    // Logica condivisa
    const handlePress = () => {
        console.log('Button pressed');
        onPress?.();
    };

    // React Native (Mobile)
    return (
        <TouchableOpacity
            onPress={handlePress}
            style={[styles.button, style]}
        >
            <Text style={styles.text}>{title}</Text>
        </TouchableOpacity>
    );

    // React DOM (Web)
    // return (
    //     <button onClick={handlePress} className="button" style={style}>
    //         {title}
    //     </button>
    // );
}
\end{lstlisting}

\section{Quando Usare React}

\subsection{React è Ideale Per:}

\textbf{1. Single Page Applications (SPA):}

Applicazioni web complesse dove l'utente naviga senza ricaricare la pagina.

\begin{lstlisting}[language=JavaScript, caption=SPA con routing]
import { BrowserRouter, Routes, Route } from 'react-router-dom';

function App() {
    return (
        <BrowserRouter>
            <Navigation />
            <Routes>
                <Route path="/" element={<Home />} />
                <Route path="/products" element={<Products />} />
                <Route path="/products/:id" element={<ProductDetail />} />
                <Route path="/cart" element={<Cart />} />
                <Route path="/checkout" element={<Checkout />} />
            </Routes>
        </BrowserRouter>
    );
}
\end{lstlisting}

\textbf{2. Dashboard e Admin Panels:}

Interfacce con molti componenti interattivi e aggiornamenti in tempo reale.

\begin{lstlisting}[language=JavaScript, caption=Dashboard interattiva]
function AdminDashboard() {
    const [stats, setStats] = useState(null);
    const [liveUsers, setLiveUsers] = useState([]);

    // Polling per aggiornamenti in tempo reale
    useEffect(() => {
        const interval = setInterval(async () => {
            const data = await fetchLiveStats();
            setStats(data.stats);
            setLiveUsers(data.users);
        }, 5000);

        return () => clearInterval(interval);
    }, []);

    return (
        <div className="dashboard">
            <StatsCards stats={stats} />
            <LiveUsersTable users={liveUsers} />
            <SalesChart data={stats?.sales} />
            <RecentOrders limit={10} />
        </div>
    );
}
\end{lstlisting}

\textbf{3. Applicazioni con UI Complesse:}

Interfacce con molti stati e interazioni.

\begin{lstlisting}[language=JavaScript, caption=UI complessa con drag and drop]
import { DndContext, closestCenter } from '@dnd-kit/core';

function KanbanBoard() {
    const [columns, setColumns] = useState({
        todo: ['Task 1', 'Task 2'],
        inProgress: ['Task 3'],
        done: ['Task 4', 'Task 5']
    });

    const handleDragEnd = (event) => {
        const { active, over } = event;

        if (!over) return;

        // Logica per spostare task tra colonne
        const sourceColumn = findColumn(active.id);
        const destColumn = over.id;

        moveTask(active.id, sourceColumn, destColumn);
    };

    return (
        <DndContext onDragEnd={handleDragEnd} collisionDetection={closestCenter}>
            <div className="kanban">
                <Column id="todo" title="To Do" tasks={columns.todo} />
                <Column id="inProgress" title="In Progress" tasks={columns.inProgress} />
                <Column id="done" title="Done" tasks={columns.done} />
            </div>
        </DndContext>
    );
}
\end{lstlisting}

\textbf{4. Applicazioni Real-Time:}

Chat, collaboration tools, live updates.

\begin{lstlisting}[language=JavaScript, caption=Chat real-time con WebSocket]
function ChatRoom({ roomId }) {
    const [messages, setMessages] = useState([]);
    const [ws, setWs] = useState(null);

    useEffect(() => {
        const websocket = new WebSocket(`ws://api.example.com/chat/${roomId}`);

        websocket.onmessage = (event) => {
            const message = JSON.parse(event.data);
            setMessages(prev => [...prev, message]);
        };

        setWs(websocket);

        return () => websocket.close();
    }, [roomId]);

    const sendMessage = (text) => {
        ws.send(JSON.stringify({ text, roomId }));
    };

    return (
        <div className="chat">
            <MessageList messages={messages} />
            <MessageInput onSend={sendMessage} />
        </div>
    );
}
\end{lstlisting}

\subsection{Quando Considerare Alternative:}

\textbf{1. Siti Statici Semplici:}

Per siti con contenuto prevalentemente statico, HTML/CSS puri o generatori statici potrebbero essere più appropriati.

\textbf{2. SEO-Critical con Vincoli di Performance:}

Per siti dove ogni millisecondo conta e il SEO è critico, soluzioni SSR/SSG pure potrebbero essere migliori (comunque disponibili con Next.js).

\textbf{3. Applicazioni Molto Semplici:}

Per form semplici o piccole interazioni, vanilla JavaScript potrebbe essere sufficiente.

\section{L'Ecosistema React nel 2024-2025}

\subsection{React Server Components}

Una delle innovazioni più recenti che sta cambiando il modo di costruire applicazioni React:

\begin{lstlisting}[language=JavaScript, caption=Server Component esempio]
// app/products/page.js - Server Component (Next.js 13+)
async function ProductsPage() {
    // Questo codice gira SOLO sul server
    // Non aumenta il bundle JavaScript del client
    const products = await db.products.findMany();

    return (
        <div>
            <h1>I Nostri Prodotti</h1>
            <ProductList products={products} />
        </div>
    );
}

// components/ProductList.js - Client Component
'use client'; // Direttiva per Client Component

import { useState } from 'react';

function ProductList({ products }) {
    const [filter, setFilter] = useState('all');

    // Questo codice gira sul client e può usare interattività
    return (
        <div>
            <FilterBar value={filter} onChange={setFilter} />
            {products
                .filter(p => filter === 'all' || p.category === filter)
                .map(product => (
                    <ProductCard key={product.id} product={product} />
                ))
            }
        </div>
    );
}
\end{lstlisting}

\subsection{React Compiler (in arrivo)}

Un compilatore che ottimizza automaticamente le performance del codice React, eliminando la necessità di useMemo e useCallback manuali.

\subsection{Concurrent Features}

React 18+ introduce rendering concorrente che migliora la responsiveness:

\begin{lstlisting}[language=JavaScript, caption=Concurrent Features]
import { useTransition, useDeferredValue } from 'react';

function SearchPage() {
    const [query, setQuery] = useState('');
    const [isPending, startTransition] = useTransition();

    // Gli aggiornamenti nella transition sono interrompibili
    const handleChange = (e) => {
        startTransition(() => {
            setQuery(e.target.value);
        });
    };

    return (
        <div>
            <input onChange={handleChange} />
            {isPending && <Spinner />}
            <SearchResults query={query} />
        </div>
    );
}
\end{lstlisting}

\section{Prerequisiti per Questo Libro}

\subsection{Conoscenze Richieste}

Per trarre il massimo da questo libro, dovresti avere:

\begin{itemize}
    \item \textbf{JavaScript (ES6+)}: Arrow functions, destructuring, spread operator, modules
    \item \textbf{HTML/CSS}: Comprensione di base del DOM e dello styling
    \item \textbf{NPM/Yarn}: Package manager e gestione dipendenze
    \item \textbf{Concetti di Programmazione}: Funzioni, oggetti, array, asincronicità
\end{itemize}

\textbf{Ripasso JavaScript Essenziale:}

\begin{lstlisting}[language=JavaScript, caption=Concetti JavaScript per React]
// 1. Arrow Functions
const greet = (name) => `Hello, ${name}`;
const add = (a, b) => a + b;

// 2. Destructuring
const user = { name: 'Mario', age: 30 };
const { name, age } = user;

const numbers = [1, 2, 3];
const [first, second] = numbers;

// 3. Spread Operator
const newUser = { ...user, city: 'Roma' };
const newNumbers = [...numbers, 4, 5];

// 4. Template Literals
const message = `${name} ha ${age} anni`;

// 5. Array Methods
const doubled = numbers.map(n => n * 2);
const evens = numbers.filter(n => n % 2 === 0);
const sum = numbers.reduce((acc, n) => acc + n, 0);

// 6. Async/Await
const fetchData = async () => {
    try {
        const response = await fetch('/api/data');
        const data = await response.json();
        return data;
    } catch (error) {
        console.error(error);
    }
};

// 7. Modules
export const API_URL = 'https://api.example.com';
export default function Component() { /* ... */ }

// 8. Optional Chaining
const userName = user?.profile?.name ?? 'Guest';

// 9. Nullish Coalescing
const port = config.port ?? 3000;
\end{lstlisting}

\subsection{Setup dell'Ambiente di Sviluppo}

Avrai bisogno di:

\begin{itemize}
    \item \textbf{Node.js} (v18 o superiore): Runtime JavaScript
    \item \textbf{Editor}: VS Code (raccomandato) con estensioni:
    \begin{itemize}
        \item ES7+ React/Redux/React-Native snippets
        \item ESLint
        \item Prettier
        \item Auto Rename Tag
    \end{itemize}
    \item \textbf{Browser}: Chrome o Firefox con React DevTools installato
    \item \textbf{Git}: Per version control
\end{itemize}

\section{Struttura del Libro}

Questo libro è organizzato in modo progressivo:

\begin{enumerate}
    \item \textbf{Introduzione a React}: Virtual DOM, setup, primi passi
    \item \textbf{JSX e Componenti}: Sintassi, functional components, composizione
    \item \textbf{Props e State}: Gestione dati, props drilling, controlled components
    \item \textbf{Hooks}: useState, useEffect, custom hooks e pattern avanzati
    \item \textbf{Event Handling}: Gestione eventi e interazioni utente
    \item \textbf{Form}: Form controllati, validazione, librerie
\end{enumerate}

Ogni capitolo include:
\begin{itemize}
    \item Spiegazioni teoriche dettagliate
    \item Esempi di codice completi e funzionanti
    \item Best practices e pattern comuni
    \item Esercizi pratici
    \item Diagrammi e visualizzazioni
\end{itemize}

\section{Filosofia di Apprendimento}

\subsection{Imparare Facendo}

Il modo migliore per imparare React è scrivere codice. Ogni esempio in questo libro è:

\begin{itemize}
    \item \textbf{Funzionante}: Puoi copiarlo e provarlo subito
    \item \textbf{Incrementale}: Costruiamo concetti uno sopra l'altro
    \item \textbf{Pratico}: Esempi basati su casi d'uso reali
    \item \textbf{Commentato}: Spiegazioni inline nel codice
\end{itemize}

\subsection{Best Practices dal Giorno Uno}

Non imparerai solo come far funzionare le cose, ma come farle funzionare \textit{bene}:

\begin{itemize}
    \item Codice pulito e manutenibile
    \item Performance optimization
    \item Accessibilità (a11y)
    \item Testing
    \item Sicurezza
\end{itemize}

\section{Oltre Questo Libro}

\subsection{Risorse per Continuare}

Una volta completato questo libro, ecco dove proseguire:

\textbf{Documentazione Ufficiale:}
\begin{itemize}
    \item \url{https://react.dev} - Nuova documentazione ufficiale
    \item \url{https://react.dev/learn} - Tutorial interattivi
\end{itemize}

\textbf{Framework e Tool:}
\begin{itemize}
    \item Next.js per SSR e SSG
    \item Remix per web apps full-stack
    \item React Native per mobile
\end{itemize}

\textbf{Community:}
\begin{itemize}
    \item Reddit: r/reactjs
    \item Discord: Reactiflux
    \item Twitter: \#ReactJS
    \item Stack Overflow: tag [reactjs]
\end{itemize}

\subsection{Il Futuro di React}

React continua ad evolversi rapidamente:

\begin{itemize}
    \item \textbf{React Forget}: Compilatore automatico per ottimizzazioni
    \item \textbf{Server Components}: Architettura ibrida client-server
    \item \textbf{Concurrent Features}: Rendering più fluido e responsivo
    \item \textbf{Suspense for Data}: Loading states dichiarativi
    \item \textbf{Actions}: Gestione form e mutations migliorata
\end{itemize}

\section{Un Ultimo Pensiero}

React non è solo una libreria da imparare: è una nuova mentalità per pensare alle interfacce utente. I concetti che apprenderai - componenti, immutabilità, composizione, flusso dati unidirezionale - ti serviranno indipendentemente dalle tecnologie che userai in futuro.

Non ti preoccupare se all'inizio qualcosa sembra complesso o contro-intuitivo. React ha una curva di apprendimento, ma una volta che "fa click", diventa uno strumento incredibilmente potente e piacevole da usare.

\textbf{Il viaggio inizia ora. Buon divertimento con React!}

\vspace{1cm}

\begin{lstlisting}[language=JavaScript, caption=Il tuo primo componente React ti aspetta...]
import { useState } from 'react';

function Welcome() {
    const [started, setStarted] = useState(false);

    return (
        <div className="welcome">
            <h1>Benvenuto nel Mondo di React!</h1>
            {!started ? (
                <button onClick={() => setStarted(true)}>
                    Inizia il Viaggio
                </button>
            ) : (
                <p>Sei pronto a costruire cose incredibili! 🚀</p>
            )}
        </div>
    );
}

export default Welcome;
\end{lstlisting}
