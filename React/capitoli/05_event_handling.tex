\chapter{Event Handling}

\section{Eventi in React}

React gestisce gli eventi in modo simile al DOM nativo, ma con alcune differenze importanti. Gli eventi in React sono chiamati "Synthetic Events" (eventi sintetici).

\subsection{Differenze tra Eventi React e DOM}

\begin{lstlisting}[language=JavaScript, caption=React vs DOM eventi]
// ❌ DOM nativo (HTML)
<button onclick="handleClick()">Click me</button>

// ✅ React (JSX)
<button onClick={handleClick}>Click me</button>

// Differenze principali:
// 1. camelCase invece di lowercase (onClick vs onclick)
// 2. Funzione invece di stringa ({handleClick} vs "handleClick()")
// 3. preventDefault() invece di return false
\end{lstlisting}

\begin{center}
\begin{tabular}{|l|l|l|}
\hline
\textbf{Aspetto} & \textbf{DOM} & \textbf{React} \\
\hline
Naming & lowercase & camelCase \\
\hline
Handler & Stringa & Funzione \\
\hline
Prevent default & return false & e.preventDefault() \\
\hline
Event object & Event nativo & SyntheticEvent \\
\hline
\end{tabular}
\end{center}

\section{Synthetic Events}

React wrappa gli eventi nativi del browser in un SyntheticEvent cross-browser.

\subsection{Cos'è un SyntheticEvent?}

\begin{lstlisting}[language=JavaScript, caption=SyntheticEvent structure]
function Component() {
    const handleClick = (event) => {
        // event è un SyntheticEvent
        console.log(event);

        // Proprietà comuni
        console.log(event.type);           // "click"
        console.log(event.target);         // Elemento DOM che ha generato l'evento
        console.log(event.currentTarget);  // Elemento con l'handler
        console.log(event.timeStamp);      // Timestamp
        console.log(event.bubbles);        // true/false

        // Metodi
        event.preventDefault();  // Previeni comportamento default
        event.stopPropagation(); // Ferma propagazione

        // Accesso all'evento nativo
        const nativeEvent = event.nativeEvent;
        console.log(nativeEvent);
    };

    return <button onClick={handleClick}>Click</button>;
}
\end{lstlisting}

\subsection{Event Pooling (React < 17)}

\begin{lstlisting}[language=JavaScript, caption=Event pooling in React 16]
// In React 16 e precedenti, gli eventi venivano "pooled"
function OldComponent() {
    const handleClick = (event) => {
        console.log(event.type); // "click"

        setTimeout(() => {
            // ❌ event è null qui!
            console.log(event.type); // null
        }, 1000);

        // ✅ Soluzione: persist()
        event.persist();
        setTimeout(() => {
            console.log(event.type); // "click"
        }, 1000);
    };

    return <button onClick={handleClick}>Click</button>;
}

// React 17+: Event pooling rimosso!
function NewComponent() {
    const handleClick = (event) => {
        setTimeout(() => {
            // ✅ Funziona in React 17+
            console.log(event.type); // "click"
        }, 1000);
    };

    return <button onClick={handleClick}>Click</button>;
}
\end{lstlisting}

\section{onClick: Eventi Click}

onClick è l'evento più comune in React.

\subsection{onClick Base}

\begin{lstlisting}[language=JavaScript, caption=onClick esempi]
function ClickExamples() {
    // 1. Inline arrow function
    return (
        <button onClick={() => console.log('Clicked!')}>
            Click 1
        </button>
    );

    // 2. Named function
    const handleClick = () => {
        console.log('Clicked!');
    };

    return <button onClick={handleClick}>Click 2</button>;

    // 3. Con parametri
    const handleClickWithParam = (id) => {
        console.log('Clicked item:', id);
    };

    return (
        <button onClick={() => handleClickWithParam(123)}>
            Click 3
        </button>
    );

    // 4. Con event object
    const handleClickWithEvent = (e) => {
        console.log('Clicked:', e.target);
    };

    return <button onClick={handleClickWithEvent}>Click 4</button>;

    // 5. Con event e parametri
    const handleClickBoth = (id, e) => {
        console.log('ID:', id, 'Event:', e);
    };

    return (
        <button onClick={(e) => handleClickBoth(123, e)}>
            Click 5
        </button>
    );
}
\end{lstlisting}

\subsection{onClick Patterns}

\begin{lstlisting}[language=JavaScript, caption=Pattern comuni onClick]
function ClickPatterns() {
    const [count, setCount] = useState(0);
    const [items, setItems] = useState(['A', 'B', 'C']);

    // Pattern 1: Aggiornamento state
    const increment = () => {
        setCount(count + 1);
    };

    // Pattern 2: Toggle boolean
    const [isOpen, setIsOpen] = useState(false);
    const toggle = () => setIsOpen(!isOpen);

    // Pattern 3: Handler con parametro da lista
    const handleItemClick = (item) => {
        console.log('Clicked:', item);
    };

    // Pattern 4: Prevent default
    const handleLinkClick = (e) => {
        e.preventDefault();
        console.log('Link clicked but not followed');
    };

    // Pattern 5: Conditional handler
    const handleConditionalClick = () => {
        if (count < 10) {
            setCount(count + 1);
        } else {
            alert('Limite raggiunto!');
        }
    };

    return (
        <div>
            {/* Pattern 1 */}
            <button onClick={increment}>Count: {count}</button>

            {/* Pattern 2 */}
            <button onClick={toggle}>
                {isOpen ? 'Chiudi' : 'Apri'}
            </button>

            {/* Pattern 3 */}
            {items.map(item => (
                <button key={item} onClick={() => handleItemClick(item)}>
                    {item}
                </button>
            ))}

            {/* Pattern 4 */}
            <a href="https://example.com" onClick={handleLinkClick}>
                Link
            </a>

            {/* Pattern 5 */}
            <button onClick={handleConditionalClick}>
                Incrementa (max 10)
            </button>
        </div>
    );
}
\end{lstlisting}

\subsection{Gestione Click Multipli}

\begin{lstlisting}[language=JavaScript, caption=Double click e timing]
function MultiClickExample() {
    const [clicks, setClicks] = useState(0);
    const [lastClick, setLastClick] = useState(0);

    // Single click
    const handleClick = () => {
        setClicks(prev => prev + 1);
    };

    // Double click
    const handleDoubleClick = () => {
        console.log('Double clicked!');
        setClicks(0); // Reset
    };

    // Debounced click
    const timeoutRef = useRef(null);

    const handleDebouncedClick = () => {
        clearTimeout(timeoutRef.current);

        timeoutRef.current = setTimeout(() => {
            console.log('Clicked (debounced)');
        }, 300);
    };

    // Click con delay detection
    const handleClickWithDelay = () => {
        const now = Date.now();
        const timeSinceLastClick = now - lastClick;

        if (timeSinceLastClick < 300) {
            console.log('Quick click!');
        } else {
            console.log('Slow click!');
        }

        setLastClick(now);
    };

    return (
        <div>
            <button onClick={handleClick} onDoubleClick={handleDoubleClick}>
                Clicks: {clicks} (double click to reset)
            </button>

            <button onClick={handleDebouncedClick}>
                Debounced Click
            </button>

            <button onClick={handleClickWithDelay}>
                Click with timing
            </button>
        </div>
    );
}
\end{lstlisting}

\section{onChange: Eventi Input}

onChange è usato per gestire cambiamenti nei form elements.

\subsection{onChange con Input}

\begin{lstlisting}[language=JavaScript, caption=onChange input types]
function InputExamples() {
    const [text, setText] = useState('');
    const [number, setNumber] = useState(0);
    const [email, setEmail] = useState('');
    const [password, setPassword] = useState('');

    return (
        <div>
            {/* Text input */}
            <input
                type="text"
                value={text}
                onChange={(e) => setText(e.target.value)}
                placeholder="Nome"
            />
            <p>Testo: {text}</p>

            {/* Number input */}
            <input
                type="number"
                value={number}
                onChange={(e) => setNumber(Number(e.target.value))}
                placeholder="Età"
            />
            <p>Numero: {number}</p>

            {/* Email input con validazione */}
            <input
                type="email"
                value={email}
                onChange={(e) => setEmail(e.target.value)}
                placeholder="Email"
            />
            <p style={{ color: email.includes('@') ? 'green' : 'red' }}>
                Email: {email}
            </p>

            {/* Password input */}
            <input
                type="password"
                value={password}
                onChange={(e) => setPassword(e.target.value)}
                placeholder="Password"
            />
            <p>Password length: {password.length}</p>
        </div>
    );
}
\end{lstlisting}

\subsection{onChange con Select}

\begin{lstlisting}[language=JavaScript, caption=onChange select]
function SelectExamples() {
    const [country, setCountry] = useState('');
    const [language, setLanguage] = useState('it');
    const [categories, setCategories] = useState([]);

    return (
        <div>
            {/* Select singolo */}
            <select value={country} onChange={(e) => setCountry(e.target.value)}>
                <option value="">Seleziona paese...</option>
                <option value="IT">Italia</option>
                <option value="US">USA</option>
                <option value="UK">UK</option>
            </select>
            <p>Paese selezionato: {country}</p>

            {/* Select con default */}
            <select value={language} onChange={(e) => setLanguage(e.target.value)}>
                <option value="it">Italiano</option>
                <option value="en">English</option>
                <option value="es">Español</option>
            </select>
            <p>Lingua: {language}</p>

            {/* Select multiplo */}
            <select
                multiple
                value={categories}
                onChange={(e) => {
                    const selected = Array.from(
                        e.target.selectedOptions,
                        option => option.value
                    );
                    setCategories(selected);
                }}
            >
                <option value="tech">Tech</option>
                <option value="sport">Sport</option>
                <option value="music">Musica</option>
                <option value="travel">Viaggi</option>
            </select>
            <p>Categorie: {categories.join(', ')}</p>
        </div>
    );
}
\end{lstlisting}

\subsection{onChange con Checkbox e Radio}

\begin{lstlisting}[language=JavaScript, caption=onChange checkbox/radio]
function CheckboxRadioExamples() {
    const [isChecked, setIsChecked] = useState(false);
    const [gender, setGender] = useState('');
    const [interests, setInterests] = useState([]);

    // Checkbox singola
    const handleCheckboxChange = (e) => {
        setIsChecked(e.target.checked);
    };

    // Radio buttons
    const handleRadioChange = (e) => {
        setGender(e.target.value);
    };

    // Checkbox multiple
    const handleInterestChange = (interest) => {
        setInterests(prev =>
            prev.includes(interest)
                ? prev.filter(i => i !== interest)
                : [...prev, interest]
        );
    };

    return (
        <div>
            {/* Checkbox singola */}
            <label>
                <input
                    type="checkbox"
                    checked={isChecked}
                    onChange={handleCheckboxChange}
                />
                Accetto i termini
            </label>
            <p>Accettato: {isChecked ? 'Sì' : 'No'}</p>

            {/* Radio buttons */}
            <div>
                <label>
                    <input
                        type="radio"
                        name="gender"
                        value="male"
                        checked={gender === 'male'}
                        onChange={handleRadioChange}
                    />
                    Maschio
                </label>
                <label>
                    <input
                        type="radio"
                        name="gender"
                        value="female"
                        checked={gender === 'female'}
                        onChange={handleRadioChange}
                    />
                    Femmina
                </label>
            </div>
            <p>Genere: {gender}</p>

            {/* Checkbox multiple */}
            <div>
                {['Sport', 'Musica', 'Lettura', 'Viaggi'].map(interest => (
                    <label key={interest}>
                        <input
                            type="checkbox"
                            checked={interests.includes(interest)}
                            onChange={() => handleInterestChange(interest)}
                        />
                        {interest}
                    </label>
                ))}
            </div>
            <p>Interessi: {interests.join(', ')}</p>
        </div>
    );
}
\end{lstlisting}

\subsection{onChange con Textarea}

\begin{lstlisting}[language=JavaScript, caption=onChange textarea]
function TextareaExample() {
    const [message, setMessage] = useState('');
    const maxLength = 200;

    const handleChange = (e) => {
        const value = e.target.value;

        // Limita lunghezza
        if (value.length <= maxLength) {
            setMessage(value);
        }
    };

    const remainingChars = maxLength - message.length;

    return (
        <div>
            <textarea
                value={message}
                onChange={handleChange}
                placeholder="Scrivi un messaggio..."
                rows={5}
                cols={50}
            />

            <p>
                Caratteri rimanenti: {remainingChars} / {maxLength}
            </p>

            <p>Righe: {message.split('\n').length}</p>

            <p style={{ color: remainingChars < 20 ? 'red' : 'black' }}>
                {remainingChars < 20 && 'Stai raggiungendo il limite!'}
            </p>
        </div>
    );
}
\end{lstlisting}

\section{Altri Eventi Comuni}

\subsection{onSubmit: Form Submission}

\begin{lstlisting}[language=JavaScript, caption=onSubmit form]
function FormSubmitExample() {
    const [formData, setFormData] = useState({
        username: '',
        email: '',
        password: ''
    });

    const handleChange = (e) => {
        const { name, value } = e.target;
        setFormData(prev => ({
            ...prev,
            [name]: value
        }));
    };

    const handleSubmit = (e) => {
        e.preventDefault(); // ⚠️ IMPORTANTE: Previeni reload pagina

        console.log('Form submitted:', formData);

        // Validazione
        if (!formData.username || !formData.email || !formData.password) {
            alert('Compila tutti i campi!');
            return;
        }

        // Invia dati
        fetch('/api/register', {
            method: 'POST',
            headers: { 'Content-Type': 'application/json' },
            body: JSON.stringify(formData)
        })
        .then(response => response.json())
        .then(data => {
            console.log('Success:', data);
            // Reset form
            setFormData({ username: '', email: '', password: '' });
        });
    };

    return (
        <form onSubmit={handleSubmit}>
            <input
                name="username"
                value={formData.username}
                onChange={handleChange}
                placeholder="Username"
            />

            <input
                name="email"
                type="email"
                value={formData.email}
                onChange={handleChange}
                placeholder="Email"
            />

            <input
                name="password"
                type="password"
                value={formData.password}
                onChange={handleChange}
                placeholder="Password"
            />

            <button type="submit">Registrati</button>
        </form>
    );
}
\end{lstlisting}

\subsection{onFocus e onBlur}

\begin{lstlisting}[language=JavaScript, caption=onFocus e onBlur]
function FocusBlurExample() {
    const [isFocused, setIsFocused] = useState(false);
    const [value, setValue] = useState('');
    const [touched, setTouched] = useState(false);

    const handleFocus = () => {
        setIsFocused(true);
    };

    const handleBlur = () => {
        setIsFocused(false);
        setTouched(true);
    };

    // Validazione solo dopo blur
    const error = touched && value.length < 3
        ? 'Minimo 3 caratteri'
        : '';

    return (
        <div>
            <input
                value={value}
                onChange={(e) => setValue(e.target.value)}
                onFocus={handleFocus}
                onBlur={handleBlur}
                placeholder="Nome"
                style={{
                    borderColor: isFocused ? 'blue' : error ? 'red' : 'gray'
                }}
            />

            <p>Focus: {isFocused ? 'Sì' : 'No'}</p>

            {error && <span style={{ color: 'red' }}>{error}</span>}
        </div>
    );
}
\end{lstlisting}

\subsection{onKeyDown, onKeyUp, onKeyPress}

\begin{lstlisting}[language=JavaScript, caption=Eventi keyboard]
function KeyboardEvents() {
    const [input, setInput] = useState('');
    const [lastKey, setLastKey] = useState('');

    const handleKeyDown = (e) => {
        setLastKey(e.key);

        // Enter per submit
        if (e.key === 'Enter') {
            console.log('Enter pressed!');
        }

        // Escape per cancellare
        if (e.key === 'Escape') {
            setInput('');
        }

        // Ctrl + S per salvare
        if (e.ctrlKey && e.key === 's') {
            e.preventDefault();
            console.log('Saving...');
        }

        // Arrow keys
        if (e.key === 'ArrowUp') {
            console.log('Up arrow');
        }
    };

    const handleKeyPress = (e) => {
        // Solo caratteri alfanumerici
        if (!/[a-zA-Z0-9]/.test(e.key)) {
            e.preventDefault();
        }
    };

    return (
        <div>
            <input
                value={input}
                onChange={(e) => setInput(e.target.value)}
                onKeyDown={handleKeyDown}
                onKeyPress={handleKeyPress}
                placeholder="Prova i tasti..."
            />

            <p>Ultimo tasto: {lastKey}</p>
            <p>Input: {input}</p>

            <ul>
                <li>Enter: Submit</li>
                <li>Escape: Cancella</li>
                <li>Ctrl+S: Salva</li>
            </ul>
        </div>
    );
}
\end{lstlisting}

\subsection{onMouseEnter e onMouseLeave}

\begin{lstlisting}[language=JavaScript, caption=Eventi mouse hover]
function HoverExample() {
    const [isHovering, setIsHovering] = useState(false);
    const [hoverCount, setHoverCount] = useState(0);
    const [position, setPosition] = useState({ x: 0, y: 0 });

    const handleMouseEnter = () => {
        setIsHovering(true);
        setHoverCount(prev => prev + 1);
    };

    const handleMouseLeave = () => {
        setIsHovering(false);
    };

    const handleMouseMove = (e) => {
        // Posizione relativa all'elemento
        const rect = e.currentTarget.getBoundingClientRect();
        setPosition({
            x: e.clientX - rect.left,
            y: e.clientY - rect.top
        });
    };

    return (
        <div>
            <div
                onMouseEnter={handleMouseEnter}
                onMouseLeave={handleMouseLeave}
                onMouseMove={handleMouseMove}
                style={{
                    width: '300px',
                    height: '200px',
                    backgroundColor: isHovering ? 'lightblue' : 'lightgray',
                    display: 'flex',
                    alignItems: 'center',
                    justifyContent: 'center',
                    cursor: 'pointer',
                    transition: 'background-color 0.3s'
                }}
            >
                <div>
                    <p>Hovering: {isHovering ? 'Sì' : 'No'}</p>
                    <p>Hover count: {hoverCount}</p>
                    <p>Position: ({position.x.toFixed(0)}, {position.y.toFixed(0)})</p>
                </div>
            </div>
        </div>
    );
}
\end{lstlisting}

\subsection{onScroll}

\begin{lstlisting}[language=JavaScript, caption=Evento scroll]
function ScrollExample() {
    const [scrollPosition, setScrollPosition] = useState(0);
    const [isTop, setIsTop] = useState(true);
    const scrollRef = useRef(null);

    const handleScroll = (e) => {
        const scrollTop = e.target.scrollTop;
        setScrollPosition(scrollTop);
        setIsTop(scrollTop === 0);
    };

    // Window scroll
    useEffect(() => {
        const handleWindowScroll = () => {
            setScrollPosition(window.scrollY);
        };

        window.addEventListener('scroll', handleWindowScroll);
        return () => window.removeEventListener('scroll', handleWindowScroll);
    }, []);

    return (
        <div>
            {/* Scrollable container */}
            <div
                ref={scrollRef}
                onScroll={handleScroll}
                style={{
                    height: '300px',
                    overflow: 'auto',
                    border: '1px solid black'
                }}
            >
                <div style={{ height: '1000px', padding: '20px' }}>
                    <p>Scroll position: {scrollPosition}px</p>
                    <p>At top: {isTop ? 'Sì' : 'No'}</p>
                    <p>Contenuto scrollabile...</p>
                </div>
            </div>
        </div>
    );
}
\end{lstlisting}

\section{Event Delegation e Bubbling}

\subsection{Event Bubbling}

\begin{lstlisting}[language=JavaScript, caption=Event bubbling]
function BubblingExample() {
    const handleParentClick = (e) => {
        console.log('Parent clicked');
        console.log('Target:', e.target.tagName);        // Elemento cliccato
        console.log('CurrentTarget:', e.currentTarget.tagName); // Elemento con handler
    };

    const handleChildClick = (e) => {
        console.log('Child clicked');

        // Ferma la propagazione
        // e.stopPropagation();
    };

    return (
        <div
            onClick={handleParentClick}
            style={{ padding: '20px', backgroundColor: 'lightblue' }}
        >
            Parent
            <button onClick={handleChildClick}>
                Child (clicca qui)
            </button>
        </div>
    );
    // Se clicchi il button, vedi:
    // "Child clicked"
    // "Parent clicked" (bubbling!)
}
\end{lstlisting}

\subsection{Event Delegation}

\begin{lstlisting}[language=JavaScript, caption=Event delegation pattern]
function TodoList() {
    const [todos, setTodos] = useState([
        { id: 1, text: 'Todo 1' },
        { id: 2, text: 'Todo 2' },
        { id: 3, text: 'Todo 3' }
    ]);

    // ❌ Handler per ogni elemento (non scalabile)
    const BadApproach = () => (
        <ul>
            {todos.map(todo => (
                <li key={todo.id}>
                    {todo.text}
                    <button onClick={() => deleteTodo(todo.id)}>Delete</button>
                </li>
            ))}
        </ul>
    );

    // ✅ Event delegation (un solo handler)
    const handleListClick = (e) => {
        // Verifica se è un button delete
        if (e.target.classList.contains('delete-btn')) {
            const id = Number(e.target.dataset.id);
            deleteTodo(id);
        }
    };

    const deleteTodo = (id) => {
        setTodos(todos.filter(todo => todo.id !== id));
    };

    return (
        <ul onClick={handleListClick}>
            {todos.map(todo => (
                <li key={todo.id}>
                    {todo.text}
                    <button
                        className="delete-btn"
                        data-id={todo.id}
                    >
                        Delete
                    </button>
                </li>
            ))}
        </ul>
    );
}
\end{lstlisting}

\section{Custom Events}

\subsection{Eventi Personalizzati}

\begin{lstlisting}[language=JavaScript, caption=Custom events pattern]
function CustomEventExample() {
    const handleCustomEvent = (data) => {
        console.log('Custom event:', data);
    };

    return (
        <div>
            <CustomButton onCustomClick={handleCustomEvent} />
        </div>
    );
}

function CustomButton({ onCustomClick }) {
    const handleClick = (e) => {
        // Raccogli dati aggiuntivi
        const data = {
            timestamp: Date.now(),
            position: {
                x: e.clientX,
                y: e.clientY
            },
            button: e.button
        };

        // Chiama callback con dati custom
        onCustomClick?.(data);
    };

    return <button onClick={handleClick}>Click me</button>;
}
\end{lstlisting}

\section{Performance e Best Practices}

\subsection{1. Evita Inline Functions nei Render}

\begin{lstlisting}[language=JavaScript, caption=Ottimizzazione event handlers]
function TodoItem({ todo, onToggle, onDelete }) {
    // ❌ MALE: Crea nuova funzione ad ogni render
    return (
        <li>
            <input
                type="checkbox"
                onChange={() => onToggle(todo.id)}
            />
            <button onClick={() => onDelete(todo.id)}>Delete</button>
        </li>
    );
}

// ✅ MEGLIO: useCallback
function TodoList() {
    const [todos, setTodos] = useState([...]);

    const handleToggle = useCallback((id) => {
        setTodos(prev => prev.map(todo =>
            todo.id === id ? { ...todo, completed: !todo.completed } : todo
        ));
    }, []);

    const handleDelete = useCallback((id) => {
        setTodos(prev => prev.filter(todo => todo.id !== id));
    }, []);

    return (
        <ul>
            {todos.map(todo => (
                <TodoItem
                    key={todo.id}
                    todo={todo}
                    onToggle={handleToggle}
                    onDelete={handleDelete}
                />
            ))}
        </ul>
    );
}
\end{lstlisting}

\subsection{2. Debouncing e Throttling}

\begin{lstlisting}[language=JavaScript, caption=Debounce e throttle]
// Debounce: Esegui dopo che l'utente smette di digitare
function SearchWithDebounce() {
    const [query, setQuery] = useState('');
    const [results, setResults] = useState([]);

    useEffect(() => {
        const timer = setTimeout(() => {
            if (query) {
                fetch(`/api/search?q=${query}`)
                    .then(res => res.json())
                    .then(setResults);
            }
        }, 500); // Aspetta 500ms dopo l'ultimo input

        return () => clearTimeout(timer);
    }, [query]);

    return (
        <div>
            <input
                value={query}
                onChange={(e) => setQuery(e.target.value)}
                placeholder="Cerca..."
            />
            <ul>
                {results.map(result => (
                    <li key={result.id}>{result.name}</li>
                ))}
            </ul>
        </div>
    );
}

// Throttle: Esegui al massimo una volta ogni X ms
function ScrollProgress() {
    const [progress, setProgress] = useState(0);
    const throttleRef = useRef(null);

    useEffect(() => {
        const handleScroll = () => {
            if (throttleRef.current) return;

            throttleRef.current = setTimeout(() => {
                const winScroll = document.documentElement.scrollTop;
                const height = document.documentElement.scrollHeight -
                              document.documentElement.clientHeight;
                const scrolled = (winScroll / height) * 100;

                setProgress(scrolled);
                throttleRef.current = null;
            }, 100); // Max una volta ogni 100ms
        };

        window.addEventListener('scroll', handleScroll);
        return () => window.removeEventListener('scroll', handleScroll);
    }, []);

    return (
        <div style={{
            position: 'fixed',
            top: 0,
            left: 0,
            width: `${progress}%`,
            height: '4px',
            backgroundColor: 'blue'
        }} />
    );
}
\end{lstlisting}

\subsection{3. Passive Event Listeners}

\begin{lstlisting}[language=JavaScript, caption=Passive listeners per performance]
function PassiveListenerExample() {
    useEffect(() => {
        const handleTouchMove = (e) => {
            // Non chiama preventDefault()
            console.log('Touch move');
        };

        // Passive listener per scroll performance
        document.addEventListener(
            'touchmove',
            handleTouchMove,
            { passive: true }
        );

        return () => {
            document.removeEventListener('touchmove', handleTouchMove);
        };
    }, []);

    return <div>Swipe qui</div>;
}
\end{lstlisting}

\section{Conclusione}

In questo capitolo abbiamo esplorato:

\begin{itemize}
    \item Synthetic Events in React
    \item onClick per gestione click
    \item onChange per input e form
    \item Altri eventi: onSubmit, onFocus, onBlur, keyboard, mouse
    \item Event bubbling e delegation
    \item Performance optimization: debounce, throttle, useCallback
    \item Best practices per event handling
\end{itemize}

Nel prossimo capitolo approfondiremo i form!
