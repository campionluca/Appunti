\chapter{Introduzione a React}

\section{Cos'è React?}

React è una libreria JavaScript dichiarativa, efficiente e flessibile per costruire interfacce utente. A differenza di framework completi come Angular, React si concentra esclusivamente sul layer di visualizzazione, lasciando agli sviluppatori la libertà di scegliere le soluzioni migliori per routing, state management e altre funzionalità.

\subsection{Caratteristiche Principali}

\textbf{1. Libreria, Non Framework}

React è una libreria focalizzata sulla UI, non un framework completo:

\begin{lstlisting}[language=JavaScript, caption=React fornisce solo la UI]
// React fornisce
import { useState } from 'react';

function App() {
    return <div>Hello React</div>;
}

// Tu scegli il resto:
// - Routing: React Router, TanStack Router, etc.
// - State: Redux, Zustand, MobX, etc.
// - Data fetching: TanStack Query, SWR, etc.
// - Forms: React Hook Form, Formik, etc.
\end{lstlisting}

\textbf{2. Component-Based}

Tutto in React è un componente riutilizzabile:

\begin{lstlisting}[language=JavaScript, caption=Componenti come building blocks]
// Componente Button
function Button({ children, onClick }) {
    return <button onClick={onClick}>{children}</button>;
}

// Componente Form che usa Button
function LoginForm() {
    return (
        <form>
            <input type="email" />
            <input type="password" />
            <Button onClick={() => console.log('Login')}>
                Accedi
            </Button>
        </form>
    );
}
\end{lstlisting}

\textbf{3. Dichiarativo}

Descrivi cosa vuoi vedere, React si occupa del resto:

\begin{lstlisting}[language=JavaScript, caption=Codice dichiarativo]
function Counter() {
    const [count, setCount] = useState(0);

    // Dichiari COSA mostrare in base allo stato
    return (
        <div>
            <p>Hai cliccato {count} volte</p>
            <button onClick={() => setCount(count + 1)}>
                Incrementa
            </button>
        </div>
    );
    // React aggiorna il DOM automaticamente quando count cambia
}
\end{lstlisting}

\section{Il Virtual DOM}

Il Virtual DOM è uno dei concetti fondamentali che rendono React efficiente e performante.

\subsection{Cos'è il Virtual DOM?}

Il Virtual DOM è una rappresentazione leggera del DOM reale mantenuta in memoria come oggetto JavaScript. Invece di manipolare direttamente il DOM (operazione costosa), React lavora con questa rappresentazione virtuale.

\textbf{Struttura del Virtual DOM:}

\begin{lstlisting}[language=JavaScript, caption=Rappresentazione Virtual DOM]
// JSX
<div className="container">
    <h1>Titolo</h1>
    <p>Paragrafo</p>
</div>

// Viene trasformato in Virtual DOM object
{
    type: 'div',
    props: {
        className: 'container',
        children: [
            {
                type: 'h1',
                props: {
                    children: 'Titolo'
                }
            },
            {
                type: 'p',
                props: {
                    children: 'Paragrafo'
                }
            }
        ]
    }
}
\end{lstlisting}

\subsection{Come Funziona il Virtual DOM}

Il processo di rendering in React segue questi passaggi:

\textbf{Fase 1: Render Iniziale}

\begin{lstlisting}[language=JavaScript, caption=Primo rendering]
function App() {
    const [items, setItems] = useState(['A', 'B', 'C']);

    return (
        <ul>
            {items.map(item => <li key={item}>{item}</li>)}
        </ul>
    );
}

// React crea Virtual DOM:
{
    type: 'ul',
    props: {
        children: [
            { type: 'li', props: { children: 'A', key: 'A' } },
            { type: 'li', props: { children: 'B', key: 'B' } },
            { type: 'li', props: { children: 'C', key: 'C' } }
        ]
    }
}

// React crea il DOM reale basandosi sul Virtual DOM
\end{lstlisting}

\textbf{Fase 2: Update}

\begin{lstlisting}[language=JavaScript, caption=Aggiornamento stato]
// Quando lo stato cambia (aggiungiamo 'D')
setItems(['A', 'B', 'C', 'D']);

// React crea un NUOVO Virtual DOM tree:
{
    type: 'ul',
    props: {
        children: [
            { type: 'li', props: { children: 'A', key: 'A' } },
            { type: 'li', props: { children: 'B', key: 'B' } },
            { type: 'li', props: { children: 'C', key: 'C' } },
            { type: 'li', props: { children: 'D', key: 'D' } } // NUOVO
        ]
    }
}
\end{lstlisting}

\textbf{Fase 3: Diffing}

React confronta il nuovo Virtual DOM con quello precedente:

\begin{lstlisting}[language=JavaScript, caption=Algoritmo di Diffing]
// Pseudo-codice del diffing
function diff(oldVTree, newVTree) {
    // Confronta tipo
    if (oldVTree.type !== newVTree.type) {
        return { action: 'REPLACE', node: newVTree };
    }

    // Confronta props
    const propsDiff = diffProps(oldVTree.props, newVTree.props);

    // Confronta children ricorsivamente
    const childrenDiff = diffChildren(
        oldVTree.props.children,
        newVTree.props.children
    );

    return {
        action: 'UPDATE',
        propChanges: propsDiff,
        childChanges: childrenDiff
    };
}

// Nel nostro esempio, React identifica:
// - 3 elementi uguali (A, B, C)
// - 1 elemento nuovo (D) da aggiungere
\end{lstlisting}

\textbf{Fase 4: Reconciliation}

React applica solo le modifiche necessarie al DOM reale:

\begin{lstlisting}[language=JavaScript, caption=Applicazione modifiche]
// Invece di ricreare tutta la lista:
// ❌ ul.innerHTML = '<li>A</li><li>B</li><li>C</li><li>D</li>';

// React fa solo:
// ✅ ul.appendChild(document.createElement('li')); // Aggiunge solo D
\end{lstlisting}

\subsection{Diagramma del Processo Virtual DOM}

\begin{verbatim}
┌─────────────────────────────────────────────────────────────┐
│                    VIRTUAL DOM WORKFLOW                      │
└─────────────────────────────────────────────────────────────┘

1. STATE CHANGE
   └─> setItems(['A', 'B', 'C', 'D'])
        │
        ▼
2. NEW VIRTUAL DOM TREE CREATED
   ┌──────────────────┐
   │  Virtual DOM v2  │
   │  [A, B, C, D]    │
   └──────────────────┘
        │
        ▼
3. DIFFING ALGORITHM
   ┌──────────────────┐      ┌──────────────────┐
   │  Virtual DOM v1  │ vs   │  Virtual DOM v2  │
   │  [A, B, C]       │      │  [A, B, C, D]    │
   └──────────────────┘      └──────────────────┘
        │
        └──────> Identifica: "Aggiungere D alla fine"
                 │
                 ▼
4. RECONCILIATION
   Applica SOLO le modifiche minime al DOM reale
   └─> appendChild(<li>D</li>)
\end{verbatim}

\subsection{Vantaggi del Virtual DOM}

\textbf{1. Performance}

\begin{lstlisting}[language=JavaScript, caption=Confronto performance]
// ❌ Manipolazione diretta del DOM (LENTA)
function updateList(items) {
    const ul = document.getElementById('list');
    ul.innerHTML = ''; // Rimuove tutto

    items.forEach(item => {
        const li = document.createElement('li');
        li.textContent = item;
        ul.appendChild(li); // Ogni append causa reflow
    });
}

// ✅ Con React Virtual DOM (VELOCE)
function List({ items }) {
    return (
        <ul>
            {items.map(item => <li key={item}>{item}</li>)}
        </ul>
    );
    // React aggiorna solo gli elementi modificati
}
\end{lstlisting}

\textbf{2. Batch Updates}

React raggruppa più aggiornamenti:

\begin{lstlisting}[language=JavaScript, caption=Batching automatico]
function MultiUpdate() {
    const [count, setCount] = useState(0);
    const [flag, setFlag] = useState(false);

    const handleClick = () => {
        setCount(count + 1);  // Update 1
        setFlag(!flag);       // Update 2
        // React raggruppa entrambi e fa un solo re-render
    };

    console.log('Render!'); // Viene chiamato UNA sola volta

    return (
        <div>
            <p>Count: {count}</p>
            <p>Flag: {flag.toString()}</p>
            <button onClick={handleClick}>Update</button>
        </div>
    );
}
\end{lstlisting}

\textbf{3. Prevedibilità}

Il rendering diventa deterministico:

\begin{lstlisting}[language=JavaScript, caption=Rendering prevedibile]
// Stesso stato = stesso output SEMPRE
function UserCard({ user }) {
    return (
        <div>
            <h3>{user.name}</h3>
            <p>{user.email}</p>
        </div>
    );
}

// Se user è identico, output identico
// Facile da testare e debuggare
\end{lstlisting}

\section{React vs Altri Framework}

\subsection{React vs Angular}

\textbf{Angular}

\begin{lstlisting}[language=TypeScript, caption=Componente Angular]
// user.component.ts
import { Component, Input } from '@angular/core';

@Component({
    selector: 'app-user',
    template: `
        <div class="user-card">
            <h3>{{ user.name }}</h3>
            <p>{{ user.email }}</p>
            <button (click)="onEdit()">Edit</button>
        </div>
    `,
    styleUrls: ['./user.component.css']
})
export class UserComponent {
    @Input() user: User;

    onEdit() {
        console.log('Editing', this.user);
    }
}
\end{lstlisting}

\textbf{React}

\begin{lstlisting}[language=JavaScript, caption=Componente React equivalente]
// UserCard.jsx
import './UserCard.css';

function UserCard({ user }) {
    const handleEdit = () => {
        console.log('Editing', user);
    };

    return (
        <div className="user-card">
            <h3>{user.name}</h3>
            <p>{user.email}</p>
            <button onClick={handleEdit}>Edit</button>
        </div>
    );
}

export default UserCard;
\end{lstlisting}

\textbf{Confronto:}

\begin{center}
\begin{tabular}{|l|l|l|}
\hline
\textbf{Aspetto} & \textbf{React} & \textbf{Angular} \\
\hline
Tipo & Libreria & Framework completo \\
\hline
Linguaggio & JavaScript/TypeScript & TypeScript \\
\hline
Learning Curve & Media & Ripida \\
\hline
Bundle Size & \textasciitilde45KB & \textasciitilde150KB \\
\hline
Data Binding & Unidirezionale & Bidirezionale \\
\hline
Dependency Injection & No (context/props) & Sì (built-in) \\
\hline
Routing & Libreria esterna & Built-in \\
\hline
State Management & Libreria esterna & RxJS/Services \\
\hline
Template & JSX & HTML + direttive \\
\hline
Mobile & React Native & Ionic/NativeScript \\
\hline
\end{tabular}
\end{center}

\subsection{React vs Vue}

\textbf{Vue}

\begin{lstlisting}[language=JavaScript, caption=Componente Vue]
<!-- UserCard.vue -->
<template>
    <div class="user-card">
        <h3>{{ user.name }}</h3>
        <p>{{ user.email }}</p>
        <button @click="handleEdit">Edit</button>
    </div>
</template>

<script setup>
import { defineProps } from 'vue';

const props = defineProps({
    user: Object
});

const handleEdit = () => {
    console.log('Editing', props.user);
};
</script>

<style scoped>
.user-card {
    border: 1px solid #ccc;
}
</style>
\end{lstlisting}

\textbf{React}

\begin{lstlisting}[language=JavaScript, caption=Componente React equivalente]
// UserCard.jsx
import styles from './UserCard.module.css';

function UserCard({ user }) {
    const handleEdit = () => {
        console.log('Editing', user);
    };

    return (
        <div className={styles.userCard}>
            <h3>{user.name}</h3>
            <p>{user.email}</p>
            <button onClick={handleEdit}>Edit</button>
        </div>
    );
}

export default UserCard;
\end{lstlisting}

\textbf{Confronto:}

\begin{center}
\begin{tabular}{|l|l|l|}
\hline
\textbf{Aspetto} & \textbf{React} & \textbf{Vue} \\
\hline
Tipo & Libreria & Framework progressivo \\
\hline
Learning Curve & Media & Facile \\
\hline
Template & JSX & HTML-based \\
\hline
Reattività & useState/useReducer & Proxy-based \\
\hline
Community & Molto grande & Grande \\
\hline
Job Market & Ampio & Buono \\
\hline
Ecosistema & Vasto & Integrato \\
\hline
Performance & Eccellente & Eccellente \\
\hline
\end{tabular}
\end{center}

\subsection{Quando Scegliere React}

\textbf{Scegli React se:}

\begin{itemize}
    \item Vuoi massima flessibilità nella scelta delle librerie
    \item Hai bisogno di un'ottima integrazione con TypeScript
    \item Il mercato del lavoro è un fattore importante
    \item Vuoi costruire anche app mobile (React Native)
    \item Preferisci JavaScript/JSX ai template HTML
    \item Hai bisogno di un ecosistema vasto
\end{itemize}

\textbf{Scegli Vue se:}

\begin{itemize}
    \item Vuoi una curva di apprendimento più dolce
    \item Preferisci template HTML con direttive
    \item Vuoi un framework più "opinionated" ma flessibile
    \item Apprezzi la documentazione eccellente
\end{itemize}

\textbf{Scegli Angular se:}

\begin{itemize}
    \item Lavori su applicazioni enterprise di grandi dimensioni
    \item Hai bisogno di TypeScript robusto
    \item Vuoi un framework completo "tutto incluso"
    \item Apprezzi structure e convention rigide
\end{itemize}

\section{Setup dell'Ambiente di Sviluppo}

\subsection{Prerequisiti}

Prima di iniziare, assicurati di avere installato:

\begin{lstlisting}[language=bash, caption=Verifica installazioni]
# Node.js (versione 18 o superiore)
node --version
# Output: v20.10.0

# npm (viene installato con Node.js)
npm --version
# Output: 10.2.3

# Git (opzionale ma raccomandato)
git --version
# Output: git version 2.42.0
\end{lstlisting}

\subsection{Opzione 1: Vite (Raccomandato 2024)}

Vite è il build tool moderno più veloce per progetti React:

\begin{lstlisting}[language=bash, caption=Creare progetto con Vite]
# Crea nuovo progetto
npm create vite@latest my-react-app -- --template react

# Entra nella directory
cd my-react-app

# Installa dipendenze
npm install

# Avvia development server
npm run dev
# Server running at http://localhost:5173
\end{lstlisting}

\textbf{Struttura progetto Vite:}

\begin{verbatim}
my-react-app/
├── node_modules/
├── public/
│   └── vite.svg
├── src/
│   ├── assets/
│   ├── App.css
│   ├── App.jsx
│   ├── index.css
│   └── main.jsx
├── .gitignore
├── index.html
├── package.json
└── vite.config.js
\end{verbatim}

\textbf{File principali:}

\begin{lstlisting}[language=JavaScript, caption=main.jsx - Entry point]
import React from 'react'
import ReactDOM from 'react-dom/client'
import App from './App.jsx'
import './index.css'

ReactDOM.createRoot(document.getElementById('root')).render(
    <React.StrictMode>
        <App />
    </React.StrictMode>,
)
\end{lstlisting}

\begin{lstlisting}[language=JavaScript, caption=App.jsx - Componente principale]
import { useState } from 'react'
import './App.css'

function App() {
    const [count, setCount] = useState(0)

    return (
        <div className="App">
            <h1>Vite + React</h1>
            <div className="card">
                <button onClick={() => setCount((count) => count + 1)}>
                    count is {count}
                </button>
            </div>
        </div>
    )
}

export default App
\end{lstlisting}

\begin{lstlisting}[language=JavaScript, caption=vite.config.js - Configurazione]
import { defineConfig } from 'vite'
import react from '@vitejs/plugin-react'

export default defineConfig({
    plugins: [react()],
    server: {
        port: 3000, // Cambia porta se necessario
        open: true  // Apri browser automaticamente
    }
})
\end{lstlisting}

\textbf{Vantaggi di Vite:}

\begin{itemize}
    \item \textbf{Velocissimo}: HMR (Hot Module Replacement) istantaneo
    \item \textbf{Leggero}: Build ottimizzate con Rollup
    \item \textbf{Moderno}: ESM nativo, supporto TypeScript out-of-the-box
    \item \textbf{Semplice}: Configurazione minimale
\end{itemize}

\subsection{Opzione 2: Create React App (Legacy)}

Create React App (CRA) è stato lo standard per anni, ma ora è meno raccomandato:

\begin{lstlisting}[language=bash, caption=Creare progetto con CRA]
# Crea nuovo progetto
npx create-react-app my-app

# Entra nella directory
cd my-app

# Avvia development server
npm start
# Server running at http://localhost:3000
\end{lstlisting}

\textbf{Nota}: Create React App è ora in maintenance mode. La documentazione ufficiale React raccomanda Vite o framework come Next.js.

\subsection{Opzione 3: Setup Manuale}

Per comprendere meglio come funziona:

\begin{lstlisting}[language=bash, caption=Setup manuale da zero]
# Crea directory e inizializza progetto
mkdir my-react-app
cd my-react-app
npm init -y

# Installa React e ReactDOM
npm install react react-dom

# Installa Vite come dev dependency
npm install -D vite @vitejs/plugin-react

# Crea struttura
mkdir src public
touch src/main.jsx src/App.jsx index.html vite.config.js
\end{lstlisting}

\begin{lstlisting}[language=HTML, caption=index.html]
<!DOCTYPE html>
<html lang="it">
<head>
    <meta charset="UTF-8">
    <meta name="viewport" content="width=device-width, initial-scale=1.0">
    <title>My React App</title>
</head>
<body>
    <div id="root"></div>
    <script type="module" src="/src/main.jsx"></script>
</body>
</html>
\end{lstlisting}

\begin{lstlisting}[language=JSON, caption=package.json - Aggiungi scripts]
{
    "name": "my-react-app",
    "version": "1.0.0",
    "scripts": {
        "dev": "vite",
        "build": "vite build",
        "preview": "vite preview"
    },
    "dependencies": {
        "react": "^18.2.0",
        "react-dom": "^18.2.0"
    },
    "devDependencies": {
        "vite": "^5.0.0",
        "@vitejs/plugin-react": "^4.2.0"
    }
}
\end{lstlisting}

\subsection{Opzione 4: Framework Full-Stack}

Per applicazioni più complesse, considera framework completi:

\textbf{Next.js (Raccomandato per produzione):}

\begin{lstlisting}[language=bash, caption=Setup Next.js]
npx create-next-app@latest my-next-app

# Scegli opzioni:
# ✓ TypeScript? No/Yes
# ✓ ESLint? Yes
# ✓ Tailwind CSS? No/Yes
# ✓ App Router? Yes (raccomandato)
# ✓ Customize import alias? No

cd my-next-app
npm run dev
\end{lstlisting}

\textbf{Remix:}

\begin{lstlisting}[language=bash, caption=Setup Remix]
npx create-remix@latest my-remix-app
cd my-remix-app
npm run dev
\end{lstlisting}

\section{Il Primo Progetto React}

Creiamo una semplice applicazione TODO per esplorare i concetti base:

\begin{lstlisting}[language=bash, caption=Setup progetto TODO]
npm create vite@latest react-todo -- --template react
cd react-todo
npm install
npm run dev
\end{lstlisting}

\subsection{App.jsx - TODO List Completa}

\begin{lstlisting}[language=JavaScript, caption=Applicazione TODO completa]
import { useState } from 'react';
import './App.css';

function App() {
    const [todos, setTodos] = useState([
        { id: 1, text: 'Imparare React', completed: false },
        { id: 2, text: 'Costruire un progetto', completed: false }
    ]);
    const [inputValue, setInputValue] = useState('');

    // Aggiungi nuovo todo
    const addTodo = () => {
        if (inputValue.trim() === '') return;

        const newTodo = {
            id: Date.now(),
            text: inputValue,
            completed: false
        };

        setTodos([...todos, newTodo]);
        setInputValue('');
    };

    // Toggle completamento
    const toggleTodo = (id) => {
        setTodos(todos.map(todo =>
            todo.id === id
                ? { ...todo, completed: !todo.completed }
                : todo
        ));
    };

    // Elimina todo
    const deleteTodo = (id) => {
        setTodos(todos.filter(todo => todo.id !== id));
    };

    // Gestisci Enter key
    const handleKeyPress = (e) => {
        if (e.key === 'Enter') {
            addTodo();
        }
    };

    // Statistiche
    const totalTodos = todos.length;
    const completedTodos = todos.filter(t => t.completed).length;
    const pendingTodos = totalTodos - completedTodos;

    return (
        <div className="app">
            <div className="container">
                <h1>📝 My TODO List</h1>

                {/* Statistiche */}
                <div className="stats">
                    <span>Totale: {totalTodos}</span>
                    <span>Completati: {completedTodos}</span>
                    <span>Da fare: {pendingTodos}</span>
                </div>

                {/* Input nuovo todo */}
                <div className="input-section">
                    <input
                        type="text"
                        value={inputValue}
                        onChange={(e) => setInputValue(e.target.value)}
                        onKeyPress={handleKeyPress}
                        placeholder="Aggiungi un nuovo task..."
                    />
                    <button onClick={addTodo}>Aggiungi</button>
                </div>

                {/* Lista todos */}
                <ul className="todo-list">
                    {todos.length === 0 ? (
                        <li className="empty">Nessun task. Aggiungine uno!</li>
                    ) : (
                        todos.map(todo => (
                            <li
                                key={todo.id}
                                className={todo.completed ? 'completed' : ''}
                            >
                                <input
                                    type="checkbox"
                                    checked={todo.completed}
                                    onChange={() => toggleTodo(todo.id)}
                                />
                                <span onClick={() => toggleTodo(todo.id)}>
                                    {todo.text}
                                </span>
                                <button
                                    className="delete-btn"
                                    onClick={() => deleteTodo(todo.id)}
                                >
                                    🗑️
                                </button>
                            </li>
                        ))
                    )}
                </ul>

                {/* Clear completed */}
                {completedTodos > 0 && (
                    <button
                        className="clear-completed"
                        onClick={() => setTodos(todos.filter(t => !t.completed))}
                    >
                        Rimuovi completati ({completedTodos})
                    </button>
                )}
            </div>
        </div>
    );
}

export default App;
\end{lstlisting}

\subsection{App.css - Styling}

\begin{lstlisting}[language=CSS, caption=Stili per TODO app]
.app {
    min-height: 100vh;
    background: linear-gradient(135deg, #667eea 0%, #764ba2 100%);
    display: flex;
    justify-content: center;
    align-items: center;
    padding: 20px;
}

.container {
    background: white;
    border-radius: 15px;
    padding: 30px;
    max-width: 500px;
    width: 100%;
    box-shadow: 0 10px 40px rgba(0,0,0,0.2);
}

h1 {
    text-align: center;
    color: #333;
    margin-bottom: 20px;
}

.stats {
    display: flex;
    justify-content: space-around;
    margin-bottom: 20px;
    padding: 15px;
    background: #f5f5f5;
    border-radius: 8px;
}

.stats span {
    font-size: 14px;
    color: #666;
}

.input-section {
    display: flex;
    gap: 10px;
    margin-bottom: 20px;
}

.input-section input {
    flex: 1;
    padding: 12px;
    border: 2px solid #e0e0e0;
    border-radius: 8px;
    font-size: 16px;
}

.input-section input:focus {
    outline: none;
    border-color: #667eea;
}

.input-section button {
    padding: 12px 24px;
    background: #667eea;
    color: white;
    border: none;
    border-radius: 8px;
    cursor: pointer;
    font-weight: 600;
}

.input-section button:hover {
    background: #5568d3;
}

.todo-list {
    list-style: none;
    padding: 0;
    margin: 0;
}

.todo-list li {
    display: flex;
    align-items: center;
    padding: 15px;
    border-bottom: 1px solid #e0e0e0;
    gap: 10px;
}

.todo-list li.completed span {
    text-decoration: line-through;
    color: #999;
}

.todo-list li span {
    flex: 1;
    cursor: pointer;
}

.todo-list li input[type="checkbox"] {
    width: 20px;
    height: 20px;
    cursor: pointer;
}

.delete-btn {
    background: none;
    border: none;
    cursor: pointer;
    font-size: 20px;
    opacity: 0.6;
    transition: opacity 0.2s;
}

.delete-btn:hover {
    opacity: 1;
}

.clear-completed {
    width: 100%;
    padding: 12px;
    margin-top: 15px;
    background: #f44336;
    color: white;
    border: none;
    border-radius: 8px;
    cursor: pointer;
    font-weight: 600;
}

.clear-completed:hover {
    background: #da190b;
}

.empty {
    text-align: center;
    color: #999;
    padding: 40px;
    font-style: italic;
}
\end{lstlisting}

\section{React Developer Tools}

\subsection{Installazione}

React DevTools è un'estensione essenziale per il debugging:

\begin{itemize}
    \item \textbf{Chrome}: Chrome Web Store → "React Developer Tools"
    \item \textbf{Firefox}: Firefox Add-ons → "React DevTools"
    \item \textbf{Edge}: Edge Add-ons → "React Developer Tools"
\end{itemize}

\subsection{Funzionalità Principali}

\textbf{1. Components Tab}

Visualizza l'albero dei componenti e le loro props/state:

\begin{verbatim}
Components
├─ App
   ├─ TodoList
   │  ├─ props: { todos: [...] }
   │  └─ state: { filter: 'all' }
   └─ TodoItem (x3)
      └─ props: { todo: {...}, onToggle: f }
\end{verbatim}

\textbf{2. Profiler Tab}

Analizza le performance dei rendering:

\begin{lstlisting}[language=JavaScript, caption=Usare il Profiler]
import { Profiler } from 'react';

function App() {
    const onRenderCallback = (
        id,
        phase,
        actualDuration,
        baseDuration,
        startTime,
        commitTime
    ) => {
        console.log(`${id} (${phase}) took ${actualDuration}ms`);
    };

    return (
        <Profiler id="App" onRender={onRenderCallback}>
            <TodoList />
        </Profiler>
    );
}
\end{lstlisting}

\section{Best Practices Iniziali}

\subsection{1. Organizzazione File}

\begin{verbatim}
src/
├── components/
│   ├── TodoList/
│   │   ├── TodoList.jsx
│   │   ├── TodoList.css
│   │   └── index.js
│   └── TodoItem/
│       ├── TodoItem.jsx
│       ├── TodoItem.css
│       └── index.js
├── hooks/
│   └── useTodos.js
├── utils/
│   └── helpers.js
├── App.jsx
└── main.jsx
\end{verbatim}

\subsection{2. Naming Convention}

\begin{lstlisting}[language=JavaScript, caption=Convenzioni di naming]
// Componenti: PascalCase
function TodoList() {}
function UserProfile() {}

// File componenti: stesso nome del componente
// TodoList.jsx, UserProfile.jsx

// Funzioni/variabili: camelCase
const handleClick = () => {};
const userData = {};

// Costanti: UPPER_SNAKE_CASE
const API_URL = 'https://api.example.com';
const MAX_ITEMS = 100;

// Custom hooks: use + PascalCase
function useTodos() {}
function useLocalStorage() {}
\end{lstlisting}

\subsection{3. Componenti Puliti}

\begin{lstlisting}[language=JavaScript, caption=Componente ben strutturato]
import { useState, useEffect } from 'react';
import PropTypes from 'prop-types';
import './TodoList.css';

// 1. Imports

// 2. Costanti
const FILTER_OPTIONS = ['all', 'active', 'completed'];

// 3. Componente
function TodoList({ initialTodos = [] }) {
    // 3.1 State
    const [todos, setTodos] = useState(initialTodos);
    const [filter, setFilter] = useState('all');

    // 3.2 Effects
    useEffect(() => {
        console.log('Todos changed:', todos);
    }, [todos]);

    // 3.3 Handlers
    const handleAddTodo = (text) => {
        // ...
    };

    // 3.4 Computed values
    const filteredTodos = todos.filter(todo => {
        if (filter === 'active') return !todo.completed;
        if (filter === 'completed') return todo.completed;
        return true;
    });

    // 3.5 Render
    return (
        <div className="todo-list">
            {/* JSX */}
        </div>
    );
}

// 4. PropTypes
TodoList.propTypes = {
    initialTodos: PropTypes.arrayOf(PropTypes.shape({
        id: PropTypes.number.isRequired,
        text: PropTypes.string.isRequired,
        completed: PropTypes.bool.isRequired
    }))
};

// 5. Export
export default TodoList;
\end{lstlisting}

\section{Conclusione}

In questo capitolo abbiamo esplorato:

\begin{itemize}
    \item Cos'è React e le sue caratteristiche principali
    \item Il funzionamento del Virtual DOM
    \item Confronto con Angular e Vue
    \item Setup dell'ambiente con Vite
    \item Creazione del primo progetto React
    \item React DevTools e best practices
\end{itemize}

Ora sei pronto per approfondire JSX e i componenti nel prossimo capitolo!
