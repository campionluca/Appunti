\chapter{Form in React}

\section{Introduzione ai Form}

I form sono una parte essenziale di quasi tutte le applicazioni web. React offre un approccio potente e flessibile per gestire form attraverso i "controlled components".

\subsection{Controlled vs Uncontrolled Components}

\begin{lstlisting}[language=JavaScript, caption=Controlled vs Uncontrolled]
// ❌ Uncontrolled: Il DOM gestisce lo stato
function UncontrolledForm() {
    const nameRef = useRef();

    const handleSubmit = (e) => {
        e.preventDefault();
        console.log('Name:', nameRef.current.value);
    };

    return (
        <form onSubmit={handleSubmit}>
            <input ref={nameRef} defaultValue="Mario" />
            <button type="submit">Submit</button>
        </form>
    );
}

// ✅ Controlled: React gestisce lo stato
function ControlledForm() {
    const [name, setName] = useState('Mario');

    const handleSubmit = (e) => {
        e.preventDefault();
        console.log('Name:', name);
    };

    return (
        <form onSubmit={handleSubmit}>
            <input
                value={name}
                onChange={(e) => setName(e.target.value)}
            />
            <button type="submit">Submit</button>
        </form>
    );
}
\end{lstlisting}

\textbf{Vantaggi Controlled Components:}
\begin{itemize}
    \item Validazione in tempo reale
    \item Formattazione automatica dell'input
    \item Disabilitazione condizionale del submit
    \item Stato sincronizzato con React
    \item Più facile da testare
\end{itemize}

\section{Form Base Controllati}

\subsection{Form con Input Multipli}

\begin{lstlisting}[language=JavaScript, caption=Form completo controllato]
function RegistrationForm() {
    const [formData, setFormData] = useState({
        username: '',
        email: '',
        password: '',
        confirmPassword: '',
        age: '',
        country: '',
        gender: '',
        terms: false,
        newsletter: false
    });

    // Handler generico per tutti gli input
    const handleChange = (e) => {
        const { name, value, type, checked } = e.target;

        setFormData(prev => ({
            ...prev,
            [name]: type === 'checkbox' ? checked : value
        }));
    };

    const handleSubmit = (e) => {
        e.preventDefault();
        console.log('Form data:', formData);
    };

    return (
        <form onSubmit={handleSubmit}>
            {/* Text Input */}
            <div>
                <label htmlFor="username">Username:</label>
                <input
                    id="username"
                    type="text"
                    name="username"
                    value={formData.username}
                    onChange={handleChange}
                    required
                />
            </div>

            {/* Email Input */}
            <div>
                <label htmlFor="email">Email:</label>
                <input
                    id="email"
                    type="email"
                    name="email"
                    value={formData.email}
                    onChange={handleChange}
                    required
                />
            </div>

            {/* Password Input */}
            <div>
                <label htmlFor="password">Password:</label>
                <input
                    id="password"
                    type="password"
                    name="password"
                    value={formData.password}
                    onChange={handleChange}
                    required
                />
            </div>

            {/* Confirm Password */}
            <div>
                <label htmlFor="confirmPassword">Conferma Password:</label>
                <input
                    id="confirmPassword"
                    type="password"
                    name="confirmPassword"
                    value={formData.confirmPassword}
                    onChange={handleChange}
                    required
                />
            </div>

            {/* Number Input */}
            <div>
                <label htmlFor="age">Età:</label>
                <input
                    id="age"
                    type="number"
                    name="age"
                    value={formData.age}
                    onChange={handleChange}
                    min="18"
                    max="120"
                />
            </div>

            {/* Select */}
            <div>
                <label htmlFor="country">Paese:</label>
                <select
                    id="country"
                    name="country"
                    value={formData.country}
                    onChange={handleChange}
                >
                    <option value="">Seleziona...</option>
                    <option value="IT">Italia</option>
                    <option value="US">USA</option>
                    <option value="UK">UK</option>
                    <option value="FR">Francia</option>
                </select>
            </div>

            {/* Radio Buttons */}
            <div>
                <label>Genere:</label>
                <label>
                    <input
                        type="radio"
                        name="gender"
                        value="male"
                        checked={formData.gender === 'male'}
                        onChange={handleChange}
                    />
                    Maschio
                </label>
                <label>
                    <input
                        type="radio"
                        name="gender"
                        value="female"
                        checked={formData.gender === 'female'}
                        onChange={handleChange}
                    />
                    Femmina
                </label>
                <label>
                    <input
                        type="radio"
                        name="gender"
                        value="other"
                        checked={formData.gender === 'other'}
                        onChange={handleChange}
                    />
                    Altro
                </label>
            </div>

            {/* Checkboxes */}
            <div>
                <label>
                    <input
                        type="checkbox"
                        name="terms"
                        checked={formData.terms}
                        onChange={handleChange}
                    />
                    Accetto i termini e condizioni
                </label>
            </div>

            <div>
                <label>
                    <input
                        type="checkbox"
                        name="newsletter"
                        checked={formData.newsletter}
                        onChange={handleChange}
                    />
                    Iscriviti alla newsletter
                </label>
            </div>

            {/* Submit Button */}
            <button type="submit">Registrati</button>

            {/* Debug */}
            <pre>{JSON.stringify(formData, null, 2)}</pre>
        </form>
    );
}
\end{lstlisting}

\section{Validazione Form}

\subsection{Validazione Base}

\begin{lstlisting}[language=JavaScript, caption=Validazione manuale]
function ValidatedForm() {
    const [formData, setFormData] = useState({
        username: '',
        email: '',
        password: '',
        confirmPassword: ''
    });

    const [errors, setErrors] = useState({});
    const [touched, setTouched] = useState({});

    const handleChange = (e) => {
        const { name, value } = e.target;
        setFormData(prev => ({ ...prev, [name]: value }));

        // Rimuovi errore quando l'utente inizia a digitare
        if (errors[name]) {
            setErrors(prev => ({ ...prev, [name]: '' }));
        }
    };

    const handleBlur = (e) => {
        const { name } = e.target;
        setTouched(prev => ({ ...prev, [name]: true }));
        validateField(name, formData[name]);
    };

    const validateField = (name, value) => {
        let error = '';

        switch (name) {
            case 'username':
                if (!value) {
                    error = 'Username è richiesto';
                } else if (value.length < 3) {
                    error = 'Username deve essere almeno 3 caratteri';
                } else if (!/^[a-zA-Z0-9_]+$/.test(value)) {
                    error = 'Username può contenere solo lettere, numeri e underscore';
                }
                break;

            case 'email':
                if (!value) {
                    error = 'Email è richiesta';
                } else if (!/\S+@\S+\.\S+/.test(value)) {
                    error = 'Email non valida';
                }
                break;

            case 'password':
                if (!value) {
                    error = 'Password è richiesta';
                } else if (value.length < 8) {
                    error = 'Password deve essere almeno 8 caratteri';
                } else if (!/(?=.*[a-z])(?=.*[A-Z])(?=.*\d)/.test(value)) {
                    error = 'Password deve contenere maiuscola, minuscola e numero';
                }
                break;

            case 'confirmPassword':
                if (!value) {
                    error = 'Conferma password è richiesta';
                } else if (value !== formData.password) {
                    error = 'Le password non corrispondono';
                }
                break;
        }

        setErrors(prev => ({ ...prev, [name]: error }));
        return error;
    };

    const validateForm = () => {
        const newErrors = {};

        Object.keys(formData).forEach(key => {
            const error = validateField(key, formData[key]);
            if (error) newErrors[key] = error;
        });

        setErrors(newErrors);
        return Object.keys(newErrors).length === 0;
    };

    const handleSubmit = (e) => {
        e.preventDefault();

        // Mark all fields as touched
        const allTouched = Object.keys(formData).reduce(
            (acc, key) => ({ ...acc, [key]: true }),
            {}
        );
        setTouched(allTouched);

        if (validateForm()) {
            console.log('Form valido:', formData);
            // Submit...
        } else {
            console.log('Form non valido');
        }
    };

    return (
        <form onSubmit={handleSubmit}>
            <div>
                <label htmlFor="username">Username:</label>
                <input
                    id="username"
                    type="text"
                    name="username"
                    value={formData.username}
                    onChange={handleChange}
                    onBlur={handleBlur}
                    className={touched.username && errors.username ? 'error' : ''}
                />
                {touched.username && errors.username && (
                    <span className="error-message">{errors.username}</span>
                )}
            </div>

            <div>
                <label htmlFor="email">Email:</label>
                <input
                    id="email"
                    type="email"
                    name="email"
                    value={formData.email}
                    onChange={handleChange}
                    onBlur={handleBlur}
                    className={touched.email && errors.email ? 'error' : ''}
                />
                {touched.email && errors.email && (
                    <span className="error-message">{errors.email}</span>
                )}
            </div>

            <div>
                <label htmlFor="password">Password:</label>
                <input
                    id="password"
                    type="password"
                    name="password"
                    value={formData.password}
                    onChange={handleChange}
                    onBlur={handleBlur}
                    className={touched.password && errors.password ? 'error' : ''}
                />
                {touched.password && errors.password && (
                    <span className="error-message">{errors.password}</span>
                )}
            </div>

            <div>
                <label htmlFor="confirmPassword">Conferma Password:</label>
                <input
                    id="confirmPassword"
                    type="password"
                    name="confirmPassword"
                    value={formData.confirmPassword}
                    onChange={handleChange}
                    onBlur={handleBlur}
                    className={touched.confirmPassword && errors.confirmPassword ? 'error' : ''}
                />
                {touched.confirmPassword && errors.confirmPassword && (
                    <span className="error-message">{errors.confirmPassword}</span>
                )}
            </div>

            <button type="submit">Registrati</button>
        </form>
    );
}
\end{lstlisting}

\subsection{Custom Hook per Validazione}

\begin{lstlisting}[language=JavaScript, caption=useForm custom hook]
function useForm(initialValues, validate) {
    const [values, setValues] = useState(initialValues);
    const [errors, setErrors] = useState({});
    const [touched, setTouched] = useState({});
    const [isSubmitting, setIsSubmitting] = useState(false);

    const handleChange = (e) => {
        const { name, value, type, checked } = e.target;
        const newValue = type === 'checkbox' ? checked : value;

        setValues(prev => ({ ...prev, [name]: newValue }));

        // Rimuovi errore
        if (errors[name]) {
            setErrors(prev => ({ ...prev, [name]: '' }));
        }
    };

    const handleBlur = (e) => {
        const { name } = e.target;
        setTouched(prev => ({ ...prev, [name]: true }));
    };

    const handleSubmit = async (onSubmit) => {
        // Mark all as touched
        const allTouched = Object.keys(values).reduce(
            (acc, key) => ({ ...acc, [key]: true }),
            {}
        );
        setTouched(allTouched);

        // Validate
        const validationErrors = validate(values);
        setErrors(validationErrors);

        if (Object.keys(validationErrors).length === 0) {
            setIsSubmitting(true);
            try {
                await onSubmit(values);
            } finally {
                setIsSubmitting(false);
            }
        }
    };

    const resetForm = () => {
        setValues(initialValues);
        setErrors({});
        setTouched({});
        setIsSubmitting(false);
    };

    return {
        values,
        errors,
        touched,
        isSubmitting,
        handleChange,
        handleBlur,
        handleSubmit,
        resetForm
    };
}

// Uso
function LoginForm() {
    const validate = (values) => {
        const errors = {};

        if (!values.email) {
            errors.email = 'Email richiesta';
        } else if (!/\S+@\S+\.\S+/.test(values.email)) {
            errors.email = 'Email non valida';
        }

        if (!values.password) {
            errors.password = 'Password richiesta';
        } else if (values.password.length < 6) {
            errors.password = 'Password troppo corta';
        }

        return errors;
    };

    const form = useForm(
        { email: '', password: '' },
        validate
    );

    const onSubmit = async (values) => {
        console.log('Submitting:', values);
        await new Promise(resolve => setTimeout(resolve, 1000));
        console.log('Submitted!');
        form.resetForm();
    };

    return (
        <form onSubmit={(e) => {
            e.preventDefault();
            form.handleSubmit(onSubmit);
        }}>
            <div>
                <input
                    name="email"
                    type="email"
                    value={form.values.email}
                    onChange={form.handleChange}
                    onBlur={form.handleBlur}
                    placeholder="Email"
                />
                {form.touched.email && form.errors.email && (
                    <span>{form.errors.email}</span>
                )}
            </div>

            <div>
                <input
                    name="password"
                    type="password"
                    value={form.values.password}
                    onChange={form.handleChange}
                    onBlur={form.handleBlur}
                    placeholder="Password"
                />
                {form.touched.password && form.errors.password && (
                    <span>{form.errors.password}</span>
                )}
            </div>

            <button type="submit" disabled={form.isSubmitting}>
                {form.isSubmitting ? 'Accedendo...' : 'Accedi'}
            </button>
        </form>
    );
}
\end{lstlisting}

\section{React Hook Form}

React Hook Form è una libreria popolare per la gestione dei form con performance ottimali.

\subsection{Setup React Hook Form}

\begin{lstlisting}[language=bash, caption=Installazione]
npm install react-hook-form
\end{lstlisting}

\subsection{Form Base con React Hook Form}

\begin{lstlisting}[language=JavaScript, caption=React Hook Form esempio base]
import { useForm } from 'react-hook-form';

function BasicForm() {
    const {
        register,
        handleSubmit,
        formState: { errors, isSubmitting },
        reset
    } = useForm({
        defaultValues: {
            username: '',
            email: '',
            password: ''
        }
    });

    const onSubmit = async (data) => {
        console.log('Form data:', data);
        await new Promise(resolve => setTimeout(resolve, 1000));
        console.log('Submitted!');
        reset();
    };

    return (
        <form onSubmit={handleSubmit(onSubmit)}>
            {/* Username */}
            <div>
                <label htmlFor="username">Username:</label>
                <input
                    id="username"
                    {...register('username', {
                        required: 'Username è richiesto',
                        minLength: {
                            value: 3,
                            message: 'Minimo 3 caratteri'
                        },
                        pattern: {
                            value: /^[a-zA-Z0-9_]+$/,
                            message: 'Solo lettere, numeri e underscore'
                        }
                    })}
                />
                {errors.username && (
                    <span className="error">{errors.username.message}</span>
                )}
            </div>

            {/* Email */}
            <div>
                <label htmlFor="email">Email:</label>
                <input
                    id="email"
                    type="email"
                    {...register('email', {
                        required: 'Email è richiesta',
                        pattern: {
                            value: /\S+@\S+\.\S+/,
                            message: 'Email non valida'
                        }
                    })}
                />
                {errors.email && (
                    <span className="error">{errors.email.message}</span>
                )}
            </div>

            {/* Password */}
            <div>
                <label htmlFor="password">Password:</label>
                <input
                    id="password"
                    type="password"
                    {...register('password', {
                        required: 'Password è richiesta',
                        minLength: {
                            value: 8,
                            message: 'Minimo 8 caratteri'
                        },
                        validate: {
                            hasUpperCase: (value) =>
                                /[A-Z]/.test(value) || 'Deve contenere maiuscola',
                            hasLowerCase: (value) =>
                                /[a-z]/.test(value) || 'Deve contenere minuscola',
                            hasNumber: (value) =>
                                /\d/.test(value) || 'Deve contenere un numero'
                        }
                    })}
                />
                {errors.password && (
                    <span className="error">{errors.password.message}</span>
                )}
            </div>

            <button type="submit" disabled={isSubmitting}>
                {isSubmitting ? 'Invio...' : 'Invia'}
            </button>
        </form>
    );
}
\end{lstlisting}

\subsection{Validazione Avanzata React Hook Form}

\begin{lstlisting}[language=JavaScript, caption=Validazione custom]
import { useForm } from 'react-hook-form';

function AdvancedForm() {
    const { register, handleSubmit, watch, formState: { errors } } = useForm();

    // Watch per validazione dipendente
    const password = watch('password');

    const onSubmit = (data) => {
        console.log(data);
    };

    return (
        <form onSubmit={handleSubmit(onSubmit)}>
            {/* Password */}
            <input
                type="password"
                {...register('password', {
                    required: 'Password richiesta',
                    minLength: { value: 8, message: 'Minimo 8 caratteri' }
                })}
            />
            {errors.password && <span>{errors.password.message}</span>}

            {/* Confirm Password - dipende da password */}
            <input
                type="password"
                {...register('confirmPassword', {
                    required: 'Conferma password richiesta',
                    validate: (value) =>
                        value === password || 'Le password non corrispondono'
                })}
            />
            {errors.confirmPassword && (
                <span>{errors.confirmPassword.message}</span>
            )}

            {/* Age - validazione numerica */}
            <input
                type="number"
                {...register('age', {
                    required: 'Età richiesta',
                    min: { value: 18, message: 'Devi avere almeno 18 anni' },
                    max: { value: 120, message: 'Età non valida' },
                    valueAsNumber: true
                })}
            />
            {errors.age && <span>{errors.age.message}</span>}

            {/* Terms - checkbox */}
            <label>
                <input
                    type="checkbox"
                    {...register('terms', {
                        required: 'Devi accettare i termini'
                    })}
                />
                Accetto i termini
            </label>
            {errors.terms && <span>{errors.terms.message}</span>}

            <button type="submit">Invia</button>
        </form>
    );
}
\end{lstlisting}

\subsection{Form Arrays con React Hook Form}

\begin{lstlisting}[language=JavaScript, caption=Dynamic form fields]
import { useForm, useFieldArray } from 'react-hook-form';

function DynamicForm() {
    const { register, control, handleSubmit } = useForm({
        defaultValues: {
            users: [{ name: '', email: '' }]
        }
    });

    const { fields, append, remove } = useFieldArray({
        control,
        name: 'users'
    });

    const onSubmit = (data) => {
        console.log('Users:', data.users);
    };

    return (
        <form onSubmit={handleSubmit(onSubmit)}>
            {fields.map((field, index) => (
                <div key={field.id}>
                    <input
                        {...register(`users.${index}.name`, {
                            required: 'Nome richiesto'
                        })}
                        placeholder="Nome"
                    />

                    <input
                        {...register(`users.${index}.email`, {
                            required: 'Email richiesta',
                            pattern: {
                                value: /\S+@\S+\.\S+/,
                                message: 'Email non valida'
                            }
                        })}
                        placeholder="Email"
                    />

                    <button type="button" onClick={() => remove(index)}>
                        Rimuovi
                    </button>
                </div>
            ))}

            <button
                type="button"
                onClick={() => append({ name: '', email: '' })}
            >
                Aggiungi Utente
            </button>

            <button type="submit">Invia</button>
        </form>
    );
}
\end{lstlisting}

\section{Formik}

Formik è un'altra libreria popolare per la gestione dei form in React.

\subsection{Setup Formik}

\begin{lstlisting}[language=bash, caption=Installazione Formik]
npm install formik yup
\end{lstlisting}

\subsection{Form Base con Formik}

\begin{lstlisting}[language=JavaScript, caption=Formik esempio base]
import { Formik, Form, Field, ErrorMessage } from 'formik';
import * as Yup from 'yup';

// Schema validazione con Yup
const validationSchema = Yup.object({
    username: Yup.string()
        .min(3, 'Minimo 3 caratteri')
        .required('Username richiesto'),
    email: Yup.string()
        .email('Email non valida')
        .required('Email richiesta'),
    password: Yup.string()
        .min(8, 'Minimo 8 caratteri')
        .matches(/[A-Z]/, 'Deve contenere maiuscola')
        .matches(/[a-z]/, 'Deve contenere minuscola')
        .matches(/\d/, 'Deve contenere numero')
        .required('Password richiesta'),
    confirmPassword: Yup.string()
        .oneOf([Yup.ref('password')], 'Le password non corrispondono')
        .required('Conferma password richiesta'),
    age: Yup.number()
        .min(18, 'Devi avere almeno 18 anni')
        .max(120, 'Età non valida')
        .required('Età richiesta'),
    terms: Yup.boolean()
        .oneOf([true], 'Devi accettare i termini')
});

function FormikExample() {
    const initialValues = {
        username: '',
        email: '',
        password: '',
        confirmPassword: '',
        age: '',
        terms: false
    };

    const handleSubmit = async (values, { setSubmitting, resetForm }) => {
        console.log('Form values:', values);
        await new Promise(resolve => setTimeout(resolve, 1000));
        console.log('Submitted!');
        resetForm();
        setSubmitting(false);
    };

    return (
        <Formik
            initialValues={initialValues}
            validationSchema={validationSchema}
            onSubmit={handleSubmit}
        >
            {({ isSubmitting, errors, touched }) => (
                <Form>
                    {/* Username */}
                    <div>
                        <label htmlFor="username">Username:</label>
                        <Field
                            id="username"
                            name="username"
                            type="text"
                        />
                        <ErrorMessage name="username" component="span" className="error" />
                    </div>

                    {/* Email */}
                    <div>
                        <label htmlFor="email">Email:</label>
                        <Field
                            id="email"
                            name="email"
                            type="email"
                        />
                        <ErrorMessage name="email" component="span" className="error" />
                    </div>

                    {/* Password */}
                    <div>
                        <label htmlFor="password">Password:</label>
                        <Field
                            id="password"
                            name="password"
                            type="password"
                        />
                        <ErrorMessage name="password" component="span" className="error" />
                    </div>

                    {/* Confirm Password */}
                    <div>
                        <label htmlFor="confirmPassword">Conferma Password:</label>
                        <Field
                            id="confirmPassword"
                            name="confirmPassword"
                            type="password"
                        />
                        <ErrorMessage name="confirmPassword" component="span" className="error" />
                    </div>

                    {/* Age */}
                    <div>
                        <label htmlFor="age">Età:</label>
                        <Field
                            id="age"
                            name="age"
                            type="number"
                        />
                        <ErrorMessage name="age" component="span" className="error" />
                    </div>

                    {/* Terms Checkbox */}
                    <div>
                        <label>
                            <Field
                                type="checkbox"
                                name="terms"
                            />
                            Accetto i termini e condizioni
                        </label>
                        <ErrorMessage name="terms" component="span" className="error" />
                    </div>

                    <button type="submit" disabled={isSubmitting}>
                        {isSubmitting ? 'Invio...' : 'Invia'}
                    </button>
                </Form>
            )}
        </Formik>
    );
}
\end{lstlisting}

\subsection{Formik con Custom Components}

\begin{lstlisting}[language=JavaScript, caption=Custom input components con Formik]
import { Formik, Form, Field, ErrorMessage } from 'formik';

// Custom Input Component
function TextInput({ label, ...props }) {
    return (
        <div className="form-field">
            <label htmlFor={props.id || props.name}>{label}</label>
            <Field {...props} />
            <ErrorMessage name={props.name} component="div" className="error" />
        </div>
    );
}

// Custom Select Component
function Select({ label, options, ...props }) {
    return (
        <div className="form-field">
            <label htmlFor={props.id || props.name}>{label}</label>
            <Field as="select" {...props}>
                <option value="">Seleziona...</option>
                {options.map(option => (
                    <option key={option.value} value={option.value}>
                        {option.label}
                    </option>
                ))}
            </Field>
            <ErrorMessage name={props.name} component="div" className="error" />
        </div>
    );
}

// Custom Checkbox Component
function Checkbox({ children, ...props }) {
    return (
        <div className="form-field">
            <label className="checkbox-label">
                <Field type="checkbox" {...props} />
                {children}
            </label>
            <ErrorMessage name={props.name} component="div" className="error" />
        </div>
    );
}

// Uso
function CustomComponentsForm() {
    return (
        <Formik
            initialValues={{
                name: '',
                email: '',
                country: '',
                terms: false
            }}
            onSubmit={(values) => console.log(values)}
        >
            <Form>
                <TextInput
                    label="Nome"
                    name="name"
                    type="text"
                />

                <TextInput
                    label="Email"
                    name="email"
                    type="email"
                />

                <Select
                    label="Paese"
                    name="country"
                    options={[
                        { value: 'IT', label: 'Italia' },
                        { value: 'US', label: 'USA' },
                        { value: 'UK', label: 'UK' }
                    ]}
                />

                <Checkbox name="terms">
                    Accetto i termini e condizioni
                </Checkbox>

                <button type="submit">Invia</button>
            </Form>
        </Formik>
    );
}
\end{lstlisting}

\section{Form Patterns Avanzati}

\subsection{Multi-Step Form}

\begin{lstlisting}[language=JavaScript, caption=Form multi-step]
function MultiStepForm() {
    const [step, setStep] = useState(1);
    const [formData, setFormData] = useState({
        // Step 1
        personalInfo: {
            firstName: '',
            lastName: '',
            email: ''
        },
        // Step 2
        address: {
            street: '',
            city: '',
            country: ''
        },
        // Step 3
        account: {
            username: '',
            password: ''
        }
    });

    const nextStep = () => setStep(prev => prev + 1);
    const prevStep = () => setStep(prev => prev - 1);

    const updateFormData = (section, data) => {
        setFormData(prev => ({
            ...prev,
            [section]: { ...prev[section], ...data }
        }));
    };

    const handleSubmit = () => {
        console.log('Final form data:', formData);
    };

    return (
        <div className="multi-step-form">
            {/* Progress Indicator */}
            <div className="progress">
                <div className={`step ${step >= 1 ? 'active' : ''}`}>1</div>
                <div className={`step ${step >= 2 ? 'active' : ''}`}>2</div>
                <div className={`step ${step >= 3 ? 'active' : ''}`}>3</div>
            </div>

            {/* Step 1: Personal Info */}
            {step === 1 && (
                <Step1
                    data={formData.personalInfo}
                    onNext={(data) => {
                        updateFormData('personalInfo', data);
                        nextStep();
                    }}
                />
            )}

            {/* Step 2: Address */}
            {step === 2 && (
                <Step2
                    data={formData.address}
                    onNext={(data) => {
                        updateFormData('address', data);
                        nextStep();
                    }}
                    onPrev={prevStep}
                />
            )}

            {/* Step 3: Account */}
            {step === 3 && (
                <Step3
                    data={formData.account}
                    onSubmit={(data) => {
                        updateFormData('account', data);
                        handleSubmit();
                    }}
                    onPrev={prevStep}
                />
            )}
        </div>
    );
}

// Step 1 Component
function Step1({ data, onNext }) {
    const [formData, setFormData] = useState(data);

    const handleSubmit = (e) => {
        e.preventDefault();
        onNext(formData);
    };

    return (
        <form onSubmit={handleSubmit}>
            <h2>Informazioni Personali</h2>
            <input
                value={formData.firstName}
                onChange={(e) => setFormData({ ...formData, firstName: e.target.value })}
                placeholder="Nome"
                required
            />
            <input
                value={formData.lastName}
                onChange={(e) => setFormData({ ...formData, lastName: e.target.value })}
                placeholder="Cognome"
                required
            />
            <input
                type="email"
                value={formData.email}
                onChange={(e) => setFormData({ ...formData, email: e.target.value })}
                placeholder="Email"
                required
            />
            <button type="submit">Avanti</button>
        </form>
    );
}

// Step 2 e Step 3 simili...
\end{lstlisting}

\subsection{Form con File Upload}

\begin{lstlisting}[language=JavaScript, caption=File upload form]
function FileUploadForm() {
    const [file, setFile] = useState(null);
    const [preview, setPreview] = useState(null);
    const [uploading, setUploading] = useState(false);
    const [progress, setProgress] = useState(0);

    const handleFileChange = (e) => {
        const selectedFile = e.target.files[0];

        if (selectedFile) {
            setFile(selectedFile);

            // Preview per immagini
            if (selectedFile.type.startsWith('image/')) {
                const reader = new FileReader();
                reader.onloadend = () => {
                    setPreview(reader.result);
                };
                reader.readAsDataURL(selectedFile);
            }
        }
    };

    const handleSubmit = async (e) => {
        e.preventDefault();

        if (!file) {
            alert('Seleziona un file');
            return;
        }

        setUploading(true);

        const formData = new FormData();
        formData.append('file', file);
        formData.append('description', 'File upload');

        try {
            const xhr = new XMLHttpRequest();

            // Progress tracking
            xhr.upload.addEventListener('progress', (e) => {
                if (e.lengthComputable) {
                    const percentComplete = (e.loaded / e.total) * 100;
                    setProgress(percentComplete);
                }
            });

            xhr.addEventListener('load', () => {
                if (xhr.status === 200) {
                    console.log('Upload completato!');
                    setFile(null);
                    setPreview(null);
                    setProgress(0);
                }
            });

            xhr.open('POST', '/api/upload');
            xhr.send(formData);
        } catch (error) {
            console.error('Errore upload:', error);
        } finally {
            setUploading(false);
        }
    };

    return (
        <form onSubmit={handleSubmit}>
            <div>
                <input
                    type="file"
                    onChange={handleFileChange}
                    accept="image/*"
                />
            </div>

            {preview && (
                <div>
                    <h3>Anteprima:</h3>
                    <img src={preview} alt="Preview" style={{ maxWidth: '300px' }} />
                </div>
            )}

            {file && (
                <div>
                    <p>File: {file.name}</p>
                    <p>Dimensione: {(file.size / 1024).toFixed(2)} KB</p>
                </div>
            )}

            {uploading && (
                <div>
                    <progress value={progress} max="100" />
                    <p>{progress.toFixed(0)}%</p>
                </div>
            )}

            <button type="submit" disabled={!file || uploading}>
                {uploading ? 'Caricamento...' : 'Carica'}
            </button>
        </form>
    );
}
\end{lstlisting}

\subsection{Form con Async Validation}

\begin{lstlisting}[language=JavaScript, caption=Validazione asincrona]
function AsyncValidationForm() {
    const [username, setUsername] = useState('');
    const [checking, setChecking] = useState(false);
    const [isAvailable, setIsAvailable] = useState(null);
    const debounceRef = useRef(null);

    // Check username availability
    const checkUsername = async (value) => {
        if (value.length < 3) {
            setIsAvailable(null);
            return;
        }

        setChecking(true);

        try {
            // Simula API call
            await new Promise(resolve => setTimeout(resolve, 1000));
            const response = await fetch(`/api/check-username?username=${value}`);
            const data = await response.json();

            setIsAvailable(data.available);
        } catch (error) {
            console.error('Error checking username:', error);
        } finally {
            setChecking(false);
        }
    };

    const handleUsernameChange = (e) => {
        const value = e.target.value;
        setUsername(value);
        setIsAvailable(null);

        // Debounce check
        clearTimeout(debounceRef.current);
        debounceRef.current = setTimeout(() => {
            checkUsername(value);
        }, 500);
    };

    const handleSubmit = (e) => {
        e.preventDefault();

        if (isAvailable) {
            console.log('Submitting username:', username);
        }
    };

    return (
        <form onSubmit={handleSubmit}>
            <div>
                <label>Username:</label>
                <input
                    value={username}
                    onChange={handleUsernameChange}
                    placeholder="Scegli username"
                />

                {checking && <span>Controllo disponibilità...</span>}

                {!checking && isAvailable === true && (
                    <span style={{ color: 'green' }}>✓ Disponibile</span>
                )}

                {!checking && isAvailable === false && (
                    <span style={{ color: 'red' }}>✗ Non disponibile</span>
                )}
            </div>

            <button type="submit" disabled={!isAvailable}>
                Registra
            </button>
        </form>
    );
}
\end{lstlisting}

\section{Best Practices}

\subsection{1. Accessibilità}

\begin{lstlisting}[language=JavaScript, caption=Form accessibili]
function AccessibleForm() {
    return (
        <form>
            {/* ✅ Label associata a input */}
            <label htmlFor="email">Email:</label>
            <input id="email" type="email" name="email" />

            {/* ✅ Aria attributes */}
            <input
                id="username"
                aria-label="Username"
                aria-required="true"
                aria-invalid={errors.username ? 'true' : 'false'}
                aria-describedby="username-error"
            />
            <span id="username-error" role="alert">
                {errors.username}
            </span>

            {/* ✅ Fieldset per raggruppamento */}
            <fieldset>
                <legend>Informazioni di contatto</legend>
                <input type="tel" name="phone" />
                <input type="email" name="email" />
            </fieldset>
        </form>
    );
}
\end{lstlisting}

\subsection{2. Performance}

\begin{lstlisting}[language=JavaScript, caption=Ottimizzazione performance]
// ✅ Debounce validation
const useDebounce = (value, delay) => {
    const [debouncedValue, setDebouncedValue] = useState(value);

    useEffect(() => {
        const timer = setTimeout(() => setDebouncedValue(value), delay);
        return () => clearTimeout(timer);
    }, [value, delay]);

    return debouncedValue;
};

// ✅ Memoizza callbacks
const handleSubmit = useCallback(async (data) => {
    await submitForm(data);
}, []);

// ✅ Controlled components solo quando necessario
// Se non hai bisogno di validazione real-time, usa uncontrolled
\end{lstlisting}

\subsection{3. Security}

\begin{lstlisting}[language=JavaScript, caption=Sicurezza form]
function SecureForm() {
    const handleSubmit = async (data) => {
        // ✅ Sanitize input prima di inviare
        const sanitized = {
            username: DOMPurify.sanitize(data.username),
            email: data.email.toLowerCase().trim()
        };

        // ✅ HTTPS sempre
        await fetch('https://api.example.com/submit', {
            method: 'POST',
            headers: {
                'Content-Type': 'application/json',
                // ✅ CSRF token
                'X-CSRF-Token': getCsrfToken()
            },
            body: JSON.stringify(sanitized)
        });
    };

    return <form onSubmit={handleSubmit}>...</form>;
}
\end{lstlisting}

\section{Conclusione}

In questo capitolo abbiamo esplorato:

\begin{itemize}
    \item Controlled vs Uncontrolled components
    \item Form base con validazione manuale
    \item React Hook Form per form performanti
    \item Formik per form complessi
    \item Pattern avanzati: multi-step, file upload, async validation
    \item Best practices: accessibilità, performance, security
\end{itemize}

Ora hai tutti gli strumenti per gestire form complessi in React!
