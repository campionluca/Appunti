\chapter{JSX e Componenti}

\section{Introduzione a JSX}

JSX (JavaScript XML) è un'estensione della sintassi JavaScript che permette di scrivere markup simile a HTML all'interno di JavaScript. È una delle caratteristiche più distintive di React.

\subsection{Cos'è JSX?}

JSX sembra HTML, ma è JavaScript:

\begin{lstlisting}[language=JavaScript, caption=Esempio JSX base]
// Questo è JSX
const element = <h1>Hello, World!</h1>;

// Viene trasformato in JavaScript puro:
const element = React.createElement('h1', null, 'Hello, World!');
\end{lstlisting}

\textbf{JSX NON è:}
\begin{itemize}
    \item HTML (anche se somiglia)
    \item Un template engine
    \item Obbligatorio (ma altamente raccomandato)
\end{itemize}

\textbf{JSX È:}
\begin{itemize}
    \item JavaScript con sintassi estesa
    \item Type-safe (con TypeScript)
    \item Più espressivo e leggibile
\end{itemize}

\subsection{Sintassi JSX Fondamentale}

\textbf{1. Espressioni JavaScript in JSX}

Usa le parentesi graffe \{\} per inserire espressioni JavaScript:

\begin{lstlisting}[language=JavaScript, caption=Espressioni in JSX]
function Greeting() {
    const name = 'Mario';
    const age = 25;
    const hobbies = ['coding', 'reading', 'gaming'];

    return (
        <div>
            {/* Variabili */}
            <h1>Ciao, {name}!</h1>

            {/* Operazioni */}
            <p>Tra 5 anni avrai {age + 5} anni</p>

            {/* Condizioni ternarie */}
            <p>Sei {age >= 18 ? 'maggiorenne' : 'minorenne'}</p>

            {/* Chiamate a funzioni */}
            <p>Nome uppercase: {name.toUpperCase()}</p>

            {/* Map per liste */}
            <ul>
                {hobbies.map((hobby, index) => (
                    <li key={index}>{hobby}</li>
                ))}
            </ul>

            {/* Template literals */}
            <p>{`${name} ha ${age} anni`}</p>
        </div>
    );
}
\end{lstlisting}

\textbf{2. Attributi in JSX}

Gli attributi seguono la convenzione camelCase:

\begin{lstlisting}[language=JavaScript, caption=Attributi JSX]
function AttributesExample() {
    const imageUrl = 'https://example.com/image.jpg';
    const altText = 'Una bella immagine';
    const isActive = true;

    return (
        <div>
            {/* className invece di class */}
            <div className="container"></div>

            {/* htmlFor invece di for */}
            <label htmlFor="input-id">Nome:</label>
            <input id="input-id" type="text" />

            {/* Attributi booleani */}
            <input type="checkbox" checked={isActive} />
            <input type="text" disabled />
            <button type="submit" autoFocus>Invia</button>

            {/* onClick invece di onclick */}
            <button onClick={() => console.log('Click!')}>
                Click me
            </button>

            {/* Style come oggetto */}
            <div style={{
                color: 'red',
                fontSize: '20px',
                backgroundColor: '#f0f0f0'
            }}>
                Testo stilizzato
            </div>

            {/* Attributi dinamici */}
            <img src={imageUrl} alt={altText} />

            {/* Data attributes */}
            <div data-user-id="123" data-role="admin">
                User info
            </div>

            {/* Spread attributes */}
            <input {...{type: 'text', placeholder: 'Nome'}} />
        </div>
    );
}
\end{lstlisting}

\textbf{3. Differenze HTML vs JSX}

\begin{center}
\begin{tabular}{|l|l|l|}
\hline
\textbf{HTML} & \textbf{JSX} & \textbf{Motivo} \\
\hline
class & className & class è keyword JS \\
\hline
for & htmlFor & for è keyword JS \\
\hline
onclick & onClick & camelCase convention \\
\hline
tabindex & tabIndex & camelCase convention \\
\hline
style="..." & style=\{\{\}\} & Oggetto JS \\
\hline
<!-- --> & \{/* */\} & Commenti JS \\
\hline
<br> & <br /> & Tag auto-chiudenti \\
\hline
\end{tabular}
\end{center}

\textbf{4. Rendering Condizionale}

\begin{lstlisting}[language=JavaScript, caption=Condizioni in JSX]
function ConditionalRendering({ isLoggedIn, username, role }) {
    // 1. If-else con ternario
    return (
        <div>
            {isLoggedIn ? (
                <h1>Benvenuto, {username}!</h1>
            ) : (
                <h1>Per favore, effettua il login</h1>
            )}
        </div>
    );

    // 2. Logical AND (&&) per rendering condizionale
    return (
        <div>
            {isLoggedIn && <h1>Benvenuto, {username}!</h1>}
            {role === 'admin' && <button>Pannello Admin</button>}
        </div>
    );

    // 3. Variabile intermedia
    let content;
    if (isLoggedIn) {
        content = <h1>Benvenuto, {username}!</h1>;
    } else {
        content = <h1>Per favore, effettua il login</h1>;
    }
    return <div>{content}</div>;

    // 4. Early return
    if (!isLoggedIn) {
        return <LoginForm />;
    }
    return <Dashboard username={username} />;

    // 5. Switch con oggetto
    const statusMessages = {
        loading: <Spinner />,
        error: <ErrorMessage />,
        success: <SuccessMessage />,
        idle: null
    };
    return statusMessages[status];
}
\end{lstlisting}

\textbf{5. Liste e Keys}

\begin{lstlisting}[language=JavaScript, caption=Rendering liste]
function ListRendering() {
    const users = [
        { id: 1, name: 'Mario', age: 25 },
        { id: 2, name: 'Laura', age: 30 },
        { id: 3, name: 'Giuseppe', age: 28 }
    ];

    return (
        <div>
            {/* ✅ Con key unica (id) */}
            <ul>
                {users.map(user => (
                    <li key={user.id}>
                        {user.name} - {user.age} anni
                    </li>
                ))}
            </ul>

            {/* ❌ Senza key (warning in console) */}
            <ul>
                {users.map(user => (
                    <li>{user.name}</li>
                ))}
            </ul>

            {/* ⚠️ Con index come key (evitare se possibile) */}
            <ul>
                {users.map((user, index) => (
                    <li key={index}>{user.name}</li>
                ))}
            </ul>

            {/* ✅ Liste complesse */}
            <div>
                {users
                    .filter(user => user.age >= 28)
                    .sort((a, b) => a.name.localeCompare(b.name))
                    .map(user => (
                        <UserCard key={user.id} user={user} />
                    ))
                }
            </div>
        </div>
    );
}
\end{lstlisting}

\textbf{Perché le Keys sono Importanti:}

\begin{lstlisting}[language=JavaScript, caption=Importanza delle keys]
// Senza keys corrette, React può confondersi
// nei riordinamenti

function TodoList() {
    const [todos, setTodos] = useState([
        { id: 1, text: 'Primo' },
        { id: 2, text: 'Secondo' },
        { id: 3, text: 'Terzo' }
    ]);

    // Se riordini e usi index come key
    // React potrebbe aggiornare l'elemento sbagliato

    // ❌ Male - usa index
    return (
        <ul>
            {todos.map((todo, index) => (
                <li key={index}>
                    <input type="checkbox" />
                    {todo.text}
                </li>
            ))}
        </ul>
    );

    // ✅ Bene - usa id univoco
    return (
        <ul>
            {todos.map(todo => (
                <li key={todo.id}>
                    <input type="checkbox" />
                    {todo.text}
                </li>
            ))}
        </ul>
    );
}
\end{lstlisting}

\subsection{JSX Avanzato}

\textbf{1. Fragments}

Raggruppa elementi senza aggiungere nodi DOM:

\begin{lstlisting}[language=JavaScript, caption=React Fragments]
import { Fragment } from 'react';

function Table() {
    // ❌ Aggiunge un div extra nel DOM
    return (
        <div>
            <td>Cella 1</td>
            <td>Cella 2</td>
        </div>
    );

    // ✅ Sintassi lunga
    return (
        <Fragment>
            <td>Cella 1</td>
            <td>Cella 2</td>
        </Fragment>
    );

    // ✅ Sintassi corta (più comune)
    return (
        <>
            <td>Cella 1</td>
            <td>Cella 2</td>
        </>
    );

    // ✅ Con key (necessario per liste)
    return items.map(item => (
        <Fragment key={item.id}>
            <td>{item.name}</td>
            <td>{item.value}</td>
        </Fragment>
    ));
}
\end{lstlisting}

\textbf{2. Spread Props}

\begin{lstlisting}[language=JavaScript, caption=Spread di props]
function Button({ type = 'button', className, ...rest }) {
    // Passa tutte le altre props al button
    return (
        <button
            type={type}
            className={`btn ${className}`}
            {...rest}
        />
    );
}

// Uso
function App() {
    return (
        <Button
            onClick={() => console.log('Click')}
            disabled
            aria-label="Submit button"
            data-testid="submit-btn"
        >
            Invia
        </Button>
    );
}
\end{lstlisting}

\textbf{3. Children come Function}

\begin{lstlisting}[language=JavaScript, caption=Render Props pattern]
function DataFetcher({ url, children }) {
    const [data, setData] = useState(null);
    const [loading, setLoading] = useState(true);

    useEffect(() => {
        fetch(url)
            .then(res => res.json())
            .then(data => {
                setData(data);
                setLoading(false);
            });
    }, [url]);

    // Children è una funzione
    return children({ data, loading });
}

// Uso
function App() {
    return (
        <DataFetcher url="/api/users">
            {({ data, loading }) => (
                loading ? <Spinner /> : <UserList users={data} />
            )}
        </DataFetcher>
    );
}
\end{lstlisting}

\section{Componenti Funzionali}

I componenti funzionali sono il modo moderno e raccomandato di creare componenti React.

\subsection{Anatomia di un Componente}

\begin{lstlisting}[language=JavaScript, caption=Componente funzionale completo]
// 1. Imports
import { useState, useEffect } from 'react';
import PropTypes from 'prop-types';
import './UserCard.css';

// 2. Componente
function UserCard({ user, onEdit, onDelete }) {
    // 3. State locale
    const [isExpanded, setIsExpanded] = useState(false);

    // 4. Effects
    useEffect(() => {
        console.log('UserCard montato');
        return () => console.log('UserCard smontato');
    }, []);

    // 5. Event handlers
    const handleToggle = () => {
        setIsExpanded(!isExpanded);
    };

    // 6. Computed values
    const fullName = `${user.firstName} ${user.lastName}`;
    const initials = `${user.firstName[0]}${user.lastName[0]}`;

    // 7. Render
    return (
        <div className="user-card">
            <div className="avatar">{initials}</div>

            <div className="info">
                <h3>{fullName}</h3>
                <p>{user.email}</p>

                {isExpanded && (
                    <div className="details">
                        <p>Ruolo: {user.role}</p>
                        <p>Registrato: {user.registeredAt}</p>
                    </div>
                )}
            </div>

            <div className="actions">
                <button onClick={handleToggle}>
                    {isExpanded ? 'Nascondi' : 'Mostra'}
                </button>
                <button onClick={() => onEdit(user.id)}>
                    Modifica
                </button>
                <button onClick={() => onDelete(user.id)}>
                    Elimina
                </button>
            </div>
        </div>
    );
}

// 8. PropTypes
UserCard.propTypes = {
    user: PropTypes.shape({
        id: PropTypes.number.isRequired,
        firstName: PropTypes.string.isRequired,
        lastName: PropTypes.string.isRequired,
        email: PropTypes.string.isRequired,
        role: PropTypes.string,
        registeredAt: PropTypes.string
    }).isRequired,
    onEdit: PropTypes.func.isRequired,
    onDelete: PropTypes.func.isRequired
};

// 9. Export
export default UserCard;
\end{lstlisting}

\subsection{Props: Comunicazione tra Componenti}

Props (properties) sono il meccanismo per passare dati dai componenti parent ai componenti child.

\textbf{1. Props Base}

\begin{lstlisting}[language=JavaScript, caption=Props fondamentali]
// Parent component
function App() {
    const user = {
        name: 'Mario Rossi',
        age: 30,
        email: 'mario@example.com'
    };

    return (
        <div>
            {/* Passaggio props */}
            <Greeting name="Mario" />
            <UserInfo user={user} />
            <Button label="Click me" onClick={() => alert('Clicked!')} />
        </div>
    );
}

// Child components
function Greeting({ name }) {
    return <h1>Ciao, {name}!</h1>;
}

function UserInfo({ user }) {
    return (
        <div>
            <p>Nome: {user.name}</p>
            <p>Età: {user.age}</p>
            <p>Email: {user.email}</p>
        </div>
    );
}

function Button({ label, onClick }) {
    return <button onClick={onClick}>{label}</button>;
}
\end{lstlisting}

\textbf{2. Props Destructuring}

\begin{lstlisting}[language=JavaScript, caption=Destructuring di props]
// ❌ Senza destructuring
function UserCard(props) {
    return (
        <div>
            <h3>{props.name}</h3>
            <p>{props.email}</p>
        </div>
    );
}

// ✅ Con destructuring (raccomandato)
function UserCard({ name, email }) {
    return (
        <div>
            <h3>{name}</h3>
            <p>{email}</p>
        </div>
    );
}

// ✅ Con valori di default
function UserCard({
    name = 'Guest',
    email = 'No email',
    role = 'user',
    isActive = true
}) {
    return (
        <div>
            <h3>{name}</h3>
            <p>{email}</p>
            <span>{role}</span>
            <span>{isActive ? 'Active' : 'Inactive'}</span>
        </div>
    );
}

// ✅ Rest props
function Button({ children, className, ...rest }) {
    return (
        <button className={`btn ${className}`} {...rest}>
            {children}
        </button>
    );
}
\end{lstlisting}

\textbf{3. Children Prop}

\begin{lstlisting}[language=JavaScript, caption=Children prop]
// Children come contenuto
function Card({ children, title }) {
    return (
        <div className="card">
            <h2>{title}</h2>
            <div className="card-body">
                {children}
            </div>
        </div>
    );
}

// Uso
function App() {
    return (
        <Card title="User Info">
            <p>Nome: Mario</p>
            <p>Email: mario@example.com</p>
            <button>Contatta</button>
        </Card>
    );
}

// Children multipli
function Layout({ header, sidebar, children, footer }) {
    return (
        <div className="layout">
            <header>{header}</header>
            <aside>{sidebar}</aside>
            <main>{children}</main>
            <footer>{footer}</footer>
        </div>
    );
}

// Uso
function App() {
    return (
        <Layout
            header={<Header />}
            sidebar={<Sidebar />}
            footer={<Footer />}
        >
            <h1>Contenuto principale</h1>
            <p>Lorem ipsum...</p>
        </Layout>
    );
}
\end{lstlisting}

\textbf{4. Callback Props}

\begin{lstlisting}[language=JavaScript, caption=Props come callbacks]
// Parent passa funzioni ai child
function App() {
    const [count, setCount] = useState(0);

    const handleIncrement = () => {
        setCount(count + 1);
    };

    const handleDecrement = () => {
        setCount(count - 1);
    };

    const handleReset = () => {
        setCount(0);
    };

    return (
        <div>
            <Display value={count} />
            <Controls
                onIncrement={handleIncrement}
                onDecrement={handleDecrement}
                onReset={handleReset}
            />
        </div>
    );
}

function Display({ value }) {
    return <h1>Contatore: {value}</h1>;
}

function Controls({ onIncrement, onDecrement, onReset }) {
    return (
        <div>
            <button onClick={onDecrement}>-</button>
            <button onClick={onReset}>Reset</button>
            <button onClick={onIncrement}>+</button>
        </div>
    );
}
\end{lstlisting}

\subsection{Composizione di Componenti}

La composizione è il pattern principale per costruire UI complesse in React.

\textbf{1. Composizione Base}

\begin{lstlisting}[language=JavaScript, caption=Composizione semplice]
// Componenti atomici
function Avatar({ src, alt }) {
    return <img src={src} alt={alt} className="avatar" />;
}

function UserName({ name }) {
    return <h3 className="username">{name}</h3>;
}

function UserEmail({ email }) {
    return <p className="email">{email}</p>;
}

// Componente composto
function UserCard({ user }) {
    return (
        <div className="user-card">
            <Avatar src={user.avatar} alt={user.name} />
            <div>
                <UserName name={user.name} />
                <UserEmail email={user.email} />
            </div>
        </div>
    );
}

// Lista di UserCard
function UserList({ users }) {
    return (
        <div className="user-list">
            {users.map(user => (
                <UserCard key={user.id} user={user} />
            ))}
        </div>
    );
}
\end{lstlisting}

\textbf{2. Container/Presentational Pattern}

\begin{lstlisting}[language=JavaScript, caption=Container e Presentational components]
// Presentational Component (UI pura)
function TodoListView({ todos, onToggle, onDelete }) {
    return (
        <ul className="todo-list">
            {todos.map(todo => (
                <TodoItem
                    key={todo.id}
                    todo={todo}
                    onToggle={onToggle}
                    onDelete={onDelete}
                />
            ))}
        </ul>
    );
}

function TodoItem({ todo, onToggle, onDelete }) {
    return (
        <li className={todo.completed ? 'completed' : ''}>
            <input
                type="checkbox"
                checked={todo.completed}
                onChange={() => onToggle(todo.id)}
            />
            <span>{todo.text}</span>
            <button onClick={() => onDelete(todo.id)}>Delete</button>
        </li>
    );
}

// Container Component (logica)
function TodoListContainer() {
    const [todos, setTodos] = useState([]);
    const [loading, setLoading] = useState(true);

    useEffect(() => {
        fetchTodos().then(data => {
            setTodos(data);
            setLoading(false);
        });
    }, []);

    const handleToggle = (id) => {
        setTodos(todos.map(todo =>
            todo.id === id
                ? { ...todo, completed: !todo.completed }
                : todo
        ));
    };

    const handleDelete = (id) => {
        setTodos(todos.filter(todo => todo.id !== id));
    };

    if (loading) return <Spinner />;

    return (
        <TodoListView
            todos={todos}
            onToggle={handleToggle}
            onDelete={handleDelete}
        />
    );
}
\end{lstlisting}

\textbf{3. Specializzazione}

\begin{lstlisting}[language=JavaScript, caption=Componenti specializzati]
// Componente generico
function Dialog({ title, message, onClose, children }) {
    return (
        <div className="dialog">
            <div className="dialog-header">
                <h2>{title}</h2>
                <button onClick={onClose}>×</button>
            </div>
            <div className="dialog-body">
                {message && <p>{message}</p>}
                {children}
            </div>
        </div>
    );
}

// Componenti specializzati
function WelcomeDialog({ onClose }) {
    return (
        <Dialog
            title="Benvenuto"
            message="Grazie per esserti registrato!"
            onClose={onClose}
        />
    );
}

function ConfirmDialog({ title, message, onConfirm, onCancel }) {
    return (
        <Dialog title={title} message={message} onClose={onCancel}>
            <div className="dialog-actions">
                <button onClick={onCancel}>Annulla</button>
                <button onClick={onConfirm}>Conferma</button>
            </div>
        </Dialog>
    );
}

function AlertDialog({ message, onClose }) {
    return (
        <Dialog title="Attenzione" message={message} onClose={onClose}>
            <button onClick={onClose}>OK</button>
        </Dialog>
    );
}
\end{lstlisting}

\textbf{4. Compound Components}

\begin{lstlisting}[language=JavaScript, caption=Compound Components pattern]
// Tabs compound component
function Tabs({ children, defaultActive = 0 }) {
    const [activeIndex, setActiveIndex] = useState(defaultActive);

    return (
        <div className="tabs">
            {React.Children.map(children, (child, index) => {
                return React.cloneElement(child, {
                    isActive: index === activeIndex,
                    onClick: () => setActiveIndex(index),
                    index
                });
            })}
        </div>
    );
}

function Tab({ label, children, isActive, onClick }) {
    return (
        <div className={`tab ${isActive ? 'active' : ''}`}>
            <button onClick={onClick}>{label}</button>
            {isActive && <div className="tab-content">{children}</div>}
        </div>
    );
}

// Uso
function App() {
    return (
        <Tabs defaultActive={1}>
            <Tab label="Tab 1">
                <p>Contenuto Tab 1</p>
            </Tab>
            <Tab label="Tab 2">
                <p>Contenuto Tab 2</p>
            </Tab>
            <Tab label="Tab 3">
                <p>Contenuto Tab 3</p>
            </Tab>
        </Tabs>
    );
}
\end{lstlisting}

\section{Component Tree}

\subsection{Diagramma Component Tree}

Esempio di albero componenti per un'applicazione TODO:

\begin{verbatim}
App
├── Header
│   ├── Logo
│   └── UserMenu
│       ├── Avatar
│       └── Dropdown
│           ├── MenuItem (Profile)
│           ├── MenuItem (Settings)
│           └── MenuItem (Logout)
│
├── TodoApp
│   ├── TodoInput
│   │   ├── Input
│   │   └── Button (Add)
│   │
│   ├── FilterBar
│   │   ├── FilterButton (All)
│   │   ├── FilterButton (Active)
│   │   └── FilterButton (Completed)
│   │
│   ├── TodoList
│   │   ├── TodoItem (x5)
│   │   │   ├── Checkbox
│   │   │   ├── Text
│   │   │   └── Button (Delete)
│   │   └── EmptyState (se vuoto)
│   │
│   └── TodoStats
│       ├── Stat (Total)
│       ├── Stat (Completed)
│       └── Stat (Pending)
│
└── Footer
    ├── Copyright
    └── Links
        ├── Link (Privacy)
        └── Link (Terms)
\end{verbatim}

\subsection{Implementazione Component Tree}

\begin{lstlisting}[language=JavaScript, caption=Implementazione albero completo]
// App - Root component
function App() {
    const [user, setUser] = useState({
        name: 'Mario Rossi',
        avatar: '/avatar.jpg'
    });

    return (
        <div className="app">
            <Header user={user} onLogout={() => setUser(null)} />
            <TodoApp />
            <Footer />
        </div>
    );
}

// Header component
function Header({ user, onLogout }) {
    return (
        <header className="header">
            <Logo />
            <UserMenu user={user} onLogout={onLogout} />
        </header>
    );
}

function Logo() {
    return <div className="logo">📝 TodoApp</div>;
}

function UserMenu({ user, onLogout }) {
    const [isOpen, setIsOpen] = useState(false);

    return (
        <div className="user-menu">
            <Avatar src={user.avatar} onClick={() => setIsOpen(!isOpen)} />
            {isOpen && (
                <Dropdown>
                    <MenuItem label="Profile" onClick={() => {}} />
                    <MenuItem label="Settings" onClick={() => {}} />
                    <MenuItem label="Logout" onClick={onLogout} />
                </Dropdown>
            )}
        </div>
    );
}

// TodoApp component
function TodoApp() {
    const [todos, setTodos] = useState([]);
    const [filter, setFilter] = useState('all');

    const addTodo = (text) => {
        setTodos([...todos, {
            id: Date.now(),
            text,
            completed: false
        }]);
    };

    const toggleTodo = (id) => {
        setTodos(todos.map(todo =>
            todo.id === id
                ? { ...todo, completed: !todo.completed }
                : todo
        ));
    };

    const deleteTodo = (id) => {
        setTodos(todos.filter(todo => todo.id !== id));
    };

    const filteredTodos = todos.filter(todo => {
        if (filter === 'active') return !todo.completed;
        if (filter === 'completed') return todo.completed;
        return true;
    });

    return (
        <main className="todo-app">
            <TodoInput onAdd={addTodo} />
            <FilterBar filter={filter} onFilterChange={setFilter} />
            <TodoList
                todos={filteredTodos}
                onToggle={toggleTodo}
                onDelete={deleteTodo}
            />
            <TodoStats todos={todos} />
        </main>
    );
}

// Componenti foglia (leaf components)
function TodoInput({ onAdd }) {
    const [text, setText] = useState('');

    const handleSubmit = (e) => {
        e.preventDefault();
        if (text.trim()) {
            onAdd(text);
            setText('');
        }
    };

    return (
        <form onSubmit={handleSubmit} className="todo-input">
            <Input
                value={text}
                onChange={setText}
                placeholder="Nuovo task..."
            />
            <Button type="submit">Aggiungi</Button>
        </form>
    );
}

function TodoList({ todos, onToggle, onDelete }) {
    if (todos.length === 0) {
        return <EmptyState />;
    }

    return (
        <ul className="todo-list">
            {todos.map(todo => (
                <TodoItem
                    key={todo.id}
                    todo={todo}
                    onToggle={onToggle}
                    onDelete={onDelete}
                />
            ))}
        </ul>
    );
}

function TodoStats({ todos }) {
    const total = todos.length;
    const completed = todos.filter(t => t.completed).length;
    const pending = total - completed;

    return (
        <div className="todo-stats">
            <Stat label="Totale" value={total} />
            <Stat label="Completati" value={completed} />
            <Stat label="Da fare" value={pending} />
        </div>
    );
}
\end{lstlisting}

\section{Best Practices}

\subsection{1. Naming e Organizzazione}

\begin{lstlisting}[language=JavaScript, caption=Convenzioni di naming]
// ✅ Componenti: PascalCase
function UserProfile() {}
function TodoList() {}

// ✅ File: stesso nome del componente
// UserProfile.jsx
// TodoList.jsx

// ✅ Props: camelCase descrittivo
function Button({ onClick, isDisabled, primaryColor }) {}

// ✅ Event handlers: handle + Event
const handleClick = () => {};
const handleSubmit = () => {};
const handleInputChange = () => {};

// ✅ Boolean props: is/has/should prefix
const isLoading = false;
const hasError = true;
const shouldRender = true;

// ❌ Evitare nomi generici
function Component() {} // Troppo generico
function MyComponent() {} // "My" non aggiunge informazioni
\end{lstlisting}

\subsection{2. Componenti Piccoli e Focalizzati}

\begin{lstlisting}[language=JavaScript, caption=Single Responsibility]
// ❌ Componente che fa troppo
function UserDashboard() {
    // 100+ righe di logica mista
    // Fetching dati, gestione form, rendering complesso
}

// ✅ Dividi responsabilità
function UserDashboard() {
    return (
        <div>
            <UserStats />
            <UserActivity />
            <UserSettings />
        </div>
    );
}

function UserStats() {
    // Solo statistiche
}

function UserActivity() {
    // Solo attività recente
}

function UserSettings() {
    // Solo impostazioni
}
\end{lstlisting}

\subsection{3. Props Validation}

\begin{lstlisting}[language=JavaScript, caption=PropTypes per validazione]
import PropTypes from 'prop-types';

function UserCard({ user, onEdit, onDelete, isSelected }) {
    // ...
}

UserCard.propTypes = {
    user: PropTypes.shape({
        id: PropTypes.number.isRequired,
        name: PropTypes.string.isRequired,
        email: PropTypes.string.isRequired,
        avatar: PropTypes.string,
        role: PropTypes.oneOf(['admin', 'user', 'guest'])
    }).isRequired,
    onEdit: PropTypes.func.isRequired,
    onDelete: PropTypes.func,
    isSelected: PropTypes.bool
};

UserCard.defaultProps = {
    isSelected: false,
    onDelete: () => {}
};
\end{lstlisting}

\subsection{4. Evitare Prop Drilling Eccessivo}

\begin{lstlisting}[language=JavaScript, caption=Problema prop drilling]
// ❌ Prop drilling eccessivo
function App() {
    const [theme, setTheme] = useState('light');
    return <Layout theme={theme} setTheme={setTheme} />;
}

function Layout({ theme, setTheme }) {
    return <Sidebar theme={theme} setTheme={setTheme} />;
}

function Sidebar({ theme, setTheme }) {
    return <ThemeToggle theme={theme} setTheme={setTheme} />;
}

function ThemeToggle({ theme, setTheme }) {
    return <button onClick={() => setTheme(/* ... */)}>Toggle</button>;
}

// ✅ Usa Context (vedi capitoli successivi)
const ThemeContext = createContext();

function App() {
    const [theme, setTheme] = useState('light');
    return (
        <ThemeContext.Provider value={{ theme, setTheme }}>
            <Layout />
        </ThemeContext.Provider>
    );
}

function ThemeToggle() {
    const { theme, setTheme } = useContext(ThemeContext);
    return <button onClick={() => setTheme(/* ... */)}>Toggle</button>;
}
\end{lstlisting}

\section{Conclusione}

In questo capitolo abbiamo coperto:

\begin{itemize}
    \item Sintassi JSX completa e avanzata
    \item Componenti funzionali moderni
    \item Props e comunicazione tra componenti
    \item Patterns di composizione
    \item Component tree e architettura
    \item Best practices fondamentali
\end{itemize}

Ora sei pronto per approfondire props, state e la gestione dei dati!
