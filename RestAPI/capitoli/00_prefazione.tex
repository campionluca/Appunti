\chapter*{Prefazione}
\addcontentsline{toc}{chapter}{Prefazione}

\section*{A chi si rivolge questo manuale}

Questi appunti sono stati pensati per sviluppatori, studenti e professionisti IT che desiderano approfondire la progettazione e implementazione di REST API moderne e scalabili. Il manuale copre dai principi fondamentali del REST alle best practices professionali per la costruzione di API robuste, sicure e ben documentate.

\section*{Struttura del corso}

Il corso è organizzato in 12 capitoli che coprono l'intero ciclo di progettazione e sviluppo di REST API:

\textbf{Parte I - Fondamenti REST} (Capitoli 1-3)
\begin{itemize}
    \item Principi architetturali REST e Richardson Maturity Model
    \item Confronto REST vs SOAP vs GraphQL
    \item Metodi HTTP: GET, POST, PUT, PATCH, DELETE
    \item Proprietà di idempotenza e safety
    \item Status codes HTTP completi (1xx-5xx) e loro utilizzo appropriato
\end{itemize}

\textbf{Parte II - Design e Struttura} (Capitoli 4-6)
\begin{itemize}
    \item Resource design: URI patterns, naming conventions
    \item Nesting, filtering, sorting e searching
    \item Formati dati: JSON, HAL, JSON:API
    \item Content negotiation e hypermedia
    \item Strategie di versioning: URI, header, semantic versioning
\end{itemize}

\textbf{Parte III - Sicurezza e Performance} (Capitoli 7-9)
\begin{itemize}
    \item Autenticazione: Basic Auth, API Keys, OAuth 2.0, JWT
    \item Rate limiting e throttling
    \item Pagination, filtering avanzato, caching
    \item CORS e security headers
\end{itemize}

\textbf{Parte IV - Professionalizzazione} (Capitoli 10-12)
\begin{itemize}
    \item Error handling standardizzato (Problem Details RFC 7807)
    \item Documentazione con OpenAPI/Swagger
    \item Best practices: HATEOAS, testing, monitoring
    \item Casi di studio reali
\end{itemize}

\section*{Prerequisiti}

Per affrontare questo corso è consigliabile avere:
\begin{itemize}
    \item Conoscenza di base del protocollo HTTP
    \item Familiarità con JSON e formati di dati strutturati
    \item Esperienza con almeno un linguaggio di programmazione (Python, Java, JavaScript, PHP)
    \item Comprensione dei concetti di client-server e web services
    \item (Opzionale) Nozioni di database relazionali
\end{itemize}

\section*{Strumenti necessari}

\textbf{Tool per testing e sviluppo}:
\begin{itemize}
    \item \textbf{Postman}: Client API completo per testing e documentazione
    \item \textbf{Insomnia}: Alternative a Postman, focus su GraphQL e REST
    \item \textbf{curl}: Command-line tool per richieste HTTP
    \item \textbf{HTTPie}: CLI user-friendly per API testing
    \item \textbf{jq}: Processore JSON da command line
\end{itemize}

\textbf{Framework e runtime}:
\begin{itemize}
    \item \textbf{Node.js + Express}: Framework JavaScript/TypeScript
    \item \textbf{Python + FastAPI/Flask}: Framework Python moderni
    \item \textbf{Java + Spring Boot}: Enterprise Java framework
    \item \textbf{PHP + Laravel/Symfony}: Framework PHP per API
    \item \textbf{ASP.NET Core}: Framework Microsoft .NET
\end{itemize}

\textbf{Documentazione e design}:
\begin{itemize}
    \item \textbf{Swagger Editor}: Editor OpenAPI online
    \item \textbf{Swagger UI}: Interfaccia interattiva per API docs
    \item \textbf{Stoplight Studio}: Design-first API development
    \item \textbf{Redoc}: Generatore documentazione da OpenAPI
\end{itemize}

\textbf{Testing e monitoring}:
\begin{itemize}
    \item \textbf{Newman}: Test runner per Postman collections
    \item \textbf{REST Assured}: Testing library Java
    \item \textbf{Gatling/K6}: Load testing e performance
    \item \textbf{Datadog/New Relic}: APM e monitoring
\end{itemize}

\section*{Come studiare}

Per ottenere il massimo da questi appunti:

\begin{enumerate}
    \item \textbf{Comprendi i principi}: Il REST è prima di tutto un'architettura
    \item \textbf{Testa ogni esempio}: Usa Postman/curl per provare le richieste
    \item \textbf{Analizza gli status codes}: Impara quando usare 200 vs 201 vs 204
    \item \textbf{Progetta prima di implementare}: Disegna la struttura delle risorse
    \item \textbf{Leggi le specifiche}: RFC HTTP, OpenAPI spec, JSON:API
    \item \textbf{Studia API reali}: GitHub, Stripe, Twitter API come reference
    \item \textbf{Implementa progetti}: Crea una API completa end-to-end
    \item \textbf{Documenta sempre}: API senza documentazione è API inutilizzabile
\end{enumerate}

\begin{tcolorbox}[title=Nota importante]
Questo manuale segue gli standard \textbf{HTTP/1.1 RFC 7231-7235}, \textbf{REST RFC 6570}, \textbf{OpenAPI 3.1}, e \textbf{JSON:API 1.1}. Gli esempi sono language-agnostic ma includono snippet in Python, JavaScript e PHP quando necessario per chiarezza.
\end{tcolorbox}

\section*{Convenzioni tipografiche}

Nel testo vengono utilizzate le seguenti convenzioni:

\begin{itemize}
    \item \texttt{Codice HTTP/JSON}: Request, response e payload in monospace
    \item \textbf{Metodi HTTP}: In grassetto (GET, POST, PUT, DELETE)
    \item \textit{Header HTTP}: In corsivo (Content-Type, Authorization)
    \item \verb|/api/resources|: URI endpoints e path parameters
    \item Box colorati: Best practices, Errori comuni, Attenzioni, Esempi
\end{itemize}

\textbf{Format esempio HTTP request/response}:
\begin{lstlisting}
GET /api/v1/users/123 HTTP/1.1
Host: api.example.com
Accept: application/json
Authorization: Bearer eyJhbGc...

HTTP/1.1 200 OK
Content-Type: application/json

{
  "id": 123,
  "name": "Mario Rossi"
}
\end{lstlisting}

\section*{API di riferimento studiate}

Vengono analizzate come case study le seguenti API pubbliche:

\begin{itemize}
    \item \textbf{GitHub REST API v3}: Design classico e maturo
    \item \textbf{Stripe API}: Eccellente developer experience
    \item \textbf{Twitter API v2}: Filtering e pagination avanzata
    \item \textbf{OpenWeatherMap API}: Esempio di API semplice e chiara
    \item \textbf{JSONPlaceholder}: Mock API per testing e prototyping
    \item \textbf{REST Countries}: Dataset geografici RESTful
\end{itemize}

\section*{Sito web e risorse}

Materiale aggiuntivo disponibile su:
\begin{itemize}
    \item Repository GitHub: \url{https://github.com/campionluca/Appunti}
    \item Collezioni Postman scaricabili
    \item Specifica OpenAPI 3.1 completa
    \item Mock server per testing
    \item Esempi di implementazione in vari linguaggi
\end{itemize}

\textbf{Risorse online consigliate}:
\begin{itemize}
    \item \url{https://restfulapi.net/} - Tutorial e best practices
    \item \url{https://www.ietf.org/rfc/rfc7231.txt} - HTTP/1.1 Specification
    \item \url{https://jsonapi.org/} - JSON:API specification
    \item \url{https://swagger.io/specification/} - OpenAPI Specification
    \item \url{https://developer.mozilla.org/en-US/docs/Web/HTTP} - MDN HTTP docs
\end{itemize}

\section*{Ringraziamenti}

Si ringrazia l'Istituto Tecnico Antonio Scarpa per il supporto nella realizzazione di questo materiale didattico. Un ringraziamento speciale alla community open source e agli autori delle specifiche REST, HTTP e OpenAPI.

\vspace{1cm}

\begin{flushright}
\textit{Prof. Luca Campion}\\
Novembre 2025
\end{flushright}

\section*{Note sulla versione}

\textbf{Versione 1.0} - Novembre 2025
\begin{itemize}
    \item Prima release completa
    \item 12 capitoli + 2 appendici
    \item Coverage: REST principles, HTTP methods, status codes, resource design
    \item Versioning, authentication, rate limiting, pagination
    \item Error handling, OpenAPI documentation, best practices
    \item Esempi completi con JSON request/response
    \item Diagrammi HTTP flow con TikZ
    \item Case study: GitHub, Stripe, Twitter API
\end{itemize}

\textbf{Aggiornamenti previsti}:
\begin{itemize}
    \item GraphQL comparison approfondita
    \item WebSocket integration pattern
    \item gRPC e HTTP/2 considerations
    \item API Gateway patterns
    \item Microservices e distributed tracing
\end{itemize}

\section*{Feedback e contributi}

Questo è un documento in evoluzione. Segnalazioni di errori, suggerimenti e contributi sono benvenuti:
\begin{itemize}
    \item Issue tracker GitHub: \url{https://github.com/campionluca/Appunti/issues}
    \item Email: \texttt{luca.campion@example.com}
    \item Pull requests per correzioni e miglioramenti
\end{itemize}

\section*{Licenza}

Questo materiale è rilasciato sotto licenza \textbf{Creative Commons BY-NC-SA 4.0}:
\begin{itemize}
    \item Attribuzione: citare l'autore originale
    \item Non commerciale: uso libero per didattica e studio
    \item Condividi allo stesso modo: derivati con stessa licenza
\end{itemize}

\vspace{0.5cm}

\begin{center}
\textit{Buono studio e buon design di API RESTful!}
\end{center}
