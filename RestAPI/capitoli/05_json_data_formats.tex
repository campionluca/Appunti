\chapter{JSON e Formati Dati}

\begin{tcolorbox}[title=Mappa del capitolo]
JSON fundamentals, Response structure, Envelope vs no-envelope, HAL (Hypertext Application Language), JSON:API specification, Siren, Collection+JSON, Content negotiation, Media types, Accept header, Esempi pratici completi, Best practices.
\end{tcolorbox}

\section*{Obiettivi di apprendimento}
\begin{itemize}
  \item Strutturare JSON response consistenti e leggibili
  \item Comprendere HAL per hypermedia
  \item Applicare JSON:API specification
  \item Implementare content negotiation
  \item Scegliere formato appropriato per use case
  \item Costruire API auto-documentanti con hypermedia
\end{itemize}

\section{JSON Fundamentals}

\subsection{Perché JSON}

\textbf{JSON} (JavaScript Object Notation) è formato standard per REST API moderne.

\textbf{Vantaggi}:
\begin{itemize}
    \item \textbf{Leggibile}: Human-readable
    \item \textbf{Compatto}: Meno verbose di XML
    \item \textbf{Type-safe}: String, number, boolean, null, array, object
    \item \textbf{Universale}: Supporto nativo in tutti linguaggi
    \item \textbf{Parsing veloce}: Performance migliori di XML
\end{itemize}

\textbf{Tipi JSON}:
\begin{lstlisting}[caption=JSON data types]
{
  "string": "Hello World",
  "number": 42,
  "float": 3.14159,
  "boolean": true,
  "null_value": null,
  "array": [1, 2, 3],
  "object": {
    "nested": "value"
  }
}
\end{lstlisting}

\subsection{JSON vs XML}

\begin{lstlisting}[caption=Same data: JSON vs XML]
# JSON (conciso)
{
  "id": 123,
  "name": "Mario Rossi",
  "email": "mario@example.com"
}

# XML (verbose)
<?xml version="1.0"?>
<user>
  <id>123</id>
  <name>Mario Rossi</name>
  <email>mario@example.com</email>
</user>
\end{lstlisting}

\textbf{JSON}: 87 bytes, \textbf{XML}: 135 bytes (55\% più grande)

\section{Response Structure}

\subsection{Envelope Pattern}

\textbf{Envelope}: Wrapping data in container object.

\begin{lstlisting}[caption=Envelope pattern]
GET /api/users/123 HTTP/1.1

HTTP/1.1 200 OK
Content-Type: application/json

{
  "status": "success",
  "data": {
    "id": 123,
    "name": "Mario Rossi",
    "email": "mario@example.com"
  },
  "meta": {
    "timestamp": "2025-11-15T15:00:00Z",
    "version": "1.0"
  }
}
\end{lstlisting}

\textbf{Pro envelope}:
\begin{itemize}
    \item Metadata separato da data
    \item Consistenza structure anche per errori
    \item Extensibility
\end{itemize}

\textbf{Contro envelope}:
\begin{itemize}
    \item Overhead extra nesting
    \item Ridondanza (status già in HTTP status code)
\end{itemize}

\subsection{No Envelope (Naked Response)}

\textbf{No envelope}: Data direttamente nel root.

\begin{lstlisting}[caption=No envelope - naked response]
GET /api/users/123 HTTP/1.1

HTTP/1.1 200 OK
Content-Type: application/json

{
  "id": 123,
  "name": "Mario Rossi",
  "email": "mario@example.com"
}
\end{lstlisting}

\textbf{Pro no-envelope}:
\begin{itemize}
    \item Più conciso
    \item HTTP headers per metadata
    \item Meno parsing client-side
\end{itemize}

\begin{tcolorbox}[colback=blue!5, colframe=blue!60, title=Raccomandazione]
\textbf{Modern REST API}: Preferire \textbf{no envelope} per singole risorse, \textbf{envelope leggero} per collections.

\textbf{Singola risorsa}: Data in root

\textbf{Collection}: Wrapper minimo per metadata pagination
\end{tcolorbox}

\subsection{Collection Response Structure}

\begin{lstlisting}[caption=Collection con metadata]
GET /api/users?page=1&limit=10 HTTP/1.1

HTTP/1.1 200 OK
Content-Type: application/json
X-Total-Count: 150

{
  "data": [
    {
      "id": 1,
      "name": "Mario Rossi",
      "email": "mario@example.com"
    },
    {
      "id": 2,
      "name": "Luigi Verdi",
      "email": "luigi@example.com"
    }
  ],
  "pagination": {
    "page": 1,
    "per_page": 10,
    "total": 150,
    "total_pages": 15
  },
  "links": {
    "self": "/api/users?page=1&limit=10",
    "first": "/api/users?page=1&limit=10",
    "last": "/api/users?page=15&limit=10",
    "next": "/api/users?page=2&limit=10"
  }
}
\end{lstlisting}

\section{HAL - Hypertext Application Language}

\subsection{Cos'è HAL}

\textbf{HAL} (RFC 8288) è formato hypermedia che aggiunge link navigabili a JSON/XML.

\textbf{Media type}: \texttt{application/hal+json}

\textbf{Elementi chiave}:
\begin{itemize}
    \item \texttt{\_links}: Collegamenti ipertestuali ad altre risorse
    \item \texttt{\_embedded}: Risorse correlate embedded nella response
    \item Resource properties: Dati effettivi risorsa
\end{itemize}

\subsection{HAL Structure}

\begin{lstlisting}[caption=HAL - Single resource]
GET /api/users/123 HTTP/1.1
Accept: application/hal+json

HTTP/1.1 200 OK
Content-Type: application/hal+json

{
  "id": 123,
  "name": "Mario Rossi",
  "email": "mario@example.com",
  "role": "admin",
  "_links": {
    "self": {
      "href": "/api/users/123"
    },
    "orders": {
      "href": "/api/users/123/orders"
    },
    "profile": {
      "href": "/api/users/123/profile"
    }
  }
}
\end{lstlisting}

\subsection{HAL Embedded Resources}

\begin{lstlisting}[caption=HAL - Embedded resources]
GET /api/orders/456 HTTP/1.1
Accept: application/hal+json

HTTP/1.1 200 OK
Content-Type: application/hal+json

{
  "id": 456,
  "status": "shipped",
  "total": 99.99,
  "_links": {
    "self": {
      "href": "/api/orders/456"
    },
    "customer": {
      "href": "/api/customers/789"
    },
    "invoice": {
      "href": "/api/orders/456/invoice"
    }
  },
  "_embedded": {
    "customer": {
      "id": 789,
      "name": "Mario Rossi",
      "email": "mario@example.com",
      "_links": {
        "self": {
          "href": "/api/customers/789"
        }
      }
    },
    "items": [
      {
        "id": 1,
        "product_id": 100,
        "quantity": 2,
        "price": 49.99,
        "_links": {
          "self": {
            "href": "/api/order-items/1"
          },
          "product": {
            "href": "/api/products/100"
          }
        }
      }
    ]
  }
}
\end{lstlisting}

\subsection{HAL Collection}

\begin{lstlisting}[caption=HAL - Collection]
GET /api/users HTTP/1.1
Accept: application/hal+json

HTTP/1.1 200 OK
Content-Type: application/hal+json

{
  "_links": {
    "self": {
      "href": "/api/users?page=1"
    },
    "first": {
      "href": "/api/users?page=1"
    },
    "last": {
      "href": "/api/users?page=10"
    },
    "next": {
      "href": "/api/users?page=2"
    }
  },
  "_embedded": {
    "users": [
      {
        "id": 1,
        "name": "Mario Rossi",
        "_links": {
          "self": {
            "href": "/api/users/1"
          }
        }
      },
      {
        "id": 2,
        "name": "Luigi Verdi",
        "_links": {
          "self": {
            "href": "/api/users/2"
          }
        }
      }
    ]
  },
  "page": 1,
  "total": 100
}
\end{lstlisting}

\subsection{HAL Link Relations}

\textbf{Standard IANA link relations}:

\begin{itemize}
    \item \texttt{self}: URI della risorsa corrente
    \item \texttt{first/last/next/prev}: Pagination
    \item \texttt{up}: Parent resource
    \item \texttt{related}: Risorsa correlata
    \item \texttt{alternate}: Rappresentazione alternativa
    \item \texttt{edit}: URI per modificare risorsa
    \item \texttt{delete}: URI per eliminare
\end{itemize}

\begin{lstlisting}[caption=HAL - Rich link relations]
{
  "id": 123,
  "name": "Mario Rossi",
  "_links": {
    "self": {
      "href": "/api/users/123"
    },
    "edit": {
      "href": "/api/users/123",
      "title": "Edit this user"
    },
    "delete": {
      "href": "/api/users/123",
      "title": "Delete this user",
      "method": "DELETE"
    },
    "avatar": {
      "href": "/api/users/123/avatar",
      "type": "image/jpeg"
    }
  }
}
\end{lstlisting}

\section{JSON:API Specification}

\subsection{Cos'è JSON:API}

\textbf{JSON:API} è specification completa per costruire API JSON consistenti.

\textbf{Media type}: \texttt{application/vnd.api+json}

\textbf{Obiettivi}:
\begin{itemize}
    \item Struttura standardizzata
    \item Ridurre roundtrips (compound documents)
    \item Sparse fieldsets
    \item Filtering, sorting, pagination built-in
\end{itemize}

\subsection{JSON:API Document Structure}

\begin{lstlisting}[caption=JSON:API - Single resource]
GET /api/users/123 HTTP/1.1
Accept: application/vnd.api+json

HTTP/1.1 200 OK
Content-Type: application/vnd.api+json

{
  "data": {
    "type": "users",
    "id": "123",
    "attributes": {
      "name": "Mario Rossi",
      "email": "mario@example.com",
      "role": "admin"
    },
    "relationships": {
      "orders": {
        "links": {
          "self": "/api/users/123/relationships/orders",
          "related": "/api/users/123/orders"
        }
      },
      "company": {
        "data": {
          "type": "companies",
          "id": "456"
        }
      }
    },
    "links": {
      "self": "/api/users/123"
    }
  }
}
\end{lstlisting}

\textbf{Componenti}:
\begin{itemize}
    \item \texttt{data}: Resource object principale
    \item \texttt{type}: Tipo risorsa (required)
    \item \texttt{id}: Identificatore (required per existing resources)
    \item \texttt{attributes}: Attributi risorsa
    \item \texttt{relationships}: Relazioni con altre risorse
    \item \texttt{links}: Hypermedia links
\end{itemize}

\subsection{JSON:API Collection}

\begin{lstlisting}[caption=JSON:API - Collection]
GET /api/users HTTP/1.1
Accept: application/vnd.api+json

HTTP/1.1 200 OK
Content-Type: application/vnd.api+json

{
  "data": [
    {
      "type": "users",
      "id": "1",
      "attributes": {
        "name": "Mario Rossi",
        "email": "mario@example.com"
      },
      "links": {
        "self": "/api/users/1"
      }
    },
    {
      "type": "users",
      "id": "2",
      "attributes": {
        "name": "Luigi Verdi",
        "email": "luigi@example.com"
      },
      "links": {
        "self": "/api/users/2"
      }
    }
  ],
  "links": {
    "self": "/api/users?page[number]=1",
    "first": "/api/users?page[number]=1",
    "last": "/api/users?page[number]=10",
    "next": "/api/users?page[number]=2"
  },
  "meta": {
    "total": 100
  }
}
\end{lstlisting}

\subsection{JSON:API Compound Documents (Included)}

\textbf{Riduce N+1 queries}: Include risorse correlate.

\begin{lstlisting}[caption=JSON:API - Compound document]
GET /api/orders/456?include=customer,items HTTP/1.1
Accept: application/vnd.api+json

HTTP/1.1 200 OK
Content-Type: application/vnd.api+json

{
  "data": {
    "type": "orders",
    "id": "456",
    "attributes": {
      "status": "shipped",
      "total": 99.99
    },
    "relationships": {
      "customer": {
        "data": {
          "type": "customers",
          "id": "789"
        }
      },
      "items": {
        "data": [
          {
            "type": "order-items",
            "id": "1"
          },
          {
            "type": "order-items",
            "id": "2"
          }
        ]
      }
    }
  },
  "included": [
    {
      "type": "customers",
      "id": "789",
      "attributes": {
        "name": "Mario Rossi",
        "email": "mario@example.com"
      }
    },
    {
      "type": "order-items",
      "id": "1",
      "attributes": {
        "product_id": 100,
        "quantity": 2,
        "price": 49.99
      }
    },
    {
      "type": "order-items",
      "id": "2",
      "attributes": {
        "product_id": 200,
        "quantity": 1,
        "price": 49.99
      }
    }
  ]
}
\end{lstlisting}

\subsection{JSON:API Sparse Fieldsets}

\begin{lstlisting}[caption=JSON:API - Sparse fieldsets]
GET /api/users?fields[users]=name,email HTTP/1.1
Accept: application/vnd.api+json

HTTP/1.1 200 OK
Content-Type: application/vnd.api+json

{
  "data": [
    {
      "type": "users",
      "id": "1",
      "attributes": {
        "name": "Mario Rossi",
        "email": "mario@example.com"
      }
    }
  ]
}
\end{lstlisting}

\subsection{JSON:API Errors}

\begin{lstlisting}[caption=JSON:API - Error format]
POST /api/users HTTP/1.1
Content-Type: application/vnd.api+json

{
  "data": {
    "type": "users",
    "attributes": {
      "name": "",
      "email": "invalid-email"
    }
  }
}

HTTP/1.1 422 Unprocessable Entity
Content-Type: application/vnd.api+json

{
  "errors": [
    {
      "status": "422",
      "source": {
        "pointer": "/data/attributes/name"
      },
      "title": "Invalid Attribute",
      "detail": "Name cannot be blank"
    },
    {
      "status": "422",
      "source": {
        "pointer": "/data/attributes/email"
      },
      "title": "Invalid Attribute",
      "detail": "Email is not a valid email address"
    }
  ]
}
\end{lstlisting}

\section{Altri Formati Hypermedia}

\subsection{Siren}

\textbf{Siren}: Hypermedia specification con focus su actions.

\begin{lstlisting}[caption=Siren format]
GET /api/orders/456 HTTP/1.1
Accept: application/vnd.siren+json

HTTP/1.1 200 OK
Content-Type: application/vnd.siren+json

{
  "class": ["order"],
  "properties": {
    "orderNumber": 456,
    "status": "pending",
    "total": 99.99
  },
  "entities": [
    {
      "class": ["customer"],
      "rel": ["customer"],
      "properties": {
        "customerId": 789,
        "name": "Mario Rossi"
      },
      "links": [
        {
          "rel": ["self"],
          "href": "/api/customers/789"
        }
      ]
    }
  ],
  "actions": [
    {
      "name": "cancel-order",
      "title": "Cancel Order",
      "method": "POST",
      "href": "/api/orders/456/cancel"
    },
    {
      "name": "update-order",
      "title": "Update Order",
      "method": "PUT",
      "href": "/api/orders/456",
      "type": "application/json",
      "fields": [
        {
          "name": "status",
          "type": "text"
        }
      ]
    }
  ],
  "links": [
    {
      "rel": ["self"],
      "href": "/api/orders/456"
    }
  ]
}
\end{lstlisting}

\subsection{Collection+JSON}

\textbf{Collection+JSON}: Focus su collections e forms.

\begin{lstlisting}[caption=Collection+JSON format]
GET /api/users HTTP/1.1
Accept: application/vnd.collection+json

HTTP/1.1 200 OK
Content-Type: application/vnd.collection+json

{
  "collection": {
    "version": "1.0",
    "href": "/api/users",
    "items": [
      {
        "href": "/api/users/1",
        "data": [
          {
            "name": "name",
            "value": "Mario Rossi"
          },
          {
            "name": "email",
            "value": "mario@example.com"
          }
        ]
      }
    ],
    "template": {
      "data": [
        {
          "name": "name",
          "value": "",
          "prompt": "Full name"
        },
        {
          "name": "email",
          "value": "",
          "prompt": "Email address"
        }
      ]
    }
  }
}
\end{lstlisting}

\section{Content Negotiation}

\subsection{Accept Header}

\textbf{Client specifica formato desiderato via Accept header}.

\begin{lstlisting}[caption=Content negotiation - Accept header]
# Request JSON
GET /api/users/123 HTTP/1.1
Accept: application/json

HTTP/1.1 200 OK
Content-Type: application/json

{"id": 123, "name": "Mario"}

# Request HAL
GET /api/users/123 HTTP/1.1
Accept: application/hal+json

HTTP/1.1 200 OK
Content-Type: application/hal+json

{
  "id": 123,
  "name": "Mario",
  "_links": {...}
}

# Request JSON:API
GET /api/users/123 HTTP/1.1
Accept: application/vnd.api+json

HTTP/1.1 200 OK
Content-Type: application/vnd.api+json

{
  "data": {
    "type": "users",
    "id": "123",
    "attributes": {...}
  }
}
\end{lstlisting}

\subsection{Multiple Accept Types}

\textbf{Client può specificare preferenze con quality values}:

\begin{lstlisting}[caption=Accept with quality values]
GET /api/users/123 HTTP/1.1
Accept: application/hal+json;q=1.0, application/json;q=0.8, */*;q=0.5

# Server sceglie in ordine di preferenza:
# 1. application/hal+json (q=1.0)
# 2. application/json (q=0.8)
# 3. Qualsiasi altro formato (q=0.5)

HTTP/1.1 200 OK
Content-Type: application/hal+json

{...}
\end{lstlisting}

\subsection{406 Not Acceptable}

\textbf{Server non supporta formato richiesto}:

\begin{lstlisting}[caption=406 quando formato non supportato]
GET /api/users/123 HTTP/1.1
Accept: application/xml

HTTP/1.1 406 Not Acceptable
Content-Type: application/json

{
  "error": "not_acceptable",
  "message": "Requested format not supported",
  "supported_formats": [
    "application/json",
    "application/hal+json"
  ]
}
\end{lstlisting}

\subsection{Content-Type Header}

\textbf{Client specifica formato payload in POST/PUT}:

\begin{lstlisting}[caption=Content-Type in POST]
POST /api/users HTTP/1.1
Content-Type: application/json
Accept: application/hal+json

{
  "name": "Mario Rossi",
  "email": "mario@example.com"
}

HTTP/1.1 201 Created
Location: /api/users/123
Content-Type: application/hal+json

{
  "id": 123,
  "name": "Mario Rossi",
  "_links": {...}
}
\end{lstlisting}

\section{Confronto Formati}

\begin{table}[h]
\centering
\small
\begin{tabular}{|l|c|c|c|c|}
\hline
\textbf{Aspetto} & \textbf{Plain JSON} & \textbf{HAL} & \textbf{JSON:API} & \textbf{Siren} \\ \hline
\textbf{Semplicità} & ★★★★★ & ★★★★☆ & ★★★☆☆ & ★★★☆☆ \\ \hline
\textbf{Hypermedia} & ☆☆☆☆☆ & ★★★★★ & ★★★★☆ & ★★★★★ \\ \hline
\textbf{Spec rigida} & ☆☆☆☆☆ & ★★☆☆☆ & ★★★★★ & ★★★☆☆ \\ \hline
\textbf{Tooling} & ★★★★★ & ★★★☆☆ & ★★★★☆ & ★★☆☆☆ \\ \hline
\textbf{Learning curve} & ★★★★★ & ★★★★☆ & ★★☆☆☆ & ★★★☆☆ \\ \hline
\textbf{Compattezza} & ★★★★★ & ★★★☆☆ & ★★☆☆☆ & ★★☆☆☆ \\ \hline
\textbf{Adoption} & ★★★★★ & ★★★☆☆ & ★★★☆☆ & ★☆☆☆☆ \\ \hline
\end{tabular}
\caption{Confronto formati JSON}
\end{table}

\section{Best Practices}

\begin{tcolorbox}[title=Best Practices JSON e Formati]
\begin{enumerate}
    \item \textbf{Plain JSON per default}: Semplicità prima di tutto
    \item \textbf{HAL per hypermedia}: Se serve navigabilità
    \item \textbf{JSON:API per complex API}: Spec completa, ecosystem
    \item \textbf{Consistenza naming}: camelCase o snake\_case (scegliere uno)
    \item \textbf{ISO 8601 per date}: \texttt{2025-11-15T15:00:00Z}
    \item \textbf{null vs omissione}: Null per valore assente, ometti per non applicabile
    \item \textbf{Array sempre array}: Anche singolo elemento \texttt{[item]}
    \item \textbf{Evita deep nesting}: Max 2-3 livelli
    \item \textbf{Pagination metadata}: In response body o headers
    \item \textbf{Content negotiation}: Accept e Content-Type corretti
    \item \textbf{Pretty print in dev}: Readability, minify in prod
    \item \textbf{UTF-8 encoding}: \texttt{Content-Type: application/json; charset=utf-8}
\end{enumerate}
\end{tcolorbox}

\begin{tcolorbox}[colback=red!5, colframe=red!60, title=Anti-patterns]
\begin{itemize}
    \item \textbf{Stringified JSON}: \texttt{\{"data": "\{\textbackslash"id\textbackslash": 123\}"\}}
    \item \textbf{Inconsistent naming}: Mix camelCase/snake\_case
    \item \textbf{String numbers}: \texttt{\{"age": "25"\}} invece di \texttt{\{"age": 25\}}
    \item \textbf{Date as string non-ISO}: \texttt{"15/11/2025"} invece di \texttt{"2025-11-15"}
    \item \textbf{Array singolo come object}: \texttt{\{"user": \{...\}\}} quando potrebbe essere \texttt{[\{...\}]}
    \item \textbf{Envelope inutile}: Wrapping quando HTTP headers sufficienti
    \item \textbf{BOM in UTF-8}: Causa parsing errors
    \item \textbf{Trailing commas}: Non valido in JSON strict
\end{itemize}
\end{tcolorbox}

\section{Esempi Pratici}

\subsection{E-commerce API - Multiple Formats}

\begin{lstlisting}[caption=Plain JSON]
GET /api/products/123 HTTP/1.1
Accept: application/json

{
  "id": 123,
  "name": "Laptop XYZ",
  "price": 999.99,
  "category": "electronics",
  "in_stock": true
}
\end{lstlisting}

\begin{lstlisting}[caption=HAL]
GET /api/products/123 HTTP/1.1
Accept: application/hal+json

{
  "id": 123,
  "name": "Laptop XYZ",
  "price": 999.99,
  "_links": {
    "self": {"href": "/api/products/123"},
    "category": {"href": "/api/categories/electronics"},
    "reviews": {"href": "/api/products/123/reviews"},
    "add-to-cart": {"href": "/api/cart/items", "method": "POST"}
  }
}
\end{lstlisting}

\begin{lstlisting}[caption=JSON:API]
GET /api/products/123 HTTP/1.1
Accept: application/vnd.api+json

{
  "data": {
    "type": "products",
    "id": "123",
    "attributes": {
      "name": "Laptop XYZ",
      "price": 999.99,
      "in_stock": true
    },
    "relationships": {
      "category": {
        "data": {"type": "categories", "id": "electronics"},
        "links": {"related": "/api/categories/electronics"}
      }
    }
  }
}
\end{lstlisting}

\section{Riepilogo}

\begin{itemize}
    \item JSON è standard de facto per REST API moderne
    \item Plain JSON: semplice, universale, ottimo default
    \item HAL: aggiunge hypermedia con \texttt{\_links} e \texttt{\_embedded}
    \item JSON:API: spec completa con compound documents, sparse fields
    \item Siren: focus su actions e forms
    \item Content negotiation via Accept e Content-Type headers
    \item Consistenza naming e structure fondamentale
    \item ISO 8601 per date/time
\end{itemize}

\section*{Esercizi}

\begin{enumerate}
    \item Implementa HAL per API blog (posts, comments, authors)
    \item Converti API esistente a JSON:API spec
    \item Implementa content negotiation: JSON, HAL, JSON:API
    \item Progetta error response standardizzato RFC 7807
    \item Benchmark: Plain JSON vs HAL vs JSON:API (payload size)
    \item Crea client che consuma HAL hypermedia links
\end{enumerate}

\section*{Riferimenti}

\begin{itemize}
    \item JSON Specification: \url{https://www.json.org/}
    \item HAL Specification: \url{http://stateless.co/hal_specification.html}
    \item JSON:API Specification: \url{https://jsonapi.org/}
    \item Siren Specification: \url{https://github.com/kevinswiber/siren}
    \item RFC 7159 - JSON: \url{https://tools.ietf.org/html/rfc7159}
    \item RFC 8288 - Web Linking: \url{https://tools.ietf.org/html/rfc8288}
\end{itemize}
