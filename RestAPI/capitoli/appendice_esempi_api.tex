% appendice_esempi_api.tex
\chapter{Appendice: Esempi di API Complete}

Questa appendice presenta tre esempi completi di REST API: E-commerce, Social Media e Payments. Ogni esempio include architettura, endpoint, esempi curl e spec OpenAPI.

\section{E-commerce API}

\subsection{Architettura}

\begin{center}
\begin{tikzpicture}[
  node distance=1.5cm,
  box/.style={rectangle, draw, minimum width=3cm, minimum height=1cm, align=center},
  db/.style={cylinder, draw, shape border rotate=90, minimum width=2cm, minimum height=1.5cm, align=center}
]
  % Client
  \node[box] (client) {Client\\(Web/Mobile)};

  % API Gateway
  \node[box, below=of client] (gateway) {API Gateway\\Load Balancer};

  % Services
  \node[box, below left=1.5cm and 1cm of gateway] (products) {Products\\Service};
  \node[box, below=of gateway] (cart) {Cart\\Service};
  \node[box, below right=1.5cm and 1cm of gateway] (orders) {Orders\\Service};

  % Databases
  \node[db, below=of products] (productsdb) {Products\\DB};
  \node[db, below=of cart] (cartdb) {Redis\\Cache};
  \node[db, below=of orders] (ordersdb) {Orders\\DB};

  % Connections
  \draw[->] (client) -- (gateway);
  \draw[->] (gateway) -- (products);
  \draw[->] (gateway) -- (cart);
  \draw[->] (gateway) -- (orders);
  \draw[->] (products) -- (productsdb);
  \draw[->] (cart) -- (cartdb);
  \draw[->] (orders) -- (ordersdb);
\end{tikzpicture}
\end{center}

\subsection{Endpoint Principali}

\begin{table}[h]
\centering
\small
\begin{tabular}{|l|l|p{6cm}|}
\hline
\textbf{Method} & \textbf{Endpoint} & \textbf{Descrizione} \\ \hline
GET & /products & Lista prodotti con filtri \\ \hline
GET & /products/\{id\} & Dettagli prodotto \\ \hline
GET & /categories & Lista categorie \\ \hline
POST & /cart/items & Aggiungi item al carrello \\ \hline
GET & /cart & Visualizza carrello \\ \hline
DELETE & /cart/items/\{id\} & Rimuovi item da carrello \\ \hline
POST & /orders & Crea ordine da carrello \\ \hline
GET & /orders & Lista ordini utente \\ \hline
GET & /orders/\{id\} & Dettagli ordine \\ \hline
POST & /orders/\{id\}/cancel & Annulla ordine \\ \hline
\end{tabular}
\caption{E-commerce API Endpoints}
\end{table}

\subsection{Esempi cURL}

\begin{lstlisting}[language=bash, caption=E-commerce API - Esempi cURL]
#!/bin/bash

API_URL="https://api.ecommerce.example.com/v1"
TOKEN="eyJhbGciOiJIUzI1NiIsInR5cCI6IkpXVCJ9..."

echo "=== E-commerce API Examples ==="

# 1. Lista prodotti con filtri
echo -e "\n1. Lista prodotti categoria elettronica"
curl -s "$API_URL/products" \
  -H "Authorization: Bearer $TOKEN" \
  -G \
  --data-urlencode "category=electronics" \
  --data-urlencode "price_min=100" \
  --data-urlencode "price_max=1000" \
  --data-urlencode "in_stock=true" \
  --data-urlencode "sort=-price" \
  --data-urlencode "page=1" \
  --data-urlencode "per_page=10" \
  | jq .

# 2. Dettagli prodotto
echo -e "\n2. Dettagli prodotto specifico"
curl -s "$API_URL/products/prod_123" \
  -H "Authorization: Bearer $TOKEN" \
  | jq .

# 3. Aggiungi al carrello
echo -e "\n3. Aggiungi prodotto al carrello"
curl -s -X POST "$API_URL/cart/items" \
  -H "Authorization: Bearer $TOKEN" \
  -H "Content-Type: application/json" \
  -d '{
    "product_id": "prod_123",
    "quantity": 2,
    "variant_id": "var_456"
  }' \
  | jq .

# 4. Visualizza carrello
echo -e "\n4. Visualizza carrello corrente"
curl -s "$API_URL/cart" \
  -H "Authorization: Bearer $TOKEN" \
  | jq .

# 5. Aggiorna quantita item
echo -e "\n5. Aggiorna quantita item carrello"
curl -s -X PATCH "$API_URL/cart/items/item_789" \
  -H "Authorization: Bearer $TOKEN" \
  -H "Content-Type: application/json" \
  -d '{
    "quantity": 3
  }' \
  | jq .

# 6. Applica coupon
echo -e "\n6. Applica codice sconto"
curl -s -X POST "$API_URL/cart/coupons" \
  -H "Authorization: Bearer $TOKEN" \
  -H "Content-Type: application/json" \
  -d '{
    "code": "SAVE20"
  }' \
  | jq .

# 7. Crea ordine
echo -e "\n7. Crea ordine da carrello"
ORDER_RESPONSE=$(curl -s -X POST "$API_URL/orders" \
  -H "Authorization: Bearer $TOKEN" \
  -H "Content-Type: application/json" \
  -H "Idempotency-Key: $(uuidgen)" \
  -d '{
    "shipping_address": {
      "name": "Mario Rossi",
      "street": "Via Roma 123",
      "city": "Milano",
      "postal_code": "20100",
      "country": "IT",
      "phone": "+39 333 1234567"
    },
    "billing_address": {
      "name": "Mario Rossi",
      "street": "Via Roma 123",
      "city": "Milano",
      "postal_code": "20100",
      "country": "IT"
    },
    "payment_method": "credit_card",
    "payment_details": {
      "card_token": "tok_visa_1234"
    }
  }')

echo "$ORDER_RESPONSE" | jq .
ORDER_ID=$(echo "$ORDER_RESPONSE" | jq -r '.id')

# 8. Traccia ordine
echo -e "\n8. Traccia ordine"
curl -s "$API_URL/orders/$ORDER_ID/tracking" \
  -H "Authorization: Bearer $TOKEN" \
  | jq .

# 9. Storia ordini
echo -e "\n9. Storia ordini utente"
curl -s "$API_URL/orders" \
  -H "Authorization: Bearer $TOKEN" \
  -G \
  --data-urlencode "status=delivered" \
  --data-urlencode "date_from=2023-01-01" \
  --data-urlencode "sort=-created_at" \
  | jq .

# 10. Richiedi reso
echo -e "\n10. Richiedi reso prodotto"
curl -s -X POST "$API_URL/orders/$ORDER_ID/returns" \
  -H "Authorization: Bearer $TOKEN" \
  -H "Content-Type: application/json" \
  -d '{
    "items": [
      {
        "order_item_id": "item_123",
        "quantity": 1,
        "reason": "defective",
        "description": "Schermo difettoso"
      }
    ]
  }' \
  | jq .
\end{lstlisting}

\subsection{Response Esempi}

\begin{lstlisting}[caption=E-commerce - Product Response]
{
  "id": "prod_123",
  "sku": "LAPTOP-DELL-XPS13",
  "name": "Dell XPS 13 Laptop",
  "description": "Laptop ultraleggero con display InfinityEdge",
  "category": {
    "id": "cat_electronics",
    "name": "Electronics",
    "slug": "electronics"
  },
  "brand": "Dell",
  "price": {
    "amount": 1299.99,
    "currency": "EUR",
    "formatted": "€1.299,99"
  },
  "compare_at_price": {
    "amount": 1499.99,
    "currency": "EUR"
  },
  "discount_percentage": 13,
  "images": [
    {
      "url": "https://cdn.example.com/products/prod_123/img1.jpg",
      "alt": "Dell XPS 13 - Vista frontale",
      "position": 1
    }
  ],
  "variants": [
    {
      "id": "var_456",
      "name": "16GB RAM / 512GB SSD",
      "sku": "LAPTOP-DELL-XPS13-16-512",
      "price": 1299.99,
      "in_stock": true,
      "stock_quantity": 15
    }
  ],
  "attributes": {
    "color": "Silver",
    "weight": "1.2kg",
    "dimensions": "30.2 x 19.9 x 1.5 cm"
  },
  "rating": {
    "average": 4.7,
    "count": 342
  },
  "in_stock": true,
  "stock_quantity": 15,
  "shipping": {
    "free_shipping": true,
    "estimated_days": 2
  },
  "created_at": "2023-10-01T10:00:00Z",
  "updated_at": "2023-11-13T12:00:00Z",
  "_links": {
    "self": {
      "href": "https://api.example.com/products/prod_123"
    },
    "category": {
      "href": "https://api.example.com/categories/cat_electronics"
    },
    "reviews": {
      "href": "https://api.example.com/products/prod_123/reviews"
    }
  }
}
\end{lstlisting}

\begin{lstlisting}[caption=E-commerce - Cart Response]
{
  "id": "cart_789",
  "user_id": "user_456",
  "items": [
    {
      "id": "item_001",
      "product": {
        "id": "prod_123",
        "name": "Dell XPS 13 Laptop",
        "image": "https://cdn.example.com/products/prod_123/thumb.jpg"
      },
      "variant": {
        "id": "var_456",
        "name": "16GB RAM / 512GB SSD"
      },
      "quantity": 2,
      "unit_price": 1299.99,
      "total_price": 2599.98
    }
  ],
  "coupon": {
    "code": "SAVE20",
    "discount_amount": 520.00,
    "discount_type": "percentage",
    "discount_value": 20
  },
  "subtotal": 2599.98,
  "discount": 520.00,
  "shipping": 0.00,
  "tax": 415.99,
  "total": 2495.97,
  "currency": "EUR",
  "item_count": 2,
  "expires_at": "2023-11-14T12:00:00Z",
  "_links": {
    "checkout": {
      "href": "https://api.example.com/checkout"
    }
  }
}
\end{lstlisting}

\begin{lstlisting}[caption=E-commerce - Order Response]
{
  "id": "ord_abc123",
  "order_number": "ORD-2023-001234",
  "status": "processing",
  "payment_status": "paid",
  "fulfillment_status": "unfulfilled",
  "customer": {
    "id": "user_456",
    "email": "mario@example.com",
    "name": "Mario Rossi"
  },
  "items": [
    {
      "id": "item_001",
      "product_id": "prod_123",
      "variant_id": "var_456",
      "name": "Dell XPS 13 Laptop - 16GB/512GB",
      "quantity": 2,
      "unit_price": 1299.99,
      "total": 2599.98
    }
  ],
  "shipping_address": {
    "name": "Mario Rossi",
    "street": "Via Roma 123",
    "city": "Milano",
    "postal_code": "20100",
    "country": "IT",
    "phone": "+39 333 1234567"
  },
  "billing_address": {
    "name": "Mario Rossi",
    "street": "Via Roma 123",
    "city": "Milano",
    "postal_code": "20100",
    "country": "IT"
  },
  "subtotal": 2599.98,
  "discount": 520.00,
  "shipping": 0.00,
  "tax": 415.99,
  "total": 2495.97,
  "currency": "EUR",
  "payment_method": "credit_card",
  "payment_details": {
    "last4": "1234",
    "brand": "Visa"
  },
  "tracking": {
    "carrier": "DHL",
    "tracking_number": "DHL123456789",
    "tracking_url": "https://dhl.com/track/DHL123456789"
  },
  "created_at": "2023-11-13T12:00:00Z",
  "updated_at": "2023-11-13T12:00:00Z",
  "_links": {
    "self": {
      "href": "https://api.example.com/orders/ord_abc123"
    },
    "tracking": {
      "href": "https://api.example.com/orders/ord_abc123/tracking"
    },
    "invoice": {
      "href": "https://api.example.com/orders/ord_abc123/invoice"
    },
    "cancel": {
      "href": "https://api.example.com/orders/ord_abc123/cancel",
      "method": "POST"
    }
  }
}
\end{lstlisting}

\section{Social Media API}

\subsection{Architettura}

\begin{center}
\begin{tikzpicture}[
  node distance=1.5cm,
  box/.style={rectangle, draw, minimum width=2.5cm, minimum height=0.8cm, align=center, font=\small},
  db/.style={cylinder, draw, shape border rotate=90, minimum width=1.8cm, minimum height=1.2cm, align=center, font=\small}
]
  % API Gateway
  \node[box] (gateway) {API Gateway};

  % Microservices
  \node[box, below left=1cm and -1.5cm of gateway] (users) {Users};
  \node[box, below=1cm of gateway] (posts) {Posts};
  \node[box, below right=1cm and -1.5cm of gateway] (social) {Social};

  % Databases
  \node[db, below=of users] (usersdb) {Users\\DB};
  \node[db, below=of posts] (postsdb) {Posts\\DB};
  \node[db, below=of social] (socialdb) {Graph\\DB};

  % Message Queue
  \node[box, right=2cm of posts] (queue) {Message\\Queue};
  \node[box, below=of queue] (notifications) {Notifications};

  % Connections
  \draw[->] (gateway) -- (users);
  \draw[->] (gateway) -- (posts);
  \draw[->] (gateway) -- (social);
  \draw[->] (users) -- (usersdb);
  \draw[->] (posts) -- (postsdb);
  \draw[->] (social) -- (socialdb);
  \draw[->] (posts) -- (queue);
  \draw[->] (queue) -- (notifications);
\end{tikzpicture}
\end{center}

\subsection{Endpoint Principali}

\begin{table}[h]
\centering
\small
\begin{tabular}{|l|l|p{5.5cm}|}
\hline
\textbf{Method} & \textbf{Endpoint} & \textbf{Descrizione} \\ \hline
GET & /users/\{id\} & Profilo utente \\ \hline
GET & /users/\{id\}/posts & Post dell'utente \\ \hline
POST & /users/\{id\}/follow & Segui utente \\ \hline
DELETE & /users/\{id\}/follow & Smetti di seguire \\ \hline
POST & /posts & Crea post \\ \hline
GET & /feed & Feed personalizzato \\ \hline
POST & /posts/\{id\}/like & Like a post \\ \hline
POST & /posts/\{id\}/comments & Commenta post \\ \hline
GET & /notifications & Notifiche utente \\ \hline
POST & /messages & Invia messaggio \\ \hline
\end{tabular}
\caption{Social Media API Endpoints}
\end{table}

\subsection{Esempi cURL}

\begin{lstlisting}[language=bash, caption=Social Media API - Esempi cURL]
#!/bin/bash

API_URL="https://api.social.example.com/v1"
TOKEN="eyJhbGciOiJIUzI1NiIsInR5cCI6IkpXVCJ9..."

echo "=== Social Media API Examples ==="

# 1. Ottieni profilo utente
echo -e "\n1. Profilo utente"
curl -s "$API_URL/users/me" \
  -H "Authorization: Bearer $TOKEN" \
  | jq .

# 2. Aggiorna profilo
echo -e "\n2. Aggiorna bio profilo"
curl -s -X PATCH "$API_URL/users/me" \
  -H "Authorization: Bearer $TOKEN" \
  -H "Content-Type: application/json" \
  -d '{
    "bio": "Software developer, coffee enthusiast",
    "website": "https://example.com"
  }' \
  | jq .

# 3. Upload foto profilo
echo -e "\n3. Upload foto profilo"
curl -s -X POST "$API_URL/users/me/avatar" \
  -H "Authorization: Bearer $TOKEN" \
  -F "file=@avatar.jpg" \
  | jq .

# 4. Crea post
echo -e "\n4. Crea nuovo post"
POST_RESPONSE=$(curl -s -X POST "$API_URL/posts" \
  -H "Authorization: Bearer $TOKEN" \
  -H "Content-Type: application/json" \
  -d '{
    "content": "Just deployed my new REST API! 🚀",
    "visibility": "public",
    "media": [
      {
        "type": "image",
        "url": "https://cdn.example.com/images/screenshot.jpg"
      }
    ],
    "tags": ["#api", "#development"]
  }')

echo "$POST_RESPONSE" | jq .
POST_ID=$(echo "$POST_RESPONSE" | jq -r '.id')

# 5. Feed personalizzato
echo -e "\n5. Feed personalizzato"
curl -s "$API_URL/feed" \
  -H "Authorization: Bearer $TOKEN" \
  -G \
  --data-urlencode "page=1" \
  --data-urlencode "per_page=20" \
  --data-urlencode "filter=following" \
  | jq .

# 6. Like a post
echo -e "\n6. Like post"
curl -s -X POST "$API_URL/posts/$POST_ID/likes" \
  -H "Authorization: Bearer $TOKEN" \
  | jq .

# 7. Commenta post
echo -e "\n7. Commenta post"
curl -s -X POST "$API_URL/posts/$POST_ID/comments" \
  -H "Authorization: Bearer $TOKEN" \
  -H "Content-Type: application/json" \
  -d '{
    "content": "Great work! 👏",
    "mentions": ["@user123"]
  }' \
  | jq .

# 8. Segui utente
echo -e "\n8. Segui utente"
curl -s -X POST "$API_URL/users/user_789/follow" \
  -H "Authorization: Bearer $TOKEN" \
  | jq .

# 9. Lista follower
echo -e "\n9. Lista follower"
curl -s "$API_URL/users/me/followers" \
  -H "Authorization: Bearer $TOKEN" \
  -G \
  --data-urlencode "page=1" \
  --data-urlencode "per_page=50" \
  | jq .

# 10. Cerca utenti
echo -e "\n10. Cerca utenti"
curl -s "$API_URL/search/users" \
  -H "Authorization: Bearer $TOKEN" \
  -G \
  --data-urlencode "q=mario" \
  --data-urlencode "limit=10" \
  | jq .

# 11. Notifiche
echo -e "\n11. Notifiche non lette"
curl -s "$API_URL/notifications" \
  -H "Authorization: Bearer $TOKEN" \
  -G \
  --data-urlencode "unread=true" \
  | jq .

# 12. Marca notifiche come lette
echo -e "\n12. Marca tutte notifiche lette"
curl -s -X POST "$API_URL/notifications/mark-read" \
  -H "Authorization: Bearer $TOKEN" \
  | jq .

# 13. Invia messaggio diretto
echo -e "\n13. Invia messaggio diretto"
curl -s -X POST "$API_URL/messages" \
  -H "Authorization: Bearer $TOKEN" \
  -H "Content-Type: application/json" \
  -d '{
    "recipient_id": "user_789",
    "content": "Ciao! Come va?",
    "attachments": []
  }' \
  | jq .

# 14. Conversazioni
echo -e "\n14. Lista conversazioni"
curl -s "$API_URL/conversations" \
  -H "Authorization: Bearer $TOKEN" \
  | jq .
\end{lstlisting}

\subsection{Response Esempi}

\begin{lstlisting}[caption=Social Media - User Profile]
{
  "id": "user_456",
  "username": "mario_rossi",
  "display_name": "Mario Rossi",
  "bio": "Software developer, coffee enthusiast ☕",
  "avatar_url": "https://cdn.example.com/avatars/user_456.jpg",
  "cover_url": "https://cdn.example.com/covers/user_456.jpg",
  "website": "https://mariorossi.com",
  "location": "Milano, Italy",
  "verified": false,
  "stats": {
    "followers": 1234,
    "following": 567,
    "posts": 89
  },
  "joined_at": "2022-01-15T10:00:00Z",
  "is_following": false,
  "is_followed_by": false,
  "_links": {
    "self": {
      "href": "https://api.example.com/users/user_456"
    },
    "posts": {
      "href": "https://api.example.com/users/user_456/posts"
    },
    "followers": {
      "href": "https://api.example.com/users/user_456/followers"
    },
    "following": {
      "href": "https://api.example.com/users/user_456/following"
    }
  }
}
\end{lstlisting}

\begin{lstlisting}[caption=Social Media - Post]
{
  "id": "post_123",
  "author": {
    "id": "user_456",
    "username": "mario_rossi",
    "display_name": "Mario Rossi",
    "avatar_url": "https://cdn.example.com/avatars/user_456.jpg",
    "verified": false
  },
  "content": "Just deployed my new REST API! 🚀",
  "media": [
    {
      "type": "image",
      "url": "https://cdn.example.com/posts/post_123/img1.jpg",
      "width": 1200,
      "height": 800
    }
  ],
  "tags": ["#api", "#development"],
  "mentions": [],
  "visibility": "public",
  "stats": {
    "likes": 42,
    "comments": 7,
    "shares": 3,
    "views": 234
  },
  "is_liked": false,
  "is_bookmarked": false,
  "created_at": "2023-11-13T12:00:00Z",
  "edited_at": null,
  "_links": {
    "self": {
      "href": "https://api.example.com/posts/post_123"
    },
    "comments": {
      "href": "https://api.example.com/posts/post_123/comments"
    },
    "likes": {
      "href": "https://api.example.com/posts/post_123/likes"
    }
  }
}
\end{lstlisting}

\section{Payments API}

\subsection{Architettura}

\begin{center}
\begin{tikzpicture}[
  node distance=1.5cm,
  box/.style={rectangle, draw, minimum width=2.8cm, minimum height=0.8cm, align=center, font=\small},
  external/.style={rectangle, draw, dashed, minimum width=2.8cm, minimum height=0.8cm, align=center, font=\small}
]
  % Client
  \node[box] (client) {Client App};

  % API
  \node[box, below=of client] (api) {Payments API};

  % Internal Services
  \node[box, below left=1.5cm and 0.5cm of api] (processing) {Payment\\Processing};
  \node[box, below right=1.5cm and 0.5cm of api] (fraud) {Fraud\\Detection};

  % External
  \node[external, below=of processing] (stripe) {Stripe};
  \node[external, below=of fraud] (risk) {Risk\\Service};

  % Database
  \node[box, right=2cm of api] (db) {Database};
  \node[box, below=of db] (queue) {Event\\Queue};

  % Connections
  \draw[->] (client) -- (api);
  \draw[->] (api) -- (processing);
  \draw[->] (api) -- (fraud);
  \draw[->] (api) -- (db);
  \draw[->] (processing) -- (stripe);
  \draw[->] (fraud) -- (risk);
  \draw[->] (processing) -- (queue);
  \draw[->] (fraud) -- (queue);
\end{tikzpicture}
\end{center}

\subsection{Endpoint Principali}

\begin{table}[h]
\centering
\small
\begin{tabular}{|l|l|p{6cm}|}
\hline
\textbf{Method} & \textbf{Endpoint} & \textbf{Descrizione} \\ \hline
POST & /payment-methods & Aggiungi metodo pagamento \\ \hline
GET & /payment-methods & Lista metodi pagamento \\ \hline
DELETE & /payment-methods/\{id\} & Elimina metodo \\ \hline
POST & /payments & Crea pagamento \\ \hline
GET & /payments/\{id\} & Dettagli pagamento \\ \hline
POST & /payments/\{id\}/refund & Rimborso \\ \hline
POST & /payouts & Crea payout \\ \hline
GET & /balance & Saldo account \\ \hline
GET & /transactions & Storia transazioni \\ \hline
\end{tabular}
\caption{Payments API Endpoints}
\end{table}

\subsection{Esempi cURL}

\begin{lstlisting}[language=bash, caption=Payments API - Esempi cURL]
#!/bin/bash

API_URL="https://api.payments.example.com/v1"
TOKEN="sk_live_abc123..."

echo "=== Payments API Examples ==="

# 1. Aggiungi carta di credito
echo -e "\n1. Aggiungi metodo di pagamento"
CARD_RESPONSE=$(curl -s -X POST "$API_URL/payment-methods" \
  -H "Authorization: Bearer $TOKEN" \
  -H "Content-Type: application/json" \
  -d '{
    "type": "card",
    "card": {
      "number": "4242424242424242",
      "exp_month": 12,
      "exp_year": 2025,
      "cvc": "123"
    },
    "billing_details": {
      "name": "Mario Rossi",
      "email": "mario@example.com",
      "address": {
        "line1": "Via Roma 123",
        "city": "Milano",
        "postal_code": "20100",
        "country": "IT"
      }
    }
  }')

echo "$CARD_RESPONSE" | jq .
PAYMENT_METHOD_ID=$(echo "$CARD_RESPONSE" | jq -r '.id')

# 2. Lista metodi di pagamento
echo -e "\n2. Lista metodi di pagamento"
curl -s "$API_URL/payment-methods" \
  -H "Authorization: Bearer $TOKEN" \
  | jq .

# 3. Crea pagamento
echo -e "\n3. Crea pagamento"
PAYMENT_RESPONSE=$(curl -s -X POST "$API_URL/payments" \
  -H "Authorization: Bearer $TOKEN" \
  -H "Content-Type: application/json" \
  -H "Idempotency-Key: $(uuidgen)" \
  -d "{
    \"amount\": 10000,
    \"currency\": \"eur\",
    \"payment_method\": \"$PAYMENT_METHOD_ID\",
    \"description\": \"Ordine #ORD-2023-001234\",
    \"metadata\": {
      \"order_id\": \"ord_abc123\",
      \"customer_id\": \"cus_456\"
    },
    \"receipt_email\": \"mario@example.com\"
  }")

echo "$PAYMENT_RESPONSE" | jq .
PAYMENT_ID=$(echo "$PAYMENT_RESPONSE" | jq -r '.id')

# 4. Conferma pagamento (per 3D Secure)
echo -e "\n4. Conferma pagamento"
curl -s -X POST "$API_URL/payments/$PAYMENT_ID/confirm" \
  -H "Authorization: Bearer $TOKEN" \
  -H "Content-Type: application/json" \
  -d '{
    "return_url": "https://example.com/payment/complete"
  }' \
  | jq .

# 5. Dettagli pagamento
echo -e "\n5. Dettagli pagamento"
curl -s "$API_URL/payments/$PAYMENT_ID" \
  -H "Authorization: Bearer $TOKEN" \
  | jq .

# 6. Lista pagamenti
echo -e "\n6. Lista pagamenti"
curl -s "$API_URL/payments" \
  -H "Authorization: Bearer $TOKEN" \
  -G \
  --data-urlencode "limit=10" \
  --data-urlencode "status=succeeded" \
  --data-urlencode "created[gte]=2023-11-01" \
  | jq .

# 7. Crea rimborso
echo -e "\n7. Crea rimborso parziale"
curl -s -X POST "$API_URL/payments/$PAYMENT_ID/refunds" \
  -H "Authorization: Bearer $TOKEN" \
  -H "Content-Type: application/json" \
  -H "Idempotency-Key: $(uuidgen)" \
  -d '{
    "amount": 5000,
    "reason": "requested_by_customer",
    "metadata": {
      "reason_details": "Prodotto danneggiato"
    }
  }' \
  | jq .

# 8. Saldo account
echo -e "\n8. Saldo account"
curl -s "$API_URL/balance" \
  -H "Authorization: Bearer $TOKEN" \
  | jq .

# 9. Crea payout
echo -e "\n9. Crea payout"
curl -s -X POST "$API_URL/payouts" \
  -H "Authorization: Bearer $TOKEN" \
  -H "Content-Type: application/json" \
  -H "Idempotency-Key: $(uuidgen)" \
  -d '{
    "amount": 50000,
    "currency": "eur",
    "method": "bank_account",
    "description": "Payout mensile Novembre 2023"
  }' \
  | jq .

# 10. Transazioni
echo -e "\n10. Storia transazioni"
curl -s "$API_URL/transactions" \
  -H "Authorization: Bearer $TOKEN" \
  -G \
  --data-urlencode "limit=20" \
  --data-urlencode "type=payment" \
  | jq .

# 11. Crea subscription
echo -e "\n11. Crea abbonamento"
curl -s -X POST "$API_URL/subscriptions" \
  -H "Authorization: Bearer $TOKEN" \
  -H "Content-Type: application/json" \
  -d "{
    \"customer\": \"cus_456\",
    \"items\": [
      {
        \"price\": \"price_pro_monthly\",
        \"quantity\": 1
      }
    ],
    \"payment_method\": \"$PAYMENT_METHOD_ID\",
    \"trial_period_days\": 14
  }" \
  | jq .

# 12. Webhook endpoint test
echo -e "\n12. Test webhook signature"
curl -s -X POST "$API_URL/webhooks/test" \
  -H "Authorization: Bearer $TOKEN" \
  -H "Content-Type: application/json" \
  -d '{
    "webhook_id": "wh_abc123",
    "event_type": "payment.succeeded"
  }' \
  | jq .
\end{lstlisting}

\subsection{Response Esempi}

\begin{lstlisting}[caption=Payments - Payment Response]
{
  "id": "pay_abc123",
  "object": "payment",
  "amount": 10000,
  "amount_received": 10000,
  "currency": "eur",
  "status": "succeeded",
  "payment_method": {
    "id": "pm_xyz789",
    "type": "card",
    "card": {
      "brand": "visa",
      "last4": "4242",
      "exp_month": 12,
      "exp_year": 2025,
      "fingerprint": "abc123"
    },
    "billing_details": {
      "name": "Mario Rossi",
      "email": "mario@example.com",
      "address": {
        "line1": "Via Roma 123",
        "city": "Milano",
        "postal_code": "20100",
        "country": "IT"
      }
    }
  },
  "description": "Ordine #ORD-2023-001234",
  "receipt_email": "mario@example.com",
  "receipt_url": "https://pay.example.com/receipts/pay_abc123",
  "metadata": {
    "order_id": "ord_abc123",
    "customer_id": "cus_456"
  },
  "fees": [
    {
      "type": "stripe_fee",
      "amount": 320,
      "currency": "eur",
      "description": "Stripe processing fee"
    }
  ],
  "net": 9680,
  "refunded": false,
  "refunds": {
    "total_count": 0,
    "total_amount": 0,
    "data": []
  },
  "risk_score": 15,
  "risk_level": "normal",
  "created_at": "2023-11-13T12:00:00Z",
  "updated_at": "2023-11-13T12:00:05Z",
  "_links": {
    "self": {
      "href": "https://api.example.com/payments/pay_abc123"
    },
    "refund": {
      "href": "https://api.example.com/payments/pay_abc123/refunds",
      "method": "POST"
    },
    "receipt": {
      "href": "https://pay.example.com/receipts/pay_abc123"
    }
  }
}
\end{lstlisting}

\begin{lstlisting}[caption=Payments - Webhook Event]
{
  "id": "evt_abc123",
  "type": "payment.succeeded",
  "created_at": "2023-11-13T12:00:05Z",
  "data": {
    "object": {
      "id": "pay_abc123",
      "amount": 10000,
      "currency": "eur",
      "status": "succeeded",
      "metadata": {
        "order_id": "ord_abc123"
      }
    },
    "previous_attributes": {
      "status": "processing"
    }
  },
  "livemode": true,
  "pending_webhooks": 1,
  "request": {
    "id": "req_xyz789",
    "idempotency_key": "a7b3c9d2-1234-5678-9abc-def012345678"
  }
}
\end{lstlisting}

\section{Postman Collections}

\subsection{E-commerce Collection}

\begin{lstlisting}[caption=Postman Collection - E-commerce]
{
  "info": {
    "name": "E-commerce API",
    "schema": "https://schema.getpostman.com/json/collection/v2.1.0/"
  },
  "auth": {
    "type": "bearer",
    "bearer": [
      {"key": "token", "value": "{{access_token}}"}
    ]
  },
  "variable": [
    {"key": "base_url", "value": "https://api.ecommerce.example.com/v1"},
    {"key": "access_token", "value": ""},
    {"key": "product_id", "value": ""},
    {"key": "order_id", "value": ""}
  ],
  "item": [
    {
      "name": "Products",
      "item": [
        {
          "name": "List Products",
          "request": {
            "method": "GET",
            "url": {
              "raw": "{{base_url}}/products?category=electronics&page=1",
              "query": [
                {"key": "category", "value": "electronics"},
                {"key": "page", "value": "1"}
              ]
            }
          }
        },
        {
          "name": "Get Product",
          "request": {
            "method": "GET",
            "url": "{{base_url}}/products/{{product_id}}"
          }
        }
      ]
    },
    {
      "name": "Cart",
      "item": [
        {
          "name": "Add to Cart",
          "event": [
            {
              "listen": "test",
              "script": {
                "exec": [
                  "pm.test(\"Status is 200\", () => {",
                  "  pm.response.to.have.status(200);",
                  "});"
                ]
              }
            }
          ],
          "request": {
            "method": "POST",
            "url": "{{base_url}}/cart/items",
            "body": {
              "mode": "raw",
              "raw": "{\n  \"product_id\": \"{{product_id}}\",\n  \"quantity\": 2\n}"
            }
          }
        }
      ]
    }
  ]
}
\end{lstlisting}

\section{Riepilogo}

Questa appendice ha fornito tre esempi completi di REST API:

\begin{enumerate}
\item \textbf{E-commerce API}: Gestione prodotti, carrello, ordini
\item \textbf{Social Media API}: Profili, post, feed, notifiche
\item \textbf{Payments API}: Pagamenti, rimborsi, abbonamenti
\end{enumerate}

Ogni esempio include:
\begin{itemize}
\item Architettura sistema
\item Endpoint completi
\item Esempi cURL pratici
\item Response dettagliate
\item Postman collections
\end{itemize}

Usa questi esempi come riferimento per progettare le tue REST API.
