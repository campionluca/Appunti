% Appendice A — Cheat Sheet Comandi
\chapter{Cheat Sheet Comandi}

\section{Introduzione}
Questa appendice fornisce una reference rapida dei comandi Linux più utilizzati, organizzati per categoria. Ogni sezione include sintassi, opzioni comuni ed esempi pratici.

\section{Navigazione File System}

\begin{tcolorbox}[title=pwd - Print Working Directory]
\begin{lstlisting}
# Mostra directory corrente
pwd

# Mostra path fisico (segue symlink)
pwd -P
\end{lstlisting}
\end{tcolorbox}

\begin{tcolorbox}[title=cd - Change Directory]
\begin{lstlisting}
cd /path/to/directory    # Vai a directory assoluta
cd relative/path         # Vai a directory relativa
cd ~                     # Vai a home directory
cd                       # Vai a home (equivalente)
cd -                     # Torna a directory precedente
cd ..                    # Vai a directory parent
cd ../..                 # Sali due livelli
\end{lstlisting}
\end{tcolorbox}

\begin{tcolorbox}[title=ls - List Directory Contents]
\begin{lstlisting}
ls                       # Lista file directory corrente
ls -l                    # Long format (permessi, owner, size)
ls -a                    # Mostra file nascosti (iniziano con .)
ls -lh                   # Long format con dimensioni human-readable
ls -la                   # Long format + file nascosti
ls -lt                   # Ordina per tempo modifica (recenti prima)
ls -ltr                  # Ordina per tempo (vecchi prima)
ls -lS                   # Ordina per dimensione
ls -R                    # Ricorsivo
ls -d */                 # Solo directory
ls -1                    # Un file per riga
\end{lstlisting}
\end{tcolorbox}

\section{Operazioni su File e Directory}

\begin{tcolorbox}[title=Creazione]
\begin{lstlisting}
touch file.txt           # Crea file vuoto
mkdir directory          # Crea directory
mkdir -p path/to/dir     # Crea path completo (parent directories)
mkdir dir1 dir2 dir3     # Crea multiple directory
\end{lstlisting}
\end{tcolorbox}

\begin{tcolorbox}[title=Copia]
\begin{lstlisting}
cp source dest           # Copia file
cp -r source/ dest/      # Copia directory ricorsivamente
cp -p file dest          # Preserva permessi, timestamps
cp -i file dest          # Interactive (chiedi conferma)
cp -u file dest          # Copia solo se source più recente
cp file{,.bak}           # Crea backup (file.bak)
cp file1 file2 dir/      # Copia multipli file in directory
\end{lstlisting}
\end{tcolorbox}

\begin{tcolorbox}[title=Spostamento e Rinomina]
\begin{lstlisting}
mv oldname newname       # Rinomina file/directory
mv source dest/          # Sposta in altra directory
mv -i source dest        # Interactive (chiedi conferma)
mv -u source dest        # Sposta solo se più recente
mv *.txt dir/            # Sposta tutti .txt in directory
\end{lstlisting}
\end{tcolorbox}

\begin{tcolorbox}[title=Eliminazione]
\begin{lstlisting}
rm file.txt              # Elimina file
rm -i file.txt           # Interactive (chiedi conferma)
rm -f file.txt           # Force (no conferma, ignora errori)
rm -r directory/         # Elimina directory ricorsivamente
rm -rf directory/        # Force recursive (PERICOLOSO!)
rmdir directory          # Elimina directory vuota
rm *.log                 # Elimina tutti .log
\end{lstlisting}
\end{tcolorbox}

\section{Visualizzazione Contenuti File}

\begin{tcolorbox}[title=cat - Concatenate and Print]
\begin{lstlisting}
cat file.txt             # Mostra intero file
cat file1 file2          # Concatena e mostra file multipli
cat -n file.txt          # Mostra con numeri riga
cat -b file.txt          # Numera solo righe non vuote
cat > file.txt           # Crea file (Ctrl+D per terminare)
cat >> file.txt          # Append a file
\end{lstlisting}
\end{tcolorbox}

\begin{tcolorbox}[title=less/more - Paginazione]
\begin{lstlisting}
less file.txt            # Visualizza con paginazione
# Comandi in less:
# Spazio: pagina avanti
# b: pagina indietro
# /pattern: cerca
# n: prossima occorrenza
# q: esci

more file.txt            # Paginazione semplice (solo avanti)
\end{lstlisting}
\end{tcolorbox}

\begin{tcolorbox}[title=head/tail - Inizio/Fine File]
\begin{lstlisting}
head file.txt            # Prime 10 righe
head -n 20 file.txt      # Prime 20 righe
head -n -5 file.txt      # Tutte tranne ultime 5

tail file.txt            # Ultime 10 righe
tail -n 20 file.txt      # Ultime 20 righe
tail -n +10 file.txt     # Dalla riga 10 in poi
tail -f file.log         # Follow (aggiornamenti real-time)
tail -f -n 50 file.log   # Follow ultime 50 righe
\end{lstlisting}
\end{tcolorbox}

\section{Ricerca File}

\begin{tcolorbox}[title=find - Cerca File]
\begin{lstlisting}
# Per nome
find . -name "*.txt"                    # File .txt
find . -iname "*.TXT"                   # Case-insensitive
find /home -name "file.txt"             # In path specifico

# Per tipo
find . -type f                          # Solo file
find . -type d                          # Solo directory
find . -type l                          # Solo symlink

# Per dimensione
find . -size +100M                      # Più grandi di 100MB
find . -size -1k                        # Più piccoli di 1KB
find . -size 50M                        # Esattamente 50MB

# Per tempo
find . -mtime -7                        # Modificati ultimi 7 giorni
find . -mtime +30                       # Modificati più di 30 giorni fa
find . -mmin -60                        # Modificati ultima ora
find . -atime -1                        # Acceduti ieri

# Per permessi
find . -perm 644                        # Permessi esatti
find . -perm -644                       # Almeno questi permessi
find . -perm /u+w,g+w                   # Owner O group writable

# Azioni
find . -name "*.log" -delete            # Elimina trovati
find . -name "*.txt" -exec cat {} \;    # Esegui comando
find . -type f -exec chmod 644 {} \;    # Cambia permessi
find . -name "*.tmp" -ok rm {} \;       # Interactive delete

# Combinazioni
find . -name "*.txt" -size +1M -mtime -7
find /var/log -name "*.log" -mtime +30 -delete
\end{lstlisting}
\end{tcolorbox}

\begin{tcolorbox}[title=grep - Cerca Pattern nei File]
\begin{lstlisting}
grep "pattern" file.txt                 # Cerca pattern
grep -i "pattern" file.txt              # Case-insensitive
grep -r "pattern" directory/            # Ricorsivo
grep -n "pattern" file.txt              # Mostra numeri riga
grep -v "pattern" file.txt              # Inverted (righe NON matching)
grep -c "pattern" file.txt              # Conta occorrenze
grep -l "pattern" *.txt                 # Solo nomi file
grep -w "word" file.txt                 # Match parola intera
grep -A 3 "pattern" file.txt            # 3 righe After match
grep -B 3 "pattern" file.txt            # 3 righe Before match
grep -C 3 "pattern" file.txt            # 3 righe Context (before+after)

# Regex
grep "^start" file.txt                  # Inizia con "start"
grep "end$" file.txt                    # Finisce con "end"
grep "err.*fatal" file.txt              # err seguito da fatal
grep -E "error|warn" file.txt           # Extended regex (OR)
\end{lstlisting}
\end{tcolorbox}

\section{Permessi e Ownership}

\begin{tcolorbox}[title=chmod - Cambia Permessi]
\begin{lstlisting}
# Notazione simbolica
chmod u+x file.sh                       # User: aggiungi execute
chmod g-w file.txt                      # Group: rimuovi write
chmod o+r file.txt                      # Others: aggiungi read
chmod a+x file.sh                       # All: aggiungi execute
chmod u+rwx,g+rx,o+r file              # Combinato

# Notazione numerica
chmod 644 file.txt                      # rw-r--r--
chmod 755 script.sh                     # rwxr-xr-x
chmod 700 private.key                   # rwx------
chmod 600 config.conf                   # rw-------
chmod 777 shared/                       # rwxrwxrwx (sconsigliato)

# Ricorsivo
chmod -R 755 directory/

# Permessi speciali
chmod u+s executable                    # SUID
chmod g+s directory                     # SGID
chmod +t directory                      # Sticky bit
chmod 4755 file                         # SUID + 755
\end{lstlisting}
\end{tcolorbox}

\begin{tcolorbox}[title=chown - Cambia Owner]
\begin{lstlisting}
chown user file.txt                     # Cambia owner
chown user:group file.txt               # Cambia owner e group
chown :group file.txt                   # Solo group
chown -R user:group directory/          # Ricorsivo
chown --reference=ref.txt file.txt      # Copia ownership da altro file
\end{lstlisting}
\end{tcolorbox}

\section{Gestione Processi}

\begin{tcolorbox}[title=ps - Process Status]
\begin{lstlisting}
ps                                      # Processi sessione corrente
ps aux                                  # Tutti processi (BSD style)
ps -ef                                  # Tutti processi (Unix style)
ps -u username                          # Processi di user
ps -p 1234                              # Processo specifico PID
ps aux | grep nginx                     # Filtra processi
ps aux --sort=-%cpu                     # Ordina per CPU
ps aux --sort=-%mem                     # Ordina per memoria
\end{lstlisting}
\end{tcolorbox}

\begin{tcolorbox}[title=top/htop - Monitor Processi]
\begin{lstlisting}
top                                     # Monitor interattivo
# Comandi in top:
# k: kill processo
# r: renice (cambia priorità)
# M: ordina per memoria
# P: ordina per CPU
# q: quit

htop                                    # top migliorato (se installato)
\end{lstlisting}
\end{tcolorbox}

\begin{tcolorbox}[title=kill - Termina Processi]
\begin{lstlisting}
kill PID                                # SIGTERM (graceful shutdown)
kill -9 PID                             # SIGKILL (force kill)
kill -15 PID                            # SIGTERM (default)
kill -HUP PID                           # SIGHUP (reload config)
killall process_name                    # Kill per nome
pkill pattern                           # Kill per pattern
pgrep pattern                           # Trova PID per pattern
\end{lstlisting}
\end{tcolorbox}

\begin{tcolorbox}[title=Background e Jobs]
\begin{lstlisting}
command &                               # Esegui in background
jobs                                    # Lista jobs correnti
fg                                      # Porta job in foreground
fg %1                                   # Porta job 1 in foreground
bg                                      # Continua job in background
Ctrl+Z                                  # Sospendi processo corrente
nohup command &                         # Esegui immune a hangup
\end{lstlisting}
\end{tcolorbox}

\section{Compressione e Archivi}

\begin{tcolorbox}[title=tar - Tape Archive]
\begin{lstlisting}
# Creare archivio
tar -cf archive.tar files/              # Crea tar
tar -czf archive.tar.gz files/          # Crea tar.gz (gzip)
tar -cjf archive.tar.bz2 files/         # Crea tar.bz2 (bzip2)
tar -cJf archive.tar.xz files/          # Crea tar.xz (xz)

# Estrarre
tar -xf archive.tar                     # Estrai tar
tar -xzf archive.tar.gz                 # Estrai tar.gz
tar -xjf archive.tar.bz2                # Estrai tar.bz2
tar -xf archive.tar -C /path/           # Estrai in path specifico

# Visualizzare contenuti
tar -tf archive.tar                     # Lista file
tar -tzf archive.tar.gz                 # Lista file in tar.gz

# Append
tar -rf archive.tar newfile             # Aggiungi file

# Opzioni comuni
tar -cvzf archive.tar.gz files/         # v=verbose
tar -xvzf archive.tar.gz                # verbose extract
\end{lstlisting}
\end{tcolorbox}

\begin{tcolorbox}[title=gzip/gunzip - Compressione]
\begin{lstlisting}
gzip file.txt                           # Comprimi (crea file.txt.gz)
gzip -k file.txt                        # Comprimi, mantieni originale
gzip -9 file.txt                        # Max compressione
gunzip file.txt.gz                      # Decomprimi
gzip -d file.txt.gz                     # Decomprimi (alternativa)
gzip -l file.txt.gz                     # Info file compresso
zcat file.txt.gz                        # Visualizza senza decomprimere
\end{lstlisting}
\end{tcolorbox}

\begin{tcolorbox}[title=zip/unzip]
\begin{lstlisting}
zip archive.zip file1 file2             # Crea zip
zip -r archive.zip directory/           # Zip ricorsivo
zip -e secure.zip file.txt              # Zip con password
unzip archive.zip                       # Estrai zip
unzip archive.zip -d /path/             # Estrai in directory
unzip -l archive.zip                    # Lista contenuti
\end{lstlisting}
\end{tcolorbox}

\section{Networking}

\begin{tcolorbox}[title=Configurazione Interfacce]
\begin{lstlisting}
# ip (moderno)
ip addr show                            # Mostra interfacce
ip link show                            # Mostra link status
ip route show                           # Mostra routing table
ip neigh show                           # Mostra ARP

# ifconfig (legacy)
ifconfig                                # Mostra interfacce
ifconfig eth0                           # Interfaccia specifica
\end{lstlisting}
\end{tcolorbox}

\begin{tcolorbox}[title=Test Connettività]
\begin{lstlisting}
ping google.com                         # Ping hostname
ping -c 4 8.8.8.8                       # 4 ping
ping -i 0.2 192.168.1.1                 # Interval 0.2s

traceroute google.com                   # Traccia route
traceroute -n 8.8.8.8                   # No DNS resolution

nc -zv host 80                          # Test porta TCP
nc -zvu host 53                         # Test porta UDP
\end{lstlisting}
\end{tcolorbox}

\begin{tcolorbox}[title=Connessioni e Porte]
\begin{lstlisting}
# ss (moderno)
ss -tulpn                               # TCP/UDP listening ports
ss -t                                   # TCP connections
ss -ta                                  # All TCP sockets
ss -s                                   # Statistiche

# netstat (legacy)
netstat -tulpn                          # Listening ports
netstat -an                             # All connections
\end{lstlisting}
\end{tcolorbox}

\begin{tcolorbox}[title=DNS]
\begin{lstlisting}
nslookup google.com                     # Query DNS
nslookup google.com 8.8.8.8             # Query server specifico

dig google.com                          # DNS lookup dettagliato
dig google.com +short                   # Solo risposta
dig google.com MX                       # Record MX
dig -x 8.8.8.8                          # Reverse lookup

host google.com                         # Semplice lookup
\end{lstlisting}
\end{tcolorbox}

\begin{tcolorbox}[title=Trasferimento File]
\begin{lstlisting}
# scp
scp file.txt user@host:/path/           # Upload file
scp user@host:/path/file.txt ./         # Download file
scp -r dir/ user@host:/path/            # Upload directory
scp -P 2222 file.txt user@host:/path/   # Porta custom

# rsync
rsync -av source/ dest/                 # Sync directories
rsync -avz source/ user@host:/dest/     # Sync remoto
rsync -avz --delete source/ dest/       # Sync + delete
rsync -avzP source/ dest/               # Con progress
\end{lstlisting}
\end{tcolorbox}

\begin{tcolorbox}[title=Download]
\begin{lstlisting}
# wget
wget https://example.com/file.zip       # Download file
wget -O output.zip URL                  # Nome custom
wget -c URL                             # Resume download
wget -r -np URL                         # Download ricorsivo

# curl
curl https://api.example.com/data       # GET request
curl -o file.zip URL                    # Salva file
curl -O URL                             # Usa nome originale
curl -L URL                             # Segui redirect
curl -X POST -d "data" URL              # POST request
curl -H "Header: value" URL             # Custom header
\end{lstlisting}
\end{tcolorbox}

\section{Gestione Utenti}

\begin{tcolorbox}[title=User Management]
\begin{lstlisting}
# Creazione/modifica utenti
sudo useradd username                   # Crea utente
sudo useradd -m -s /bin/bash user       # Con home e shell
sudo passwd username                    # Imposta password
sudo usermod -aG group username         # Aggiungi a gruppo
sudo userdel username                   # Elimina utente
sudo userdel -r username                # Elimina + home

# Info utenti
id                                      # Info utente corrente
id username                             # Info utente specifico
whoami                                  # Username corrente
who                                     # Utenti loggati
w                                       # Utenti loggati + attività
last                                    # Storico login
\end{lstlisting}
\end{tcolorbox}

\begin{tcolorbox}[title=Group Management]
\begin{lstlisting}
sudo groupadd groupname                 # Crea gruppo
sudo groupdel groupname                 # Elimina gruppo
groups username                         # Gruppi di utente
getent group groupname                  # Info gruppo
\end{lstlisting}
\end{tcolorbox}

\section{Systemd e Servizi}

\begin{tcolorbox}[title=Service Management]
\begin{lstlisting}
sudo systemctl start service            # Avvia servizio
sudo systemctl stop service             # Ferma servizio
sudo systemctl restart service          # Riavvia servizio
sudo systemctl reload service           # Ricarica config
sudo systemctl status service           # Status servizio

sudo systemctl enable service           # Abilita avvio automatico
sudo systemctl disable service          # Disabilita avvio
sudo systemctl enable --now service     # Enable + start

systemctl is-active service             # Check se attivo
systemctl is-enabled service            # Check se enabled

systemctl list-units --type=service     # Lista servizi
systemctl --failed                      # Servizi falliti
\end{lstlisting}
\end{tcolorbox}

\begin{tcolorbox}[title=Logs con journalctl]
\begin{lstlisting}
journalctl                              # Tutti i log
journalctl -u service                   # Log di servizio
journalctl -f                           # Follow (real-time)
journalctl -u nginx -f                  # Follow servizio
journalctl -n 50                        # Ultime 50 righe
journalctl --since today                # Da oggi
journalctl --since "1 hour ago"         # Ultima ora
journalctl --since "2025-01-01"         # Da data
journalctl -p err                       # Solo errori
journalctl -b                           # Boot corrente
journalctl -k                           # Kernel messages
\end{lstlisting}
\end{tcolorbox}

\section{Disk Usage}

\begin{tcolorbox}[title=df - Disk Free]
\begin{lstlisting}
df                                      # Spazio disco
df -h                                   # Human-readable
df -i                                   # Inode usage
df -T                                   # Mostra filesystem type
df /path                                # Spazio per path specifico
\end{lstlisting}
\end{tcolorbox}

\begin{tcolorbox}[title=du - Disk Usage]
\begin{lstlisting}
du -sh directory/                       # Dimensione directory
du -sh *                                # Dimensione file/dir correnti
du -sh /* | sort -rh | head -10         # Top 10 directory
du -ah directory/                       # All files (human-readable)
du -d 1 directory/                      # Depth 1
\end{lstlisting}
\end{tcolorbox}

\section{Text Processing}

\begin{tcolorbox}[title=awk]
\begin{lstlisting}
awk '{print $1}' file.txt               # Prima colonna
awk '{print $1, $3}' file.txt           # Colonne 1 e 3
awk -F: '{print $1}' /etc/passwd        # Delimiter custom
awk 'NR==5' file.txt                    # Riga 5
awk '/pattern/' file.txt                # Righe con pattern
awk '$3 > 100' file.txt                 # Condizione numerica
awk '{sum+=$1} END {print sum}' file    # Somma colonna
\end{lstlisting}
\end{tcolorbox}

\begin{tcolorbox}[title=sed - Stream Editor]
\begin{lstlisting}
sed 's/old/new/' file.txt               # Sostituisci (prima occorrenza)
sed 's/old/new/g' file.txt              # Sostituisci (tutte)
sed -i 's/old/new/g' file.txt           # Modifica file in-place
sed -n '5,10p' file.txt                 # Stampa righe 5-10
sed '/pattern/d' file.txt               # Elimina righe con pattern
sed 's/^/prefix /' file.txt             # Aggiungi prefix
\end{lstlisting}
\end{tcolorbox}

\begin{tcolorbox}[title=cut]
\begin{lstlisting}
cut -d: -f1 /etc/passwd                 # Campo 1 (delimiter :)
cut -c1-10 file.txt                     # Caratteri 1-10
cut -d, -f1,3 file.csv                  # Campi 1 e 3 (CSV)
\end{lstlisting}
\end{tcolorbox}

\begin{tcolorbox}[title=sort]
\begin{lstlisting}
sort file.txt                           # Ordina alfabeticamente
sort -r file.txt                        # Reverse
sort -n file.txt                        # Ordine numerico
sort -u file.txt                        # Unique (rimuovi duplicati)
sort -k2 file.txt                       # Ordina per colonna 2
sort -t: -k3 -n /etc/passwd             # Ordina per campo 3 numerico
\end{lstlisting}
\end{tcolorbox}

\begin{tcolorbox}[title=uniq]
\begin{lstlisting}
uniq file.txt                           # Rimuovi righe duplicate consecutive
uniq -c file.txt                        # Conta occorrenze
uniq -d file.txt                        # Solo duplicati
uniq -u file.txt                        # Solo unique

# Tipico uso con sort
sort file.txt | uniq                    # Rimuovi tutti duplicati
sort file.txt | uniq -c | sort -rn      # Conta + ordina
\end{lstlisting}
\end{tcolorbox}

\section{Variabili e Environment}

\begin{tcolorbox}[title=Environment Variables]
\begin{lstlisting}
# Visualizzare
echo $HOME                              # Variabile specifica
echo $PATH
env                                     # Tutte le variabili
printenv                                # Alternativa a env
printenv HOME                           # Variabile specifica

# Impostare
export VAR=value                        # Export variabile
VAR=value                               # Solo sessione corrente
export PATH=$PATH:/new/path             # Aggiungi a PATH

# File configurazione
~/.bashrc                               # Bash configuration
~/.bash_profile                         # Login shell
~/.profile                              # Generic shell
/etc/environment                        # System-wide
\end{lstlisting}
\end{tcolorbox}

\section{Redirection e Pipe}

\begin{tcolorbox}[title=Redirection]
\begin{lstlisting}
command > file.txt                      # Redirect stdout (overwrite)
command >> file.txt                     # Redirect stdout (append)
command 2> error.log                    # Redirect stderr
command &> all.log                      # Redirect stdout + stderr
command > output.txt 2>&1               # Redirect both (alternativa)
command < input.txt                     # Input da file
command > /dev/null                     # Scarta output
command 2>&1 | tee file.log             # Output + salva
\end{lstlisting}
\end{tcolorbox}

\begin{tcolorbox}[title=Pipe e Combinazioni]
\begin{lstlisting}
command1 | command2                     # Pipe output a command2
cmd1 && cmd2                            # Esegui cmd2 se cmd1 OK
cmd1 || cmd2                            # Esegui cmd2 se cmd1 fallisce
cmd1 ; cmd2                             # Esegui sequenzialmente
cmd1 & cmd2                             # Esegui in parallelo
(cmd1; cmd2)                            # Subshell
\end{lstlisting}
\end{tcolorbox}

\section{Package Management}

\begin{tcolorbox}[title=APT (Debian/Ubuntu)]
\begin{lstlisting}
sudo apt update                         # Aggiorna package list
sudo apt upgrade                        # Aggiorna packages
sudo apt install package                # Installa package
sudo apt remove package                 # Rimuovi package
sudo apt purge package                  # Rimuovi + config
sudo apt autoremove                     # Rimuovi dipendenze non usate
apt search keyword                      # Cerca package
apt show package                        # Info package
apt list --installed                    # Lista packages installati
\end{lstlisting}
\end{tcolorbox}

\begin{tcolorbox}[title=YUM/DNF (RedHat/CentOS/Fedora)]
\begin{lstlisting}
sudo yum update                         # Aggiorna packages
sudo yum install package                # Installa package
sudo yum remove package                 # Rimuovi package
yum search keyword                      # Cerca package
yum info package                        # Info package
yum list installed                      # Lista installed

# DNF (newer)
sudo dnf update
sudo dnf install package
\end{lstlisting}
\end{tcolorbox}

\section{Shortcuts Bash}

\begin{tcolorbox}[title=Editing Command Line]
\begin{lstlisting}
Ctrl+A          # Vai a inizio riga
Ctrl+E          # Vai a fine riga
Ctrl+U          # Cancella da cursore a inizio
Ctrl+K          # Cancella da cursore a fine
Ctrl+W          # Cancella parola precedente
Ctrl+Y          # Incolla ultimo cancellato
Ctrl+L          # Clear screen
Ctrl+C          # Interrompi comando
Ctrl+D          # EOF / logout
Ctrl+Z          # Sospendi processo
Ctrl+R          # Ricerca history
!!              # Ripeti ultimo comando
!$              # Ultimo argomento comando precedente
!*              # Tutti argomenti comando precedente
\end{lstlisting}
\end{tcolorbox}

\section{Reference Rapida Simboli}

\begin{tcolorbox}[title=Special Characters]
\begin{tabular}{ll}
\textbf{Simbolo} & \textbf{Significato} \\
\hline
\texttt{.} & Directory corrente \\
\texttt{..} & Directory parent \\
\texttt{\textasciitilde} & Home directory \\
\texttt{/} & Root directory \\
\texttt{*} & Wildcard (qualsiasi caratteri) \\
\texttt{?} & Wildcard (un carattere) \\
\texttt{[]} & Character class \\
\texttt{\textbackslash} & Escape character \\
\texttt{|} & Pipe \\
\texttt{>} & Redirect output \\
\texttt{>>} & Append output \\
\texttt{<} & Input redirect \\
\texttt{\&} & Background execution \\
\texttt{;} & Command separator \\
\texttt{\&\&} & AND operator \\
\texttt{||} & OR operator \\
\texttt{\$} & Variable prefix \\
\texttt{\#} & Comment \\
\end{tabular}
\end{tcolorbox}

\section{Permessi Numerici Reference}

\begin{tcolorbox}[title=Permission Numbers]
\begin{tabular}{ll}
\textbf{Numero} & \textbf{Permessi} \\
\hline
0 & --- \\
1 & --x \\
2 & -w- \\
3 & -wx \\
4 & r-- \\
5 & r-x \\
6 & rw- \\
7 & rwx \\
\end{tabular}

\vspace{0.5em}
Esempi comuni:
\begin{itemize}
\item \texttt{644}: rw-r--r-- (file normali)
\item \texttt{755}: rwxr-xr-x (script, directory)
\item \texttt{700}: rwx------ (file privati)
\item \texttt{600}: rw------- (chiavi SSH, config)
\item \texttt{777}: rwxrwxrwx (tutto permesso - sconsigliato!)
\end{itemize}
\end{tcolorbox}

\section{Signal Reference}

\begin{tcolorbox}[title=Common Signals]
\begin{tabular}{lll}
\textbf{Signal} & \textbf{Numero} & \textbf{Azione} \\
\hline
SIGHUP & 1 & Hangup (reload config) \\
SIGINT & 2 & Interrupt (Ctrl+C) \\
SIGQUIT & 3 & Quit \\
SIGKILL & 9 & Kill (non catchable) \\
SIGTERM & 15 & Terminate (graceful) \\
SIGSTOP & 19 & Stop (non catchable) \\
SIGTSTP & 20 & Stop (Ctrl+Z) \\
SIGCONT & 18 & Continue \\
\end{tabular}
\end{tcolorbox}

\section{Exit Codes}

\begin{tcolorbox}[title=Common Exit Codes]
\begin{tabular}{ll}
\textbf{Code} & \textbf{Significato} \\
\hline
0 & Success \\
1 & General errors \\
2 & Misuse of shell command \\
126 & Command cannot execute \\
127 & Command not found \\
128 & Invalid exit argument \\
130 & Terminated by Ctrl+C \\
255 & Exit status out of range \\
\end{tabular}

Verificare exit code ultimo comando:
\begin{lstlisting}
command
echo $?
\end{lstlisting}
\end{tcolorbox}

\section{Formati Data e Tempo}

\begin{tcolorbox}[title=date Command Formats]
\begin{lstlisting}
date                                    # Data/ora corrente
date +%Y-%m-%d                          # 2025-01-15
date +%Y%m%d                            # 20250115
date +%H:%M:%S                          # 14:30:45
date +%s                                # Unix timestamp
date -d "yesterday"                     # Ieri
date -d "2 days ago"                    # 2 giorni fa
date -d "next Monday"                   # Prossimo lunedì
date -d @1234567890                     # Da timestamp

# Formati comuni
date +%Y%m%d_%H%M%S                     # 20250115_143045
date +"%Y-%m-%d %H:%M:%S"               # 2025-01-15 14:30:45
\end{lstlisting}
\end{tcolorbox}

\section{One-Liners Utili}

\begin{tcolorbox}[title=Useful One-Liners]
\begin{lstlisting}
# Top 10 comandi più usati
history | awk '{print $2}' | sort | uniq -c | sort -rn | head -10

# File più grandi
find . -type f -exec du -h {} \; | sort -rh | head -10

# Processi che usano più CPU
ps aux --sort=-%cpu | head -10

# Processi che usano più memoria
ps aux --sort=-%mem | head -10

# IP connessi
netstat -tn 2>/dev/null | grep :80 | awk '{print $5}' | \
    cut -d: -f1 | sort | uniq -c | sort -rn

# File modificati oggi
find . -type f -mtime 0

# Backup rapido
tar -czf backup_$(date +%Y%m%d).tar.gz directory/

# Kill processi per nome
pkill -9 -f "process_name"

# Port check
for p in {1..1024}; do nc -zv localhost $p 2>&1 | grep succeeded; done

# Monitor file size real-time
watch -n 1 'du -sh /var/log'

# Download intero sito
wget --mirror --convert-links --page-requisites --no-parent URL
\end{lstlisting}
\end{tcolorbox}
