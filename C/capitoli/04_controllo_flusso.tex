\chapter{Controllo di Flusso}
\label{cap:controllo_flusso}

\section{Introduzione}

Il controllo di flusso è il meccanismo che permette di modificare l'ordine di esecuzione delle istruzioni in un programma. Senza strutture di controllo, le istruzioni verrebbero eseguite sequenzialmente dall'inizio alla fine, limitando considerevolmente la capacità espressiva dei programmi. Le strutture di controllo permettono di prendere decisioni nel codice attraverso istruzioni condizionali (come \texttt{if} e \texttt{switch}), di ripetere blocchi di codice utilizzando cicli (come \texttt{for}, \texttt{while}, \texttt{do-while}), e di saltare parti di codice quando necessario.

Il controllo di flusso è fondamentale per creare programmi dinamici e interattivi. Immagina di scrivere un programma che valuta i voti degli studenti: senza istruzioni condizionali, non potresti distinguere tra sufficiente e insufficiente. Oppure pensa a un programma che deve elaborare 100 numeri: senza cicli, dovresti scrivere la stessa istruzione 100 volte!

\subsection{Obiettivi di Apprendimento}

Al termine di questo capitolo sarai in grado di:

\begin{itemize}
    \item Utilizzare le istruzioni condizionali (\texttt{if}, \texttt{if-else}, \texttt{switch}) per prendere decisioni nel codice
    \item Implementare cicli (\texttt{while}, \texttt{do-while}, \texttt{for}) per ripetere operazioni
    \item Controllare il flusso dei cicli con \texttt{break} e \texttt{continue}
    \item Scegliere la struttura di controllo più appropriata per ogni situazione
    \item Evitare errori comuni come cicli infiniti e condizioni mal formulate
    \item Scrivere programmi complessi che combinano diverse strutture di controllo
\end{itemize}

\section{Istruzioni Condizionali}

\subsection{Istruzione if}

L'istruzione \texttt{if} esegue un blocco di codice solo se una condizione è vera.

\subsubsection{Sintassi Base}

\begin{lstlisting}
if (condizione) {
    // codice eseguito se condizione e' vera
}
\end{lstlisting}

\subsubsection{Esempi}

\begin{lstlisting}
#include <stdio.h>

int main(int argc, char** argv) {
    int eta = 18;

    if (eta >= 18) {
        printf("Sei maggiorenne\n");
    }

    // Esempio con espressione complessa
    int voto = 28;
    if (voto >= 18 && voto <= 30) {
        printf("Voto valido: %d\n", voto);
    }

    return 0;
}
\end{lstlisting}

\subsection{Istruzione if-else}

L'istruzione \texttt{if-else} permette di specificare un blocco alternativo da eseguire se la condizione è falsa.

\subsubsection{Sintassi}

\begin{lstlisting}
if (condizione) {
    // codice se condizione e' vera
} else {
    // codice se condizione e' falsa
}
\end{lstlisting}

\subsubsection{Esempi}

\begin{lstlisting}
#include <stdio.h>

int main(int argc, char** argv) {
    int numero = 7;

    if (numero % 2 == 0) {
        printf("%d e' pari\n", numero);
    } else {
        printf("%d e' dispari\n", numero);
    }

    // Esempio con input
    float temperatura;
    printf("Inserisci la temperatura: ");
    scanf("%f", &temperatura);

    if (temperatura > 30) {
        printf("Fa caldo!\n");
    } else {
        printf("La temperatura e' accettabile.\n");
    }

    return 0;
}
\end{lstlisting}

\subsection{Istruzione if-else if-else}

Per testare condizioni multiple in sequenza.

\subsubsection{Sintassi}

\begin{lstlisting}
if (condizione1) {
    // codice se condizione1 e' vera
} else if (condizione2) {
    // codice se condizione2 e' vera
} else if (condizione3) {
    // codice se condizione3 e' vera
} else {
    // codice se nessuna condizione e' vera
}
\end{lstlisting}

\subsubsection{Esempi}

\begin{lstlisting}
#include <stdio.h>

int main(int argc, char** argv) {
    int voto;
    printf("Inserisci il voto: ");
    scanf("%d", &voto);

    if (voto < 18) {
        printf("Insufficiente\n");
    } else if (voto >= 18 && voto < 24) {
        printf("Sufficiente\n");
    } else if (voto >= 24 && voto < 27) {
        printf("Buono\n");
    } else if (voto >= 27 && voto <= 30) {
        printf("Ottimo\n");
    } else {
        printf("Voto non valido\n");
    }

    return 0;
}
\end{lstlisting}

\subsection{if Annidati}

È possibile inserire istruzioni \texttt{if} all'interno di altre istruzioni \texttt{if}.

\begin{lstlisting}
#include <stdio.h>

int main(int argc, char** argv) {
    int eta, patente;

    printf("Inserisci eta': ");
    scanf("%d", &eta);
    printf("Hai la patente? (1=si, 0=no): ");
    scanf("%d", &patente);

    if (eta >= 18) {
        if (patente == 1) {
            printf("Puoi guidare!\n");
        } else {
            printf("Devi prendere la patente.\n");
        }
    } else {
        printf("Sei troppo giovane per guidare.\n");
    }

    return 0;
}
\end{lstlisting}

\begin{attenzione}
Troppi livelli di annidamento rendono il codice difficile da leggere. Considera di usare operatori logici o di semplificare la logica.
\end{attenzione}

\subsection{Istruzione switch}

L'istruzione \texttt{switch} permette di confrontare una variabile con diversi valori costanti.

\subsubsection{Sintassi}

\begin{lstlisting}
switch (espressione) {
    case valore1:
        // codice per valore1
        break;
    case valore2:
        // codice per valore2
        break;
    default:
        // codice se nessun case corrisponde
}
\end{lstlisting}

\subsubsection{Esempi}

\begin{lstlisting}
#include <stdio.h>

int main(int argc, char** argv) {
    int giorno;
    printf("Inserisci un numero (1-7): ");
    scanf("%d", &giorno);

    switch (giorno) {
        case 1:
            printf("Lunedi\n");
            break;
        case 2:
            printf("Martedi\n");
            break;
        case 3:
            printf("Mercoledi\n");
            break;
        case 4:
            printf("Giovedi\n");
            break;
        case 5:
            printf("Venerdi\n");
            break;
        case 6:
            printf("Sabato\n");
            break;
        case 7:
            printf("Domenica\n");
            break;
        default:
            printf("Giorno non valido\n");
    }

    return 0;
}
\end{lstlisting}

\subsubsection{Fall-through}

Se ometti il \texttt{break}, l'esecuzione continua nel caso successivo.

\begin{lstlisting}
#include <stdio.h>

int main(int argc, char** argv) {
    int mese;
    printf("Inserisci il numero del mese (1-12): ");
    scanf("%d", &mese);

    int giorni;
    switch (mese) {
        case 1: case 3: case 5: case 7:
        case 8: case 10: case 12:
            giorni = 31;
            break;
        case 4: case 6: case 9: case 11:
            giorni = 30;
            break;
        case 2:
            giorni = 28;  // Semplificato, non considera anni bisestili
            break;
        default:
            printf("Mese non valido\n");
            return 1;
    }

    printf("Il mese %d ha %d giorni\n", mese, giorni);
    return 0;
}
\end{lstlisting}

\begin{nota}
Il \texttt{break} è essenziale nella maggior parte dei casi. Il fall-through può essere utile, ma deve essere usato intenzionalmente e documentato.
\end{nota}

\section{Cicli (Loop)}

I cicli permettono di ripetere un blocco di codice più volte.

\subsection{Ciclo while}

Ripete un blocco di codice finché una condizione è vera. La condizione viene controllata all'inizio.

\subsubsection{Sintassi}

\begin{lstlisting}
while (condizione) {
    // codice da ripetere
}
\end{lstlisting}

\subsubsection{Esempi}

\begin{lstlisting}
#include <stdio.h>

int main(int argc, char** argv) {
    // Stampa numeri da 1 a 5
    int i = 1;
    while (i <= 5) {
        printf("%d ", i);
        i++;
    }
    printf("\n");

    // Calcola somma di numeri inseriti dall'utente
    int numero, somma = 0;
    printf("Inserisci numeri (0 per terminare):\n");
    scanf("%d", &numero);

    while (numero != 0) {
        somma += numero;
        scanf("%d", &numero);
    }

    printf("La somma e': %d\n", somma);
    return 0;
}
\end{lstlisting}

\subsection{Ciclo do-while}

Simile al \texttt{while}, ma la condizione viene controllata alla fine. Il blocco viene eseguito almeno una volta.

\subsubsection{Sintassi}

\begin{lstlisting}
do {
    // codice da ripetere
} while (condizione);
\end{lstlisting}

\subsubsection{Esempi}

\begin{lstlisting}
#include <stdio.h>

int main(int argc, char** argv) {
    int numero;

    // Richiede input finche' non e' valido
    do {
        printf("Inserisci un numero tra 1 e 10: ");
        scanf("%d", &numero);

        if (numero < 1 || numero > 10) {
            printf("Numero non valido! Riprova.\n");
        }
    } while (numero < 1 || numero > 10);

    printf("Hai inserito: %d\n", numero);

    // Menu con do-while
    int scelta;
    do {
        printf("\n=== MENU ===\n");
        printf("1. Opzione 1\n");
        printf("2. Opzione 2\n");
        printf("3. Opzione 3\n");
        printf("0. Esci\n");
        printf("Scelta: ");
        scanf("%d", &scelta);

        switch (scelta) {
            case 1:
                printf("Hai scelto l'opzione 1\n");
                break;
            case 2:
                printf("Hai scelto l'opzione 2\n");
                break;
            case 3:
                printf("Hai scelto l'opzione 3\n");
                break;
            case 0:
                printf("Uscita...\n");
                break;
            default:
                printf("Scelta non valida\n");
        }
    } while (scelta != 0);

    return 0;
}
\end{lstlisting}

\begin{nota}
Usa \texttt{do-while} quando vuoi che il codice venga eseguito almeno una volta, come nei menu o nella validazione dell'input.
\end{nota}

\subsection{Ciclo for}

Il ciclo \texttt{for} è ideale quando si conosce in anticipo quante iterazioni fare.

\subsubsection{Sintassi}

\begin{lstlisting}
for (inizializzazione; condizione; incremento) {
    // codice da ripetere
}
\end{lstlisting}

\subsubsection{Componenti del for}

Il ciclo \texttt{for} è composto da tre componenti essenziali, ognuno dei quali svolge un ruolo specifico nel controllo del ciclo. L'\textbf{inizializzazione} è eseguita una sola volta, all'inizio del ciclo, e tipicamente imposta la variabile contatore al suo valore iniziale. La \textbf{condizione} è controllata prima di ogni iterazione: se la condizione è vera, il corpo del ciclo viene eseguito; se è falsa, il ciclo termina. L'\textbf{incremento} è eseguito dopo ogni iterazione e solitamente incrementa o decrementa la variabile contatore, permettendo così di avanzare verso la condizione di terminazione del ciclo.

\subsubsection{Esempi}

\begin{lstlisting}
#include <stdio.h>

int main(int argc, char** argv) {
    // Stampa numeri da 1 a 10
    for (int i = 1; i <= 10; i++) {
        printf("%d ", i);
    }
    printf("\n");

    // Stampa numeri pari da 2 a 20
    for (int i = 2; i <= 20; i += 2) {
        printf("%d ", i);
    }
    printf("\n");

    // Conta alla rovescia
    for (int i = 10; i >= 1; i--) {
        printf("%d ", i);
    }
    printf("\n");

    // Calcola fattoriale
    int n = 5;
    int fattoriale = 1;
    for (int i = 1; i <= n; i++) {
        fattoriale *= i;
    }
    printf("Il fattoriale di %d e': %d\n", n, fattoriale);

    return 0;
}
\end{lstlisting}

\subsubsection{for Annidati}

\begin{lstlisting}
#include <stdio.h>

int main(int argc, char** argv) {
    // Stampa una tabella pitagorica 5x5
    printf("Tabella Pitagorica:\n");
    for (int i = 1; i <= 5; i++) {
        for (int j = 1; j <= 5; j++) {
            printf("%4d", i * j);
        }
        printf("\n");
    }

    // Disegna un triangolo di asterischi
    printf("\nTriangolo:\n");
    for (int i = 1; i <= 5; i++) {
        for (int j = 1; j <= i; j++) {
            printf("*");
        }
        printf("\n");
    }

    return 0;
}
\end{lstlisting}

\section{Istruzioni di Controllo dei Cicli}

\subsection{Istruzione break}

L'istruzione \texttt{break} interrompe immediatamente il ciclo più interno.

\begin{lstlisting}
#include <stdio.h>

int main(int argc, char** argv) {
    // Trova il primo numero divisibile per 7
    for (int i = 1; i <= 100; i++) {
        if (i % 7 == 0) {
            printf("Primo numero divisibile per 7: %d\n", i);
            break;  // Esce dal ciclo
        }
    }

    // Esempio con while
    int numero;
    printf("Inserisci numeri (999 per uscire):\n");
    while (1) {  // Ciclo infinito
        scanf("%d", &numero);
        if (numero == 999) {
            break;  // Esce dal ciclo
        }
        printf("Hai inserito: %d\n", numero);
    }

    return 0;
}
\end{lstlisting}

\subsection{Istruzione continue}

L'istruzione \texttt{continue} salta alla prossima iterazione del ciclo.

\begin{lstlisting}
#include <stdio.h>

int main(int argc, char** argv) {
    // Stampa numeri da 1 a 10, saltando i multipli di 3
    for (int i = 1; i <= 10; i++) {
        if (i % 3 == 0) {
            continue;  // Salta questa iterazione
        }
        printf("%d ", i);  // Non viene eseguito per 3, 6, 9
    }
    printf("\n");

    // Calcola somma di numeri positivi
    int numeri[] = {5, -3, 8, -1, 12, -7, 4};
    int somma = 0;

    for (int i = 0; i < 7; i++) {
        if (numeri[i] < 0) {
            continue;  // Salta i numeri negativi
        }
        somma += numeri[i];
    }

    printf("Somma dei numeri positivi: %d\n", somma);
    return 0;
}
\end{lstlisting}

\begin{attenzione}
\texttt{break} e \texttt{continue} devono essere usati con cautela: un uso eccessivo può rendere il codice difficile da seguire.
\end{attenzione}

\subsection{Istruzione goto}

L'istruzione \texttt{goto} permette di saltare a un'etichetta specifica nel codice.

\begin{lstlisting}
#include <stdio.h>

int main(int argc, char** argv) {
    int i = 0;

inizio:  // Etichetta
    if (i < 5) {
        printf("%d ", i);
        i++;
        goto inizio;  // Salta all'etichetta
    }

    printf("\nFine\n");
    return 0;
}
\end{lstlisting}

\begin{attenzione}
L'uso di \texttt{goto} è generalmente sconsigliato perché rende il codice difficile da comprendere e mantenere. Usa le strutture di controllo standard quando possibile.
\end{attenzione}

\section{Confronto tra i Cicli}

\subsection{Quando usare quale ciclo}

La scelta tra i diversi tipi di cicli dipende dalla natura del problema che stai cercando di risolvere. Utilizza il ciclo \texttt{for} quando conosci in anticipo il numero esatto di iterazioni che devi eseguire:
\begin{lstlisting}
for (int i = 0; i < 10; i++) {
    // Esegui 10 volte
}
\end{lstlisting}
Scegli il ciclo \texttt{while} quando la condizione di terminazione deve essere controllata prima di ogni iterazione, poiché il corpo potrebbe non eseguirsi nemmeno una volta se la condizione è inizialmente falsa:
\begin{lstlisting}
while (numero != 0) {
    // Potrebbe non eseguire mai
}
\end{lstlisting}
Infine, utilizza il ciclo \texttt{do-while} quando devi garantire che il corpo del ciclo venga eseguito almeno una volta, indipendentemente dalla condizione iniziale:
\begin{lstlisting}
do {
    // Esegue sempre almeno una volta
} while (condizione);
\end{lstlisting}

\subsection{Equivalenza tra i Cicli}

Questi tre cicli sono equivalenti:

\begin{lstlisting}
// Ciclo for
for (int i = 0; i < 10; i++) {
    printf("%d ", i);
}

// Equivalente con while
int i = 0;
while (i < 10) {
    printf("%d ", i);
    i++;
}

// Equivalente con do-while (se si esegue almeno una volta)
int i = 0;
if (i < 10) {
    do {
        printf("%d ", i);
        i++;
    } while (i < 10);
}
\end{lstlisting}

\section{Esempi Completi}

\subsection{Calcolatrice con Menu}

\begin{lstlisting}
#include <stdio.h>

int main(int argc, char** argv) {
    float num1, num2, risultato;
    int scelta;

    do {
        printf("\n=== CALCOLATRICE ===\n");
        printf("1. Addizione\n");
        printf("2. Sottrazione\n");
        printf("3. Moltiplicazione\n");
        printf("4. Divisione\n");
        printf("0. Esci\n");
        printf("Scelta: ");
        scanf("%d", &scelta);

        if (scelta == 0) {
            break;
        }

        if (scelta < 1 || scelta > 4) {
            printf("Scelta non valida!\n");
            continue;
        }

        printf("Inserisci primo numero: ");
        scanf("%f", &num1);
        printf("Inserisci secondo numero: ");
        scanf("%f", &num2);

        switch (scelta) {
            case 1:
                risultato = num1 + num2;
                printf("%.2f + %.2f = %.2f\n", num1, num2, risultato);
                break;
            case 2:
                risultato = num1 - num2;
                printf("%.2f - %.2f = %.2f\n", num1, num2, risultato);
                break;
            case 3:
                risultato = num1 * num2;
                printf("%.2f * %.2f = %.2f\n", num1, num2, risultato);
                break;
            case 4:
                if (num2 != 0) {
                    risultato = num1 / num2;
                    printf("%.2f / %.2f = %.2f\n",
                           num1, num2, risultato);
                } else {
                    printf("Errore: divisione per zero!\n");
                }
                break;
        }
    } while (1);

    printf("Arrivederci!\n");
    return 0;
}
\end{lstlisting}

\subsection{Gioco: Indovina il Numero}

\begin{lstlisting}
#include <stdio.h>
#include <stdlib.h>
#include <time.h>

int main(int argc, char** argv) {
    int numero_segreto, tentativo, tentativi = 0;

    // Inizializza il generatore di numeri casuali
    srand(time(NULL));
    numero_segreto = rand() % 100 + 1;  // Numero tra 1 e 100

    printf("=== INDOVINA IL NUMERO ===\n");
    printf("Ho pensato un numero tra 1 e 100.\n");

    do {
        printf("\nInserisci il tuo tentativo: ");
        scanf("%d", &tentativo);
        tentativi++;

        if (tentativo < numero_segreto) {
            printf("Troppo basso! Riprova.\n");
        } else if (tentativo > numero_segreto) {
            printf("Troppo alto! Riprova.\n");
        } else {
            printf("\nComplimenti! Hai indovinato!\n");
            printf("Numero di tentativi: %d\n", tentativi);
        }
    } while (tentativo != numero_segreto);

    return 0;
}
\end{lstlisting}

\section{Esercizi}

\subsection{Livello Base}

\begin{enumerate}
    \item Scrivi un programma che determina se un numero è positivo, negativo o zero.
    \item Crea un programma che stampa i numeri da 1 a 100.
    \item Scrivi un programma che calcola la somma dei primi N numeri naturali.
    \item Implementa un programma che stampa la tavola pitagorica di un numero dato.
\end{enumerate}

\subsection{Livello Intermedio}

\begin{enumerate}
    \item Scrivi un programma che verifica se un numero è primo.
    \item Crea un programma che inverte un numero (es: 1234 diventa 4321).
    \item Implementa un programma che calcola il massimo comun divisore (MCD) di due numeri usando l'algoritmo di Euclide.
    \item Scrivi un programma che disegna un triangolo di numeri:
    \begin{verbatim}
    1
    12
    123
    1234
    12345
    \end{verbatim}
\end{enumerate}

\subsection{Livello Avanzato}

\begin{enumerate}
    \item Crea un programma che genera la sequenza di Fibonacci fino a N termini.
    \item Scrivi un programma che verifica se un numero è un palindromo.
    \item Implementa un programma che trova tutti i numeri perfetti minori di 1000 (un numero perfetto è uguale alla somma dei suoi divisori propri).
    \item Crea un programma che disegna il triangolo di Floyd:
    \begin{verbatim}
    1
    2 3
    4 5 6
    7 8 9 10
    \end{verbatim}
\end{enumerate}
