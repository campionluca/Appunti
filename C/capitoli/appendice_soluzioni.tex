\chapter*{Appendice: Soluzioni ad Alcuni Esercizi}
\addcontentsline{toc}{chapter}{Appendice: Soluzioni ad Alcuni Esercizi}

Questa appendice contiene le soluzioni di alcuni esercizi selezionati dai capitoli precedenti. Per ogni esercizio viene fornita una soluzione commentata.

\section*{Capitolo 2: Variabili e Tipi di Dati}

\subsection*{Esercizio Base 2: Area di un Cerchio}

\begin{lstlisting}
#include <stdio.h>
#define PI 3.14159

int main() {
    float raggio, area;

    printf("Inserisci il raggio del cerchio: ");
    scanf("%f", &raggio);

    // Formula: area = PI * r^2
    area = PI * raggio * raggio;

    printf("L'area del cerchio e': %.2f\n", area);

    return 0;
}
\end{lstlisting}

\subsection*{Esercizio Intermedio 1: Secondi in Ore, Minuti, Secondi}

\begin{lstlisting}
#include <stdio.h>

int main() {
    int secondi_totali, ore, minuti, secondi;

    printf("Inserisci il numero di secondi: ");
    scanf("%d", &secondi_totali);

    ore = secondi_totali / 3600;
    minuti = (secondi_totali % 3600) / 60;
    secondi = secondi_totali % 60;

    printf("%d secondi corrispondono a:\n", secondi_totali);
    printf("%d ore, %d minuti, %d secondi\n", ore, minuti, secondi);

    return 0;
}
\end{lstlisting}

\section*{Capitolo 3: Operatori ed Espressioni}

\subsection*{Esercizio Base 3: Pari o Dispari}

\begin{lstlisting}
#include <stdio.h>

int main() {
    int numero;

    printf("Inserisci un numero: ");
    scanf("%d", &numero);

    if (numero % 2 == 0) {
        printf("%d e' pari\n", numero);
    } else {
        printf("%d e' dispari\n", numero);
    }

    return 0;
}
\end{lstlisting}

\subsection*{Esercizio Intermedio 3: Anno Bisestile}

\begin{lstlisting}
#include <stdio.h>

int main() {
    int anno;

    printf("Inserisci un anno: ");
    scanf("%d", &anno);

    // Un anno e' bisestile se:
    // - divisibile per 4 E
    // - (non divisibile per 100 OPPURE divisibile per 400)
    if ((anno % 4 == 0) && (anno % 100 != 0 || anno % 400 == 0)) {
        printf("%d e' un anno bisestile\n", anno);
    } else {
        printf("%d non e' un anno bisestile\n", anno);
    }

    return 0;
}
\end{lstlisting}

\section*{Capitolo 4: Controllo di Flusso}

\subsection*{Esercizio Intermedio 1: Numero Primo}

\begin{lstlisting}
#include <stdio.h>

int main() {
    int numero, i, primo = 1;

    printf("Inserisci un numero: ");
    scanf("%d", &numero);

    if (numero <= 1) {
        primo = 0;
    } else {
        for (i = 2; i * i <= numero; i++) {
            if (numero % i == 0) {
                primo = 0;
                break;
            }
        }
    }

    if (primo) {
        printf("%d e' un numero primo\n", numero);
    } else {
        printf("%d non e' un numero primo\n", numero);
    }

    return 0;
}
\end{lstlisting}

\subsection*{Esercizio Intermedio 2: Inversione di un Numero}

\begin{lstlisting}
#include <stdio.h>

int main() {
    int numero, inverso = 0, cifra;

    printf("Inserisci un numero: ");
    scanf("%d", &numero);

    int originale = numero;

    while (numero != 0) {
        cifra = numero % 10;
        inverso = inverso * 10 + cifra;
        numero /= 10;
    }

    printf("Il numero %d invertito e': %d\n", originale, inverso);

    return 0;
}
\end{lstlisting}

\subsection*{Esercizio Avanzato 1: Sequenza di Fibonacci}

\begin{lstlisting}
#include <stdio.h>

int main() {
    int n, i;
    int primo = 0, secondo = 1, successivo;

    printf("Quanti termini della sequenza Fibonacci? ");
    scanf("%d", &n);

    printf("Sequenza di Fibonacci:\n");

    for (i = 0; i < n; i++) {
        if (i <= 1) {
            successivo = i;
        } else {
            successivo = primo + secondo;
            primo = secondo;
            secondo = successivo;
        }
        printf("%d ", successivo);
    }
    printf("\n");

    return 0;
}
\end{lstlisting}

\section*{Capitolo 5: Funzioni}

\subsection*{Esercizio Base 2: Verifica Numero Pari}

\begin{lstlisting}
#include <stdio.h>

int is_pari(int n) {
    return (n % 2 == 0);
}

int main() {
    int numero;

    printf("Inserisci un numero: ");
    scanf("%d", &numero);

    if (is_pari(numero)) {
        printf("%d e' pari\n", numero);
    } else {
        printf("%d e' dispari\n", numero);
    }

    return 0;
}
\end{lstlisting}

\subsection*{Esercizio Intermedio 1: Inversione Numero con Funzione}

\begin{lstlisting}
#include <stdio.h>

int inverti_numero(int n) {
    int inverso = 0;

    while (n != 0) {
        inverso = inverso * 10 + (n % 10);
        n /= 10;
    }

    return inverso;
}

int main() {
    int numero;

    printf("Inserisci un numero: ");
    scanf("%d", &numero);

    int invertito = inverti_numero(numero);

    printf("%d invertito e': %d\n", numero, invertito);

    return 0;
}
\end{lstlisting}

\subsection*{Esercizio Intermedio 2: Somma Cifre Ricorsiva}

\begin{lstlisting}
#include <stdio.h>

int somma_cifre(int n) {
    // Caso base
    if (n == 0) {
        return 0;
    }
    // Caso ricorsivo
    return (n % 10) + somma_cifre(n / 10);
}

int main() {
    int numero;

    printf("Inserisci un numero: ");
    scanf("%d", &numero);

    int somma = somma_cifre(numero);

    printf("La somma delle cifre di %d e': %d\n", numero, somma);

    return 0;
}
\end{lstlisting}

\section*{Capitolo 6: Array}

\subsection*{Esercizio Base 1: Conta Pari e Dispari}

\begin{lstlisting}
#include <stdio.h>

int main() {
    int numeri[10];
    int pari = 0, dispari = 0;

    printf("Inserisci 10 numeri:\n");
    for (int i = 0; i < 10; i++) {
        printf("Numero %d: ", i + 1);
        scanf("%d", &numeri[i]);

        if (numeri[i] % 2 == 0) {
            pari++;
        } else {
            dispari++;
        }
    }

    printf("\nNumeri pari: %d\n", pari);
    printf("Numeri dispari: %d\n", dispari);

    return 0;
}
\end{lstlisting}

\subsection*{Esercizio Base 4: Array Palindromo}

\begin{lstlisting}
#include <stdio.h>

int main() {
    int arr[] = {1, 2, 3, 2, 1};
    int n = sizeof(arr) / sizeof(arr[0]);
    int palindromo = 1;

    for (int i = 0; i < n / 2; i++) {
        if (arr[i] != arr[n - 1 - i]) {
            palindromo = 0;
            break;
        }
    }

    if (palindromo) {
        printf("L'array e' palindromo\n");
    } else {
        printf("L'array non e' palindromo\n");
    }

    return 0;
}
\end{lstlisting}

\subsection*{Esercizio Intermedio 1: Ricerca Binaria}

\begin{lstlisting}
#include <stdio.h>

int ricerca_binaria(int arr[], int n, int valore) {
    int sinistra = 0, destra = n - 1;

    while (sinistra <= destra) {
        int medio = sinistra + (destra - sinistra) / 2;

        if (arr[medio] == valore) {
            return medio;
        }

        if (arr[medio] < valore) {
            sinistra = medio + 1;
        } else {
            destra = medio - 1;
        }
    }

    return -1;  // Non trovato
}

int main() {
    int arr[] = {2, 5, 8, 12, 16, 23, 38, 45, 56, 67, 78};
    int n = sizeof(arr) / sizeof(arr[0]);
    int valore = 23;

    int posizione = ricerca_binaria(arr, n, valore);

    if (posizione != -1) {
        printf("%d trovato alla posizione %d\n", valore, posizione);
    } else {
        printf("%d non trovato\n", valore);
    }

    return 0;
}
\end{lstlisting}

\section*{Capitolo 7: Puntatori}

\subsection*{Esercizio Base 1: Scambio Ciclico}

\begin{lstlisting}
#include <stdio.h>

void scambio_ciclico(int *a, int *b, int *c) {
    int temp = *a;
    *a = *b;
    *b = *c;
    *c = temp;
}

int main() {
    int x = 1, y = 2, z = 3;

    printf("Prima: x=%d, y=%d, z=%d\n", x, y, z);
    scambio_ciclico(&x, &y, &z);
    printf("Dopo: x=%d, y=%d, z=%d\n", x, y, z);

    return 0;
}
\end{lstlisting}

\subsection*{Esercizio Base 3: Max e Min con Puntatori}

\begin{lstlisting}
#include <stdio.h>

void trova_max_min(int arr[], int n, int *max, int *min) {
    *max = arr[0];
    *min = arr[0];

    for (int i = 1; i < n; i++) {
        if (arr[i] > *max) {
            *max = arr[i];
        }
        if (arr[i] < *min) {
            *min = arr[i];
        }
    }
}

int main() {
    int numeri[] = {45, 12, 78, 23, 67, 89, 34};
    int n = sizeof(numeri) / sizeof(numeri[0]);
    int max, min;

    trova_max_min(numeri, n, &max, &min);

    printf("Massimo: %d\n", max);
    printf("Minimo: %d\n", min);

    return 0;
}
\end{lstlisting}

\section*{Capitolo 8: Stringhe}

\subsection*{Esercizio Base 1: Conta Vocali}

\begin{lstlisting}
#include <stdio.h>
#include <ctype.h>

int conta_vocali(char *str) {
    int conta = 0;

    for (int i = 0; str[i] != '\0'; i++) {
        char c = tolower(str[i]);
        if (c == 'a' || c == 'e' || c == 'i' ||
            c == 'o' || c == 'u') {
            conta++;
        }
    }

    return conta;
}

int main() {
    char frase[100];

    printf("Inserisci una frase: ");
    fgets(frase, sizeof(frase), stdin);

    int vocali = conta_vocali(frase);

    printf("Numero di vocali: %d\n", vocali);

    return 0;
}
\end{lstlisting}

\subsection*{Esercizio Intermedio 2: Anagrammi}

\begin{lstlisting}
#include <stdio.h>
#include <string.h>
#include <ctype.h>

int sono_anagrammi(char *str1, char *str2) {
    int freq1[26] = {0}, freq2[26] = {0};

    // Conta frequenza caratteri prima stringa
    for (int i = 0; str1[i] != '\0'; i++) {
        if (isalpha(str1[i])) {
            freq1[tolower(str1[i]) - 'a']++;
        }
    }

    // Conta frequenza caratteri seconda stringa
    for (int i = 0; str2[i] != '\0'; i++) {
        if (isalpha(str2[i])) {
            freq2[tolower(str2[i]) - 'a']++;
        }
    }

    // Confronta frequenze
    for (int i = 0; i < 26; i++) {
        if (freq1[i] != freq2[i]) {
            return 0;
        }
    }

    return 1;
}

int main() {
    char str1[] = "listen";
    char str2[] = "silent";

    if (sono_anagrammi(str1, str2)) {
        printf("\"%s\" e \"%s\" sono anagrammi\n", str1, str2);
    } else {
        printf("\"%s\" e \"%s\" non sono anagrammi\n", str1, str2);
    }

    return 0;
}
\end{lstlisting}

\section*{Capitolo 9: Struct}

\subsection*{Esercizio Base 1: Struct Rettangolo}

\begin{lstlisting}
#include <stdio.h>

typedef struct {
    float base;
    float altezza;
} Rettangolo;

float area(Rettangolo r) {
    return r.base * r.altezza;
}

float perimetro(Rettangolo r) {
    return 2 * (r.base + r.altezza);
}

int main() {
    Rettangolo r;

    printf("Inserisci base: ");
    scanf("%f", &r.base);
    printf("Inserisci altezza: ");
    scanf("%f", &r.altezza);

    printf("\nArea: %.2f\n", area(r));
    printf("Perimetro: %.2f\n", perimetro(r));

    return 0;
}
\end{lstlisting}

\subsection*{Esercizio Base 3: Conto Corrente}

\begin{lstlisting}
#include <stdio.h>

typedef struct {
    int numero;
    char intestatario[50];
    float saldo;
} ContoCorrente;

void deposita(ContoCorrente *conto, float importo) {
    if (importo > 0) {
        conto->saldo += importo;
        printf("Depositati %.2f euro\n", importo);
        printf("Nuovo saldo: %.2f euro\n", conto->saldo);
    } else {
        printf("Importo non valido!\n");
    }
}

void preleva(ContoCorrente *conto, float importo) {
    if (importo > 0 && importo <= conto->saldo) {
        conto->saldo -= importo;
        printf("Prelevati %.2f euro\n", importo);
        printf("Nuovo saldo: %.2f euro\n", conto->saldo);
    } else {
        printf("Operazione non consentita!\n");
    }
}

void stampa_saldo(ContoCorrente conto) {
    printf("\n=== Conto %d ===\n", conto.numero);
    printf("Intestatario: %s\n", conto.intestatario);
    printf("Saldo: %.2f euro\n", conto.saldo);
}

int main() {
    ContoCorrente conto = {12345, "Mario Rossi", 1000.00};

    stampa_saldo(conto);

    deposita(&conto, 500.00);
    preleva(&conto, 200.00);
    preleva(&conto, 2000.00);  // Saldo insufficiente

    stampa_saldo(conto);

    return 0;
}
\end{lstlisting}

\section*{Capitolo 10: File}

\subsection*{Esercizio Base 3: Conta Carattere in File}

\begin{lstlisting}
#include <stdio.h>

int main() {
    FILE *file;
    char nome_file[100];
    char carattere_cerca, c;
    int conta = 0;

    printf("Nome del file: ");
    scanf("%s", nome_file);

    printf("Carattere da cercare: ");
    scanf(" %c", &carattere_cerca);

    file = fopen(nome_file, "r");
    if (file == NULL) {
        printf("Impossibile aprire il file!\n");
        return 1;
    }

    while ((c = fgetc(file)) != EOF) {
        if (c == carattere_cerca) {
            conta++;
        }
    }

    fclose(file);

    printf("Il carattere '%c' appare %d volte\n",
           carattere_cerca, conta);

    return 0;
}
\end{lstlisting}

\subsection*{Esercizio Intermedio 3: Cifrario di Cesare}

\begin{lstlisting}
#include <stdio.h>
#include <ctype.h>

void cifra_file(char *input, char *output, int chiave) {
    FILE *fin = fopen(input, "r");
    FILE *fout = fopen(output, "w");

    if (fin == NULL || fout == NULL) {
        printf("Errore apertura file!\n");
        return;
    }

    char c;
    while ((c = fgetc(fin)) != EOF) {
        if (isalpha(c)) {
            char base = isupper(c) ? 'A' : 'a';
            c = base + (c - base + chiave) % 26;
        }
        fputc(c, fout);
    }

    fclose(fin);
    fclose(fout);
    printf("File cifrato con successo!\n");
}

void decifra_file(char *input, char *output, int chiave) {
    cifra_file(input, output, 26 - chiave);
}

int main() {
    int scelta, chiave;
    char input[100], output[100];

    printf("1. Cifra file\n");
    printf("2. Decifra file\n");
    printf("Scelta: ");
    scanf("%d", &scelta);

    printf("File input: ");
    scanf("%s", input);
    printf("File output: ");
    scanf("%s", output);
    printf("Chiave (0-25): ");
    scanf("%d", &chiave);

    if (scelta == 1) {
        cifra_file(input, output, chiave);
    } else if (scelta == 2) {
        decifra_file(input, output, chiave);
    }

    return 0;
}
\end{lstlisting}

\section*{Note Finali}

Queste soluzioni rappresentano solo uno dei possibili approcci per risolvere gli esercizi. Ci sono spesso modi alternativi, ugualmente validi, per affrontare un problema.

Quando studi queste soluzioni:
\begin{itemize}
    \item Cerca di capire la logica, non solo copiare il codice
    \item Prova a modificare le soluzioni per renderle migliori
    \item Confronta le tue soluzioni con queste
    \item Impara dai diversi approcci utilizzati
\end{itemize}
