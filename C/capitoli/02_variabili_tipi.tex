\chapter{Variabili e Tipi di Dati}
\label{cap:variabili_tipi}

\section{Introduzione}

Le variabili sono uno dei concetti fondamentali della programmazione. Una \textbf{variabile} è un contenitore che memorizza un valore che può cambiare durante l'esecuzione del programma. Pensate a una variabile come a una scatola etichettata dove potete conservare informazioni.

\section{Obiettivi di Apprendimento}

Alla fine di questo capitolo sarai in grado di:

\begin{itemize}
    \item Comprendere il concetto di variabile
    \item Dichiarare e inizializzare variabili
    \item Conoscere i tipi di dati base del C
    \item Usare gli specificatori di formato della funzione printf()
    \item Leggere input dall'utente con la funzione scanf() della libreria stdio
    \item Lavorare con costanti
\end{itemize}

\section{Concetto di Variabile}

Una variabile è caratterizzata da tre attributi fondamentali che la identificano e ne definiscono il comportamento. Il \textbf{nome} è un identificatore univoco assegnato alla variabile (ad esempio \texttt{eta} o \texttt{altezza}), che permette al programmatore di fare riferimento ad essa nel codice. Il \textbf{tipo} determina quale categoria di dati la variabile può contenere, come numeri interi, numeri decimali, o caratteri; il tipo vincola sia i valori rappresentabili che le operazioni consentite. Infine, il \textbf{valore} è il dato effettivamente memorizzato dalla variabile in memoria in un certo momento dell'esecuzione del programma, che può essere modificato durante l'esecuzione.

\begin{lstlisting}[caption={Esempio concettuale di variabile}]
int eta = 17;
// nome della variabile: eta
// tipo della variabile: int (intero)
// valore della variabile: 17
\end{lstlisting}

\section{Tipi di Dati Base}

Il C è un linguaggio \textbf{fortemente tipizzato}: ogni variabile deve avere un tipo dichiarato esplicitamente.

\subsection{Tipi Interi}

I tipi interi memorizzano numeri senza parte decimale.

\begin{table}[h]
\centering
\begin{tabular}{|l|l|l|l|}
\hline
\textbf{Tipo} & \textbf{Dimensione} & \textbf{Range (tipico)} & \textbf{Formato} \\
\hline
\texttt{char} & 1 byte & -128 a 127 & \texttt{\%c} \\
\texttt{unsigned char} & 1 byte & 0 a 255 & \texttt{\%c} \\
\texttt{short} & 2 byte & -32,768 a 32,767 & \texttt{\%hd} \\
\texttt{unsigned short} & 2 byte & 0 a 65,535 & \texttt{\%hu} \\
\texttt{int} & 4 byte & -2.147.483.648 a 2.147.483.647 & \texttt{\%d} \\
\texttt{unsigned int} & 4 byte & 0 a 4.294.967.295 & \texttt{\%u} \\
\texttt{long} & 4-8 byte & Dipende dalla piattaforma & \texttt{\%ld} \\
\texttt{unsigned long} & 4-8 byte & Dipende dalla piattaforma & \texttt{\%lu} \\
\texttt{long long} & 8 byte & Molto ampio & \texttt{\%lld} \\
\hline
\end{tabular}
\caption{Tipi interi in C}
\end{table}

\begin{nota}
La parola chiave \texttt{unsigned} indica che la variabile può contenere solo valori positivi (senza segno). Questo raddoppia il valore massimo rappresentabile.
\end{nota}

\subsection{Tipi a Virgola Mobile}

I tipi a virgola mobile memorizzano numeri con parte decimale.

\begin{table}[h]
\centering
\begin{tabular}{|l|l|l|l|}
\hline
\textbf{Tipo} & \textbf{Dimensione} & \textbf{Precisione} & \textbf{Formato} \\
\hline
\texttt{float} & 4 byte & 6-7 cifre decimali & \texttt{\%f} \\
\texttt{double} & 8 byte & 15-16 cifre decimali & \texttt{\%lf} \\
\texttt{long double} & 10-16 byte & 18-19 cifre decimali & \texttt{\%Lf} \\
\hline
\end{tabular}
\caption{Tipi a virgola mobile in C}
\end{table}

\subsection{Tipo Carattere}

\begin{lstlisting}[caption={Tipo char}]
char lettera = 'A';     // La variabile lettera contiene un singolo carattere
char simbolo = '#';     // La variabile simbolo contiene il carattere '#'
char numero = '5';      // La variabile numero contiene il carattere '5', non il numero 5!
\end{lstlisting}

\begin{attenzione}
Un carattere deve essere racchiuso tra apici singoli \texttt{'A'}, mentre una stringa usa apici doppi \texttt{"Ciao"}. Non confonderli!
\end{attenzione}

\section{Dichiarazione di Variabili}

\subsection{Sintassi Base}

\begin{lstlisting}[caption={Sintassi di dichiarazione}]
tipo nome_variabile;
\end{lstlisting}

\subsection{Esempi di Dichiarazione}

\begin{lstlisting}[caption={Dichiarazione di variabili}]
int eta;              // Dichiarazione semplice
float altezza;
char iniziale;

int x, y, z;          // Dichiarazione multipla
float base, altezza, area;
\end{lstlisting}

\subsection{Inizializzazione}

\textbf{Inizializzare} una variabile significa assegnarle un valore iniziale al momento della dichiarazione.

\begin{lstlisting}[caption={Inizializzazione di variabili}]
int eta = 17;                    // Dichiarazione + inizializzazione della variabile eta
float altezza = 1.75;            // Dichiarazione + inizializzazione della variabile altezza
char grado = 'A';                // Dichiarazione + inizializzazione della variabile grado

// Si puo' anche fare in due passaggi
int numero;                      // Dichiarazione della variabile numero
numero = 42;                     // Assegnamento del valore 42 alla variabile numero
\end{lstlisting}

\begin{nota}
È buona pratica inizializzare sempre le variabili. Una variabile non inizializzata contiene un valore casuale (quello che c'era in memoria prima).
\end{nota}

\section{Regole per i Nomi delle Variabili}

I nomi delle variabili (identificatori) devono seguire queste regole:

\subsection{Regole Obbligatorie}

\begin{enumerate}
    \item Possono contenere lettere (a-z, A-Z), cifre (0-9) e underscore (\_)
    \item Devono iniziare con una lettera o underscore (non con una cifra)
    \item Non possono essere parole chiave del C (es. \texttt{int}, \texttt{return}, \texttt{if})
    \item Sono case-sensitive: \texttt{eta}, \texttt{Eta} e \texttt{ETA} sono variabili diverse
\end{enumerate}

\subsection{Convenzioni Consigliate}

\begin{enumerate}
    \item Usa nomi significativi: \texttt{eta} invece di \texttt{x}
    \item Per nomi composti usa snake\_case: \texttt{numero\_studenti}
    \item Usa nomi brevi ma descrittivi
    \item Evita nomi di una sola lettera (tranne per contatori come \texttt{i}, \texttt{j}, \texttt{k})
\end{enumerate}

\begin{lstlisting}[caption={Esempi di nomi validi e non}]
// VALIDI
int eta;
int numero_studenti;
float _temp;
char scelta1;

// NON VALIDI
int 1numero;          // Inizia con una cifra
int for;              // Parola chiave
int numero-studenti;  // Contiene trattino
\end{lstlisting}

\section{L'operatore sizeof}

L'operatore \texttt{sizeof} del linguaggio C restituisce la dimensione in byte di un tipo di dato o di una variabile.

\begin{lstlisting}[caption={Uso di sizeof}]
#include <stdio.h>

int main(int argc, char** argv) {
    printf("Dimensione di int: %lu byte\n", sizeof(int));
    printf("Dimensione di float: %lu byte\n", sizeof(float));
    printf("Dimensione di double: %lu byte\n", sizeof(double));
    printf("Dimensione di char: %lu byte\n", sizeof(char));

    int x;
    printf("Dimensione della variabile x: %lu byte\n", sizeof(x));

    return 0;
}
\end{lstlisting}

Output tipico:
\begin{verbatim}
Dimensione di int: 4 byte
Dimensione di float: 4 byte
Dimensione di double: 8 byte
Dimensione di char: 1 byte
Dimensione di x: 4 byte
\end{verbatim}

\begin{nota}
Le dimensioni possono variare in base all'architettura del sistema (32-bit vs 64-bit) e al compilatore.
\end{nota}

\section{Specificatori di Formato per printf}

Gli specificatori di formato indicano alla funzione printf() della libreria stdio come stampare un valore.

\begin{table}[h]
\centering
\begin{tabular}{|l|l|l|}
\hline
\textbf{Specificatore} & \textbf{Tipo} & \textbf{Descrizione} \\
\hline
\texttt{\%d} o \texttt{\%i} & int & Intero decimale con segno \\
\texttt{\%u} & unsigned int & Intero decimale senza segno \\
\texttt{\%f} & float/double & Numero a virgola mobile \\
\texttt{\%lf} & double (scanf) & Double (per input) \\
\texttt{\%c} & char & Singolo carattere \\
\texttt{\%s} & char* & Stringa \\
\texttt{\%x} & int & Intero in esadecimale \\
\texttt{\%o} & int & Intero in ottale \\
\texttt{\%p} & puntatore & Indirizzo di memoria \\
\texttt{\%\%} & - & Stampa il simbolo \% \\
\hline
\end{tabular}
\caption{Principali specificatori di formato}
\end{table}

\subsection{Formattazione Avanzata}

\begin{lstlisting}[caption={Controllo del formato di stampa}]
#include <stdio.h>

int main(int argc, char** argv) {
    int numero = 42;
    float pi = 3.14159;

    printf("Numero: %d\n", numero);           // Stampa il valore della variabile numero: 42
    printf("Con padding: %5d\n", numero);     // Stampa il valore della variabile numero con padding:    42 (5 caratteri)
    printf("Con zero: %05d\n", numero);       // Stampa il valore della variabile numero con zeri: 00042

    printf("Float: %f\n", pi);                // Stampa il valore della variabile pi: 3.141590
    printf("2 decimali: %.2f\n", pi);         // Stampa il valore della variabile pi con 2 decimali: 3.14
    printf("6 decimali: %.6f\n", pi);         // Stampa il valore della variabile pi con 6 decimali: 3.141590

    return 0;
}
\end{lstlisting}

\section{Input da Tastiera: scanf}

La funzione scanf() della libreria stdio legge input dall'utente.

\subsection{Sintassi Base}

\begin{lstlisting}[caption={Sintassi di scanf}]
scanf("specificatore", &variabile);
\end{lstlisting}

\begin{attenzione}
Nota l'operatore \texttt{\&} (ampersand) prima del nome della variabile! Questo operatore indica alla funzione scanf() l'indirizzo di memoria della variabile dove salvare il valore letto. Dimenticarlo è uno degli errori più comuni.
\end{attenzione}

\subsection{Esempi di Input}

\begin{lstlisting}[caption={Leggere input con scanf}]
#include <stdio.h>

int main(int argc, char** argv) {
    int eta;
    float altezza;
    char iniziale;

    printf("Inserisci la tua eta': ");
    scanf("%d", &eta);  // Legge un intero e lo salva nella variabile eta

    printf("Inserisci la tua altezza (in metri): ");
    scanf("%f", &altezza);  // Legge un float e lo salva nella variabile altezza

    printf("Inserisci l'iniziale del tuo nome: ");
    scanf(" %c", &iniziale);  // Legge un char e lo salva nella variabile iniziale (nota lo spazio prima di %c)

    printf("\nRiepilogo:\n");
    printf("Eta': %d anni\n", eta);  // Stampa il valore della variabile eta
    printf("Altezza: %.2f m\n", altezza);  // Stampa il valore della variabile altezza
    printf("Iniziale: %c\n", iniziale);  // Stampa il valore della variabile iniziale

    return 0;
}
\end{lstlisting}

\begin{errore}
Quando si legge un carattere con la funzione scanf() dopo aver letto numeri, è necessario aggiungere uno spazio prima di \texttt{\%c} per consumare eventuali caratteri di whitespace rimasti nel buffer di input:
\begin{lstlisting}
scanf(" %c", &carattere);  // Lo spazio prima di %c e' importante!
\end{lstlisting}
\end{errore}

\section{Costanti}

Le \textbf{costanti} sono valori che non possono essere modificati durante l'esecuzione del programma.

\subsection{Costanti con \#define}

\begin{lstlisting}[caption={Costanti con \#define}]
#include <stdio.h>

#define PI 3.14159
#define MAX_STUDENTI 30
#define SALUTO "Ciao a tutti!"

int main(int argc, char** argv) {
    float raggio = 5.0;
    float area = PI * raggio * raggio;  // Calcola l'area usando la costante PI e la variabile raggio

    printf("Area del cerchio: %.2f\n", area);  // Stampa il valore della variabile area
    printf("Numero massimo studenti: %d\n", MAX_STUDENTI);  // Stampa il valore della costante MAX_STUDENTI
    printf("%s\n", SALUTO);  // Stampa il valore della costante SALUTO

    // PI = 3.14;  // ERRORE: non si puo' modificare il valore della costante PI

    return 0;
}
\end{lstlisting}

\begin{nota}
Per convenzione, i nomi delle costanti si scrivono in MAIUSCOLO con underscore.
\end{nota}

\subsection{Costanti con const}

\begin{lstlisting}[caption={Costanti con const}]
#include <stdio.h>

int main(int argc, char** argv) {
    const float PI = 3.14159;
    const int GIORNI_SETTIMANA = 7;

    float raggio = 3.0;
    float area = PI * raggio * raggio;  // Calcola l'area usando la costante PI e la variabile raggio

    printf("Area: %.2f\n", area);  // Stampa il valore della variabile area
    printf("Giorni in una settimana: %d\n", GIORNI_SETTIMANA);  // Stampa il valore della costante GIORNI_SETTIMANA

    // PI = 3.14;  // ERRORE: la costante PI non e' modificabile

    return 0;
}
\end{lstlisting}

\subsection{Differenza tra \#define e const}

\begin{table}[h]
\centering
\begin{tabular}{|l|p{5cm}|p{5cm}|}
\hline
\textbf{Aspetto} & \textbf{\#define} & \textbf{const} \\
\hline
Elaborazione & Preprocessore (sostituzione testuale) & Compilatore (variabile vera) \\
Tipo & No tipo esplicito & Ha un tipo esplicito \\
Debug & Più difficile & Più facile \\
Scope & Globale (dal punto di definizione) & Rispetta lo scope \\
\hline
\end{tabular}
\caption{Confronto tra \#define e const}
\end{table}

\section{Conversioni di Tipo}

\subsection{Conversione Implicita}

Il C converte automaticamente i tipi quando necessario:

\begin{lstlisting}[caption={Conversione implicita}]
#include <stdio.h>

int main(int argc, char** argv) {
    int intero = 10;
    float decimale = 3.5;

    float risultato = intero + decimale;  // Il valore della variabile intero viene convertito automaticamente in float
    printf("Risultato: %.2f\n", risultato);  // Stampa il valore della variabile risultato: 13.50

    int troncato = intero + decimale;  // Il risultato float viene troncato e assegnato alla variabile troncato
    printf("Troncato: %d\n", troncato);  // Stampa il valore della variabile troncato: 13

    return 0;
}
\end{lstlisting}

\begin{attenzione}
Quando si assegna un float a un int, la parte decimale viene \textbf{troncata} (tagliata via), non arrotondata!
\end{attenzione}

\subsection{Conversione Esplicita (Cast)}

\begin{lstlisting}[caption={Cast esplicito}]
#include <stdio.h>

int main(int argc, char** argv) {
    int a = 7, b = 2;

    // Divisione intera
    float ris1 = a / b;  // Le variabili a e b sono int, quindi la divisione e' intera
    printf("Senza cast: %.2f\n", ris1);  // Stampa il valore della variabile ris1: 3.00 (divisione intera!)

    // Con cast
    float ris2 = (float)a / b;  // Il valore della variabile a viene convertito in float prima della divisione
    printf("Con cast: %.2f\n", ris2);  // Stampa il valore della variabile ris2: 3.50 (divisione float)

    // Cast per arrotondamento
    float x = 3.7;
    int arrotondato = (int)(x + 0.5);  // Il valore della variabile x viene arrotondato
    printf("Arrotondato: %d\n", arrotondato);  // Stampa il valore della variabile arrotondato: 4

    return 0;
}
\end{lstlisting}

\section{Esempi Pratici Completi}

\subsection{Esempio 1: Calcolo Area Rettangolo}

\begin{lstlisting}[caption={Calcolo area e perimetro di un rettangolo}]
#include <stdio.h>

int main(int argc, char** argv) {
    float base, altezza;
    float area, perimetro;

    printf("=== CALCOLO AREA RETTANGOLO ===\n\n");

    printf("Inserisci la base: ");
    scanf("%f", &base);

    printf("Inserisci l'altezza: ");
    scanf("%f", &altezza);

    area = base * altezza;
    perimetro = 2 * (base + altezza);

    printf("\nRisultati:\n");
    printf("Area: %.2f\n", area);
    printf("Perimetro: %.2f\n", perimetro);

    return 0;
}
\end{lstlisting}

\subsection{Esempio 2: Conversione Temperature}

\begin{lstlisting}[caption={Conversione da Celsius a Fahrenheit}]
#include <stdio.h>

int main(int argc, char** argv) {
    float celsius, fahrenheit;

    printf("Inserisci la temperatura in Celsius: ");
    scanf("%f", &celsius);

    fahrenheit = (celsius * 9.0 / 5.0) + 32.0;

    printf("%.2f gradi Celsius = %.2f gradi Fahrenheit\n",
           celsius, fahrenheit);

    return 0;
}
\end{lstlisting}

\subsection{Esempio 3: Calcolo Media}

\begin{lstlisting}[caption={Calcolo media di tre numeri}]
#include <stdio.h>

int main(int argc, char** argv) {
    float num1, num2, num3;
    float somma, media;

    printf("Inserisci tre numeri:\n");

    printf("Numero 1: ");
    scanf("%f", &num1);

    printf("Numero 2: ");
    scanf("%f", &num2);

    printf("Numero 3: ");
    scanf("%f", &num3);

    somma = num1 + num2 + num3;
    media = somma / 3.0;

    printf("\nSomma: %.2f\n", somma);
    printf("Media: %.2f\n", media);

    return 0;
}
\end{lstlisting}

\section{Overflow e Underflow}

\subsection{Overflow}

Si verifica quando un valore supera il massimo rappresentabile:

\begin{lstlisting}[caption={Esempio di overflow}]
#include <stdio.h>
#include <limits.h>

int main(int argc, char** argv) {
    int massimo = INT_MAX;  // Valore massimo per int
    printf("Valore massimo: %d\n", massimo);

    int overflow = massimo + 1;
    printf("Dopo overflow: %d\n", overflow);  // Numero negativo!

    return 0;
}
\end{lstlisting}

\subsection{Underflow}

Si verifica quando un numero è troppo piccolo per essere rappresentato:

\begin{lstlisting}[caption={Esempio di underflow}]
#include <stdio.h>

int main(int argc, char** argv) {
    float piccolo = 0.0000001;
    float piccolissimo = piccolo / 1000000000;

    printf("Numero piccolo: %f\n", piccolo);
    printf("Numero piccolissimo: %f\n", piccolissimo);  // Potrebbe essere 0

    return 0;
}
\end{lstlisting}

\section{Libreria limits.h}

La libreria \texttt{limits.h} definisce le costanti per i valori limite dei tipi:

\begin{lstlisting}[caption={Uso di limits.h}]
#include <stdio.h>
#include <limits.h>

int main(int argc, char** argv) {
    printf("Range di char: %d a %d\n", CHAR_MIN, CHAR_MAX);
    printf("Range di short: %d a %d\n", SHRT_MIN, SHRT_MAX);
    printf("Range di int: %d a %d\n", INT_MIN, INT_MAX);
    printf("Range di long: %ld a %ld\n", LONG_MIN, LONG_MAX);

    return 0;
}
\end{lstlisting}

\section{Esercizi}

\subsection{Livello Base}

\begin{enumerate}
    \item Scrivi un programma che chiede nome ed età all'utente e li stampa.
    \item Crea un programma che legge due numeri interi e stampa la loro somma, differenza, prodotto e quoziente.
    \item Scrivi un programma che converte un valore in euro in dollari (tasso di cambio fisso: 1 euro = 1.10 dollari).
    \item Crea un programma che calcola l'area di un cerchio dato il raggio (usa \texttt{\#define} per PI).
\end{enumerate}

\subsection{Livello Intermedio}

\begin{enumerate}
    \item Scrivi un programma che calcola l'IMC (Indice di Massa Corporea): IMC = peso / (altezza * altezza).
    \item Crea un programma che converte un numero di secondi in ore, minuti e secondi.
    \item Scrivi un programma che calcola il perimetro e l'area di un triangolo rettangolo dati i due cateti (usa il teorema di Pitagora per l'ipotenusa).
    \item Crea un programma che simula una cassa: l'utente inserisce il prezzo, la quantità e viene calcolato il totale con IVA al 22\%.
\end{enumerate}

\subsection{Livello Avanzato}

\begin{enumerate}
    \item Scrivi un programma che legge tre coefficienti (a, b, c) di un'equazione di secondo grado e calcola il discriminante (delta = b² - 4ac).
    \item Crea un programma che converte un numero decimale in binario (solo per numeri piccoli, es. 0-15).
    \item Scrivi un programma che calcola l'interesse composto: M = C(1 + r)ⁿ dove C è il capitale, r il tasso di interesse, n gli anni.
\end{enumerate}

\section{Riepilogo}

In questo capitolo abbiamo imparato:

\begin{itemize}
    \item Il concetto di variabile e tipo di dato
    \item I tipi base del C: interi, float, char
    \item Come dichiarare e inizializzare variabili
    \item Le regole per i nomi delle variabili
    \item Gli specificatori di formato per printf
    \item Come leggere input con scanf
    \item La differenza tra costanti (\#define e const)
    \item Le conversioni di tipo implicite ed esplicite
    \item I problemi di overflow e underflow
\end{itemize}

\section{Approfondimenti}

\begin{itemize}
    \item \textbf{Documentazione}: C Data Types Reference
    \item \textbf{Esercizi online}: HackerRank C Practice
    \item \textbf{Video}: "C Variables and Data Types" su YouTube
\end{itemize}

Nel prossimo capitolo vedremo gli operatori ed espressioni, che ci permetteranno di effettuare calcoli e manipolazioni più complesse!
