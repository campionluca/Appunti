\chapter{Stringhe}
\label{cap:stringhe}

\section{Introduzione alle Stringhe}

In C, le stringhe sono array di caratteri terminati da un carattere speciale: \texttt{'\textbackslash 0'} (carattere nullo o null terminator).

\subsection{Obiettivi di Apprendimento}

Alla fine di questo capitolo sarai in grado di:

\begin{itemize}
    \item Comprendere come le stringhe sono rappresentate in C come array di caratteri
    \item Utilizzare il terminatore null \texttt{'\textbackslash 0'} correttamente
    \item Dichiarare e inizializzare stringhe in vari modi
    \item Leggere e stampare stringhe con diverse tecniche di I/O
    \item Utilizzare le principali funzioni della libreria \texttt{string.h}
    \item Manipolare stringhe (conversione, inversione, concatenazione)
    \item Evitare buffer overflow e altri errori comuni con le stringhe
    \item Creare array di stringhe e gestire strutture dati complesse
    \item Convertire tra stringhe e numeri in entrambe le direzioni
\end{itemize}

\begin{nota}
A differenza di altri linguaggi, C non ha un tipo di dato "stringa" nativo. Le stringhe sono gestite come array di \texttt{char}.
\end{nota}

\subsection{Caratteristiche delle Stringhe}

Le stringhe in C hanno caratteristiche distintive che ne determinano il comportamento e l'uso. Una stringa è essenzialmente un \textbf{array di caratteri con terminatore null}: ogni stringa termina con il carattere speciale \texttt{'\textbackslash 0'}, anche conosciuto come terminatore nullo. Questo terminatore \texttt{'\textbackslash 0'} \textbf{marca la fine della stringa}, permettendo alle funzioni di sapere dove termina il contenuto effettivo della stringa. Questa caratteristica ha un'implicazione importante: \textbf{la dimensione dell'array deve includere spazio per il terminatore}. Se desideri una stringa di 5 caratteri (ad esempio "Hello"), devi allocare un array di almeno 6 elementi per far spazio al terminatore. Le stringhe \textbf{possono essere manipolate come array o tramite puntatori}, il che offre flessibilità nel modo in cui accedi e modifichi il contenuto della stringa.

\section{Dichiarazione e Inizializzazione}

\subsection{Modi di Dichiarare Stringhe}

\begin{lstlisting}
#include <stdio.h>

int main() {
    // Metodo 1: Array di caratteri con inizializzazione
    char str1[6] = {'H', 'e', 'l', 'l', 'o', '\0'};

    // Metodo 2: Inizializzazione con letterale stringa
    char str2[] = "Hello";  // Dimensione automatica (6 caratteri)

    // Metodo 3: Specifica dimensione esplicita
    char str3[20] = "Hello";  // str3 ha 20 caratteri, usa solo 6

    // Metodo 4: Puntatore a stringa letterale (costante)
    char *str4 = "Hello";  // Non modificabile!

    printf("%s\n", str1);
    printf("%s\n", str2);
    printf("%s\n", str3);
    printf("%s\n", str4);

    return 0;
}
\end{lstlisting}

\begin{attenzione}
Le stringhe letterali (come \texttt{"Hello"}) sono memorizzate in una zona di memoria di sola lettura. Non tentare di modificarle!
\end{attenzione}

\subsection{Differenza tra Array e Puntatore}

\begin{lstlisting}
#include <stdio.h>

int main() {
    // Array di caratteri (modificabile)
    char str1[] = "Hello";
    str1[0] = 'J';  // OK: modifica in "Jello"

    // Puntatore a stringa letterale (non modificabile)
    char *str2 = "Hello";
    // str2[0] = 'J';  // ERRORE: crash a runtime!

    printf("%s\n", str1);
    printf("%s\n", str2);

    return 0;
}
\end{lstlisting}

\section{Input e Output di Stringhe}

\subsection{Funzioni di I/O}

\begin{lstlisting}
#include <stdio.h>

int main() {
    char nome[50];

    // Metodo 1: scanf (legge fino allo spazio)
    printf("Inserisci nome: ");
    scanf("%s", nome);  // Nota: nome, non &nome!
    printf("Nome: %s\n", nome);

    // Pulisci il buffer
    while (getchar() != '\n');

    // Metodo 2: gets (deprecato, non usare!)
    // gets(nome);  // PERICOLOSO: non controlla i limiti!

    // Metodo 3: fgets (sicuro, legge anche spazi)
    printf("Inserisci nome completo: ");
    fgets(nome, sizeof(nome), stdin);
    printf("Nome completo: %s", nome);

    return 0;
}
\end{lstlisting}

\begin{attenzione}
\texttt{scanf("\%s", str)} si ferma al primo spazio. Usa \texttt{fgets()} per leggere stringhe con spazi.
\end{attenzione}

\subsection{Carattere per Carattere}

\begin{lstlisting}
#include <stdio.h>

int main() {
    char str[100];
    char ch;
    int i = 0;

    printf("Inserisci una stringa (max 99 caratteri):\n");

    // Lettura carattere per carattere
    while ((ch = getchar()) != '\n' && i < 99) {
        str[i++] = ch;
    }
    str[i] = '\0';  // Aggiungi terminatore

    printf("Hai inserito: %s\n", str);

    return 0;
}
\end{lstlisting}

\section{Funzioni Standard per Stringhe}

La libreria \texttt{string.h} fornisce molte funzioni utili.

\subsection{strlen - Lunghezza della Stringa}

\begin{lstlisting}
#include <stdio.h>
#include <string.h>

int main() {
    char str[] = "Hello World";

    int lunghezza = strlen(str);
    printf("Lunghezza: %d\n", lunghezza);  // 11

    // Implementazione manuale
    int len = 0;
    while (str[len] != '\0') {
        len++;
    }
    printf("Lunghezza (manuale): %d\n", len);

    return 0;
}
\end{lstlisting}

\subsection{strcpy - Copia di Stringhe}

\begin{lstlisting}
#include <stdio.h>
#include <string.h>

int main() {
    char sorgente[] = "Hello";
    char destinazione[20];

    // Copia sorgente in destinazione
    strcpy(destinazione, sorgente);
    printf("Destinazione: %s\n", destinazione);

    // Versione sicura con limite
    strncpy(destinazione, "World", sizeof(destinazione) - 1);
    destinazione[sizeof(destinazione) - 1] = '\0';
    printf("Destinazione: %s\n", destinazione);

    return 0;
}
\end{lstlisting}

\subsection{strcat - Concatenazione}

\begin{lstlisting}
#include <stdio.h>
#include <string.h>

int main() {
    char str1[50] = "Hello ";
    char str2[] = "World";

    // Concatena str2 a str1
    strcat(str1, str2);
    printf("Risultato: %s\n", str1);  // "Hello World"

    // Versione sicura con limite
    char str3[20] = "C ";
    strncat(str3, "Programming", sizeof(str3) - strlen(str3) - 1);
    printf("Risultato: %s\n", str3);

    return 0;
}
\end{lstlisting}

\subsection{strcmp - Confronto di Stringhe}

\begin{lstlisting}
#include <stdio.h>
#include <string.h>

int main() {
    char str1[] = "apple";
    char str2[] = "banana";
    char str3[] = "apple";

    // strcmp restituisce:
    // < 0 se str1 < str2
    // 0 se str1 == str2
    // > 0 se str1 > str2

    if (strcmp(str1, str2) < 0) {
        printf("%s viene prima di %s\n", str1, str2);
    }

    if (strcmp(str1, str3) == 0) {
        printf("%s e' uguale a %s\n", str1, str3);
    }

    // Confronto case-insensitive (non standard ovunque)
    #ifdef _WIN32
    if (stricmp(str1, "APPLE") == 0) {
        printf("Uguali ignorando maiuscole/minuscole\n");
    }
    #else
    if (strcasecmp(str1, "APPLE") == 0) {
        printf("Uguali ignorando maiuscole/minuscole\n");
    }
    #endif

    return 0;
}
\end{lstlisting}

\subsection{Altre Funzioni Utili}

\begin{lstlisting}
#include <stdio.h>
#include <string.h>
#include <ctype.h>

int main() {
    char str[] = "Hello World 123";

    // strchr: trova un carattere
    char *pos = strchr(str, 'o');
    if (pos != NULL) {
        printf("Prima 'o' trovata alla posizione %ld\n",
               pos - str);
    }

    // strstr: trova una sottostringa
    pos = strstr(str, "World");
    if (pos != NULL) {
        printf("'World' trovato: %s\n", pos);
    }

    // strrchr: trova l'ultima occorrenza
    pos = strrchr(str, 'o');
    if (pos != NULL) {
        printf("Ultima 'o' alla posizione %ld\n", pos - str);
    }

    return 0;
}
\end{lstlisting}

\section{Manipolazione di Stringhe}

\subsection{Conversione Maiuscolo/Minuscolo}

\begin{lstlisting}
#include <stdio.h>
#include <ctype.h>

void maiuscolo(char *str) {
    for (int i = 0; str[i] != '\0'; i++) {
        str[i] = toupper(str[i]);
    }
}

void minuscolo(char *str) {
    for (int i = 0; str[i] != '\0'; i++) {
        str[i] = tolower(str[i]);
    }
}

int main() {
    char str1[] = "Hello World";
    char str2[] = "HELLO WORLD";

    maiuscolo(str1);
    printf("Maiuscolo: %s\n", str1);

    minuscolo(str2);
    printf("Minuscolo: %s\n", str2);

    return 0;
}
\end{lstlisting}

\subsection{Inversione di una Stringa}

\begin{lstlisting}
#include <stdio.h>
#include <string.h>

void inverti(char *str) {
    int n = strlen(str);
    for (int i = 0; i < n / 2; i++) {
        char temp = str[i];
        str[i] = str[n - 1 - i];
        str[n - 1 - i] = temp;
    }
}

int main() {
    char str[] = "Hello";

    printf("Originale: %s\n", str);
    inverti(str);
    printf("Invertita: %s\n", str);

    return 0;
}
\end{lstlisting}

\subsection{Rimozione degli Spazi}

\begin{lstlisting}
#include <stdio.h>
#include <ctype.h>

void rimuovi_spazi(char *str) {
    int i = 0, j = 0;

    while (str[i] != '\0') {
        if (!isspace(str[i])) {
            str[j++] = str[i];
        }
        i++;
    }
    str[j] = '\0';
}

int main() {
    char str[] = "H e l l o   W o r l d";

    printf("Originale: [%s]\n", str);
    rimuovi_spazi(str);
    printf("Senza spazi: [%s]\n", str);

    return 0;
}
\end{lstlisting}

\subsection{Conta Parole}

\begin{lstlisting}
#include <stdio.h>
#include <ctype.h>

int conta_parole(char *str) {
    int conta = 0;
    int in_parola = 0;

    for (int i = 0; str[i] != '\0'; i++) {
        if (isspace(str[i])) {
            in_parola = 0;
        } else if (!in_parola) {
            in_parola = 1;
            conta++;
        }
    }

    return conta;
}

int main() {
    char str[] = "Questa e' una frase di esempio";

    int parole = conta_parole(str);
    printf("La stringa contiene %d parole\n", parole);

    return 0;
}
\end{lstlisting}

\section{Array di Stringhe}

\subsection{Array Bidimensionale di char}

\begin{lstlisting}
#include <stdio.h>

int main() {
    // Array di 5 stringhe, ognuna max 20 caratteri
    char nomi[5][20] = {
        "Mario",
        "Lucia",
        "Giovanni",
        "Anna",
        "Francesco"
    };

    printf("Lista nomi:\n");
    for (int i = 0; i < 5; i++) {
        printf("%d. %s\n", i + 1, nomi[i]);
    }

    return 0;
}
\end{lstlisting}

\subsection{Array di Puntatori a Stringhe}

\begin{lstlisting}
#include <stdio.h>

int main() {
    // Array di puntatori a stringhe
    char *giorni[] = {
        "Lunedi",
        "Martedi",
        "Mercoledi",
        "Giovedi",
        "Venerdi",
        "Sabato",
        "Domenica"
    };

    printf("Giorni della settimana:\n");
    for (int i = 0; i < 7; i++) {
        printf("%s\n", giorni[i]);
    }

    return 0;
}
\end{lstlisting}

\section{Conversioni tra Stringhe e Numeri}

\subsection{Da Stringa a Numero}

\begin{lstlisting}
#include <stdio.h>
#include <stdlib.h>

int main() {
    char str_int[] = "12345";
    char str_float[] = "3.14159";
    char str_long[] = "9876543210";

    // Conversione a int
    int num = atoi(str_int);
    printf("Intero: %d\n", num);

    // Conversione a float
    float fnum = atof(str_float);
    printf("Float: %.2f\n", fnum);

    // Conversione a long
    long lnum = atol(str_long);
    printf("Long: %ld\n", lnum);

    // Conversione con strtol (piu' robusta)
    char *endptr;
    long val = strtol("  123abc", &endptr, 10);
    printf("Valore: %ld, Resto: %s\n", val, endptr);

    return 0;
}
\end{lstlisting}

\subsection{Da Numero a Stringa}

\begin{lstlisting}
#include <stdio.h>
#include <stdlib.h>

int main() {
    int num = 12345;
    float fnum = 3.14159;

    char buffer[50];

    // sprintf: stampa su stringa
    sprintf(buffer, "%d", num);
    printf("Intero come stringa: %s\n", buffer);

    sprintf(buffer, "%.2f", fnum);
    printf("Float come stringa: %s\n", buffer);

    // Formato complesso
    sprintf(buffer, "x=%d, y=%.1f", num, fnum);
    printf("Formato: %s\n", buffer);

    return 0;
}
\end{lstlisting}

\section{Tokenizzazione di Stringhe}

\subsection{Uso di strtok}

\begin{lstlisting}
#include <stdio.h>
#include <string.h>

int main() {
    char str[] = "apple,banana,cherry,date";
    char *token;

    printf("Parole separate da virgola:\n");

    // Primo token
    token = strtok(str, ",");

    // Token successivi
    while (token != NULL) {
        printf("- %s\n", token);
        token = strtok(NULL, ",");
    }

    // Esempio con spazi multipli
    char frase[] = "Questa  e'   una   frase";
    printf("\nParole nella frase:\n");

    token = strtok(frase, " ");
    while (token != NULL) {
        printf("- %s\n", token);
        token = strtok(NULL, " ");
    }

    return 0;
}
\end{lstlisting}

\begin{attenzione}
\texttt{strtok()} modifica la stringa originale inserendo caratteri null. Se hai bisogno della stringa originale, fai prima una copia.
\end{attenzione}

\section{Validazione di Stringhe}

\subsection{Verifica se è un Numero}

\begin{lstlisting}
#include <stdio.h>
#include <ctype.h>

int is_numero(char *str) {
    if (*str == '-' || *str == '+') {
        str++;  // Salta il segno
    }

    if (*str == '\0') {
        return 0;  // Stringa vuota o solo segno
    }

    while (*str != '\0') {
        if (!isdigit(*str)) {
            return 0;
        }
        str++;
    }

    return 1;
}

int main() {
    char str1[] = "12345";
    char str2[] = "-678";
    char str3[] = "123abc";
    char str4[] = "";

    printf("%s: %s\n", str1, is_numero(str1) ? "Numero" : "Non numero");
    printf("%s: %s\n", str2, is_numero(str2) ? "Numero" : "Non numero");
    printf("%s: %s\n", str3, is_numero(str3) ? "Numero" : "Non numero");
    printf("%s: %s\n", str4, is_numero(str4) ? "Numero" : "Non numero");

    return 0;
}
\end{lstlisting}

\subsection{Verifica Palindromo}

\begin{lstlisting}
#include <stdio.h>
#include <string.h>
#include <ctype.h>

int is_palindromo(char *str) {
    int sinistra = 0;
    int destra = strlen(str) - 1;

    while (sinistra < destra) {
        // Salta caratteri non alfanumerici
        while (sinistra < destra && !isalnum(str[sinistra])) {
            sinistra++;
        }
        while (sinistra < destra && !isalnum(str[destra])) {
            destra--;
        }

        // Confronta ignorando maiuscole/minuscole
        if (tolower(str[sinistra]) != tolower(str[destra])) {
            return 0;
        }

        sinistra++;
        destra--;
    }

    return 1;
}

int main() {
    char str1[] = "radar";
    char str2[] = "hello";
    char str3[] = "A man, a plan, a canal: Panama";

    printf("%s: %s\n", str1,
           is_palindromo(str1) ? "Palindromo" : "Non palindromo");
    printf("%s: %s\n", str2,
           is_palindromo(str2) ? "Palindromo" : "Non palindromo");
    printf("%s: %s\n", str3,
           is_palindromo(str3) ? "Palindromo" : "Non palindromo");

    return 0;
}
\end{lstlisting}

\section{Esempi Pratici}

\subsection{Sistema di Login}

\begin{lstlisting}
#include <stdio.h>
#include <string.h>

#define MAX_USERNAME 20
#define MAX_PASSWORD 20

int verifica_login(char *username, char *password) {
    // Username e password corretti (esempio)
    char user_corretto[] = "admin";
    char pass_corretta[] = "password123";

    return (strcmp(username, user_corretto) == 0 &&
            strcmp(password, pass_corretta) == 0);
}

int main() {
    char username[MAX_USERNAME];
    char password[MAX_PASSWORD];

    printf("=== SISTEMA DI LOGIN ===\n");
    printf("Username: ");
    scanf("%s", username);

    printf("Password: ");
    scanf("%s", password);

    if (verifica_login(username, password)) {
        printf("\nAccesso consentito!\n");
    } else {
        printf("\nCredenziali non valide!\n");
    }

    return 0;
}
\end{lstlisting}

\subsection{Analisi di Testo}

\begin{lstlisting}
#include <stdio.h>
#include <string.h>
#include <ctype.h>

void analizza_testo(char *testo) {
    int lettere = 0, cifre = 0, spazi = 0, altri = 0;

    for (int i = 0; testo[i] != '\0'; i++) {
        if (isalpha(testo[i])) {
            lettere++;
        } else if (isdigit(testo[i])) {
            cifre++;
        } else if (isspace(testo[i])) {
            spazi++;
        } else {
            altri++;
        }
    }

    printf("=== ANALISI TESTO ===\n");
    printf("Lunghezza: %lu\n", strlen(testo));
    printf("Lettere: %d\n", lettere);
    printf("Cifre: %d\n", cifre);
    printf("Spazi: %d\n", spazi);
    printf("Altri caratteri: %d\n", altri);
}

int main() {
    char testo[200];

    printf("Inserisci un testo:\n");
    fgets(testo, sizeof(testo), stdin);

    // Rimuovi newline finale se presente
    testo[strcspn(testo, "\n")] = '\0';

    analizza_testo(testo);

    return 0;
}
\end{lstlisting}

\section{Esercizi}

\subsection{Livello Base}

\begin{enumerate}
    \item Scrivi una funzione che conta quante vocali ci sono in una stringa.
    \item Crea una funzione che verifica se una stringa contiene solo caratteri alfabetici.
    \item Implementa una funzione che rimuove tutti i caratteri non alfanumerici da una stringa.
    \item Scrivi un programma che legge una frase e stampa ogni parola su una riga separata.
\end{enumerate}

\subsection{Livello Intermedio}

\begin{enumerate}
    \item Implementa una funzione che sostituisce tutte le occorrenze di una sottostringa con un'altra.
    \item Crea una funzione che verifica se due stringhe sono anagrammi.
    \item Scrivi un programma che ordina un array di stringhe in ordine alfabetico.
    \item Implementa una funzione che comprime una stringa (es: "aaabbc" → "a3b2c1").
\end{enumerate}

\subsection{Livello Avanzato}

\begin{enumerate}
    \item Implementa un parser di espressioni matematiche semplici (es: "3 + 5 * 2").
    \item Crea un programma che valida indirizzi email usando analisi di stringhe.
    \item Scrivi una funzione che implementa il pattern matching semplice (con wildcard * e ?).
    \item Implementa un sistema di cifratura/decifratura Caesar per stringhe.
\end{enumerate}
