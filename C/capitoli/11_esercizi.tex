\chapter{Esercizi Completi}
\label{cap:esercizi}

Questo capitolo raccoglie tutti gli esercizi presenti nei capitoli precedenti, organizzati per argomento e livello di difficoltà.

\section{Variabili e Tipi di Dati}

\subsection{Livello Base}

\begin{enumerate}
    \item Scrivi un programma che dichiara variabili di tutti i tipi base e stampa i loro valori.
    \item Crea un programma che calcola l'area di un cerchio dato il raggio.
    \item Implementa un programma che converte euro in dollari (tasso fisso 1.2).
    \item Scrivi un programma che calcola la media di tre numeri float inseriti dall'utente.
\end{enumerate}

\subsection{Livello Intermedio}

\begin{enumerate}
    \item Crea un programma che converte un numero di secondi in ore, minuti e secondi.
    \item Implementa un convertitore tra diverse unità di temperatura (Celsius, Fahrenheit, Kelvin).
    \item Scrivi un programma che calcola l'IMC (Indice di Massa Corporea) e fornisce una valutazione.
    \item Crea un programma che calcola interessi composti dato capitale, tasso e periodo.
\end{enumerate}

\subsection{Livello Avanzato}

\begin{enumerate}
    \item Implementa un programma che risolve equazioni di secondo grado complete.
    \item Crea un convertitore universale di unità di misura (lunghezza, peso, volume, temperatura).
    \item Scrivi un programma che calcola la distanza tra due coordinate GPS.
    \item Implementa un sistema di calcolo del codice fiscale dato nome, cognome, data e luogo di nascita.
\end{enumerate}

\section{Operatori ed Espressioni}

\subsection{Livello Base}

\begin{enumerate}
    \item Scrivi un programma che calcola l'area e il perimetro di un rettangolo dati base e altezza.
    \item Crea un programma che converte una temperatura da Celsius a Fahrenheit.
    \item Scrivi un programma che verifica se un numero è pari o dispari.
    \item Implementa un programma che calcola la media di tre numeri float.
\end{enumerate}

\subsection{Livello Intermedio}

\begin{enumerate}
    \item Scrivi un programma che dato un numero di secondi, calcola quante ore, minuti e secondi rappresenta.
    \item Crea un programma che scambia i valori di due variabili senza usare una variabile temporanea.
    \item Implementa un programma che verifica se un anno è bisestile.
    \item Scrivi un programma che estrae le singole cifre di un numero intero a tre cifre.
\end{enumerate}

\subsection{Livello Avanzato}

\begin{enumerate}
    \item Crea un programma che implementa una calcolatrice semplice con le quattro operazioni base.
    \item Scrivi un programma che usa operatori bit a bit per verificare se un numero è una potenza di 2.
    \item Implementa l'algoritmo di Euclide per il calcolo del MCD.
    \item Crea un programma che conta i bit impostati a 1 nella rappresentazione binaria di un numero.
\end{enumerate}

\section{Controllo di Flusso}

\subsection{Livello Base}

\begin{enumerate}
    \item Scrivi un programma che determina se un numero è positivo, negativo o zero.
    \item Crea un programma che stampa i numeri da 1 a 100.
    \item Scrivi un programma che calcola la somma dei primi N numeri naturali.
    \item Implementa la tavola pitagorica di un numero dato.
\end{enumerate}

\subsection{Livello Intermedio}

\begin{enumerate}
    \item Scrivi un programma che verifica se un numero è primo.
    \item Crea un programma che inverte un numero (es: 1234 diventa 4321).
    \item Implementa l'algoritmo di Euclide per il MCD.
    \item Disegna un triangolo di numeri crescenti.
\end{enumerate}

\subsection{Livello Avanzato}

\begin{enumerate}
    \item Genera la sequenza di Fibonacci fino a N termini.
    \item Verifica se un numero è palindromo.
    \item Trova tutti i numeri perfetti minori di 1000.
    \item Disegna il triangolo di Floyd.
\end{enumerate}

\section{Funzioni}

\subsection{Livello Base}

\begin{enumerate}
    \item Scrivi una funzione che converte temperatura da Celsius a Fahrenheit.
    \item Crea una funzione che verifica se un numero è pari.
    \item Implementa una funzione che calcola l'area di un cerchio dato il raggio.
    \item Scrivi una funzione che stampa un triangolo di asterischi di altezza N.
\end{enumerate}

\subsection{Livello Intermedio}

\begin{enumerate}
    \item Crea una funzione che inverte un numero intero.
    \item Scrivi una funzione ricorsiva che calcola la somma delle cifre di un numero.
    \item Implementa una funzione che verifica se una stringa è palindroma.
    \item Crea funzioni per conversione tra unità di misura.
\end{enumerate}

\subsection{Livello Avanzato}

\begin{enumerate}
    \item Implementa le Torre di Hanoi in modo ricorsivo.
    \item Genera numeri primi con il Crivello di Eratostene.
    \item Calcola ricorsivamente il coefficiente binomiale.
    \item Risolvi il problema delle N regine con backtracking.
\end{enumerate}

\section{Array}

\subsection{Livello Base}

\begin{enumerate}
    \item Leggi 10 numeri e stampa quanti sono pari e quanti dispari.
    \item Trova la seconda cifra più grande in un array.
    \item Conta le occorrenze di un numero in un array.
    \item Verifica se un array è palindromo.
\end{enumerate}

\subsection{Livello Intermedio}

\begin{enumerate}
    \item Implementa la ricerca binaria su array ordinato.
    \item Ruota gli elementi di un array di K posizioni.
    \item Fondi due array ordinati in un unico array ordinato.
    \item Trova sottoinsiemi con somma uguale a un valore dato.
\end{enumerate}

\subsection{Livello Avanzato}

\begin{enumerate}
    \item Implementa Quick Sort ricorsivo.
    \item Moltiplica due matrici.
    \item Trova la sotto-sequenza contigua con somma massima (Kadane).
    \item Risolvi il problema dello zaino 0/1 con programmazione dinamica.
\end{enumerate}

\section{Puntatori}

\subsection{Livello Base}

\begin{enumerate}
    \item Usa puntatori per scambiare tre valori ciclicamente.
    \item Inverti un array usando puntatori.
    \item Trova max e min restituendoli tramite puntatori.
    \item Alloca dinamicamente un array con i quadrati dei numeri da 1 a N.
\end{enumerate}

\subsection{Livello Intermedio}

\begin{enumerate}
    \item Concatena due array allocati dinamicamente.
    \item Rimuovi duplicati da array dinamico.
    \item Implementa matrice dinamica (array 2D).
    \item Ordina usando solo aritmetica dei puntatori.
\end{enumerate}

\subsection{Livello Avanzato}

\begin{enumerate}
    \item Implementa lista collegata con inserimento, ricerca e cancellazione.
    \item Crea uno stack dinamico.
    \item Gestisci matrice sparsa con allocazione dinamica.
    \item Implementa albero binario di ricerca.
\end{enumerate}

\section{Stringhe}

\subsection{Livello Base}

\begin{enumerate}
    \item Conta le vocali in una stringa.
    \item Verifica se una stringa contiene solo caratteri alfabetici.
    \item Rimuovi caratteri non alfanumerici da stringa.
    \item Stampa ogni parola di una frase su una riga separata.
\end{enumerate}

\subsection{Livello Intermedio}

\begin{enumerate}
    \item Sostituisci tutte le occorrenze di una sottostringa.
    \item Verifica se due stringhe sono anagrammi.
    \item Ordina array di stringhe in ordine alfabetico.
    \item Comprimi stringa (es: "aaabbc" → "a3b2c1").
\end{enumerate}

\subsection{Livello Avanzato}

\begin{enumerate}
    \item Parser di espressioni matematiche semplici.
    \item Valida indirizzi email.
    \item Pattern matching con wildcard * e ?.
    \item Sistema di cifratura/decifratura Caesar.
\end{enumerate}

\section{Struct}

\subsection{Livello Base}

\begin{enumerate}
    \item Struct Rettangolo con funzioni per area e perimetro.
    \item Struct Tempo con conversione in secondi totali.
    \item Struct ContoCorrente con operazioni deposito/prelievo.
    \item Struct Prodotto per gestire inventario.
\end{enumerate}

\subsection{Livello Intermedio}

\begin{enumerate}
    \item Sistema rubrica telefonica con array di struct.
    \item Struct Matrice2D con allocazione dinamica.
    \item Database libri con ricerca e ordinamento.
    \item Struct per poligoni con calcolo perimetro e area.
\end{enumerate}

\subsection{Livello Avanzato}

\begin{enumerate}
    \item Lista collegata con struct e puntatori.
    \item Sistema gestione parcheggio.
    \item Implementa grafo con struct per nodi e archi.
    \item Sistema bancario con conti, transazioni e storici.
\end{enumerate}

\section{File}

\subsection{Livello Base}

\begin{enumerate}
    \item Crea file con 10 numeri casuali.
    \item Leggi file e stampa righe contenenti parola specifica.
    \item Conta occorrenze di un carattere in file.
    \item Unisci due file in un terzo file.
\end{enumerate}

\subsection{Livello Intermedio}

\begin{enumerate}
    \item Ordina righe di file in ordine alfabetico.
    \item Sistema di log con data, ora e messaggio.
    \item Cifra e decifra file con cifrario Caesar.
    \item Rubrica telefonica salvata su CSV.
\end{enumerate}

\subsection{Livello Avanzato}

\begin{enumerate}
    \item Editor di testo per modificare righe specifiche.
    \item Sistema inventario con file binario e ricerca.
    \item Comprimi file eliminando spazi multipli e righe vuote.
    \item Backup incrementale con salvataggio modifiche.
\end{enumerate}

\section{Progetti Completi}

\subsection{Progetto 1: Sistema di Gestione Biblioteca}

Crea un sistema completo per gestire una biblioteca con:
\begin{itemize}
    \item Struct per Libro (titolo, autore, ISBN, disponibilità)
    \item Struct per Utente (nome, tessera, libri presi in prestito)
    \item Funzioni per aggiungere/rimuovere libri
    \item Funzioni per prestito e restituzione
    \item Ricerca libri per autore/titolo
    \item Salvataggio su file
\end{itemize}

\subsection{Progetto 2: Gioco del Tris}

Implementa il gioco del tris con:
\begin{itemize}
    \item Matrice 3x3 per la griglia
    \item Funzione per stampare la griglia
    \item Funzione per verificare vittoria/pareggio
    \item Modalità 2 giocatori
    \item Validazione input
    \item Possibilità di giocare più partite
\end{itemize}

\subsection{Progetto 3: Gestionale Studenti}

Sistema completo per gestire studenti con:
\begin{itemize}
    \item Struct Studente con dati personali e voti
    \item Array dinamico di studenti
    \item Menu interattivo
    \item Funzioni CRUD (Create, Read, Update, Delete)
    \item Calcolo medie e statistiche
    \item Ordinamento per voto/nome
    \item Salvataggio su file binario
\end{itemize}

\subsection{Progetto 4: Calcolatrice Avanzata}

Calcolatrice con funzionalità avanzate:
\begin{itemize}
    \item Operazioni base (+, -, *, /)
    \item Potenze e radici
    \item Funzioni trigonometriche
    \item Conversioni tra basi numeriche
    \item Storico operazioni
    \item Salvataggio storico su file
\end{itemize}

\subsection{Progetto 5: Sistema di Prenotazioni}

Sistema per gestire prenotazioni (es: ristorante, hotel):
\begin{itemize}
    \item Struct per Prenotazione
    \item Gestione disponibilità
    \item Ricerca per data/nome
    \item Cancellazione prenotazioni
    \item Report statistici
    \item Salvataggio persistente
\end{itemize}

\section{Esercizi di Riepilogo}

\subsection{Esercizio Complessivo 1: Gestione Magazzino}

Crea un programma completo che gestisce un magazzino con:
\begin{itemize}
    \item Prodotti (codice, nome, quantità, prezzo)
    \item Carico e scarico merci
    \item Inventario con valore totale
    \item Ricerca prodotti sotto scorta
    \item Report su file
\end{itemize}

\subsection{Esercizio Complessivo 2: Agenda Personale}

Implementa un'agenda con:
\begin{itemize}
    \item Eventi con data, ora, descrizione
    \item Visualizzazione giornaliera/settimanale
    \item Ricerca eventi
    \item Promemoria
    \item Esportazione in formato testo
\end{itemize}

\subsection{Esercizio Complessivo 3: Analizzatore di Testo}

Programma che analizza file di testo fornendo:
\begin{itemize}
    \item Numero righe, parole, caratteri
    \item Parole più frequenti
    \item Lunghezza media parole
    \item Indice di leggibilità
    \item Report completo su file
\end{itemize}

\section{Consigli per gli Esercizi}

\subsection{Come Affrontare gli Esercizi}

\begin{enumerate}
    \item \textbf{Leggi attentamente}: comprendi cosa ti viene chiesto
    \item \textbf{Scomponi il problema}: dividi in sotto-problemi più semplici
    \item \textbf{Pianifica}: pensa alla struttura prima di scrivere codice
    \item \textbf{Inizia dal semplice}: parti da una versione base e migliora
    \item \textbf{Testa frequentemente}: verifica il codice passo dopo passo
    \item \textbf{Gestisci gli errori}: prevedi input non validi
    \item \textbf{Commenta il codice}: spiega la logica complessa
    \item \textbf{Refactoring}: migliora il codice dopo che funziona
\end{enumerate}

\subsection{Debugging}

Quando il programma non funziona:

\begin{enumerate}
    \item Usa \texttt{printf()} per stampare valori intermedi
    \item Verifica che gli indici degli array siano corretti
    \item Controlla i puntatori (NULL, inizializzazione)
    \item Verifica apertura/chiusura file
    \item Controlla le condizioni nei cicli e negli if
    \item Usa il debugger (gdb su Linux/Mac, Visual Studio su Windows)
\end{enumerate}

\subsection{Best Practices}

\begin{itemize}
    \item Usa nomi di variabili descrittivi
    \item Mantieni funzioni piccole e focalizzate
    \item Evita duplicazione di codice
    \item Gestisci sempre gli errori
    \item Libera sempre la memoria allocata
    \item Chiudi sempre i file aperti
    \item Commenta il codice complesso
    \item Usa costanti invece di "numeri magici"
    \item Testa casi limite (0, negativi, stringhe vuote)
\end{itemize}

\section{Risorse Aggiuntive}

\subsection{Compilazione}

Comandi base per compilare programmi C:

\begin{verbatim}
# Compilazione semplice
gcc programma.c -o programma

# Con warning
gcc -Wall -Wextra programma.c -o programma

# Con debugging
gcc -g programma.c -o programma

# Con ottimizzazione
gcc -O2 programma.c -o programma

# Con libreria matematica
gcc programma.c -o programma -lm
\end{verbatim}

\subsection{Online Judge per Esercitarsi}

\begin{itemize}
    \item HackerRank
    \item LeetCode
    \item Codeforces
    \item Project Euler
    \item Codewars
\end{itemize}

\subsection{Documenta zione di Riferimento}

\begin{itemize}
    \item cppreference.com
    \item The C Programming Language (K\&R)
    \item man pages (Linux/Mac)
    \item MSDN (Windows)
\end{itemize}
