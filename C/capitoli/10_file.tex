\chapter{Gestione dei File}
\label{cap:file}

\section{Introduzione ai File}

La gestione dei file permette di salvare dati in modo permanente su disco e di leggerli in seguito. In C, i file vengono gestiti attraverso puntatori a \texttt{FILE}.

\subsection{Tipi di File}

I file possono essere classificati in due categorie principali in base al loro contenuto. I \textbf{file di testo} contengono caratteri leggibili in formato ASCII o Unicode, come file con estensione .txt, .csv, .json, .html, e altri formati testuali. Questi file possono essere aperti e visualizzati in qualsiasi editor di testo e sono facilmente interpretabili da umani. I \textbf{file binari}, al contrario, contengono dati in formato binario direttamente leggibile dalla macchina, come immagini (.jpg, .png), file eseguibili (.exe, .elf), archivi compressi (.zip), database, e altri formati che non sono direttamente leggibili in forma testuale. La scelta tra file di testo e binari dipende dalla natura dei dati che si desidera memorizzare e dalla necessità di leggibilità umana.

\begin{nota}
Per gestire i file in C è necessario includere la libreria \texttt{stdio.h}.
\end{nota}

\section{Apertura e Chiusura di File}

\subsection{Funzione fopen}

\begin{lstlisting}
FILE *fopen(const char *nome_file, const char *modalita);
\end{lstlisting}

\subsection{Modalità di Apertura}

\begin{itemize}
    \item \texttt{"r"}: lettura (read) - il file deve esistere
    \item \texttt{"w"}: scrittura (write) - crea nuovo file o sovrascrive
    \item \texttt{"a"}: append - aggiunge in coda al file
    \item \texttt{"r+"}: lettura e scrittura - il file deve esistere
    \item \texttt{"w+"}: lettura e scrittura - crea nuovo file
    \item \texttt{"a+"}: lettura e append
    \item \texttt{"rb"}, \texttt{"wb"}, ecc.: modalità binaria
\end{itemize}

\subsection{Esempio Base}

\begin{lstlisting}
#include <stdio.h>
#include <stdlib.h>

int main(int argc, char** argv) {
    FILE *file;

    // Apertura file
    file = fopen("test.txt", "w");

    // Controlla se l'apertura e' riuscita
    if (file == NULL) {
        printf("Errore nell'apertura del file!\n");
        return 1;
    }

    printf("File aperto con successo!\n");

    // Chiusura file
    fclose(file);

    return 0;
}
\end{lstlisting}

\begin{attenzione}
Controlla sempre che \texttt{fopen()} non restituisca NULL prima di usare il file. Ricorda sempre di chiudere i file con \texttt{fclose()}.
\end{attenzione}

\section{Scrittura su File}

\subsection{fprintf}

Scrive dati formattati su file (simile a printf).

\begin{lstlisting}
#include <stdio.h>

int main(int argc, char** argv) {
    FILE *file = fopen("output.txt", "w");

    if (file == NULL) {
        printf("Errore apertura file!\n");
        return 1;
    }

    // Scrittura formattata
    fprintf(file, "Questa e' la prima riga\n");
    fprintf(file, "Numero: %d\n", 42);
    fprintf(file, "Float: %.2f\n", 3.14159);

    fclose(file);
    printf("Dati scritti su output.txt\n");

    return 0;
}
\end{lstlisting}

\subsection{fputs}

Scrive una stringa su file.

\begin{lstlisting}
#include <stdio.h>

int main(int argc, char** argv) {
    FILE *file = fopen("testo.txt", "w");

    if (file == NULL) {
        printf("Errore!\n");
        return 1;
    }

    fputs("Prima riga\n", file);
    fputs("Seconda riga\n", file);
    fputs("Terza riga\n", file);

    fclose(file);
    return 0;
}
\end{lstlisting}

\subsection{fputc}

Scrive un singolo carattere.

\begin{lstlisting}
#include <stdio.h>

int main(int argc, char** argv) {
    FILE *file = fopen("caratteri.txt", "w");

    if (file == NULL) {
        printf("Errore!\n");
        return 1;
    }

    // Scrive caratteri singoli
    fputc('H', file);
    fputc('e', file);
    fputc('l', file);
    fputc('l', file);
    fputc('o', file);
    fputc('\n', file);

    fclose(file);
    return 0;
}
\end{lstlisting}

\section{Lettura da File}

\subsection{fscanf}

Legge dati formattati da file.

\begin{lstlisting}
#include <stdio.h>

int main(int argc, char** argv) {
    FILE *file = fopen("dati.txt", "r");

    if (file == NULL) {
        printf("File non trovato!\n");
        return 1;
    }

    int numero;
    float decimale;
    char parola[50];

    // Legge dati formattati
    fscanf(file, "%d", &numero);
    fscanf(file, "%f", &decimale);
    fscanf(file, "%s", parola);

    printf("Numero: %d\n", numero);
    printf("Decimale: %.2f\n", decimale);
    printf("Parola: %s\n", parola);

    fclose(file);
    return 0;
}
\end{lstlisting}

\subsection{fgets}

Legge una riga da file.

\begin{lstlisting}
#include <stdio.h>

int main(int argc, char** argv) {
    FILE *file = fopen("testo.txt", "r");

    if (file == NULL) {
        printf("File non trovato!\n");
        return 1;
    }

    char linea[200];

    // Legge riga per riga
    while (fgets(linea, sizeof(linea), file) != NULL) {
        printf("%s", linea);
    }

    fclose(file);
    return 0;
}
\end{lstlisting}

\subsection{fgetc}

Legge un carattere alla volta.

\begin{lstlisting}
#include <stdio.h>

int main(int argc, char** argv) {
    FILE *file = fopen("testo.txt", "r");

    if (file == NULL) {
        printf("File non trovato!\n");
        return 1;
    }

    char c;

    // Legge carattere per carattere
    while ((c = fgetc(file)) != EOF) {
        putchar(c);
    }

    fclose(file);
    return 0;
}
\end{lstlisting}

\section{Controllo Fine File e Errori}

\subsection{feof e ferror}

\begin{lstlisting}
#include <stdio.h>

int main(int argc, char** argv) {
    FILE *file = fopen("dati.txt", "r");

    if (file == NULL) {
        perror("Errore apertura file");
        return 1;
    }

    int num;

    while (1) {
        int result = fscanf(file, "%d", &num);

        // Controlla fine file
        if (feof(file)) {
            printf("Fine del file raggiunta\n");
            break;
        }

        // Controlla errori
        if (ferror(file)) {
            printf("Errore di lettura!\n");
            break;
        }

        // Controlla risultato fscanf
        if (result != 1) {
            printf("Formato non valido\n");
            break;
        }

        printf("Numero letto: %d\n", num);
    }

    fclose(file);
    return 0;
}
\end{lstlisting}

\section{Posizionamento nel File}

\subsection{fseek}

Sposta il puntatore del file in una posizione specifica.

\begin{lstlisting}
#include <stdio.h>

int main(int argc, char** argv) {
    FILE *file = fopen("test.txt", "r");

    if (file == NULL) {
        printf("Errore!\n");
        return 1;
    }

    // Vai all'inizio del file
    fseek(file, 0, SEEK_SET);

    // Vai alla fine del file
    fseek(file, 0, SEEK_END);

    // Ottieni la dimensione del file
    long dimensione = ftell(file);
    printf("Dimensione del file: %ld byte\n", dimensione);

    // Torna all'inizio
    rewind(file);  // Equivalente a fseek(file, 0, SEEK_SET)

    fclose(file);
    return 0;
}
\end{lstlisting}

\subsection{ftell}

Restituisce la posizione corrente nel file.

\begin{lstlisting}
#include <stdio.h>

int main(int argc, char** argv) {
    FILE *file = fopen("test.txt", "r");

    if (file == NULL) {
        printf("Errore!\n");
        return 1;
    }

    printf("Posizione iniziale: %ld\n", ftell(file));

    fgetc(file);  // Legge un carattere
    printf("Dopo 1 carattere: %ld\n", ftell(file));

    fgetc(file);
    printf("Dopo 2 caratteri: %ld\n", ftell(file));

    fclose(file);
    return 0;
}
\end{lstlisting}

\section{File Binari}

\subsection{fwrite}

Scrive dati binari su file.

\begin{lstlisting}
#include <stdio.h>

typedef struct {
    int id;
    char nome[30];
    float voto;
} Studente;

int main(int argc, char** argv) {
    FILE *file = fopen("studenti.dat", "wb");

    if (file == NULL) {
        printf("Errore!\n");
        return 1;
    }

    Studente s1 = {1, "Mario Rossi", 27.5};
    Studente s2 = {2, "Lucia Bianchi", 29.0};

    // Scrivi struct binarie
    fwrite(&s1, sizeof(Studente), 1, file);
    fwrite(&s2, sizeof(Studente), 1, file);

    fclose(file);
    printf("Dati scritti in formato binario\n");

    return 0;
}
\end{lstlisting}

\subsection{fread}

Legge dati binari da file.

\begin{lstlisting}
#include <stdio.h>

typedef struct {
    int id;
    char nome[30];
    float voto;
} Studente;

int main(int argc, char** argv) {
    FILE *file = fopen("studenti.dat", "rb");

    if (file == NULL) {
        printf("File non trovato!\n");
        return 1;
    }

    Studente s;

    // Leggi struct dal file
    while (fread(&s, sizeof(Studente), 1, file) == 1) {
        printf("ID: %d\n", s.id);
        printf("Nome: %s\n", s.nome);
        printf("Voto: %.1f\n\n", s.voto);
    }

    fclose(file);
    return 0;
}
\end{lstlisting}

\section{Esempi Pratici}

\subsection{Copia di File}

\begin{lstlisting}
#include <stdio.h>

int main(int argc, char** argv) {
    FILE *sorgente, *destinazione;
    char ch;

    sorgente = fopen("originale.txt", "r");
    if (sorgente == NULL) {
        printf("Impossibile aprire il file sorgente!\n");
        return 1;
    }

    destinazione = fopen("copia.txt", "w");
    if (destinazione == NULL) {
        printf("Impossibile creare il file destinazione!\n");
        fclose(sorgente);
        return 1;
    }

    // Copia carattere per carattere
    while ((ch = fgetc(sorgente)) != EOF) {
        fputc(ch, destinazione);
    }

    printf("File copiato con successo!\n");

    fclose(sorgente);
    fclose(destinazione);

    return 0;
}
\end{lstlisting}

\subsection{Conta Righe, Parole e Caratteri}

\begin{lstlisting}
#include <stdio.h>
#include <ctype.h>

int main(int argc, char** argv) {
    FILE *file = fopen("testo.txt", "r");

    if (file == NULL) {
        printf("File non trovato!\n");
        return 1;
    }

    int righe = 0, parole = 0, caratteri = 0;
    int in_parola = 0;
    char c;

    while ((c = fgetc(file)) != EOF) {
        caratteri++;

        if (c == '\n') {
            righe++;
        }

        if (isspace(c)) {
            in_parola = 0;
        } else if (!in_parola) {
            in_parola = 1;
            parole++;
        }
    }

    printf("Statistiche del file:\n");
    printf("Righe: %d\n", righe);
    printf("Parole: %d\n", parole);
    printf("Caratteri: %d\n", caratteri);

    fclose(file);
    return 0;
}
\end{lstlisting}

\subsection{Database Studenti su File}

\begin{lstlisting}
#include <stdio.h>
#include <string.h>

typedef struct {
    int matricola;
    char nome[30];
    char cognome[30];
    float media;
} Studente;

void aggiungi_studente() {
    FILE *file = fopen("studenti.dat", "ab");
    if (file == NULL) {
        printf("Errore apertura file!\n");
        return;
    }

    Studente s;
    printf("Matricola: ");
    scanf("%d", &s.matricola);
    printf("Nome: ");
    scanf("%s", s.nome);
    printf("Cognome: ");
    scanf("%s", s.cognome);
    printf("Media: ");
    scanf("%f", &s.media);

    fwrite(&s, sizeof(Studente), 1, file);
    printf("Studente aggiunto!\n");

    fclose(file);
}

void visualizza_studenti() {
    FILE *file = fopen("studenti.dat", "rb");
    if (file == NULL) {
        printf("Nessun dato disponibile!\n");
        return;
    }

    Studente s;
    printf("\n=== ELENCO STUDENTI ===\n");

    while (fread(&s, sizeof(Studente), 1, file) == 1) {
        printf("Matricola: %d\n", s.matricola);
        printf("Nome: %s %s\n", s.nome, s.cognome);
        printf("Media: %.2f\n", s.media);
        printf("-------------------\n");
    }

    fclose(file);
}

void cerca_studente(int matricola) {
    FILE *file = fopen("studenti.dat", "rb");
    if (file == NULL) {
        printf("Nessun dato disponibile!\n");
        return;
    }

    Studente s;
    int trovato = 0;

    while (fread(&s, sizeof(Studente), 1, file) == 1) {
        if (s.matricola == matricola) {
            printf("\nStudente trovato:\n");
            printf("Matricola: %d\n", s.matricola);
            printf("Nome: %s %s\n", s.nome, s.cognome);
            printf("Media: %.2f\n", s.media);
            trovato = 1;
            break;
        }
    }

    if (!trovato) {
        printf("Studente non trovato!\n");
    }

    fclose(file);
}

int main(int argc, char** argv) {
    int scelta, matricola;

    do {
        printf("\n=== GESTIONE STUDENTI ===\n");
        printf("1. Aggiungi studente\n");
        printf("2. Visualizza tutti\n");
        printf("3. Cerca studente\n");
        printf("0. Esci\n");
        printf("Scelta: ");
        scanf("%d", &scelta);

        switch (scelta) {
            case 1:
                aggiungi_studente();
                break;
            case 2:
                visualizza_studenti();
                break;
            case 3:
                printf("Inserisci matricola: ");
                scanf("%d", &matricola);
                cerca_studente(matricola);
                break;
            case 0:
                printf("Arrivederci!\n");
                break;
            default:
                printf("Scelta non valida!\n");
        }
    } while (scelta != 0);

    return 0;
}
\end{lstlisting}

\subsection{File CSV}

\begin{lstlisting}
#include <stdio.h>
#include <string.h>

typedef struct {
    char nome[30];
    char cognome[30];
    int eta;
    char citta[30];
} Persona;

void scrivi_csv() {
    FILE *file = fopen("persone.csv", "w");
    if (file == NULL) {
        printf("Errore!\n");
        return;
    }

    // Intestazione
    fprintf(file, "Nome,Cognome,Eta,Citta\n");

    // Dati
    fprintf(file, "Mario,Rossi,30,Milano\n");
    fprintf(file, "Lucia,Bianchi,25,Roma\n");
    fprintf(file, "Giovanni,Verdi,35,Napoli\n");

    fclose(file);
    printf("File CSV creato!\n");
}

void leggi_csv() {
    FILE *file = fopen("persone.csv", "r");
    if (file == NULL) {
        printf("File non trovato!\n");
        return;
    }

    char linea[200];
    Persona p;

    // Salta l'intestazione
    fgets(linea, sizeof(linea), file);

    printf("\n=== DATI DA CSV ===\n");

    while (fgets(linea, sizeof(linea), file) != NULL) {
        // Parse della riga CSV
        sscanf(linea, "%[^,],%[^,],%d,%[^\n]",
               p.nome, p.cognome, &p.eta, p.citta);

        printf("%s %s, %d anni, %s\n",
               p.nome, p.cognome, p.eta, p.citta);
    }

    fclose(file);
}

int main(int argc, char** argv) {
    scrivi_csv();
    leggi_csv();
    return 0;
}
\end{lstlisting}

\section{Gestione Errori}

\subsection{perror}

Stampa un messaggio di errore descrittivo.

\begin{lstlisting}
#include <stdio.h>
#include <errno.h>

int main(int argc, char** argv) {
    FILE *file = fopen("nonexistent.txt", "r");

    if (file == NULL) {
        perror("Errore");  // Stampa: "Errore: No such file or directory"
        return 1;
    }

    fclose(file);
    return 0;
}
\end{lstlisting}

\section{Esercizi}

\subsection{Livello Base}

\begin{enumerate}
    \item Scrivi un programma che crea un file di testo e ci scrive 10 numeri casuali.
    \item Crea un programma che legge un file di testo e stampa solo le righe che contengono una parola specifica.
    \item Implementa un programma che conta quante volte appare un carattere in un file.
    \item Scrivi un programma che unisce due file di testo in un terzo file.
\end{enumerate}

\subsection{Livello Intermedio}

\begin{enumerate}
    \item Crea un programma che ordina le righe di un file in ordine alfabetico.
    \item Implementa un sistema di log che registra data, ora e messaggio su file.
    \item Scrivi un programma che cifra e decifra un file di testo con cifrario di Cesare.
    \item Crea un programma che gestisce una rubrica telefonica salvata su file CSV.
\end{enumerate}

\subsection{Livello Avanzato}

\begin{enumerate}
    \item Implementa un editor di testo semplice che permette di modificare righe specifiche di un file.
    \item Crea un sistema di gestione inventario con salvataggio su file binario e funzioni di ricerca.
    \item Scrivi un programma che comprime un file di testo eliminando spazi multipli e righe vuote.
    \item Implementa un sistema di backup incrementale che salva solo le modifiche rispetto all'ultima versione.
\end{enumerate}
