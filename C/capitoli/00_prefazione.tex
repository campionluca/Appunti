\chapter*{Prefazione}
\addcontentsline{toc}{chapter}{Prefazione}

\section*{Benvenuti al Corso di Programmazione C}

Questo libro nasce dall'esigenza di fornire agli studenti del terzo anno dell'Istituto Tecnico un materiale didattico completo, chiaro e pratico per l'apprendimento del linguaggio C. Il C è un linguaggio di programmazione fondamentale, la cui conoscenza rappresenta una solida base per qualsiasi percorso nel mondo della programmazione e dell'informatica.

\section*{Perché Studiare il C?}

Il linguaggio C, nato nei primi anni '70 nei laboratori Bell, è ancora oggi uno dei linguaggi più utilizzati e influenti nella storia dell'informatica. Nonostante la sua età, il C rimane attuale per diversi motivi:

\begin{itemize}
    \item \textbf{Fondamenta solide}: Il C insegna i concetti fondamentali della programmazione procedurale e della gestione della memoria.
    \item \textbf{Efficienza}: È uno dei linguaggi più veloci ed efficienti, utilizzato per sistemi operativi, driver e software embedded.
    \item \textbf{Portabilità}: Il codice C può essere compilato su praticamente qualsiasi piattaforma.
    \item \textbf{Controllo}: Offre un controllo diretto sull'hardware e sulla memoria.
    \item \textbf{Base per altri linguaggi}: C++, Java, C\# e molti altri linguaggi derivano dal C o ne condividono la sintassi.
    \item \textbf{Richiesto nel mondo professionale}: Molte aziende cercano programmatori C, specialmente nei settori automotive, aerospaziale, telecomunicazioni e sistemi embedded.
\end{itemize}

\begin{nota}
Imparare il C vi renderà programmatori migliori, indipendentemente dal linguaggio che userete in futuro. Il C vi obbliga a comprendere realmente cosa succede "sotto il cofano" del computer.
\end{nota}

\section*{A Chi si Rivolge Questo Libro}

Questo materiale è stato pensato specificamente per studenti del terzo anno di un Istituto Tecnico con indirizzo informatico. I prerequisiti richiesti sono:

\begin{itemize}
    \item Conoscenza di base della matematica (aritmetica, algebra elementare)
    \item Capacità di ragionamento logico
    \item Familiarità con l'uso del computer
    \item Voglia di imparare e sperimentare
\end{itemize}

\textbf{Non è richiesta} alcuna esperienza precedente di programmazione. Partiremo dalle basi e procederemo gradualmente verso argomenti più complessi.

\section*{Struttura del Libro}

Il materiale è organizzato in 11 capitoli, progettati per essere affrontati sequenzialmente:

\begin{description}
    \item[Capitolo 1 - Introduzione al C] Storia del linguaggio, ambiente di sviluppo, primo programma.
    \item[Capitolo 2 - Variabili e Tipi di Dati] Dichiarazione variabili, tipi base, input/output.
    \item[Capitolo 3 - Operatori ed Espressioni] Operatori aritmetici, logici, relazionali e precedenza.
    \item[Capitolo 4 - Controllo di Flusso] Strutture condizionali (if, switch) e cicli (for, while).
    \item[Capitolo 5 - Array] Array monodimensionali e multidimensionali.
    \item[Capitolo 6 - Funzioni] Dichiarazione, definizione, parametri, ricorsione.
    \item[Capitolo 7 - Puntatori] Concetti fondamentali, dereferenziazione, aritmetica dei puntatori.
    \item[Capitolo 8 - Stringhe] Manipolazione stringhe, funzioni della libreria standard.
    \item[Capitolo 9 - Struct] Strutture dati personalizzate, typedef, union, enum.
    \item[Capitolo 10 - File] Input/output su file, gestione degli errori.
    \item[Capitolo 11 - Preprocessore] Direttive, macro, compilazione condizionale.
\end{description}

Ogni capitolo include:
\begin{itemize}
    \item Spiegazioni teoriche chiare e concise
    \item Esempi di codice commentati
    \item Box informativi (attenzione, note, errori comuni)
    \item Esercizi di difficoltà crescente
    \item Riepilogo finale
\end{itemize}

\section*{Come Usare Questo Libro}

Per ottenere il massimo da questo materiale didattico, segui questi consigli:

\subsection*{1. Leggi Attivamente}

Non limitarti a leggere passivamente. Prendi appunti, sottolinea, fai schemi. Cerca di riformulare i concetti con parole tue.

\subsection*{2. Scrivi il Codice}

\begin{attenzione}
NON copiare e incollare il codice dagli esempi! Digitalo personalmente, carattere per carattere. Questo ti aiuterà a:
\begin{itemize}
    \item Memorizzare la sintassi
    \item Notare i dettagli
    \item Fare esperienza con gli errori di compilazione
    \item Sviluppare "memoria muscolare" per la programmazione
\end{itemize}
\end{attenzione}

\subsection*{3. Sperimenta}

Una volta che un esempio funziona:
\begin{itemize}
    \item Modificalo
    \item Prova a "romperlo"
    \item Cambia i valori
    \item Aggiungi funzionalità
    \item Chiediti "cosa succede se..."
\end{itemize}

La sperimentazione è fondamentale per l'apprendimento della programmazione.

\subsection*{4. Fai Gli Esercizi}

Gli esercizi sono organizzati in tre livelli:

\begin{description}
    \item[Base] Applicazione diretta dei concetti spiegati nel capitolo. Sono essenziali per verificare di aver compreso le basi.
    \item[Intermedio] Richiedono di combinare più concetti o di ragionare un po' di più. Sono il cuore dell'apprendimento.
    \item[Avanzato] Problemi più complessi che richiedono creatività e ragionamento. Sono sfide che vi prepareranno per situazioni reali.
\end{description}

\textbf{Strategia consigliata}:
\begin{enumerate}
    \item Prova a risolvere l'esercizio autonomamente
    \item Se ti blocchi, rileggi la sezione teorica
    \item Se ancora non riesci, consulta la soluzione nell'appendice
    \item Analizza la soluzione e cerca di capire ogni riga
    \item Prova a riscrivere la soluzione senza guardarla
\end{enumerate}

\subsection*{5. Gestisci gli Errori}

\begin{nota}
Gli errori sono \textbf{normali} e \textbf{utili}! Ogni programmatore, dal principiante al professionista, incontra continuamente errori. La differenza sta nel saperli affrontare:

\begin{itemize}
    \item Leggi attentamente i messaggi di errore del compilatore
    \item Google è tuo amico: cerca l'errore online
    \item Controlla riga per riga il codice
    \item Usa printf per "debuggare" e capire cosa sta succedendo
    \item Chiedi aiuto ai compagni o all'insegnante
\end{itemize}
\end{nota}

\subsection*{6. Pratica Costante}

La programmazione è come imparare a suonare uno strumento: richiede pratica regolare. Dedica tempo ogni giorno, anche solo 30 minuti, piuttosto che sessioni lunghe saltuarie.

\section*{Ambiente di Sviluppo}

Per seguire questo corso avrai bisogno di:

\begin{itemize}
    \item \textbf{Un compilatore C}: Consigliamo GCC (GNU Compiler Collection)
    \begin{itemize}
        \item Linux: già installato o disponibile tramite package manager
        \item macOS: installabile tramite Xcode Command Line Tools
        \item Windows: MinGW o Cygwin
    \end{itemize}
    \item \textbf{Un editor di testo o IDE}:
    \begin{itemize}
        \item Visual Studio Code (consigliato, gratuito, multipiattaforma)
        \item Code::Blocks (IDE specifico per C/C++, gratuito)
        \item Dev-C++ (solo Windows, semplice per iniziare)
        \item CLion (professionale, a pagamento ma gratuito per studenti)
    \end{itemize}
\end{itemize}

Nel Capitolo 1 troverai le istruzioni dettagliate per installare e configurare l'ambiente di sviluppo.

\section*{Convenzioni Tipografiche}

In questo libro utilizziamo diverse convenzioni per rendere il contenuto più chiaro:

\begin{itemize}
    \item \texttt{Codice inline}: parole chiave, nomi di variabili, funzioni (es. \texttt{printf}, \texttt{int}, \texttt{main})
    \item \textbf{Grassetto}: concetti importanti, termini tecnici alla prima occorrenza
    \item \textit{Corsivo}: enfasi, termini stranieri
\end{itemize}

\subsection*{Box Informativi}

Troverai diversi tipi di box colorati:

\begin{attenzione}
I box \textbf{Attenzione} (arancione) evidenziano concetti critici, errori da evitare o informazioni particolarmente importanti che non devono essere trascurate.
\end{attenzione}

\begin{nota}
I box \textbf{Nota} (blu) contengono suggerimenti, consigli pratici, best practices e informazioni utili che arricchiscono la comprensione.
\end{nota}

\begin{errore}
I box \textbf{Errore Comune} (rosso) descrivono errori tipici che i principianti (e a volte anche i programmatori esperti!) commettono frequentemente, aiutandoti a evitarli.
\end{errore}

\subsection*{Esempi di Codice}

Gli esempi di codice sono presentati in box con sfondo grigio chiaro, numerazione delle righe e sintassi colorata:

\begin{lstlisting}[caption={Esempio di codice C}]
#include <stdio.h>

int main() {
    // Questo e' un commento
    printf("Ciao, mondo!\n");  // Stampa un messaggio
    return 0;
}
\end{lstlisting}

\section*{Risorse Online}

Oltre a questo libro, ti consigliamo di consultare:

\begin{itemize}
    \item \textbf{C Reference}: \url{https://en.cppreference.com/w/c}
    \item \textbf{Learn C}: \url{https://www.learn-c.org/}
    \item \textbf{Stack Overflow}: \url{https://stackoverflow.com/questions/tagged/c}
    \item \textbf{Online GDB}: \url{https://www.onlinegdb.com/} (compilatore online)
\end{itemize}

\section*{Un Ultimo Consiglio}

\begin{center}
\Large\textit{``La programmazione è un'arte pratica. \\
Non imparerai a programmare leggendo, \\
ma scrivendo codice.''}
\end{center}

Non scoraggiarti se all'inizio alcuni concetti sembrano difficili. Il C è un linguaggio che richiede tempo e pratica per essere padroneggiato. La frustrazione è parte del processo di apprendimento. Ogni errore risolto è una lezione appresa.

Prenditi il tempo necessario, procedi con calma, e soprattutto: \textbf{divertiti}! La programmazione è un'attività creativa e gratificante. Vedrai che, capitolo dopo capitolo, diventerai sempre più sicuro e capace.

\vspace{1cm}

Buono studio e buona programmazione!

\vspace{0.5cm}

\hfill \textit{L'Autore}

\vspace{0.5cm}

\hfill \textit{Novembre 2025}
