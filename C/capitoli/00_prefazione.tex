\chapter*{Prefazione}
\addcontentsline{toc}{chapter}{Prefazione}

\section*{Benvenuti al Corso di Programmazione C}

Questo libro nasce dall'esigenza di fornire agli studenti del terzo anno dell'Istituto Tecnico un materiale didattico completo, chiaro e pratico per l'apprendimento del linguaggio C. Il C è un linguaggio di programmazione fondamentale, la cui conoscenza rappresenta una solida base per qualsiasi percorso nel mondo della programmazione e dell'informatica.

\section*{Perché Studiare il C?}

Il linguaggio C, nato nei primi anni '70 nei laboratori Bell della AT\&T, è ancora oggi uno dei linguaggi più utilizzati e influenti nella storia dell'informatica. Nonostante la sua età, il C rimane straordinariamente attuale per molteplici ragioni. Innanzitutto, il C fornisce le \textbf{fondamenta solide} della programmazione, insegnandovi i concetti fondamentali della programmazione procedurale e della gestione della memoria direttamente, senza gli "aiutini" che linguaggi moderni vi forniscono. In termini di prestazioni, il C è uno dei linguaggi più \textbf{efficienti} ed è per questo motivo ampiamente utilizzato per la realizzazione di sistemi operativi, driver hardware e software embedded dove la velocità è critica. Un grande vantaggio del C è la sua \textbf{portabilità}: il codice C può essere compilato e eseguito su praticamente qualsiasi piattaforma, da computer desktop a microcontrollori, consentendo una vera "write once, compile anywhere" philosophy. Il C offre inoltre un \textbf{controllo diretto} sull'hardware e sulla memoria, permettendovi di ottimizzare il codice al massimo livello di dettaglio. Molti linguaggi moderni, tra cui C++, Java e C\#, \textbf{derivano dal C} o ne condividono la sintassi, quindi imparare il C vi fornirà una base solida per apprendere quasi qualsiasi altro linguaggio. Infine, il C rimane \textbf{richiesto nel mondo professionale}: molte aziende, specialmente nei settori automotive, aerospaziale, telecomunicazioni e sistemi embedded, cercano continuamente programmatori competenti in C.

\begin{nota}
Imparare il C vi renderà programmatori migliori, indipendentemente dal linguaggio che userete in futuro. Il C vi obbliga a comprendere realmente cosa succede "sotto il cofano" del computer.
\end{nota}

\section*{A Chi si Rivolge Questo Libro}

Questo materiale è stato pensato specificamente per studenti del terzo anno di un Istituto Tecnico con indirizzo informatico. Per affrontare con successo questo corso, è necessario possedere una \textbf{conoscenza di base della matematica}, in particolare aritmetica e algebra elementare, per comprendere le espressioni e i calcoli che affronteremo. Dovete avere sviluppato una buona \textbf{capacità di ragionamento logico}, in quanto la programmazione è fondamentalmente il trasformare la logica in codice. È importante avere una certa \textbf{familiarità con l'uso del computer}, compresa la capacità di navigare il file system, usare il terminale/prompt dei comandi e lanciare programmi. Infine, ma non meno importante, dovete avere una genuina \textbf{voglia di imparare e sperimentare}, poiché la programmazione è un'attività che richiede curiosità e la volontà di scoprire come le cose funzionano.

Tuttavia, \textbf{non è richiesta} alcuna esperienza precedente di programmazione. Questo corso è progettato per i principianti assoluti, e partiremo dalle basi più elementari, procedendo gradualmente verso argomenti più complessi e sofisticati.

\section*{Struttura del Libro}

Il materiale è organizzato in 11 capitoli, progettati per essere affrontati sequenzialmente:

\begin{description}
    \item[Capitolo 1 - Introduzione al C] Storia del linguaggio, ambiente di sviluppo, primo programma.
    \item[Capitolo 2 - Variabili e Tipi di Dati] Dichiarazione variabili, tipi base, input/output.
    \item[Capitolo 3 - Operatori ed Espressioni] Operatori aritmetici, logici, relazionali e precedenza.
    \item[Capitolo 4 - Controllo di Flusso] Strutture condizionali (if, switch) e cicli (for, while).
    \item[Capitolo 5 - Array] Array monodimensionali e multidimensionali.
    \item[Capitolo 6 - Funzioni] Dichiarazione, definizione, parametri, ricorsione.
    \item[Capitolo 7 - Puntatori] Concetti fondamentali, dereferenziazione, aritmetica dei puntatori.
    \item[Capitolo 8 - Stringhe] Manipolazione stringhe, funzioni della libreria standard.
    \item[Capitolo 9 - Struct] Strutture dati personalizzate, typedef, union, enum.
    \item[Capitolo 10 - File] Input/output su file, gestione degli errori.
    \item[Capitolo 11 - Preprocessore] Direttive, macro, compilazione condizionale.
\end{description}

Ogni capitolo include:
\begin{itemize}
    \item Spiegazioni teoriche chiare e concise
    \item Esempi di codice commentati
    \item Box informativi (attenzione, note, errori comuni)
    \item Esercizi di difficoltà crescente
    \item Riepilogo finale
\end{itemize}

\section*{Come Usare Questo Libro}

Per ottenere il massimo da questo materiale didattico, segui questi consigli:

\subsection*{1. Leggi Attivamente}

Non limitarti a leggere passivamente. Prendi appunti, sottolinea, fai schemi. Cerca di riformulare i concetti con parole tue.

\subsection*{2. Scrivi il Codice}

\begin{attenzione}
NON copiare e incollare il codice dagli esempi! Digitalo personalmente, carattere per carattere. Questo semplice atto di digitazione manuale ha importanti benefici didattici. In primo luogo, vi aiuterà a \textbf{memorizzare la sintassi} del linguaggio, poiché il vostro cervello processerà ogni carattere. Inoltre, dovrete \textbf{notare i dettagli}, come la posizione di ogni parentesi, punto e virgola o operatore, il che vi renderà consapevoli di sottili particolarità del linguaggio. Digitando il codice manualmente, farete esperienza con gli errori di compilazione in modo naturale, imparando a leggere e interpretare i messaggi di errore. Infine, svilupperete quella che i programmatori chiamano "memoria muscolare" - un'intuizione automatica di come scrivere il codice correttamente.
\end{attenzione}

\subsection*{3. Sperimenta}

Una volta che un esempio funziona correttamente, non fermatevi lì. La vera apprendimento avviene quando iniziate a sperimentare. \textbf{Modificate il codice} in modo deliberato per vedere come cambiano i risultati. \textbf{Provate a "romperlo"} intenzionalmente - cambiate operatori, rimuovete righe, vedete cosa succede quando il programma fallisce; questo vi insegnerà quali parti del codice sono critiche. \textbf{Cambiate i valori} per capire come il programma reagisce a input diversi. \textbf{Aggiungete funzionalità} al programma - magari fate sì che accetti input diversi, che stampi informazioni aggiuntive, o che faccia calcoli più complessi. Soprattutto, \textbf{chiedetevi "cosa succede se..."} continuamente. Questa curiosità è il motore dell'apprendimento della programmazione. La sperimentazione non è una distrazione dal vostro studio - è il vostro studio principale.

\subsection*{4. Fai Gli Esercizi}

Gli esercizi sono organizzati in tre livelli:

\begin{description}
    \item[Base] Applicazione diretta dei concetti spiegati nel capitolo. Sono essenziali per verificare di aver compreso le basi.
    \item[Intermedio] Richiedono di combinare più concetti o di ragionare un po' di più. Sono il cuore dell'apprendimento.
    \item[Avanzato] Problemi più complessi che richiedono creatività e ragionamento. Sono sfide che vi prepareranno per situazioni reali.
\end{description}

\textbf{Strategia consigliata}:
\begin{enumerate}
    \item Prova a risolvere l'esercizio autonomamente
    \item Se ti blocchi, rileggi la sezione teorica
    \item Se ancora non riesci, consulta la soluzione nell'appendice
    \item Analizza la soluzione e cerca di capire ogni riga
    \item Prova a riscrivere la soluzione senza guardarla
\end{enumerate}

\subsection*{5. Gestisci gli Errori}

\begin{nota}
Gli errori sono \textbf{normali} e \textbf{utili}! Ogni programmatore, dal principiante al professionista, incontra continuamente errori. La differenza tra un programmatore esperto e uno inesperto non è che il primo non fa errori, ma che sa come affrontarli efficacemente. Quando incontrate un errore, \textbf{leggete attentamente i messaggi di errore del compilatore}: il compilatore sta cercando di dirvi esattamente cosa è andato storto, di solito con il numero di linea e una descrizione del problema. Non abbiate paura di usare Google o risorse online quando siete bloccati - \textbf{Google è vostro amico}, e cercare l'errore online vi darà spesso la risposta istantaneamente, poiché qualcuno ha probabilmente incontrato lo stesso problema prima di voi. Quando cercate di capire cosa non funziona, \textbf{controllate il codice riga per riga}, idealmente con un amico che vi aiuta, poiché gli occhi di qualcun altro possono spesso notare errori che voi avete tralasciato. Un'ottima tecnica è \textbf{usare printf per "debuggare"}, cioè inserire statement che stampano i valori delle variabili in punti strategici del codice, per capire cosa sta realmente succedendo durante l'esecuzione. Infine, non siate riluttanti a \textbf{chiedere aiuto ai compagni o all'insegnante}: la collaborazione è una parte essenziale della programmazione nel mondo reale.
\end{nota}

\subsection*{6. Pratica Costante}

La programmazione è come imparare a suonare uno strumento: richiede pratica regolare. Dedica tempo ogni giorno, anche solo 30 minuti, piuttosto che sessioni lunghe saltuarie.

\section*{Ambiente di Sviluppo}

Per seguire questo corso avrai bisogno di:

\begin{itemize}
    \item \textbf{Un compilatore C}: Consigliamo GCC (GNU Compiler Collection)
    \begin{itemize}
        \item Linux: già installato o disponibile tramite package manager
        \item macOS: installabile tramite Xcode Command Line Tools
        \item Windows: MinGW o Cygwin
    \end{itemize}
    \item \textbf{Un editor di testo o IDE}:
    \begin{itemize}
        \item Visual Studio Code (consigliato, gratuito, multipiattaforma)
        \item Code::Blocks (IDE specifico per C/C++, gratuito)
        \item Dev-C++ (solo Windows, semplice per iniziare)
        \item CLion (professionale, a pagamento ma gratuito per studenti)
    \end{itemize}
\end{itemize}

Nel Capitolo 1 troverai le istruzioni dettagliate per installare e configurare l'ambiente di sviluppo.

\section*{Convenzioni Tipografiche}

In questo libro utilizziamo diverse convenzioni per rendere il contenuto più chiaro:

\begin{itemize}
    \item \texttt{Codice inline}: parole chiave, nomi di variabili, funzioni (es. \texttt{printf}, \texttt{int}, \texttt{main})
    \item \textbf{Grassetto}: concetti importanti, termini tecnici alla prima occorrenza
    \item \textit{Corsivo}: enfasi, termini stranieri
\end{itemize}

\subsection*{Box Informativi}

Troverai diversi tipi di box colorati:

\begin{attenzione}
I box \textbf{Attenzione} (arancione) evidenziano concetti critici, errori da evitare o informazioni particolarmente importanti che non devono essere trascurate.
\end{attenzione}

\begin{nota}
I box \textbf{Nota} (blu) contengono suggerimenti, consigli pratici, best practices e informazioni utili che arricchiscono la comprensione.
\end{nota}

\begin{errore}
I box \textbf{Errore Comune} (rosso) descrivono errori tipici che i principianti (e a volte anche i programmatori esperti!) commettono frequentemente, aiutandoti a evitarli.
\end{errore}

\subsection*{Esempi di Codice}

Gli esempi di codice sono presentati in box con sfondo grigio chiaro, numerazione delle righe e sintassi colorata:

\begin{lstlisting}[caption={Esempio di codice C}]
#include <stdio.h>

int main() {
    // Questo e' un commento
    printf("Ciao, mondo!\n");  // Stampa un messaggio
    return 0;
}
\end{lstlisting}

\section*{Risorse Online}

Oltre a questo libro, ti consigliamo di consultare:

\begin{itemize}
    \item \textbf{C Reference}: \url{https://en.cppreference.com/w/c}
    \item \textbf{Learn C}: \url{https://www.learn-c.org/}
    \item \textbf{Stack Overflow}: \url{https://stackoverflow.com/questions/tagged/c}
    \item \textbf{Online GDB}: \url{https://www.onlinegdb.com/} (compilatore online)
\end{itemize}

\section*{Un Ultimo Consiglio}

\begin{center}
\Large\textit{``La programmazione è un'arte pratica. \\
Non imparerai a programmare leggendo, \\
ma scrivendo codice.''}
\end{center}

Non scoraggiarti se all'inizio alcuni concetti sembrano difficili. Il C è un linguaggio che richiede tempo e pratica per essere padroneggiato. La frustrazione è parte del processo di apprendimento. Ogni errore risolto è una lezione appresa.

Prenditi il tempo necessario, procedi con calma, e soprattutto: \textbf{divertiti}! La programmazione è un'attività creativa e gratificante. Vedrai che, capitolo dopo capitolo, diventerai sempre più sicuro e capace.

\vspace{1cm}

Buono studio e buona programmazione!

\vspace{0.5cm}

\hfill \textit{L'Autore}

\vspace{0.5cm}

\hfill \textit{Novembre 2025}
