\chapter*{Bibliografia e Risorse}
\addcontentsline{toc}{chapter}{Bibliografia e Risorse}

\section*{Libri di Testo}

\subsection*{Testi Fondamentali}

\begin{enumerate}
    \item \textbf{Kernighan, B. W., \& Ritchie, D. M.} (1988). \textit{The C Programming Language} (2nd ed.). Prentice Hall.
    \begin{itemize}
        \item Il libro classico scritto dai creatori del linguaggio C
        \item Considerato la "bibbia" del C
        \item Copre tutte le caratteristiche fondamentali del linguaggio
    \end{itemize}

    \item \textbf{King, K. N.} (2008). \textit{C Programming: A Modern Approach} (2nd ed.). W. W. Norton \& Company.
    \begin{itemize}
        \item Approccio moderno alla programmazione in C
        \item Molti esempi pratici ed esercizi
        \item Ottimo per principianti
    \end{itemize}

    \item \textbf{Deitel, P., \& Deitel, H.} (2015). \textit{C How to Program} (8th ed.). Pearson.
    \begin{itemize}
        \item Trattazione completa con approccio didattico
        \item Numerosi esempi e case study
        \item Include esercizi graduali
    \end{itemize}
\end{enumerate}

\subsection*{Testi Avanzati}

\begin{enumerate}
    \item \textbf{Kochan, S. G.} (2014). \textit{Programming in C} (4th ed.). Addison-Wesley.
    \item \textbf{Prata, S.} (2013). \textit{C Primer Plus} (6th ed.). Addison-Wesley.
    \item \textbf{Summit, S.} (1995). \textit{C Programming FAQs}. Addison-Wesley.
\end{enumerate}

\section*{Risorse Online}

\subsection*{Tutorial e Guide}

\begin{enumerate}
    \item \textbf{Learn-C.org}
    \begin{itemize}
        \item \url{https://www.learn-c.org/}
        \item Tutorial interattivo gratuito
        \item Esercizi pratici con verifica automatica
    \end{itemize}

    \item \textbf{GeeksforGeeks - C Programming}
    \begin{itemize}
        \item \url{https://www.geeksforgeeks.org/c-programming-language/}
        \item Articoli approfonditi su ogni aspetto del C
        \item Molti esempi e spiegazioni
    \end{itemize}

    \item \textbf{Tutorialspoint - C Programming}
    \begin{itemize}
        \item \url{https://www.tutorialspoint.com/cprogramming/}
        \item Tutorial completo con esempi
        \item Possibilità di compilare online
    \end{itemize}

    \item \textbf{C Programming at Programiz}
    \begin{itemize}
        \item \url{https://www.programiz.com/c-programming}
        \item Guide chiare per principianti
        \item Esempi pratici e ben commentati
    \end{itemize}
\end{enumerate}

\subsection*{Documentazione di Riferimento}

\begin{enumerate}
    \item \textbf{cppreference.com}
    \begin{itemize}
        \item \url{https://en.cppreference.com/w/c}
        \item Documentazione completa e accurata
        \item Include C11 e C17
    \end{itemize}

    \item \textbf{C Standard Library Reference}
    \begin{itemize}
        \item \url{https://www.cplusplus.com/reference/clibrary/}
        \item Riferimento per tutte le funzioni standard
        \item Esempi per ogni funzione
    \end{itemize}
\end{enumerate}

\subsection*{Piattaforme di Pratica}

\begin{enumerate}
    \item \textbf{HackerRank}
    \begin{itemize}
        \item \url{https://www.hackerrank.com/domains/c}
        \item Sfide di programmazione in C
        \item Livelli di difficoltà progressivi
    \end{itemize}

    \item \textbf{LeetCode}
    \begin{itemize}
        \item \url{https://leetcode.com/}
        \item Problemi di algoritmi e strutture dati
        \item Supporto per C
    \end{itemize}

    \item \textbf{Codewars}
    \begin{itemize}
        \item \url{https://www.codewars.com/}
        \item Kata (esercizi) di vari livelli
        \item Sistema di ranking e progressione
    \end{itemize}

    \item \textbf{Project Euler}
    \begin{itemize}
        \item \url{https://projecteuler.net/}
        \item Problemi matematici da risolvere con programmazione
        \item Ottimo per migliorare il problem solving
    \end{itemize}

    \item \textbf{Codeforces}
    \begin{itemize}
        \item \url{https://codeforces.com/}
        \item Competizioni di programmazione
        \item Archivio di problemi storici
    \end{itemize}
\end{enumerate}

\section*{Strumenti di Sviluppo}

\subsection*{Compilatori}

\begin{enumerate}
    \item \textbf{GCC (GNU Compiler Collection)}
    \begin{itemize}
        \item Disponibile su Linux, macOS, Windows
        \item Open source e molto diffuso
        \item \url{https://gcc.gnu.org/}
    \end{itemize}

    \item \textbf{Clang}
    \begin{itemize}
        \item Compilatore moderno con ottimi messaggi di errore
        \item Integrato in Xcode su macOS
        \item \url{https://clang.llvm.org/}
    \end{itemize}

    \item \textbf{Microsoft Visual C++}
    \begin{itemize}
        \item Compilatore Microsoft per Windows
        \item Integrato in Visual Studio
    \end{itemize}
\end{enumerate}

\subsection*{IDE (Integrated Development Environment)}

\begin{enumerate}
    \item \textbf{Visual Studio Code}
    \begin{itemize}
        \item Editor leggero e versatile
        \item Estensioni per C/C++
        \item \url{https://code.visualstudio.com/}
    \end{itemize}

    \item \textbf{Code::Blocks}
    \begin{itemize}
        \item IDE gratuito specifico per C/C++
        \item Interfaccia semplice
        \item \url{http://www.codeblocks.org/}
    \end{itemize}

    \item \textbf{CLion}
    \begin{itemize}
        \item IDE professionale di JetBrains
        \item Gratuito per studenti
        \item \url{https://www.jetbrains.com/clion/}
    \end{itemize}

    \item \textbf{Dev-C++}
    \begin{itemize}
        \item IDE semplice per Windows
        \item Buono per principianti
        \item \url{https://www.bloodshed.net/devcpp.html}
    \end{itemize}
\end{enumerate}

\subsection*{Compilatori Online}

\begin{enumerate}
    \item \textbf{OnlineGDB}
    \begin{itemize}
        \item \url{https://www.onlinegdb.com/}
        \item Compilatore e debugger online
    \end{itemize}

    \item \textbf{Repl.it}
    \begin{itemize}
        \item \url{https://replit.com/}
        \item Ambiente di sviluppo online completo
    \end{itemize}

    \item \textbf{JDoodle}
    \begin{itemize}
        \item \url{https://www.jdoodle.com/c-online-compiler}
        \item Compilatore C online semplice
    \end{itemize}
\end{enumerate}

\subsection*{Debugger}

\begin{enumerate}
    \item \textbf{GDB (GNU Debugger)}
    \begin{itemize}
        \item Debugger a riga di comando
        \item Molto potente
        \item \url{https://www.gnu.org/software/gdb/}
    \end{itemize}

    \item \textbf{Valgrind}
    \begin{itemize}
        \item Analisi di memoria
        \item Rileva memory leak
        \item \url{https://valgrind.org/}
    \end{itemize}
\end{enumerate}

\section*{Standard e Specifiche}

\subsection*{Standard C}

\begin{enumerate}
    \item \textbf{C89/C90} - ANSI C / ISO C
    \begin{itemize}
        \item Prima standardizzazione del C
    \end{itemize}

    \item \textbf{C99} - ISO/IEC 9899:1999
    \begin{itemize}
        \item Array a lunghezza variabile
        \item Commenti in stile C++
        \item Tipo \texttt{long long}
    \end{itemize}

    \item \textbf{C11} - ISO/IEC 9899:2011
    \begin{itemize}
        \item Supporto multi-threading
        \item Analisi statica migliorata
    \end{itemize}

    \item \textbf{C17/C18} - ISO/IEC 9899:2018
    \begin{itemize}
        \item Correzioni e chiarimenti su C11
    \end{itemize}
\end{enumerate}

\section*{Community e Forum}

\begin{enumerate}
    \item \textbf{Stack Overflow}
    \begin{itemize}
        \item \url{https://stackoverflow.com/questions/tagged/c}
        \item Domande e risposte sulla programmazione in C
    \end{itemize}

    \item \textbf{Reddit - r/C\_Programming}
    \begin{itemize}
        \item \url{https://www.reddit.com/r/C_Programming/}
        \item Community attiva di programmatori C
    \end{itemize}

    \item \textbf{C Board}
    \begin{itemize}
        \item \url{https://cboard.cprogramming.com/}
        \item Forum dedicato alla programmazione in C
    \end{itemize}
\end{enumerate}

\section*{Canali YouTube}

\begin{enumerate}
    \item \textbf{freeCodeCamp.org} - Tutorial completi C
    \item \textbf{The Cherno} - Programmazione C/C++
    \item \textbf{Jacob Sorber} - C programming concepts
    \item \textbf{Code Vault} - Advanced C Programming
\end{enumerate}

\section*{Best Practices e Style Guide}

\begin{enumerate}
    \item \textbf{GNU Coding Standards}
    \begin{itemize}
        \item \url{https://www.gnu.org/prep/standards/}
        \item Standard di codifica GNU
    \end{itemize}

    \item \textbf{Linux Kernel Coding Style}
    \begin{itemize}
        \item \url{https://www.kernel.org/doc/html/latest/process/coding-style.html}
        \item Style guide del kernel Linux
    \end{itemize}

    \item \textbf{MISRA C}
    \begin{itemize}
        \item Standard per C sicuro in sistemi critici
        \item Usato nell'automotive e aerospace
    \end{itemize}
\end{enumerate}

\section*{Progetti Open Source in C}

Studiare codice di qualità è un ottimo modo per imparare:

\begin{enumerate}
    \item \textbf{Linux Kernel} - \url{https://github.com/torvalds/linux}
    \item \textbf{Git} - \url{https://github.com/git/git}
    \item \textbf{SQLite} - \url{https://www.sqlite.org/}
    \item \textbf{Redis} - \url{https://github.com/redis/redis}
    \item \textbf{Nginx} - \url{https://github.com/nginx/nginx}
\end{enumerate}

\section*{Certificazioni}

\begin{enumerate}
    \item \textbf{C Programming Language Certified Associate (CLA)}
    \begin{itemize}
        \item Certificazione entry-level
        \item Offerta da C++ Institute
    \end{itemize}

    \item \textbf{C Programming Language Certified Professional (CLP)}
    \begin{itemize}
        \item Certificazione avanzata
        \item Per programmatori esperti
    \end{itemize}
\end{enumerate}

\section*{Note Finali}

Questa bibliografia rappresenta solo un punto di partenza. Il mondo della programmazione in C è vasto e in continua evoluzione. Alcuni consigli:

\begin{itemize}
    \item \textbf{Pratica costante}: la programmazione si impara programmando
    \item \textbf{Leggi codice}: studia progetti open source ben scritti
    \item \textbf{Partecipa alle community}: fai domande e aiuta altri
    \item \textbf{Mantieniti aggiornato}: segui blog e newsletter
    \item \textbf{Sperimenta}: non aver paura di provare cose nuove
\end{itemize}

Ricorda: il miglior programmatore non è quello che conosce tutto, ma quello che sa dove trovare le informazioni quando serve!
