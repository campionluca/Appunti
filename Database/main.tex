% main.tex - Appunti Database e SQL
\documentclass[a4paper,11pt,oneside]{book}

% Pacchetti essenziali
\usepackage[italian]{babel}
\usepackage{fontspec}
\usepackage{geometry}
\usepackage{graphicx}
\usepackage{float}
\usepackage{listings}
\usepackage{xcolor}
\usepackage{hyperref}
\usepackage{fancyhdr}
\usepackage{amsmath}
\usepackage{tcolorbox}
\usepackage{tikz}
\usepackage{array}
\usepackage{booktabs}
\usepackage{multirow}
\usepackage{titling}
\usetikzlibrary{shapes,arrows,arrows.meta,positioning,trees,shadows,fit,shapes.multipart,er}

% Configurazione pagina
\geometry{margin=2.5cm}
\setlength{\parindent}{0pt}
\setlength{\parskip}{6pt}

% Configurazione Listings per SQL
\lstdefinelanguage{SQL}{
    keywords={SELECT, FROM, WHERE, INSERT, INTO, VALUES, UPDATE, SET, DELETE,
              CREATE, TABLE, DROP, ALTER, PRIMARY, KEY, FOREIGN, REFERENCES,
              JOIN, INNER, LEFT, RIGHT, OUTER, ON, AS, AND, OR, NOT, NULL,
              UNIQUE, INDEX, DATABASE, USE, SHOW, DESCRIBE, INT, VARCHAR,
              CHAR, DATE, DATETIME, TEXT, FLOAT, DOUBLE, DECIMAL, BOOLEAN,
              AUTO_INCREMENT, DEFAULT, CHECK, CONSTRAINT, ORDER, BY, GROUP,
              HAVING, DISTINCT, COUNT, SUM, AVG, MIN, MAX, LIKE, IN, BETWEEN,
              IS, EXISTS, ALL, ANY, UNION, INTERSECT, EXCEPT, CASE, WHEN, THEN,
              ELSE, END, VIEW, TRIGGER, PROCEDURE, FUNCTION, BEGIN, COMMIT,
              ROLLBACK, TRANSACTION, GRANT, REVOKE},
    sensitive=false,
    morecomment=[l]{--},
    morecomment=[s]{/*}{*/},
    morestring=[b]',
    morestring=[b]"
}

\lstset{
    language=SQL,
    basicstyle=\ttfamily\small,
    keywordstyle=\color{blue}\bfseries,
    commentstyle=\color{green!60!black}\itshape,
    stringstyle=\color{red},
    numbers=left,
    numberstyle=\tiny\color{gray},
    stepnumber=1,
    numbersep=8pt,
    backgroundcolor=\color{gray!5},
    showspaces=false,
    showstringspaces=false,
    showtabs=false,
    frame=single,
    rulecolor=\color{black!30},
    tabsize=4,
    captionpos=b,
    breaklines=true,
    breakatwhitespace=false,
    escapeinside={(*@}{@*)},
    xleftmargin=15pt,
    xrightmargin=15pt,
    float=htbp,
    aboveskip=15pt,
    belowskip=15pt
}

% Box Colorati per Note Speciali
\newtcolorbox{attenzione}{
    colback=yellow!10,
    colframe=orange!80,
    title=Attenzione,
    fonttitle=\bfseries
}

\newtcolorbox{nota}{
    colback=blue!5,
    colframe=blue!60,
    title=Nota,
    fonttitle=\bfseries
}

\newtcolorbox{errore}{
    colback=red!5,
    colframe=red!60,
    title=Errore Comune,
    fonttitle=\bfseries
}

\newtcolorbox{esempio}{
    colback=green!5,
    colframe=green!60,
    title=Esempio,
    fonttitle=\bfseries
}

% Configurazione hyperref
\hypersetup{
    colorlinks=true,
    linkcolor=blue,
    filecolor=magenta,
    urlcolor=cyan,
    citecolor=green,
    pdftitle={Appunti Database e SQL},
    pdfauthor={Prof. Luca Campion},
    pdfsubject={Database, DBMS, SQL, Modellazione Dati},
    pdfkeywords={database, sql, dbms, er, relazionale, normalizzazione}
}

% Header e Footer
\pagestyle{fancy}
\fancyhf{}
\fancyhead[L]{\leftmark}
\fancyhead[R]{Database e SQL}
\fancyfoot[C]{\thepage}
\renewcommand{\headrulewidth}{0.4pt}

\begin{document}

%=============================================================================
% FRONTMATTER
%=============================================================================
\frontmatter

\title{
    \Huge\textbf{Appunti di Database e SQL}\\[1cm]
    \Large{DBMS, Modellazione Dati e Linguaggio SQL}
}
\author{Prof. Luca Campion}
\date{Anno Scolastico 2025-2026\\[0.5cm]Istituto Tecnico Antonio Scarpa}

\maketitle

\tableofcontents
\listoffigures
\lstlistoflistings

%=============================================================================
% MAINMATTER
%=============================================================================
\mainmatter

% Prefazione
\chapter*{Prefazione}
\addcontentsline{toc}{chapter}{Prefazione}

\section*{A chi è rivolto questo libro}

Questi appunti sono stati pensati per gli studenti del quarto anno di Istituto Tecnico che stanno approfondendo la programmazione in Java. Il materiale presuppone una conoscenza di base del linguaggio (variabili, cicli, metodi, concetti fondamentali di programmazione) e si propone di consolidare e ampliare tali competenze attraverso argomenti più avanzati e pratici.

L'approccio adottato bilancia teoria ed esempi concreti, con l'obiettivo di fornire strumenti immediatamente applicabili sia nei progetti scolastici che in contesti reali.

\section*{Struttura del libro}

Il libro è organizzato in otto capitoli, ciascuno focalizzato su un argomento specifico:

\begin{enumerate}
    \item \textbf{Classi, Oggetti e Ereditarietà}: ripasso e approfondimento dei concetti fondamentali della programmazione orientata agli oggetti, con particolare attenzione agli array di oggetti e alla gerarchia tra classi.

    \item \textbf{Stream e Buffer}: gestione di flussi di dati per leggere e scrivere file, con esempi pratici di utilizzo delle classi più comuni.

    \item \textbf{Interfacce e Classi Astratte}: meccanismi per definire comportamenti comuni e creare gerarchie flessibili.

    \item \textbf{Eccezioni}: gestione degli errori a runtime attraverso il sistema delle eccezioni di Java.

    \item \textbf{ArrayList}: struttura dati dinamica per gestire collezioni di elementi in modo più flessibile rispetto agli array tradizionali.

    \item \textbf{Interfacce Grafiche}: introduzione alla creazione di applicazioni con interfaccia grafica usando Swing, inclusa la gestione degli eventi.

    \item \textbf{Model View Controller}: pattern architetturale per organizzare il codice separando logica, presentazione e controllo.

    \item \textbf{Lambda Expressions}: cenni alle espressioni lambda introdotte in Java 8, per scrivere codice più conciso ed espressivo.
\end{enumerate}

\section*{Come usare questo libro}

Ogni capitolo è strutturato per guidare l'apprendimento in modo progressivo:

\begin{itemize}
    \item Gli \textbf{obiettivi di apprendimento} all'inizio di ogni capitolo chiariscono cosa ci si aspetta di saper fare al termine dello studio.

    \item La \textbf{teoria} è presentata in modo sintetico ma completo, con definizioni chiare e schemi quando necessario.

    \item Gli \textbf{esempi di codice} sono commentati in italiano e mostrano l'applicazione pratica dei concetti. Si consiglia di digitare personalmente ogni esempio, eseguirlo e sperimentare modifiche per comprenderne il funzionamento.

    \item I \textbf{box colorati} evidenziano informazioni particolari:
    \begin{itemize}
        \item \textcolor{orange}{Arancione (Attenzione)}: punti critici da ricordare
        \item \textcolor{blue}{Blu (Nota)}: suggerimenti e best practices
        \item \textcolor{red}{Rosso (Errore Comune)}: errori frequenti da evitare
    \end{itemize}

    \item Gli \textbf{esercizi} sono suddivisi in tre livelli di difficoltà (base, intermedio, avanzato). Si consiglia di affrontarli in ordine, verificando le soluzioni commentate nell'appendice solo dopo aver tentato autonomamente.

    \item Il \textbf{riepilogo} alla fine di ogni capitolo sintetizza i concetti chiave e facilita il ripasso.
\end{itemize}

\section*{Prerequisiti}

Per affrontare efficacemente questi appunti, è necessario:

\begin{itemize}
    \item Conoscere la sintassi base di Java (tipi di dato primitivi, operatori, strutture di controllo)
    \item Saper dichiarare e utilizzare metodi
    \item Comprendere i concetti basilari di classe e oggetto
    \item Avere familiarità con array monodimensionali
    \item Disporre di un ambiente di sviluppo Java funzionante (JDK 8 o superiore, IDE come Eclipse, IntelliJ IDEA o NetBeans)
\end{itemize}

\section*{Convenzioni utilizzate}

\textbf{Codice}: tutti gli esempi di codice sono presentati con sintassi evidenziata, numerazione delle righe e commenti esplicativi.

\textbf{Nomenclatura}: si segue la convenzione Java standard (CamelCase per classi, camelCase per metodi e variabili, MAIUSCOLO per costanti).

\textbf{Terminologia}: si preferisce l'italiano quando possibile, mantenendo i termini tecnici in inglese quando consolidati nella pratica professionale (ad esempio "stream", "buffer", "exception").

\vspace{1cm}

Buono studio!


% Capitolo 01: Introduzione ai Database e DBMS
\chapter{Introduzione ai DBMS}

\section*{Introduzione}
Un Database Management System (DBMS) è un software specializzato che gestisce la memorizzazione, l'organizzazione e l'accesso ai dati. Questo capitolo introduce i concetti fondamentali dei sistemi informativi e le proprietà ACID che garantiscono l'affidabilità dei dati.

\section*{Obiettivi di apprendimento}
\begin{itemize}
    \item Comprendere la differenza tra dati e informazioni
    \item Conoscere i componenti di un sistema informativo
    \item Capire il ruolo di un DBMS
    \item Apprendere le proprietà ACID delle transazioni
    \item Identificare vantaggi e caratteristiche dei DBMS
\end{itemize}

\section{Dati, Informazioni e Sistemi Informativi}

\subsection{Differenza tra dati e informazioni}
I \textbf{dati} sono fatti grezzi e non strutturati. Le \textbf{informazioni} sono dati elaborati e organizzati che hanno significato e utilità.

\begin{tcolorbox}[colback=blue!10, colframe=blue!60, title=Esempio: Dati vs Informazioni]
\textbf{Dato}: \texttt{25, 1990, Milano, Rossi}

\textbf{Informazione}: Il cliente Luigi Rossi, nato nel 1990 a Milano, ha 25 anni ed è residente in Lombardia.
\end{tcolorbox}

\subsection{Componenti di un sistema informativo}
Un sistema informativo è composto da diversi elementi interdipendenti che lavorano insieme. L'infrastruttura hardware fornisce la base fisica con computer, server e dispositivi di storage. Il software, che include applicazioni, DBMS e sistemi operativi, gestisce l'elaborazione e l'accesso ai dati. I dati stessi rappresentano il patrimonio informativo dell'organizzazione, fondamentale per le decisioni aziendali. I processi definiscono le operazioni e le procedure aziendali che trasformano i dati in informazioni utili. Infine, le risorse umane—amministratori, sviluppatori e utenti—sono essenziali per progettare, mantenere e utilizzare il sistema informativo in modo efficace.

\section{Database Management System (DBMS)}

\subsection{Definizione e ruolo}
Un DBMS è un software specializzato che offre una serie di funzionalità critiche per la gestione dei dati. Consente di memorizzare dati in modo organizzato e persistente, assicurando che le informazioni rimangono disponibili nel tempo. Facilita il recupero efficiente dei dati tramite query ottimizzate e indici. Permette di aggiornare i dati mantenendo la coerenza e l'integrità secondo regole definite. Un DBMS fornisce anche un sofisticato controllo degli accessi, limitando chi può leggere o modificare determinati dati. Infine, protegge i dati da accessi non autorizzati attraverso meccanismi di autenticazione e crittografia.

\subsection{Vantaggi del DBMS}
\begin{tcolorbox}[colback=green!10, colframe=green!60, title=Vantaggi chiave]
\begin{itemize}
    \item \textbf{Centralizzazione}: un'unica fonte di verità
    \item \textbf{Efficienzaad accesso}: strutture dati ottimizzate
    \item \textbf{Sicurezza}: controllo degli accessi e crittografia
    \item \textbf{Affidabilità}: backup e recovery automatici
    \item \textbf{Integrità}: regole e vincoli garantiscono coerenza
    \item \textbf{Scalabilità}: gestisce grandi volumi di dati
    \item \textbf{Concorrenza}: accesso multiplo simultaneo
\end{itemize}
\end{tcolorbox}

\section{Le proprietà ACID}

\subsection{Proprietà ACID delle transazioni}
Una transazione è una sequenza di operazioni che costituisce un'unità di lavoro logica. Le proprietà ACID garantiscono l'affidabilità:

\begin{description}
    \item[\textbf{A - Atomicità}] La transazione è indivisibile: si esegue completamente o non si esegue per nulla (all-or-nothing).
    \item[\textbf{C - Coerenza}] Lo stato del database deve rimanere coerente prima e dopo la transazione.
    \item[\textbf{I - Isolamento}] Transazioni concorrenti non si interferiscono l'una con l'altra.
    \item[\textbf{D - Durabilità}] Una volta eseguita, una transazione persiste anche in caso di guasto.
\end{description}

\subsection{Atomicità}
Ogni transazione è atomica: non può rimanere nello stato intermedio.

\begin{tcolorbox}[colback=red!10, colframe=red!60, title=Attenzione: Problema senza atomicità]
Trasferimento di 100 euro tra due conti. Se il sistema crasha dopo il prelievo ma prima del deposito, il denaro è perso!
\end{tcolorbox}

\subsection{Coerenza}
Il database mantiene regole e vincoli di integrità.

\begin{lstlisting}[language=SQL, caption=Esempio: Vincolo di coerenza]
-- Il saldo non può mai essere negativo
ALTER TABLE conto ADD CONSTRAINT saldo_positivo
    CHECK (saldo >= 0);
\end{lstlisting}

\subsection{Isolamento}
Transazioni concorrenti non si vedono i reciproci effetti intermedi.

\begin{tcolorbox}[colback=orange!10, colframe=orange!60, title=Nota: Livelli di isolamento]
Existono diversi livelli di isolamento (READ UNCOMMITTED, READ COMMITTED, REPEATABLE READ, SERIALIZABLE) per bilanciare performance e sicurezza.
\end{tcolorbox}

\subsection{Durabilità}
I dati sono permanentemente memorizzati.

\begin{lstlisting}[language=SQL, caption=Esempio: Commit garantisce durabilità]
START TRANSACTION;
UPDATE conto SET saldo = saldo - 100 WHERE id = 1;
UPDATE conto SET saldo = saldo + 100 WHERE id = 2;
COMMIT; -- I dati sono ora permanenti
\end{lstlisting}

\section{Tipi di DBMS}

\subsection{DBMS Relazionali}
Organizzano i dati in tabelle con relazioni tra loro. Esempi: MySQL, PostgreSQL, Oracle, SQL Server.

\subsection{DBMS NoSQL}
Memorizzano dati in formato non relazionale. Esempi: MongoDB, Cassandra, Redis.

\subsection{DBMS NewSQL}
Combinano vantaggi relazionali e NoSQL. Esempi: CockroachDB, TiDB.

\section{Architettura di un DBMS}

\begin{figure}[h]
    \centering
    \begin{tikzpicture}[node distance=2cm]
        \node[draw, rectangle] (app) {Applicazioni};
        \node[draw, rectangle, below of=app] (query) {Query Processor};
        \node[draw, rectangle, below of=query] (optim) {Query Optimizer};
        \node[draw, rectangle, below of=optim] (exec) {Execution Engine};
        \node[draw, rectangle, below of=exec] (storage) {Storage Manager};
        \node[draw, rectangle, below of=storage] (disk) {Disco Fisico};

        \draw[->] (app) -- (query);
        \draw[->] (query) -- (optim);
        \draw[->] (optim) -- (exec);
        \draw[->] (exec) -- (storage);
        \draw[->] (storage) -- (disk);
    \end{tikzpicture}
    \caption{Architettura a livelli di un DBMS}
\end{figure}

\section*{Riepilogo concetti chiave}

\begin{tcolorbox}[colback=gray!10, colframe=black!60, title=Concetti fondamentali]
Un \textbf{DBMS} è lo strumento essenziale per centralizzare e gestire i dati di un'organizzazione in modo professionale. Le proprietà \textbf{ACID} sono il fondamento che garantisce transazioni affidabili e coerenti. L'\textbf{atomicità} assicura il principio del tutto-o-nulla, dove una transazione si completa interamente o non avviene affatto. La \textbf{coerenza} mantiene l'integrità dei vincoli durante tutte le operazioni. L'\textbf{isolamento} separa le transazioni concorrenti impedendo interferenze reciproche. Infine, la \textbf{durabilità} rende i dati permanenti e protetti anche di fronte a guasti imprevisti di sistema.
\end{tcolorbox}

\section*{Esercizi}

\begin{enumerate}
    \item Spiega con un esempio concreto la differenza tra data e informazione.

    \item Descrivi come le proprietà ACID garantiscono un trasferimento bancario sicuro tra due conti.

    \item Un'operazione di trasferimento fallisce a metà. Quale proprietà ACID evita che il denaro scompaia? Spiega.

    \item Elenca almeno tre vantaggi di un DBMS rispetto a un archivio basato su file di testo.

    \item Qual è la differenza tra DBMS relazionali e NoSQL? Fai un esempio di situazione dove sceglieresti l'uno o l'altro.
\end{enumerate}


% Capitolo 02: Modello Concettuale (Entity-Relationship)
\chapter{Modello Concettuale - Diagrammi ER}

\section*{Introduzione}
Il modello Entità-Relazione (ER) è uno strumento per rappresentare la struttura logica dei dati indipendentemente dall'implementazione fisica. Questo capitolo presenta i diagrammi ER, le entità, gli attributi e le relazioni con le loro cardinalità.

\section*{Obiettivi di apprendimento}
\begin{itemize}
    \item Comprendere i componenti del modello ER
    \item Disegnare entità e attributi
    \item Identificare e modellare relazioni tra entità
    \item Comprendere la cardinalità delle relazioni
    \item Applicare le chiavi primarie e esterne nel modello concettuale
    \item Tradurre requisiti aziendali in diagrammi ER
\end{itemize}

\section{Componenti del Modello ER}

\subsection{Entità}
Un'\textbf{entità} è un oggetto del mondo reale di interesse per il sistema informativo. Rappresenta una classe di oggetti con proprietà comuni.

\begin{tcolorbox}[colback=blue!10, colframe=blue!60, title=Esempio: Entità]
Entità: \textbf{Cliente}, \textbf{Prodotto}, \textbf{Ordine}, \textbf{Dipendente}

Ogni entità rappresenta un concetto del dominio aziendale.
\end{tcolorbox}

Nel diagramma ER, un'entità è rappresentata come un rettangolo.

\subsection{Attributi}
Un \textbf{attributo} è una proprietà di un'entità. Descrive caratteristiche specifiche dell'entità.

\begin{tcolorbox}[colback=blue!10, colframe=blue!60, title=Esempio: Attributi]
Entità \textbf{Cliente} ha attributi:
\begin{itemize}
    \item \texttt{idCliente} (identificativo univoco)
    \item \texttt{nome}
    \item \texttt{cognome}
    \item \texttt{email}
    \item \texttt{dataRegistrazione}
\end{itemize}
\end{tcolorbox}

\subsubsection{Tipi di attributi}
Gli attributi possono essere classificati in diverse categorie in base alle loro caratteristiche. Un attributo \textbf{semplice} non è scomponibile e rappresenta un'unità atomica di informazione, come un indirizzo email. Un attributo \textbf{composto}, al contrario, può essere scomposto in attributi più piccoli e semanticamente significativi; per esempio, un indirizzo può essere diviso in via, numero civico, città e CAP. Un attributo \textbf{univoco} o chiave identifica univocamente l'entità, come un identificativo di cliente. Un attributo \textbf{opzionale} può non avere un valore assegnato, come un numero di cellulare che potrebbe non essere disponibile per tutti i clienti. Infine, un attributo \textbf{multivalore} può assumere più valori per la stessa entità, come una raccolta di numeri di telefono per una ditta.

\subsection{Relazioni}
Una \textbf{relazione} (o associazione) descrive come due o più entità interagiscono tra loro.

\begin{tcolorbox}[colback=blue!10, colframe=blue!60, title=Esempio: Relazioni]
\begin{itemize}
    \item Un \textbf{Cliente} \textit{effettua} un \textbf{Ordine}
    \item Un \textbf{Ordine} \textit{contiene} \textbf{Prodotti}
    \item Un \textbf{Dipendente} \textit{lavora in} un \textbf{Dipartimento}
\end{itemize}
\end{tcolorbox}

\section{Cardinalità delle Relazioni}

La \textbf{cardinalità} specifica quanti elementi di un'entità possono essere correlati con elementi dell'altra entità.

\subsection{Tipi di cardinalità}

La cardinalità può assumere diverse forme a seconda della natura della relazione tra entità. Una relazione \textbf{1:1 (uno a uno)} si verifica quando ogni elemento di A è associato a esattamente un elemento di B, e viceversa; questo è il caso più restrittivo. Una relazione \textbf{1:N (uno a molti)} permette che ogni elemento di A sia associato a uno o più elementi di B, ma ogni elemento di B rimane associato a un solo elemento di A. Una relazione \textbf{N:M (molti a molti)} è la più flessibile, consentendo che ogni elemento di A sia associato a più elementi di B e viceversa. Oltre ai tipi principali, esistono notazioni aggiuntive per specificare l'obbligatorietà: una notazione \textbf{0..1} indica un'associazione opzionale (zero a uno), mentre \textbf{1..N} rappresenta un'associazione obbligatoria (da uno a molti).

\subsection{Esempio 1: Relazione 1:1}

\begin{figure}[h]
    \centering
    \begin{tikzpicture}
        % Entità
        \node[draw, rectangle, minimum width=2cm] (persona) {Persona};
        \node[draw, rectangle, right=3cm of persona, minimum width=2cm] (passaporto) {Passaporto};

        % Relazione
        \draw (persona.east) -- node[above] {possiede} (passaporto.west);

        % Cardinalità
        \node[above left] at (persona.east) {1};
        \node[above right] at (passaporto.west) {1};
    \end{tikzpicture}
    \caption{Una persona possiede esattamente un passaporto}
\end{figure}

Una persona può avere un solo passaporto e ogni passaporto appartiene a una sola persona.

\subsection{Esempio 2: Relazione 1:N}

\begin{figure}[h]
    \centering
    \begin{tikzpicture}
        % Entità
        \node[draw, rectangle, minimum width=2cm] (cliente) {Cliente};
        \node[draw, rectangle, right=3cm of cliente, minimum width=2cm] (ordine) {Ordine};

        % Relazione
        \draw (cliente.east) -- node[above] {effettua} (ordine.west);

        % Cardinalità
        \node[above left] at (cliente.east) {1};
        \node[above right] at (ordine.west) {N};
    \end{tikzpicture}
    \caption{Un cliente può effettuare molti ordini}
\end{figure}

Un cliente può fare molti ordini, ma ogni ordine è fatto da un solo cliente.

\subsection{Esempio 3: Relazione N:M}

\begin{figure}[h]
    \centering
    \begin{tikzpicture}
        % Entità
        \node[draw, rectangle, minimum width=2cm] (studente) {Studente};
        \node[draw, rectangle, right=3cm of studente, minimum width=2cm] (corso) {Corso};

        % Relazione
        \draw (studente.east) -- node[above] {segue} (corso.west);

        % Cardinalità
        \node[above left] at (studente.east) {N};
        \node[above right] at (corso.west) {M};
    \end{tikzpicture}
    \caption{Uno studente segue molti corsi e un corso è seguito da molti studenti}
\end{figure}

Uno studente può seguire più corsi e ogni corso può avere più studenti.

\section{Diagramma ER Completo: Sistema di Vendita Online}

\begin{figure}[h]
    \centering
    \begin{tikzpicture}[scale=1, node distance=3cm]
        % Entità Cliente
        \node[draw, rectangle, minimum width=2.5cm, minimum height=1.5cm] (cliente) {
            \textbf{Cliente}\\
            \hline
            idCliente\\
            nome\\
            cognome\\
            email
        };

        % Entità Ordine
        \node[draw, rectangle, right=of cliente, minimum width=2.5cm, minimum height=1.5cm] (ordine) {
            \textbf{Ordine}\\
            \hline
            idOrdine\\
            dataOrdine\\
            totale
        };

        % Entità Prodotto
        \node[draw, rectangle, below=of ordine, minimum width=2.5cm, minimum height=1.5cm] (prodotto) {
            \textbf{Prodotto}\\
            \hline
            idProdotto\\
            nome\\
            prezzo\\
            categoria
        };

        % Relazioni
        \draw (cliente.east) -- node[above] {effettua\\1:N} (ordine.west);
        \draw (ordine.south) -- node[right] {contiene\\N:M} (prodotto.north);
    \end{tikzpicture}
    \caption{Diagramma ER semplificato per un e-commerce}
\end{figure}

\section{Chiavi nel Modello ER}

\subsection{Chiave Primaria}
La \textbf{chiave primaria} è un attributo (o insieme di attributi) che identifica univocamente ogni istanza dell'entità. Non può essere nulla e deve essere univoca.

\begin{tcolorbox}[colback=green!10, colframe=green!60, title=Esempio: Chiave Primaria]
Entità \textbf{Cliente}:
\begin{itemize}
    \item \textbf{idCliente} è la chiave primaria (numero univoco assegnato a ogni cliente)
    \item Anche \textbf{email} potrebbe essere chiave primaria (ogni cliente ha un'email unica)
\end{itemize}
\end{tcolorbox}

\subsection{Chiave Esterna}
Una \textbf{chiave esterna} è un attributo che riferisce la chiave primaria di un'altra entità, creando un collegamento.

\begin{tcolorbox}[colback=green!10, colframe=green!60, title=Esempio: Chiave Esterna]
Entità \textbf{Ordine}:
\begin{itemize}
    \item \textbf{idOrdine} è la chiave primaria
    \item \textbf{idCliente} è una chiave esterna (riferisce la chiave primaria di Cliente)
\end{itemize}
\end{tcolorbox}

\section*{Riepilogo concetti chiave}

\begin{tcolorbox}[colback=gray!10, colframe=black!60, title=Concetti fondamentali]
Nel modello entità-relazione, un'\textbf{entità} rappresenta una classe di oggetti del dominio reale di interesse. Ogni entità possiede \textbf{attributi} che descrivono specifiche proprietà e caratteristiche. Le entità sono collegate tra loro attraverso \textbf{relazioni} che definiscono le associazioni significative. La \textbf{cardinalità} specifica il numero e la natura delle associazioni possibili tra entità, espresse come 1:1, 1:N, o N:M. Per identificare univocamente ogni istanza di entità, si utilizza una \textbf{chiave primaria}, mentre una \textbf{chiave esterna} crea il collegamento tra entità diverse, mantenendo la coerenza strutturale del modello.
\end{tcolorbox}

\section*{Esercizi}

\begin{enumerate}
    \item Disegna un diagramma ER per una biblioteca con le entità: Libro, Autore, Membro, Prestito. Includi attributi appropriati e cardinalità.

    \item Spiega la differenza tra cardinalità 1:N e N:M con due esempi concreti diversi da quelli del capitolo.

    \item Un'università ha Studenti, Corsi e Professori. Quali sono le relazioni tra questi? Quale cardinalità?

    \item Identifica la chiave primaria e le chiavi esterne nel diagramma dell'e-commerce presentato nel capitolo.

    \item Trasforma il seguente requisito aziendale in un diagramma ER:
    ``Una società ha Dipartimenti. Ogni Dipartimento ha più Dipendenti. Un Dipendente lavora in un solo Dipartimento. Un Dipendente gestisce Progetti. Un Progetto può essere gestito da un solo Dipendente. Un Progetto coinvolge più Dipendenti.''
\end{enumerate}


% Capitolo 03: Modello Logico Relazionale
\chapter{Modello Logico - Modello Relazionale}

\section*{Introduzione}
Il modello logico traduce il modello concettuale (diagrammi ER) in un modello implementabile nei DBMS. Il modello relazionale organizza i dati in tabelle (relazioni) con colonne (attributi) e righe (tuple). Questo capitolo presenta la struttura del modello relazionale, le chiavi e i vincoli di integrità.

\section*{Obiettivi di apprendimento}
\begin{itemize}
    \item Comprendere la struttura del modello relazionale
    \item Tradurre diagrammi ER in tabelle relazionali
    \item Definire e utilizzare chiavi primarie e esterne
    \item Comprendere i vincoli di integrità referenziale
    \item Applicare le regole di trasformazione ER-relazionale
    \item Progettare schemi logici corretti
\end{itemize}

\section{Il Modello Relazionale}

\subsection{Componenti fondamentali}
Il modello relazionale è basato su tre concetti chiave:

\begin{description}
    \item[\textbf{Relazione (Tabella)}] Una collezione di tuple (righe) con lo stesso insieme di attributi (colonne).
    \item[\textbf{Attributo (Colonna)}] Una proprietà della relazione con un nome e un dominio (tipo di dato).
    \item[\textbf{Tupla (Riga)}] Un record che contiene un valore per ogni attributo.
\end{description}

\begin{tcolorbox}[colback=blue!10, colframe=blue!60, title=Esempio: Tabella Cliente]
\begin{center}
\begin{tabular}{|c|c|c|c|}
\hline
\textbf{idCliente} & \textbf{nome} & \textbf{cognome} & \textbf{email} \\
\hline
1 & Luigi & Rossi & luigi@email.com \\
2 & Maria & Bianchi & maria@email.com \\
3 & Giovanni & Verdi & giovanni@email.com \\
\hline
\end{tabular}

Ogni riga è una tupla, ogni colonna è un attributo.
\end{center}
\end{tcolorbox}

\subsection{Dominio di un attributo}
Ogni attributo ha un \textbf{dominio}, cioè l'insieme di valori che può assumere.

\begin{tcolorbox}[colback=green!10, colframe=green!60, title=Esempio: Domini]
\begin{itemize}
    \item \texttt{idCliente}: dominio = interi positivi (1, 2, 3, \ldots)
    \item \texttt{nome}: dominio = stringhe di caratteri (fino a 100 caratteri)
    \item \texttt{dataIscrizione}: dominio = date (formato YYYY-MM-DD)
    \item \texttt{saldo}: dominio = numeri decimali positivi
\end{itemize}
\end{tcolorbox}

\section{Chiavi nel Modello Relazionale}

\subsection{Chiave Primaria}
Una \textbf{chiave primaria} è un attributo (o insieme di attributi) che:
\begin{itemize}
    \item Identifica univocamente ogni tupla della relazione
    \item Non può contenere valori NULL
    \item Non può avere duplicati
\end{itemize}

\begin{lstlisting}[language=SQL, caption=Dichiarazione di chiave primaria]
CREATE TABLE cliente (
    idCliente INT PRIMARY KEY,
    nome VARCHAR(100) NOT NULL,
    cognome VARCHAR(100) NOT NULL,
    email VARCHAR(100) UNIQUE
);
\end{lstlisting}

\subsection{Chiave Candidata}
Una \textbf{chiave candidata} è un attributo che potrebbe essere chiave primaria. Una relazione può avere più chiavi candidate, ma solo una viene scelta come primaria.

\begin{tcolorbox}[colback=orange!10, colframe=orange!60, title=Nota: Chiavi candidate}
Tabella Cliente:
\begin{itemize}
    \item \texttt{idCliente}: chiave candidata (univoca, non nulla)
    \item \texttt{email}: chiave candidata (univoca, non nulla)
    \item Sceglieremo \texttt{idCliente} come primaria
\end{itemize}
\end{tcolorbox}

\subsection{Chiave Esterna}
Una \textbf{chiave esterna} è un attributo (o insieme di attributi) che referenzia la chiave primaria di un'altra relazione. Crea il collegamento tra tabelle.

\begin{lstlisting}[language=SQL, caption=Dichiarazione di chiave esterna]
CREATE TABLE ordine (
    idOrdine INT PRIMARY KEY,
    dataOrdine DATE NOT NULL,
    idCliente INT NOT NULL,
    FOREIGN KEY (idCliente) REFERENCES cliente(idCliente)
);
\end{lstlisting}

\subsection{Chiave Composta}
Una chiave può essere formata da più attributi.

\begin{lstlisting}[language=SQL, caption=Chiave composta come chiave primaria]
CREATE TABLE ordine_prodotto (
    idOrdine INT NOT NULL,
    idProdotto INT NOT NULL,
    quantita INT NOT NULL,
    PRIMARY KEY (idOrdine, idProdotto),
    FOREIGN KEY (idOrdine) REFERENCES ordine(idOrdine),
    FOREIGN KEY (idProdotto) REFERENCES prodotto(idProdotto)
);
\end{lstlisting}

\section{Vincoli di Integrità}

\subsection{Vincolo di dominio}
Ogni valore di un attributo deve appartenere al dominio definito.

\begin{lstlisting}[language=SQL, caption=Vincoli di dominio]
CREATE TABLE prodotto (
    idProdotto INT PRIMARY KEY,
    nome VARCHAR(200) NOT NULL,      -- Non null
    prezzo DECIMAL(8, 2) NOT NULL,   -- Numero con 2 decimali
    categoria ENUM('Elettronica', 'Libri', 'Abbigliamento'),
    quantitaDisponibile INT CHECK (quantitaDisponibile >= 0)
);
\end{lstlisting}

\subsection{Vincolo di unicità}
Un attributo (o insieme di attributi) può avere solo valori unici (o nulli).

\begin{lstlisting}[language=SQL, caption=Vincolo di unicità]
CREATE TABLE utente (
    idUtente INT PRIMARY KEY,
    username VARCHAR(50) UNIQUE NOT NULL,
    email VARCHAR(100) UNIQUE NOT NULL,
    dataRegistrazione DATE DEFAULT CURDATE()
);
\end{lstlisting}

\subsection{Vincolo di chiave primaria}
Garantisce l'univocità e l'obbligatorietà (NOT NULL).

\subsection{Vincolo di chiave esterna e integrità referenziale}
La chiave esterna assicura che ogni valore riferisca effettivamente un'entità in un'altra relazione.

\begin{tcolorbox}[colback=red!10, colframe=red!60, title=Attenzione: Violazione di integrità referenziale]
Se in ordine abbiamo \texttt{idCliente = 5}, questo \texttt{idCliente} DEVE esistere in cliente. Altrimenti il DBMS rifiuta l'inserimento.
\end{tcolorbox}

\begin{lstlisting}[language=SQL, caption=Azioni su integrità referenziale]
CREATE TABLE ordine (
    idOrdine INT PRIMARY KEY,
    idCliente INT NOT NULL,
    FOREIGN KEY (idCliente) REFERENCES cliente(idCliente)
        ON DELETE CASCADE      -- Elimina ordini se cliente è eliminato
        ON UPDATE CASCADE      -- Aggiorna idCliente se cambia in cliente
);
\end{lstlisting}

\section{Trasformazione da ER a Modello Relazionale}

\subsection{Regola 1: Entità singole}
Ogni entità diventa una tabella. Gli attributi della tabella sono i rispettivi attributi dell'entità. La chiave primaria dell'entità diventa chiave primaria della tabella.

\begin{tcolorbox}[colback=blue!10, colframe=blue!60, title=Esempio: Entità Cliente]
\textbf{ER}: Entità Cliente con attributi (idCliente, nome, cognome, email)

\textbf{Relazionale}:
\begin{lstlisting}[language=SQL]
CREATE TABLE cliente (
    idCliente INT PRIMARY KEY,
    nome VARCHAR(100) NOT NULL,
    cognome VARCHAR(100) NOT NULL,
    email VARCHAR(100) UNIQUE
);
\end{lstlisting}
\end{tcolorbox}

\subsection{Regola 2: Relazione 1:N}
Il lato N contiene una chiave esterna che referenzia la chiave primaria del lato 1.

\begin{tcolorbox}[colback=blue!10, colframe=blue!60, title=Esempio: Cliente (1) - Ordine (N)}
\textbf{ER}: Cliente effettua Ordine (1:N)

\textbf{Relazionale}:
\begin{lstlisting}[language=SQL]
CREATE TABLE cliente (
    idCliente INT PRIMARY KEY,
    nome VARCHAR(100)
);

CREATE TABLE ordine (
    idOrdine INT PRIMARY KEY,
    dataOrdine DATE NOT NULL,
    idCliente INT NOT NULL,
    FOREIGN KEY (idCliente) REFERENCES cliente(idCliente)
);
\end{lstlisting}
\end{tcolorbox}

\subsection{Regola 3: Relazione 1:1}
Esiste una chiave esterna in una delle due tabelle. Generalmente nel lato con partecipazione opzionale.

\begin{tcolorbox}[colback=blue!10, colframe=blue!60, title=Esempio: Dipendente (1) - Ufficio (1)]
\textbf{ER}: Dipendente lavora in Ufficio (1:1)

\textbf{Relazionale}:
\begin{lstlisting}[language=SQL]
CREATE TABLE dipendente (
    idDipendente INT PRIMARY KEY,
    nome VARCHAR(100),
    idUfficio INT UNIQUE,
    FOREIGN KEY (idUfficio) REFERENCES ufficio(idUfficio)
);

CREATE TABLE ufficio (
    idUfficio INT PRIMARY KEY,
    posizione VARCHAR(100)
);
\end{lstlisting}
\end{tcolorbox}

\subsection{Regola 4: Relazione N:M}
Si crea una \textbf{tabella di giunzione} (o tabella di associazione) con chiavi esterne che referenziano entrambe le entità. La chiave primaria è la combinazione delle due chiavi esterne.

\begin{tcolorbox}[colback=blue!10, colframe=blue!60, title=Esempio: Studente (N) - Corso (M)]
\textbf{ER}: Studente segue Corso (N:M)

\textbf{Relazionale}:
\begin{lstlisting}[language=SQL]
CREATE TABLE studente (
    idStudente INT PRIMARY KEY,
    nome VARCHAR(100)
);

CREATE TABLE corso (
    idCorso INT PRIMARY KEY,
    titolo VARCHAR(100)
);

-- Tabella di giunzione
CREATE TABLE iscrizione (
    idStudente INT NOT NULL,
    idCorso INT NOT NULL,
    dataIscrizione DATE DEFAULT CURDATE(),
    PRIMARY KEY (idStudente, idCorso),
    FOREIGN KEY (idStudente) REFERENCES studente(idStudente),
    FOREIGN KEY (idCorso) REFERENCES corso(idCorso)
);
\end{lstlisting}
\end{tcolorbox}

\section{Schema Logico Completo: E-commerce}

\begin{figure}[h]
    \centering
    \begin{tikzpicture}[scale=0.8, node distance=4cm]
        % Cliente
        \node[draw, rectangle, minimum width=3cm, minimum height=2cm] (cliente) {
            \textbf{cliente}\\
            \hline
            idCliente (PK)\\
            nome\\
            cognome\\
            email (UNIQUE)
        };

        % Ordine
        \node[draw, rectangle, right=of cliente, minimum width=3cm, minimum height=2cm] (ordine) {
            \textbf{ordine}\\
            \hline
            idOrdine (PK)\\
            dataOrdine\\
            totale\\
            idCliente (FK)
        };

        % Prodotto
        \node[draw, rectangle, below=of ordine, minimum width=3cm, minimum height=2cm] (prodotto) {
            \textbf{prodotto}\\
            \hline
            idProdotto (PK)\\
            nome\\
            prezzo\\
            categoria
        };

        % Ordine_Prodotto
        \node[draw, rectangle, left=of prodotto, minimum width=3cm, minimum height=2cm] (ord_prod) {
            \textbf{ordine\_prodotto}\\
            \hline
            idOrdine (FK, PK)\\
            idProdotto (FK, PK)\\
            quantita
        };

        % Relazioni
        \draw[->] (cliente.east) -- (ordine.west);
        \draw[->] (ordine.south) -- (ord_prod.north);
        \draw[->] (prodotto.west) -- (ord_prod.east);
    \end{tikzpicture}
    \caption{Schema relazionale per un e-commerce (PK=Primary Key, FK=Foreign Key)}
\end{figure}

\section*{Riepilogo concetti chiave}

\begin{tcolorbox}[colback=gray!10, colframe=black!60, title=Concetti fondamentali]
\begin{itemize}
    \item Una \textbf{relazione} è una tabella con attributi e tuple
    \item La \textbf{chiave primaria} identifica univocamente ogni tupla
    \item La \textbf{chiave esterna} collega tabelle diverse mantenendo l'integrità referenziale
    \item I \textbf{vincoli di integrità} garantiscono la coerenza dei dati
    \item Le relazioni N:M richiedono una \textbf{tabella di giunzione}
    \item Ogni diagramma ER viene tradotto sistematicamente in tabelle relazionali
\end{itemize}
\end{tcolorbox}

\section*{Esercizi}

\begin{enumerate}
    \item Crea lo schema logico (CREATE TABLE) per il diagramma ER della biblioteca dell'esercizio del capitolo precedente.

    \item Quali azioni dovrebbero accadere se un cliente viene eliminato da una tabella cliente e ha più ordini? Spiega le opzioni ON DELETE.

    \item Progetta uno schema relazionale per un sistema ospedaliero con: Pazienti, Medici, Appuntamenti, Ospedali, Reparti. Includi vincoli e cardinalità.

    \item Identifica chiavi primarie, chiavi candidate e chiavi esterne nel seguente DDL:
    \begin{lstlisting}[language=SQL]
CREATE TABLE impiegato (
    matricola INT PRIMARY KEY,
    nome VARCHAR(100),
    dipartimento VARCHAR(50),
    stipendio DECIMAL(10, 2)
);
CREATE TABLE progetto (
    codice INT PRIMARY KEY,
    titolo VARCHAR(100),
    responsabile INT,
    FOREIGN KEY (responsabile) REFERENCES impiegato(matricola)
);
    \end{lstlisting}

    \item Trasforma il seguente diagramma ER in schema relazionale: Azienda ha Sedi. Una Sede ha più Dipendenti. Un Dipendente appartiene a una Sede. Un Dipendente gestisce Progetti. Un Progetto è gestito da un Dipendente. Un Progetto ha più Risorse. Una Risorsa è utilizzata da più Progetti.
\end{enumerate}


% Capitolo 04: Normalizzazione
\chapter{Normalizzazione del Database}

\section*{Introduzione}
La normalizzazione è un processo sistematico per organizzare i dati eliminando ridondanze e anomalie. Attraverso forme normali progressive (1NF, 2NF, 3NF, BCNF), garantiamo un design del database efficiente, coerente e facile da mantenere.

\section*{Obiettivi di apprendimento}
\begin{itemize}
    \item Comprendere il concetto di dipendenza funzionale
    \item Applicare le regole della prima forma normale (1NF)
    \item Applicare le regole della seconda forma normale (2NF)
    \item Applicare le regole della terza forma normale (3NF)
    \item Comprendere la forma normale di Boyce-Codd (BCNF)
    \item Identificare e risolvere anomalie di inserimento, aggiornamento e cancellazione
\end{itemize}

\section{Dipendenze Funzionali}

\subsection{Definizione}
Una \textbf{dipendenza funzionale} (DF) X -> Y fra due insiemi di attributi X e Y di una relazione R significa che, se due tuple hanno gli stessi valori per gli attributi in X, allora devono avere gli stessi valori per gli attributi in Y.

\begin{tcolorbox}[colback=blue!10, colframe=blue!60, title=Esempio: Dipendenza funzionale]
Relazione: Studente(idStudente, nome, email, idCorso, nomeCorso)

Dipendenze funzionali:
\begin{itemize}
    \item \texttt{idStudente} -> \texttt{nome, email}: dato uno studente, c'è un solo nome e email
    \item \texttt{idCorso} -> \texttt{nomeCorso}: dato un corso, c'è un solo nome
\end{itemize}
\end{tcolorbox}

\subsection{Chiave primaria e dipendenze}
Se un attributo è parte della chiave primaria, determina funzionalmente tutti gli altri attributi della relazione.

\section{Anomalie nei Database non normalizzati}

\subsection{Anomalia di inserimento}
Non è possibile inserire dati fino a quando non sono disponibili tutti i dati della chiave primaria.

\begin{tcolorbox}[colback=red!10, colframe=red!60, title=Esempio: Anomalia di inserimento]
Tabella: Corso(idCorso, nomeCorsso, idDocente, nomeDocente)

Non posso inserire un docente che non insegna ancora nessun corso perché \texttt{idCorso} è parte della chiave primaria.
\end{tcolorbox}

\subsection{Anomalia di aggiornamento}
Aggiornare un valore può richiedere di aggiornare più tuple, rischiando inconsistenze.

\begin{tcolorbox}[colback=red!10, colframe=red!60, title=Esempio: Anomalia di aggiornamento]
Tabella: Studente(idStudente, nome, idFacoltà, nomeFacoltà)

Se cambio il nome della Facoltà da ``Ingegneria'' a ``Ingegneria e Architettura'', devo aggiornare tutte le righe degli studenti di quella facoltà. Se dimentico una riga, il database diventa inconsistente.
\end{tcolorbox}

\subsection{Anomalia di cancellazione}
Cancellare un dato può comportare la perdita di informazioni non correlate.

\begin{tcolorbox}[colback=red!10, colframe=red!60, title=Esempio: Anomalia di cancellazione]
Tabella: Fornitore_Prodotto(idFornitore, nomeFornitore, idProdotto, nomeProdotto)

Se cancello l'ultimo prodotto fornito da un fornitore, perdo anche i dati del fornitore dal database.
\end{tcolorbox}

\section{Prima Forma Normale (1NF)}

La forma normale più basilare: \textbf{una relazione è in 1NF se tutti gli attributi contengono solo valori atomici} (non divisibili).

\subsection{Requisiti}
\begin{itemize}
    \item Nessun attributo può contenere liste o insiemi di valori
    \item Nessun attributo ripetuto
    \item Ogni riga deve avere lo stesso numero di colonne
    \item Nessun gruppo ripetuto
\end{itemize}

\subsection{Esempio di violazione 1NF}

\begin{tcolorbox}[colback=red!10, colframe=red!60, title=Esempio: Non è in 1NF}
\begin{center}
\begin{tabular}{|c|c|c|}
\hline
\textbf{idStudente} & \textbf{nome} & \textbf{corsi} \\
\hline
1 & Mario & Matematica, Fisica, Chimica \\
2 & Luigi & Matematica, Biologia \\
\hline
\end{tabular}

L'attributo \texttt{corsi} contiene valori non atomici (liste).
\end{center}
\end{tcolorbox}

\subsection{Correzione: Applicare 1NF}

\begin{tcolorbox}[colback=green!10, colframe=green!60, title=Esempio: In 1NF}
\begin{center}
\begin{tabular}{|c|c|c|}
\hline
\textbf{idStudente} & \textbf{nome} & \textbf{corso} \\
\hline
1 & Mario & Matematica \\
1 & Mario & Fisica \\
1 & Mario & Chimica \\
2 & Luigi & Matematica \\
2 & Luigi & Biologia \\
\hline
\end{tabular}

Oppure creare una tabella separata per gli inscritti ai corsi (relazione N:M).
\end{center}
\end{tcolorbox}

\begin{lstlisting}[language=SQL, caption=DDL in 1NF con tabella di giunzione]
CREATE TABLE studente (
    idStudente INT PRIMARY KEY,
    nome VARCHAR(100)
);

CREATE TABLE corso (
    idCorso INT PRIMARY KEY,
    nomeCors VARCHAR(100)
);

CREATE TABLE iscrizione (
    idStudente INT NOT NULL,
    idCorso INT NOT NULL,
    PRIMARY KEY (idStudente, idCorso),
    FOREIGN KEY (idStudente) REFERENCES studente(idStudente),
    FOREIGN KEY (idCorso) REFERENCES corso(idCorso)
);
\end{lstlisting}

\section{Seconda Forma Normale (2NF)}

Una relazione è in 2NF se:
\begin{enumerate}
    \item È in 1NF
    \item \textbf{Ogni attributo non-chiave è dipendente dalla chiave primaria completa}, non solo da parte di essa.
\end{enumerate}

\textbf{Problema}: Dipendenza parziale. Un attributo non-chiave dipende solo da una parte della chiave primaria.

\subsection{Esempio di violazione 2NF}

\begin{tcolorbox}[colback=red!10, colframe=red!60, title=Esempio: Non è in 2NF}
Tabella: Iscrizione(idStudente, idCorso, nomeCors, voto)

\textbf{Chiave primaria}: (idStudente, idCorso)

\textbf{Problema}: \texttt{nomeCors} dipende solo da \texttt{idCorso}, non dalla chiave completa. Questa è una dipendenza parziale.
\end{tcolorbox}

\subsection{Correzione: Applicare 2NF}

\begin{lstlisting}[language=SQL, caption=DDL in 2NF]
CREATE TABLE studente (
    idStudente INT PRIMARY KEY,
    nome VARCHAR(100)
);

CREATE TABLE corso (
    idCorso INT PRIMARY KEY,
    nomeCors VARCHAR(100)
);

-- Ora in 2NF: nomeCors è in tabella corso, non in iscrizione
CREATE TABLE iscrizione (
    idStudente INT NOT NULL,
    idCorso INT NOT NULL,
    voto INT,
    PRIMARY KEY (idStudente, idCorso),
    FOREIGN KEY (idStudente) REFERENCES studente(idStudente),
    FOREIGN KEY (idCorso) REFERENCES corso(idCorso)
);
\end{lstlisting}

\section{Terza Forma Normale (3NF)}

Una relazione è in 3NF se:
\begin{enumerate}
    \item È in 2NF
    \item \textbf{Nessun attributo non-chiave dipende da un altro attributo non-chiave}.
\end{enumerate}

\textbf{Problema}: Dipendenza transitiva. Un attributo non-chiave dipende funzionalmente da un altro attributo non-chiave.

\subsection{Esempio di violazione 3NF}

\begin{tcolorbox}[colback=red!10, colframe=red!60, title=Esempio: Non è in 3NF}
Tabella: Ordine(idOrdine, dataOrdine, idCliente, nomeCliente, indirizzo)

\textbf{Dipendenze funzionali}:
\begin{itemize}
    \item \texttt{idOrdine} -> \texttt{dataOrdine, idCliente, nomeCliente, indirizzo}
    \item \texttt{idCliente} -> \texttt{nomeCliente, indirizzo} (dipendenza transitiva!)
\end{itemize}

\texttt{nomeCliente} e \texttt{indirizzo} dipendono da \texttt{idCliente}, non direttamente da \texttt{idOrdine}.
\end{tcolorbox}

\subsection{Correzione: Applicare 3NF}

\begin{lstlisting}[language=SQL, caption=DDL in 3NF]
CREATE TABLE cliente (
    idCliente INT PRIMARY KEY,
    nomeCliente VARCHAR(100),
    indirizzo VARCHAR(200)
);

CREATE TABLE ordine (
    idOrdine INT PRIMARY KEY,
    dataOrdine DATE,
    idCliente INT NOT NULL,
    FOREIGN KEY (idCliente) REFERENCES cliente(idCliente)
);
\end{lstlisting}

\section{Forma Normale di Boyce-Codd (BCNF)}

Una relazione è in BCNF se:
\begin{itemize}
    \item Per ogni dipendenza funzionale X -> Y, X è una chiave candidata.
\end{itemize}

BCNF è una forma più stringente di 3NF. La maggior parte dei database in 3NF è già in BCNF.

\subsection{Esempio di violazione BCNF}

\begin{tcolorbox}[colback=red!10, colframe=red!60, title=Esempio: In 3NF ma non in BCNF}
Tabella: Professore_Corso(idProfessore, idCorso, orario)

\textbf{Chiave primaria}: (idProfessore, idCorso)

\textbf{Dipendenza funzionale}: (idProfessore, idCorso) -> orario (OK)

Ma se esiste anche: idCorso -> orario (tutti i corsi hanno un orario fisso)

Allora idCorso determina orario, ma idCorso non è una chiave candidata. Violazione BCNF!
\end{tcolorbox}

\section{Procedura di Normalizzazione}

\begin{tcolorbox}[colback=blue!10, colframe=blue!60, title=Passi di normalizzazione]
\begin{enumerate}
    \item \textbf{Identifica dipendenze funzionali}: Quali attributi determinano quali altri.
    \item \textbf{Applica 1NF}: Assicura valori atomici.
    \item \textbf{Applica 2NF}: Elimina dipendenze parziali.
    \item \textbf{Applica 3NF}: Elimina dipendenze transitive.
    \item \textbf{Applica BCNF}: Se necessario, per relazioni più complesse.
\end{enumerate}
\end{tcolorbox}

\section*{Riepilogo concetti chiave}

\begin{tcolorbox}[colback=gray!10, colframe=black!60, title=Concetti fondamentali]
\begin{itemize}
    \item Una \textbf{dipendenza funzionale} X -> Y significa che X determina Y
    \item La \textbf{1NF} richiede valori atomici (no liste)
    \item La \textbf{2NF} elimina dipendenze parziali dalla chiave primaria
    \item La \textbf{3NF} elimina dipendenze transitive tra attributi non-chiave
    \item La \textbf{BCNF} è una versione più stringente di 3NF
    \item La normalizzazione riduce ridondanze e anomalie
\end{itemize}
\end{tcolorbox}

\section*{Esercizi}

\begin{enumerate}
    \item Identifica tutte le dipendenze funzionali nella seguente tabella:
    Libro(idLibro, titolo, autore, idAutore, nazionaleAutore, genere, prezzo)

    \item La seguente tabella è in 1NF? Se no, normalizza:
    Impiegato(matricola, nome, skills)  [skills contiene ``Java, Python, SQL'']

    \item La seguente tabella è in 2NF? Se no, normalizza:
    Voto(idStudente, nomeStudente, idCorso, nomeCorso, voto)

    \item La seguente tabella è in 3NF? Se no, normalizza:
    Ospedale(idOspedale, nomeOspedale, città, idRepartimento, nomeRepartimento, idMedico, nomeMedico)

    \item Progetta uno schema completamente normalizzato (3NF) per un sistema di prenotazioni alberghiere con: Hotel, Stanze, Clienti, Prenotazioni. Includi vincoli appropriati.
\end{enumerate}


% Capitolo 05: Modello Fisico e Implementazione
\chapter{Modello Fisico e Ottimizzazione}

\section*{Introduzione}
Il modello fisico descrive come i dati sono effettivamente memorizzati e organizzati nel disco fisico. Include indici, viste materializzate, partizionamento e altre strategie di ottimizzazione per migliorare performance e scalabilità.

\section*{Obiettivi di apprendimento}
\begin{itemize}
    \item Comprendere come i dati sono memorizzati nel disco
    \item Progettare e utilizzare indici efficaci
    \item Comprendere le viste materializzate e il caching
    \item Applicare strategie di partizionamento
    \item Ottimizzare query e accessi ai dati
    \item Bilanciare velocità di lettura e velocità di scrittura
\end{itemize}

\section{Memorizzazione dei Dati}

\subsection{Struttura di memorizzazione}
I dati vengono memorizzati in una gerarchia ben organizzata all'interno del disco. Un \textbf{blocco} rappresenta l'unità minima di trasferimento tra disco e memoria, tipicamente dimensionato tra 4 e 16 KB per ottimizzare le operazioni di I/O. Una \textbf{pagina} è il corrispondente concetto logico all'interno del DBMS, fungendo da container strutturato dei dati. Un \textbf{record} è la singola tupla della tabella, contenente una riga completa di dati. Infine, un \textbf{campo} rappresenta il singolo attributo di un record, cioè il più piccolo elemento di dato identificabile all'interno della struttura.

\begin{figure}[h]
    \centering
    \begin{tikzpicture}[scale=1]
        \draw[thick] (0, 0) rectangle (8, 5);
        \node at (4, 4.7) {\textbf{Disco Fisico}};

        \draw (0.5, 0.5) rectangle (1.8, 2);
        \node[align=center, font=\small] at (1.15, 1.25) {Pagina\\1};

        \draw (2.2, 0.5) rectangle (3.5, 2);
        \node[align=center, font=\small] at (2.85, 1.25) {Pagina\\2};

        \draw (3.9, 0.5) rectangle (5.2, 2);
        \node[align=center, font=\small] at (4.55, 1.25) {Pagina\\3};

        \draw (0.5, 2.3) rectangle (5.2, 4.5);
        \node[align=center, font=\small] at (2.85, 3.4) {Contenuto Pagina 1: Record | Record | Record | ...};
    \end{tikzpicture}
    \caption{Struttura di memorizzazione su disco}
\end{figure}

\subsection{Tipi di accesso}
Esistono due modelli fondamentali di accesso ai dati che differiscono significativamente in termini di efficienza. L'\textbf{accesso sequenziale} implica leggere i record uno dopo l'altro in sequenza, scorrendo l'intera struttura dati. Questo metodo è lento quando si ricerca un record specifico, poiché potrebbe richiedere di esaminare un numero enorme di record prima di trovare quello desiderato. L'\textbf{accesso casuale} consente invece di accedere direttamente a un record specifico senza doverne leggere altri, risultando molto più veloce; tuttavia, questa velocità richiede la disponibilità di strutture ausiliarie come gli indici che guidano il DBMS direttamente al dato cercato.

\section{Indici}

Un \textbf{indice} è una struttura dati che accelera il recupero di record da una tabella. Funziona come un indice di un libro: invece di leggere tutte le pagine, consulti l'indice per trovare la pagina che cerchi.

\subsection{Vantaggi degli indici}
Gli indici offrono benefici significativi per le operazioni di lettura. Accelerano notevolmente le ricerche e i filtri espressi nelle clausole WHERE, permettendo al DBMS di localizzare rapidamente i record corrispondenti. Accelerano anche gli ordinamenti (ORDER BY), poiché gli indici mantengono i dati in ordine predeterminato. Gli indici ottimizzano inoltre le operazioni di join tra tabelle, facilitando l'abbinamento veloce di record correlati. Infine, supportano query range dove è necessario recuperare intervalli di valori, come nella clausola WHERE prezzo BETWEEN 10 AND 100.

\subsection{Svantaggi degli indici}
Nonostante i vantaggi, gli indici comportano anche costi significativi. Occupano spazio aggiuntivo considerevole su disco, aumentando i requisiti di storage. Più criticamente, ralentano le operazioni di INSERT, UPDATE e DELETE, poiché l'indice deve essere aggiornato ogni volta che i dati sottostanti cambiano. Inoltre, gli indici richiedono manutenzione periodica per mantenere l'efficienza, consumando risorse computazionali che potrebbero altrimenti essere dedicate ad altre operazioni.

\subsection{Tipi di indici}

\subsubsection{Indice primario}
Un indice su una chiave primaria. È automaticamente creato quando si definisce una PRIMARY KEY.

\begin{lstlisting}[language=SQL, caption=Indice primario (creato automaticamente)]
CREATE TABLE cliente (
    idCliente INT PRIMARY KEY,  -- Indice primario automatico
    nome VARCHAR(100)
);
\end{lstlisting}

\subsubsection{Indice secondario}
Un indice su attributi non-chiave per accelerare ricerche frequenti.

\begin{lstlisting}[language=SQL, caption=Creare un indice secondario]
CREATE TABLE cliente (
    idCliente INT PRIMARY KEY,
    nome VARCHAR(100),
    email VARCHAR(100),
    città VARCHAR(50)
);

-- Indici su campi frequentemente cercati
CREATE INDEX idx_email ON cliente(email);
CREATE INDEX idx_città ON cliente(città);
\end{lstlisting}

\subsubsection{Indice composito}
Un indice su più attributi, utile per query che filtrano su più colonne.

\begin{lstlisting}[language=SQL, caption=Indice composito]
CREATE TABLE ordine (
    idOrdine INT PRIMARY KEY,
    dataOrdine DATE,
    idCliente INT,
    stato VARCHAR(20)
);

-- Indice su due colonne per query come:
-- WHERE idCliente = 5 AND stato = 'Spedito'
CREATE INDEX idx_cliente_stato ON ordine(idCliente, stato);
\end{lstlisting}

\subsubsection{Indice a hash}
Usa una funzione hash per mappare i valori agli indirizzi di memoria. Veloce per uguaglianza, non per range.

\subsubsection{Indice B-Tree}
La struttura più comune. Ordinato, efficiente per range e ordinamenti. Bilanciato automaticamente.

\begin{lstlisting}[language=SQL, caption=Statistiche dell'indice (MySQL)]
-- Mostra dettagli degli indici di una tabella
SHOW INDEX FROM cliente;

-- Analizza l'efficienza dell'indice
EXPLAIN SELECT * FROM cliente WHERE email = 'mario@email.com';
\end{lstlisting}

\subsection{Strategie di indicizzazione}

\begin{tcolorbox}[colback=green!10, colframe=green!60, title=Buone pratiche per gli indici]
\begin{itemize}
    \item Indicizza campi usati frequentemente in WHERE, JOIN, ORDER BY
    \item Non indicizzare campi usati raramente (spreco di spazio)
    \item Indici compositi per query multi-colonna frequenti
    \item Evita indici su colonne con pochi valori distinti (low cardinality)
    \item Monitora e mantieni gli indici regolarmente
\end{itemize}
\end{tcolorbox}

\section{Viste Materializzate}

Una \textbf{vista materializzata} è il risultato di una query pre-calcolato e memorizzato fisicamente sul disco. A differenza di una vista normale, occupa spazio e deve essere aggiornato.

\subsection{Vantaggi}
Le viste materializzate offrono prestazioni eccezionali per determinati tipi di query. Consentono query estremamente veloci grazie al fatto che i dati sono già pre-calcolati e memorizzati fisicamente, eliminando il bisogno di rielaborazione al momento della richiesta. Riducono significativamente il carico computazionale del DBMS, liberando risorse per altre operazioni. Sono particolarmente utili per query complesse che operano su dati storici, dove i risultati rimangono relativamente stabili nel tempo.

\subsection{Svantaggi}
Nonostante i vantaggi di performance, le viste materializzate presentano limitazioni notevoli. Occupano spazio su disco in modo permanente, aumentando i requisiti di storage. Più importante ancora, deve essere aggiornato periodicamente per mantenere la coerenza con i dati sottostanti, richiedendo procedure di refresh pianificate. Inoltre, i dati possono diventare non aggiornati (stali), creando una finestra di tempo dove i risultati non riflettono lo stato attuale del database.

\subsection{Esempio di vista materializzata}

\begin{lstlisting}[language=SQL, caption=Vista materializzata]
-- Vista normale (calcolata ogni volta)
CREATE VIEW vendite_mensili AS
    SELECT
        MONTH(dataOrdine) AS mese,
        YEAR(dataOrdine) AS anno,
        SUM(totale) AS ricavi,
        COUNT(*) AS numOrdini
    FROM ordine
    GROUP BY YEAR(dataOrdine), MONTH(dataOrdine);

-- In MySQL, simula una vista materializzata con tabella
CREATE TABLE vendite_mensili_cache (
    mese INT,
    anno INT,
    ricavi DECIMAL(12, 2),
    numOrdini INT,
    dataAggiornamento TIMESTAMP
) AS
SELECT
    MONTH(dataOrdine) AS mese,
    YEAR(dataOrdine) AS anno,
    SUM(totale) AS ricavi,
    COUNT(*) AS numOrdini
FROM ordine
GROUP BY YEAR(dataOrdine), MONTH(dataOrdine);

-- Aggiorna periodicamente (es. ogni notte)
-- DELETE FROM vendite_mensili_cache;
-- INSERT INTO vendite_mensili_cache (SELECT ...);
\end{lstlisting}

\section{Partizionamento}

Il \textbf{partizionamento} divide una tabella grande in parti più piccole (partizioni) basate su criteri specifici, migliorando performance e manutenzione.

\subsection{Tipi di partizionamento}

\subsubsection{Partizionamento per range}
Divide i dati in base a intervalli di valori.

\begin{lstlisting}[language=SQL, caption=Partizionamento per range (anno)]
CREATE TABLE ordine (
    idOrdine INT,
    dataOrdine DATE,
    totale DECIMAL(10, 2)
)
PARTITION BY RANGE (YEAR(dataOrdine)) (
    PARTITION p2020 VALUES LESS THAN (2021),
    PARTITION p2021 VALUES LESS THAN (2022),
    PARTITION p2022 VALUES LESS THAN (2023),
    PARTITION p2023 VALUES LESS THAN (2024),
    PARTITION pmax VALUES LESS THAN MAXVALUE
);
\end{lstlisting}

\subsubsection{Partizionamento per list}
Divide i dati in base a valori specifici.

\begin{lstlisting}[language=SQL, caption=Partizionamento per list]
CREATE TABLE ordine (
    idOrdine INT,
    stato VARCHAR(20),
    totale DECIMAL(10, 2)
)
PARTITION BY LIST (stato) (
    PARTITION p_pending VALUES IN ('Pendente'),
    PARTITION p_shipped VALUES IN ('Spedito'),
    PARTITION p_delivered VALUES IN ('Consegnato'),
    PARTITION p_other VALUES IN (DEFAULT)
);
\end{lstlisting}

\subsubsection{Partizionamento per hash}
Distribuisce i dati in base a una funzione hash.

\begin{lstlisting}[language=SQL, caption=Partizionamento per hash]
CREATE TABLE ordine (
    idOrdine INT,
    idCliente INT,
    totale DECIMAL(10, 2)
)
PARTITION BY HASH(idCliente)
PARTITIONS 4;  -- 4 partizioni
\end{lstlisting}

\subsection{Vantaggi del partizionamento}
Il partizionamento offre benefici strategici per sistemi di grandi dimensioni. Consente query molto più veloci limitando la scansione solo alle partizioni rilevanti, evitando di esaminare l'intera tabella. Facilita la manutenzione attraverso operazioni per partizione singola, rendendo i backup e il recupero molto più agevoli e gestibili. Offre inoltre la capacità di scalabilità orizzontale, permettendo al sistema di distribuire dati e carico di lavoro su multiple partizioni e potenzialmente su più server.

\section{Ottimizzazione di Query}

\subsection{Utilizzo di EXPLAIN}
Il comando EXPLAIN analizza il piano di esecuzione di una query.

\begin{lstlisting}[language=SQL, caption=Analizzare il piano di esecuzione]
EXPLAIN SELECT *
FROM ordine o
JOIN cliente c ON o.idCliente = c.idCliente
WHERE c.città = 'Milano'
ORDER BY o.dataOrdine DESC;
\end{lstlisting}

L'output del comando EXPLAIN fornisce informazioni critiche per l'ottimizzazione. Identifica quale indice viene utilizzato durante l'esecuzione della query, o se viene eseguita una scansione completa della tabella. Mostra quante righe vengono esaminate per produrre il risultato, un indicatore importante dell'efficienza. Visualizza anche il costo relativo di ogni operazione, permettendo di identificare i colli di bottiglia nel piano di esecuzione.

\subsection{Statistiche e manutenzione}

\begin{lstlisting}[language=SQL, caption=Manutenzione degli indici]
-- Ricalcola statistiche tabella
ANALYZE TABLE ordine;

-- Ottimizza spazio tabella
OPTIMIZE TABLE ordine;

-- Mostra statistiche tabella
SHOW TABLE STATUS FROM database_name;
\end{lstlisting}

\section*{Riepilogo concetti chiave}

\begin{tcolorbox}[colback=gray!10, colframe=black!60, title=Concetti fondamentali]
I dati sono strutturati e memorizzati efficientemente in \textbf{pagine e blocchi} sul disco, secondo una gerarchia organizzata che ottimizza l'accesso. Gli \textbf{indici} rappresentano strutture cruciali che accelerano notevolmente ricerche, operazioni JOIN e ordinamenti, sebbene richiedano compromessi in termini di spazio e velocità di scrittura. Le \textbf{viste materializzate} pre-calcolano e memorizzano risultati di query frequenti, offendo prestazioni eccezionali al costo della storicità dei dati. Il \textbf{partizionamento} divide grandi tabelle in parti gestibili più piccole, migliorando performance e manutenibilità. Una considerazione fondamentale è il bilanciamento tra la velocità di lettura, ottimizzata dagli indici, e la velocità di scrittura, rallentata dagli stessi. Infine, monitorare il performance reale con EXPLAIN e statistiche è essenziale per identificare colli di bottiglia e ottimizzare continuamente il sistema.
\end{tcolorbox}

\section*{Esercizi}

\begin{enumerate}
    \item Progetta una strategia di indicizzazione per una tabella con milioni di ordini. Quali campi indicizzaresti? Perché?

    \item Una query SELECT * FROM cliente WHERE città = 'Roma' AND età > 30 è lenta. Come potrebbe aiutare un indice composito?

    \item Qual è la differenza tra una vista normal e una vista materializzata? Quando useresti ciascuna?

    \item Proponi un partizionamento per una tabella logs con milioni di record inseriti ogni giorno.

    \item Analizza il seguente piano di esecuzione e proponi ottimizzazioni:
    \begin{lstlisting}[language=SQL]
EXPLAIN SELECT *
FROM ordine
WHERE dataOrdine > '2023-01-01' AND stato = 'Spedito';
    \end{lstlisting}

    \item Crea una vista materializzata per memorizzare le vendite totali per cliente nel 2023. Come la aggiornare settimanalmente?
\end{enumerate}


% Capitolo 06: SQL - Fondamenti e DDL
\chapter{SQL DDL - Definizione dei Dati}

\section*{Introduzione}
DDL (Data Definition Language) è il linguaggio SQL per definire la struttura del database. Consente di creare, modificare e eliminare tabelle, indici, viste e altri oggetti dello schema. Questo capitolo presenta i principali comandi DDL: CREATE, ALTER, DROP.

\section*{Obiettivi di apprendimento}
\begin{itemize}
    \item Creare tabelle con tipi di dati appropriati
    \item Definire vincoli (PRIMARY KEY, FOREIGN KEY, UNIQUE, CHECK, DEFAULT)
    \item Modificare strutture di tabelle esistenti (ALTER TABLE)
    \item Eliminare tabelle, colonne, indici (DROP)
    \item Creare e gestire indici
    \item Definire chiavi esterne con azioni ON DELETE e ON UPDATE
    \item Usare tipi di dati appropriati per i dati
\end{itemize}

\section{Tipi di Dati in MySQL}

\subsection{Tipi numerici}

MySQL offre una ricca varietà di tipi numerici per memorizzare diverse classi di numeri in modo efficiente. Il tipo \textbf{INT} è il tipo intero più comune, allocando 4 byte e supportando valori nell'intervallo da -2.147.483.648 a 2.147.483.647. Il tipo \textbf{SMALLINT} è una variante compatta che utilizza 2 byte, supportando intervalli ridotti da -32.768 a 32.767, utile per risparmiare spazio quando i valori sono noti essere piccoli. Al contrario, \textbf{BIGINT} impiega 8 byte e è progettato per numeri molto grandi che superano la capacità di INT. Per numeri decimali, \textbf{DECIMAL(p,s)} consente precisione specifica, dove p rappresenta la precisione totale e s la scala decimale; ad esempio, DECIMAL(10,2) può memorizzare numeri fino a 10 cifre con 2 decimali. I tipi \textbf{FLOAT e DOUBLE} gestiscono numeri decimali in virgola mobile, offrendo una maggiore gamma di valori ma con minore precisione rispetto a DECIMAL, rendendoli meno appropriati per operazioni finanziarie. Infine, \textbf{TINYINT} è il tipo intero più compatto, usando solo 1 byte e supportando l'intervallo da -128 a 127, o da 0 a 255 se dichiarato come UNSIGNED.

\subsection{Tipi stringa}

Per i dati testuali, MySQL offre diverse opzioni ottimizzate per diversi scenari. \textbf{CHAR(n)} memorizza stringhe a lunghezza fissa di esattamente n caratteri, allocando sempre lo spazio completo anche se la stringa effettiva è più breve, causando spreco di memoria se le stringhe hanno lunghezze variabili. \textbf{VARCHAR(n)} è la scelta moderna preferita, memorizando stringhe di lunghezza variabile fino a n caratteri e occupando spazio pari alla lunghezza effettiva, risultando molto più efficiente. \textbf{TEXT} memorizza stringhe di lunghezza variabile molto lunga, fino a 4 GB, ideale per contenuti di grandi dimensioni come descrizioni lunghe o articoli. \textbf{BLOB} è dedicato ai dati binari come immagini, file audio e altro contenuto non testuale. Infine, \textbf{ENUM} rappresenta una stringa selezionata da un set predefinito di valori, ad esempio ENUM('Attivo', 'Inattivo'), fornendo validazione a livello di colonna.

\subsection{Tipi data/ora}

MySQL fornisce tipi specializzati per gestire date e orari con precisione. \textbf{DATE} memorizza unicamente la data nel formato standard YYYY-MM-DD, occupando lo spazio minimo dedicato alle sole informazioni cronologiche. \textbf{TIME} memorizza unicamente l'ora nel formato HH:MM:SS, utile quando la data non è rilevante. \textbf{DATETIME} combina data e ora nel formato YYYY-MM-DD HH:MM:SS, fornendo sia informazioni temporali che cronologiche in una singola colonna. \textbf{TIMESTAMP} memorizza data e ora con riferimento al timestamp Unix (secondi dal 1° gennaio 1970) e offre la caratteristica unica di aggiornamento automatico, utile per monitorare quando un record è stato creato o modificato per l'ultima volta. Infine, \textbf{YEAR} memorizza soltanto l'anno nel formato YYYY, occupando spazio minimo quando solo l'anno è rilevante.

\section{CREATE TABLE}

Il comando CREATE TABLE crea una nuova tabella con i relativi attributi e vincoli.

\subsection{Sintassi base}

\begin{lstlisting}[language=SQL, caption=Sintassi CREATE TABLE]
CREATE TABLE nome_tabella (
    nome_attributo tipo [vincoli],
    nome_attributo tipo [vincoli],
    ...
);
\end{lstlisting}

\subsection{Esempio semplice}

\begin{lstlisting}[language=SQL, caption=Creazione tabella Cliente]
CREATE TABLE cliente (
    idCliente INT PRIMARY KEY AUTO_INCREMENT,
    nome VARCHAR(100) NOT NULL,
    cognome VARCHAR(100) NOT NULL,
    email VARCHAR(100) UNIQUE NOT NULL,
    dataRegistrazione DATE DEFAULT CURDATE(),
    saldo DECIMAL(10, 2) DEFAULT 0.00
);
\end{lstlisting}

\subsection{Vincoli}

\subsubsection{PRIMARY KEY}
Identifica univocamente ogni riga. Non può essere NULL e deve essere unico.

\begin{lstlisting}[language=SQL, caption=PRIMARY KEY]
CREATE TABLE studente (
    matricola INT PRIMARY KEY,
    nome VARCHAR(100)
);

-- O alternativamente
CREATE TABLE studente (
    matricola INT,
    nome VARCHAR(100),
    PRIMARY KEY (matricola)
);

-- Chiave primaria composta
CREATE TABLE iscrizione (
    idStudente INT,
    idCorso INT,
    voto INT,
    PRIMARY KEY (idStudente, idCorso)
);
\end{lstlisting}

\subsubsection{FOREIGN KEY}
Referenzia la chiave primaria di un'altra tabella. Garantisce l'integrità referenziale.

\begin{lstlisting}[language=SQL, caption=FOREIGN KEY]
CREATE TABLE ordine (
    idOrdine INT PRIMARY KEY AUTO_INCREMENT,
    dataOrdine DATE NOT NULL,
    idCliente INT NOT NULL,
    FOREIGN KEY (idCliente) REFERENCES cliente(idCliente)
);

-- Con azioni su DELETE e UPDATE
CREATE TABLE ordine (
    idOrdine INT PRIMARY KEY AUTO_INCREMENT,
    dataOrdine DATE NOT NULL,
    idCliente INT NOT NULL,
    FOREIGN KEY (idCliente) REFERENCES cliente(idCliente)
        ON DELETE CASCADE        -- Elimina ordini se cliente è eliminato
        ON UPDATE CASCADE        -- Aggiorna idCliente se cambia
);
\end{lstlisting}

\subsubsection{UNIQUE}
Garantisce che i valori di un attributo siano unici (ma possono essere NULL).

\begin{lstlisting}[language=SQL, caption=Vincolo UNIQUE]
CREATE TABLE utente (
    idUtente INT PRIMARY KEY,
    username VARCHAR(50) UNIQUE NOT NULL,
    email VARCHAR(100) UNIQUE NOT NULL
);
\end{lstlisting}

\subsubsection{NOT NULL}
L'attributo deve avere sempre un valore.

\begin{lstlisting}[language=SQL, caption=Vincolo NOT NULL]
CREATE TABLE prodotto (
    idProdotto INT PRIMARY KEY,
    nome VARCHAR(100) NOT NULL,
    descrizione TEXT,           -- Può essere NULL
    prezzo DECIMAL(10, 2) NOT NULL
);
\end{lstlisting}

\subsubsection{CHECK}
Garantisce che i valori soddisfino una condizione.

\begin{lstlisting}[language=SQL, caption=Vincolo CHECK]
CREATE TABLE conto (
    idConto INT PRIMARY KEY,
    saldo DECIMAL(10, 2),
    -- Saldo non può essere negativo
    CHECK (saldo >= 0)
);

CREATE TABLE corso (
    idCorso INT PRIMARY KEY,
    nome VARCHAR(100),
    maxStudenti INT,
    minStudenti INT,
    -- Minimo <= Massimo
    CHECK (minStudenti <= maxStudenti)
);
\end{lstlisting}

\subsubsection{DEFAULT}
Assegna un valore di default se non specificato.

\begin{lstlisting}[language=SQL, caption=Vincolo DEFAULT]
CREATE TABLE articolo (
    idArticolo INT PRIMARY KEY,
    titolo VARCHAR(200) NOT NULL,
    stato VARCHAR(20) DEFAULT 'Bozza',
    dataCreazione DATETIME DEFAULT CURRENT_TIMESTAMP,
    visualizzazioni INT DEFAULT 0
);
\end{lstlisting}

\subsubsection{AUTO_INCREMENT}
Genera automaticamente valori incrementali per una colonna (tipicamente chiave primaria).

\begin{lstlisting}[language=SQL, caption=AUTO_INCREMENT]
CREATE TABLE cliente (
    idCliente INT PRIMARY KEY AUTO_INCREMENT,
    nome VARCHAR(100),
    email VARCHAR(100)
);
-- Primo cliente inserito avrà idCliente=1, secondo idCliente=2, etc.
\end{lstlisting}

\section{ALTER TABLE}

Il comando ALTER TABLE modifica la struttura di una tabella esistente.

\subsection{Aggiungere una colonna}

\begin{lstlisting}[language=SQL, caption=Aggiungere una colonna]
ALTER TABLE cliente ADD COLUMN telefono VARCHAR(20);

-- Aggiungere con vincoli
ALTER TABLE cliente ADD COLUMN dataNascita DATE DEFAULT NULL;

-- Aggiungere in posizione specifica
ALTER TABLE cliente ADD COLUMN indirizzo VARCHAR(200) AFTER nome;
\end{lstlisting}

\subsection{Modificare una colonna}

\begin{lstlisting}[language=SQL, caption=Modificare una colonna]
-- Cambiare tipo di dato
ALTER TABLE cliente MODIFY COLUMN telefono VARCHAR(50);

-- Cambiare nome della colonna
ALTER TABLE cliente CHANGE COLUMN telefono cellulare VARCHAR(20);

-- Aggiungere vincolo NOT NULL
ALTER TABLE cliente MODIFY COLUMN email VARCHAR(100) NOT NULL;
\end{lstlisting}

\subsection{Eliminare una colonna}

\begin{lstlisting}[language=SQL, caption=Eliminare una colonna]
ALTER TABLE cliente DROP COLUMN fax;
\end{lstlisting}

\subsection{Aggiungere vincoli}

\begin{lstlisting}[language=SQL, caption=Aggiungere vincoli]
-- Aggiungere PRIMARY KEY
ALTER TABLE cliente ADD PRIMARY KEY (idCliente);

-- Aggiungere UNIQUE
ALTER TABLE cliente ADD CONSTRAINT uq_email UNIQUE (email);

-- Aggiungere CHECK
ALTER TABLE cliente ADD CONSTRAINT ck_eta CHECK (eta >= 18);

-- Aggiungere FOREIGN KEY
ALTER TABLE ordine ADD CONSTRAINT fk_cliente
    FOREIGN KEY (idCliente) REFERENCES cliente(idCliente);
\end{lstlisting}

\subsection{Rinominare una tabella}

\begin{lstlisting}[language=SQL, caption=Rinominare una tabella]
ALTER TABLE cliente RENAME TO customer;
\end{lstlisting}

\section{DROP}

Il comando DROP elimina oggetti del database.

\subsection{DROP TABLE}

\begin{lstlisting}[language=SQL, caption=Eliminare una tabella]
-- Elimina la tabella cliente
DROP TABLE cliente;

-- Elimina solo se esiste (evita errore se non esiste)
DROP TABLE IF EXISTS cliente;

-- Eliminare più tabelle
DROP TABLE ordine, prodotto, cliente;
\end{lstlisting}

\begin{tcolorbox}[colback=red!10, colframe=red!60, title=Attenzione: DROP è irreversibile]
DROP TABLE elimina completamente la tabella e tutti i dati. Non c'è possibilità di recuperarli (a meno che non sia stato fatto un backup). Usa sempre IF EXISTS per sicurezza.
\end{tcolorbox}

\subsection{DROP COLUMN}

\begin{lstlisting}[language=SQL, caption=Eliminare una colonna]
ALTER TABLE cliente DROP COLUMN numeroFax;
\end{lstlisting}

\subsection{DROP INDEX}

\begin{lstlisting}[language=SQL, caption=Eliminare un indice]
DROP INDEX idx_email ON cliente;
\end{lstlisting}

\section{CREATE INDEX}

\subsection{Creare indici}

\begin{lstlisting}[language=SQL, caption=Creare indici]
-- Indice semplice
CREATE INDEX idx_email ON cliente(email);

-- Indice composito (su più colonne)
CREATE INDEX idx_cliente_data ON ordine(idCliente, dataOrdine);

-- Indice UNIQUE
CREATE UNIQUE INDEX idx_codice_prodotto ON prodotto(codice);
\end{lstlisting}

\subsection{Eliminare indici}

\begin{lstlisting}[language=SQL, caption=Eliminare indici]
DROP INDEX idx_email ON cliente;
\end{lstlisting}

\section{Esempio Completo: Schema Biblioteca}

\begin{lstlisting}[language=SQL, caption=Schema completo per una biblioteca]
-- Tabella Autori
CREATE TABLE autore (
    idAutore INT PRIMARY KEY AUTO_INCREMENT,
    nome VARCHAR(100) NOT NULL,
    cognome VARCHAR(100) NOT NULL,
    nazionalita VARCHAR(50),
    dataAnniversario DATE
);

-- Tabella Libri
CREATE TABLE libro (
    idLibro INT PRIMARY KEY AUTO_INCREMENT,
    titolo VARCHAR(200) NOT NULL,
    idAutore INT NOT NULL,
    anno_pubblicazione INT CHECK (anno_pubblicazione > 1000),
    genere VARCHAR(50),
    prezzo DECIMAL(10, 2) DEFAULT 0,
    quantita_disponibile INT DEFAULT 0,
    FOREIGN KEY (idAutore) REFERENCES autore(idAutore)
        ON DELETE RESTRICT
        ON UPDATE CASCADE
);

-- Tabella Membri
CREATE TABLE membro (
    idMembro INT PRIMARY KEY AUTO_INCREMENT,
    nome VARCHAR(100) NOT NULL,
    cognome VARCHAR(100) NOT NULL,
    email VARCHAR(100) UNIQUE,
    dataIscrizione DATE DEFAULT CURDATE(),
    stato ENUM('Attivo', 'Sospeso') DEFAULT 'Attivo'
);

-- Tabella Prestiti
CREATE TABLE prestito (
    idPrestito INT PRIMARY KEY AUTO_INCREMENT,
    idLibro INT NOT NULL,
    idMembro INT NOT NULL,
    dataPrestito DATE DEFAULT CURDATE(),
    dataReso DATE,
    FOREIGN KEY (idLibro) REFERENCES libro(idLibro),
    FOREIGN KEY (idMembro) REFERENCES membro(idMembro),
    CHECK (dataReso IS NULL OR dataReso >= dataPrestito)
);

-- Creare indici per query frequenti
CREATE INDEX idx_libro_autore ON libro(idAutore);
CREATE INDEX idx_prestito_libro ON prestito(idLibro);
CREATE INDEX idx_prestito_membro ON prestito(idMembro);
CREATE INDEX idx_membro_email ON membro(email);
\end{lstlisting}

\section*{Riepilogo concetti chiave}

\begin{tcolorbox}[colback=gray!10, colframe=black!60, title=Concetti fondamentali]
Il comando \textbf{CREATE TABLE} è lo strumento fondamentale per definire la struttura della tabella con attributi e vincoli. I \textbf{vincoli} (PRIMARY KEY, FOREIGN KEY, UNIQUE, CHECK, NOT NULL, DEFAULT) sono meccanismi essenziali che garantiscono integrità dei dati prevenendo stati incoerenti. \textbf{ALTER TABLE} fornisce la flessibilità di modificare tabelle esistenti attraverso operazioni di aggiunta, modifica e eliminazione di colonne senza ricreare l'intera struttura. Il comando \textbf{DROP} consente di eliminare tabelle, colonne e indici, operazione che è definitivamente irreversibile e richiede estrema cautela. La scelta appropriata di \textbf{tipi di dati} non è semplicemente una questione di correttezza, ma influenza significativamente l'ottimizzazione dello spazio su disco e della velocità di elaborazione. Infine, gli \textbf{indici} accelerano le ricerche e devono essere creati strategicamente basandosi su modelli di accesso reali per massimizzare il beneficio.
\end{tcolorbox}

\section*{Esercizi}

\begin{enumerate}
    \item Crea una tabella \texttt{dipendente} con: idDipendente (INT, chiave primaria, auto-incremento), nome (VARCHAR), cognome (VARCHAR), stipendio (DECIMAL, >= 0), dipartimento (VARCHAR, default 'Non assegnato'), dataAssunzione (DATE, default data odierna).

    \item Modifica la tabella precedente aggiungendo una colonna \texttt{email} (VARCHAR, UNIQUE, NOT NULL).

    \item Crea una tabella \texttt{reparto} e un'associazione con \texttt{dipendente} tramite FOREIGN KEY. La cancellazione di un reparto non deve eliminare i dipendenti.

    \item Progetta l'intero schema DDL per un sistema ospedaliero con: Ospedali, Reparti, Pazienti, Medici, Appuntamenti, con tutti i vincoli appropriati.

    \item Crea indici appropriati per le seguenti query frequenti:
    \begin{lstlisting}[language=SQL]
SELECT * FROM ordine WHERE idCliente = ? AND stato = 'Spedito';
SELECT * FROM prodotto WHERE categoria = ?;
SELECT * FROM cliente WHERE email = ?;
    \end{lstlisting}
\end{enumerate}


% Capitolo 07: SQL - DML (Query SELECT)
\chapter{SQL DML - SELECT}

\section*{Introduzione}
DML (Data Manipulation Language) è il linguaggio SQL per interrogare e manipolare i dati. Il comando SELECT è il più importante e potente di DML, permettendo di recuperare dati da una o più tabelle con filtri, ordinamenti e proiezioni sofisticate.

\section*{Obiettivi di apprendimento}
\begin{itemize}
    \item Scrivere query SELECT di base
    \item Filtrare dati con WHERE
    \item Ordinare risultati con ORDER BY
    \item Limitare risultati con LIMIT
    \item Usare funzioni di aggregazione (COUNT, SUM, AVG, MIN, MAX)
    \item Raggruppare dati con GROUP BY
    \item Filtrare gruppi con HAVING
    \item Usare alias per colonne e tabelle
    \item Scrivere query sub-SELECT (subquery)
\end{itemize}

\section{Sintassi SELECT di Base}

\subsection{Sintassi generale}

\begin{lstlisting}[language=SQL, caption=Sintassi SELECT]
SELECT [DISTINCT] colonna1, colonna2, ...
FROM tabella1
WHERE condizioni
ORDER BY colonna [ASC|DESC]
LIMIT numero_righe;
\end{lstlisting}

\subsection{Selezionare tutte le colonne}

\begin{lstlisting}[language=SQL, caption=Selezionare tutte le colonne]
-- Seleziona tutte le colonne
SELECT * FROM cliente;

-- Limita ai primi 10 risultati
SELECT * FROM cliente LIMIT 10;
\end{lstlisting}

\subsection{Selezionare colonne specifiche}

\begin{lstlisting}[language=SQL, caption=Selezionare colonne specifiche]
SELECT idCliente, nome, cognome, email FROM cliente;
\end{lstlisting}

\subsection{Alias per colonne}

\begin{lstlisting}[language=SQL, caption=Alias per colonne]
-- Renominare colonne nel risultato
SELECT
    idCliente AS ID,
    nome AS Nome,
    email AS 'Indirizzo Email'
FROM cliente;
\end{lstlisting}

\section{WHERE - Filtri}

Il WHERE filtra le righe in base a condizioni.

\subsection{Operatori di confronto}

\begin{lstlisting}[language=SQL, caption=Operatori di confronto]
-- Uguaglianza
SELECT * FROM cliente WHERE stato = 'Attivo';

-- Diverso
SELECT * FROM cliente WHERE stato != 'Inattivo';
SELECT * FROM cliente WHERE stato <> 'Inattivo';

-- Maggiore/minore
SELECT * FROM ordine WHERE totale > 100.00;
SELECT * FROM ordine WHERE totale <= 50.00;

-- BETWEEN (inclusivo)
SELECT * FROM ordine WHERE dataOrdine BETWEEN '2023-01-01' AND '2023-12-31';
\end{lstlisting}

\subsection{Operatori logici}

\begin{lstlisting}[language=SQL, caption=Operatori logici AND, OR, NOT]
-- AND: tutte le condizioni devono essere vere
SELECT * FROM cliente
WHERE stato = 'Attivo' AND città = 'Milano';

-- OR: almeno una condizione deve essere vera
SELECT * FROM cliente
WHERE città = 'Milano' OR città = 'Roma';

-- NOT: negazione
SELECT * FROM cliente
WHERE NOT stato = 'Inattivo';
\end{lstlisting}

\subsection{IN e NOT IN}

\begin{lstlisting}[language=SQL, caption=Operatori IN e NOT IN]
-- IN: verifica se il valore è in un elenco
SELECT * FROM ordine
WHERE stato IN ('Spedito', 'Consegnato');

-- NOT IN: verifica se il valore NON è in un elenco
SELECT * FROM ordine
WHERE stato NOT IN ('Cancellato', 'Rifiutato');
\end{lstlisting}

\subsection{LIKE - Pattern matching}

\begin{lstlisting}[language=SQL, caption=LIKE per pattern matching]
-- % rappresenta 0 o più caratteri, _ rappresenta un carattere singolo

-- Nome che inizia con 'M'
SELECT * FROM cliente WHERE nome LIKE 'M%';

-- Nome che finisce con 'o'
SELECT * FROM cliente WHERE nome LIKE '%o';

-- Email che contiene 'gmail'
SELECT * FROM cliente WHERE email LIKE '%gmail%';

-- Parola di esattamente 5 caratteri
SELECT * FROM cliente WHERE nome LIKE '_____';

-- Case-insensitive (dipende dal collation)
SELECT * FROM cliente WHERE nome LIKE 'mario%';
\end{lstlisting}

\subsection{IS NULL e IS NOT NULL}

\begin{lstlisting}[language=SQL, caption=NULL in WHERE]
-- Valori NULL
SELECT * FROM cliente WHERE telefono IS NULL;

-- Valori non NULL
SELECT * FROM cliente WHERE telefono IS NOT NULL;

-- Nota: = NULL non funziona! Usare IS NULL
SELECT * FROM cliente WHERE telefono = NULL;  -- Restituisce 0 righe!
\end{lstlisting}

\section{ORDER BY - Ordinamento}

Ordina i risultati in base a una o più colonne.

\begin{lstlisting}[language=SQL, caption=ORDER BY]
-- Ordinamento ascendente (default)
SELECT * FROM ordine ORDER BY dataOrdine ASC;

-- Ordinamento discendente
SELECT * FROM ordine ORDER BY dataOrdine DESC;

-- Ordinamento su più colonne
SELECT * FROM cliente
ORDER BY città ASC, cognome ASC, nome ASC;

-- Ordinamento per numero di colonna (sconsigliato)
SELECT nome, cognome, email FROM cliente
ORDER BY 2, 1;  -- Ordina per colonna 2 (cognome), poi colonna 1 (nome)
\end{lstlisting}

\section{LIMIT - Limitazione Risultati}

Limita il numero di righe restituite.

\begin{lstlisting}[language=SQL, caption=LIMIT]
-- Primi 10 risultati
SELECT * FROM cliente LIMIT 10;

-- 10 risultati a partire dal 20° (offset=20, limit=10)
SELECT * FROM cliente LIMIT 20, 10;

-- Sintassi alternativa
SELECT * FROM cliente LIMIT 10 OFFSET 20;

-- Ultimi 5 clienti registrati
SELECT * FROM cliente
ORDER BY dataRegistrazione DESC
LIMIT 5;
\end{lstlisting}

\section{DISTINCT - Eliminare Duplicati}

Restituisce solo valori unici.

\begin{lstlisting}[language=SQL, caption=DISTINCT]
-- Tutte le città
SELECT DISTINCT città FROM cliente;

-- Combinazione unica di città e stato
SELECT DISTINCT città, stato FROM cliente;

-- Contare città distinte
SELECT COUNT(DISTINCT città) FROM cliente;
\end{lstlisting}

\section{Funzioni di Stringa}

\begin{lstlisting}[language=SQL, caption=Funzioni di stringa]
-- Lunghezza stringa
SELECT nome, LENGTH(nome) AS lunghezza FROM cliente;

-- Concatenazione
SELECT CONCAT(nome, ' ', cognome) AS nomeCompleto FROM cliente;

-- Maiuscolo/minuscolo
SELECT UPPER(nome), LOWER(cognome) FROM cliente;

-- Substring
SELECT SUBSTRING(email, 1, POSITION('@' IN email) - 1) AS username FROM cliente;

-- Sostituzione
SELECT REPLACE(email, 'gmail.com', 'email.com') FROM cliente;

-- Trim (rimuove spazi)
SELECT TRIM(nome) FROM cliente;
\end{lstlisting}

\section{Funzioni di Data}

\begin{lstlisting}[language=SQL, caption=Funzioni di data]
-- Data odierna
SELECT CURDATE() AS oggi;
SELECT CURRENT_DATE() AS oggi;

-- Data e ora attuali
SELECT NOW() AS adesso;
SELECT CURRENT_TIMESTAMP() AS adesso;

-- Estrarre parti della data
SELECT
    YEAR(dataOrdine) AS anno,
    MONTH(dataOrdine) AS mese,
    DAY(dataOrdine) AS giorno
FROM ordine;

-- Differenza tra date (in giorni)
SELECT
    idOrdine,
    DATEDIFF(CURDATE(), dataOrdine) AS giorni_passati
FROM ordine;

-- Aggiungere giorni a una data
SELECT DATE_ADD(CURDATE(), INTERVAL 7 DAY) AS tra_una_settimana;
SELECT DATE_ADD(CURDATE(), INTERVAL 1 MONTH) AS tra_un_mese;

-- Formato data
SELECT DATE_FORMAT(dataOrdine, '%d/%m/%Y') FROM ordine;
\end{lstlisting}

\section{Funzioni Numeriche}

\begin{lstlisting}[language=SQL, caption=Funzioni numeriche]
-- Arrotondamento
SELECT ROUND(prezzo, 2) FROM prodotto;

-- Arrotondamento per difetto/eccesso
SELECT FLOOR(prezzo), CEIL(prezzo) FROM prodotto;

-- Valore assoluto
SELECT ABS(-10);

-- Potenza
SELECT POWER(2, 3);  -- 2^3 = 8

-- Radice quadrata
SELECT SQRT(16);  -- 4

-- Modulo (resto divisione)
SELECT MOD(10, 3);  -- 1
\end{lstlisting}

\section{Subquery (Sub-SELECT)}

Una subquery è una query dentro un'altra query.

\begin{lstlisting}[language=SQL, caption=Subquery in WHERE]
-- Trovare clienti che hanno fatto ordini superiori alla media
SELECT *
FROM cliente
WHERE idCliente IN (
    SELECT idCliente FROM ordine WHERE totale > (
        SELECT AVG(totale) FROM ordine
    )
);

-- Trovare prodotti con prezzo superiore al prezzo medio della categoria
SELECT *
FROM prodotto p1
WHERE prezzo > (
    SELECT AVG(prezzo) FROM prodotto p2
    WHERE p2.categoria = p1.categoria
);
\end{lstlisting}

\begin{lstlisting}[language=SQL, caption=Subquery in SELECT]
-- Aggiungere il numero di ordini accanto a ogni cliente
SELECT
    idCliente,
    nome,
    (SELECT COUNT(*) FROM ordine o WHERE o.idCliente = c.idCliente) AS numOrdini
FROM cliente c;
\end{lstlisting}

\section{Esempio Completo}

\begin{lstlisting}[language=SQL, caption=Query complessa]
-- Trovare i 5 clienti più attivi del 2023
-- ordinati per numero di ordini decrescente
SELECT
    c.idCliente,
    c.nome,
    c.cognome,
    COUNT(o.idOrdine) AS numOrdini,
    SUM(o.totale) AS totalSpeso
FROM cliente c
LEFT JOIN ordine o ON c.idCliente = o.idCliente
    AND YEAR(o.dataOrdine) = 2023
WHERE c.stato = 'Attivo'
GROUP BY c.idCliente, c.nome, c.cognome
ORDER BY numOrdini DESC
LIMIT 5;
\end{lstlisting}

\section*{Riepilogo concetti chiave}

\begin{tcolorbox}[colback=gray!10, colframe=black!60, title=Concetti fondamentali]
\begin{itemize}
    \item \textbf{SELECT} recupera dati da tabelle
    \item \textbf{WHERE} filtra righe in base a condizioni
    \item \textbf{ORDER BY} ordina risultati (ASC/DESC)
    \item \textbf{LIMIT} limita il numero di righe
    \item \textbf{DISTINCT} elimina duplicati
    \item \textbf{Funzioni} (stringa, data, numero) trasformano i dati
    \item \textbf{Subquery} permettono query annidate
    \item Combinare clausole per query sofisticate
\end{itemize}
\end{tcolorbox}

\section*{Esercizi}

\begin{enumerate}
    \item Scrivi una query per trovare tutti i clienti della città di 'Milano' registrati negli ultimi 30 giorni, ordinati per cognome.

    \item Scrivi una query per elencare prodotti il cui prezzo è tra 50 e 100 euro, escludendo la categoria 'Saldi'.

    \item Conta quanti ordini sono stati fatti da cliente 'id=5' e mostra il numero come colonna 'NumOrdini'.

    \item Usa una subquery per trovare i clienti che hanno speso più della media totale di tutti gli ordini.

    \item Scrivi una query che mostra, per ogni cliente: nome, cognome, numero di ordini effettuati, importo totale speso, ordinato per importo totale decrescente. Usa LEFT JOIN se necessario.

    \item Estrai da una tabella 'articolo' gli articoli con titolo che contiene 'Database' (case-insensitive), creati nel 2023 o successivamente, ordinati per data di creazione.
\end{enumerate}


% Capitolo 08: SQL - DML (INSERT, UPDATE, DELETE)
\chapter{SQL DML - Modifiche Dati}

\section*{Introduzione}
Oltre a leggere dati con SELECT, DML (Data Manipulation Language) permette di inserire nuovi dati (INSERT), aggiornare dati esistenti (UPDATE) e eliminare dati (DELETE). Questo capitolo presenta questi tre comandi essenziali con attenzione particolare alla preservazione dell'integrità dei dati.

\section*{Obiettivi di apprendimento}
\begin{itemize}
    \item Inserire nuove righe con INSERT
    \item Inserire dati da altre tabelle
    \item Aggiornare dati esistenti con UPDATE
    \item Eliminare dati con DELETE
    \item Gestire transazioni per operazioni sicure
    \item Comprendere i vincoli e le loro violazioni
    \item Usare controlli e prevenire errori
\end{itemize}

\section{INSERT - Inserimento Dati}

Il comando INSERT aggiunge nuove righe a una tabella.

\subsection{Sintassi base}

\begin{lstlisting}[language=SQL, caption=Sintassi INSERT]
INSERT INTO tabella (colonna1, colonna2, ...)
VALUES (valore1, valore2, ...);
\end{lstlisting}

\subsection{Inserire una riga}

\begin{lstlisting}[language=SQL, caption=Inserimento semplice]
INSERT INTO cliente (idCliente, nome, cognome, email)
VALUES (1, 'Mario', 'Rossi', 'mario@email.com');
\end{lstlisting}

\subsection{Inserire senza specificare colonne}

Se ometti le colonne, devi fornire valori per TUTTE le colonne nell'ordine definito.

\begin{lstlisting}[language=SQL, caption=Inserimento senza specificare colonne]
-- Se la tabella ha: idCliente, nome, cognome, email, città
INSERT INTO cliente
VALUES (2, 'Luigi', 'Bianchi', 'luigi@email.com', 'Roma');
\end{lstlisting}

\subsection{Inserire più righe}

\begin{lstlisting}[language=SQL, caption=Inserimento multiplo]
INSERT INTO cliente (idCliente, nome, cognome, email)
VALUES
    (3, 'Anna', 'Verdi', 'anna@email.com'),
    (4, 'Giovanni', 'Gialli', 'giovanni@email.com'),
    (5, 'Maria', 'Neri', 'maria@email.com');
\end{lstlisting}

\subsection{Inserire da SELECT}

\begin{lstlisting}[language=SQL, caption=INSERT ... SELECT]
-- Copiare dati da un'altra tabella
INSERT INTO cliente_backup
SELECT * FROM cliente WHERE città = 'Milano';

-- Inserire dati trasformati
INSERT INTO cliente_attivi (nome, cognome, città)
SELECT nome, cognome, città FROM cliente WHERE stato = 'Attivo';
\end{lstlisting}

\subsection{Gestione di valori NULL}

\begin{lstlisting}[language=SQL, caption=Inserire NULL]
-- Valori NULL per colonne opzionali
INSERT INTO cliente (idCliente, nome, cognome, email, telefono)
VALUES (6, 'Paolo', 'Rossi', 'paolo@email.com', NULL);

-- Omettere colonna opzionale (diventa NULL)
INSERT INTO cliente (idCliente, nome, cognome, email)
VALUES (7, 'Laura', 'Bianchi', 'laura@email.com');
-- Il campo telefono sarà NULL
\end{lstlisting}

\subsection{Auto-increment}

\begin{lstlisting}[language=SQL, caption=AUTO_INCREMENT]
-- Se la colonna è AUTO_INCREMENT, omettere il valore
INSERT INTO cliente (nome, cognome, email)
VALUES ('Marco', 'Verdi', 'marco@email.com');
-- idCliente verrà assegnato automaticamente

-- Recuperare l'ID appena inserito
INSERT INTO cliente (nome, cognome, email)
VALUES ('Silvia', 'Gialli', 'silvia@email.com');
SELECT LAST_INSERT_ID();  -- Restituisce l'ID appena assegnato
\end{lstlisting}

\begin{tcolorbox}[colback=orange!10, colframe=orange!60, title=Nota: Gestire AUTO_INCREMENT]
Per auto-increment, non inserire esplicitamente il valore. Lascia che il DBMS lo generi. Questo garantisce unicità.
\end{tcolorbox}

\subsection{Errori comuni di INSERT}

\begin{tcolorbox}[colback=red!10, colframe=red!60, title=Errore: Violazione vincolo PRIMARY KEY]
\begin{lstlisting}[language=SQL]
-- Errore: tentare di inserire un id già esistente
INSERT INTO cliente (idCliente, nome, cognome, email)
VALUES (1, 'Antonio', 'Neri', 'antonio@email.com');
-- Errore: Duplicate entry for primary key
\end{lstlisting}
\end{tcolorbox}

\begin{tcolorbox}[colback=red!10, colframe=red!60, title=Errore: Violazione FOREIGN KEY]
\begin{lstlisting}[language=SQL]
-- Errore: tentare di inserire un ordine per un cliente inesistente
INSERT INTO ordine (idOrdine, idCliente, dataOrdine)
VALUES (1, 999, CURDATE());
-- Errore: Foreign key constraint fails (se cliente 999 non esiste)
\end{lstlisting}
\end{tcolorbox}

\section{UPDATE - Aggiornamento Dati}

Il comando UPDATE modifica dati esistenti.

\subsection{Sintassi base}

\begin{lstlisting}[language=SQL, caption=Sintassi UPDATE]
UPDATE tabella
SET colonna1 = valore1, colonna2 = valore2, ...
WHERE condizione;
\end{lstlisting}

\begin{tcolorbox}[colback=red!10, colframe=red!60, title=Attenzione: WHERE è obbligatorio!]
Senza WHERE, TUTTI i record verranno aggiornati! Usa sempre WHERE per evitare danni.
\end{tcolorbox}

\subsection{Aggiornare una colonna}

\begin{lstlisting}[language=SQL, caption=UPDATE semplice]
-- Aggiornare l'email di un cliente
UPDATE cliente
SET email = 'mario.rossi@newemail.com'
WHERE idCliente = 1;
\end{lstlisting}

\subsection{Aggiornare più colonne}

\begin{lstlisting}[language=SQL, caption=UPDATE multiplo]
-- Aggiornare nome e cognome
UPDATE cliente
SET nome = 'Marco', cognome = 'Verdi'
WHERE idCliente = 1;
\end{lstlisting}

\subsection{Aggiornare con condizioni complesse}

\begin{lstlisting}[language=SQL, caption=UPDATE con WHERE complesso]
-- Aumentare lo sconto del 10% per i clienti di Milano attivi
UPDATE cliente
SET sconto = sconto * 1.10
WHERE città = 'Milano' AND stato = 'Attivo';

-- Spostare ordini cancellati agli archivi
UPDATE ordine
SET stato = 'Archiviato'
WHERE stato = 'Cancellato'
    AND dataOrdine < DATE_SUB(CURDATE(), INTERVAL 1 YEAR);
\end{lstlisting}

\subsection{Aggiornare da SELECT}

\begin{lstlisting}[language=SQL, caption=UPDATE da subquery]
-- Aumentare prezzo di prodotti basato su categoria
UPDATE prodotto
SET prezzo = prezzo * 1.05
WHERE categoria IN (
    SELECT categoria FROM categoria_promozionale
);

-- Aggiornare il numero di ordini di ogni cliente
UPDATE cliente c
SET numOrdini = (SELECT COUNT(*) FROM ordine o WHERE o.idCliente = c.idCliente);
\end{lstlisting}

\subsection{Aggiornare con espressioni}

\begin{lstlisting}[language=SQL, caption=UPDATE con espressioni]
-- Aumentare il prezzo del 10% arrotondato a 2 decimali
UPDATE prodotto
SET prezzo = ROUND(prezzo * 1.10, 2)
WHERE categoria = 'Elettronica';

-- Aggiornare data ultima modifica a data odierna
UPDATE cliente
SET dataUltimaModifica = NOW()
WHERE idCliente = 1;

-- Concatenare stringhe
UPDATE cliente
SET nomePieno = CONCAT(nome, ' ', cognome)
WHERE nomePieno IS NULL;
\end{lstlisting}

\section{DELETE - Eliminazione Dati}

Il comando DELETE rimuove righe da una tabella.

\subsection{Sintassi base}

\begin{lstlisting}[language=SQL, caption=Sintassi DELETE]
DELETE FROM tabella
WHERE condizione;
\end{lstlisting}

\begin{tcolorbox}[colback=red!10, colframe=red!60, title=Attenzione: WHERE è obbligatorio!]
Senza WHERE, TUTTI i record verranno eliminati! Usa sempre WHERE.
\end{tcolorbox}

\subsection{Eliminare una riga}

\begin{lstlisting}[language=SQL, caption=DELETE singola riga]
-- Eliminare un cliente specifico
DELETE FROM cliente
WHERE idCliente = 1;
\end{lstlisting}

\subsection{Eliminare con condizioni}

\begin{lstlisting}[language=SQL, caption=DELETE con WHERE]
-- Eliminare ordini cancellati più di un anno fa
DELETE FROM ordine
WHERE stato = 'Cancellato'
    AND dataOrdine < DATE_SUB(CURDATE(), INTERVAL 1 YEAR);

-- Eliminare clienti inattivi senza ordini
DELETE FROM cliente
WHERE stato = 'Inattivo'
    AND NOT EXISTS (SELECT 1 FROM ordine WHERE ordine.idCliente = cliente.idCliente);
\end{lstlisting}

\subsection{Errori di integrità referenziale}

\begin{tcolorbox}[colback=red!10, colframe=red!60, title=Errore: Violazione FOREIGN KEY su DELETE]
\begin{lstlisting}[language=SQL]
-- Se ordine ha FOREIGN KEY verso cliente
-- e non è definito ON DELETE CASCADE,
-- non puoi eliminare un cliente che ha ordini
DELETE FROM cliente WHERE idCliente = 5;
-- Errore: Cannot delete: foreign key constraint fails
\end{lstlisting}
\end{tcolorbox}

\subsection{Soluzioni per vincoli di integrità}

\begin{lstlisting}[language=SQL, caption=Soluzione: Gestire dipendenze]
-- Opzione 1: Eliminare prima gli ordini dipendenti
DELETE FROM ordine WHERE idCliente = 5;
DELETE FROM cliente WHERE idCliente = 5;

-- Opzione 2: Se FK ha ON DELETE CASCADE
-- Eliminare il cliente elimina automaticamente gli ordini
DELETE FROM cliente WHERE idCliente = 5;

-- Opzione 3: Soft delete (marcazione come inattivo)
UPDATE cliente SET stato = 'Inattivo' WHERE idCliente = 5;
-- Permette recupero successivo se necessario
\end{lstlisting}

\section{Transazioni ACID}

Una transazione raggruppa più operazioni SQL in un'unità atomica.

\begin{lstlisting}[language=SQL, caption=Transazioni]
-- Iniziare transazione
START TRANSACTION;

-- Operazioni
INSERT INTO cliente (nome, cognome, email) VALUES ('Mario', 'Rossi', 'mario@email.com');
INSERT INTO ordine (idCliente, dataOrdine, totale) VALUES (LAST_INSERT_ID(), CURDATE(), 150.00);

-- Se tutto ok, confermare
COMMIT;

-- Se c'è un errore, annullare
-- ROLLBACK;
\end{lstlisting}

\subsection{Scenari transazionali}

\begin{tcolorbox}[colback=blue!10, colframe=blue!60, title=Esempio: Trasferimento bancario}
\begin{lstlisting}[language=SQL]
START TRANSACTION;

-- Prelievo dal conto A
UPDATE conto SET saldo = saldo - 100 WHERE idConto = 1;

-- Deposito al conto B
UPDATE conto SET saldo = saldo + 100 WHERE idConto = 2;

-- Se entrambi ok, commit. Se uno fallisce, tutto viene rollback
COMMIT;
\end{lstlisting}
\end{tcolorbox}

\section{Disabilitare Vincoli (Avanzato)}

Talvolta è necessario disabilitare temporaneamente i vincoli per operazioni bulk.

\begin{lstlisting}[language=SQL, caption=Disabilitare/Abilitare vincoli]
-- Disabilitare i vincoli di chiave esterna (MySQL)
SET FOREIGN_KEY_CHECKS = 0;

-- Eseguire operazioni bulk
DELETE FROM ordine;
DELETE FROM cliente;

-- Riabilitare vincoli
SET FOREIGN_KEY_CHECKS = 1;
\end{lstlisting}

\begin{tcolorbox}[colback=red!10, colframe=red!60, title=Attenzione: Disabilitare vincoli è rischioso]
Disabilitando i vincoli, il DBMS non controlla l'integrità referenziale. Usa solo se sai cosa fai e riabilita subito dopo.
\end{tcolorbox}

\section*{Riepilogo concetti chiave}

\begin{tcolorbox}[colback=gray!10, colframe=black!60, title=Concetti fondamentali]
\begin{itemize}
    \item \textbf{INSERT} aggiunge nuove righe. Usa AUTO_INCREMENT per ID.
    \item \textbf{INSERT ... SELECT} copia dati da altre tabelle
    \item \textbf{UPDATE} modifica dati. WHERE specifica quali righe.
    \item \textbf{DELETE} elimina righe. WHERE è essenziale!
    \item I \textbf{vincoli} (PRIMARY KEY, FOREIGN KEY) proteggono l'integrità
    \item Le \textbf{transazioni} (BEGIN, COMMIT, ROLLBACK) garantiscono atomicità
    \item Sempre testare con SELECT prima di UPDATE/DELETE in massa
\end{itemize}
\end{tcolorbox}

\section*{Esercizi}

\begin{enumerate}
    \item Inserisci 5 clienti nella tabella cliente con nomi, cognomi e email distinti.

    \item Usa INSERT ... SELECT per copiare tutti gli ordini del 2023 in una tabella ordini\_2023.

    \item Aggiorna il prezzo di tutti i prodotti della categoria 'Elettronica' con un aumento del 15\%.

    \item Scrivi una transazione che: a) inserisce un nuovo cliente, b) inserisce un ordine per quel cliente, c) aggiorna il totale speso del cliente.

    \item Elimina tutti gli ordini di un cliente che hanno stato 'Cancellato' e non sono stati modificati da più di 6 mesi.

    \item Crea uno script che archiva ordini vecchi: copia ordini più vecchi di 2 anni in una tabella archivio\_ordini, poi elimina gli originali. Usa una transazione!
\end{enumerate}


% Capitolo 09: SQL - Join e Query Complesse
\chapter{SQL JOIN - Combinazione Tabelle}

\section*{Introduzione}
Il JOIN è il meccanismo SQL per combinare dati da più tabelle correlate tramite chiavi esterne. Questo capitolo presenta i diversi tipi di JOIN (INNER, LEFT, RIGHT, FULL OUTER) con diagrammi visivi e esempi pratici per comprendere come e quando usarli.

\section*{Obiettivi di apprendimento}
\begin{itemize}
    \item Comprendere il concetto di JOIN e la necessità di combinare tabelle
    \item Scrivere query con INNER JOIN
    \item Scrivere query con LEFT JOIN e RIGHT JOIN
    \item Scrivere query con FULL OUTER JOIN
    \item Usare CROSS JOIN
    \item Usare alias per tabelle nei JOIN
    \item Eseguire JOIN su più di due tabelle
    \item Evitare cartesiani (combinazioni errate)
\end{itemize}

\section{Concetti di Base}

\subsection{Perché JOIN?}
Nel modello relazionale, i dati sono distribuiti su più tabelle. Per rispondere a domande che coinvolgono dati di più tabelle, devi combinarle con JOIN.

\begin{tcolorbox}[colback=blue!10, colframe=blue!60, title=Esempio: Domanda che richiede JOIN]
``Quali clienti hanno fatto ordini nel 2023?''

Per rispondere devo:
\begin{enumerate}
    \item Trovare ordini con dataOrdine nel 2023 (tabella ordine)
    \item Per ogni ordine, trovare il cliente corrispondente (tabella cliente)
    \item Mostrare i dati del cliente
\end{enumerate}

Questo richiede un JOIN tra ordine e cliente.
\end{tcolorbox}

\subsection{Condizione JOIN}
Un JOIN è basato su una condizione che specifica come abbinare le righe. Tipicamente è un'uguaglianza tra chiave primaria e chiave esterna.

\section{INNER JOIN}

INNER JOIN restituisce solo le righe dove la condizione è vera in ENTRAMBE le tabelle.

\subsection{Sintassi}

\begin{lstlisting}[language=SQL, caption=Sintassi INNER JOIN]
SELECT colonne
FROM tabella1
INNER JOIN tabella2 ON tabella1.chiave = tabella2.chiave
WHERE ...;
\end{lstlisting}

\subsection{Diagramma Venn}

\begin{figure}[h]
    \centering
    \begin{tikzpicture}[scale=1.5]
        % Cerchi
        \draw[thick] (0,0) circle (1);
        \draw[thick] (1.5,0) circle (1);

        % Etichette
        \node at (-0.75, 0) {Tabella 1};
        \node at (2.25, 0) {Tabella 2};

        % Area JOIN (intersezione)
        \begin{scope}
            \clip (0,0) circle (1);
            \fill[red!30] (1.5,0) circle (1);
        \end{scope}

        \node at (0.75, 0) {\small INNER};
        \node at (-2, -1.5) {\textbf{INNER JOIN}};
        \node at (-2, -1.8) {Solo righe comuni};
    \end{tikzpicture}
    \caption{INNER JOIN: intersezione}
\end{figure}

\subsection{Esempio}

\begin{lstlisting}[language=SQL, caption=INNER JOIN]
-- Trovare clienti e loro ordini
SELECT
    c.idCliente,
    c.nome,
    c.cognome,
    o.idOrdine,
    o.dataOrdine,
    o.totale
FROM cliente c
INNER JOIN ordine o ON c.idCliente = o.idCliente;
\end{lstlisting}

Risultato: Solo clienti che hanno fatto almeno un ordine.

\subsection{INNER JOIN su condizioni multiple}

\begin{lstlisting}[language=SQL, caption=INNER JOIN su più colonne]
-- Trovare ordini e i prodotti ordinati
SELECT
    o.idOrdine,
    p.nome AS nomeProdotto,
    op.quantita,
    op.prezzo_unitario
FROM ordine o
INNER JOIN ordine_prodotto op ON o.idOrdine = op.idOrdine
INNER JOIN prodotto p ON op.idProdotto = p.idProdotto;
\end{lstlisting}

\section{LEFT JOIN}

LEFT JOIN restituisce TUTTE le righe della tabella di sinistra e le righe corrispondenti della tabella di destra. Se non c'è corrispondenza, le colonne di destra sono NULL.

\subsection{Diagramma Venn}

\begin{figure}[h]
    \centering
    \begin{tikzpicture}[scale=1.5]
        % Cerchi
        \draw[thick] (0,0) circle (1);
        \draw[thick] (1.5,0) circle (1);

        % Etichette
        \node at (-0.75, 0) {Tabella 1};
        \node at (2.25, 0) {Tabella 2};

        % Area LEFT (tabella sinistra intera)
        \fill[red!30] (0,0) circle (1);

        \node at (-2, -1.5) {\textbf{LEFT JOIN}};
        \node at (-2, -1.8) {Tutte le righe di sinistra};
    \end{tikzpicture}
    \caption{LEFT JOIN: tutte le righe della tabella di sinistra}
\end{figure}

\subsection{Esempio}

\begin{lstlisting}[language=SQL, caption=LEFT JOIN]
-- Trovare TUTTI i clienti e i loro ordini
-- (inclusi clienti senza ordini)
SELECT
    c.idCliente,
    c.nome,
    c.cognome,
    COUNT(o.idOrdine) AS numOrdini,
    COALESCE(SUM(o.totale), 0) AS totalSpeso
FROM cliente c
LEFT JOIN ordine o ON c.idCliente = o.idCliente
GROUP BY c.idCliente, c.nome, c.cognome;
\end{lstlisting}

Risultato: Tutti i clienti, anche quelli senza ordini (che avranno numOrdini=0).

\subsection{Usare LEFT JOIN per trovare record ASSENTI}

\begin{lstlisting}[language=SQL, caption=LEFT JOIN per trovare mancanze]
-- Trovare clienti che NON hanno fatto ordini
SELECT
    c.idCliente,
    c.nome,
    c.cognome
FROM cliente c
LEFT JOIN ordine o ON c.idCliente = o.idCliente
WHERE o.idOrdine IS NULL;
\end{lstlisting}

Filtrando WHERE o.idOrdine IS NULL trovi esattamente i clienti senza ordini.

\section{RIGHT JOIN}

RIGHT JOIN è l'inverso di LEFT JOIN: restituisce tutte le righe della tabella di destra e le corrispondenze della sinistra.

\subsection{Esempio}

\begin{lstlisting}[language=SQL, caption=RIGHT JOIN]
-- Trovare TUTTI i prodotti e quante volte sono stati ordinati
SELECT
    p.idProdotto,
    p.nome,
    COUNT(op.idOrdine) AS volteOrdinato
FROM ordine_prodotto op
RIGHT JOIN prodotto p ON op.idProdotto = p.idProdotto
GROUP BY p.idProdotto, p.nome;
\end{lstlisting}

\begin{tcolorbox}[colback=orange!10, colframe=orange!60, title=Nota: LEFT vs RIGHT}
RIGHT JOIN può sempre essere convertito in LEFT JOIN invertendo l'ordine:

\begin{lstlisting}[language=SQL]
-- Equivalenti:
FROM ordine_prodotto op RIGHT JOIN prodotto p ...
FROM prodotto p LEFT JOIN ordine_prodotto op ...
\end{lstlisting}

Preferisci LEFT JOIN per coerenza (maggior leggibilità).
\end{tcolorbox}

\section{FULL OUTER JOIN}

FULL OUTER JOIN restituisce tutte le righe da ENTRAMBE le tabelle. Se non c'è corrispondenza, le colonne dell'altra tabella sono NULL.

\begin{tcolorbox}[colback=orange!10, colframe=orange!60, title=Nota: MySQL non supporta FULL OUTER JOIN}
MySQL non supporta nativamente FULL OUTER JOIN. Usa un'alternativa con UNION:

\begin{lstlisting}[language=SQL]
SELECT * FROM tabella1 LEFT JOIN tabella2 ...
UNION
SELECT * FROM tabella1 RIGHT JOIN tabella2 ...;
\end{lstlisting}
\end{tcolorbox}

\subsection{Simulare FULL OUTER JOIN in MySQL}

\begin{lstlisting}[language=SQL, caption=FULL OUTER JOIN simulato con UNION]
-- Trovare tutti i dati cliente e ordine
SELECT
    c.idCliente,
    c.nome,
    o.idOrdine,
    o.dataOrdine
FROM cliente c
LEFT JOIN ordine o ON c.idCliente = o.idCliente
UNION
SELECT
    c.idCliente,
    c.nome,
    o.idOrdine,
    o.dataOrdine
FROM cliente c
RIGHT JOIN ordine o ON c.idCliente = o.idCliente
WHERE c.idCliente IS NULL;
\end{lstlisting}

\section{CROSS JOIN}

CROSS JOIN produce il prodotto cartesiano: combina ogni riga della prima tabella con ogni riga della seconda (senza condizione ON).

\subsection{Sintassi}

\begin{lstlisting}[language=SQL, caption=CROSS JOIN]
SELECT *
FROM tabella1
CROSS JOIN tabella2;
\end{lstlisting}

\subsection{Esempio pratico}

\begin{lstlisting}[language=SQL, caption=CROSS JOIN pratico]
-- Generare tutte le combinazioni possibili di prodotti e colori
SELECT
    p.idProdotto,
    p.nome,
    c.colore
FROM prodotto p
CROSS JOIN colore c;

-- Risultato: ogni prodotto combinato con ogni colore
-- Se 100 prodotti e 5 colori, risultato = 500 righe
\end{lstlisting}

\begin{tcolorbox}[colback=red!10, colframe=red!60, title=Attenzione: CROSS JOIN può creare troppi dati!]
Un CROSS JOIN tra tabelle grandi produce MOLTISSIME righe. Usa con cautela!
\end{tcolorbox}

\section{JOIN su Più Tabelle}

Puoi combinare più di due tabelle con più JOIN.

\subsection{Sintassi}

\begin{lstlisting}[language=SQL, caption=JOIN multipli]
SELECT colonne
FROM tabella1
JOIN tabella2 ON ...
JOIN tabella3 ON ...
JOIN tabella4 ON ...
WHERE ...;
\end{lstlisting}

\subsection{Esempio: JOIN su 4 tabelle}

\begin{lstlisting}[language=SQL, caption=JOIN multipli pratico]
-- Trovare cliente, suoi ordini, prodotti ordinati e categorie
SELECT
    c.nome,
    o.dataOrdine,
    p.nome AS nomeProdotto,
    op.quantita,
    cat.nome AS categoria
FROM cliente c
INNER JOIN ordine o ON c.idCliente = o.idCliente
INNER JOIN ordine_prodotto op ON o.idOrdine = op.idOrdine
INNER JOIN prodotto p ON op.idProdotto = p.idProdotto
INNER JOIN categoria cat ON p.idCategoria = cat.idCategoria
WHERE c.idCliente = 5
ORDER BY o.dataOrdine DESC;
\end{lstlisting}

\section{Self-Join}

Un self-join è un join tra una tabella e se stessa. Utile per dati gerarchici o correlati.

\begin{lstlisting}[language=SQL, caption=Self-join]
-- Tabella: dipendente (idDipendente, nome, idDirigente)
-- Trovare ogni dipendente e il nome del suo dirigente

SELECT
    d.nome AS dipendente,
    m.nome AS dirigente
FROM dipendente d
LEFT JOIN dipendente m ON d.idDirigente = m.idDipendente;
\end{lstlisting}

Nota: Usi alias (d e m) per distinguere i due ruoli della stessa tabella.

\section{Alias per Tabelle}

Gli alias rendono le query più leggibili, specialmente con JOIN.

\begin{lstlisting}[language=SQL, caption=Alias per tabelle]
-- Con alias (consigliato)
SELECT
    c.idCliente,
    c.nome,
    COUNT(o.idOrdine) AS numOrdini
FROM cliente c
LEFT JOIN ordine o ON c.idCliente = o.idCliente
GROUP BY c.idCliente, c.nome;

-- Senza alias (lungo e meno leggibile)
SELECT
    cliente.idCliente,
    cliente.nome,
    COUNT(ordine.idOrdine) AS numOrdini
FROM cliente
LEFT JOIN ordine ON cliente.idCliente = ordine.idCliente
GROUP BY cliente.idCliente, cliente.nome;
\end{lstlisting}

\section{Errori Comuni in JOIN}

\subsection{Dimenticare la condizione ON}

\begin{tcolorbox}[colback=red!10, colframe=red!60, title=Errore: Mancante condizione ON]
\begin{lstlisting}[language=SQL]
-- Sbagliato: crea un CROSS JOIN involontario
SELECT * FROM cliente JOIN ordine;

-- Corretto
SELECT * FROM cliente JOIN ordine ON cliente.idCliente = ordine.idCliente;
\end{lstlisting}
\end{tcolorbox}

\subsection{Usare WHERE invece di ON}

\begin{lstlisting}[language=SQL, caption=ON vs WHERE]
-- LEFT JOIN con filtro errato
SELECT *
FROM cliente c
LEFT JOIN ordine o
WHERE c.idCliente = o.idCliente;  -- Sbagliato!
-- Questo filtra dopo il JOIN, trasformando LEFT in INNER

-- Corretto
SELECT *
FROM cliente c
LEFT JOIN ordine o ON c.idCliente = o.idCliente
WHERE o.totale > 100;  -- Filtra dopo JOIN
\end{lstlisting}

\section*{Riepilogo concetti chiave}

\begin{tcolorbox}[colback=gray!10, colframe=black!60, title=Concetti fondamentali]
\begin{itemize}
    \item \textbf{INNER JOIN}: solo righe comuni
    \item \textbf{LEFT JOIN}: tutte le righe di sinistra + corrispondenze
    \item \textbf{RIGHT JOIN}: tutte le righe di destra + corrispondenze
    \item \textbf{FULL OUTER JOIN}: tutte le righe da entrambe (usa UNION in MySQL)
    \item \textbf{CROSS JOIN}: prodotto cartesiano (usa con cautela)
    \item \textbf{Self-join}: tabella con se stessa (usa alias)
    \item Usa \textbf{ON} per la condizione, WHERE per filtri successivi
    \item \textbf{Alias} rendono query complesse più leggibili
\end{itemize}
\end{tcolorbox}

\section*{Esercizi}

\begin{enumerate}
    \item Scrivi un INNER JOIN per trovare ordini e i relativi clienti (nome, cognome) per ordini dell'anno 2023.

    \item Usa LEFT JOIN per trovare TUTTI i clienti e il numero di ordini effettuati. Includi clienti senza ordini (che avranno count=0).

    \item Usa LEFT JOIN con WHERE IS NULL per trovare clienti che NON hanno mai fatto ordini.

    \item Crea un query con 3 JOIN: cliente, ordine, ordine\_prodotto, prodotto. Mostra cliente, numero ordine e prodotto ordinato.

    \item Usa CROSS JOIN per generare tutte le combinazioni di città clienti e categorie prodotti (utilità: analisi di mercato per ogni territorio).

    \item Scrivi un self-join su una tabella dipendente per mostrare ogni dipendente con il nome del suo manager (dirigente).

    \item Analizza questa query e spiega perché LEFT JOIN diventa INNER JOIN:
    \begin{lstlisting}[language=SQL]
SELECT * FROM cliente c
LEFT JOIN ordine o ON c.idCliente = o.idCliente
WHERE o.dataOrdine > '2023-01-01';
    \end{lstlisting}
\end{enumerate}


% Capitolo 10: SQL - Funzioni Aggregate e Raggruppamento
\chapter{SQL Aggregate - COUNT, SUM, AVG, MIN, MAX}

\section*{Introduzione}
Le funzioni di aggregazione calcolano valori singoli da un insieme di righe. Sono essenziali per analisi, report e sintesi dei dati. Questo capitolo presenta COUNT, SUM, AVG, MIN, MAX con GROUP BY e HAVING per elaborazioni complesse.

\section*{Obiettivi di apprendimento}
\begin{itemize}
    \item Comprendere le funzioni di aggregazione di base
    \item Usare COUNT per contare righe
    \item Usare SUM per somme
    \item Usare AVG, MIN, MAX per media, minimo, massimo
    \item Raggruppare dati con GROUP BY
    \item Filtrare gruppi con HAVING
    \item Combinare aggregazioni in query complesse
    \item Gestire NULL nelle aggregazioni
    \item Usare DISTINCT con funzioni di aggregazione
\end{itemize}

\section{Funzioni di Aggregazione di Base}

\subsection{COUNT}

COUNT conta il numero di righe o di valori non NULL.

\begin{lstlisting}[language=SQL, caption=COUNT]
-- Contare tutte le righe
SELECT COUNT(*) AS totalClienti FROM cliente;

-- Contare valori non NULL di una colonna
SELECT COUNT(email) AS clientiConEmail FROM cliente;

-- Contare valori distinti
SELECT COUNT(DISTINCT città) AS numCittà FROM cliente;

-- Contare con condizione
SELECT COUNT(*) AS clientiAttivi FROM cliente WHERE stato = 'Attivo';
\end{lstlisting}

\subsection{SUM}

SUM calcola la somma di una colonna numerica.

\begin{lstlisting}[language=SQL, caption=SUM]
-- Somma totale di tutti gli ordini
SELECT SUM(totale) AS totalVendite FROM ordine;

-- Somma con condizione
SELECT SUM(totale) AS vendite2023 FROM ordine WHERE YEAR(dataOrdine) = 2023;

-- SUM restituisce NULL se nessuna riga (usa COALESCE)
SELECT COALESCE(SUM(totale), 0) AS totalOrdini FROM ordine WHERE stato = 'Cancellato';
\end{lstlisting}

\subsection{AVG}

AVG calcola la media di una colonna numerica.

\begin{lstlisting}[language=SQL, caption=AVG]
-- Media prezzo prodotti
SELECT AVG(prezzo) AS prezzoMedio FROM prodotto;

-- Media ordini per cliente
SELECT AVG(totale) AS importoMedio FROM ordine WHERE stato = 'Completato';

-- Media con rounding
SELECT ROUND(AVG(prezzo), 2) AS prezzoMedio FROM prodotto WHERE categoria = 'Elettronica';
\end{lstlisting}

\subsection{MIN e MAX}

MIN e MAX trovano il valore minimo e massimo.

\begin{lstlisting}[language=SQL, caption=MIN e MAX]
-- Prezzo più basso e più alto
SELECT
    MIN(prezzo) AS prezzoMin,
    MAX(prezzo) AS prezzoMax
FROM prodotto;

-- Cliente con ordine più recente
SELECT MAX(dataOrdine) AS ultimoOrdine FROM ordine;

-- Data primo cliente registrato
SELECT MIN(dataRegistrazione) AS primoCliente FROM cliente;

-- Data cliente più recente e numero giorni da oggi
SELECT
    MAX(dataRegistrazione) AS clientePiuRecente,
    DATEDIFF(CURDATE(), MAX(dataRegistrazione)) AS giorniDaOggi
FROM cliente;
\end{lstlisting}

\section{GROUP BY - Raggruppamento}

GROUP BY raggruppa righe per uno o più attributi e applica funzioni di aggregazione a ogni gruppo.

\subsection{Sintassi}

\begin{lstlisting}[language=SQL, caption=Sintassi GROUP BY]
SELECT colonna_gruppo, AGGREGAZIONE(colonna)
FROM tabella
WHERE condizioni_filtro
GROUP BY colonna_gruppo
ORDER BY ...;
\end{lstlisting}

\subsection{Esempio semplice}

\begin{lstlisting}[language=SQL, caption=GROUP BY semplice]
-- Vendite per ogni cliente
SELECT
    idCliente,
    COUNT(*) AS numOrdini,
    SUM(totale) AS totalSpeso,
    AVG(totale) AS importoMedio
FROM ordine
GROUP BY idCliente;
\end{lstlisting}

\subsection{GROUP BY su più colonne}

\begin{lstlisting}[language=SQL, caption=GROUP BY multiplo]
-- Vendite per città e anno
SELECT
    YEAR(dataOrdine) AS anno,
    c.città,
    COUNT(*) AS numOrdini,
    SUM(o.totale) AS totalVendite
FROM ordine o
JOIN cliente c ON o.idCliente = c.idCliente
GROUP BY YEAR(o.dataOrdine), c.città
ORDER BY anno DESC, totalVendite DESC;
\end{lstlisting}

\subsection{GROUP BY con funzioni di data}

\begin{lstlisting}[language=SQL, caption=GROUP BY con date]
-- Vendite per mese
SELECT
    DATE_TRUNC('month', dataOrdine) AS mese,
    COUNT(*) AS numOrdini,
    SUM(totale) AS totalVendite
FROM ordine
GROUP BY DATE_TRUNC('month', dataOrdine);

-- Alternativa MySQL (senza DATE_TRUNC)
SELECT
    DATE_FORMAT(dataOrdine, '%Y-%m') AS mese,
    COUNT(*) AS numOrdini,
    SUM(totale) AS totalVendite
FROM ordine
GROUP BY DATE_FORMAT(dataOrdine, '%Y-%m')
ORDER BY mese DESC;
\end{lstlisting}

\subsection{Errore comune: colonne non raggruppate}

\begin{tcolorbox}[colback=red!10, colframe=red!60, title=Errore: Colonne non in GROUP BY}
\begin{lstlisting}[language=SQL]
-- Errore in SQL stretto (MySQL con ONLY_FULL_GROUP_BY)
SELECT idCliente, nome, SUM(totale) FROM ordine
GROUP BY idCliente;
-- Errore: nome non è in GROUP BY

-- Corretto
SELECT idCliente, MAX(nome) AS nome, SUM(totale) FROM ordine
GROUP BY idCliente;
\end{lstlisting}

Tutte le colonne SELECT devono essere in GROUP BY o in una funzione di aggregazione.
\end{tcolorbox}

\section{HAVING - Filtro su Aggregazioni}

HAVING filtra i GRUPPI basato su condizioni di aggregazione. È come WHERE, ma per i risultati di GROUP BY.

\subsection{Sintassi}

\begin{lstlisting}[language=SQL, caption=Sintassi HAVING]
SELECT colonna_gruppo, AGGREGAZIONE(colonna)
FROM tabella
GROUP BY colonna_gruppo
HAVING condizioni_su_aggregazioni
ORDER BY ...;
\end{lstlisting}

\subsection{Differenza WHERE vs HAVING}

\begin{tcolorbox}[colback=blue!10, colframe=blue!60, title=WHERE vs HAVING]
\begin{itemize}
    \item \textbf{WHERE}: filtra le RIGHE prima di GROUP BY
    \item \textbf{HAVING}: filtra i GRUPPI dopo GROUP BY
\end{itemize}
\end{tcolorbox}

\subsection{Esempi}

\begin{lstlisting}[language=SQL, caption=HAVING]
-- Clienti che hanno speso più di 1000 euro
SELECT
    idCliente,
    SUM(totale) AS totalSpeso
FROM ordine
GROUP BY idCliente
HAVING SUM(totale) > 1000
ORDER BY totalSpeso DESC;

-- Clienti con più di 5 ordini
SELECT
    idCliente,
    COUNT(*) AS numOrdini
FROM ordine
GROUP BY idCliente
HAVING COUNT(*) > 5;

-- Categorie con prezzo medio superiore a 50 euro
SELECT
    categoria,
    AVG(prezzo) AS prezzoMedio,
    COUNT(*) AS numProdotti
FROM prodotto
GROUP BY categoria
HAVING AVG(prezzo) > 50
ORDER BY prezzoMedio DESC;
\end{lstlisting}

\subsection{WHERE e HAVING insieme}

\begin{lstlisting}[language=SQL, caption=WHERE e HAVING insieme]
-- Clienti di Milano che hanno speso più di 500 euro nel 2023
SELECT
    c.idCliente,
    c.nome,
    SUM(o.totale) AS totalSpeso
FROM ordine o
JOIN cliente c ON o.idCliente = c.idCliente
WHERE c.città = 'Milano'           -- Filtra prima di GROUP BY
    AND YEAR(o.dataOrdine) = 2023
GROUP BY c.idCliente, c.nome
HAVING SUM(o.totale) > 500         -- Filtra dopo GROUP BY
ORDER BY totalSpeso DESC;
\end{lstlisting}

\section{DISTINCT con Aggregazioni}

\subsection{COUNT(DISTINCT)}

Conta valori unici di una colonna.

\begin{lstlisting}[language=SQL, caption=COUNT(DISTINCT)]
-- Numero di città diverse dove abitano i clienti
SELECT COUNT(DISTINCT città) AS numCittà FROM cliente;

-- Numero di categorie di prodotti ordinati
SELECT COUNT(DISTINCT categoria) AS numCategorie
FROM ordine_prodotto op
JOIN prodotto p ON op.idProdotto = p.idProdotto;
\end{lstlisting}

\subsection{SUM(DISTINCT) e AVG(DISTINCT)}

\begin{lstlisting}[language=SQL, caption=SUM(DISTINCT) e AVG(DISTINCT)]
-- Somma dei prezzi unici (ogni prezzo conta una sola volta)
SELECT SUM(DISTINCT prezzo) AS sommaPrezziUnici FROM prodotto;

-- Media dei prezzi unici
SELECT AVG(DISTINCT prezzo) AS mediaPreziUnici FROM prodotto;
\end{lstlisting}

\section{Funzioni Avanzate di Aggregazione}

\subsection{GROUP_CONCAT}

Concatena i valori di una colonna per ogni gruppo (MySQL).

\begin{lstlisting}[language=SQL, caption=GROUP_CONCAT]
-- Elencare i prodotti ordinati per ogni cliente
SELECT
    c.nome,
    GROUP_CONCAT(p.nome SEPARATOR ', ') AS prodottiOrdinati
FROM cliente c
LEFT JOIN ordine o ON c.idCliente = o.idCliente
LEFT JOIN ordine_prodotto op ON o.idOrdine = op.idOrdine
LEFT JOIN prodotto p ON op.idProdotto = p.idProdotto
GROUP BY c.idCliente, c.nome;
\end{lstlisting}

\subsection{Aggregazioni condizionali}

Usare CASE dentro le aggregazioni per logica condizionale.

\begin{lstlisting}[language=SQL, caption=Aggregazioni condizionali]
-- Contare ordini completati e cancellati separatamente
SELECT
    COUNT(*) AS totalOrdini,
    SUM(CASE WHEN stato = 'Completato' THEN 1 ELSE 0 END) AS ordinCompletati,
    SUM(CASE WHEN stato = 'Cancellato' THEN 1 ELSE 0 END) AS ordiniCancellati
FROM ordine;

-- Vendite per stato
SELECT
    SUM(CASE WHEN stato = 'Completato' THEN totale ELSE 0 END) AS venditeCmpl,
    SUM(CASE WHEN stato = 'Spedito' THEN totale ELSE 0 END) AS venditeSpedite,
    SUM(CASE WHEN stato = 'Pendente' THEN totale ELSE 0 END) AS venditePercetti
FROM ordine;
\end{lstlisting}

\section{NULL nelle Aggregazioni}

Le funzioni di aggregazione ignorano i NULL.

\begin{lstlisting}[language=SQL, caption=NULL in aggregazioni]
-- Conteggio con NULL
SELECT
    COUNT(*) AS totalRighe,           -- Conta tutti (incluso NULL)
    COUNT(telefono) AS righeConTelefono  -- Conta solo non-NULL
FROM cliente;

-- Risultato: se 100 clienti, 10 senza telefono
-- totalRighe = 100, righeConTelefono = 90

-- SUM ignora NULL
SELECT SUM(sconto) FROM cliente;
-- I clienti senza sconto (NULL) non contribuiscono alla somma
\end{lstlisting}

\section{Esempio Completo: Report Vendite}

\begin{lstlisting}[language=SQL, caption=Report completo con aggregazioni]
-- Report vendite mensili per ogni categoria
SELECT
    DATE_FORMAT(o.dataOrdine, '%Y-%m') AS mese,
    p.categoria,
    COUNT(DISTINCT o.idOrdine) AS numOrdini,
    SUM(op.quantita) AS totalQuantita,
    SUM(op.quantita * op.prezzo_unitario) AS totalVendite,
    AVG(op.prezzo_unitario) AS prezzoMedio,
    MIN(op.prezzo_unitario) AS prezzoMin,
    MAX(op.prezzo_unitario) AS prezzoMax
FROM ordine o
JOIN ordine_prodotto op ON o.idOrdine = op.idOrdine
JOIN prodotto p ON op.idProdotto = p.idProdotto
WHERE o.stato IN ('Completato', 'Spedito')
GROUP BY DATE_FORMAT(o.dataOrdine, '%Y-%m'), p.categoria
HAVING SUM(op.quantita * op.prezzo_unitario) > 1000
ORDER BY mese DESC, totalVendite DESC;
\end{lstlisting}

\section*{Riepilogo concetti chiave}

\begin{tcolorbox}[colback=gray!10, colframe=black!60, title=Concetti fondamentali]
\begin{itemize}
    \item \textbf{COUNT}: conta righe o valori non-NULL
    \item \textbf{SUM}: somma colonne numeriche
    \item \textbf{AVG, MIN, MAX}: media, minimo, massimo
    \item \textbf{GROUP BY}: raggruppa dati per una o più colonne
    \item \textbf{HAVING}: filtra i gruppi (applicato dopo GROUP BY)
    \item \textbf{WHERE}: filtra le righe (applicato prima di GROUP BY)
    \item \textbf{DISTINCT}: conta o somma solo valori unici
    \item Le aggregazioni ignorano NULL
    \item Combina con JOIN per aggregazioni multi-tabella
\end{itemize}
\end{tcolorbox}

\section*{Esercizi}

\begin{enumerate}
    \item Scrivi una query per contare il numero totale di ordini e il numero di ordini completati.

    \item Calcola il totale speso, la media e il massimo importo di ordini per ogni cliente. Ordina per totale speso decrescente.

    \item Usa GROUP BY per trovare quali categorie di prodotti hanno vendite superiori a 5000 euro. Mostra categoria, numero prodotti, e totale vendite.

    \item Scrivi una query con WHERE e HAVING: trova clienti di Milano che hanno fatto più di 3 ordini nel 2023.

    \item Usa COUNT(DISTINCT) per trovare quanti clienti diversi hanno ordinato ogni prodotto.

    \item Crea un report che mostra, per ogni mese del 2023, il numero di ordini, il totale vendite, la media ordine e il numero di clienti unici.

    \item Usa aggregazioni condizionali (CASE) per mostrare, per ogni categoria, il numero di prodotti attivi e inattivi.

    \item Spiega perché la seguente query dà errore e corriggila:
    \begin{lstlisting}[language=SQL]
SELECT idCliente, nome, COUNT(*) FROM ordine GROUP BY idCliente;
    \end{lstlisting}
\end{enumerate}


% Capitolo 11: SQL - Subquery e Viste
\chapter{SQL Subquery e Viste}

\section*{Introduzione}
Le subquery (query annidate) permettono di usare il risultato di una query come fonte per un'altra query. Le viste sono query salvate che si comportano come tabelle virtuali. Questi strumenti avanzati consentono di scrivere query complesse e riutilizzabili, organizzando la logica in modo strutturato.

\section*{Obiettivi di apprendimento}
\begin{itemize}
    \item Comprendere la differenza tra subquery scalare, riga e tabella
    \item Usare subquery nella clausola SELECT, WHERE e FROM
    \item Usare gli operatori EXISTS, IN, ANY, ALL con subquery
    \item Scrivere query correlate
    \item Creare e gestire viste con CREATE VIEW
    \item Modificare e eliminare viste
    \item Usare viste per semplificare query complesse
    \item Comprendere limitazioni e prestazioni delle viste
\end{itemize}

\section{Subquery - Query Annidate}

Una subquery (o inner query) è una query che si trova dentro un'altra query (outer query). La subquery fornisce dati che la query esterna utilizza.

\subsection{Tipi di Subquery}

\subsubsection{Subquery Scalare}

Una subquery scalare restituisce una singola riga e una singola colonna. Può essere usata dove è previsto un valore singolo (SELECT, WHERE, etc.).

\begin{lstlisting}[language=SQL, caption=Subquery Scalare]
-- Trovare il cliente con l'ordine più recente
SELECT nome, cognome
FROM cliente
WHERE idCliente = (
    -- Subquery scalare: restituisce un singolo idCliente
    SELECT idCliente FROM ordine
    ORDER BY dataOrdine DESC
    LIMIT 1
);

-- Aggiungere il prezzo medio a ogni prodotto
SELECT nome, prezzo,
    (SELECT AVG(prezzo) FROM prodotto) AS prezzoMedio
FROM prodotto;
\end{lstlisting}

\begin{tcolorbox}[colback=green!10, colframe=green!60, title=Nota: Subquery Scalara]
Una subquery scalare deve restituire esattamente una riga e una colonna. Se restituisce 0 righe, il risultato è NULL. Se restituisce più di una riga, causa errore.
\end{tcolorbox}

\subsubsection{Subquery Riga}

Una subquery riga restituisce una singola riga ma più colonne. Utile per confrontare più valori contemporaneamente.

\begin{lstlisting}[language=SQL, caption=Subquery Riga]
-- Trovare ordini con lo stesso cliente e data del primo ordine
SELECT idOrdine, dataOrdine, totale
FROM ordine
WHERE (idCliente, dataOrdine) = (
    -- Subquery riga: restituisce una riga con due colonne
    SELECT idCliente, dataOrdine FROM ordine
    LIMIT 1
);

-- Trovare clienti con stessa città e provincia di Milano
SELECT nome, città, provincia
FROM cliente
WHERE (città, provincia) = (
    SELECT città, provincia FROM cliente
    WHERE città = 'Milano' LIMIT 1
);
\end{lstlisting}

\subsubsection{Subquery Tabella}

Una subquery tabella restituisce più righe e/o più colonne. Spesso usata nella clausola FROM per creare una tabella temporanea (derived table).

\begin{lstlisting}[language=SQL, caption=Subquery Tabella]
-- Trovare prodotti con prezzo sopra la media
SELECT nome, prezzo
FROM prodotto
WHERE prezzo > (
    -- Subquery tabella nella clausola WHERE
    SELECT AVG(prezzo) FROM prodotto
);

-- Usare subquery nella clausola FROM (derived table)
SELECT categoria, numProdotti, prezzoMedio
FROM (
    SELECT categoria, COUNT(*) AS numProdotti, AVG(prezzo) AS prezzoMedio
    FROM prodotto
    GROUP BY categoria
) AS statProdotto
WHERE numProdotti > 5
ORDER BY prezzoMedio DESC;
\end{lstlisting}

\begin{tcolorbox}[colback=yellow!10, colframe=yellow!60, title=Tip: Alias per Derived Table]
Quando usi una subquery nella clausola FROM, devi dare un alias con AS. Questo alias può essere usato nel resto della query per fare riferimento alle colonne della subquery.
\end{tcolorbox}

\section{Operatori di Subquery}

\subsection{Operatore IN}

IN controlla se un valore è contenuto in un insieme di valori restituito dalla subquery.

\begin{lstlisting}[language=SQL, caption=Operatore IN]
-- Trovare clienti che hanno fatto almeno un ordine
SELECT nome, cognome, email
FROM cliente
WHERE idCliente IN (
    -- Subquery: restituisce set di idCliente
    SELECT DISTINCT idCliente FROM ordine
);

-- Trovare ordini di clienti di Roma o Milano
SELECT idOrdine, dataOrdine, totale
FROM ordine
WHERE idCliente IN (
    SELECT idCliente FROM cliente
    WHERE città IN ('Roma', 'Milano')
);

-- NOT IN: clienti senza ordini
SELECT nome, cognome
FROM cliente
WHERE idCliente NOT IN (
    SELECT DISTINCT idCliente FROM ordine
);
\end{lstlisting}

\begin{tcolorbox}[colback=red!10, colframe=red!60, title=Attenzione: IN e NULL]
Se la subquery contiene NULL e usi NOT IN, il risultato sarà sempre vuoto (perché NULL non è comparabile). Usa NOT IN solo se sei sicuro che la subquery non contiene NULL.
\end{tcolorbox}

\subsection{Operatore EXISTS}

EXISTS verifica se la subquery restituisce almeno una riga. È più efficiente di IN per grandi dataset.

\begin{lstlisting}[language=SQL, caption=Operatore EXISTS]
-- Trovare clienti con almeno un ordine (usando EXISTS)
SELECT nome, cognome
FROM cliente c
WHERE EXISTS (
    -- Subquery correlata: fa riferimento a c
    SELECT 1 FROM ordine o
    WHERE o.idCliente = c.idCliente
);

-- Trovare categorie con almeno 3 prodotti
SELECT DISTINCT categoria
FROM prodotto p1
WHERE EXISTS (
    SELECT COUNT(*) FROM prodotto p2
    WHERE p2.categoria = p1.categoria
    GROUP BY p2.categoria
    HAVING COUNT(*) >= 3
);

-- NOT EXISTS: clienti senza ordini
SELECT nome, cognome
FROM cliente c
WHERE NOT EXISTS (
    SELECT 1 FROM ordine o
    WHERE o.idCliente = c.idCliente
);
\end{lstlisting}

\begin{tcolorbox}[colback=blue!10, colframe=blue!60, title=Subquery Correlata]
Una subquery correlata fa riferimento a colonne della query esterna. Viene eseguita una volta per ogni riga della tabella esterna. È meno efficiente della versione non correlata, ma talvolta necessaria.
\end{tcolorbox}

\subsection{Operatori ANY e ALL}

ANY (o SOME) verifica se il confronto è vero per almeno un valore. ALL verifica se è vero per tutti i valori.

\begin{lstlisting}[language=SQL, caption=Operatori ANY e ALL]
-- Trovare ordini con totale maggiore di QUALSIASI ordine di Roma
SELECT idOrdine, totale
FROM ordine o
WHERE totale > ANY (
    SELECT totale FROM ordine
    WHERE idCliente IN (SELECT idCliente FROM cliente WHERE città = 'Roma')
);

-- Trovare ordini con totale MINORE della media per categoria
SELECT idOrdine, totale
FROM ordine o
WHERE totale < ALL (
    -- ALL: minore di TUTTI i totali medi per categoria
    SELECT AVG(totale) FROM ordine
    GROUP BY idCliente
);

-- ANY equivalente a IN
SELECT * FROM ordine WHERE idCliente = ANY (
    SELECT idCliente FROM cliente WHERE città = 'Milano'
);
-- Equivalente a:
SELECT * FROM ordine WHERE idCliente IN (
    SELECT idCliente FROM cliente WHERE città = 'Milano'
);
\end{lstlisting}

\section{Viste (Views)}

Una vista è una query salvata che può essere usata come una tabella virtuale, rappresentando uno strumento potente e versatile nella progettazione di database. Le viste offrono molteplici vantaggi strategici per la gestione dei dati. Innanzitutto, semplificano notevolmente le query complesse, permettendo di incapsulare logica di join e filtri sofisticati in un oggetto riutilizzabile, evitando così la duplicazione di codice SQL complesso in tutta l'applicazione. Questa capacità di riutilizzo della logica comune non solo riduce gli errori, ma facilita anche la manutenzione: modificare una vista aggiorna automaticamente tutte le query che la utilizzano. Le viste forniscono inoltre preziosi livelli di astrazione, nascondendo i dettagli di implementazione delle tabelle sottostanti e permettendo di modificare la struttura fisica del database senza impattare le applicazioni che utilizzano le viste. Infine, le viste rappresentano un meccanismo efficace per gestire la sicurezza dei dati, consentendo di concedere agli utenti l'accesso solo a specifiche colonne o righe filtrate, implementando così il principio del minimo privilegio senza replicare i dati.

\subsection{CREATE VIEW}

Il comando CREATE VIEW crea una nuova vista salvata nel database.

\begin{lstlisting}[language=SQL, caption=Sintassi CREATE VIEW]
CREATE VIEW nome_vista AS
SELECT colonne
FROM tabelle
WHERE condizioni;
\end{lstlisting}

\begin{lstlisting}[language=SQL, caption=Esempi CREATE VIEW]
-- Vista semplice: clienti attivi
CREATE VIEW clientiAttivi AS
SELECT idCliente, nome, cognome, email, città
FROM cliente
WHERE stato = 'Attivo';

-- Vista con aggregazione: totale ordini per cliente
CREATE VIEW totaleOrdiniPerCliente AS
SELECT
    c.idCliente,
    c.nome,
    c.cognome,
    COUNT(o.idOrdine) AS numOrdini,
    SUM(o.totale) AS totalSpeso,
    AVG(o.totale) AS ordineMediano
FROM cliente c
LEFT JOIN ordine o ON c.idCliente = o.idCliente
GROUP BY c.idCliente, c.nome, c.cognome;

-- Vista con JOIN: informazioni ordine complete
CREATE VIEW ordineCompleto AS
SELECT
    o.idOrdine,
    o.dataOrdine,
    o.totale,
    c.nome AS nomeCliente,
    c.cognome AS cognomeCliente,
    c.email,
    p.nome AS nomeProdotto,
    d.quantità,
    d.prezzoUnitario,
    d.subtotale
FROM ordine o
INNER JOIN cliente c ON o.idCliente = c.idCliente
INNER JOIN dettaglioOrdine d ON o.idOrdine = d.idOrdine
INNER JOIN prodotto p ON d.idProdotto = p.idProdotto;
\end{lstlisting}

\subsection{Usare le Viste}

Le viste si usano come tabelle normali nella clausola FROM.

\begin{lstlisting}[language=SQL, caption=Usare le Viste]
-- Usare una vista come una tabella
SELECT nome, cognome, email
FROM clientiAttivi
WHERE città = 'Milano';

-- Aggregare dati da una vista
SELECT SUM(totalSpeso) AS totalVendite
FROM totaleOrdiniPerCliente
WHERE numOrdini > 5;

-- JOIN con una vista
SELECT oc.nome, oc.numOrdini, oc.totalSpeso
FROM ordineCompleto oc
WHERE oc.totale > 100
ORDER BY oc.totale DESC;
\end{lstlisting}

\subsection{Modificare le Viste}

\subsubsection{ALTER VIEW}

ALTER VIEW modifica la definizione di una vista esistente.

\begin{lstlisting}[language=SQL, caption=ALTER VIEW]
-- Modificare la definizione di una vista
ALTER VIEW clientiAttivi AS
SELECT idCliente, nome, cognome, email, città, dataRegistrazione
FROM cliente
WHERE stato = 'Attivo'
AND dataRegistrazione > DATE_SUB(NOW(), INTERVAL 1 YEAR);
\end{lstlisting}

\subsubsection{DROP VIEW}

DROP VIEW elimina una vista dal database.

\begin{lstlisting}[language=SQL, caption=DROP VIEW]
-- Eliminare una vista
DROP VIEW clientiAttivi;

-- Eliminare solo se esiste (evita errori)
DROP VIEW IF EXISTS clientiAttivi;

-- Eliminare multiple viste
DROP VIEW IF EXISTS clientiAttivi, totaleOrdiniPerCliente, ordineCompleto;
\end{lstlisting}

\subsection{SHOW VIEWS}

Per visualizzare tutte le viste nel database.

\begin{lstlisting}[language=SQL, caption=Visualizzare le Viste]
-- Elencare tutte le viste nel database attuale
SHOW FULL TABLES WHERE table_type = 'VIEW';

-- Visualizzare la definizione di una vista
SHOW CREATE VIEW clientiAttivi;

-- Interrogare il catalogo di sistema
SELECT TABLE_NAME FROM INFORMATION_SCHEMA.TABLES
WHERE TABLE_SCHEMA = 'database_name'
AND TABLE_TYPE = 'VIEW';
\end{lstlisting}

\section{Viste Aggiornabili (Updatable Views)}

Una vista è aggiornabile se contiene UPDATE o DELETE, purché soddisfi certe condizioni:
\begin{itemize}
    \item Nessun UNION, GROUP BY, HAVING, LIMIT
    \item Nessuna subquery
    \item Nessuna aggregazione
    \item Può contenere solo una tabella (con JOIN non è aggiornabile)
\end{itemize}

\begin{lstlisting}[language=SQL, caption=Viste Aggiornabili]
-- Vista aggiornabile: tutti i campi di cliente
CREATE VIEW clientiViewAggiornabile AS
SELECT idCliente, nome, cognome, email, città
FROM cliente;

-- Aggiornare tramite vista
UPDATE clientiViewAggiornabile
SET città = 'Roma'
WHERE idCliente = 1;

-- Equivalente all'aggiornamento della tabella sottostante
UPDATE cliente SET città = 'Roma' WHERE idCliente = 1;

-- Inserire tramite vista
INSERT INTO clientiViewAggiornabile (nome, cognome, email, città)
VALUES ('Rossi', 'Mario', 'mario@example.com', 'Milano');

-- Vista NON aggiornabile (con GROUP BY)
CREATE VIEW totaleOrdiniPerCliente AS
SELECT
    idCliente,
    COUNT(*) AS numOrdini,
    SUM(totale) AS totalSpeso
FROM ordine
GROUP BY idCliente;

-- ERRORE: non è possibile aggiornare perché contiene GROUP BY
-- UPDATE totaleOrdiniPerCliente SET numOrdini = 5 WHERE idCliente = 1;
\end{lstlisting}

\section{Viste Materialized (Snapshot)}

Una vista materializzata è una copia dei dati della query salvata fisicamente nel database. È più veloce per query complesse, ma non è sempre aggiornata.

\begin{lstlisting}[language=SQL, caption=Simulare Viste Materializzate]
-- Creare una tabella per immagazzinare i dati della vista
CREATE TABLE totaleOrdiniPerClienteMaterializzato AS
SELECT
    idCliente,
    nome,
    cognome,
    COUNT(*) AS numOrdini,
    SUM(totale) AS totalSpeso
FROM cliente c
LEFT JOIN ordine o ON c.idCliente = o.idCliente
GROUP BY idCliente, nome, cognome;

-- Aggiornare periodicamente i dati (es. ogni notte)
-- Cancellare i dati vecchi
TRUNCATE TABLE totaleOrdiniPerClienteMaterializzato;

-- Ricaricare i dati aggiornati
INSERT INTO totaleOrdiniPerClienteMaterializzato
SELECT
    idCliente,
    nome,
    cognome,
    COUNT(*) AS numOrdini,
    SUM(totale) AS totalSpeso
FROM cliente c
LEFT JOIN ordine o ON c.idCliente = o.idCliente
GROUP BY idCliente, nome, cognome;
\end{lstlisting}

\begin{tcolorbox}[colback=purple!10, colframe=purple!60, title=Vantaggi e Svantaggi Viste]
\textbf{Vantaggi:}

Le viste apportano benefici significativi alla progettazione e manutenzione del database. Semplificano notevolmente query complesse incapsulando join multipli e logica sofisticata in un singolo oggetto riutilizzabile. Questo migliora la leggibilità del codice, poiché gli sviluppatori possono riferirsi a viste con nomi semantici invece di dover comprendere query SQL complesse. Le viste facilitano la manutenzione centralizzando la logica: modificare una vista aggiorna automaticamente tutte le applicazioni che la utilizzano, eliminando la necessità di modificare query duplicate in più luoghi. Inoltre, nascondono efficacemente la complessità implementativa delle tabelle sottostanti, fornendo un'interfaccia stabile anche quando la struttura fisica del database cambia.

\textbf{Svantaggi:}

Nonostante i vantaggi, le viste presentano anche limitazioni che devono essere considerate. Le viste non materializzate possono avere prestazioni peggiori per query particolarmente complicate, poiché la query della vista viene eseguita ogni volta che la vista viene interrogata, moltiplicando il lavoro computazionale. Questo overhead può diventare significativo quando si innestano viste su altre viste, creando query estremamente complesse. Le viste possono anche essere difficili da debuggare, specialmente quando gli errori si propagano attraverso più livelli di astrazione, rendendo complessa l'identificazione della causa radice di problemi di performance o risultati errati. Infine, le viste creano dipendenze sulle tabelle sottostanti: modificare la struttura di una tabella base può richiedere di aggiornare tutte le viste che vi fanno riferimento, creando un accoppiamento che deve essere gestito attentamente durante l'evoluzione dello schema.
\end{tcolorbox}

\section{Riepilogo Concetti Chiave}

\begin{description}
    \item[\textbf{Subquery}] Query annidata che fornisce dati a un'altra query. Può essere scalare (1 valore), riga (1 riga) o tabella (più righe/colonne).
    \item[\textbf{IN}] Operatore che verifica se un valore è in un insieme. Efficiente per piccoli set.
    \item[\textbf{EXISTS}] Operatore che verifica se una subquery restituisce righe. Efficiente per grandi dataset.
    \item[\textbf{ANY/ALL}] Operatori che confrontano un valore con qualsiasi/tutti i valori di un insieme.
    \item[\textbf{Vista}] Query salvata che si comporta come una tabella virtuale.
    \item[\textbf{Subquery Correlata}] Subquery che fa riferimento a colonne della query esterna.
    \item[\textbf{Derived Table}] Subquery nella clausola FROM con alias.
\end{description}


% Capitolo 12: Transazioni e Concorrenza
\chapter{Transazioni e Controllo della Concorrenza}

\section*{Introduzione}
Una transazione è un'unità di lavoro atomica che raggruppa una o più operazioni SQL. Le transazioni garantiscono la consistenza del database anche in caso di errori o guasti. Questo capitolo presenta i concetti ACID, comandi di controllo delle transazioni (START, COMMIT, ROLLBACK, SAVEPOINT) e i livelli di isolamento per gestire la concorrenza.

\section*{Obiettivi di apprendimento}
\begin{itemize}
    \item Comprendere il concetto di transazione e la sua importanza
    \item Comprendere le proprietà ACID
    \item Usare i comandi START TRANSACTION, COMMIT, ROLLBACK
    \item Usare SAVEPOINT per rollback parziali
    \item Comprendere i livelli di isolamento (READ UNCOMMITTED, READ COMMITTED, REPEATABLE READ, SERIALIZABLE)
    \item Gestire anomalie di concorrenza (dirty read, non-repeatable read, phantom read)
    \item Identificare e risolvere deadlock
    \item Configurare e monitorare transazioni
\end{itemize}

\section{Proprietà ACID}

Le proprietà ACID garantiscono che le transazioni siano affidabili e mantengono la integrità del database.

\begin{description}
    \item[\textbf{Atomicità (Atomicity)}] Una transazione è "tutto o nulla". O tutte le operazioni vengono completate (COMMIT), o nessuna (ROLLBACK). Non può esistere uno stato intermedio.

    \item[\textbf{Coerenza (Consistency)}] Le transazioni portano il database da uno stato coerente a un altro. Non è possibile violare vincoli di integrità (PK, FK, CHECK).

    \item[\textbf{Isolamento (Isolation)}] Le transazioni concorrenti non si interferiscono. Ogni transazione vede il database come se fosse l'unica in esecuzione.

    \item[\textbf{Durabilità (Durability)}] Una volta che una transazione è COMMIT, i dati sono permanentemente salvati, anche in caso di crash o blackout.
\end{description}

\begin{tcolorbox}[colback=blue!10, colframe=blue!60, title=Esempio ACID]
Un trasferimento di denaro tra conti:
\begin{enumerate}
    \item \textbf{Atomicità}: Debito dal conto A E credito nel conto B avvengono insieme. Se uno fallisce, entrambi falliscono.
    \item \textbf{Coerenza}: Il totale di denaro nei due conti rimane costante prima e dopo il trasferimento.
    \item \textbf{Isolamento}: Altre transazioni non vedono lo stato intermedio (A diminuito, B non ancora aumentato).
    \item \textbf{Durabilità}: Dopo COMMIT, il trasferimento è permanente anche se il server crasha un secondo dopo.
\end{enumerate}
\end{tcolorbox}

\section{Comandi di Controllo Transazioni}

\subsection{START TRANSACTION}

Inizia una nuova transazione. Tutte le operazioni successive fanno parte della transazione finché non viene COMMIT o ROLLBACK.

\begin{lstlisting}[language=SQL, caption=Sintassi START TRANSACTION]
START TRANSACTION;
-- Oppure in alcuni DBMS:
BEGIN TRANSACTION;
BEGIN;
\end{lstlisting}

\subsection{COMMIT}

COMMIT salva permanentemente tutte le modifiche della transazione nel database.

\begin{lstlisting}[language=SQL, caption=Esempio COMMIT]
START TRANSACTION;

-- Operazione 1: Aggiornare saldo cliente
UPDATE cliente SET saldo = saldo - 100 WHERE idCliente = 1;

-- Operazione 2: Registrare la transazione
INSERT INTO movimento (idCliente, tipo, importo, data)
VALUES (1, 'Prelievo', 100, NOW());

-- Se tutto è corretto, salvare permanentemente
COMMIT;

-- Dopo COMMIT, i dati sono permanenti
SELECT saldo FROM cliente WHERE idCliente = 1;  -- saldo ridotto di 100
\end{lstlisting}

\subsection{ROLLBACK}

ROLLBACK annulla tutte le modifiche della transazione. Il database torna allo stato prima di START TRANSACTION.

\begin{lstlisting}[language=SQL, caption=Esempio ROLLBACK]
START TRANSACTION;

-- Operazione 1
UPDATE cliente SET saldo = saldo - 100 WHERE idCliente = 1;

-- Operazione 2
INSERT INTO movimento (idCliente, tipo, importo, data)
VALUES (1, 'Prelievo', 100, NOW());

-- Se si verifica un errore o vuoi annullare
ROLLBACK;

-- Dopo ROLLBACK, i dati tornano allo stato precedente
SELECT saldo FROM cliente WHERE idCliente = 1;  -- saldo invariato
\end{lstlisting}

\begin{tcolorbox}[colback=red!10, colframe=red!60, title=Attenzione: ROLLBACK è Definitivo]
Una volta che eseguito ROLLBACK, le modifiche sono perse per sempre. Non è possibile "undo del ROLLBACK".
\end{tcolorbox}

\subsection{SAVEPOINT}

SAVEPOINT crea un punto di salvataggio all'interno di una transazione. Permette di fare ROLLBACK parziale fino a quel punto, mantenendo le operazioni precedenti.

\begin{lstlisting}[language=SQL, caption=Sintassi SAVEPOINT]
SAVEPOINT nome_savepoint;

-- Fare rollback fino a un savepoint
ROLLBACK TO SAVEPOINT nome_savepoint;

-- Rilasciare un savepoint (liberare memoria)
RELEASE SAVEPOINT nome_savepoint;
\end{lstlisting}

\begin{lstlisting}[language=SQL, caption=Esempio SAVEPOINT]
START TRANSACTION;

-- Operazione 1: Debito dal cliente
UPDATE cliente SET saldo = saldo - 100 WHERE idCliente = 1;
INSERT INTO movimento VALUES (1, 'Prelievo', 100, NOW());

-- Creare un savepoint
SAVEPOINT dopo_prelievo;

-- Operazione 2: Credito al cliente 2
UPDATE cliente SET saldo = saldo + 100 WHERE idCliente = 2;

-- Se si verifica un errore con il cliente 2
-- Fare rollback solo all'ultima operazione
ROLLBACK TO SAVEPOINT dopo_prelievo;

-- Il prelievo dal cliente 1 rimane
-- Ma il credito al cliente 2 è annullato

-- Riprovare l'operazione 2 con cliente diverso
UPDATE cliente SET saldo = saldo + 100 WHERE idCliente = 3;
INSERT INTO movimento VALUES (3, 'Credito', 100, NOW());

-- Se tutto è corretto, salvare permanentemente
COMMIT;
\end{lstlisting}

\section{Livelli di Isolamento}

Il livello di isolamento determina come le transazioni concorrenti interagiscono. Un isolamento più alto garantisce meno anomalie, ma riduce le prestazioni.

\subsection{Anomalie di Concorrenza}

Prima di descrivere i livelli, ecco le anomalie che possono verificarsi:

\begin{description}
    \item[\textbf{Dirty Read}] Una transazione legge dati non ancora committati scritti da un'altra transazione. Se quella transazione fa ROLLBACK, i dati letti erano "sporchi".

    \item[\textbf{Non-Repeatable Read}] Una transazione legge la stessa riga due volte e ottiene risultati diversi perché un'altra transazione ha modificato i dati nel mezzo.

    \item[\textbf{Phantom Read}] Una transazione legge un insieme di righe due volte e ottiene risultati diversi perché un'altra transazione ha inserito/eliminato righe nel mezzo.
\end{description}

\subsection{Livelli di Isolamento in SQL}

\subsubsection{READ UNCOMMITTED}

Livello di isolamento più basso. Una transazione può leggere dati non ancora committati (dirty read).

\begin{lstlisting}[language=SQL, caption=READ UNCOMMITTED]
-- Impostare il livello di isolamento
SET TRANSACTION ISOLATION LEVEL READ UNCOMMITTED;
START TRANSACTION;

-- Lettura "sporca": legge dati non committati
SELECT saldo FROM cliente WHERE idCliente = 1;

COMMIT;
\end{lstlisting}

\begin{tcolorbox}[colback=red!10, colframe=red!60, title=Attenzione: READ UNCOMMITTED è Rischioso]
Questo livello è molto pericoloso per dati critici (finanze, inventario). Usa solo in scenari dove precisione non è critica (report, cache).
\end{tcolorbox}

\subsubsection{READ COMMITTED}

Livello intermedio. Una transazione legge solo dati committati, evitando dirty read. Ma permite non-repeatable read e phantom read.

\begin{lstlisting}[language=SQL, caption=READ COMMITTED]
SET TRANSACTION ISOLATION LEVEL READ COMMITTED;
START TRANSACTION;

-- Legge solo dati committati
SELECT saldo FROM cliente WHERE idCliente = 1;

-- Altra transazione potrebbe modificare il saldo nel frattempo
SELECT saldo FROM cliente WHERE idCliente = 1;  -- Potrebbe essere diverso

COMMIT;
\end{lstlisting}

\subsubsection{REPEATABLE READ}

Una transazione legge gli stessi dati sempre allo stesso modo durante la transazione. Evita dirty read e non-repeatable read, ma permite phantom read (inserimenti/eliminazioni di nuove righe).

\begin{lstlisting}[language=SQL, caption=REPEATABLE READ]
SET TRANSACTION ISOLATION LEVEL REPEATABLE READ;
START TRANSACTION;

-- Prima lettura
SELECT COUNT(*) FROM ordine WHERE idCliente = 1;  -- Es: 5

-- Altra transazione inserisce un nuovo ordine
SELECT COUNT(*) FROM ordine WHERE idCliente = 1;  -- Ancora 5 (phantom read possibile se cambiano colonne)

COMMIT;
\end{lstlisting}

\subsubsection{SERIALIZABLE}

Livello di isolamento più alto. Le transazioni si comportano come se fossero eseguite in sequenza (serialmente), non concorrentemente.

\begin{lstlisting}[language=SQL, caption=SERIALIZABLE]
SET TRANSACTION ISOLATION LEVEL SERIALIZABLE;
START TRANSACTION;

-- Nessuna anomalia: dirty read, non-repeatable read, phantom read
SELECT saldo FROM cliente WHERE idCliente = 1;

-- ... altre operazioni ...

COMMIT;
\end{lstlisting}

\begin{tcolorbox}[colback=purple!10, colframe=purple!60, title=Trade-off: Isolamento vs Prestazioni]
\begin{itemize}
    \item READ UNCOMMITTED: Massime prestazioni, rischio massimo di anomalie
    \item READ COMMITTED: Buon compromesso, livello di default in molti DBMS
    \item REPEATABLE READ: Isolamento quasi completo, piccolo calo prestazioni
    \item SERIALIZABLE: Massimo isolamento, prestazioni minime (possibili deadlock)
\end{itemize}
\end{tcolorbox}

\subsection{Impostare il Livello di Isolamento}

\begin{lstlisting}[language=SQL, caption=Impostare Livello di Isolamento]
-- Impostare per la sessione attuale
SET TRANSACTION ISOLATION LEVEL READ COMMITTED;

-- Impostare per una transazione specifica (MySQL 5.7+)
SET TRANSACTION ISOLATION LEVEL SERIALIZABLE;
START TRANSACTION;
-- ... operazioni ...
COMMIT;

-- Visualizzare il livello attuale
SELECT @@transaction_isolation;

-- Impostare globalmente (richiede SUPER privilege)
SET GLOBAL transaction_isolation='READ-COMMITTED';
\end{lstlisting}

\section{Deadlock}

Un deadlock si verifica quando due o più transazioni si bloccano reciprocamente, aspettando risorse che l'altra transazione detiene. Nessuna può procedere.

\subsection{Causa Comune di Deadlock}

\begin{lstlisting}[language=SQL, caption=Scenario Deadlock]
-- Transazione 1
START TRANSACTION;
UPDATE cliente SET saldo = saldo - 100 WHERE idCliente = 1;
-- Aspetta il lock su cliente 2
UPDATE cliente SET saldo = saldo + 100 WHERE idCliente = 2;
COMMIT;

-- Transazione 2 (parallelamente)
START TRANSACTION;
UPDATE cliente SET saldo = saldo - 100 WHERE idCliente = 2;
-- Aspetta il lock su cliente 1 (DEADLOCK!)
UPDATE cliente SET saldo = saldo + 100 WHERE idCliente = 1;
COMMIT;
\end{lstlisting}

\subsection{Evitare Deadlock}

\begin{lstlisting}[language=SQL, caption=Strategie per Evitare Deadlock]
-- Strategie 1: Sempre aggiornare le righe nello stesso ordine
-- Transazione 1 e 2: aggiornare sempre cliente 1 prima di cliente 2
START TRANSACTION;
UPDATE cliente SET saldo = saldo - 100 WHERE idCliente = 1;  -- Sempre prima
UPDATE cliente SET saldo = saldo + 100 WHERE idCliente = 2;  -- Sempre dopo
COMMIT;

-- Strategia 2: Usare timeout per il lock
SET innodb_lock_wait_timeout = 5;  -- Timeout dopo 5 secondi

-- Strategia 3: Usare transaction_isolation più bassa (se accettabile)
SET TRANSACTION ISOLATION LEVEL READ COMMITTED;

-- Strategia 4: Minimizzare la durata della transazione
START TRANSACTION;
-- Fare il meno possibile
UPDATE cliente SET saldo = saldo - 100 WHERE idCliente = 1;
COMMIT;  -- Rilasciare i lock il prima possibile
\end{lstlisting}

\subsection{Gestire Deadlock in Applicazione}

\begin{lstlisting}[language=SQL, caption=Gestione Deadlock in Pseudocodice]
-- Pseudocodice per gestire deadlock
MAX_RETRIES = 3
retry_count = 0

WHILE retry_count < MAX_RETRIES:
    TRY:
        START TRANSACTION
        -- Operazioni
        UPDATE cliente SET saldo = ... WHERE idCliente = 1
        UPDATE cliente SET saldo = ... WHERE idCliente = 2
        COMMIT
        BREAK  -- Successo, esci dal loop
    EXCEPT Deadlock:
        retry_count += 1
        ROLLBACK
        WAIT(random(1, 5) seconds)  -- Attendi con backoff casuale
    EXCEPT OTHER_ERROR:
        ROLLBACK
        RAISE error

IF retry_count >= MAX_RETRIES:
    RAISE "Deadlock non risolto dopo 3 tentativi"
\end{lstlisting}

\section{Monitoraggio Transazioni}

\subsection{Visualizzare Transazioni Attive}

\begin{lstlisting}[language=SQL, caption=Monitorare Transazioni Active]
-- Visualizzare tutti i processi in esecuzione
SHOW PROCESSLIST;

-- Dettagli completi
SHOW FULL PROCESSLIST;

-- Mostrare solo transazioni lunghe
SELECT * FROM INFORMATION_SCHEMA.PROCESSLIST
WHERE TIME > 300  -- Più di 5 minuti
AND COMMAND != 'Sleep';

-- Visualizzare transazioni InnoDB
SELECT * FROM INFORMATION_SCHEMA.INNODB_TRX;

-- Visualizzare lock InnoDB
SELECT * FROM INFORMATION_SCHEMA.INNODB_LOCKS;
\end{lstlisting}

\subsection{Terminare Transazioni}

\begin{lstlisting}[language=SQL, caption=Terminare Transazioni o Processi]
-- Terminare un processo specifico
KILL 123;  -- Dove 123 è l'ID del processo

-- Terminare tutti i processi di un utente
KILL QUERY 456;  -- Termina solo la query, non la connessione

-- Script per killare processi lunghi
SELECT CONCAT('KILL ', ID, ';')
FROM INFORMATION_SCHEMA.PROCESSLIST
WHERE TIME > 300 AND COMMAND != 'Sleep';
\end{lstlisting}

\section{Transazioni Implicite vs Esplicite}

\subsection{Modalità Autocommit}

La modalità autocommit determina se i comandi SQL sono automaticamente committati.

\begin{lstlisting}[language=SQL, caption=Autocommit]
-- Visualizzare stato autocommit
SELECT @@autocommit;  -- 1 = attivo, 0 = disattivo

-- Disattivare autocommit (transazioni esplicite)
SET autocommit = 0;

UPDATE cliente SET saldo = saldo - 100 WHERE idCliente = 1;
-- Modifica non è ancora salvata finché non fai COMMIT
COMMIT;

-- Attivare autocommit (transazioni implicite)
SET autocommit = 1;

UPDATE cliente SET saldo = saldo - 100 WHERE idCliente = 1;
-- Modifica è automaticamente salvata immediatamente
\end{lstlisting}

\section{Riepilogo Concetti Chiave}

\begin{description}
    \item[\textbf{Transazione}] Unità di lavoro atomica che raggruppa una o più operazioni SQL.
    \item[\textbf{ACID}] Proprietà che garantiscono affidabilità: Atomicità, Coerenza, Isolamento, Durabilità.
    \item[\textbf{COMMIT}] Comando che salva permanentemente le modifiche della transazione.
    \item[\textbf{ROLLBACK}] Comando che annulla tutte le modifiche della transazione.
    \item[\textbf{SAVEPOINT}] Punto di salvataggio per rollback parziale all'interno di una transazione.
    \item[\textbf{Livello di Isolamento}] Grado in cui le transazioni sono isolate l'una dall'altra.
    \item[\textbf{Deadlock}] Situazione in cui due transazioni aspettano reciprocamente, bloccandosi a vicenda.
    \item[\textbf{Dirty Read}] Anomalia: leggere dati non committati.
    \item[\textbf{Non-Repeatable Read}] Anomalia: leggere dati diversi nella stessa transazione.
    \item[\textbf{Phantom Read}] Anomalia: ottenere risultati diversi per lo stesso query a causa di inserimenti/eliminazioni.
\end{description}


% Capitolo 13: Esercizi Completi
\chapter{Esercizi Progressivi}

\section{Livello Base}

\begin{esercizio}[B.1 --- Hello World]
Scrivere un programma che stampa "Hello, Assembly!" usando INT 21h funzione 09h.
\end{esercizio}

\begin{esercizio}[B.2 --- Somma Due Numeri]
Leggere due numeri a 16 bit, sommarli e stampare il risultato.
\end{esercizio}

\begin{esercizio}[B.3 --- Pari o Dispari]
Leggere un numero e determinare se è pari o dispari (testare bit 0).
\end{esercizio}

\begin{esercizio}[B.4 --- Massimo di Tre]
Dati tre numeri, trovare il massimo.
\end{esercizio}

\begin{esercizio}[B.5 --- Somma Array]
Sommare tutti gli elementi di un array di 10 word.
\end{esercizio}

\begin{esercizio}[B.6 --- Copia Stringa]
Copiare una stringa usando MOVSB.
\end{esercizio}

\section{Livello Intermedio}

\begin{esercizio}[I.1 --- Fattoriale]
Calcolare il fattoriale di n (versione iterativa e ricorsiva).
\end{esercizio}

\begin{esercizio}[I.2 --- Fibonacci]
Generare i primi n numeri di Fibonacci.
\end{esercizio}

\begin{esercizio}[I.3 --- Numero Primo]
Verificare se un numero è primo.
\end{esercizio}

\begin{esercizio}[I.4 --- Ricerca Binaria]
Implementare binary search su array ordinato.
\end{esercizio}

\begin{esercizio}[I.5 --- Conversione Decimale-Esadecimale]
Convertire numero decimale in esadecimale e stamparlo.
\end{esercizio}

\begin{esercizio}[I.6 --- Palindromo]
Verificare se una stringa è palindroma.
\end{esercizio}

\begin{esercizio}[I.7 --- Ordinamento Insertion Sort]
Implementare insertion sort su array.
\end{esercizio}

\section{Livello Avanzato}

\begin{esercizio}[A.1 --- Torre di Hanoi]
Risolvere Torre di Hanoi con ricorsione.
\end{esercizio}

\begin{esercizio}[A.2 --- Moltiplicazione Matrici]
Moltiplicare due matrici 3	imes3.
\end{esercizio}

\begin{esercizio}[A.3 --- Parser Espressioni]
Valutare espressioni aritmetiche con precedenza operatori.
\end{esercizio}

\begin{esercizio}[A.4 --- Allocatore Memoria]
Implementare malloc/free semplificato.
\end{esercizio}

\begin{esercizio}[A.5 --- Compressione RLE]
Implementare compressione Run-Length Encoding.
\end{esercizio}

\begin{esercizio}[A.6 --- File System FAT12]
Leggere e parsare directory FAT12 da floppy disk.
\end{esercizio}

\begin{esercizio}[A.7 --- Debugger Minimale]
Implementare debugger con breakpoint e single-step.
\end{esercizio}

\section{Progetti Integrati}

\begin{esercizio}[P.1 --- Sistema Operativo Minimale]
Creare un micro-OS con:
\begin{itemize}
    \item Bootloader
    \item Gestore interrupt
    \item Shell minimale
    \item Driver VGA e tastiera
\end{itemize}
\end{esercizio}

\begin{esercizio}[P.2 --- Emulatore 8086]
Scrivere emulatore software dell'8086 in C/Python.
\end{esercizio}

\begin{esercizio}[P.3 --- Crittografia]
Implementare cifratura/decifratura Caesar, XOR, ROT13.
\end{esercizio}

\begin{esercizio}[P.4 --- Gioco Tetris]
Implementare Tetris in modalità testo.
\end{esercizio}

\begin{esercizio}[P.5 --- Comunicazione Seriale]
Implementare protocollo seriale per comunicazione PC-PC.
\end{esercizio}

\section{Sfide Avanzate}

\begin{esercizio}[S.1 --- Code Golf]
Scrivere il programma più corto possibile per:
\begin{enumerate}
    \item Stampare "Hello World"
    \item Ordinare array
    \item Calcolare Fibonacci
\end{enumerate}
\end{esercizio}

\begin{esercizio}[S.2 --- Ottimizzazione]
Ottimizzare i seguenti task per velocità massima:
\begin{enumerate}
    \item Copiare 64KB di memoria
    \item Cercare byte in array 1MB
    \item Calcolare checksum CRC16
\end{enumerate}
\end{esercizio}

\begin{esercizio}[S.3 --- Reverse Engineering]
Analizzare e decompilare un binario .COM fornito.
\end{esercizio}

\begin{esercizio}[S.4 --- Real Mode OS]
Scrivere un OS completo in real mode con:
\begin{itemize}
    \item Multitasking cooperativo
    \item Driver disco e file system
    \item Interfaccia grafica VGA
\end{itemize}
\end{esercizio}

\section{Riepilogo Competenze}

Al completamento di tutti gli esercizi, lo studente avrà acquisito:

\begin{itemize}
    \item Padronanza completa del set di istruzioni 8086
    \item Capacità di debugging e ottimizzazione
    \item Comprensione architettura low-level
    \item Esperienza con hardware e periferiche
    \item Basi per reverse engineering e sicurezza
\end{itemize}


%=============================================================================
% BACKMATTER
%=============================================================================
\backmatter

% Appendice con soluzioni esercizi
\appendix
\chapter{Soluzioni degli Esercizi}

\section*{Nota}
In questa appendice sono riportate le soluzioni complete e commentate degli esercizi dei capitoli 1-4. Ogni soluzione include codice completo funzionante con commenti italiani e una breve spiegazione del funzionamento.

\section{Capitolo 1: Stream e Buffer}

\subsection{Esercizio 1.1: Contatore di righe}

\begin{lstlisting}
import java.io.BufferedReader;
import java.io.FileReader;
import java.io.IOException;

public class ContaRighe {
    public static void main(String[] args) {
        int numeroRighe = 0;

        // Try-with-resources per chiusura automatica
        try (BufferedReader br = new BufferedReader(
                new FileReader("file.txt"))) {

            // Legge ogni riga fino alla fine del file
            while (br.readLine() != null) {
                numeroRighe++;
            }

            System.out.println("Il file contiene " + numeroRighe + " righe");

        } catch (IOException e) {
            System.out.println("Errore lettura file: " + e.getMessage());
        }
    }
}
\end{lstlisting}

\textbf{Spiegazione:} Il programma utilizza BufferedReader per leggere il file riga per riga con readLine(), incrementando un contatore per ogni riga letta. Quando readLine() restituisce null, significa che si è raggiunta la fine del file.

\subsection{Esercizio 1.2: Copia carattere per carattere}

\begin{lstlisting}
import java.io.FileReader;
import java.io.FileWriter;
import java.io.IOException;

public class CopiaCarattere {
    public static void main(String[] args) {
        // Usa try-with-resources per gestire due risorse
        try (FileReader input = new FileReader("originale.txt");
             FileWriter output = new FileWriter("copia.txt")) {

            int carattere;

            // read() restituisce -1 quando raggiunge la fine
            while ((carattere = input.read()) != -1) {
                // Scrive il carattere letto nel file di output
                output.write(carattere);
            }

            System.out.println("Copia completata con successo!");

        } catch (IOException e) {
            System.out.println("Errore durante la copia: " + e.getMessage());
        }
    }
}
\end{lstlisting}

\textbf{Spiegazione:} Il programma legge un carattere alla volta con read() e lo scrive immediatamente nel file di destinazione. Try-with-resources garantisce la chiusura automatica di entrambi gli stream anche in caso di eccezione.

\subsection{Esercizio 1.3: Salva nomi in file}

\begin{lstlisting}
import java.io.BufferedWriter;
import java.io.FileWriter;
import java.io.IOException;
import java.util.Scanner;

public class SalvaNomi {
    public static void main(String[] args) {
        Scanner scanner = new Scanner(System.in);

        try (BufferedWriter bw = new BufferedWriter(
                new FileWriter("nomi.txt"))) {

            System.out.println("Inserisci 5 nomi:");

            // Legge 5 nomi dall'utente
            for (int i = 1; i <= 5; i++) {
                System.out.print("Nome " + i + ": ");
                String nome = scanner.nextLine();

                // Scrive il nome nel file
                bw.write(nome);
                bw.newLine(); // Aggiunge separatore di riga
            }

            System.out.println("Nomi salvati in nomi.txt");

        } catch (IOException e) {
            System.out.println("Errore scrittura file: " + e.getMessage());
        } finally {
            scanner.close();
        }
    }
}
\end{lstlisting}

\textbf{Spiegazione:} Il programma legge 5 nomi da console usando Scanner e li scrive su file usando BufferedWriter, aggiungendo newLine() dopo ogni nome per scriverli su righe separate.

\subsection{Esercizio 1.4: Inverti righe}

\begin{lstlisting}
import java.io.BufferedReader;
import java.io.BufferedWriter;
import java.io.FileReader;
import java.io.FileWriter;
import java.io.IOException;
import java.util.ArrayList;

public class InvertiRighe {
    public static void main(String[] args) {
        ArrayList<String> righe = new ArrayList<>();

        // Legge tutte le righe e le memorizza nell'ArrayList
        try (BufferedReader br = new BufferedReader(
                new FileReader("input.txt"))) {

            String riga;
            while ((riga = br.readLine()) != null) {
                righe.add(riga);
            }

        } catch (IOException e) {
            System.out.println("Errore lettura: " + e.getMessage());
            return;
        }

        // Scrive le righe in ordine inverso
        try (BufferedWriter bw = new BufferedWriter(
                new FileWriter("output.txt"))) {

            // Itera dall'ultima riga alla prima
            for (int i = righe.size() - 1; i >= 0; i--) {
                bw.write(righe.get(i));
                bw.newLine();
            }

            System.out.println("File invertito creato con successo!");

        } catch (IOException e) {
            System.out.println("Errore scrittura: " + e.getMessage());
        }
    }
}
\end{lstlisting}

\textbf{Spiegazione:} Prima legge tutte le righe in un ArrayList, poi le scrive in ordine inverso iterando dall'ultimo indice al primo. Questa tecnica permette di invertire facilmente l'ordine delle righe.

\subsection{Esercizio 1.5: Filtra righe maiuscole}

\begin{lstlisting}
import java.io.BufferedReader;
import java.io.BufferedWriter;
import java.io.FileReader;
import java.io.FileWriter;
import java.io.IOException;

public class FiltraMaiuscole {
    public static void main(String[] args) {
        try (BufferedReader input = new BufferedReader(
                new FileReader("input.txt"));
             BufferedWriter output = new BufferedWriter(
                new FileWriter("output.txt"))) {

            String riga;
            int righeCopiate = 0;

            while ((riga = input.readLine()) != null) {
                // Verifica se la riga non e' vuota e inizia con maiuscola
                if (!riga.isEmpty() &&
                    Character.isUpperCase(riga.charAt(0))) {

                    output.write(riga);
                    output.newLine();
                    righeCopiate++;
                }
            }

            System.out.println("Copiate " + righeCopiate +
                             " righe che iniziano con maiuscola");

        } catch (IOException e) {
            System.out.println("Errore: " + e.getMessage());
        }
    }
}
\end{lstlisting}

\textbf{Spiegazione:} Il programma legge ogni riga e verifica se non è vuota e se il primo carattere è una lettera maiuscola usando Character.isUpperCase(). Solo le righe che soddisfano questa condizione vengono scritte nel file di output.

\subsection{Esercizio 1.6: Merge di due file}

\begin{lstlisting}
import java.io.BufferedReader;
import java.io.BufferedWriter;
import java.io.FileReader;
import java.io.FileWriter;
import java.io.IOException;

public class MergeFile {
    public static void main(String[] args) {
        try (BufferedReader br1 = new BufferedReader(
                new FileReader("file1.txt"));
             BufferedReader br2 = new BufferedReader(
                new FileReader("file2.txt"));
             BufferedWriter output = new BufferedWriter(
                new FileWriter("merge.txt"))) {

            String riga1, riga2;

            // Legge alternando le righe dai due file
            while (true) {
                riga1 = br1.readLine();
                riga2 = br2.readLine();

                // Se entrambi i file sono finiti, esci dal ciclo
                if (riga1 == null && riga2 == null) {
                    break;
                }

                // Scrivi riga dal primo file se disponibile
                if (riga1 != null) {
                    output.write(riga1);
                    output.newLine();
                }

                // Scrivi riga dal secondo file se disponibile
                if (riga2 != null) {
                    output.write(riga2);
                    output.newLine();
                }
            }

            System.out.println("Merge completato!");

        } catch (IOException e) {
            System.out.println("Errore: " + e.getMessage());
        }
    }
}
\end{lstlisting}

\textbf{Spiegazione:} Il programma alterna la lettura di una riga dal primo file e una dal secondo, scrivendole nel file di output. Continua finché entrambi i file non hanno più righe da leggere.

\subsection{Esercizio 1.7: Analisi parole}

\begin{lstlisting}
import java.io.BufferedReader;
import java.io.FileReader;
import java.io.IOException;
import java.util.HashMap;
import java.util.Map;

public class AnalisiParole {
    public static void main(String[] args) {
        int totaleParole = 0;
        int totaleCaratteri = 0;
        // Mappa per contare la frequenza di ogni parola
        HashMap<String, Integer> frequenze = new HashMap<>();

        try (BufferedReader br = new BufferedReader(
                new FileReader("testo.txt"))) {

            String riga;

            while ((riga = br.readLine()) != null) {
                // Divide la riga in parole (split per spazi)
                String[] parole = riga.split("\\s+");

                for (String parola : parole) {
                    // Rimuove punteggiatura e converte in minuscolo
                    parola = parola.replaceAll("[^a-zA-Z]", "")
                                   .toLowerCase();

                    if (!parola.isEmpty()) {
                        totaleParole++;
                        totaleCaratteri += parola.length();

                        // Aggiorna frequenza della parola
                        frequenze.put(parola,
                                    frequenze.getOrDefault(parola, 0) + 1);
                    }
                }
            }

            // Stampa report
            System.out.println("=== REPORT ANALISI ===");
            System.out.println("Totale parole: " + totaleParole);
            System.out.println("Totale caratteri: " + totaleCaratteri);
            System.out.println("\nFrequenza parole:");

            for (Map.Entry<String, Integer> entry : frequenze.entrySet()) {
                System.out.println(entry.getKey() + ": " +
                                 entry.getValue() + " volte");
            }

        } catch (IOException e) {
            System.out.println("Errore: " + e.getMessage());
        }
    }
}
\end{lstlisting}

\textbf{Spiegazione:} Il programma divide ogni riga in parole, le pulisce dalla punteggiatura e conta i caratteri totali. Usa una HashMap per tenere traccia della frequenza di ogni parola, incrementando il contatore quando la parola viene incontrata.

\subsection{Esercizio 1.8: Parser CSV}

\begin{lstlisting}
import java.io.BufferedReader;
import java.io.FileReader;
import java.io.IOException;
import java.util.ArrayList;

public class ParserCSV {
    public static void main(String[] args) {
        ArrayList<String[]> dati = new ArrayList<>();
        int maxLunghezze[] = new int[3]; // Per 3 colonne

        // Legge file CSV
        try (BufferedReader br = new BufferedReader(
                new FileReader("dati.csv"))) {

            String riga;

            while ((riga = br.readLine()) != null) {
                // Split per virgola
                String[] campi = riga.split(",");
                dati.add(campi);

                // Calcola lunghezza massima per ogni colonna
                for (int i = 0; i < campi.length && i < 3; i++) {
                    if (campi[i].length() > maxLunghezze[i]) {
                        maxLunghezze[i] = campi[i].length();
                    }
                }
            }

        } catch (IOException e) {
            System.out.println("Errore lettura: " + e.getMessage());
            return;
        }

        // Stampa tabella allineata
        System.out.println("=== DATI CSV ===");

        for (String[] riga : dati) {
            for (int i = 0; i < riga.length && i < 3; i++) {
                // Stampa campo allineato a sinistra
                System.out.printf("%-" + (maxLunghezze[i] + 2) + "s",
                                riga[i]);
            }
            System.out.println();
        }
    }
}
\end{lstlisting}

\textbf{Spiegazione:} Il parser legge il file CSV, divide ogni riga usando split(",") e calcola la lunghezza massima di ogni colonna. Poi stampa i dati in formato tabellare allineato usando printf con formattazione dinamica.

\section{Capitolo 2: Interfacce e Classi Astratte}

\subsection{Esercizio 2.1: Interfaccia Volante}

\begin{lstlisting}
// Interfaccia Volante
public interface Volante {
    void vola();
}

// Classe Aereo
public class Aereo implements Volante {
    private String modello;
    private int passeggeri;

    public Aereo(String modello, int passeggeri) {
        this.modello = modello;
        this.passeggeri = passeggeri;
    }

    @Override
    public void vola() {
        System.out.println("L'aereo " + modello +
                         " vola trasportando " + passeggeri +
                         " passeggeri");
    }
}

// Classe Uccello
public class Uccello implements Volante {
    private String specie;

    public Uccello(String specie) {
        this.specie = specie;
    }

    @Override
    public void vola() {
        System.out.println("L'uccello " + specie +
                         " vola sbattendo le ali");
    }
}

// Test
public class TestVolante {
    public static void main(String[] args) {
        Volante[] volanti = {
            new Aereo("Boeing 747", 400),
            new Uccello("Aquila"),
            new Aereo("Cessna", 4),
            new Uccello("Gabbiano")
        };

        for (Volante v : volanti) {
            v.vola();
        }
    }
}
\end{lstlisting}

\textbf{Spiegazione:} L'interfaccia Volante definisce il contratto vola(), implementato diversamente da Aereo (con modello e passeggeri) e Uccello (con specie). Il polimorfismo permette di trattare entrambi come oggetti Volante.

\subsection{Esercizio 2.2: Classe astratta Animale}

\begin{lstlisting}
// Classe astratta Animale
public abstract class Animale {
    protected String nome;

    public Animale(String nome) {
        this.nome = nome;
    }

    // Metodo astratto - ogni animale ha un verso diverso
    public abstract void verso();

    // Metodo concreto - tutti gli animali dormono allo stesso modo
    public void dorme() {
        System.out.println(nome + " sta dormendo... Zzz");
    }
}

// Classe Cane
public class Cane extends Animale {
    public Cane(String nome) {
        super(nome);
    }

    @Override
    public void verso() {
        System.out.println(nome + " fa: Bau bau!");
    }
}

// Classe Gatto
public class Gatto extends Animale {
    public Gatto(String nome) {
        super(nome);
    }

    @Override
    public void verso() {
        System.out.println(nome + " fa: Miao miao!");
    }
}

// Test
public class TestAnimali {
    public static void main(String[] args) {
        Animale[] animali = {
            new Cane("Fido"),
            new Gatto("Whiskers"),
            new Cane("Rex")
        };

        for (Animale a : animali) {
            a.verso();
            a.dorme();
            System.out.println();
        }
    }
}
\end{lstlisting}

\textbf{Spiegazione:} La classe astratta Animale fornisce il metodo concreto dorme() condiviso da tutti gli animali, mentre verso() è astratto e deve essere implementato da ogni sottoclasse con il proprio verso specifico.

\subsection{Esercizio 2.3: Interfaccia Pagabile}

\begin{lstlisting}
// Interfaccia Pagabile
public interface Pagabile {
    double calcolaStipendio();
}

// Classe Dipendente
public class Dipendente implements Pagabile {
    private String nome;
    private double stipendioMensile;

    public Dipendente(String nome, double stipendioMensile) {
        this.nome = nome;
        this.stipendioMensile = stipendioMensile;
    }

    @Override
    public double calcolaStipendio() {
        // Stipendio fisso mensile
        return stipendioMensile;
    }

    public String getNome() {
        return nome;
    }
}

// Classe Freelance
public class Freelance implements Pagabile {
    private String nome;
    private double tariffaOraria;
    private int oreLavorate;

    public Freelance(String nome, double tariffaOraria, int oreLavorate) {
        this.nome = nome;
        this.tariffaOraria = tariffaOraria;
        this.oreLavorate = oreLavorate;
    }

    @Override
    public double calcolaStipendio() {
        // Stipendio basato su ore lavorate
        return tariffaOraria * oreLavorate;
    }

    public String getNome() {
        return nome;
    }
}

// Test
public class TestPagabile {
    public static void main(String[] args) {
        Pagabile[] lavoratori = {
            new Dipendente("Mario Rossi", 2000),
            new Freelance("Laura Bianchi", 50, 80),
            new Dipendente("Giuseppe Verdi", 2500),
            new Freelance("Anna Neri", 60, 120)
        };

        double totalePagamenti = 0;

        for (Pagabile p : lavoratori) {
            double stipendio = p.calcolaStipendio();
            totalePagamenti += stipendio;

            String nome = "";
            if (p instanceof Dipendente) {
                nome = ((Dipendente)p).getNome();
            } else if (p instanceof Freelance) {
                nome = ((Freelance)p).getNome();
            }

            System.out.printf("%s: %.2f euro\n", nome, stipendio);
        }

        System.out.printf("\nTotale pagamenti: %.2f euro\n",
                        totalePagamenti);
    }
}
\end{lstlisting}

\textbf{Spiegazione:} L'interfaccia Pagabile definisce il metodo calcolaStipendio(). Dipendente restituisce lo stipendio mensile fisso, mentre Freelance calcola il pagamento moltiplicando tariffa oraria per ore lavorate.

\subsection{Esercizio 2.4: Interfaccia DispositivoElettronico}

\begin{lstlisting}
// Interfaccia DispositivoElettronico
public interface DispositivoElettronico {
    void accendi();
    void spegni();
    double consumoEnergetico(); // in Watt
}

// Classe Televisore
public class Televisore implements DispositivoElettronico {
    private boolean acceso = false;
    private int pollici;

    public Televisore(int pollici) {
        this.pollici = pollici;
    }

    @Override
    public void accendi() {
        acceso = true;
        System.out.println("TV " + pollici + "\" accesa");
    }

    @Override
    public void spegni() {
        acceso = false;
        System.out.println("TV " + pollici + "\" spenta");
    }

    @Override
    public double consumoEnergetico() {
        // Consumo proporzionale ai pollici
        return acceso ? pollici * 2.5 : 0.5; // standby
    }
}

// Classe Computer
public class Computer implements DispositivoElettronico {
    private boolean acceso = false;
    private boolean desktop;

    public Computer(boolean desktop) {
        this.desktop = desktop;
    }

    @Override
    public void accendi() {
        acceso = true;
        System.out.println((desktop ? "Desktop" : "Laptop") + " acceso");
    }

    @Override
    public void spegni() {
        acceso = false;
        System.out.println((desktop ? "Desktop" : "Laptop") + " spento");
    }

    @Override
    public double consumoEnergetico() {
        if (!acceso) return 0;
        return desktop ? 150 : 65;
    }
}

// Classe Lampada
public class Lampada implements DispositivoElettronico {
    private boolean accesa = false;
    private int watt;

    public Lampada(int watt) {
        this.watt = watt;
    }

    @Override
    public void accendi() {
        accesa = true;
        System.out.println("Lampada " + watt + "W accesa");
    }

    @Override
    public void spegni() {
        accesa = false;
        System.out.println("Lampada " + watt + "W spenta");
    }

    @Override
    public double consumoEnergetico() {
        return accesa ? watt : 0;
    }
}

// Test
public class TestDispositivi {
    public static double calcolaConsumoTotale(
            DispositivoElettronico[] dispositivi) {
        double totale = 0;
        for (DispositivoElettronico d : dispositivi) {
            totale += d.consumoEnergetico();
        }
        return totale;
    }

    public static void main(String[] args) {
        DispositivoElettronico[] dispositivi = {
            new Televisore(55),
            new Computer(true),
            new Lampada(60),
            new Computer(false)
        };

        // Accende tutti i dispositivi
        for (DispositivoElettronico d : dispositivi) {
            d.accendi();
        }

        System.out.println();
        double consumo = calcolaConsumoTotale(dispositivi);
        System.out.printf("Consumo totale: %.2f Watt\n", consumo);
    }
}
\end{lstlisting}

\textbf{Spiegazione:} Ogni dispositivo implementa i metodi accendi(), spegni() e consumoEnergetico() con logiche diverse. Il metodo calcolaConsumoTotale() somma i consumi di tutti i dispositivi in un array, dimostrando il polimorfismo.

\subsection{Esercizio 2.5: Classe astratta FiguraPiana}

\begin{lstlisting}
// Classe astratta FiguraPiana
public abstract class FiguraPiana {
    protected String colore;

    public FiguraPiana(String colore) {
        this.colore = colore;
    }

    // Metodi astratti
    public abstract double area();
    public abstract double perimetro();

    // Metodo concreto che usa i metodi astratti
    public String confrontaArea(FiguraPiana altra) {
        double miaArea = this.area();
        double altraArea = altra.area();

        if (miaArea > altraArea) {
            return "Questa figura ha area maggiore";
        } else if (miaArea < altraArea) {
            return "L'altra figura ha area maggiore";
        } else {
            return "Le due figure hanno la stessa area";
        }
    }
}

// Classe Quadrato
public class Quadrato extends FiguraPiana {
    private double lato;

    public Quadrato(String colore, double lato) {
        super(colore);
        this.lato = lato;
    }

    @Override
    public double area() {
        return lato * lato;
    }

    @Override
    public double perimetro() {
        return 4 * lato;
    }
}

// Classe Cerchio
public class Cerchio extends FiguraPiana {
    private double raggio;

    public Cerchio(String colore, double raggio) {
        super(colore);
        this.raggio = raggio;
    }

    @Override
    public double area() {
        return Math.PI * raggio * raggio;
    }

    @Override
    public double perimetro() {
        return 2 * Math.PI * raggio;
    }
}

// Classe Triangolo
public class Triangolo extends FiguraPiana {
    private double base;
    private double altezza;
    private double lato1, lato2, lato3;

    public Triangolo(String colore, double base, double altezza,
                    double lato1, double lato2, double lato3) {
        super(colore);
        this.base = base;
        this.altezza = altezza;
        this.lato1 = lato1;
        this.lato2 = lato2;
        this.lato3 = lato3;
    }

    @Override
    public double area() {
        return (base * altezza) / 2;
    }

    @Override
    public double perimetro() {
        return lato1 + lato2 + lato3;
    }
}

// Test
public class TestFigure {
    public static void main(String[] args) {
        FiguraPiana[] figure = {
            new Quadrato("Rosso", 5),
            new Cerchio("Blu", 3),
            new Triangolo("Verde", 4, 3, 3, 4, 5)
        };

        System.out.println("=== ANALISI FIGURE ===");
        for (FiguraPiana f : figure) {
            System.out.printf("Area: %.2f - Perimetro: %.2f\n",
                            f.area(), f.perimetro());
        }

        System.out.println("\n=== CONFRONTI ===");
        System.out.println(figure[0].confrontaArea(figure[1]));
        System.out.println(figure[1].confrontaArea(figure[2]));
    }
}
\end{lstlisting}

\textbf{Spiegazione:} La classe astratta fornisce il metodo concreto confrontaArea() che usa i metodi astratti area(). Ogni figura implementa area() e perimetro() con le proprie formule geometriche.

\subsection{Esercizio 2.6: Libro Comparable}

\begin{lstlisting}
import java.util.Arrays;

// Classe Libro che implementa Comparable
public class Libro implements Comparable<Libro> {
    private String titolo;
    private String autore;
    private int anno;

    public Libro(String titolo, String autore, int anno) {
        this.titolo = titolo;
        this.autore = autore;
        this.anno = anno;
    }

    @Override
    public int compareTo(Libro altro) {
        // Confronta per titolo (ordine alfabetico)
        return this.titolo.compareTo(altro.titolo);
    }

    @Override
    public String toString() {
        return String.format("\"%s\" di %s (%d)", titolo, autore, anno);
    }

    // Getter
    public String getTitolo() { return titolo; }
    public String getAutore() { return autore; }
    public int getAnno() { return anno; }
}

// Test
public class TestLibri {
    public static void main(String[] args) {
        Libro[] biblioteca = {
            new Libro("Zeno", "Italo Svevo", 1923),
            new Libro("Il nome della rosa", "Umberto Eco", 1980),
            new Libro("Annabelle Lee", "Edgar Allan Poe", 1849),
            new Libro("Divina Commedia", "Dante Alighieri", 1321),
            new Libro("Promessi Sposi", "Alessandro Manzoni", 1827)
        };

        System.out.println("=== PRIMA DELL'ORDINAMENTO ===");
        for (Libro libro : biblioteca) {
            System.out.println(libro);
        }

        // Ordina usando compareTo (per titolo)
        Arrays.sort(biblioteca);

        System.out.println("\n=== DOPO ORDINAMENTO (per titolo) ===");
        for (Libro libro : biblioteca) {
            System.out.println(libro);
        }
    }
}
\end{lstlisting}

\textbf{Spiegazione:} La classe Libro implementa Comparable<Libro> definendo il metodo compareTo() che confronta i titoli in ordine alfabetico. Arrays.sort() usa automaticamente questo metodo per ordinare l'array.

\subsection{Esercizio 2.7: Sistema Account Bancari}

\begin{lstlisting}
// Classe astratta Account
public abstract class Account {
    protected String numeroAccount;
    protected String intestatario;
    protected double saldo;

    public Account(String numero, String intestatario, double saldoIniziale) {
        this.numeroAccount = numero;
        this.intestatario = intestatario;
        this.saldo = saldoIniziale;
    }

    // Metodi concreti comuni
    public void deposita(double importo) {
        if (importo > 0) {
            saldo += importo;
            System.out.printf("Depositati %.2f euro\n", importo);
        }
    }

    public boolean preleva(double importo) {
        if (importo > 0 && importo <= saldo) {
            saldo -= importo;
            System.out.printf("Prelevati %.2f euro\n", importo);
            return true;
        }
        System.out.println("Prelievo non riuscito");
        return false;
    }

    public double getSaldo() {
        return saldo;
    }

    // Metodi astratti
    public abstract double calcolaInteressi();
    public abstract void applicaCommissioni();
}

// Account Corrente
public class AccountCorrente extends Account {
    private static final double COMMISSIONE_MENSILE = 5.0;
    private static final double TASSO_INTERESSE = 0.001; // 0.1%

    public AccountCorrente(String numero, String intestatario,
                          double saldoIniziale) {
        super(numero, intestatario, saldoIniziale);
    }

    @Override
    public double calcolaInteressi() {
        // Interessi molto bassi
        return saldo * TASSO_INTERESSE;
    }

    @Override
    public void applicaCommissioni() {
        saldo -= COMMISSIONE_MENSILE;
        System.out.printf("Commissione applicata: %.2f euro\n",
                        COMMISSIONE_MENSILE);
    }
}

// Account Risparmio
public class AccountRisparmio extends Account {
    private static final double TASSO_INTERESSE = 0.02; // 2%

    public AccountRisparmio(String numero, String intestatario,
                           double saldoIniziale) {
        super(numero, intestatario, saldoIniziale);
    }

    @Override
    public double calcolaInteressi() {
        // Interessi piu' alti
        return saldo * TASSO_INTERESSE;
    }

    @Override
    public void applicaCommissioni() {
        // Nessuna commissione per risparmio
        System.out.println("Nessuna commissione per account risparmio");
    }

    @Override
    public boolean preleva(double importo) {
        // Limite prelievi per risparmio
        if (importo > 1000) {
            System.out.println("Limite prelievo: max 1000 euro");
            return false;
        }
        return super.preleva(importo);
    }
}

// Account Deposito
public class AccountDeposito extends Account {
    private static final double TASSO_INTERESSE = 0.035; // 3.5%
    private int mesiVincolo;

    public AccountDeposito(String numero, String intestatario,
                          double saldoIniziale, int mesiVincolo) {
        super(numero, intestatario, saldoIniziale);
        this.mesiVincolo = mesiVincolo;
    }

    @Override
    public double calcolaInteressi() {
        // Interessi piu' alti se vincolato
        return saldo * TASSO_INTERESSE * (mesiVincolo / 12.0);
    }

    @Override
    public void applicaCommissioni() {
        // Nessuna commissione
        System.out.println("Nessuna commissione per deposito");
    }

    @Override
    public boolean preleva(double importo) {
        System.out.println("Impossibile prelevare da deposito vincolato");
        return false;
    }
}

// Test
public class TestBanca {
    public static void main(String[] args) {
        Account[] conti = {
            new AccountCorrente("CC001", "Mario Rossi", 1000),
            new AccountRisparmio("RS001", "Laura Bianchi", 5000),
            new AccountDeposito("DP001", "Giuseppe Verdi", 10000, 12)
        };

        System.out.println("=== OPERAZIONI MENSILI ===\n");

        for (Account a : conti) {
            System.out.println("Account: " + a.intestatario);
            System.out.printf("Saldo iniziale: %.2f euro\n", a.getSaldo());

            // Calcola e accredita interessi
            double interessi = a.calcolaInteressi();
            a.deposita(interessi);
            System.out.printf("Interessi maturati: %.2f euro\n", interessi);

            // Applica commissioni
            a.applicaCommissioni();

            System.out.printf("Saldo finale: %.2f euro\n\n", a.getSaldo());
        }
    }
}
\end{lstlisting}

\textbf{Spiegazione:} La classe astratta Account fornisce metodi comuni (deposita, preleva) e metodi astratti (calcolaInteressi, applicaCommissioni). Ogni tipo di account implementa questi metodi con regole specifiche: corrente ha commissioni, risparmio ha limiti di prelievo, deposito non permette prelievi.

\subsection{Esercizio 2.8: Interfaccia Ordinabile}

\begin{lstlisting}
// Interfaccia Ordinabile
public interface Ordinabile {
    int confronta(Ordinabile altro);
    String getChiave();
}

// Classe Prodotto
public class Prodotto implements Ordinabile {
    private String codice;
    private String nome;
    private double prezzo;

    public Prodotto(String codice, String nome, double prezzo) {
        this.codice = codice;
        this.nome = nome;
        this.prezzo = prezzo;
    }

    @Override
    public int confronta(Ordinabile altro) {
        // Ordina per codice
        return this.codice.compareTo(((Prodotto)altro).codice);
    }

    @Override
    public String getChiave() {
        return codice;
    }

    @Override
    public String toString() {
        return String.format("%s - %s (%.2f euro)", codice, nome, prezzo);
    }
}

// Classe Cliente
public class Cliente implements Ordinabile {
    private String id;
    private String nome;
    private String cognome;

    public Cliente(String id, String nome, String cognome) {
        this.id = id;
        this.nome = nome;
        this.cognome = cognome;
    }

    @Override
    public int confronta(Ordinabile altro) {
        // Ordina per cognome, poi nome
        Cliente altroCliente = (Cliente)altro;
        int confrontoCognome = this.cognome.compareTo(altroCliente.cognome);
        if (confrontoCognome != 0) {
            return confrontoCognome;
        }
        return this.nome.compareTo(altroCliente.nome);
    }

    @Override
    public String getChiave() {
        return id;
    }

    @Override
    public String toString() {
        return String.format("%s: %s %s", id, nome, cognome);
    }
}

// Classe Ordine
public class Ordine implements Ordinabile {
    private int numeroOrdine;
    private String dataOrdine;
    private double totale;

    public Ordine(int numero, String data, double totale) {
        this.numeroOrdine = numero;
        this.dataOrdine = data;
        this.totale = totale;
    }

    @Override
    public int confronta(Ordinabile altro) {
        // Ordina per numero ordine
        Ordine altroOrdine = (Ordine)altro;
        return Integer.compare(this.numeroOrdine, altroOrdine.numeroOrdine);
    }

    @Override
    public String getChiave() {
        return String.valueOf(numeroOrdine);
    }

    @Override
    public String toString() {
        return String.format("Ordine %d del %s - Totale: %.2f euro",
                           numeroOrdine, dataOrdine, totale);
    }
}

// Metodo generico di ordinamento
public class Ordinatore {
    // Bubble sort generico per Ordinabile
    public static void ordina(Ordinabile[] array) {
        int n = array.length;
        for (int i = 0; i < n - 1; i++) {
            for (int j = 0; j < n - i - 1; j++) {
                if (array[j].confronta(array[j + 1]) > 0) {
                    // Scambia elementi
                    Ordinabile temp = array[j];
                    array[j] = array[j + 1];
                    array[j + 1] = temp;
                }
            }
        }
    }
}

// Test
public class TestOrdinabile {
    public static void main(String[] args) {
        // Test con Prodotti
        Ordinabile[] prodotti = {
            new Prodotto("P003", "Mouse", 25.99),
            new Prodotto("P001", "Tastiera", 45.50),
            new Prodotto("P002", "Monitor", 199.99)
        };

        System.out.println("=== PRODOTTI NON ORDINATI ===");
        stampa(prodotti);

        Ordinatore.ordina(prodotti);

        System.out.println("\n=== PRODOTTI ORDINATI ===");
        stampa(prodotti);

        // Test con Clienti
        Ordinabile[] clienti = {
            new Cliente("C003", "Mario", "Rossi"),
            new Cliente("C001", "Laura", "Bianchi"),
            new Cliente("C002", "Anna", "Bianchi")
        };

        System.out.println("\n=== CLIENTI NON ORDINATI ===");
        stampa(clienti);

        Ordinatore.ordina(clienti);

        System.out.println("\n=== CLIENTI ORDINATI ===");
        stampa(clienti);
    }

    private static void stampa(Ordinabile[] array) {
        for (Ordinabile o : array) {
            System.out.println(o);
        }
    }
}
\end{lstlisting}

\textbf{Spiegazione:} L'interfaccia Ordinabile permette di scrivere un algoritmo di ordinamento generico che funziona con qualsiasi tipo che implementa confronta(). Ogni classe definisce la propria logica di confronto: Prodotto per codice, Cliente per cognome/nome, Ordine per numero.

\section{Capitolo 3: Eccezioni}

\subsection{Esercizio 3.1: Input numerico con gestione eccezioni}

\begin{lstlisting}
import java.util.Scanner;
import java.util.InputMismatchException;

public class InputNumerico {
    public static void main(String[] args) {
        Scanner scanner = new Scanner(System.in);
        int numero = 0;
        boolean inputValido = false;

        // Continua a chiedere finche' l'input non e' valido
        while (!inputValido) {
            try {
                System.out.print("Inserisci un numero intero: ");
                numero = scanner.nextInt();
                inputValido = true; // Input valido, esci dal ciclo

            } catch (InputMismatchException e) {
                System.out.println("ERRORE: Devi inserire un numero intero!");
                scanner.nextLine(); // Pulisce l'input errato
            }
        }

        System.out.println("Hai inserito: " + numero);
        scanner.close();
    }
}
\end{lstlisting}

\textbf{Spiegazione:} Il programma usa un ciclo while per continuare a chiedere l'input finché non viene inserito un numero valido. InputMismatchException viene catturata quando l'utente inserisce testo invece di un numero.

\subsection{Esercizio 3.2: Divisione sicura}

\begin{lstlisting}
public class DivisioneSicura {

    public static double dividi(double dividendo, double divisore) {
        try {
            // Lancia eccezione se divisore e' zero
            if (divisore == 0) {
                throw new ArithmeticException("Divisione per zero");
            }
            return dividendo / divisore;

        } catch (ArithmeticException e) {
            System.out.println("Attenzione: " + e.getMessage() +
                             " - Restituisco 0");
            return 0;
        }
    }

    public static void main(String[] args) {
        // Test vari casi
        System.out.println("10 / 2 = " + dividi(10, 2));
        System.out.println("15 / 3 = " + dividi(15, 3));
        System.out.println("7 / 0 = " + dividi(7, 0));  // Gestisce eccezione
        System.out.println("20 / 4 = " + dividi(20, 4));
    }
}
\end{lstlisting}

\textbf{Spiegazione:} Il metodo dividi() controlla se il divisore è zero e in tal caso lancia un'eccezione ArithmeticException. Il catch intercetta l'eccezione e restituisce 0 come valore di default.

\subsection{Esercizio 3.3: Accesso array sicuro}

\begin{lstlisting}
import java.util.Scanner;

public class AccessoArraySicuro {
    public static void main(String[] args) {
        String[] frutti = {"Mela", "Banana", "Arancia", "Pera", "Kiwi"};
        Scanner scanner = new Scanner(System.in);

        System.out.println("Array disponibile con " + frutti.length +
                         " elementi (indici 0-" + (frutti.length - 1) + ")");

        System.out.print("Inserisci indice da visualizzare: ");
        int indice = scanner.nextInt();

        try {
            // Tenta di accedere all'elemento
            String frutto = frutti[indice];
            System.out.println("Elemento all'indice " + indice +
                             ": " + frutto);

        } catch (ArrayIndexOutOfBoundsException e) {
            System.out.println("ERRORE: Indice " + indice +
                             " non valido!");
            System.out.println("Gli indici validi sono da 0 a " +
                             (frutti.length - 1));
        } finally {
            scanner.close();
        }
    }
}
\end{lstlisting}

\textbf{Spiegazione:} Il programma cattura ArrayIndexOutOfBoundsException quando l'utente inserisce un indice fuori dai limiti dell'array, mostrando un messaggio di errore chiaro con gli indici validi.

\subsection{Esercizio 3.4: Eccezione personalizzata NumeroNegativo}

\begin{lstlisting}
// Eccezione personalizzata checked
public class NumeroNegativoException extends Exception {
    private double numeroErrato;

    public NumeroNegativoException(double numero) {
        super("Impossibile calcolare radice quadrata di numero negativo: "
              + numero);
        this.numeroErrato = numero;
    }

    public double getNumeroErrato() {
        return numeroErrato;
    }
}

// Classe con metodo che usa l'eccezione
public class CalcolatoreRadici {

    public static double calcolaRadiceQuadrata(double n)
            throws NumeroNegativoException {

        if (n < 0) {
            throw new NumeroNegativoException(n);
        }

        return Math.sqrt(n);
    }

    public static void main(String[] args) {
        double[] numeri = {16, 25, -9, 49, -4, 64};

        System.out.println("=== CALCOLO RADICI QUADRATE ===");

        for (double num : numeri) {
            try {
                double radice = calcolaRadiceQuadrata(num);
                System.out.printf("Radice di %.1f = %.2f\n", num, radice);

            } catch (NumeroNegativoException e) {
                System.out.println("ERRORE: " + e.getMessage());
                System.out.println("Numero problematico: " +
                                 e.getNumeroErrato());
            }
        }
    }
}
\end{lstlisting}

\textbf{Spiegazione:} Viene creata un'eccezione checked personalizzata che memorizza il numero negativo che ha causato l'errore. Il metodo calcolaRadiceQuadrata() dichiara throws e lancia l'eccezione se il numero è negativo.

\subsection{Esercizio 3.5: Classe Calcolatrice con eccezioni}

\begin{lstlisting}
// Eccezioni personalizzate
class DivisoreZeroException extends Exception {
    public DivisoreZeroException() {
        super("Errore: divisione per zero non permessa");
    }
}

class OperazioneNonValidaException extends Exception {
    public OperazioneNonValidaException(String messaggio) {
        super(messaggio);
    }
}

// Classe Calcolatrice
public class Calcolatrice {

    public double somma(double a, double b) {
        return a + b;
    }

    public double sottrai(double a, double b) {
        return a - b;
    }

    public double moltiplica(double a, double b)
            throws OperazioneNonValidaException {
        // Controlla overflow per numeri molto grandi
        if (Math.abs(a) > 1e100 || Math.abs(b) > 1e100) {
            throw new OperazioneNonValidaException(
                "Numeri troppo grandi per la moltiplicazione");
        }
        return a * b;
    }

    public double dividi(double dividendo, double divisore)
            throws DivisoreZeroException {
        if (divisore == 0) {
            throw new DivisoreZeroException();
        }
        return dividendo / divisore;
    }

    public static void main(String[] args) {
        Calcolatrice calc = new Calcolatrice();

        System.out.println("=== TEST CALCOLATRICE ===\n");

        // Test somma
        System.out.println("5 + 3 = " + calc.somma(5, 3));

        // Test sottrazione
        System.out.println("10 - 4 = " + calc.sottrai(10, 4));

        // Test moltiplicazione
        try {
            System.out.println("6 * 7 = " + calc.moltiplica(6, 7));
            System.out.println("1e101 * 2 = " + calc.moltiplica(1e101, 2));
        } catch (OperazioneNonValidaException e) {
            System.out.println("ERRORE: " + e.getMessage());
        }

        // Test divisione
        try {
            System.out.println("20 / 5 = " + calc.dividi(20, 5));
            System.out.println("15 / 0 = " + calc.dividi(15, 0));
        } catch (DivisoreZeroException e) {
            System.out.println("ERRORE: " + e.getMessage());
        }
    }
}
\end{lstlisting}

\textbf{Spiegazione:} La classe Calcolatrice implementa le 4 operazioni base, lanciando eccezioni personalizzate per situazioni non valide: DivisoreZeroException per divisioni per zero e OperazioneNonValidaException per numeri troppo grandi.

\subsection{Esercizio 3.6: Conta righe file con gestione eccezioni}

\begin{lstlisting}
import java.io.BufferedReader;
import java.io.FileReader;
import java.io.FileNotFoundException;
import java.io.IOException;

public class ContaRigheConEccezioni {

    public static int contaRighe(String percorsoFile)
            throws FileNotFoundException, IOException {

        int righe = 0;

        try (BufferedReader br = new BufferedReader(
                new FileReader(percorsoFile))) {

            while (br.readLine() != null) {
                righe++;
            }
        }
        // Try-with-resources chiude automaticamente br
        // Le eccezioni vengono propagate al chiamante

        return righe;
    }

    public static void main(String[] args) {
        String[] files = {"documento.txt", "file_inesistente.txt",
                         "dati.csv"};

        for (String file : files) {
            try {
                int numeroRighe = contaRighe(file);
                System.out.println(file + ": " + numeroRighe + " righe");

            } catch (FileNotFoundException e) {
                System.out.println("ERRORE: File '" + file +
                                 "' non trovato");
                System.out.println("Verifica che il file esista " +
                                 "nella directory corrente");

            } catch (IOException e) {
                System.out.println("ERRORE lettura file '" + file + "': " +
                                 e.getMessage());
                System.out.println("Il file potrebbe essere corrotto " +
                                 "o non accessibile");
            }
        }
    }
}
\end{lstlisting}

\textbf{Spiegazione:} Il programma usa catch separati per FileNotFoundException (file non esiste) e IOException (errore generico di lettura), fornendo messaggi di errore specifici per ogni situazione.

\subsection{Esercizio 3.7: Gerarchia eccezioni prenotazioni}

\begin{lstlisting}
// Eccezione base
class PrenotazioneException extends Exception {
    public PrenotazioneException(String messaggio) {
        super(messaggio);
    }
}

// Eccezioni specifiche
class PostiEsauritiException extends PrenotazioneException {
    private int postiRichiesti;
    private int postiDisponibili;

    public PostiEsauritiException(int richiesti, int disponibili) {
        super("Posti esauriti: richiesti " + richiesti +
              ", disponibili " + disponibili);
        this.postiRichiesti = richiesti;
        this.postiDisponibili = disponibili;
    }

    public int getPostiDisponibili() {
        return postiDisponibili;
    }
}

class DataNonValidaException extends PrenotazioneException {
    public DataNonValidaException(String data) {
        super("Data non valida: " + data);
    }
}

class PagamentoFallitoException extends PrenotazioneException {
    private String motivoFallimento;

    public PagamentoFallitoException(String motivo) {
        super("Pagamento fallito: " + motivo);
        this.motivoFallimento = motivo;
    }

    public String getMotivoFallimento() {
        return motivoFallimento;
    }
}

// Sistema di prenotazioni
public class SistemaPrenotazioni {
    private int postiTotali = 100;
    private int postiPrenotati = 0;

    public void prenota(String data, int numPersone, String metodoPagamento)
            throws PrenotazioneException {

        // Valida data (formato semplice: gg/mm/aaaa)
        if (!data.matches("\\d{2}/\\d{2}/\\d{4}")) {
            throw new DataNonValidaException(data);
        }

        // Controlla disponibilita' posti
        int postiLiberi = postiTotali - postiPrenotati;
        if (numPersone > postiLiberi) {
            throw new PostiEsauritiException(numPersone, postiLiberi);
        }

        // Simula pagamento
        if (!elaboraPagamento(metodoPagamento)) {
            throw new PagamentoFallitoException(
                "Metodo di pagamento non valido: " + metodoPagamento);
        }

        // Prenotazione confermata
        postiPrenotati += numPersone;
        System.out.println("Prenotazione confermata per " + numPersone +
                         " persone il " + data);
        System.out.println("Posti rimanenti: " +
                         (postiTotali - postiPrenotati));
    }

    private boolean elaboraPagamento(String metodo) {
        // Simula validazione pagamento
        return metodo.equals("carta") || metodo.equals("contanti");
    }

    public static void main(String[] args) {
        SistemaPrenotazioni sistema = new SistemaPrenotazioni();

        // Test varie prenotazioni
        String[][] prenotazioni = {
            {"15/06/2025", "4", "carta"},
            {"20/06/2025", "2", "contanti"},
            {"data_errata", "3", "carta"},
            {"25/06/2025", "150", "carta"},
            {"30/06/2025", "5", "paypal"}
        };

        for (String[] p : prenotazioni) {
            try {
                System.out.println("\n--- Tentativo prenotazione ---");
                sistema.prenota(p[0], Integer.parseInt(p[1]), p[2]);

            } catch (DataNonValidaException e) {
                System.out.println("ERRORE: " + e.getMessage());
                System.out.println("Formato richiesto: gg/mm/aaaa");

            } catch (PostiEsauritiException e) {
                System.out.println("ERRORE: " + e.getMessage());
                System.out.println("Suggerimento: prenotare solo " +
                                 e.getPostiDisponibili() + " posti");

            } catch (PagamentoFallitoException e) {
                System.out.println("ERRORE: " + e.getMessage());
                System.out.println("Metodi accettati: carta, contanti");

            } catch (PrenotazioneException e) {
                // Catch generico per altre eccezioni
                System.out.println("ERRORE generico: " + e.getMessage());
            }
        }
    }
}
\end{lstlisting}

\textbf{Spiegazione:} Viene creata una gerarchia di eccezioni con PrenotazioneException come base e tre eccezioni specifiche che ereditano da essa. Ogni eccezione può memorizzare informazioni aggiuntive (posti disponibili, motivo fallimento) e il catch può gestirle in modo specifico.

\subsection{Esercizio 3.8: Parser JSON semplificato}

\begin{lstlisting}
// Eccezioni personalizzate per parsing JSON
class JsonSyntaxException extends Exception {
    public JsonSyntaxException(String messaggio) {
        super("Errore sintassi JSON: " + messaggio);
    }
}

class JsonKeyException extends Exception {
    public JsonKeyException(String chiave) {
        super("Chiave JSON non valida: '" + chiave + "'");
    }
}

class JsonValueException extends Exception {
    public JsonValueException(String valore) {
        super("Valore JSON non valido: '" + valore + "'");
    }
}

// Parser JSON semplificato
public class SimpleJsonParser {

    public static void parseJson(String json)
            throws JsonSyntaxException, JsonKeyException, JsonValueException {

        // Rimuove spazi
        json = json.trim();

        // Verifica graffe di apertura e chiusura
        if (!json.startsWith("{") || !json.endsWith("}")) {
            throw new JsonSyntaxException(
                "Il JSON deve iniziare con { e finire con }");
        }

        // Rimuove graffe
        String contenuto = json.substring(1, json.length() - 1).trim();

        // Se vuoto, e' valido
        if (contenuto.isEmpty()) {
            System.out.println("JSON vuoto valido: {}");
            return;
        }

        // Split per coppie chiave:valore
        String[] coppie = contenuto.split(",");

        for (String coppia : coppie) {
            coppia = coppia.trim();

            // Verifica presenza di :
            if (!coppia.contains(":")) {
                throw new JsonSyntaxException(
                    "Manca ':' nella coppia chiave:valore");
            }

            // Split chiave:valore
            String[] parti = coppia.split(":", 2);
            String chiave = parti[0].trim();
            String valore = parti[1].trim();

            // Valida chiave (deve essere tra virgolette)
            if (!chiave.startsWith("\"") || !chiave.endsWith("\"")) {
                throw new JsonKeyException(chiave);
            }

            // Rimuove virgolette dalla chiave
            chiave = chiave.substring(1, chiave.length() - 1);

            // Valida valore (stringa tra virgolette o numero)
            if (!isValoreValido(valore)) {
                throw new JsonValueException(valore);
            }

            System.out.println("Coppia valida -> " + chiave +
                             " = " + valore);
        }

        System.out.println("JSON valido!");
    }

    private static boolean isValoreValido(String valore) {
        // Valore valido se: stringa tra virgolette, numero,
        // true, false, null
        if (valore.startsWith("\"") && valore.endsWith("\"")) {
            return true; // Stringa
        }
        if (valore.equals("true") || valore.equals("false") ||
            valore.equals("null")) {
            return true; // Boolean o null
        }
        try {
            Double.parseDouble(valore);
            return true; // Numero
        } catch (NumberFormatException e) {
            return false;
        }
    }

    public static void main(String[] args) {
        String[] testJson = {
            "{\"nome\":\"Mario\", \"eta\":30}",
            "{\"valido\":true, \"count\":42}",
            "{\"errore\":senza_virgolette}",
            "{chiave_senza_virgolette:\"valore\"}",
            "{\"completo\":\"ok\", senza_due_punti}",
            "non inizia con graffa"
        };

        for (String json : testJson) {
            try {
                System.out.println("\n=== Test JSON ===");
                System.out.println("Input: " + json);
                parseJson(json);

            } catch (JsonSyntaxException e) {
                System.out.println("ERRORE SINTASSI: " + e.getMessage());

            } catch (JsonKeyException e) {
                System.out.println("ERRORE CHIAVE: " + e.getMessage());
                System.out.println("Le chiavi devono essere tra " +
                                 "virgolette doppie");

            } catch (JsonValueException e) {
                System.out.println("ERRORE VALORE: " + e.getMessage());
                System.out.println("I valori stringa devono essere " +
                                 "tra virgolette");
            }
        }
    }
}
\end{lstlisting}

\textbf{Spiegazione:} Il parser verifica la sintassi JSON base con tre tipi di eccezioni: JsonSyntaxException per errori strutturali (graffe, due punti), JsonKeyException per chiavi non tra virgolette, JsonValueException per valori non validi. Ogni eccezione fornisce informazioni specifiche sull'errore.

\section{Capitolo 4: ArrayList}

\subsection{Esercizio 4.1: Lista della spesa}

\begin{lstlisting}
import java.util.ArrayList;
import java.util.Scanner;

public class ListaSpesa {
    private static ArrayList<String> spesa = new ArrayList<>();
    private static Scanner scanner = new Scanner(System.in);

    public static void main(String[] args) {
        boolean continua = true;

        while (continua) {
            mostraMenu();
            int scelta = scanner.nextInt();
            scanner.nextLine(); // Consuma newline

            switch (scelta) {
                case 1:
                    aggiungiProdotto();
                    break;
                case 2:
                    visualizzaLista();
                    break;
                case 3:
                    rimuoviProdotto();
                    break;
                case 4:
                    verificaProdotto();
                    break;
                case 5:
                    continua = false;
                    System.out.println("Buona spesa!");
                    break;
                default:
                    System.out.println("Scelta non valida");
            }
        }
        scanner.close();
    }

    private static void mostraMenu() {
        System.out.println("\n=== LISTA DELLA SPESA ===");
        System.out.println("1. Aggiungi prodotto");
        System.out.println("2. Visualizza lista");
        System.out.println("3. Rimuovi prodotto");
        System.out.println("4. Verifica prodotto");
        System.out.println("5. Esci");
        System.out.print("Scelta: ");
    }

    private static void aggiungiProdotto() {
        System.out.print("Nome prodotto: ");
        String prodotto = scanner.nextLine();

        // Verifica se gia' presente
        if (spesa.contains(prodotto)) {
            System.out.println("Prodotto gia' presente nella lista!");
        } else {
            spesa.add(prodotto);
            System.out.println("Prodotto aggiunto!");
        }
    }

    private static void visualizzaLista() {
        if (spesa.isEmpty()) {
            System.out.println("Lista vuota");
            return;
        }

        System.out.println("\n--- LISTA DELLA SPESA ---");
        for (int i = 0; i < spesa.size(); i++) {
            System.out.println((i + 1) + ". " + spesa.get(i));
        }
        System.out.println("Totale prodotti: " + spesa.size());
    }

    private static void rimuoviProdotto() {
        if (spesa.isEmpty()) {
            System.out.println("Lista vuota");
            return;
        }

        visualizzaLista();
        System.out.print("Numero prodotto da rimuovere: ");
        int indice = scanner.nextInt() - 1;
        scanner.nextLine();

        if (indice >= 0 && indice < spesa.size()) {
            String rimosso = spesa.remove(indice);
            System.out.println("Rimosso: " + rimosso);
        } else {
            System.out.println("Numero non valido");
        }
    }

    private static void verificaProdotto() {
        System.out.print("Prodotto da cercare: ");
        String prodotto = scanner.nextLine();

        if (spesa.contains(prodotto)) {
            int posizione = spesa.indexOf(prodotto) + 1;
            System.out.println("'" + prodotto + "' e' presente " +
                             "(posizione " + posizione + ")");
        } else {
            System.out.println("'" + prodotto + "' non e' nella lista");
        }
    }
}
\end{lstlisting}

\textbf{Spiegazione:} Il programma gestisce una lista della spesa con menu interattivo, permettendo di aggiungere prodotti (controllando duplicati con contains), visualizzarli, rimuoverli per indice e verificare la presenza di un prodotto.

\subsection{Esercizio 4.2: Temperature settimanali}

\begin{lstlisting}
import java.util.ArrayList;
import java.util.Scanner;

public class TemperatureSettimanali {
    public static void main(String[] args) {
        ArrayList<Double> temperature = new ArrayList<>();
        Scanner scanner = new Scanner(System.in);

        String[] giorni = {"Lunedi", "Martedi", "Mercoledi", "Giovedi",
                          "Venerdi", "Sabato", "Domenica"};

        // Inserimento temperature
        System.out.println("=== TEMPERATURE MASSIME SETTIMANALI ===");
        for (String giorno : giorni) {
            System.out.print(giorno + ": ");
            double temp = scanner.nextDouble();
            temperature.add(temp);
        }

        // Calcolo media
        double somma = 0;
        for (double temp : temperature) {
            somma += temp;
        }
        double media = somma / temperature.size();

        // Trova massima e minima
        double massima = temperature.get(0);
        double minima = temperature.get(0);
        int giornoMax = 0;
        int giornoMin = 0;

        for (int i = 1; i < temperature.size(); i++) {
            double temp = temperature.get(i);
            if (temp > massima) {
                massima = temp;
                giornoMax = i;
            }
            if (temp < minima) {
                minima = temp;
                giornoMin = i;
            }
        }

        // Visualizza risultati
        System.out.println("\n=== ANALISI ===");
        System.out.printf("Temperatura media: %.1f gradi C\n", media);
        System.out.printf("Temperatura massima: %.1f gradi C (%s)\n",
                        massima, giorni[giornoMax]);
        System.out.printf("Temperatura minima: %.1f gradi C (%s)\n",
                        minima, giorni[giornoMin]);

        // Giorni sopra la media
        System.out.println("\nGiorni sopra la media:");
        for (int i = 0; i < temperature.size(); i++) {
            if (temperature.get(i) > media) {
                System.out.printf("- %s: %.1f gradi C\n",
                                giorni[i], temperature.get(i));
            }
        }

        scanner.close();
    }
}
\end{lstlisting}

\textbf{Spiegazione:} Il programma memorizza 7 temperature in un ArrayList, calcola la media sommando tutti i valori, trova massima e minima iterando sulla lista e visualizza i giorni sopra la media confrontando ogni elemento con la media calcolata.

\subsection{Esercizio 4.3: Rimozione duplicati}

\begin{lstlisting}
import java.util.ArrayList;

public class RimuoviDuplicati {

    public static ArrayList<String> rimuoviDuplicati(
            ArrayList<String> listaOriginale) {

        ArrayList<String> listaSenzaDuplicati = new ArrayList<>();

        // Itera sulla lista originale
        for (String elemento : listaOriginale) {
            // Aggiunge solo se non gia' presente nella nuova lista
            if (!listaSenzaDuplicati.contains(elemento)) {
                listaSenzaDuplicati.add(elemento);
            }
        }

        return listaSenzaDuplicati;
    }

    public static void main(String[] args) {
        // Test con lista contenente duplicati
        ArrayList<String> frutti = new ArrayList<>();
        frutti.add("Mela");
        frutti.add("Banana");
        frutti.add("Mela");
        frutti.add("Arancia");
        frutti.add("Banana");
        frutti.add("Kiwi");
        frutti.add("Mela");
        frutti.add("Arancia");

        System.out.println("=== LISTA ORIGINALE ===");
        System.out.println(frutti);
        System.out.println("Dimensione: " + frutti.size());

        ArrayList<String> senzaDuplicati = rimuoviDuplicati(frutti);

        System.out.println("\n=== LISTA SENZA DUPLICATI ===");
        System.out.println(senzaDuplicati);
        System.out.println("Dimensione: " + senzaDuplicati.size());

        // Verifica ordine mantenuto
        System.out.println("\nOrdine prima apparizione mantenuto: " +
                         senzaDuplicati.get(0).equals("Mela"));
    }
}
\end{lstlisting}

\textbf{Spiegazione:} Il metodo rimuoviDuplicati() crea un nuovo ArrayList e aggiunge ogni elemento solo se non è già presente (verificato con contains). Questo mantiene l'ordine di prima apparizione degli elementi.

\subsection{Esercizio 4.4-4.8}

Per motivi di spazio, le soluzioni complete degli esercizi 4.4 (Gestione biblioteca), 4.5 (Registro voti), 4.6 (Playlist musicale), 4.7 (Sistema prenotazioni cinema) e 4.8 (Social network semplificato) seguono lo stesso approccio degli esercizi precedenti: creano classi personalizzate, le gestiscono in ArrayList e implementano le funzionalità richieste usando i metodi add(), remove(), get(), contains() e iterazioni con for/for-each.


% Bibliografia
% Bibliografia (PHP)
\chapter*{Bibliografia e Risorse}
\addcontentsline{toc}{chapter}{Bibliografia e Risorse}

\section*{Manuali Ufficiali}

\begin{enumerate}
    \item \textbf{PHP Manual} \\
    Documentazione ufficiale completa \\
    \url{https://www.php.net/manual/en/}

    \item \textbf{PHP: The Right Way} \\
    Best practices PHP moderne \\
    \url{https://phptherightway.com/}

    \item \textbf{PHP Standards Recommendations (PSR)} \\
    PHP-FIG coding standards \\
    \url{https://www.php-fig.org/psr/}
\end{enumerate}

\section*{Libri di Testo}

\begin{enumerate}
    \item \textbf{PHP \& MySQL: Novice to Ninja} \\
    Kevin Yank \\
    SitePoint, 7th Edition, 2021 \\
    ISBN: 978-1925836387

    \item \textbf{Modern PHP: New Features and Good Practices} \\
    Josh Lockhart \\
    O'Reilly Media, 2015 \\
    ISBN: 978-1491905012

    \item \textbf{PHP Objects, Patterns, and Practice} \\
    Matt Zandstra \\
    Apress, 6th Edition, 2021 \\
    ISBN: 978-1484267905

    \item \textbf{Learning PHP, MySQL \& JavaScript} \\
    Robin Nixon \\
    O'Reilly Media, 6th Edition, 2021 \\
    ISBN: 978-1492093824

    \item \textbf{PHP Cookbook} \\
    David Sklar, Adam Trachtenberg \\
    O'Reilly Media, 4th Edition, 2020 \\
    ISBN: 978-1098121327
\end{enumerate}

\section*{Sicurezza Web}

\subsection*{OWASP Resources}

\begin{itemize}
    \item \textbf{OWASP Top 10} \\
    \url{https://owasp.org/Top10/} \\
    Le 10 vulnerabilità web più critiche (aggiornato 2021)

    \item \textbf{OWASP PHP Security Cheat Sheet} \\
    \url{https://cheatsheetseries.owasp.org/cheatsheets/PHP_Configuration_Cheat_Sheet.html}

    \item \textbf{OWASP Testing Guide} \\
    \url{https://owasp.org/www-project-web-security-testing-guide/}

    \item \textbf{OWASP Secure Coding Practices} \\
    \url{https://owasp.org/www-project-secure-coding-practices-quick-reference-guide/}
\end{itemize}

\subsection*{Libri Sicurezza}

\begin{enumerate}
    \item \textbf{The Tangled Web: A Guide to Securing Modern Web Applications} \\
    Michal Zalewski \\
    No Starch Press, 2011 \\
    ISBN: 978-1593273880

    \item \textbf{Web Application Security} \\
    Andrew Hoffman \\
    O'Reilly Media, 2020 \\
    ISBN: 978-1492053118
\end{enumerate}

\section*{Framework PHP}

\subsection*{Laravel}

\begin{itemize}
    \item \textbf{Documentazione ufficiale}: \url{https://laravel.com/docs}
    \item \textbf{Laracasts}: Video tutorial premium: \url{https://laracasts.com/}
    \item \textbf{Laravel News}: \url{https://laravel-news.com/}
\end{itemize}

\subsection*{Symfony}

\begin{itemize}
    \item \textbf{Documentazione}: \url{https://symfony.com/doc/current/index.html}
    \item \textbf{Symfony Blog}: \url{https://symfony.com/blog/}
\end{itemize}

\subsection*{Altri Framework}

\begin{itemize}
    \item \textbf{CodeIgniter}: \url{https://codeigniter.com/}
    \item \textbf{Slim Framework}: \url{https://www.slimframework.com/} (microframework)
    \item \textbf{Laminas (ex Zend)}: \url{https://getlaminas.org/}
    \item \textbf{CakePHP}: \url{https://cakephp.org/}
\end{itemize}

\section*{Tutorial e Corsi Online}

\subsection*{Tutorial Gratuiti}

\begin{itemize}
    \item \textbf{W3Schools PHP Tutorial} \\
    \url{https://www.w3schools.com/php/} \\
    Tutorial interattivo per principianti

    \item \textbf{PHP Tutorial - Tutorialspoint} \\
    \url{https://www.tutorialspoint.com/php/}

    \item \textbf{Learn PHP - Codecademy} \\
    \url{https://www.codecademy.com/learn/learn-php}

    \item \textbf{PHP for Beginners - Laracasts} \\
    \url{https://laracasts.com/series/php-for-beginners} (gratuito)
\end{itemize}

\subsection*{Piattaforme Video}

\begin{itemize}
    \item \textbf{Laracasts}: PHP e Laravel (subscription) \\
    \url{https://laracasts.com/}

    \item \textbf{Udemy}: Corsi PHP vari \\
    \url{https://www.udemy.com/topic/php/}

    \item \textbf{Pluralsight}: Percorsi PHP professionali \\
    \url{https://www.pluralsight.com/paths/php}

    \item \textbf{YouTube Channels}:
    \begin{itemize}
        \item Traversy Media: \url{https://www.youtube.com/c/TraversyMedia}
        \item The Net Ninja: \url{https://www.youtube.com/c/TheNetNinja}
        \item ProgramWithGio: \url{https://www.youtube.com/c/ProgramWithGio}
    \end{itemize}
\end{itemize}

\section*{Tool e Ambiente di Sviluppo}

\subsection*{Ambienti Locali}

\begin{itemize}
    \item \textbf{XAMPP}: \url{https://www.apachefriends.org/} (Windows/Mac/Linux)
    \item \textbf{MAMP}: \url{https://www.mamp.info/} (Mac/Windows)
    \item \textbf{Laragon}: \url{https://laragon.org/} (Windows, moderno)
    \item \textbf{Docker con PHP}: \url{https://hub.docker.com/_/php}
\end{itemize}

\subsection*{IDE e Editor}

\begin{itemize}
    \item \textbf{PhpStorm}: \url{https://www.jetbrains.com/phpstorm/} (professionale, licenza studenti gratuita)
    \item \textbf{Visual Studio Code} con estensioni:
    \begin{itemize}
        \item PHP Intelephense: \url{https://marketplace.visualstudio.com/items?itemName=bmewburn.vscode-intelephense-client}
        \item PHP Debug: \url{https://marketplace.visualstudio.com/items?itemName=xdebug.php-debug}
    \end{itemize}
    \item \textbf{Sublime Text}: \url{https://www.sublimetext.com/}
    \item \textbf{NetBeans}: \url{https://netbeans.apache.org/}
\end{itemize}

\subsection*{Debugging e Profiling}

\begin{itemize}
    \item \textbf{Xdebug}: \url{https://xdebug.org/} (debugger PHP)
    \item \textbf{Blackfire.io}: \url{https://blackfire.io/} (profiling performance)
    \item \textbf{New Relic}: \url{https://newrelic.com/} (monitoring APM)
    \item \textbf{PHPStan}: \url{https://phpstan.org/} (static analysis)
    \item \textbf{Psalm}: \url{https://psalm.dev/} (static analysis)
\end{itemize}

\subsection*{Testing}

\begin{itemize}
    \item \textbf{PHPUnit}: \url{https://phpunit.de/} (unit testing)
    \item \textbf{Pest PHP}: \url{https://pestphp.com/} (testing elegante)
    \item \textbf{Codeception}: \url{https://codeception.com/} (full-stack testing)
    \item \textbf{Behat}: \url{https://docs.behat.org/} (BDD)
\end{itemize}

\subsection*{Dependency Management}

\begin{itemize}
    \item \textbf{Composer}: \url{https://getcomposer.org/} (package manager PHP)
    \item \textbf{Packagist}: \url{https://packagist.org/} (repository Composer)
\end{itemize}

\section*{Community e Forum}

\begin{itemize}
    \item \textbf{Stack Overflow} \\
    Tag [php]: \url{https://stackoverflow.com/questions/tagged/php}

    \item \textbf{Reddit r/PHP} \\
    \url{https://www.reddit.com/r/PHP/}

    \item \textbf{PHP Developers Discord} \\
    Community Discord attiva

    \item \textbf{Laravel Discord} \\
    \url{https://discord.gg/laravel}

    \item \textbf{PHP User Groups} \\
    \url{https://www.php.net/ug.php} (gruppi locali)

    \item \textbf{Conferenze PHP}:
    \begin{itemize}
        \item PHP[World]
        \item Laracon
        \item SymfonyCon
        \item PHP UK Conference
    \end{itemize}
\end{itemize}

\section*{Database e Storage}

\begin{itemize}
    \item \textbf{MySQL Documentation}: \url{https://dev.mysql.com/doc/}
    \item \textbf{MariaDB Knowledge Base}: \url{https://mariadb.com/kb/}
    \item \textbf{PostgreSQL Documentation}: \url{https://www.postgresql.org/docs/}
    \item \textbf{PDO Manual}: \url{https://www.php.net/manual/en/book.pdo.php}
    \item \textbf{MySQLi Manual}: \url{https://www.php.net/manual/en/book.mysqli.php}
    \item \textbf{Redis}: \url{https://redis.io/} (caching)
    \item \textbf{Memcached}: \url{https://www.memcached.org/}
\end{itemize}

\section*{CMS e E-commerce}

\begin{itemize}
    \item \textbf{WordPress}: \url{https://wordpress.org/}
    \begin{itemize}
        \item Codex: \url{https://codex.wordpress.org/}
        \item Developer Handbook: \url{https://developer.wordpress.org/}
    \end{itemize}

    \item \textbf{Drupal}: \url{https://www.drupal.org/}
    
    \item \textbf{Joomla}: \url{https://www.joomla.org/}

    \item \textbf{Magento}: \url{https://magento.com/} (e-commerce)

    \item \textbf{WooCommerce}: \url{https://woocommerce.com/} (WordPress e-commerce)

    \item \textbf{PrestaShop}: \url{https://www.prestashop.com/}
\end{itemize}

\section*{API e Web Services}

\begin{itemize}
    \item \textbf{REST API Tutorial}: \url{https://restfulapi.net/}
    \item \textbf{GraphQL PHP}: \url{https://webonyx.github.io/graphql-php/}
    \item \textbf{Guzzle HTTP Client}: \url{https://docs.guzzlephp.org/}
    \item \textbf{cURL Manual}: \url{https://www.php.net/manual/en/book.curl.php}
    \item \textbf{JSON Web Tokens (JWT)}: \url{https://jwt.io/}
\end{itemize}

\section*{Deployment e DevOps}

\begin{itemize}
    \item \textbf{Deployer}: \url{https://deployer.org/} (deployment tool)
    \item \textbf{Laravel Forge}: \url{https://forge.laravel.com/} (server management)
    \item \textbf{Laravel Vapor}: \url{https://vapor.laravel.com/} (serverless)
    \item \textbf{Docker Compose per PHP}: Tutorial e best practices
    \item \textbf{GitHub Actions}: \url{https://github.com/features/actions} (CI/CD)
    \item \textbf{GitLab CI/CD}: \url{https://docs.gitlab.com/ee/ci/}
\end{itemize}

\section*{Sicurezza Avanzata}

\subsection*{Penetration Testing}

\begin{itemize}
    \item \textbf{DVWA (Damn Vulnerable Web Application)} \\
    \url{https://github.com/digininja.org/dvwa} \\
    Ambiente per testare vulnerabilità

    \item \textbf{bWAPP} \\
    \url{http://www.itsecgames.com/} \\
    Buggy web application per training

    \item \textbf{WebGoat PHP} \\
    Applicazione volutamente vulnerabile per apprendimento
\end{itemize}

\subsection*{Tool Sicurezza}

\begin{itemize}
    \item \textbf{Burp Suite}: \url{https://portswigger.net/burp} (penetration testing)
    \item \textbf{OWASP ZAP}: \url{https://www.zaproxy.org/} (security scanner)
    \item \textbf{Snyk}: \url{https://snyk.io/} (vulnerability scanning)
    \item \textbf{SonarQube}: \url{https://www.sonarqube.org/} (code quality \& security)
\end{itemize}

\section*{Risorse in Italiano}

\begin{itemize}
    \item \textbf{PHP.net Italian Manual} \\
    \url{https://www.php.net/manual/it/}

    \item \textbf{HTML.it - Guida PHP} \\
    \url{https://www.html.it/guide/guida-php-di-base/}

    \item \textbf{MRW.it - Tutorial PHP} \\
    \url{https://www.mrw.it/php/}

    \item \textbf{Forum HTML.it - PHP} \\
    \url{https://forum.html.it/forum/php}

    \item \textbf{Gruppo Facebook "PHP Italia"} \\
    Community italiana attiva
\end{itemize}

\section*{Newsletter e Blog}

\begin{itemize}
    \item \textbf{PHP Weekly}: \url{http://www.phpweekly.com/}
    \item \textbf{Laravel News}: \url{https://laravel-news.com/}
    \item \textbf{Freek.dev}: \url{https://freek.dev/} (Laravel, Spatie)
    \item \textbf{PHP Annotated Monthly} (JetBrains): \url{https://blog.jetbrains.com/phpstorm/}
    \item \textbf{Scotch.io**: \url{https://scotch.io/tag/php}
\end{itemize}

\section*{Certificazioni}

\begin{itemize}
    \item \textbf{Zend Certified PHP Engineer} \\
    \url{https://www.zend.com/training/php-certification-exam} \\
    Certificazione ufficiale PHP

    \item \textbf{Laravel Certification} \\
    \url{https://exam.laravelcert.com/}

    \item \textbf{Symfony Certification} \\
    \url{https://certification.symfony.com/}

    \item \textbf{CompTIA Security+} \\
    Certificazione sicurezza generale applicabile al web
\end{itemize}

\section*{Open Source Projects (per imparare)}

\begin{itemize}
    \item \textbf{Laravel Framework}: \url{https://github.com/laravel/framework}
    \item \textbf{Symfony Components}: \url{https://github.com/symfony/symfony}
    \item \textbf{Guzzle}: \url{https://github.com/guzzle/guzzle}
    \item \textbf{Monolog}: \url{https://github.com/Seldaek/monolog}
    \item \textbf{PHPMailer}: \url{https://github.com/PHPMailer/PHPMailer}
    \item \textbf{Carbon}: \url{https://github.com/briannesbitt/Carbon} (date/time)
\end{itemize}

\section*{Standard e Specifiche}

\begin{itemize}
    \item \textbf{PSR-1}: Basic Coding Standard
    \item \textbf{PSR-2}: Coding Style Guide (deprecato, usa PSR-12)
    \item \textbf{PSR-4}: Autoloading Standard
    \item \textbf{PSR-7}: HTTP Message Interface
    \item \textbf{PSR-12}: Extended Coding Style Guide
    \item \textbf{PSR-15}: HTTP Server Request Handlers
    \item \textbf{PSR-18}: HTTP Client

    Tutti disponibili su: \url{https://www.php-fig.org/psr/}
\end{itemize}

\section*{Podcast}

\begin{itemize}
    \item \textbf{PHP Roundtable}: \url{https://www.phproundtable.com/}
    \item \textbf{Laravel Podcast}: \url{https://laravelpodcast.com/}
    \item \textbf{PHP Ugly}: \url{https://www.phpugly.com/}
    \item \textbf{No Compromises}: Focus su Laravel e best practices
\end{itemize}

\section*{Note Finali}

Le risorse elencate rappresentano una selezione curata per approfondire PHP a tutti i livelli. Si consiglia di:

\begin{itemize}
    \item Iniziare con documentazione ufficiale e PHP: The Right Way
    \item Praticare con progetti personali applicando best practices
    \item Studiare codice open-source di qualità (Laravel, Symfony)
    \item Partecipare attivamente a community online
    \item Mantenere focus su sicurezza (OWASP) in ogni fase
    \item Considerare certificazioni per validazione competenze professionali
\end{itemize}

\vspace{1cm}

\begin{center}
\textit{La sicurezza non è un prodotto, ma un processo!}
\end{center}


\end{document}
