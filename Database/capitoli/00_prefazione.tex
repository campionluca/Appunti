\chapter*{Prefazione}
\addcontentsline{toc}{chapter}{Prefazione}

\section*{A chi si rivolge questo manuale}

Questi appunti sono stati pensati per studenti di istituti tecnici e professionali che si avvicinano per la prima volta al mondo dei database e del linguaggio SQL. Il percorso è strutturato per accompagnare progressivamente dalla teoria alla pratica, dalla progettazione concettuale all'implementazione fisica di un database.

\section*{Struttura del corso}

Il corso è organizzato in 13 capitoli che coprono l'intero ciclo di vita di un database:

\textbf{Parte I - Teoria e Modellazione} (Capitoli 1-5)
\begin{itemize}
    \item Introduzione ai DBMS e ai sistemi informativi
    \item Modellazione concettuale con diagrammi Entity-Relationship
    \item Modello logico relazionale e algebra relazionale
    \item Normalizzazione per garantire integrità dei dati
    \item Progettazione fisica e ottimizzazione
\end{itemize}

\textbf{Parte II - Linguaggio SQL} (Capitoli 6-12)
\begin{itemize}
    \item DDL: creazione e gestione strutture dati
    \item DML: interrogazione e manipolazione dati
    \item Query complesse con join e subquery
    \item Funzioni aggregate e raggruppamenti
    \item Gestione transazioni e concorrenza
\end{itemize}

\textbf{Parte III - Pratica} (Capitolo 13 + Appendice)
\begin{itemize}
    \item Esercizi completi di progettazione e implementazione
    \item Casi di studio reali
    \item Soluzioni commentate
\end{itemize}

\section*{Prerequisiti}

Per affrontare questo corso è consigliabile avere:
\begin{itemize}
    \item Conoscenze base di informatica e algoritmi
    \item Familiarità con la rappresentazione dei dati
    \item Capacità di ragionamento logico e problem solving
    \item (Opzionale) Conoscenza di un linguaggio di programmazione
\end{itemize}

\section*{Strumenti necessari}

\textbf{Software consigliato}:
\begin{itemize}
    \item \textbf{MySQL/MariaDB}: DBMS relazionale open source
    \item \textbf{MySQL Workbench}: Tool grafico per progettazione e query
    \item \textbf{phpMyAdmin}: Interfaccia web per gestione database
    \item \textbf{DBeaver}: Client universale multi-database
    \item \textbf{SQLite}: Database embedded per esercitazioni rapide
\end{itemize}

\textbf{Tool di modellazione}:
\begin{itemize}
    \item \textbf{draw.io}: Diagrammi ER online gratuiti
    \item \textbf{MySQL Workbench}: Modellazione integrata
    \item \textbf{Lucidchart}: Diagrammi professionali
\end{itemize}

\section*{Come studiare}

Per ottenere il massimo da questi appunti:

\begin{enumerate}
    \item \textbf{Leggi attentamente la teoria}: Ogni concetto è spiegato con esempi e diagrammi
    \item \textbf{Disegna i diagrammi}: La modellazione si impara praticando
    \item \textbf{Scrivi le query SQL}: Digita personalmente ogni query, non copiare
    \item \textbf{Testa sul database}: Crea database di prova e verifica i risultati
    \item \textbf{Risolvi gli esercizi}: Prova autonomamente prima di vedere le soluzioni
    \item \textbf{Progetta casi reali}: Applica i concetti a scenari del mondo reale
\end{enumerate}

\begin{nota}
Questo manuale segue lo standard \textbf{SQL ANSI/ISO} con focus su \textbf{MySQL} per gli esempi pratici. La maggior parte delle query funziona su tutti i DBMS relazionali (PostgreSQL, SQL Server, Oracle) con minime modifiche.
\end{nota}

\section*{Convenzioni tipografiche}

Nel testo vengono utilizzate le seguenti convenzioni:

\begin{itemize}
    \item \texttt{Codice SQL}: Comandi e query in carattere monospace
    \item \textbf{Parole chiave SQL}: In grassetto nelle spiegazioni (SELECT, WHERE)
    \item \textit{Nomi di tabelle/attributi}: In corsivo per riferimenti (Studente, nome)
    \item Box colorati: Attenzioni, Note, Errori Comuni, Esempi
\end{itemize}

\section*{Sito web e risorse}

Materiale aggiuntivo disponibile su:
\begin{itemize}
    \item Repository GitHub: \url{https://github.com/campionluca/Appunti}
    \item Database di esempio scaricabili
    \item Script SQL per esercitazioni
    \item Soluzioni interattive
\end{itemize}

\section*{Ringraziamenti}

Si ringrazia l'Istituto Tecnico Antonio Scarpa per il supporto nella realizzazione di questo materiale didattico.

\vspace{1cm}

\begin{flushright}
\textit{Prof. Luca Campion}\\
Novembre 2025
\end{flushright}

\section*{Note sulla versione}

\textbf{Versione 1.0} - Novembre 2025
\begin{itemize}
    \item Prima release completa
    \item 13 capitoli + appendice soluzioni
    \item Coverage SQL: DDL, DML, DCL, TCL
    \item Esempi MySQL 8.0+
    \item Diagrammi ER con TikZ
\end{itemize}
