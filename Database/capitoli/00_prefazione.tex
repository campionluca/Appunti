\chapter*{Prefazione}
\addcontentsline{toc}{chapter}{Prefazione}

\section*{A chi si rivolge questo manuale}

Questi appunti sono stati pensati per studenti di istituti tecnici e professionali che si avvicinano per la prima volta al mondo dei database e del linguaggio SQL. Il percorso è strutturato per accompagnare progressivamente dalla teoria alla pratica, dalla progettazione concettuale all'implementazione fisica di un database.

\section*{Struttura del corso}

Il corso è organizzato in 13 capitoli che coprono l'intero ciclo di vita di un database:

\textbf{Parte I - Teoria e Modellazione} (Capitoli 1-5)

Questa prima parte costituisce il fondamento teorico essenziale del corso. Si introduce ai Database Management Systems (DBMS) e ai sistemi informativi, comprendendo il loro ruolo nell'architettura software moderna. Si procede con la modellazione concettuale attraverso i diagrammi Entity-Relationship, strumento fondamentale per rappresentare visivamente la struttura dei dati. Il percorso continua esplorando il modello logico relazionale e l'algebra relazionale, che traducono i concetti astratti in strutture implementabili. La normalizzazione garantisce l'integrità dei dati eliminando ridondanze e anomalie. Infine, si affronta la progettazione fisica e l'ottimizzazione per massimizzare le prestazioni del sistema.

\textbf{Parte II - Linguaggio SQL} (Capitoli 6-12)

La seconda parte si concentra sul linguaggio SQL, lo standard de facto per interagire con i database relazionali. Si inizia con il DDL (Data Definition Language) per creare e gestire le strutture dati, definendo tabelle, vincoli e indici. Il DML (Data Manipulation Language) permette di interrogare e manipolare i dati attraverso query SELECT, INSERT, UPDATE e DELETE. Si approfondiscono le query complesse utilizzando join per combinare dati da più tabelle e subquery per logiche annidate. Le funzioni aggregate e i raggruppamenti consentono di eseguire analisi e report sofisticati. La parte si conclude con la gestione delle transazioni e della concorrenza, aspetti cruciali per garantire l'integrità dei dati in ambienti multi-utente.

\textbf{Parte III - Pratica} (Capitolo 13 + Appendice)

La terza e ultima parte consolida l'apprendimento attraverso la pratica. Gli esercizi completi di progettazione e implementazione permettono di applicare tutti i concetti teorici appresi nelle parti precedenti. I casi di studio reali mostrano come affrontare problematiche concrete del mondo professionale, dalla raccolta dei requisiti all'implementazione finale. Le soluzioni commentate forniscono non solo le risposte corrette, ma anche spiegazioni dettagliate del ragionamento e delle scelte progettuali, facilitando l'apprendimento attraverso l'analisi critica.

\section*{Prerequisiti}

Per affrontare questo corso è consigliabile avere:
\begin{itemize}
    \item Conoscenze base di informatica e algoritmi
    \item Familiarità con la rappresentazione dei dati
    \item Capacità di ragionamento logico e problem solving
    \item (Opzionale) Conoscenza di un linguaggio di programmazione
\end{itemize}

\section*{Strumenti necessari}

\textbf{Software consigliato}:
\begin{itemize}
    \item \textbf{MySQL/MariaDB}: DBMS relazionale open source
    \item \textbf{MySQL Workbench}: Tool grafico per progettazione e query
    \item \textbf{phpMyAdmin}: Interfaccia web per gestione database
    \item \textbf{DBeaver}: Client universale multi-database
    \item \textbf{SQLite}: Database embedded per esercitazioni rapide
\end{itemize}

\textbf{Tool di modellazione}:
\begin{itemize}
    \item \textbf{draw.io}: Diagrammi ER online gratuiti
    \item \textbf{MySQL Workbench}: Modellazione integrata
    \item \textbf{Lucidchart}: Diagrammi professionali
\end{itemize}

\section*{Come studiare}

Per ottenere il massimo da questi appunti:

\begin{enumerate}
    \item \textbf{Leggi attentamente la teoria}: Ogni concetto è spiegato con esempi e diagrammi
    \item \textbf{Disegna i diagrammi}: La modellazione si impara praticando
    \item \textbf{Scrivi le query SQL}: Digita personalmente ogni query, non copiare
    \item \textbf{Testa sul database}: Crea database di prova e verifica i risultati
    \item \textbf{Risolvi gli esercizi}: Prova autonomamente prima di vedere le soluzioni
    \item \textbf{Progetta casi reali}: Applica i concetti a scenari del mondo reale
\end{enumerate}

\begin{nota}
Questo manuale segue lo standard \textbf{SQL ANSI/ISO} con focus su \textbf{MySQL} per gli esempi pratici. La maggior parte delle query funziona su tutti i DBMS relazionali (PostgreSQL, SQL Server, Oracle) con minime modifiche.
\end{nota}

\section*{Convenzioni tipografiche}

Nel testo vengono utilizzate le seguenti convenzioni:

\begin{itemize}
    \item \texttt{Codice SQL}: Comandi e query in carattere monospace
    \item \textbf{Parole chiave SQL}: In grassetto nelle spiegazioni (SELECT, WHERE)
    \item \textit{Nomi di tabelle/attributi}: In corsivo per riferimenti (Studente, nome)
    \item Box colorati: Attenzioni, Note, Errori Comuni, Esempi
\end{itemize}

\section*{Sito web e risorse}

Materiale aggiuntivo disponibile su:
\begin{itemize}
    \item Repository GitHub: \url{https://github.com/campionluca/Appunti}
    \item Database di esempio scaricabili
    \item Script SQL per esercitazioni
    \item Soluzioni interattive
\end{itemize}

\section*{Ringraziamenti}

Si ringrazia l'Istituto Tecnico Antonio Scarpa per il supporto nella realizzazione di questo materiale didattico.

\vspace{1cm}

\begin{flushright}
\textit{Prof. Luca Campion}\\
Novembre 2025
\end{flushright}

\section*{Note sulla versione}

\textbf{Versione 1.0} - Novembre 2025
\begin{itemize}
    \item Prima release completa
    \item 13 capitoli + appendice soluzioni
    \item Coverage SQL: DDL, DML, DCL, TCL
    \item Esempi MySQL 8.0+
    \item Diagrammi ER con TikZ
\end{itemize}
