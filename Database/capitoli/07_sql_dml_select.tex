\chapter{SQL DML - SELECT}

\section*{Introduzione}
DML (Data Manipulation Language) è il linguaggio SQL per interrogare e manipolare i dati. Il comando SELECT è il più importante e potente di DML, permettendo di recuperare dati da una o più tabelle con filtri, ordinamenti e proiezioni sofisticate.

\section*{Obiettivi di apprendimento}
\begin{itemize}
    \item Scrivere query SELECT di base
    \item Filtrare dati con WHERE
    \item Ordinare risultati con ORDER BY
    \item Limitare risultati con LIMIT
    \item Usare funzioni di aggregazione (COUNT, SUM, AVG, MIN, MAX)
    \item Raggruppare dati con GROUP BY
    \item Filtrare gruppi con HAVING
    \item Usare alias per colonne e tabelle
    \item Scrivere query sub-SELECT (subquery)
\end{itemize}

\section{Sintassi SELECT di Base}

\subsection{Sintassi generale}

\begin{lstlisting}[language=SQL, caption=Sintassi SELECT]
SELECT [DISTINCT] colonna1, colonna2, ...
FROM tabella1
WHERE condizioni
ORDER BY colonna [ASC|DESC]
LIMIT numero_righe;
\end{lstlisting}

\subsection{Selezionare tutte le colonne}

\begin{lstlisting}[language=SQL, caption=Selezionare tutte le colonne]
-- Seleziona tutte le colonne
SELECT * FROM cliente;

-- Limita ai primi 10 risultati
SELECT * FROM cliente LIMIT 10;
\end{lstlisting}

\subsection{Selezionare colonne specifiche}

\begin{lstlisting}[language=SQL, caption=Selezionare colonne specifiche]
SELECT idCliente, nome, cognome, email FROM cliente;
\end{lstlisting}

\subsection{Alias per colonne}

\begin{lstlisting}[language=SQL, caption=Alias per colonne]
-- Renominare colonne nel risultato
SELECT
    idCliente AS ID,
    nome AS Nome,
    email AS 'Indirizzo Email'
FROM cliente;
\end{lstlisting}

\section{WHERE - Filtri}

Il WHERE filtra le righe in base a condizioni.

\subsection{Operatori di confronto}

\begin{lstlisting}[language=SQL, caption=Operatori di confronto]
-- Uguaglianza
SELECT * FROM cliente WHERE stato = 'Attivo';

-- Diverso
SELECT * FROM cliente WHERE stato != 'Inattivo';
SELECT * FROM cliente WHERE stato <> 'Inattivo';

-- Maggiore/minore
SELECT * FROM ordine WHERE totale > 100.00;
SELECT * FROM ordine WHERE totale <= 50.00;

-- BETWEEN (inclusivo)
SELECT * FROM ordine WHERE dataOrdine BETWEEN '2023-01-01' AND '2023-12-31';
\end{lstlisting}

\subsection{Operatori logici}

\begin{lstlisting}[language=SQL, caption=Operatori logici AND, OR, NOT]
-- AND: tutte le condizioni devono essere vere
SELECT * FROM cliente
WHERE stato = 'Attivo' AND città = 'Milano';

-- OR: almeno una condizione deve essere vera
SELECT * FROM cliente
WHERE città = 'Milano' OR città = 'Roma';

-- NOT: negazione
SELECT * FROM cliente
WHERE NOT stato = 'Inattivo';
\end{lstlisting}

\subsection{IN e NOT IN}

\begin{lstlisting}[language=SQL, caption=Operatori IN e NOT IN]
-- IN: verifica se il valore è in un elenco
SELECT * FROM ordine
WHERE stato IN ('Spedito', 'Consegnato');

-- NOT IN: verifica se il valore NON è in un elenco
SELECT * FROM ordine
WHERE stato NOT IN ('Cancellato', 'Rifiutato');
\end{lstlisting}

\subsection{LIKE - Pattern matching}

\begin{lstlisting}[language=SQL, caption=LIKE per pattern matching]
-- % rappresenta 0 o più caratteri, _ rappresenta un carattere singolo

-- Nome che inizia con 'M'
SELECT * FROM cliente WHERE nome LIKE 'M%';

-- Nome che finisce con 'o'
SELECT * FROM cliente WHERE nome LIKE '%o';

-- Email che contiene 'gmail'
SELECT * FROM cliente WHERE email LIKE '%gmail%';

-- Parola di esattamente 5 caratteri
SELECT * FROM cliente WHERE nome LIKE '_____';

-- Case-insensitive (dipende dal collation)
SELECT * FROM cliente WHERE nome LIKE 'mario%';
\end{lstlisting}

\subsection{IS NULL e IS NOT NULL}

\begin{lstlisting}[language=SQL, caption=NULL in WHERE]
-- Valori NULL
SELECT * FROM cliente WHERE telefono IS NULL;

-- Valori non NULL
SELECT * FROM cliente WHERE telefono IS NOT NULL;

-- Nota: = NULL non funziona! Usare IS NULL
SELECT * FROM cliente WHERE telefono = NULL;  -- Restituisce 0 righe!
\end{lstlisting}

\section{ORDER BY - Ordinamento}

Ordina i risultati in base a una o più colonne.

\begin{lstlisting}[language=SQL, caption=ORDER BY]
-- Ordinamento ascendente (default)
SELECT * FROM ordine ORDER BY dataOrdine ASC;

-- Ordinamento discendente
SELECT * FROM ordine ORDER BY dataOrdine DESC;

-- Ordinamento su più colonne
SELECT * FROM cliente
ORDER BY città ASC, cognome ASC, nome ASC;

-- Ordinamento per numero di colonna (sconsigliato)
SELECT nome, cognome, email FROM cliente
ORDER BY 2, 1;  -- Ordina per colonna 2 (cognome), poi colonna 1 (nome)
\end{lstlisting}

\section{LIMIT - Limitazione Risultati}

Limita il numero di righe restituite.

\begin{lstlisting}[language=SQL, caption=LIMIT]
-- Primi 10 risultati
SELECT * FROM cliente LIMIT 10;

-- 10 risultati a partire dal 20° (offset=20, limit=10)
SELECT * FROM cliente LIMIT 20, 10;

-- Sintassi alternativa
SELECT * FROM cliente LIMIT 10 OFFSET 20;

-- Ultimi 5 clienti registrati
SELECT * FROM cliente
ORDER BY dataRegistrazione DESC
LIMIT 5;
\end{lstlisting}

\section{DISTINCT - Eliminare Duplicati}

Restituisce solo valori unici.

\begin{lstlisting}[language=SQL, caption=DISTINCT]
-- Tutte le città
SELECT DISTINCT città FROM cliente;

-- Combinazione unica di città e stato
SELECT DISTINCT città, stato FROM cliente;

-- Contare città distinte
SELECT COUNT(DISTINCT città) FROM cliente;
\end{lstlisting}

\section{Funzioni di Stringa}

\begin{lstlisting}[language=SQL, caption=Funzioni di stringa]
-- Lunghezza stringa
SELECT nome, LENGTH(nome) AS lunghezza FROM cliente;

-- Concatenazione
SELECT CONCAT(nome, ' ', cognome) AS nomeCompleto FROM cliente;

-- Maiuscolo/minuscolo
SELECT UPPER(nome), LOWER(cognome) FROM cliente;

-- Substring
SELECT SUBSTRING(email, 1, POSITION('@' IN email) - 1) AS username FROM cliente;

-- Sostituzione
SELECT REPLACE(email, 'gmail.com', 'email.com') FROM cliente;

-- Trim (rimuove spazi)
SELECT TRIM(nome) FROM cliente;
\end{lstlisting}

\section{Funzioni di Data}

\begin{lstlisting}[language=SQL, caption=Funzioni di data]
-- Data odierna
SELECT CURDATE() AS oggi;
SELECT CURRENT_DATE() AS oggi;

-- Data e ora attuali
SELECT NOW() AS adesso;
SELECT CURRENT_TIMESTAMP() AS adesso;

-- Estrarre parti della data
SELECT
    YEAR(dataOrdine) AS anno,
    MONTH(dataOrdine) AS mese,
    DAY(dataOrdine) AS giorno
FROM ordine;

-- Differenza tra date (in giorni)
SELECT
    idOrdine,
    DATEDIFF(CURDATE(), dataOrdine) AS giorni_passati
FROM ordine;

-- Aggiungere giorni a una data
SELECT DATE_ADD(CURDATE(), INTERVAL 7 DAY) AS tra_una_settimana;
SELECT DATE_ADD(CURDATE(), INTERVAL 1 MONTH) AS tra_un_mese;

-- Formato data
SELECT DATE_FORMAT(dataOrdine, '%d/%m/%Y') FROM ordine;
\end{lstlisting}

\section{Funzioni Numeriche}

\begin{lstlisting}[language=SQL, caption=Funzioni numeriche]
-- Arrotondamento
SELECT ROUND(prezzo, 2) FROM prodotto;

-- Arrotondamento per difetto/eccesso
SELECT FLOOR(prezzo), CEIL(prezzo) FROM prodotto;

-- Valore assoluto
SELECT ABS(-10);

-- Potenza
SELECT POWER(2, 3);  -- 2^3 = 8

-- Radice quadrata
SELECT SQRT(16);  -- 4

-- Modulo (resto divisione)
SELECT MOD(10, 3);  -- 1
\end{lstlisting}

\section{Subquery (Sub-SELECT)}

Una subquery è una query dentro un'altra query.

\begin{lstlisting}[language=SQL, caption=Subquery in WHERE]
-- Trovare clienti che hanno fatto ordini superiori alla media
SELECT *
FROM cliente
WHERE idCliente IN (
    SELECT idCliente FROM ordine WHERE totale > (
        SELECT AVG(totale) FROM ordine
    )
);

-- Trovare prodotti con prezzo superiore al prezzo medio della categoria
SELECT *
FROM prodotto p1
WHERE prezzo > (
    SELECT AVG(prezzo) FROM prodotto p2
    WHERE p2.categoria = p1.categoria
);
\end{lstlisting}

\begin{lstlisting}[language=SQL, caption=Subquery in SELECT]
-- Aggiungere il numero di ordini accanto a ogni cliente
SELECT
    idCliente,
    nome,
    (SELECT COUNT(*) FROM ordine o WHERE o.idCliente = c.idCliente) AS numOrdini
FROM cliente c;
\end{lstlisting}

\section{Esempio Completo}

\begin{lstlisting}[language=SQL, caption=Query complessa]
-- Trovare i 5 clienti più attivi del 2023
-- ordinati per numero di ordini decrescente
SELECT
    c.idCliente,
    c.nome,
    c.cognome,
    COUNT(o.idOrdine) AS numOrdini,
    SUM(o.totale) AS totalSpeso
FROM cliente c
LEFT JOIN ordine o ON c.idCliente = o.idCliente
    AND YEAR(o.dataOrdine) = 2023
WHERE c.stato = 'Attivo'
GROUP BY c.idCliente, c.nome, c.cognome
ORDER BY numOrdini DESC
LIMIT 5;
\end{lstlisting}

\section*{Riepilogo concetti chiave}

\begin{tcolorbox}[colback=gray!10, colframe=black!60, title=Concetti fondamentali]
\begin{itemize}
    \item \textbf{SELECT} recupera dati da tabelle
    \item \textbf{WHERE} filtra righe in base a condizioni
    \item \textbf{ORDER BY} ordina risultati (ASC/DESC)
    \item \textbf{LIMIT} limita il numero di righe
    \item \textbf{DISTINCT} elimina duplicati
    \item \textbf{Funzioni} (stringa, data, numero) trasformano i dati
    \item \textbf{Subquery} permettono query annidate
    \item Combinare clausole per query sofisticate
\end{itemize}
\end{tcolorbox}

\section*{Esercizi}

\begin{enumerate}
    \item Scrivi una query per trovare tutti i clienti della città di 'Milano' registrati negli ultimi 30 giorni, ordinati per cognome.

    \item Scrivi una query per elencare prodotti il cui prezzo è tra 50 e 100 euro, escludendo la categoria 'Saldi'.

    \item Conta quanti ordini sono stati fatti da cliente 'id=5' e mostra il numero come colonna 'NumOrdini'.

    \item Usa una subquery per trovare i clienti che hanno speso più della media totale di tutti gli ordini.

    \item Scrivi una query che mostra, per ogni cliente: nome, cognome, numero di ordini effettuati, importo totale speso, ordinato per importo totale decrescente. Usa LEFT JOIN se necessario.

    \item Estrai da una tabella 'articolo' gli articoli con titolo che contiene 'Database' (case-insensitive), creati nel 2023 o successivamente, ordinati per data di creazione.
\end{enumerate}
