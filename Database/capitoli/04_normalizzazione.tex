\chapter{Normalizzazione del Database}

\section*{Introduzione}
La normalizzazione è un processo sistematico per organizzare i dati eliminando ridondanze e anomalie. Attraverso forme normali progressive (1NF, 2NF, 3NF, BCNF), garantiamo un design del database efficiente, coerente e facile da mantenere.

\section*{Obiettivi di apprendimento}
\begin{itemize}
    \item Comprendere il concetto di dipendenza funzionale
    \item Applicare le regole della prima forma normale (1NF)
    \item Applicare le regole della seconda forma normale (2NF)
    \item Applicare le regole della terza forma normale (3NF)
    \item Comprendere la forma normale di Boyce-Codd (BCNF)
    \item Identificare e risolvere anomalie di inserimento, aggiornamento e cancellazione
\end{itemize}

\section{Dipendenze Funzionali}

\subsection{Definizione}
Una \textbf{dipendenza funzionale} (DF) X -> Y fra due insiemi di attributi X e Y di una relazione R significa che, se due tuple hanno gli stessi valori per gli attributi in X, allora devono avere gli stessi valori per gli attributi in Y.

\begin{tcolorbox}[colback=blue!10, colframe=blue!60, title=Esempio: Dipendenza funzionale]
Relazione: Studente(idStudente, nome, email, idCorso, nomeCorso)

Dipendenze funzionali:
\begin{itemize}
    \item \texttt{idStudente} -> \texttt{nome, email}: dato uno studente, c'è un solo nome e email
    \item \texttt{idCorso} -> \texttt{nomeCorso}: dato un corso, c'è un solo nome
\end{itemize}
\end{tcolorbox}

\subsection{Chiave primaria e dipendenze}
Se un attributo è parte della chiave primaria, determina funzionalmente tutti gli altri attributi della relazione.

\section{Anomalie nei Database non normalizzati}

\subsection{Anomalia di inserimento}
Non è possibile inserire dati fino a quando non sono disponibili tutti i dati della chiave primaria.

\begin{tcolorbox}[colback=red!10, colframe=red!60, title=Esempio: Anomalia di inserimento]
Tabella: Corso(idCorso, nomeCorsso, idDocente, nomeDocente)

Non posso inserire un docente che non insegna ancora nessun corso perché \texttt{idCorso} è parte della chiave primaria.
\end{tcolorbox}

\subsection{Anomalia di aggiornamento}
Aggiornare un valore può richiedere di aggiornare più tuple, rischiando inconsistenze.

\begin{tcolorbox}[colback=red!10, colframe=red!60, title=Esempio: Anomalia di aggiornamento]
Tabella: Studente(idStudente, nome, idFacoltà, nomeFacoltà)

Se cambio il nome della Facoltà da ``Ingegneria'' a ``Ingegneria e Architettura'', devo aggiornare tutte le righe degli studenti di quella facoltà. Se dimentico una riga, il database diventa inconsistente.
\end{tcolorbox}

\subsection{Anomalia di cancellazione}
Cancellare un dato può comportare la perdita di informazioni non correlate.

\begin{tcolorbox}[colback=red!10, colframe=red!60, title=Esempio: Anomalia di cancellazione]
Tabella: Fornitore_Prodotto(idFornitore, nomeFornitore, idProdotto, nomeProdotto)

Se cancello l'ultimo prodotto fornito da un fornitore, perdo anche i dati del fornitore dal database.
\end{tcolorbox}

\section{Prima Forma Normale (1NF)}

La forma normale più basilare: \textbf{una relazione è in 1NF se tutti gli attributi contengono solo valori atomici} (non divisibili).

\subsection{Requisiti}
\begin{itemize}
    \item Nessun attributo può contenere liste o insiemi di valori
    \item Nessun attributo ripetuto
    \item Ogni riga deve avere lo stesso numero di colonne
    \item Nessun gruppo ripetuto
\end{itemize}

\subsection{Esempio di violazione 1NF}

\begin{tcolorbox}[colback=red!10, colframe=red!60, title=Esempio: Non è in 1NF}
\begin{center}
\begin{tabular}{|c|c|c|}
\hline
\textbf{idStudente} & \textbf{nome} & \textbf{corsi} \\
\hline
1 & Mario & Matematica, Fisica, Chimica \\
2 & Luigi & Matematica, Biologia \\
\hline
\end{tabular}

L'attributo \texttt{corsi} contiene valori non atomici (liste).
\end{center}
\end{tcolorbox}

\subsection{Correzione: Applicare 1NF}

\begin{tcolorbox}[colback=green!10, colframe=green!60, title=Esempio: In 1NF}
\begin{center}
\begin{tabular}{|c|c|c|}
\hline
\textbf{idStudente} & \textbf{nome} & \textbf{corso} \\
\hline
1 & Mario & Matematica \\
1 & Mario & Fisica \\
1 & Mario & Chimica \\
2 & Luigi & Matematica \\
2 & Luigi & Biologia \\
\hline
\end{tabular}

Oppure creare una tabella separata per gli inscritti ai corsi (relazione N:M).
\end{center}
\end{tcolorbox}

\begin{lstlisting}[language=SQL, caption=DDL in 1NF con tabella di giunzione]
CREATE TABLE studente (
    idStudente INT PRIMARY KEY,
    nome VARCHAR(100)
);

CREATE TABLE corso (
    idCorso INT PRIMARY KEY,
    nomeCors VARCHAR(100)
);

CREATE TABLE iscrizione (
    idStudente INT NOT NULL,
    idCorso INT NOT NULL,
    PRIMARY KEY (idStudente, idCorso),
    FOREIGN KEY (idStudente) REFERENCES studente(idStudente),
    FOREIGN KEY (idCorso) REFERENCES corso(idCorso)
);
\end{lstlisting}

\section{Seconda Forma Normale (2NF)}

Una relazione è in 2NF se:
\begin{enumerate}
    \item È in 1NF
    \item \textbf{Ogni attributo non-chiave è dipendente dalla chiave primaria completa}, non solo da parte di essa.
\end{enumerate}

\textbf{Problema}: Dipendenza parziale. Un attributo non-chiave dipende solo da una parte della chiave primaria.

\subsection{Esempio di violazione 2NF}

\begin{tcolorbox}[colback=red!10, colframe=red!60, title=Esempio: Non è in 2NF}
Tabella: Iscrizione(idStudente, idCorso, nomeCors, voto)

\textbf{Chiave primaria}: (idStudente, idCorso)

\textbf{Problema}: \texttt{nomeCors} dipende solo da \texttt{idCorso}, non dalla chiave completa. Questa è una dipendenza parziale.
\end{tcolorbox}

\subsection{Correzione: Applicare 2NF}

\begin{lstlisting}[language=SQL, caption=DDL in 2NF]
CREATE TABLE studente (
    idStudente INT PRIMARY KEY,
    nome VARCHAR(100)
);

CREATE TABLE corso (
    idCorso INT PRIMARY KEY,
    nomeCors VARCHAR(100)
);

-- Ora in 2NF: nomeCors è in tabella corso, non in iscrizione
CREATE TABLE iscrizione (
    idStudente INT NOT NULL,
    idCorso INT NOT NULL,
    voto INT,
    PRIMARY KEY (idStudente, idCorso),
    FOREIGN KEY (idStudente) REFERENCES studente(idStudente),
    FOREIGN KEY (idCorso) REFERENCES corso(idCorso)
);
\end{lstlisting}

\section{Terza Forma Normale (3NF)}

Una relazione è in 3NF se:
\begin{enumerate}
    \item È in 2NF
    \item \textbf{Nessun attributo non-chiave dipende da un altro attributo non-chiave}.
\end{enumerate}

\textbf{Problema}: Dipendenza transitiva. Un attributo non-chiave dipende funzionalmente da un altro attributo non-chiave.

\subsection{Esempio di violazione 3NF}

\begin{tcolorbox}[colback=red!10, colframe=red!60, title=Esempio: Non è in 3NF}
Tabella: Ordine(idOrdine, dataOrdine, idCliente, nomeCliente, indirizzo)

\textbf{Dipendenze funzionali}:
\begin{itemize}
    \item \texttt{idOrdine} -> \texttt{dataOrdine, idCliente, nomeCliente, indirizzo}
    \item \texttt{idCliente} -> \texttt{nomeCliente, indirizzo} (dipendenza transitiva!)
\end{itemize}

\texttt{nomeCliente} e \texttt{indirizzo} dipendono da \texttt{idCliente}, non direttamente da \texttt{idOrdine}.
\end{tcolorbox}

\subsection{Correzione: Applicare 3NF}

\begin{lstlisting}[language=SQL, caption=DDL in 3NF]
CREATE TABLE cliente (
    idCliente INT PRIMARY KEY,
    nomeCliente VARCHAR(100),
    indirizzo VARCHAR(200)
);

CREATE TABLE ordine (
    idOrdine INT PRIMARY KEY,
    dataOrdine DATE,
    idCliente INT NOT NULL,
    FOREIGN KEY (idCliente) REFERENCES cliente(idCliente)
);
\end{lstlisting}

\section{Forma Normale di Boyce-Codd (BCNF)}

Una relazione è in BCNF se:
\begin{itemize}
    \item Per ogni dipendenza funzionale X -> Y, X è una chiave candidata.
\end{itemize}

BCNF è una forma più stringente di 3NF. La maggior parte dei database in 3NF è già in BCNF.

\subsection{Esempio di violazione BCNF}

\begin{tcolorbox}[colback=red!10, colframe=red!60, title=Esempio: In 3NF ma non in BCNF}
Tabella: Professore_Corso(idProfessore, idCorso, orario)

\textbf{Chiave primaria}: (idProfessore, idCorso)

\textbf{Dipendenza funzionale}: (idProfessore, idCorso) -> orario (OK)

Ma se esiste anche: idCorso -> orario (tutti i corsi hanno un orario fisso)

Allora idCorso determina orario, ma idCorso non è una chiave candidata. Violazione BCNF!
\end{tcolorbox}

\section{Procedura di Normalizzazione}

\begin{tcolorbox}[colback=blue!10, colframe=blue!60, title=Passi di normalizzazione]
\begin{enumerate}
    \item \textbf{Identifica dipendenze funzionali}: Quali attributi determinano quali altri.
    \item \textbf{Applica 1NF}: Assicura valori atomici.
    \item \textbf{Applica 2NF}: Elimina dipendenze parziali.
    \item \textbf{Applica 3NF}: Elimina dipendenze transitive.
    \item \textbf{Applica BCNF}: Se necessario, per relazioni più complesse.
\end{enumerate}
\end{tcolorbox}

\section*{Riepilogo concetti chiave}

\begin{tcolorbox}[colback=gray!10, colframe=black!60, title=Concetti fondamentali]
\begin{itemize}
    \item Una \textbf{dipendenza funzionale} X -> Y significa che X determina Y
    \item La \textbf{1NF} richiede valori atomici (no liste)
    \item La \textbf{2NF} elimina dipendenze parziali dalla chiave primaria
    \item La \textbf{3NF} elimina dipendenze transitive tra attributi non-chiave
    \item La \textbf{BCNF} è una versione più stringente di 3NF
    \item La normalizzazione riduce ridondanze e anomalie
\end{itemize}
\end{tcolorbox}

\section*{Esercizi}

\begin{enumerate}
    \item Identifica tutte le dipendenze funzionali nella seguente tabella:
    Libro(idLibro, titolo, autore, idAutore, nazionaleAutore, genere, prezzo)

    \item La seguente tabella è in 1NF? Se no, normalizza:
    Impiegato(matricola, nome, skills)  [skills contiene ``Java, Python, SQL'']

    \item La seguente tabella è in 2NF? Se no, normalizza:
    Voto(idStudente, nomeStudente, idCorso, nomeCorso, voto)

    \item La seguente tabella è in 3NF? Se no, normalizza:
    Ospedale(idOspedale, nomeOspedale, città, idRepartimento, nomeRepartimento, idMedico, nomeMedico)

    \item Progetta uno schema completamente normalizzato (3NF) per un sistema di prenotazioni alberghiere con: Hotel, Stanze, Clienti, Prenotazioni. Includi vincoli appropriati.
\end{enumerate}
