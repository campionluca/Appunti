\chapter*{Bibliografia e Risorse}

\addcontentsline{toc}{chapter}{Bibliografia e Risorse}

\section*{Introduzione}

Questo capitolo fornisce una lista curata di risorse per approfondire i concetti di database, SQL e progettazione. Include libri classici, articoli accademici, risorse online e strumenti pratici.

\section{Libri Consigliati}

\subsection{Testi Fondamentali}

\begin{description}
    \item[\textbf{Database System Concepts}] di Abraham Silberschatz, Henry Korth, S. Sudarshan
    \begin{itemize}
        \item \textit{Casa editrice}: McGraw-Hill
        \item \textit{Livello}: Intermedio-Avanzato
        \item \textit{Argomenti}: Teoria relazionale, query optimization, concurrency control, recovery
        \item \textit{Considerazione}: Testo accademico più completo, ideale per approfondimento teorico
    \end{itemize}

    \item[\textbf{SQL Performance Explained}] di Markus Winand
    \begin{itemize}
        \item \textit{Casa editrice}: Self-published
        \item \textit{Livello}: Intermedio-Avanzato
        \item \textit{Argomenti}: Indici, query execution plans, ottimizzazione
        \item \textit{Considerazione}: Pratico e focato su performance, linguaggio semplice
    \end{itemize}

    \item[\textbf{Learning SQL}] di Alan Beaulieu
    \begin{itemize}
        \item \textit{Casa editrice}: O'Reilly
        \item \textit{Livello}: Base-Intermedio
        \item \textit{Argomenti}: SQL fundamentals, joins, subqueries, set operations
        \item \textit{Considerazione}: Ottimo per principianti, esempi chiari, esercizi progressivi
    \end{itemize}

    \item[\textbf{MySQL Cookbook}] di Paul DuBois
    \begin{itemize}
        \item \textit{Casa editrice}: O'Reilly
        \item \textit{Livello}: Base-Intermedio
        \item \textit{Argomenti}: Query comuni, administration, troubleshooting specifico MySQL
        \item \textit{Considerazione}: Reference veloce per query standard
    \end{itemize}
\end{description}

\subsection{Progettazione e Normalizzazione}

\begin{description}
    \item[\textbf{Relational Database Design}] di C.J. Date
    \begin{itemize}
        \item \textit{Livello}: Avanzato
        \item \textit{Argomenti}: Teoria relazionale pura, normalizzazione, integrità referenziale
        \item \textit{Note}: Denso ma fondamentale per comprendere i principi sottostanti
    \end{itemize}

    \item[\textbf{Database Design for Mere Mortals}] di Michael J. Hernandez
    \begin{itemize}
        \item \textit{Livello}: Base-Intermedio
        \item \textit{Argomenti}: Progettazione ER, normalizzazione, best practices
        \item \textit{Note}: Accessibile, focalizzato su design pratico, non teorico
    \end{itemize}
\end{description}

\subsection{Amministrazione Database}

\begin{description}
    \item[\textbf{MySQL 8.0 Reference Manual}] (Documentazione ufficiale)
    \begin{itemize}
        \item \textit{Disponibile}: online su dev.mysql.com
        \item \textit{Livello}: Tutti i livelli
        \item \textit{Contenuto}: Guida completa, sempre aggiornata
        \item \textit{Note}: Prima fonte per MySQL specifico
    \end{itemize}

    \item[\textbf{High Performance MySQL}] di Baron Schwartz, Vadim Tkachenko, Peter Zaitsev
    \begin{itemize}
        \item \textit{Livello}: Intermedio-Avanzato
        \item \textit{Argomenti}: Optimization, replication, high availability, sharding
        \item \textit{Note}: Essenziale per production databases
    \end{itemize}
\end{description}

\section{Risorse Online}

\subsection{Documentazione Ufficiale}

\begin{description}
    \item[\textbf{MySQL Official Documentation}]
    \begin{itemize}
        \item \textit{URL}: \texttt{https://dev.mysql.com/doc/}
        \item \textit{Contenuto}: Documentazione completa per tutte le versioni
        \item \textit{Consiglio}: Bookmarkare le sezioni \textit{Language Reference} e \textit{InnoDB}
    \end{itemize}

    \item[\textbf{PostgreSQL Documentation}]
    \begin{itemize}
        \item \textit{URL}: \texttt{https://www.postgresql.org/docs/}
        \item \textit{Contenuto}: Excellente per analizzare alternative SQL (WINDOW FUNCTIONS, CTE)
        \item \textit{Consiglio}: Leggere per imparare best practices applicabili a tutti i DBMS
    \end{itemize}

    \item[\textbf{SQL Standard Documentation}]
    \begin{itemize}
        \item \textit{URL}: ISO/IEC 9075 (non libero, ma citato in documentazione)
        \item \textit{Alternativa}: Wikipedia SQL, MDN SQL Guide
        \item \textit{Consiglio}: Capire lo standard per scrivere SQL portabile
    \end{itemize}
\end{description}

\subsection{Tutorial e Corsi}

\begin{description}
    \item[\textbf{SQLZoo}]
    \begin{itemize}
        \item \textit{URL}: \texttt{https://sqlzoo.net/}
        \item \textit{Tipo}: Interactive tutorials
        \item \textit{Livello}: Base-Intermedio
        \item \textit{Vantaggi}: Impara praticando, feedback immediato
    \end{itemize}

    \item[\textbf{Mode Analytics SQL Tutorial}]
    \begin{itemize}
        \item \textit{URL}: \texttt{https://mode.com/sql-tutorial/}
        \item \textit{Tipo}: Tutorial passo passo con esercizi
        \item \textit{Livello}: Base-Intermedio
        \item \textit{Vantaggi}: Pratico, ben strutturato, gratis
    \end{itemize}

    \item[\textbf{LeetCode Database Problems}]
    \begin{itemize}
        \item \textit{URL}: \texttt{https://leetcode.com/problemset/database/}
        \item \textit{Tipo}: Problemi SQL di difficoltà crescente
        \item \textit{Livello}: Base-Avanzato
        \item \textit{Vantaggi}: Sfida il tuo livello, soluzioni commentate
    \end{itemize}

    \item[\textbf{Udemy - Complete MySQL Developer Course}]
    \begin{itemize}
        \item \textit{Docenti}: Various
        \item \textit{Livello}: Base-Intermedio
        \item \textit{Vantaggi}: Video, esercizi pratici, supporto
        \item \textit{Costo}: Spesso in offerta
    \end{itemize}

    \item[\textbf{Coursera - Databases Specialization}]
    \begin{itemize}
        \item \textit{Università}: University of Michigan
        \item \textit{Livello}: Base-Intermedio
        \item \textit{Vantaggi}: Riconoscimento accademico, strutturato
        \item \textit{Costo}: Gratuito per audit, certificato a pagamento
    \end{itemize}
\end{description}

\subsection{Blog e Articoli}

\begin{description}
    \item[\textbf{Use The Index, Luke!}]
    \begin{itemize}
        \item \textit{URL}: \texttt{https://use-the-index-luke.com/}
        \item \textit{Autore}: Markus Winand
        \item \textit{Tema}: Indici e query optimization
        \item \textit{Livello}: Intermedio-Avanzato
        \item \textit{Vantaggi}: Gratuito online, riferimento assoluto per indici
    \end{itemize}

    \item[\textbf{Percona Blog}]
    \begin{itemize}
        \item \textit{URL}: \texttt{https://www.percona.com/blog/}
        \item \textit{Tema}: MySQL performance, troubleshooting, best practices
        \item \textit{Livello}: Intermedio-Avanzato
        \item \textit{Vantaggi}: Articoli da esperti del settore
    \end{itemize}

    \item[\textbf{DB Fiddle}]
    \begin{itemize}
        \item \textit{URL}: \texttt{https://www.db-fiddle.com/}
        \item \textit{Tipo}: Editor SQL online
        \item \textit{Supporta}: MySQL, PostgreSQL, SQL Server, SQLite, Oracle
        \item \textit{Vantaggi}: Testare query online senza installare nulla
    \end{itemize}

    \item[\textbf{Stack Overflow - Tag SQL}]
    \begin{itemize}
        \item \textit{URL}: \texttt{https://stackoverflow.com/questions/tagged/sql}
        \item \textit{Tipo}: Community Q\&A
        \item \textit{Vantaggi}: Soluzioni a problemi comuni, learning dai dibattiti
    \end{itemize}
\end{description}

\section{Strumenti Pratici}

\subsection{DBMS e IDE SQL}

\begin{description}
    \item[\textbf{MySQL Community Server}]
    \begin{itemize}
        \item \textit{URL}: \texttt{https://dev.mysql.com/downloads/mysql/}
        \item \textit{Costo}: Gratuito, Open Source
        \item \textit{Piattaforme}: Linux, Windows, macOS
        \item \textit{Uso}: Produzione e sviluppo
    \end{itemize}

    \item[\textbf{PostgreSQL}]
    \begin{itemize}
        \item \textit{URL}: \texttt{https://www.postgresql.org/}
        \item \textit{Costo}: Gratuito, Open Source
        \item \textit{Vantaggi}: Più robusto di MySQL, SQL completo, JSONB, hstore
        \item \textit{Consiglio}: Valida alternativa a MySQL
    \end{itemize}

    \item[\textbf{DBeaver Community}]
    \begin{itemize}
        \item \textit{URL}: \texttt{https://dbeaver.io/}
        \item \textit{Costo}: Gratuito
        \item \textit{Tipo}: IDE SQL universale
        \item \textit{Supporta}: MySQL, PostgreSQL, Oracle, SQL Server, SQLite, molti altri
        \item \textit{Vantaggi}: UI intuitiva, query builder, ER diagram
    \end{itemize}

    \item[\textbf{MySQL Workbench}]
    \begin{itemize}
        \item \textit{URL}: \texttt{https://dev.mysql.com/downloads/workbench/}
        \item \textit{Costo}: Gratuito
        \item \textit{Tipo}: IDE ufficiale MySQL
        \item \textit{Vantaggi}: EER diagram design, reverse engineering, sync schema
    \end{itemize}

    \item[\textbf{DataGrip}]
    \begin{itemize}
        \item \textit{URL}: \texttt{https://www.jetbrains.com/datagrip/}
        \item \textit{Costo}: A pagamento (trial gratuito)
        \item \textit{Tipo}: IDE SQL premium
        \item \textit{Vantaggi}: Intelligenza artificiale, refactoring, analysis avanzata
    \end{itemize}

    \item[\textbf{VS Code + Extensions}]
    \begin{itemize}
        \item \textit{Extension}: MySQL (Weijan Chen) o Database Client
        \item \textit{Costo}: Gratuito
        \item \textit{Vantaggi}: Leggero, nel vostro editor preferito
    \end{itemize}
\end{description}

\subsection{Strumenti di Analisi e Monitoring}

\begin{description}
    \item[\textbf{EXPLAIN PLAN Analyzer}]
    \begin{itemize}
        \item \textit{URL}: \texttt{https://www.depesz.com/}
        \item \textit{Uso}: Paste EXPLAIN output per analizzare visivamente
        \item \textit{Supporta}: PostgreSQL, MySQL EXPLAIN
    \end{itemize}

    \item[\textbf{MySQL Slow Query Log Analyzer}]
    \begin{itemize}
        \item \textit{Tool}: mysqldumpslow, pt-query-digest (Percona Toolkit)
        \item \textit{Uso}: Identificare query lente
        \item \textit{Consiglio}: Essenziale per production debugging
    \end{itemize}

    \item[\textbf{Prometheus + Grafana}]
    \begin{itemize}
        \item \textit{Uso}: Monitoring database metrics
        \item \textit{Costo}: Gratuito
        \item \textit{Setup}: Più complesso, ma professionale
    \end{itemize}

    \item[\textbf{MySQL Exporter per Prometheus}]
    \begin{itemize}
        \item \textit{URL}: \texttt{https://github.com/prometheus/mysqld\_exporter}
        \item \textit{Uso}: Exportare metriche MySQL per monitoraggio
    \end{itemize}
\end{description}

\section{Articoli Accademici Importanti}

\subsection{Teoria Relazionale}

\begin{description}
    \item[\textbf{A Relational Model of Data}] di E.F. Codd (1970)
    \begin{itemize}
        \item \textit{Rilevanza}: Paper storico che ha fondato il modello relazionale
        \item \textit{Disponibilità}: ACM Digital Library
        \item \textit{Difficoltà}: Accademico, denso
    \end{itemize}

    \item[\textbf{Normalization of Database Relations}] di E.F. Codd (1972)
    \begin{itemize}
        \item \textit{Tema}: Fondamenti della normalizzazione
        \item \textit{Rilevanza}: Ancora attuale dopo 50 anni
    \end{itemize}
\end{description}

\subsection{Query Optimization}

\begin{description}
    \item[\textbf{The Architecture of a Database System}] di Hellerstein, Stonebraker, Hamilton
    \begin{itemize}
        \item \textit{Disponibilità}: Gratuito online
        \item \textit{Tema}: Interno di un DBMS, query optimization, transaction management
        \item \textit{Livello}: Avanzato
    \end{itemize}

    \item[\textbf{Selectivity Estimation}] - Vari articoli
    \begin{itemize}
        \item \textit{Tema}: Come il query optimizer stima numero di righe
        \item \textit{Rilevanza}: Capire perché optimizer sceglie certi piani
    \end{itemize}
\end{description}

\section{Comunità e Forum}

\subsection{Community Online}

\begin{description}
    \item[\textbf{MySQL Community Forums}]
    \begin{itemize}
        \item \textit{URL}: \texttt{https://forums.mysql.com/}
        \item \textit{Tipo}: Forum ufficiale MySQL
        \item \textit{Moderatori}: Team MySQL ufficiale
    \end{itemize}

    \item[\textbf{Database Administrators Stack Exchange}]
    \begin{itemize}
        \item \textit{URL}: \texttt{https://dba.stackexchange.com/}
        \item \textit{Tipo}: Community Q\&A per DBA
        \item \textit{Qualità}: Alta (moderazione rigorosa)
    \end{itemize}

    \item[\textbf{Reddit - r/databases}]
    \begin{itemize}
        \item \textit{URL}: \texttt{https://www.reddit.com/r/databases/}
        \item \textit{Tipo}: Community discussions
        \item \textit{Vantaggi}: Conversazioni informali, ultimissime news
    \end{itemize}

    \item[\textbf{Hacker News}]
    \begin{itemize}
        \item \textit{URL}: \texttt{https://news.ycombinator.com/}
        \item \textit{Filtro}: Search "database"
        \item \textit{Valore}: Scopri trends, discusisoni profonde di esperti
    \end{itemize}
\end{description}

\section{Conferenze e Meetup}

\begin{description}
    \item[\textbf{Percona Live}]
    \begin{itemize}
        \item \textit{Tipo}: Conferenza internazionale MySQL e database
        \item \textit{Frequenza}: Annuale
        \item \textit{Partecipazione}: Relatori di fama mondiale
    \end{itemize}

    \item[\textbf{MySQL User Group}]
    \begin{itemize}
        \item \textit{Ubicazione}: Molte città (locale)
        \item \textit{Frequenza}: Mensile/trimestrale
        \item \textit{Costo}: Gratuito
    \end{itemize}

    \item[\textbf{PostgreSQL Meetups}]
    \begin{itemize}
        \item \textit{Ubicazione}: Mondiale
        \item \textit{URL}: \texttt{https://www.meetup.com/}
        \item \textit{Consiglio}: Cercare per città
    \end{itemize}
\end{description}

\section{Certificazioni}

\subsection{Certificazioni Database}

\begin{description}
    \item[\textbf{Oracle Certified Associate MySQL Developer}]
    \begin{itemize}
        \item \textit{Ente}: Oracle
        \item \textit{Livello}: Base-Intermedio
        \item \textit{Costo}: Circa 245 USD per esame
        \item \textit{Preparazione}: MySQL 5.7 Certified Associate Exam Guide
        \item \textit{Valore}: Riconoscimento professionale
    \end{itemize}

    \item[\textbf{PostgreSQL Certification}]
    \begin{itemize}
        \item \textit{Ente}: PostgreSQL Association e altri enti
        \item \textit{Livello}: Vario
        \item \textit{Costo}: Vario
        \item \textit{Valore}: Meno noto di Oracle, ma valido
    \end{itemize}

    \item[\textbf{Cloud Database Certifications}]
    \begin{itemize}
        \item AWS Certified Cloud Practitioner
        \item Google Cloud Associate Cloud Engineer
        \item Azure Fundamentals
        \item \textit{Valore}: Importante nel cloud moderno
    \end{itemize}
\end{description}

\section{Checklist per Padronanza}

Prima di considerarsi "esperto", assicurati di poter:

\begin{itemize}
    \item Progettare schema da zero con ER diagram
    \item Normalizzare fino a BCNF senza errori
    \item Scrivere query complex con JOIN, subquery, window functions
    \item Leggere e interpretare EXPLAIN plans
    \item Ottimizzare query lente
    \item Configurare indici appropriati
    \item Gestire transazioni e deadlock
    \item Backup e recovery di database
    \item Replicazione e high availability
    \item Monitoraggio e troubleshooting
    \item Sicurezza: utenti, ruoli, permessi
    \item Coding: stored procedures, trigger
\end{itemize}

\section{Consigli Finali}

\begin{tcolorbox}[colback=green!10, colframe=green!60, title=Come Continuare l'Apprendimento]
\begin{enumerate}
    \item \textbf{Pratica costante}: Scrivi SQL ogni giorno, anche piccole query
    \item \textbf{Leggi codice altrui}: Su GitHub, Stack Overflow, blog tech
    \item \textbf{Contribuisci open source}: MySQL, PostgreSQL, tool SQL
    \item \textbf{Sperimenta}: Crea progetti personali, database interessanti
    \item \textbf{Segui blog/newsletter}: Use The Index Luke, Percona Blog
    \item \textbf{Partecipa comunità}: Forum, meetup, conferenze
    \item \textbf{Insegna agli altri}: Spiegare consolida la comprensione
    \item \textbf{Stai aggiornato}: Database evolvono (MySQL 8.0 features, PostgreSQL 15+)
    \item \textbf{Comprendi teoria}: Non solo sintassi SQL, capire motivazioni
    \item \textbf{Misura performance}: Non indovinare, profila sempre
\end{enumerate}
\end{tcolorbox}

\section*{Nota sulla Licenza}

Le risorse elencate sono pubblicamente disponibili e la loro inclusione non implica endorsement. Visita i siti ufficiali per dettagli su licenze e termini di utilizzo.
