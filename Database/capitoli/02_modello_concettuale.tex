\chapter{Modello Concettuale - Diagrammi ER}

\section*{Introduzione}
Il modello Entità-Relazione (ER) è uno strumento per rappresentare la struttura logica dei dati indipendentemente dall'implementazione fisica. Questo capitolo presenta i diagrammi ER, le entità, gli attributi e le relazioni con le loro cardinalità.

\section*{Obiettivi di apprendimento}
\begin{itemize}
    \item Comprendere i componenti del modello ER
    \item Disegnare entità e attributi
    \item Identificare e modellare relazioni tra entità
    \item Comprendere la cardinalità delle relazioni
    \item Applicare le chiavi primarie e esterne nel modello concettuale
    \item Tradurre requisiti aziendali in diagrammi ER
\end{itemize}

\section{Componenti del Modello ER}

\subsection{Entità}
Un'\textbf{entità} è un oggetto del mondo reale di interesse per il sistema informativo. Rappresenta una classe di oggetti con proprietà comuni.

\begin{tcolorbox}[colback=blue!10, colframe=blue!60, title=Esempio: Entità]
Entità: \textbf{Cliente}, \textbf{Prodotto}, \textbf{Ordine}, \textbf{Dipendente}

Ogni entità rappresenta un concetto del dominio aziendale.
\end{tcolorbox}

Nel diagramma ER, un'entità è rappresentata come un rettangolo.

\subsection{Attributi}
Un \textbf{attributo} è una proprietà di un'entità. Descrive caratteristiche specifiche dell'entità.

\begin{tcolorbox}[colback=blue!10, colframe=blue!60, title=Esempio: Attributi]
Entità \textbf{Cliente} ha attributi:
\begin{itemize}
    \item \texttt{idCliente} (identificativo univoco)
    \item \texttt{nome}
    \item \texttt{cognome}
    \item \texttt{email}
    \item \texttt{dataRegistrazione}
\end{itemize}
\end{tcolorbox}

\subsubsection{Tipi di attributi}
Gli attributi possono essere classificati in diverse categorie in base alle loro caratteristiche. Un attributo \textbf{semplice} non è scomponibile e rappresenta un'unità atomica di informazione, come un indirizzo email. Un attributo \textbf{composto}, al contrario, può essere scomposto in attributi più piccoli e semanticamente significativi; per esempio, un indirizzo può essere diviso in via, numero civico, città e CAP. Un attributo \textbf{univoco} o chiave identifica univocamente l'entità, come un identificativo di cliente. Un attributo \textbf{opzionale} può non avere un valore assegnato, come un numero di cellulare che potrebbe non essere disponibile per tutti i clienti. Infine, un attributo \textbf{multivalore} può assumere più valori per la stessa entità, come una raccolta di numeri di telefono per una ditta.

\subsection{Relazioni}
Una \textbf{relazione} (o associazione) descrive come due o più entità interagiscono tra loro.

\begin{tcolorbox}[colback=blue!10, colframe=blue!60, title=Esempio: Relazioni]
\begin{itemize}
    \item Un \textbf{Cliente} \textit{effettua} un \textbf{Ordine}
    \item Un \textbf{Ordine} \textit{contiene} \textbf{Prodotti}
    \item Un \textbf{Dipendente} \textit{lavora in} un \textbf{Dipartimento}
\end{itemize}
\end{tcolorbox}

\section{Cardinalità delle Relazioni}

La \textbf{cardinalità} specifica quanti elementi di un'entità possono essere correlati con elementi dell'altra entità.

\subsection{Tipi di cardinalità}

La cardinalità può assumere diverse forme a seconda della natura della relazione tra entità. Una relazione \textbf{1:1 (uno a uno)} si verifica quando ogni elemento di A è associato a esattamente un elemento di B, e viceversa; questo è il caso più restrittivo. Una relazione \textbf{1:N (uno a molti)} permette che ogni elemento di A sia associato a uno o più elementi di B, ma ogni elemento di B rimane associato a un solo elemento di A. Una relazione \textbf{N:M (molti a molti)} è la più flessibile, consentendo che ogni elemento di A sia associato a più elementi di B e viceversa. Oltre ai tipi principali, esistono notazioni aggiuntive per specificare l'obbligatorietà: una notazione \textbf{0..1} indica un'associazione opzionale (zero a uno), mentre \textbf{1..N} rappresenta un'associazione obbligatoria (da uno a molti).

\subsection{Esempio 1: Relazione 1:1}

\begin{figure}[h]
    \centering
    \begin{tikzpicture}
        % Entità
        \node[draw, rectangle, minimum width=2cm] (persona) {Persona};
        \node[draw, rectangle, right=3cm of persona, minimum width=2cm] (passaporto) {Passaporto};

        % Relazione
        \draw (persona.east) -- node[above] {possiede} (passaporto.west);

        % Cardinalità
        \node[above left] at (persona.east) {1};
        \node[above right] at (passaporto.west) {1};
    \end{tikzpicture}
    \caption{Una persona possiede esattamente un passaporto}
\end{figure}

Una persona può avere un solo passaporto e ogni passaporto appartiene a una sola persona.

\subsection{Esempio 2: Relazione 1:N}

\begin{figure}[h]
    \centering
    \begin{tikzpicture}
        % Entità
        \node[draw, rectangle, minimum width=2cm] (cliente) {Cliente};
        \node[draw, rectangle, right=3cm of cliente, minimum width=2cm] (ordine) {Ordine};

        % Relazione
        \draw (cliente.east) -- node[above] {effettua} (ordine.west);

        % Cardinalità
        \node[above left] at (cliente.east) {1};
        \node[above right] at (ordine.west) {N};
    \end{tikzpicture}
    \caption{Un cliente può effettuare molti ordini}
\end{figure}

Un cliente può fare molti ordini, ma ogni ordine è fatto da un solo cliente.

\subsection{Esempio 3: Relazione N:M}

\begin{figure}[h]
    \centering
    \begin{tikzpicture}
        % Entità
        \node[draw, rectangle, minimum width=2cm] (studente) {Studente};
        \node[draw, rectangle, right=3cm of studente, minimum width=2cm] (corso) {Corso};

        % Relazione
        \draw (studente.east) -- node[above] {segue} (corso.west);

        % Cardinalità
        \node[above left] at (studente.east) {N};
        \node[above right] at (corso.west) {M};
    \end{tikzpicture}
    \caption{Uno studente segue molti corsi e un corso è seguito da molti studenti}
\end{figure}

Uno studente può seguire più corsi e ogni corso può avere più studenti.

\section{Diagramma ER Completo: Sistema di Vendita Online}

\begin{figure}[h]
    \centering
    \begin{tikzpicture}[scale=1, node distance=3cm]
        % Entità Cliente
        \node[draw, rectangle, minimum width=2.5cm, minimum height=1.5cm] (cliente) {
            \textbf{Cliente}\\
            \hline
            idCliente\\
            nome\\
            cognome\\
            email
        };

        % Entità Ordine
        \node[draw, rectangle, right=of cliente, minimum width=2.5cm, minimum height=1.5cm] (ordine) {
            \textbf{Ordine}\\
            \hline
            idOrdine\\
            dataOrdine\\
            totale
        };

        % Entità Prodotto
        \node[draw, rectangle, below=of ordine, minimum width=2.5cm, minimum height=1.5cm] (prodotto) {
            \textbf{Prodotto}\\
            \hline
            idProdotto\\
            nome\\
            prezzo\\
            categoria
        };

        % Relazioni
        \draw (cliente.east) -- node[above] {effettua\\1:N} (ordine.west);
        \draw (ordine.south) -- node[right] {contiene\\N:M} (prodotto.north);
    \end{tikzpicture}
    \caption{Diagramma ER semplificato per un e-commerce}
\end{figure}

\section{Chiavi nel Modello ER}

\subsection{Chiave Primaria}
La \textbf{chiave primaria} è un attributo (o insieme di attributi) che identifica univocamente ogni istanza dell'entità. Non può essere nulla e deve essere univoca.

\begin{tcolorbox}[colback=green!10, colframe=green!60, title=Esempio: Chiave Primaria]
Entità \textbf{Cliente}:
\begin{itemize}
    \item \textbf{idCliente} è la chiave primaria (numero univoco assegnato a ogni cliente)
    \item Anche \textbf{email} potrebbe essere chiave primaria (ogni cliente ha un'email unica)
\end{itemize}
\end{tcolorbox}

\subsection{Chiave Esterna}
Una \textbf{chiave esterna} è un attributo che riferisce la chiave primaria di un'altra entità, creando un collegamento.

\begin{tcolorbox}[colback=green!10, colframe=green!60, title=Esempio: Chiave Esterna]
Entità \textbf{Ordine}:
\begin{itemize}
    \item \textbf{idOrdine} è la chiave primaria
    \item \textbf{idCliente} è una chiave esterna (riferisce la chiave primaria di Cliente)
\end{itemize}
\end{tcolorbox}

\section*{Riepilogo concetti chiave}

\begin{tcolorbox}[colback=gray!10, colframe=black!60, title=Concetti fondamentali]
Nel modello entità-relazione, un'\textbf{entità} rappresenta una classe di oggetti del dominio reale di interesse. Ogni entità possiede \textbf{attributi} che descrivono specifiche proprietà e caratteristiche. Le entità sono collegate tra loro attraverso \textbf{relazioni} che definiscono le associazioni significative. La \textbf{cardinalità} specifica il numero e la natura delle associazioni possibili tra entità, espresse come 1:1, 1:N, o N:M. Per identificare univocamente ogni istanza di entità, si utilizza una \textbf{chiave primaria}, mentre una \textbf{chiave esterna} crea il collegamento tra entità diverse, mantenendo la coerenza strutturale del modello.
\end{tcolorbox}

\section*{Esercizi}

\begin{enumerate}
    \item Disegna un diagramma ER per una biblioteca con le entità: Libro, Autore, Membro, Prestito. Includi attributi appropriati e cardinalità.

    \item Spiega la differenza tra cardinalità 1:N e N:M con due esempi concreti diversi da quelli del capitolo.

    \item Un'università ha Studenti, Corsi e Professori. Quali sono le relazioni tra questi? Quale cardinalità?

    \item Identifica la chiave primaria e le chiavi esterne nel diagramma dell'e-commerce presentato nel capitolo.

    \item Trasforma il seguente requisito aziendale in un diagramma ER:
    ``Una società ha Dipartimenti. Ogni Dipartimento ha più Dipendenti. Un Dipendente lavora in un solo Dipartimento. Un Dipendente gestisce Progetti. Un Progetto può essere gestito da un solo Dipendente. Un Progetto coinvolge più Dipendenti.''
\end{enumerate}
