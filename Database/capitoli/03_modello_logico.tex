\chapter{Modello Logico - Modello Relazionale}

\section*{Introduzione}
Il modello logico traduce il modello concettuale (diagrammi ER) in un modello implementabile nei DBMS. Il modello relazionale organizza i dati in tabelle (relazioni) con colonne (attributi) e righe (tuple). Questo capitolo presenta la struttura del modello relazionale, le chiavi e i vincoli di integrità.

\section*{Obiettivi di apprendimento}
\begin{itemize}
    \item Comprendere la struttura del modello relazionale
    \item Tradurre diagrammi ER in tabelle relazionali
    \item Definire e utilizzare chiavi primarie e esterne
    \item Comprendere i vincoli di integrità referenziale
    \item Applicare le regole di trasformazione ER-relazionale
    \item Progettare schemi logici corretti
\end{itemize}

\section{Il Modello Relazionale}

\subsection{Componenti fondamentali}
Il modello relazionale è basato su tre concetti chiave:

\begin{description}
    \item[\textbf{Relazione (Tabella)}] Una collezione di tuple (righe) con lo stesso insieme di attributi (colonne).
    \item[\textbf{Attributo (Colonna)}] Una proprietà della relazione con un nome e un dominio (tipo di dato).
    \item[\textbf{Tupla (Riga)}] Un record che contiene un valore per ogni attributo.
\end{description}

\begin{tcolorbox}[colback=blue!10, colframe=blue!60, title=Esempio: Tabella Cliente]
\begin{center}
\begin{tabular}{|c|c|c|c|}
\hline
\textbf{idCliente} & \textbf{nome} & \textbf{cognome} & \textbf{email} \\
\hline
1 & Luigi & Rossi & luigi@email.com \\
2 & Maria & Bianchi & maria@email.com \\
3 & Giovanni & Verdi & giovanni@email.com \\
\hline
\end{tabular}

Ogni riga è una tupla, ogni colonna è un attributo.
\end{center}
\end{tcolorbox}

\subsection{Dominio di un attributo}
Ogni attributo ha un \textbf{dominio}, cioè l'insieme di valori che può assumere.

\begin{tcolorbox}[colback=green!10, colframe=green!60, title=Esempio: Domini]
\begin{itemize}
    \item \texttt{idCliente}: dominio = interi positivi (1, 2, 3, \ldots)
    \item \texttt{nome}: dominio = stringhe di caratteri (fino a 100 caratteri)
    \item \texttt{dataIscrizione}: dominio = date (formato YYYY-MM-DD)
    \item \texttt{saldo}: dominio = numeri decimali positivi
\end{itemize}
\end{tcolorbox}

\section{Chiavi nel Modello Relazionale}

\subsection{Chiave Primaria}
Una \textbf{chiave primaria} è un attributo (o insieme di attributi) che:
\begin{itemize}
    \item Identifica univocamente ogni tupla della relazione
    \item Non può contenere valori NULL
    \item Non può avere duplicati
\end{itemize}

\begin{lstlisting}[language=SQL, caption=Dichiarazione di chiave primaria]
CREATE TABLE cliente (
    idCliente INT PRIMARY KEY,
    nome VARCHAR(100) NOT NULL,
    cognome VARCHAR(100) NOT NULL,
    email VARCHAR(100) UNIQUE
);
\end{lstlisting}

\subsection{Chiave Candidata}
Una \textbf{chiave candidata} è un attributo che potrebbe essere chiave primaria. Una relazione può avere più chiavi candidate, ma solo una viene scelta come primaria.

\begin{tcolorbox}[colback=orange!10, colframe=orange!60, title=Nota: Chiavi candidate}
Tabella Cliente:
\begin{itemize}
    \item \texttt{idCliente}: chiave candidata (univoca, non nulla)
    \item \texttt{email}: chiave candidata (univoca, non nulla)
    \item Sceglieremo \texttt{idCliente} come primaria
\end{itemize}
\end{tcolorbox}

\subsection{Chiave Esterna}
Una \textbf{chiave esterna} è un attributo (o insieme di attributi) che referenzia la chiave primaria di un'altra relazione. Crea il collegamento tra tabelle.

\begin{lstlisting}[language=SQL, caption=Dichiarazione di chiave esterna]
CREATE TABLE ordine (
    idOrdine INT PRIMARY KEY,
    dataOrdine DATE NOT NULL,
    idCliente INT NOT NULL,
    FOREIGN KEY (idCliente) REFERENCES cliente(idCliente)
);
\end{lstlisting}

\subsection{Chiave Composta}
Una chiave può essere formata da più attributi.

\begin{lstlisting}[language=SQL, caption=Chiave composta come chiave primaria]
CREATE TABLE ordine_prodotto (
    idOrdine INT NOT NULL,
    idProdotto INT NOT NULL,
    quantita INT NOT NULL,
    PRIMARY KEY (idOrdine, idProdotto),
    FOREIGN KEY (idOrdine) REFERENCES ordine(idOrdine),
    FOREIGN KEY (idProdotto) REFERENCES prodotto(idProdotto)
);
\end{lstlisting}

\section{Vincoli di Integrità}

\subsection{Vincolo di dominio}
Ogni valore di un attributo deve appartenere al dominio definito.

\begin{lstlisting}[language=SQL, caption=Vincoli di dominio]
CREATE TABLE prodotto (
    idProdotto INT PRIMARY KEY,
    nome VARCHAR(200) NOT NULL,      -- Non null
    prezzo DECIMAL(8, 2) NOT NULL,   -- Numero con 2 decimali
    categoria ENUM('Elettronica', 'Libri', 'Abbigliamento'),
    quantitaDisponibile INT CHECK (quantitaDisponibile >= 0)
);
\end{lstlisting}

\subsection{Vincolo di unicità}
Un attributo (o insieme di attributi) può avere solo valori unici (o nulli).

\begin{lstlisting}[language=SQL, caption=Vincolo di unicità]
CREATE TABLE utente (
    idUtente INT PRIMARY KEY,
    username VARCHAR(50) UNIQUE NOT NULL,
    email VARCHAR(100) UNIQUE NOT NULL,
    dataRegistrazione DATE DEFAULT CURDATE()
);
\end{lstlisting}

\subsection{Vincolo di chiave primaria}
Garantisce l'univocità e l'obbligatorietà (NOT NULL).

\subsection{Vincolo di chiave esterna e integrità referenziale}
La chiave esterna assicura che ogni valore riferisca effettivamente un'entità in un'altra relazione.

\begin{tcolorbox}[colback=red!10, colframe=red!60, title=Attenzione: Violazione di integrità referenziale]
Se in ordine abbiamo \texttt{idCliente = 5}, questo \texttt{idCliente} DEVE esistere in cliente. Altrimenti il DBMS rifiuta l'inserimento.
\end{tcolorbox}

\begin{lstlisting}[language=SQL, caption=Azioni su integrità referenziale]
CREATE TABLE ordine (
    idOrdine INT PRIMARY KEY,
    idCliente INT NOT NULL,
    FOREIGN KEY (idCliente) REFERENCES cliente(idCliente)
        ON DELETE CASCADE      -- Elimina ordini se cliente è eliminato
        ON UPDATE CASCADE      -- Aggiorna idCliente se cambia in cliente
);
\end{lstlisting}

\section{Trasformazione da ER a Modello Relazionale}

\subsection{Regola 1: Entità singole}
Ogni entità diventa una tabella. Gli attributi della tabella sono i rispettivi attributi dell'entità. La chiave primaria dell'entità diventa chiave primaria della tabella.

\begin{tcolorbox}[colback=blue!10, colframe=blue!60, title=Esempio: Entità Cliente]
\textbf{ER}: Entità Cliente con attributi (idCliente, nome, cognome, email)

\textbf{Relazionale}:
\begin{lstlisting}[language=SQL]
CREATE TABLE cliente (
    idCliente INT PRIMARY KEY,
    nome VARCHAR(100) NOT NULL,
    cognome VARCHAR(100) NOT NULL,
    email VARCHAR(100) UNIQUE
);
\end{lstlisting}
\end{tcolorbox}

\subsection{Regola 2: Relazione 1:N}
Il lato N contiene una chiave esterna che referenzia la chiave primaria del lato 1.

\begin{tcolorbox}[colback=blue!10, colframe=blue!60, title=Esempio: Cliente (1) - Ordine (N)}
\textbf{ER}: Cliente effettua Ordine (1:N)

\textbf{Relazionale}:
\begin{lstlisting}[language=SQL]
CREATE TABLE cliente (
    idCliente INT PRIMARY KEY,
    nome VARCHAR(100)
);

CREATE TABLE ordine (
    idOrdine INT PRIMARY KEY,
    dataOrdine DATE NOT NULL,
    idCliente INT NOT NULL,
    FOREIGN KEY (idCliente) REFERENCES cliente(idCliente)
);
\end{lstlisting}
\end{tcolorbox}

\subsection{Regola 3: Relazione 1:1}
Esiste una chiave esterna in una delle due tabelle. Generalmente nel lato con partecipazione opzionale.

\begin{tcolorbox}[colback=blue!10, colframe=blue!60, title=Esempio: Dipendente (1) - Ufficio (1)]
\textbf{ER}: Dipendente lavora in Ufficio (1:1)

\textbf{Relazionale}:
\begin{lstlisting}[language=SQL]
CREATE TABLE dipendente (
    idDipendente INT PRIMARY KEY,
    nome VARCHAR(100),
    idUfficio INT UNIQUE,
    FOREIGN KEY (idUfficio) REFERENCES ufficio(idUfficio)
);

CREATE TABLE ufficio (
    idUfficio INT PRIMARY KEY,
    posizione VARCHAR(100)
);
\end{lstlisting}
\end{tcolorbox}

\subsection{Regola 4: Relazione N:M}
Si crea una \textbf{tabella di giunzione} (o tabella di associazione) con chiavi esterne che referenziano entrambe le entità. La chiave primaria è la combinazione delle due chiavi esterne.

\begin{tcolorbox}[colback=blue!10, colframe=blue!60, title=Esempio: Studente (N) - Corso (M)]
\textbf{ER}: Studente segue Corso (N:M)

\textbf{Relazionale}:
\begin{lstlisting}[language=SQL]
CREATE TABLE studente (
    idStudente INT PRIMARY KEY,
    nome VARCHAR(100)
);

CREATE TABLE corso (
    idCorso INT PRIMARY KEY,
    titolo VARCHAR(100)
);

-- Tabella di giunzione
CREATE TABLE iscrizione (
    idStudente INT NOT NULL,
    idCorso INT NOT NULL,
    dataIscrizione DATE DEFAULT CURDATE(),
    PRIMARY KEY (idStudente, idCorso),
    FOREIGN KEY (idStudente) REFERENCES studente(idStudente),
    FOREIGN KEY (idCorso) REFERENCES corso(idCorso)
);
\end{lstlisting}
\end{tcolorbox}

\section{Schema Logico Completo: E-commerce}

\begin{figure}[h]
    \centering
    \begin{tikzpicture}[scale=0.8, node distance=4cm]
        % Cliente
        \node[draw, rectangle, minimum width=3cm, minimum height=2cm] (cliente) {
            \textbf{cliente}\\
            \hline
            idCliente (PK)\\
            nome\\
            cognome\\
            email (UNIQUE)
        };

        % Ordine
        \node[draw, rectangle, right=of cliente, minimum width=3cm, minimum height=2cm] (ordine) {
            \textbf{ordine}\\
            \hline
            idOrdine (PK)\\
            dataOrdine\\
            totale\\
            idCliente (FK)
        };

        % Prodotto
        \node[draw, rectangle, below=of ordine, minimum width=3cm, minimum height=2cm] (prodotto) {
            \textbf{prodotto}\\
            \hline
            idProdotto (PK)\\
            nome\\
            prezzo\\
            categoria
        };

        % Ordine_Prodotto
        \node[draw, rectangle, left=of prodotto, minimum width=3cm, minimum height=2cm] (ord_prod) {
            \textbf{ordine\_prodotto}\\
            \hline
            idOrdine (FK, PK)\\
            idProdotto (FK, PK)\\
            quantita
        };

        % Relazioni
        \draw[->] (cliente.east) -- (ordine.west);
        \draw[->] (ordine.south) -- (ord_prod.north);
        \draw[->] (prodotto.west) -- (ord_prod.east);
    \end{tikzpicture}
    \caption{Schema relazionale per un e-commerce (PK=Primary Key, FK=Foreign Key)}
\end{figure}

\section*{Riepilogo concetti chiave}

\begin{tcolorbox}[colback=gray!10, colframe=black!60, title=Concetti fondamentali]
\begin{itemize}
    \item Una \textbf{relazione} è una tabella con attributi e tuple
    \item La \textbf{chiave primaria} identifica univocamente ogni tupla
    \item La \textbf{chiave esterna} collega tabelle diverse mantenendo l'integrità referenziale
    \item I \textbf{vincoli di integrità} garantiscono la coerenza dei dati
    \item Le relazioni N:M richiedono una \textbf{tabella di giunzione}
    \item Ogni diagramma ER viene tradotto sistematicamente in tabelle relazionali
\end{itemize}
\end{tcolorbox}

\section*{Esercizi}

\begin{enumerate}
    \item Crea lo schema logico (CREATE TABLE) per il diagramma ER della biblioteca dell'esercizio del capitolo precedente.

    \item Quali azioni dovrebbero accadere se un cliente viene eliminato da una tabella cliente e ha più ordini? Spiega le opzioni ON DELETE.

    \item Progetta uno schema relazionale per un sistema ospedaliero con: Pazienti, Medici, Appuntamenti, Ospedali, Reparti. Includi vincoli e cardinalità.

    \item Identifica chiavi primarie, chiavi candidate e chiavi esterne nel seguente DDL:
    \begin{lstlisting}[language=SQL]
CREATE TABLE impiegato (
    matricola INT PRIMARY KEY,
    nome VARCHAR(100),
    dipartimento VARCHAR(50),
    stipendio DECIMAL(10, 2)
);
CREATE TABLE progetto (
    codice INT PRIMARY KEY,
    titolo VARCHAR(100),
    responsabile INT,
    FOREIGN KEY (responsabile) REFERENCES impiegato(matricola)
);
    \end{lstlisting}

    \item Trasforma il seguente diagramma ER in schema relazionale: Azienda ha Sedi. Una Sede ha più Dipendenti. Un Dipendente appartiene a una Sede. Un Dipendente gestisce Progetti. Un Progetto è gestito da un Dipendente. Un Progetto ha più Risorse. Una Risorsa è utilizzata da più Progetti.
\end{enumerate}
